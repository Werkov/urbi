\chapter{Frequently Asked Questions}
\label{sec:faq}

\section{Objects, slots and variables}

\paragraph{How do I create a new Object derivative?}
Urbi is based on prototypes. To create a new Object derivative (which
will inherit all the Object methods), you can do:

\begin{urbifixme}
var myObject = Object.new;
\end{urbifixme}

\paragraph{How do I add a new slot in an object?}
To add a slot to an object \code{O}, you have to use the \code{var} keyword:

\begin{urbifixme}
var O.mySlot = 42;
\end{urbifixme}

\noindent
or you can use the "setSlot" method:

\begin{urbifixme}
O.setSlot("mySlot", 42);
\end{urbifixme}

Note that in a method, \code{this} designates the current object. For example,

\begin{urbifixme}
var this.mySlot = 42;
\end{urbifixme}

\noindent
will create a new slot in the current object while

\begin{urbifixme}
var mySlot = 42;
\end{urbifixme}

\noindent
will create a local variable, which will not be known anymore after we
exist the current scope.

\paragraph{How do I modify a slot of my object?}
\begin{urbifixme}
O.mySlot = 51;
\end{urbifixme}

\noindent
or

\begin{urbifixme}
O.updateSlot("mySlot", 51);
\end{urbifixme}

\paragraph{How do I create or modify a local variable?}
\begin{urbifixme}
var myLocalVariable;
myLocalVariable = "foo";
\end{urbifixme}

\noindent
or

\begin{urbifixme}
var myLocalVariable = "foo";
\end{urbifixme}


\paragraph{How do I make a constructor?}
You can define a method called \code{init} which will be called
automatically by \code{new}. For example:

\begin{urbifixme}
class myObject {
  function init(x, y) {
    var this.x = x;
    var this.y = y;
  };
};
myInstance = myObject.new(10, 20);
\end{urbifixme}


\paragraph{How can I manipulate the list of prototypes of my objects?}
The \code{protos} method returns a list (which can be manipulated)
containing the list of your object prototype.

\begin{urbifixme}
var myObject = Object.new;
myObject.protos;
[00000001] [Object]
\end{urbifixme}

\paragraph{How can I know the slots available for a given object?}
The \code{slotNames} and \code{allSlotNames} methods return
respectively the local slot names and the local+inherited slot names.

\paragraph{How do I create a new function?}
Functions are first class objects. That means that you can add them as
any other slot in an object:

\begin{urbifixme}
var myObject = Object.new;
var myFunction = function (x, y)
  { echo ("myFunction called with " + x + " and " + y) };
\end{urbifixme}

You can also use the following notation to add a function to your
object:

\begin{urbifixme}
var myObject = Object.new;
function myObject.myFunction (x, y) { /* ... */ };
\end{urbifixme}

or even group definitions within a "do" scope, which will automatically
define new slots instead of local variables and functions:

\begin{urbifixme}
var myObject = Object.new;
do (myObject) {
  function myFunction (x, y) { /* ... */ };
};
\end{urbifixme}

or group those two statements by using a convenient "class" scope:

\begin{urbifixme}
class myObject {
  function myFunction (x, y) { /* ... */ };
};
\end{urbifixme}

\section{Tags}

\paragraph{How do I create a tag?}


Tags are objects, and must be created as any object by using "new" to create
derivatives of the "Tag" object.

\begin{urbifixme}
var myTag = Tag.new;
\end{urbifixme}

or

\begin{urbifixme}
var myTag = Tag.new("cool name");
\end{urbifixme}

The name is not mandatory but makes it easier to display a tag and remember
what it is.

\paragraph{How do I stop a tag?}


\begin{urbifixme}
myTag.stop;
\end{urbifixme}

\paragraph{Can tagged statements return a value?}


By default, tagged statements will return the latest evaluated expression if
they have not been stopped:

\begin{urbifixme}
var myTag = Tag.new;
var res;
res = { myTag: { do_something; 42 } };
\end{urbifixme}

After the execution, "res" will contain 42.

However, when a tag is stopped, all the statements tagged with that tag may
be forced to return a value, by giving it as a parameter to "stop":

\begin{urbifixme}
var myTag = Tag.new;
var res;
{ res = { myTag: sleep(inf) } }, // Sleep forever (infinite amount of time),
                                 // note the "," to put the tagged statement in
                                 // the background
myTag.stop(42);
res;
[00000001] 42
\end{urbifixme}

\section{Events}

\paragraph{How do I create an event?}
Events are objects, and must be created as any object by using "new" to create
derivatives of the "Event" object.

\begin{urbifixme}
var ev = Event.new;
\end{urbifixme}

\paragraph{How do I emit an event?}

\begin{urbifixme}
ev!(1, "foo");
\end{urbifixme}

\paragraph{How do I catch an event?}

\begin{urbifixme}
at(ev?(1, var msg)) {
  echo ("Received event with 1 and message " + msg);
};
\end{urbifixme}

The \code{?} marker indicates that we are looking for an event instead of a
boolean condition. The construction "var msg" indicates that the "msg"
variable will be bound (as a local variable) in the body part of the "at"
construct, with whatever value is present in the event that triggered the
"at".

\section{Lists}

\paragraph{How can I iterate over a list?}

You can use the "for" construct to iterate over the elements of a list:

\begin{urbiscript}
for (var i: [10, 11, 12]) echo (i);
[00000001] *** 10
[00000002] *** 11
[00000003] *** 12
\end{urbiscript}

\section{UObjects loading}

\paragraph{urbi-launch fails with 'myuobject: file not found'. What can I do?}
If you are sure the file is there, the most probable cause is an
undefined symbol in your shared library. A libltdl quirk prevents us
from displaying a more accurate error message.  You can use a tool
named \command{ltrace} to obtain the exact error message.  Ltrace is a
standard package on most Linux distributions.  Run it with
\samp{ltrace -C -s 1024 urbi-launch ...}, and look for lines
containing \samp{dlerror} in the output. One will contain the exact
message that occurred while trying to load your shared library.

Under Mac OS X, the \code{DYLD\_PRINT\_APIS} environment variable can
be set to 1 to make the shared library loader more verbose and display
all its attempts to load a file to stderr.

Note that this problem is Mac OS X and Linux specific.

\section{MiscStatus}

\paragraph{How can I know what has changed since the latest beta release I got?}

The file \file{RELEASE-NOTES.txt} at the top of the distribution
contains the list of user-visible changes between consecutive
releases.

\paragraph{How do I report a bug?}

See the file \file{REPORTING-BUGS.txt} at the top of the
distribution. We insist on keeping \email{k2-beta@lists.gostai.com} in
copy at all times, because it helps us to track conversations with our
users.


%%% Local Variables:
%%% mode: latex
%%% TeX-master: "urbi-sdk"
%%% End:
