\chapter{Introduction}

Urbi is a programming language primarily designed for robotics. It's a
dynamic, prototype-based, object-oriented scripting language. It
supports and emphasizes parallel and event-based programming, which
are very popular paradigms in robotics, by providing core primitives
and language constructs.

Its main features are:
\begin{itemize}
\item \urbi is designed to be syntactically close to \Cxx. If you know
  \C or \Cxx, you can easily write \urbi programs.
\item \urbi is fully integrated with \Cxx. You can bind \Cxx classes
  in \urbi seamlessly. \urbi is also integrated with many other
  languages such as \java, \matlab or \python.
\item \urbi is object oriented. It supports encapsulation, inheritance
  and inclusion polymorphism. Dynamic dispatching is available through
  monomethods --- just as \Cxx, \Cs or \java.
\item \urbi includes parallelism at the core of its semantics. It
  provides you with natural constructs to run and control high numbers
  of interacting concurrent tasks.
\item \urbi supports event-based programming. Triggering events and
  reacting to them is absolutely straightforward.
\item \urbi supports functional programming, inspired from languages
  such as \lisp or \caml. This includes first class functions and
  pattern matching.
\item \urbi uses a client/server architecture: the interpreter accepts
  multiple connections from different sources (human users, robots,
  other servers \ldots) and enables them to interact.
\item \urbi supports distributed architectures: you can run objects in
  different processes, potentially on another computer across the
  network.
\end{itemize}

%%% Local Variables:
%%% mode: latex
%%% TeX-master: "urbi-sdk"
%%% End:
