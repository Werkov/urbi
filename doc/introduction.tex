\chapter{Introduction}

\urbi SDK is a fully-featured environment to orchestrate complex
organizations of components.  It relies on a middleware architecture
that coordinates components named UObjects.  It also features \us, a
scripting language that can be used to write orchestration programs.

\section{\urbi and UObjects}

\urbi makes the orchestration of independent, concurrent, components
easier.  It was first designed for robotics: it provides all the
needed features to coordinate the execution of various components
(actuators, sensors, software devices that provide features such as
text-to-speech, face recognition and so forth).  Languages such as
\Cxx are well suited to program the local, low-level, handling of
these hardware or software devices; indeed one needs efficiency, small
memory footprint, and access to low-level hardware details.  Yet, when
it comes to component orchestration and coordination, in a word, when
it comes to addressing concurrency, it can be tedious to use such
languages.

Middleware infrastructures make possible to use remote components as
if they were local, to allow concurrent execution, to make synchronous
or asynchronous requests and so forth.  The \dfn{UObject} \Cxx
architecture provides exactly this: a common API that allows
conforming components to be used seamlessly in highly concurrent
settings.  Components need not be designed with UObjects in mind,
rather, UObjects are typically ``shells'' around ``regular''
components.

Components with an UObject interface are naturally supported by the
\us programming language.  This provides a tremendous help: one can
interact with these components (making queries, changing them,
observing their state, monitoring various kinds of events and so
forth), which provides a huge speed-up during development.

Finally, note that, although made with robots in mind, the UObject
architecture is well suited to tame any heavily concurrent
environment, such as video games or complex systems in general.

\section{\urbi and \us}

\us is a programming language primarily designed for robotics. It's a
dynamic, prototype-based, object-oriented scripting language. It
supports and emphasizes parallel and event-based programming, which
are very popular paradigms in robotics, by providing core primitives
and language constructs.

Its main features are:
\begin{itemize}
\item syntactically close to \Cxx. If you know \C or \Cxx, you can
  easily write \us programs.
\item fully integrated with \Cxx. You can bind \Cxx classes in \us
  seamlessly. \us is also integrated with many other languages such as
  \java, \matlab or \python.
\item object-oriented. It supports encapsulation, inheritance and
  inclusion polymorphism. Dynamic dispatching is available through
  monomethods --- just as \Cxx, \Cs or \java.
\item concurrent. It provides you with natural constructs to run and
  control high numbers of interacting concurrent tasks.
\item event-based. Triggering events and reacting to them is
  absolutely straightforward.
\item functional programming.  Inspired by languages such as \lisp or
  \caml, \us features first class functions and pattern matching.
\item client/server.  The interpreter accepts multiple connections
  from different sources (human users, robots, other servers \ldots)
  and enables them to interact.
\item distributed.  You can run objects in different processes,
  potentially remote computers across the network.
\end{itemize}

\section{Outline}

This multi-part document provides a complete guide to Urbi.  See
\autoref{sec:notations} for the various notations that are used in the
document.

\newenvironment{partDescription}[2]
{%
  \item[\autoref{#1} --- \nameref{#1}]~\\%
  #2
  \begin{description}%
    \let\itemOrig\item%
    \renewcommand{\item}[1][]{\itemOrig[~~\autoref{##1} --- \nameref{##1}]~\\}%
  }{%
  \end{description}%
}

%%% Keep sync with urbi-sdk.tex.
\begin{description}
%% Copyright (C) 2010, Gostai S.A.S.
%%
%% This software is provided "as is" without warranty of any kind,
%% either expressed or implied, including but not limited to the
%% implied warranties of fitness for a particular purpose.
%%
%% See the LICENSE file for more information.

\begin{partDescription}{part:tut}
  {%
    This part, also known as the ``\us tutorial'', teaches the reader
    how to program in
    \us.  It goes from the basis to concurrent and
    event-based programming.  No specific knowledge is expected.
    There is no need for a \Cxx compiler, as \UObject will not be
    covered here (see \autoref{part:uobject}).  The reference manual
    contains a terse and complete definition of the \urbi environment
    (\autoref{part:specs}).
    %
  }
\item[sec:tut:first]
  First contacts with \us.
\item[sec:tut:value]
  A quick introduction to objects and values.
\item[sec:tut:flow]
  Basic control flow: \lstinline{if}, \lstinline{for} and the like.
\item[sec:tut:function]
  Details about functions, scoped, and lexical closures.
\item[sec:tut:object]
  A more in-depth introduction to object-oriented programming in \us.
\item[sec:tut:functional]
  Functions are first-class citizens.
\item[sec:tut:concurrent]
  The \us operators for concurrency, tags.
\item[sec:tut:event-prog]
  Support for event-driven concurrency in \us.
\item[sec:tut:ros] How to use ROS from \urbi, and vice-versa.
\end{partDescription}


%%% Local Variables:
%%% mode: latex
%%% TeX-master: "../urbi-sdk"
%%% ispell-dictionary: "american"
%%% ispell-personal-dictionary: "../urbi.dict"
%%% fill-column: 76
%%% End:

%% Copyright (C) 2010, Gostai S.A.S.
%%
%% This software is provided "as is" without warranty of any kind,
%% either expressed or implied, including but not limited to the
%% implied warranties of fitness for a particular purpose.
%%
%% See the LICENSE file for more information.

\begin{partDescription}{part:tut}
  {%
    This part, also known as the ``\us tutorial'', teaches the reader
    how to program in
    \us.  It goes from the basis to concurrent and
    event-based programming.  No specific knowledge is expected.
    There is no need for a \Cxx compiler, as \UObject will not be
    covered here (see \autoref{part:uobject}).  The reference manual
    contains a terse and complete definition of the \urbi environment
    (\autoref{part:specs}).
    %
  }
\item[sec:tut:first]
  First contacts with \us.
\item[sec:tut:value]
  A quick introduction to objects and values.
\item[sec:tut:flow]
  Basic control flow: \lstinline{if}, \lstinline{for} and the like.
\item[sec:tut:function]
  Details about functions, scoped, and lexical closures.
\item[sec:tut:object]
  A more in-depth introduction to object-oriented programming in \us.
\item[sec:tut:functional]
  Functions are first-class citizens.
\item[sec:tut:concurrent]
  The \us operators for concurrency, tags.
\item[sec:tut:event-prog]
  Support for event-driven concurrency in \us.
\item[sec:tut:ros] How to use ROS from \urbi, and vice-versa.
\end{partDescription}


%%% Local Variables:
%%% mode: latex
%%% TeX-master: "../urbi-sdk"
%%% ispell-dictionary: "american"
%%% ispell-personal-dictionary: "../urbi.dict"
%%% fill-column: 76
%%% End:

%% Copyright (C) 2010, Gostai S.A.S.
%%
%% This software is provided "as is" without warranty of any kind,
%% either expressed or implied, including but not limited to the
%% implied warranties of fitness for a particular purpose.
%%
%% See the LICENSE file for more information.

\begin{partDescription}{part:tut}
  {%
    This part, also known as the ``\us tutorial'', teaches the reader
    how to program in
    \us.  It goes from the basis to concurrent and
    event-based programming.  No specific knowledge is expected.
    There is no need for a \Cxx compiler, as \UObject will not be
    covered here (see \autoref{part:uobject}).  The reference manual
    contains a terse and complete definition of the \urbi environment
    (\autoref{part:specs}).
    %
  }
\item[sec:tut:first]
  First contacts with \us.
\item[sec:tut:value]
  A quick introduction to objects and values.
\item[sec:tut:flow]
  Basic control flow: \lstinline{if}, \lstinline{for} and the like.
\item[sec:tut:function]
  Details about functions, scoped, and lexical closures.
\item[sec:tut:object]
  A more in-depth introduction to object-oriented programming in \us.
\item[sec:tut:functional]
  Functions are first-class citizens.
\item[sec:tut:concurrent]
  The \us operators for concurrency, tags.
\item[sec:tut:event-prog]
  Support for event-driven concurrency in \us.
\item[sec:tut:ros] How to use ROS from \urbi, and vice-versa.
\end{partDescription}


%%% Local Variables:
%%% mode: latex
%%% TeX-master: "../urbi-sdk"
%%% ispell-dictionary: "american"
%%% ispell-personal-dictionary: "../urbi.dict"
%%% fill-column: 76
%%% End:

%% Copyright (C) 2010, Gostai S.A.S.
%%
%% This software is provided "as is" without warranty of any kind,
%% either expressed or implied, including but not limited to the
%% implied warranties of fitness for a particular purpose.
%%
%% See the LICENSE file for more information.

\begin{partDescription}{part:tut}
  {%
    This part, also known as the ``\us tutorial'', teaches the reader
    how to program in
    \us.  It goes from the basis to concurrent and
    event-based programming.  No specific knowledge is expected.
    There is no need for a \Cxx compiler, as \UObject will not be
    covered here (see \autoref{part:uobject}).  The reference manual
    contains a terse and complete definition of the \urbi environment
    (\autoref{part:specs}).
    %
  }
\item[sec:tut:first]
  First contacts with \us.
\item[sec:tut:value]
  A quick introduction to objects and values.
\item[sec:tut:flow]
  Basic control flow: \lstinline{if}, \lstinline{for} and the like.
\item[sec:tut:function]
  Details about functions, scoped, and lexical closures.
\item[sec:tut:object]
  A more in-depth introduction to object-oriented programming in \us.
\item[sec:tut:functional]
  Functions are first-class citizens.
\item[sec:tut:concurrent]
  The \us operators for concurrency, tags.
\item[sec:tut:event-prog]
  Support for event-driven concurrency in \us.
\item[sec:tut:ros] How to use ROS from \urbi, and vice-versa.
\end{partDescription}


%%% Local Variables:
%%% mode: latex
%%% TeX-master: "../urbi-sdk"
%%% ispell-dictionary: "american"
%%% ispell-personal-dictionary: "../urbi.dict"
%%% fill-column: 76
%%% End:

\ifthen{\boolean{platforms}}
{
  %% Copyright (C) 2010, Gostai S.A.S.
%%
%% This software is provided "as is" without warranty of any kind,
%% either expressed or implied, including but not limited to the
%% implied warranties of fitness for a particular purpose.
%%
%% See the LICENSE file for more information.

\begin{partDescription}{part:tut}
  {%
    This part, also known as the ``\us tutorial'', teaches the reader
    how to program in
    \us.  It goes from the basis to concurrent and
    event-based programming.  No specific knowledge is expected.
    There is no need for a \Cxx compiler, as \UObject will not be
    covered here (see \autoref{part:uobject}).  The reference manual
    contains a terse and complete definition of the \urbi environment
    (\autoref{part:specs}).
    %
  }
\item[sec:tut:first]
  First contacts with \us.
\item[sec:tut:value]
  A quick introduction to objects and values.
\item[sec:tut:flow]
  Basic control flow: \lstinline{if}, \lstinline{for} and the like.
\item[sec:tut:function]
  Details about functions, scoped, and lexical closures.
\item[sec:tut:object]
  A more in-depth introduction to object-oriented programming in \us.
\item[sec:tut:functional]
  Functions are first-class citizens.
\item[sec:tut:concurrent]
  The \us operators for concurrency, tags.
\item[sec:tut:event-prog]
  Support for event-driven concurrency in \us.
\item[sec:tut:ros] How to use ROS from \urbi, and vice-versa.
\end{partDescription}


%%% Local Variables:
%%% mode: latex
%%% TeX-master: "../urbi-sdk"
%%% ispell-dictionary: "american"
%%% ispell-personal-dictionary: "../urbi.dict"
%%% fill-column: 76
%%% End:

}
%% Copyright (C) 2010, Gostai S.A.S.
%%
%% This software is provided "as is" without warranty of any kind,
%% either expressed or implied, including but not limited to the
%% implied warranties of fitness for a particular purpose.
%%
%% See the LICENSE file for more information.

\begin{partDescription}{part:tut}
  {%
    This part, also known as the ``\us tutorial'', teaches the reader
    how to program in
    \us.  It goes from the basis to concurrent and
    event-based programming.  No specific knowledge is expected.
    There is no need for a \Cxx compiler, as \UObject will not be
    covered here (see \autoref{part:uobject}).  The reference manual
    contains a terse and complete definition of the \urbi environment
    (\autoref{part:specs}).
    %
  }
\item[sec:tut:first]
  First contacts with \us.
\item[sec:tut:value]
  A quick introduction to objects and values.
\item[sec:tut:flow]
  Basic control flow: \lstinline{if}, \lstinline{for} and the like.
\item[sec:tut:function]
  Details about functions, scoped, and lexical closures.
\item[sec:tut:object]
  A more in-depth introduction to object-oriented programming in \us.
\item[sec:tut:functional]
  Functions are first-class citizens.
\item[sec:tut:concurrent]
  The \us operators for concurrency, tags.
\item[sec:tut:event-prog]
  Support for event-driven concurrency in \us.
\item[sec:tut:ros] How to use ROS from \urbi, and vice-versa.
\end{partDescription}


%%% Local Variables:
%%% mode: latex
%%% TeX-master: "../urbi-sdk"
%%% ispell-dictionary: "american"
%%% ispell-personal-dictionary: "../urbi.dict"
%%% fill-column: 76
%%% End:

\end{description}

%% Redefine this environment so that next time the */abstract.tex
%% files are read, they create the part instead of referencing to it.

\renewenvironment{partDescription}[2]
{%
  \chapter*{About This Part}
  #2
  \begin{description}%
    \let\itemOrig\item%
    \renewcommand{\item}[1][]{\itemOrig[~~\autoref{##1} --- \nameref{##1}]~\\}%
  }{%
  \end{description}%
}


%% \paragraph{Thanks}
%% The \us language what first designed and implemented by
%% Jean-Christophe Baillie, together with Matthieu Nottale.  Because
%% \urbi was widely acclaimed by its users, Jean-Christophe founded
%% Gostai, a France-based Company that develops software for robotics
%% with a strong emphasis on personal robotics.
%%
%% \us SDK 1 was further developped by Akim Demaille, Guillaume
%% Deslandes, Quentin Hocquet, Thomas Moulard, and Benoît Sigoure.
%%
%% The \us SDK 2 project was started and developped by Akim Demaille,
%% Quentin Hocquet, Matthieu Nottale, and Benoît Sigoure.  Samuel Tardieu
%% provided an immense help during the year 2008, in particular for the
%% concurrency and event support.
%%
%% The maintenance is currently carried out by Akim Demaille, Quentin
%% Hocquet, and Matthieu Nottale.  Jean-Christophe Baillie is still
%% deeply involved in the development of \us, he regularly submits ideas,
%% and occasionnally even code!

%%% Local Variables:
%%% mode: latex
%%% TeX-master: "urbi-sdk"
%%% ispell-dictionary: "american"
%%% ispell-personal-dictionary: "urbi.dict"
%%% fill-column: 76
%%% End:
