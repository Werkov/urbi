%% Copyright (C) 2010, 2011, Gostai S.A.S.
%%
%% This software is provided "as is" without warranty of any kind,
%% either expressed or implied, including but not limited to the
%% implied warranties of fitness for a particular purpose.
%%
%% See the LICENSE file for more information.

\section{Float.limits}
\label{sec:float-limits}

This singleton handles various limits related to the \refObject{Float}
objects.

\subsection{Prototypes}
\begin{refObjects}
\item[Singleton]
\end{refObjects}

\subsection{Slots}

\begin{urbiscriptapi}
\item[digits] Number of digits (in \refSlot{radix} base) in the mantissa.
\begin{urbiassert}
Float.limits.digits;
\end{urbiassert}

\item[digits10]
  Number of digits (in decimal base) that can be represented without
  change.
\begin{urbiassert}
Float.limits.digits10;
\end{urbiassert}

\item[epsilon]
  Machine epsilon (the difference between 1 and the least value
  greater than 1 that is representable).
\begin{urbiassert}
1 != 1 + Float.limits.epsilon;
1 == 1 + Float.limits.epsilon / 2;
\end{urbiassert}

\item[max]
  Maximum finite value.
\begin{urbiassert}
Float.limits.max     != Float.inf;
Float.limits.max * 2 == Float.inf;
\end{urbiassert}

\item[maxExponent]
  Maximum integer value for the exponent that generates a normalized
  floating-point number.
\begin{urbiassert}
Float.inf != Float.limits.radix ** (Float.limits.maxExponent - 1);
Float.inf == Float.limits.radix ** Float.limits.maxExponent;
\end{urbiassert}

\item[maxExponent10]
  Maximum integer value such that 10 raised to that power generates a
  normalized finite floating-point number.
\begin{urbiassert}
Float.inf != 10 ** Float.limits.maxExponent10;
Float.inf == 10 ** (Float.limits.maxExponent10 + 1);
\end{urbiassert}

\item[min]
  Minimum positive normalized value.
\begin{urbiassert}
0 != Float.limits.min;
\end{urbiassert}

\item[minExponent]
  Minimum negative integer value for the exponent that generates a
  normalized floating-point number.
\begin{urbiassert}
0 != Float.limits.radix ** Float.limits.minExponent;
\end{urbiassert}

\item[minExponent10]
  Minimum negative integer value such that 10 raised to that power
  generates a normalized floating-point number.
\begin{urbiassert}
0 != 10 ** Float.limits.minExponent10;
\end{urbiassert}

\item[radix]
  Base of the exponent of the representation.
\begin{urbiassert}
Float.limits.radix == 2;
\end{urbiassert}
\end{urbiscriptapi}

%%% Local Variables:
%%% mode: latex
%%% TeX-master: "../urbi-sdk"
%%% ispell-dictionary: "american"
%%% ispell-personal-dictionary: "../urbi.dict"
%%% fill-column: 76
%%% End:
