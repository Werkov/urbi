%% Copyright (C) 2009-2010, Gostai S.A.S.
%%
%% This software is provided "as is" without warranty of any kind,
%% either expressed or implied, including but not limited to the
%% implied warranties of fitness for a particular purpose.
%%
%% See the LICENSE file for more information.

\section{Hash}

A \dfn{hash} is a condensed, easily comparable representation of another
value. They are mainly used to map \refObject{Dictionary} keys to values.

Equal objects must always have the same hash. Different objects should, as
much as possible, have different hashes.

\subsection{Prototypes}
\begin{refObjects}
\item[Object]
\end{refObjects}

\subsection{Construction}
Objects can be hashed with \refSlot[Object]{hash}.
\begin{urbiassert}
Object.new.hash.isA(Hash);
\end{urbiassert}

\subsection{Slots}
\begin{urbiscriptapi}
\item[asFloat] A Float value equivalent to the \lstinline|Hash| object. Two
  hashes have the same Float representation if and only if they are equal.
\begin{urbiscript}
var h1 = Object.new.hash|;
var h2 = Object.new.hash|;
assert
{
  h1.asFloat == h1.asFloat;
  h1.asFloat != h2.asFloat;
};
\end{urbiscript}


\item[combine](<that>)%
  Combine \that's hash with \this, and return \this. This is used to hash
  composite objects based on more primitive object hashes. For instance, an
  object with two slots could be hashed by hashing its first first, and
  combining the second in.

\begin{urbiscript}
class C
{
  function init(var a, var b)
  {
    var this.a = a;
    var this.b = b;
  };

  function hash()
  {
    this.a.hash().combine(b)
  };
}|

assert
{
  C.new(0, 0).hash == C.new(0, 0).hash;
  C.new(0, 0).hash != C.new(0, 1).hash;
};
\end{urbiscript}
\end{urbiscriptapi}

%%% Local Variables:
%%% mode: latex
%%% TeX-master: "../urbi-sdk"
%%% ispell-dictionary: "american"
%%% ispell-personal-dictionary: "../urbi.dict"
%%% fill-column: 76
%%% End:
