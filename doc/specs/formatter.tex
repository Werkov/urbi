%% Copyright (C) 2009-2010, Gostai S.A.S.
%%
%% This software is provided "as is" without warranty of any kind,
%% either expressed or implied, including but not limited to the
%% implied warranties of fitness for a particular purpose.
%%
%% See the LICENSE file for more information.

\section{Formatter}

A \dfn{formatter} stores format information of a format string like
used in \code{printf} in the C library or in \code{boost::format}.

\subsection{Prototypes}

\begin{refObjects}
\item[Object]
\end{refObjects}

\subsection{Construction}

Formatters are created with the format string. It cuts the string to
separate regular parts of string and formatting patterns, and stores
them.

\begin{urbiscript}[firstnumber=1]
Formatter.new("Name:%s, Surname:%s;");
[00000001] Formatter ["Name:", %s, ", Surname:", %s, ";"]
\end{urbiscript}

Actually, formatting patterns are translated into
\refObject{FormatInfo}.

\subsection{Slots}

\begin{urbiscriptapi}
\item[asList]
  Return the content of the \dfn{formatter} as a list of strings and
  \refObject{FormatInfo}.
\begin{urbiassert}
Formatter.new("Name:%s, Surname:%s;").asList.asString
       == "[\"Name:\", %s, \", Surname:\", %s, \";\"]";
\end{urbiassert}

-\item \lstinline|'%'(\var{args})|\\
  Use \this as format string and \var{args} as the list of
  arguments, and return the result (a \refObject{String}).  The arity
  of the Formatter (i.e., the number of expected arguments) and the
  size of \var{args} must match exactly.

  This operator concatenates regular strings and the strings that are
  result of \lstinline|asString| called on members of \var{args} with
  the appropriate \refObject{FormatInfo}.
\begin{urbiassert}
Formatter.new("Name:%s, Surname:%s;") % ["Foo", "Bar"]
       == "Name:Foo, Surname:Bar;";
\end{urbiassert}

  If \var{args} is not a \refObject{List}, then the call is equivalent
  to calling \lstinline|'%'([\var{args}])|.
\begin{urbiassert}
Formatter.new("%06.3f") % Math.pi
       == "03.142";
\end{urbiassert}

  Note that \lstinline|String.'%'| provides a nicer interface to this
  operator:
\begin{urbiassert}
"%06.3f" % Math.pi == "03.142";
\end{urbiassert}

  It is nevertheless interesting to use the Formatter for performance
  reasons if the format is reused many times.
\begin{urbiscript}
{
  // Some large database of people.
  var people =
    [["Foo", "Bar" ],
     ["One", "Two" ],
     ["Un",  "Deux"],];
  var f = Formatter.new("Name:%7s, Surname:%7s;");
  for (var p: people)
    echo (f % p);
};
[00031939] *** Name:    Foo, Surname:    Bar;
[00031940] *** Name:    One, Surname:    Two;
[00031941] *** Name:     Un, Surname:   Deux;
\end{urbiscript}
\end{urbiscriptapi}

%%% Local Variables:
%%% coding: utf-8
%%% mode: latex
%%% TeX-master: "../urbi-sdk"
%%% ispell-dictionary: "american"
%%% ispell-personal-dictionary: "../urbi.dict"
%%% fill-column: 76
%%% End:
