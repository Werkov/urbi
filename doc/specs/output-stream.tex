%% Copyright (C) 2009-2010, Gostai S.A.S.
%%
%% This software is provided "as is" without warranty of any kind,
%% either expressed or implied, including but not limited to the
%% implied warranties of fitness for a particular purpose.
%%
%% See the LICENSE file for more information.

\section{OutputStream}

OutputStreams are used to write (possibly binary) files by hand.

\subsection{Prototypes}
\begin{refObjects}
\item[Stream]
\end{refObjects}

\subsection{Construction}
\label{sec:specs:OutputStream:ctor}

An OutputStream is a writing-interface to a file; its constructor
requires a \refObject{File}.  If the file already exists, content is
\emph{appended} to it.  Remove the file beforehand if you want to
override its content.

\begin{urbiscript}
var o1 = OutputStream.new(File.create("file.txt"));
[00000001] OutputStream_0x827000

var o2 = OutputStream.new(File.new("file.txt"));
[00000002] OutputStream_0x827000
\end{urbiscript}

When a stream (\refObject{OutputStream} or \refObject{InputStream}) is
opened on a File, that File cannot be removed.  On Unix systems, this is
handled gracefully (the references to the file are removed, but the content
is still there for the streams that were already bound to this file); so in
practice, the File appears to be removable.  On Windows, the File cannot be
removed at all.  Therefore, do not forget to close the streams you opened.

\begin{urbiscript}
o1.close;
o2.close;
\end{urbiscript}

\subsection{Slots}

\begin{urbiscriptapi}
\item[<<](<that>)%
  Output \lstinline|\var{this}.asString|.  Return \this to
  enable chains of calls.  Raise an error if the file is closed.
\begin{urbiscript}
var o = OutputStream.new(File.create("fresh.txt"))|;
o << 1 << "2" << [3, [4]]|;
o.close;
assert (File.new("fresh.txt").content.data == "12[3, [4]]");
o << 1;
[00000005:error] !!! <<: stream is closed
\end{urbiscript}

\item[flush]%
  To provide efficient input/output operations, \dfn[buffer]{buffers} are
  used.  As a consequence, what is put into a stream might not be
  immediately saved on the actual file.  To \dfn{flush} a buffer means to
  dump its content to the file.  Raise an error if the file is closed.
\begin{urbiscript}
var s = OutputStream.new(File.create("file.txt"))|
s.flush;
s.close;
s.flush;
[00039175:error] !!! flush: stream is closed
\end{urbiscript}

\item[put](<byte>)%
  Output the character corresponding to the numeric code \var{byte} in
  \this, and return \this.  Raise an error if the file is closed.
\begin{urbiscript}
var f = File.create("put.txt") |
var os = OutputStream.new(f) |

assert
{
  os.put(0)
    .put(255)
    .put(72).put(101).put(108).put(108).put(111)
  === os;
  f.content.data == "\0\xffHello";
};
os.put(12.5);
[00029816:error] !!! put: argument 1: bad numeric conversion: overflow or non empty fractional part: 12.5
os.put(-1);
[00034840:error] !!! put: argument 1: bad numeric conversion: negative overflow: -1
os.put(256);
[00039175:error] !!! put: argument 1: bad numeric conversion: positive overflow: 256
os.close;
os.put(0);
[00039179:error] !!! put: stream is closed
\end{urbiscript}

\end{urbiscriptapi}


%%% Local Variables:
%%% coding: utf-8
%%% mode: latex
%%% TeX-master: "../urbi-sdk"
%%% ispell-dictionary: "american"
%%% ispell-personal-dictionary: "../urbi.dict"
%%% fill-column: 76
%%% End:
