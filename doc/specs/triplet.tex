\section{Triplet}

A \dfn{triplet} (or \dfn{triple}) is a container storing three
objects.


\subsection{Prototype}
\begin{refObjects}
\item[Tuple]
\end{refObjects}

\subsection{Construction}

A \lstinline|Triplet| is constructed with three arguments.

\begin{urbiscript}[firstnumber=1]
Triplet.new(1, 2, 3);
[00000001] (1, 2, 3)

Triplet.new(1, 2);
[00000003:error] !!! Triplet.init: expected 3 arguments, given 2

Triplet.new(1, 2, 3, 4);
[00000003:error] !!! Triplet.init: expected 3 arguments, given 4
\end{urbiscript}

\subsection{Slots}
\begin{urbiscriptapi}
\item[first]
  Return the first member of the pair.
\begin{urbiassert}
Triplet.new(1, 2, 3).first == 1;
Triplet[0] === Triplet.first;
\end{urbiassert}

\item[second]
  Return the second member of the triplet.
\begin{urbiassert}
Triplet.new(1, 2, 3).second == 2;
Triplet[1] === Triplet.second;
\end{urbiassert}

\item[third]
  Return the third member of the triplet.
\begin{urbiassert}
Triplet.new(1, 2, 3).third == 3;
Triplet[2] === Triplet.third;
\end{urbiassert}
\end{urbiscriptapi}



%%% Local Variables:
%%% mode: latex
%%% TeX-master: "../urbi-sdk"
%%% ispell-dictionary: "american"
%%% ispell-personal-dictionary: "../urbi.dict"
%%% fill-column: 76
%%% End:
