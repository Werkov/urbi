\section{Dictionary}

A \dfn{dictionary} is an \dfn{associative array}, also known as a
\dfn{hash} in some programming languages.  They are arrays whose
indexes are strings.

In a way objects are dictionaries: one can use \lstinline|setSlot|,
\lstinline|updateSlot|, and \lstinline|getSlot|.  This is unsafe since
slots also contains value and methods that object depend upon to run
properly.

\begin{itemize}
\item \lstinline|asList|\\
  Return the contents of the dictionary as a \refObject{Pair} list
  (\var{key}, \var{value}).  This makes it easier to iterate over a
  Dictionary using \lstinline|for|.  No particular order is ensured.

\item \lstinline|clear|\\
  Empty the dictionary.

\item \lstinline|empty|\\
  Whether the dictionary is empty.

\item \lstinline|erase(\var{key})|\\
  Remove the mapping for \lstinline|\var{key}|.

\item \lstinline|get(\var{key})|\\
  Return the value associated to  \lstinline|\var{key}| if it exists,
  \lstinline|void| otherwise.

\item \lstinline|getWithDefault(\var{key}, \var{default-value})|\\
  Return the value associated to  \lstinline|\var{key}| if it exists,
  \lstinline|\var{default-value}| otherwise.

\item \lstinline|has(\var{key})|\\
  Whether the dictionary has a mapping for \lstinline|\var{key}|.

\item \lstinline|init(\var{key1}, \var{value1}, ...)|~\\
  Instantiate a new \lstinline|Dictionary|.  Insert the mapping from
  \lstinline|\var{key1}| to \lstinline|\var{value1}| and so forth.

\item \lstinline|keys|\\
  The list of all the keys.  No particular order is ensured.

\item \lstinline|set(\var{key}, \var{value})|\\
  Map \lstinline|\var{key}| to \lstinline|\var{value}| and return
  \lstinline|this| so that invocations to \lstinline|set| can be
  chained.  The possibly existing previous mapping is overriden.

\item \lstinline|[\var{key}]=\var{value}|\\
  Syntactic sugar for \lstinline|set(\var{key}, \var{value})|.

\item \lstinline|[\var{key}]|\\
  Syntactic sugar for \lstinline|get(\var{key})|.
\end{itemize}

See \autoref{lst:specs:dictionary} for an example.

\begin{urbiscript}[caption=Dictionaries, label=lst:specs:dictionary]
var d = Dictionary.new("one", 1, "two", 2);
[00000001] Dictionary {"one" => 1, "two" => 2}
/*HIDE*/ //Make the order deterministic.
/*HIDE*/ Dictionary.asList =
/*HIDE*/ function ()
/*HIDE*/ {
/*HIDE*/   var res = []|
/*HIDE*/   for| (var k: keys.sort)
/*HIDE*/     res.insertBack(Pair.new (k, get (k)))|
/*HIDE*/   res
/*HIDE*/ } | {};
for (var p in d)
  echo (p.first + " -> " + p.second);
[00000003] *** one -> 1
[00000002] *** two -> 2
"three" in d;
[00000004] false
d["three"] = d["one"] + d["two"] | {};
"three" in d;
[00000005] true
d.getWithDefault("four", 4);
[00000006] 4
\end{urbiscript}

%%% Local Variables:
%%% mode: latex
%%% TeX-master: "../urbi-sdk"
%%% End:
