%% Copyright (C) 2009-2010, Gostai S.A.S.
%%
%% This software is provided "as is" without warranty of any kind,
%% either expressed or implied, including but not limited to the
%% implied warranties of fitness for a particular purpose.
%%
%% See the LICENSE file for more information.

\section{Dictionary}

A \dfn{dictionary} is an \dfn{associative array}, also known as a
\dfn{hash} in some programming languages.  They are arrays whose
indexes are strings.

In a way objects are dictionaries: one can use \lstinline|setSlot|,
\lstinline|updateSlot|, and \lstinline|getSlot|.  This is unsafe since
slots also contains value and methods that object depend upon to run
properly.

\subsection{Example}

The following session demonstrates the features of the Dictionary
objects.

\begin{urbiscript}[firstnumber=1]
var d = ["one" => 1, "two" => 2];
[00000001] ["one" => 1, "two" => 2]
for (var p : d)
  echo (p.first + " => " + p.second);
[00000003] *** one => 1
[00000002] *** two => 2
"three" in d;
[00000004] false
d["three"];
[00000005:error] !!! missing key: three
d["three"] = d["one"] + d["two"] | {};
"three" in d;
[00000006] true
d.getWithDefault("four", 4);
[00000007] 4
\end{urbiscript}

\subsection{Prototypes}

\begin{refObjects}
\item[Comparable]
\item[Container]
\item[Object]
\item[RangeIterable]
\end{refObjects}

\subsection{Construction}

The Dictionary constructor takes arguments by pair (key, value).

\begin{urbiscript}
Dictionary.new("one", 1, "two", 2);
[00000000] ["one" => 1, "two" => 2]
Dictionary.new;
[00000000] [ => ]
\end{urbiscript}

Yet you are encouraged to use the specific syntax for Dictionary
literals:

\begin{urbiscript}
["one" => 1, "two" => 2];
[00000000] ["one" => 1, "two" => 2]
[=>];
[00000000] [ => ]
\end{urbiscript}

An extra comma can be added at the end of the list.

\begin{urbiscript}
[
  "one" => 1,
  "two" => 2,
];
[00000000] ["one" => 1, "two" => 2]
\end{urbiscript}

\subsection{Slots}

\begin{urbiscriptapi}
\item['=='](<that>)%
  Whether \this equals \var{that}.  This suppose that elements
  contained inside the dictionary are \lstinline|Comparable|.
\begin{urbiassert}
[ => ] == [ => ];
["a" => 1, "b" => 2] == ["b" => 2, "a" => 1];
\end{urbiassert}


\item|'[]'|(<key>)
  Syntactic sugar for \lstinline|get(\var{key})|.

\begin{urbiassert}
["one" => 1]["one"] == 1;
\end{urbiassert}


\item|'[]='|(<key>, <value>)%
  Syntactic sugar for \lstinline|set(\var{key}, \var{value})|, but returns
  \var{value}.

\begin{urbiscript}
{
  var d = ["one" =>"2"];
  assert
  {
    (d["one"] = 1) == 1;
    d["one"] == 1;
  };
};
\end{urbiscript}


\item[asBool]
  Negation of \refSlot{empty}.
\begin{urbiassert}
[=>].asBool == false;
["key" => "value"].asBool == true;
\end{urbiassert}


\item[asList]
  Return the contents of the dictionary as a \refObject{Pair} list
  (\var{key}, \var{value}).

\begin{urbiassert}
["one" => 1, "two" => 2].asList == [("one", 1), ("two", 2)];
\end{urbiassert}

  \noindent
  Since Dictionary derives from \refObject{RangeIterable}, it is easy
  to iterate over a Dictionary using a range-\lstinline|for|
  (\autoref{sec:lang:for:each}).  No particular order is ensured.
\begin{urbiscript}
{
  var res = [];
  for| (var entry: ["one" => 1, "two" => 2])
    res << entry.second;
  assert(res == [1, 2]);
};
\end{urbiscript}


\item[asString] A string representing the dictionary.  There is no guarantee
  on the order of the output.
\begin{urbiassert}
                [=>].asString == "[ => ]";
["a" => 1, "b" => 2].asString == "[\"a\" => 1, \"b\" => 2]";
\end{urbiassert}


\item[clear]
  Empty the dictionary.

\begin{urbiassert}
["one" => 1].clear.empty;
\end{urbiassert}


\item[empty]
  Whether the dictionary is empty.

\begin{urbiassert}
[=>].empty == true;
["key" => "value"].empty == false;
\end{urbiassert}


\item[erase](<key>)
  Remove the mapping for \lstinline|\var{key}|.

\begin{urbiassert}
["one" => 1, "two" => 2].erase("two") == ["one" => 1]
\end{urbiassert}

%% commented until a consensus is reached.
%%


%% \item[extend](<ext>)
%%   Extend with the dictionary \lstinline|\var{ext}|.
%%   Return the value of the new dictionary.
%% \begin{urbiscript}
%% d = ["one" => 1, "two" => 2];
%% [00000001] ["one" => 1, "two" => 2]
%% d.extend(["one" => 0, "three" => 3]);
%% [00000002] ["one" => 0, "three" => 3, "two" => 2]
%% \end{urbiscript}


\item[get](<key>)
  Return the value associated to \lstinline|\var{key}|.  A
  Dictionary.KeyError exception is thrown if the key is missing.

  % FIXME: the following exception test should be rewritten when (if)
  % we introduce the throw assertion.
\begin{urbiscript}
assert(["one" => 1, "two" => 2].get("one") == 1);
try
{
  ["one" => 1, "two" => 2].get("three");
  echo("never reached");
}
catch (var e if e.isA(Dictionary.KeyError))
{
  assert(e.key == "three")
};
\end{urbiscript}


\item[getWithDefault](<key>, <defaultValue>)%
  The value associated to \var{key} if it exists, \var{defaultValue}
  otherwise.

\begin{urbiscript}
do (["one" => 1, "two" => 2])
{
  assert
  {
    getWithDefault("one",  -1) == 1;
    getWithDefault("three", 3) == 3;
  };
}|;
\end{urbiscript}


\item[has](<key>)
  Whether the dictionary has a mapping for \var{key}.

\begin{urbiscript}
do (["one" => 1])
{
  assert(has("one"));
  assert(!has("zero"));
}|;
\end{urbiscript}

  The infix operators \lstinline|in| and \lstinline|not in| use
  \lstinline|has| (see \autoref{sec:lang:operators:containers}).

\begin{urbiassert}
"one" in     ["one" => 1];
"two" not in ["one" => 1];
\end{urbiassert}


\item[init](<key1>, <value1>, ...)%~\\
  Insert the mapping from \var{key1} to \var{value1} and so forth.

\begin{urbiscript}
Dictionary.clone.init("one", 1, "two", 2);
[00000000] ["one" => 1, "two" => 2]
\end{urbiscript}


\item[keys]\\
  The list of all the keys.  No particular order is ensured.  Since
  \refObject{List} features the same function, uniform iteration over
  a List or a Dictionary is possible.
\begin{urbiscript}
{
  var d = ["one" => 1, "two" => 2];
  assert(d.keys == ["one", "two"]);
  assert({
           var res = [];
           for (var k: d.keys)
             res << d[k];
           res
         }
         == [1, 2]);
};
\end{urbiscript}


\item[matchAgainst](<handler>, <pattern>)
  Pattern matching on members.  See \refObject{Pattern}.

\begin{urbiscript}
{
  // Match a subset of the dictionary.
  ["a" => var a] = ["a" => 1, "b" => 2];
  // get the matched value.
  assert(a == 1);
};
\end{urbiscript}


\item[set](<key>, <value>)
  Map \lstinline|\var{key}| to \lstinline|\var{value}| and return
  \this so that invocations to \lstinline|set| can be
  chained.  The possibly existing previous mapping is overridden.

\begin{urbiscript}
[=>].set("one", 2).set("one", 1);
[00000000] ["one" => 1]
\end{urbiscript}


\item[size]
  Number of element in the dictionary.

\begin{urbiscript}
{
  var d = [=>];
  assert(d.size == 0);
  d["a"] = 0;
  assert(d.size == 1);
  d["b"] = 1;
  assert(d.size == 2);
  d["a"] = 2;
  assert(d.size == 2);
};
\end{urbiscript}



\end{urbiscriptapi}


%%% Local Variables:
%%% mode: latex
%%% TeX-master: "../urbi-sdk"
%%% ispell-dictionary: "american"
%%% ispell-personal-dictionary: "../urbi.dict"
%%% fill-column: 76
%%% End:
