%% Copyright (C) 2009-2011, Gostai S.A.S.
%%
%% This software is provided "as is" without warranty of any kind,
%% either expressed or implied, including but not limited to the
%% implied warranties of fitness for a particular purpose.
%%
%% See the LICENSE file for more information.

\section{Dictionary}

A \dfn{dictionary} is an \dfn{associative array}, also known as a \dfn{hash}
in some programming languages.  They are arrays whose indexes are arbitrary
objects.

\subsection{Example}

The following session demonstrates the features of the Dictionary objects.

\begin{urbiscript}[firstnumber=1]
var d = ["one" => 1, "two" => 2];
[00000001] ["one" => 1, "two" => 2]

for (var p : d)
  echo (p.first + " => " + p.second);
[00000003] *** one => 1
[00000002] *** two => 2

"three" in d;
[00000004] false
d["three"];
[00000005:error] !!! missing key: three
d["three"] = d["one"] + d["two"]|;
"three" in d;
[00000006] true
d.getWithDefault("four", 4);
[00000007] 4
\end{urbiscript}

\subsection{Hash values}
\label{sec:dictionary:hash}

Arbitrary objects can be used as dictionary keys. To map to the same cell,
two objects used as keys must have equal hashes (retrieved with the
\refSlot[Object]{hash} method) and be equal to each other (in the
\refSlot[Object]{'=='} sense).

This means that two different objects may have the same hash: the equality
operator (\refSlot[Object]{'=='}) is checked in addition to the hash, to
handle such collision.  However a good hash algorithm should avoid this
case, since it hinders performances.

See \refSlot[Object]{hash} for more detail on how to override hash
values. Most standard value-based classes implement a reasonable hash
function: see \refSlot[Float]{hash}, \refSlot[String]{hash},
\refSlot[List]{hash}, \ldots

\subsection{Prototypes}

\begin{refObjects}
\item[Comparable]
\item[Container]
\item[Object]
\item[RangeIterable]
\end{refObjects}

\subsection{Construction}

The Dictionary constructor takes arguments by pair (key, value).

\begin{urbiscript}
Dictionary.new("one", 1, "two", 2);
[00000000] ["one" => 1, "two" => 2]
Dictionary.new;
[00000000] [ => ]
\end{urbiscript}

There must be an even number of arguments.

\begin{urbiscript}
Dictionary.new("1", 2, "3");
[00000001:error] !!! new: odd number of arguments
\end{urbiscript}

You are encouraged to use the specific syntax for Dictionary literals:

\begin{urbiscript}
["one" => 1, "two" => 2];
[00000000] ["one" => 1, "two" => 2]
[=>];
[00000000] [ => ]
\end{urbiscript}

An extra comma can be added at the end of the list.

\begin{urbiscript}
[
  "one" => 1,
  "two" => 2,
];
[00000000] ["one" => 1, "two" => 2]
\end{urbiscript}

It is guaranteed that the pairs to insert are evaluated left-to-write, key
first, the value.

\begin{urbiassert}
   ["a".fresh => "b".fresh, "c".fresh => "d".fresh]
== ["a_5"     => "b_6",     "c_7"     => "d_8"];
\end{urbiassert}

Duplicate keys in Dictionary literal are an error.
On this regards, \us departs from choices made in JavaScript, Perl, Python,
Ruby, and probably many other languages.

\begin{urbiscript}
["one" => 1, "one" => 2];
[00000001:error] !!! duplicate dictionary key: "one"
\end{urbiscript}


\subsection{Slots}

\begin{urbiscriptapi}
\item['=='](<that>)%
  Whether \this equals \var{that}.  Expects members to be
  \refObject{Comparable}.
\begin{urbiassert}
[ => ] == [ => ];
["a" => 1, "b" => 2] == ["b" => 2, "a" => 1];
\end{urbiassert}


\item|'[]'|(<key>)%
  Syntactic sugar for \lstinline|get(\var{key})|.

\begin{urbiscript}
assert (["one" => 1]["one"] == 1);
["one" => 1]["two"];
[00000012:error] !!! missing key: two
\end{urbiscript}


\item|'[]='|(<key>, <value>)%
  Syntactic sugar for \lstinline|set(\var{key}, \var{value})|, but returns
  \var{value}.

\begin{urbiassert}
var d = ["one" =>"2"];
(d["one"] = 1) == 1;
d["one"] == 1;
\end{urbiassert}


\item[asBool]
  Negation of \refSlot{empty}.
\begin{urbiassert}
[=>].asBool == false;
["key" => "value"].asBool == true;
\end{urbiassert}


\item[asList]%
  The contents of the dictionary as a \refObject{Pair} list (\var{key},
  \var{value}).

\begin{urbiassert}
["one" => 1, "two" => 2].asList == [("one", 1), ("two", 2)];
\end{urbiassert}

  \noindent
  Since Dictionary derives from \refObject{RangeIterable}, it is easy to
  iterate over a Dictionary using a range-\lstinline|for|
  (\autoref{sec:lang:foreach}).  No particular order is ensured.
\begin{urbiscript}
{
  var res = [];
  for| (var entry: ["one" => 1, "two" => 2])
    res << entry.second;
  assert(res == [1, 2]);
};
\end{urbiscript}


\item[asString] A string representing the dictionary.  There is no guarantee
  on the order of the output.
\begin{urbiassert}
                [=>].asString == "[ => ]";
["a" => 1, "b" => 2].asString == "[\"a\" => 1, \"b\" => 2]";
\end{urbiassert}

\item[asTree]%
Display the content of the List as a tree representation.
\begin{urbiscript}
echo("simple dictionary:" + ["key1" => "elt1", "key2" => ["key3" => "elt3"]].asTree);
[:][00000001] *** simple dictionary:
[:][
[:]  key1 => elt1,
[:]  key2 =>
[:]  [
[:]    key3 => elt3,
[:]  ]
[:]]
echo("dictionary with list:" +
  ["key1" => "elt1", "key2" => ["key3" => ["key4", "key5"]]].asTree);
[:][00000002] *** dictionary with list:
[:][
[:]  key1 => elt1,
[:]  key2 =>
[:]  [
[:]    key3 =>
[:]    [
[:]      key4,
[:]      key5,
[:]    ]
[:]  ]
[:]]
\end{urbiscript}

\item[clear]
  Empty the dictionary.

\begin{urbiassert}
["one" => 1].clear.empty;
\end{urbiassert}


\item[elementAdded] An event emitted each time a new element is added to
  the Dictionary.


\item[elementChanged] An event emitted each time the value associated to a
  key of the Dictionary is changed.


\item[elementRemoved] An event emitted each time an element is removed from
  the Dictionary.

\begin{urbiscript}
d = [ => ] |;
at(d.elementAdded?) echo ("added");
at(d.elementChanged?) echo ("changed");
at(d.elementRemoved?) echo ("removed");

d["key1"] = "value1";
[00000001] "value1"
[00000001] *** added

d["key2"] = "value2";
[00000001] "value2"
[00000001] *** added

d["key2"] = "value3";
[00000001] "value3"
[00000001] *** changed

d.erase("key2");
[00000002] ["key1" => "value1"]
[00000001] *** removed

d.clear;
[00000003] [ => ]
[00000001] *** removed

d.clear;
[00000003] [ => ]
\end{urbiscript}


\item[empty]
  Whether the dictionary is empty.

\begin{urbiassert}
[=>].empty == true;
["key" => "value"].empty == false;
\end{urbiassert}


\item[erase](<key>) Remove the mapping for \var{key}.
\begin{urbicomment}
removeSlot("d")|;
\end{urbicomment}
\begin{urbiscript}
{
  var d = ["one" => 1, "two" => 2];
  assert
  {
    d.erase("two") === d;
    d == ["one" => 1];
  };

  try
  {
    ["one" => 1, "two" => 2].erase("three");
    echo("never reached");
  }
  catch (var e if e.isA(Dictionary.KeyError))
  {
    assert(e.key == "three")
  };
};
\end{urbiscript}

%% commented until a consensus is reached.
%%


%% \item[extend](<ext>)
%%   Extend with the dictionary \var{ext}.
%%   Return the value of the new dictionary.
%% \begin{urbiscript}
%% d = ["one" => 1, "two" => 2];
%% [00000001] ["one" => 1, "two" => 2]
%% d.extend(["one" => 0, "three" => 3]);
%% [00000002] ["one" => 0, "three" => 3, "two" => 2]
%% \end{urbiscript}


\item[get](<key>)%
  The value associated to \var{key}.  A \lstinline|Dictionary.KeyError|
  exception is thrown if the key is missing.
  % FIXME: the following exception test should be rewritten when (if)
  % we introduce the throw assertion.
\begin{urbiscript}
var d = ["one" => 1, "two" => 2]|;

assert(d.get("one") == 1);
["one" => 1, "two" => 2].get("three");
[00000010:error] !!! missing key: three

try
{
  d.get("three");
  echo("never reached");
}
catch (var e if e.isA(Dictionary.KeyError))
{
  assert(e.key == "three")
};
\end{urbiscript}


\item[getWithDefault](<key>, <defaultValue>)%
  The value associated to \var{key} if it exists, \var{defaultValue}
  otherwise.

\begin{urbiassert}
var d = ["one" => 1, "two" => 2];
d.getWithDefault("one",  -1) == 1;
d.getWithDefault("three", 3) == 3;
\end{urbiassert}


\item[has](<key>)%
  Whether the dictionary has a mapping for \var{key}.

\begin{urbiassert}
var d = ["one" => 1];
d.has("one");
!d.has("zero");
\end{urbiassert}

  The infix operators \lstinline|in| and \lstinline|not in| use
  \lstinline|has| (see \autoref{sec:lang:op:containers}).

\begin{urbiassert}
"one" in     ["one" => 1];
"two" not in ["one" => 1];
\end{urbiassert}


\item[init](<key1>, <value1>, ...)%
  Insert the mapping from \var{key1} to \var{value1} and so forth.

\begin{urbiscript}
Dictionary.clone.init("one", 1, "two", 2);
[00000000] ["one" => 1, "two" => 2]
\end{urbiscript}


\item[keys]%
  The list of all the keys.  No particular order is ensured.  Since
  \refObject{List} features the same function, uniform iteration over
  a List or a Dictionary is possible.
\begin{urbiassert}
var d = ["one" => 1, "two" => 2];
d.keys == ["one", "two"];
\end{urbiassert}


\item[matchAgainst](<handler>, <pattern>)
  Pattern matching on members.  See \refObject{Pattern}.

\begin{urbiscript}
{
  // Match a subset of the dictionary.
  ["a" => var a] = ["a" => 1, "b" => 2];
  // get the matched value.
  assert(a == 1);
};
\end{urbiscript}


\item[set](<key>, <value>)%
  Map \var{key} to \var{value} and return \this so that invocations to
  \refSlot{set} can be chained.  The possibly existing previous mapping is
  overridden.

\begin{urbiscript}
[=>].set("one", 2)
    .set("two", 2)
    .set("one", 1);
[00000000] ["one" => 1, "two" => 2]
\end{urbiscript}


\item[size]
  Number of element in the dictionary.

\begin{urbiassert}
var d = [=>];  d.size == 0;
d["a"] = 10;   d.size == 1;
d["b"] = 20;   d.size == 2;
d["a"] = 30;   d.size == 2;
\end{urbiassert}
\end{urbiscriptapi}


%%% Local Variables:
%%% coding: utf-8
%%% mode: latex
%%% TeX-master: "../urbi-sdk"
%%% ispell-dictionary: "american"
%%% ispell-personal-dictionary: "../urbi.dict"
%%% fill-column: 76
%%% End:
