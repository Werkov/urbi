\section{Dictionary}

A \dfn{dictionary} is an \dfn{associative array}, also known as a
\dfn{hash} in some programming languages.  They are arrays whose
indexes are strings.

In a way objects are dictionaries: one can use \lstinline|setSlot|,
\lstinline|updateSlot|, and \lstinline|getSlot|.  This is unsafe since
slots also contains value and methods that object depend upon to run
properly.

\subsection{Prototypes}

\begin{itemize}
\item \refObject{Object}
\end{itemize}

\subsection{Construction}

Dictionnary are created like any other object. The constructor can
take couples of argument to store them in the dictionary.

\begin{urbiscript}
var d = Dictionary.new("one", 1, "two", 2);
[00000000] Dictionary {"one" => 1, "two" => 2}
\end{urbiscript}

\subsection{Methods}

\begin{itemize}
\item \lstinline|asList|\\
  Return the contents of the dictionary as a \refObject{Pair} list
  (\var{key}, \var{value}).  This makes it easier to iterate over a
  Dictionary using \lstinline|for|.  No particular order is ensured.

\begin{urbiscript}[firstnumber=last]
d.asList;
[00000000] [("one", 1), ("two", 2)]
\end{urbiscript}

\item \lstinline|clear|\\
  Empty the dictionary.

\begin{urbiscript}[firstnumber=last]
d.clear;
[00102179] Dictionary {}
\end{urbiscript}

\item \lstinline|empty|\\
  Whether the dictionary is empty.

\begin{urbiscript}[firstnumber=last]
assert(d.empty);
\end{urbiscript}

\item \lstinline|erase(\var{key})|\\
  Remove the mapping for \lstinline|\var{key}|.

\begin{urbiscript}[firstnumber=last]
d["un"] = 1 | d["deux"] = 2 | d["trois"] = 3;
[00329262] Dictionary {"deux" => 2, "trois" => 3, "un" => 1}
d.erase("trois");
[00386250] Dictionary {"deux" => 2, "un" => 1}
\end{urbiscript}

\item \lstinline|get(\var{key})|\\
  Return the value associated to  \lstinline|\var{key}| if it exists,
  \lstinline|void| otherwise.

\begin{urbiscript}[firstnumber=last]
d.get("un");
[00000000] 1
\end{urbiscript}


\item \lstinline|getWithDefault(\var{key}, \var{default-value})|\\
  Return the value associated to  \lstinline|\var{key}| if it exists,
  \lstinline|\var{default-value}| otherwise.

\begin{urbiscript}[firstnumber=last]
d.getWithDefault("deux", "rien");
[00000000] 2
d.getWithDefault("quatre", "rien");
[00000000] "rien"
\end{urbiscript}


\item \lstinline|has(\var{key})|\\
  Whether the dictionary has a mapping for \lstinline|\var{key}|.

\begin{urbiscript}[firstnumber=last]
assert(d.has("un"));
assert(!(d.has("zero")));
\end{urbiscript}

\item \lstinline|init(\var{key1}, \var{value1}, ...)|~\\
  Insert the mapping from \lstinline|\var{key1}| to
  \lstinline|\var{value1}| and so forth.

\begin{urbiscript}[firstnumber=last]
d.init("quatre", 4, "cinq", 5, "six", 6);
[00000000] Dictionary {"cinq" => 5, "deux" => 2, "quatre" => 4, "six" => 6, "un" => 1}
\end{urbiscript}

\item \lstinline|keys|\\
  The list of all the keys.  No particular order is ensured.

\item \lstinline|set(\var{key}, \var{value})|\\
  Map \lstinline|\var{key}| to \lstinline|\var{value}| and return
  \lstinline|this| so that invocations to \lstinline|set| can be
  chained.  The possibly existing previous mapping is overriden.

\begin{urbiscript}[firstnumber=last]
d.set("deux", "two").set("zero", 0);
[00000000] Dictionary {"cinq" => 5, "deux" => "two", "quatre" => 4, "six" => 6, "un" => 1, "zero" => 0}
\end{urbiscript}

\item \lstinline|[]=(\var{key}, \var{value})|\\
  Syntactic sugar for \lstinline|set(\var{key}, \var{value})|.

\begin{urbiscript}[firstnumber=last]
d["un"] = "hein ?";
[00000000] Dictionary {"cinq" => 5, "deux" => "two", "quatre" => 4, "six" => 6, "un" => "hein ?", "zero" => 0}
\end{urbiscript}

\item \lstinline|[](\var{key})|\\
  Syntactic sugar for \lstinline|get(\var{key})|.

\begin{urbiscript}[firstnumber=last]
d["cinq"];
[02123829] 5
\end{urbiscript}

\end{itemize}

\subsection{Example}

\begin{urbiscript}
var d = Dictionary.new("one", 1, "two", 2);
[00000001] Dictionary {"one" => 1, "two" => 2}
for (var p in d)
  echo (p.first + " -> " + p.second);
[00000003] *** one -> 1
[00000002] *** two -> 2
"three" in d;
[00000004] false
d["three"] = d["one"] + d["two"] | {};
"three" in d;
[00000005] true
d.getWithDefault("four", 4);
[00000006] 4
\end{urbiscript}

%%% Local Variables:
%%% mode: latex
%%% TeX-master: "../urbi-sdk"
%%% End:
