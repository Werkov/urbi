%% Copyright (C) 2009-2010, Gostai S.A.S.
%%
%% This software is provided "as is" without warranty of any kind,
%% either expressed or implied, including but not limited to the
%% implied warranties of fitness for a particular purpose.
%%
%% See the LICENSE file for more information.

\section{void}

The special entity \lstinline|void| is an object used to denote ``no
value''.  It has no prototype and cannot be used as a value.  In
contrast with \refObject{nil}, which is a valid object,
\lstinline|void| denotes a value one is not allowed to read.

\subsection{Prototypes}

None.

\subsection{Construction}

\lstinline|void| is the value returned by constructs that return no
value.

\begin{urbiassert}[firstnumber=1]
void.isVoid;
{}.isVoid;
{if (false) 123}.isVoid;
\end{urbiassert}

\subsection{Slots}

\begin{urbiscriptapi}
\item[acceptVoid]
  Trick \this so that, even if it is \lstinline|void| it
  can be used as a value.  See also \lstinline|unacceptVoid|.
\begin{urbiscript}
void.foo;
[00096374:error] !!! unexpected void
void.acceptVoid.foo;
[00102358:error] !!! lookup failed: foo
\end{urbiscript}

\item[isVoid]
  Whether \this is \lstinline|void|.  Therefore, return
  \lstinline|true|.
\begin{urbiassert}
void.isVoid;
void.acceptVoid.isVoid;
! 123.isVoid;
\end{urbiassert}

\item[unacceptVoid]
  Remove the magic from \this that allowed to manipulate it
  as a value, even if it \lstinline|void|.  See also
  \lstinline|acceptVoid|.
\begin{urbiscript}
void.acceptVoid.unacceptVoid.foo;
[00096374:error] !!! unexpected void
\end{urbiscript}

\end{urbiscriptapi}




%%% Local Variables:
%%% coding: utf-8
%%% mode: latex
%%% TeX-master: "../urbi-sdk"
%%% ispell-dictionary: "american"
%%% ispell-personal-dictionary: "../urbi.dict"
%%% fill-column: 76
%%% End:
