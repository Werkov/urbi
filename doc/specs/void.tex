\section{void}

The special entity \lstinline|void| is an object used to denote ``no
value''.  It has no prototype and cannot be used as a value.  In
contrast with \refObject{nil}, which is a valid object,
\lstinline|void| denotes a value one is not allowed to read.

\subsection{Prototypes}

\begin{itemize}
\item \refObject{void}
\end{itemize}

\subsection{Construction}

\lstinline|void| is the value returned by constructs that return no
value.

\begin{urbiscript}
assert(void.isVoid);
assert({}.isVoid);
assert({if (false) 123}.isVoid);
\end{urbiscript}

\subsection{Slots}

\begin{itemize}
\item \lstinline|acceptVoid|\\
  Trick \lstinline|this| so that, even if it is \lstinline|void| it
  can be used as a value.  See also \lstinline|unacceptVoid|.
\begin{urbiscript}[firstnumber=last]
void.foo;
[00096374:error] !!! unexpected void
void.acceptVoid.foo;
[00102358:error] !!! lookup failed: foo
\end{urbiscript}

\item \lstinline|isVoid|\\
  Whether \lstinline|this| is \lstinline|void|.  Therefore, return
  \lstinline|true|.
\begin{urbiscript}[firstnumber=last]
assert(void.isVoid);
assert(void.acceptVoid.isVoid);
assert(! 123.isVoid);
\end{urbiscript}

\item \lstinline|unacceptVoid|\\
  Remove the magic from \lstinline|this| that allowed to manipulate it
  as a value, even if it \lstinline|void|.  See also
  \lstinline|acceptVoid|.
\begin{urbiscript}[firstnumber=last]
void.acceptVoid.unacceptVoid.foo;
[00096374:error] !!! unexpected void
\end{urbiscript}

\end{itemize}




%%% Local Variables:
%%% mode: latex
%%% TeX-master: "../urbi-sdk"
%%% End:
