\section{Singleton}

A \dfn{singleton} is a prototype that cannot be cloned. All prototypes
derived of \lstinline{Singleton} are also singletons.

\subsection{Prototypes}
\begin{itemize}
\item \refObject{Object}
\end{itemize}

\subsection{Construction}

To be a singleton, the object must have \lstinline{Singleton} as a
prototype. The common way to do this is
%
\lstinline{var s = Singleton.new},
%
but this does not work : \lstinline|s| is not a new singleton, it is
the \lstinline|Singleton| itself since it cannot be cloned. There are
two other ways:

\begin{urbiscript}
// Defining a new class and specifying Singleton as a parent.
class NewSingleton1: Singleton
{
  var asString = "NewSingleton1";
}|
var s1 = NewSingleton1.new;
[00000001] NewSingleton1
assert(s1 === NewSingleton1);
assert(NewSingleton1 !== Singleton);

// Create a new Object and set its prototype by hand.
var NewSingleton2 = Object.new|
var NewSingleton2.asString = "NewSingleton2"|
NewSingleton2.protos = [Singleton]|
var s2 = NewSingleton2.new;
[00000001] NewSingleton2
assert(s2 === NewSingleton2);
assert(NewSingleton2 !== Singleton);
\end{urbiscript}

\subsection{Methods}
\begin{itemize}
\item \lstinline|clone|\\
  Return \lstinline|this|.

\item \lstinline|'new'|\\
  Return \lstinline|this|.
\end{itemize}

%%% Local Variables:
%%% mode: latex
%%% TeX-master: "../urbi-sdk"
%%% End:
