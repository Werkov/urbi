\section{Object}

All objects in \us must have \refObject{Object} in their
parents. \refObject{Object} is done for this purpose so that it come
with many primitives that are mandatory for all object in \us.

\subsection{Prototypes}

\begin{itemize}
\item \refObject{Orderable}
\item \refObject{Global}
\end{itemize}

\subsection{Methods}

\begin{itemize}
\item \lstinline|acceptVoid|\\

\item \lstinline|addProto(\var{proto})|\\
  Add the prototype \var{proto} as a parent into the list of
  \lstinline|this|.

\begin{urbiscript}
var Dummy = Object.new | Dummy.addProto(Orderable) | {};
assert_eq(Dummy.protos, [Orderable, Object]);
\end{urbiscript}

\item \lstinline|allProto|\\
  Return a list with \lstinline|this|, all parents of
  \lstinline|this|, the parents of the parents of
  \lstinline|this|,\ldots

% FIXME: Cannot use assert_eq here, as we use == on Comparable,
% which is an endless loop.
\begin{urbiscript}[firstnumber=last]
assert_eq(12.allProtos.map(function (t) { t.asString }),
          ["12", "0", "Orderable", "Object", "Comparable", "Global",
           "Tags", "Math", "DeprecatedCommands", "Kernel1Functions",
           "System"]);
\end{urbiscript}

\item \lstinline|allSlotNames|\\
  Return a list with the slot names of \lstinline|this| and of all
  parents in the hierarchy of \lstinline|this|.

\item \lstinline|apply|\\

\item \lstinline|as|\\

\item \lstinline|bounce|\\

\item \lstinline|bounce_named|\\

\item \lstinline|callMessage|\\

\item \lstinline|clone|\\

\item \lstinline|cloneSlot(\var{from}, \var{to})|\\
  Set the new slot \var{to} using a clone of \var{from}. This can only
  be used into the same object.

\begin{urbiscript}[firstnumber=last]
var foo = Object.new |;
cloneSlot("foo", "bar") |;
assert(!(foo === bar));
\end{urbiscript}

\item \lstinline|copySlot(\var{from}, \var{to})|\\
  Same as \lstinline|cloneSlot|, but the slot aren't cloned, so the
  two slot are the same.

\begin{urbiscript}[firstnumber=last]
var moo = Object.new |;
cloneSlot("moo", "loo") |;
assert(!(moo === loo));
\end{urbiscript}

\item \lstinline|createSlot(\var{name})|\\
  Create an empty slot (which actually means it is bound to
  \lstinline|void|) named \var{name}.  Raise an error if \var{name}
  was already defined.

\begin{urbiscript}[firstnumber=last]
assert(Object.createSlot("foo").isVoid);
assert(Object.hasSlot("foo"));
\end{urbiscript}

\item \lstinline|dump|\\

\item \lstinline|each(\var{fun})|\\
  Call \lstinline|each| with the given functional value on
  \lstinline|this.asList|.

\item \lstinline|eachBg|\\

\item \lstinline$each|$\\

\item \lstinline|getLazyLocalSlot|\\

\item \lstinline|getPeriod|\\

\item \lstinline|getProperty|\\

\item \lstinline|getSlot(\var{name})|\\
  Return the value associated to \var{name}.

\begin{urbiscript}[firstnumber=last]
assert(!Object.getSlot("getSlot").isVoid);
\end{urbiscript}

\item \lstinline|hasProperty|\\

\item \lstinline|hasLocalSlot(\var{slot})|\\
  Whether \lstinline|this| features a slot \var{slot}, locally, not
  from some ancester.  See also \lstinline|Object.hasSlot|.

\begin{urbiscript}[firstnumber=last]
class Base         { var this.base = 23; } |;
class Derive: Base { var this.derive = 43 } |;
assert(Derive.hasLocalSlot("derive"));
assert(!Derive.hasLocalSlot("base"));
\end{urbiscript}

\item \lstinline|hasSlot(\var{slot})|\\
  Whether \lstinline|this| has the slot \var{slot}, locally, or from
  some ancester.  See also \lstinline|Object.hasLocalSlot|.

\begin{urbiscript}[firstnumber=last]
assert(Derive.hasSlot("derive"));
assert(Derive.hasSlot("base"));
assert(!Base.hasSlot("derive"));
\end{urbiscript}

\item \lstinline|id|\\

\item \lstinline|isA(\var{obj})|\\
  Return true if \lstinline|this| has \var{obj} in his parents, false
  otherwise.

\begin{urbiscript}[firstnumber=last]
assert(Float.isA(Orderable));
assert(!(String.isA(Float)));
\end{urbiscript}

\item \lstinline|isNil|\\
  Return true if \lstinline|this| is nil, false otherwise.

\begin{urbiscript}[firstnumber=last]
assert(nil.isNil);
assert(!(0.isNil));
\end{urbiscript}

\item \lstinline|isProto|\\
  Return true if \lstinline|this| is a prototype, false otherwise;

\begin{urbiscript}[firstnumber=last]
assert(Float.isProto);
assert(!(42.isProto));
\end{urbiscript}

\item \lstinline|isVoid|\\
  Return true if \lstinline|this| is \lstinline|void|.

\begin{urbiscript}[firstnumber=last]
assert(void.isVoid);
assert(!(42.isVoid));
\end{urbiscript}

\item \lstinline|locateSlot(\var{slot})|\\
  Return \lstinline|nil| if \lstinline|this| don't have the slot
  \lstinline|slot|. Otherwise it returns the first lowest owner of
  \lstinline|slot| of \lstinline|this|.

\begin{urbiscript}[firstnumber=last]
assert(locateSlot("init"), Channel);
assert(locateSlot("doesNotExist").isNil);
\end{urbiscript}

\item \lstinline|print|\\

\item \lstinline|protos|\\
  Return the list of prototypes of \lstinline|this|.
\begin{urbiscript}[firstnumber=last]
assert_eq(12.protos, [0]);
\end{urbiscript}

\item \lstinline|removeProperty|\\

\item \lstinline|removeProto(\var{proto})|\\
  Remove \var{proto} from the list of prototypes of \lstinline|this|.

\item \lstinline|removeSlot(\var(slot))|\\
  Remove \var{slot} from the list of slots of \lstinline|this|.

\item \lstinline|setConstSlot|\\
  Like \lstinline|setSlot| but the created slot is const.

\begin{urbiscript}[firstnumber=last]
assert_eq(setConstSlot("fortyTwo", 42), 42);
fortyTwo = 51;
[00000000:error] !!! cannot modify const slot
\end{urbiscript}

\item \lstinline|setProperty|\\

\item \lstinline|setProtos|\\

\item \lstinline|setSlot(\var{name}, \var{value})|\\
  Create a slot \var{name} mapping to \var{value}. Raise an error if
  \var{name} was already defined.

\begin{urbiscript}[firstnumber=last]
assert_eq(Object.setSlot("theObject", Object), Object);
assert(Object.theObject === Object);
assert(theObject === Object);
\end{urbiscript}

\item \lstinline|slotNames|\\
  Returns the list of slot owned by \lstinline|this|.

\item \lstinline|tasks|\\
  Returns the list of the current running tasks;

\item \lstinline|type|\\

\item \lstinline|uid|\\
  Returns ths unique id of \lstinline|this|.

\item \lstinline|unacceptvoid|\\

\item \lstinline|uobject_init|\\

\item \lstinline|updateSlot(\var{name}, \var{value})|\\
  Map the existing slot named \var{name} to \var{value}. Raise an
  error if \var{name} was not defined.
\begin{urbiscript}[firstnumber=last]
assert_eq(Object.setSlot("one", 1), 1);
assert_eq(Object.updateSlot("one", 2), 2);
assert_eq(Object.one, 2);
\end{urbiscript}

\item \lstinline|'&&'(\var{that})|\\
  Short-circuiting logical and. If \lstinline|this| evaluates to true
  evaluate and return \var{that}, otherwise return \lstinline|this|
  without evaluating \var{that}.
\begin{urbiscript}[firstnumber=last]
assert_eq(0 && "foo", 0);
assert_eq(2 && "foo", "foo");

assert_eq(""    && "foo", "");
assert_eq("foo" && "bar", "bar");
\end{urbiscript}

\item \lstinline/'||'(\var{that})/\\
  Short-circuiting logical or. If \lstinline|this| evaluates to false
  evaluate and return \var{that}, otherwise return \lstinline|this|
  without evaluating \var{that}.
\begin{urbiscript}[firstnumber=last]
assert_eq(0 || "foo", "foo");
assert_eq(2 ||  1/0,  2);

assert_eq(""    || "foo", "foo");
assert_eq("foo" || 1/0,   "foo");
\end{urbiscript}

\item \lstinline|'!'|\\
  Logical negation. If \lstinline|this| evaluates to false return
  \lstinline|true| and vice-versa.
\begin{urbiscript}[firstnumber=last]
assert_eq(!1, false);
assert_eq(!0, true);

assert_eq(!"foo", false);
assert_eq(!"",    true);
\end{urbiscript}
\end{itemize}

%%% Local Variables:
%%% mode: latex
%%% TeX-master: "../urbi-sdk"
%%% End:
