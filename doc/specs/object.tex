\section{Object}

All objects in \us must have \refObject{Object} in their
parents. \refObject{Object} is done for this purpose so that it come
with many primitives that are mandatory for all object in \us.

\subsection{Prototypes}

\begin{itemize}
\item \refObject{Orderable}
\item \refObject{Global}
\end{itemize}

\subsection{Construction}

Fresh object can be instantiated by cloning Object itself.

\begin{urbiscript}[firstnumber=1]
Object.new;
[00000421] Object_0x00000000
\end{urbiscript}

The keyword \lstindex{class} also allows to define objects which are
intended to serve as prototype of a family of objects, similarly to
classes in traditional object-oriented programming languages (see
\autoref{sec:tut:class}).

\begin{urbiscript}
{
  class Foo
  {
    var attr = 23;
  };
  assert
  {
    Foo.localSlotNames == ["$type", "asFoo", "attr"];
    Foo.asFoo === Foo;
    Foo.attr == 23;
    Foo.'$type' == "Foo";
  };
};
\end{urbiscript}


\subsection{Slots}

\begin{itemize}
\item \lstinline|acceptVoid|\\
  Return \lstinline|this|.  See \refObject{void} to know why.
\begin{urbiscript}
{
  var o = Object.new;
  assert(o.acceptVoid === o);
};
\end{urbiscript}


\item \lstinline|addProto(\var{proto})|\\
  Add \var{proto} into the list of prototypes of \lstinline|this|.
\begin{urbiscript}
{
  var o = Object.new;
  o.addProto(Orderable);
  assert(o.protos == [Orderable, Object]);
};
\end{urbiscript}

\item \lstinline|allProto|\\
  Return a list with \lstinline|this|, all parents of
  \lstinline|this|, the parents of the parents of
  \lstinline|this|,\ldots
\begin{urbiassert}
123.allProtos.size == 12;
\end{urbiassert}

\item \lstinline|allSlotNames|\\
  Return a list with the slot names of \lstinline|this| and its
  ancesters.
\begin{urbiassert}
Object.localSlotNames
  .subset(Object.allSlotNames);
Object.protos.foldl(function (var r, var p) { r + p.localSlotNames },
                    [])
  .subset(Object.allSlotNames);
\end{urbiassert}

\item \lstinline|apply(\var{args})|\\
  ``Invoke \lstinline|this|''.  The size of the argument list,
  \var{args}, must be one.  This argument is ignored.  This function
  exists for compatibility with \lstinline|Code.apply|.
\begin{urbiassert}
Object.apply([this]) === Object;
Object.apply([1])    === Object;
\end{urbiassert}

\item \lstinline|as(\var{type})|\\
  Convert \lstinline|this| to \var{type}.  This is syntactic sugar for
  \lstinline|as\var{Type}| when \var{Type} is the \lstinline|$type| of
  \var{type}.
\begin{urbiassert}
12.as(Float) == 12;
"12".as(Float) == 12;
12.as(String) == "12";
Object.as(Object) === Object;
\end{urbiassert}

\item \lstinline|asBool|\\
  Whether \lstinline|this| is ``true'', see \autoref{sec:truth}.

\begin{urbiscript}
assert(Global.asBool == true);
assert(nil.asBool ==    false);
void.asBool;
[00000421:error] !!! unexpected void
\end{urbiscript}

\item \lstinline|bounce(\var{name})|\\
  Return \lstinline|this.\var{name}| transformed from a method into a
  function that takes its target (its ``\lstinline|this|'') as first
  and only argument.  \lstinline|this.\var{name}| must take no
  argument.
\begin{urbiassert}
{ var myCos = Object.bounce("cos"); myCos(0) }    == 0.cos;
{ var myType = bounce("$type"); myType(Object); } == "Object";
{ var myType = bounce("$type"); myType(3.14); }   == "Float";
\end{urbiassert}

\item \lstinline|callMessage(\var{msg})|\\
  Invoke the \refObject{CallMessage} \var{msg} on this.
%%% \begin{urbiscript}
%%% function f(var tgt, var msg, var args)
%%% {
%%%   call.target  = tgt;
%%%   call.message = msg;
%%%   call.code = tgt.getSlot(msg);
%%%   call.args    = args;
%%%   call.inspect;
%%%   tgt.callMessage(call);
%%% }|;
%%% assert
%%% {
%%%   f(Object, "$type", []) == "Object.f(1, 2)";
%%%
%%% };
%%% \end{urbiscript}

\item \lstinline|clone|\\
  Clone \lstinline|this|, i.e., create a fresh, empty, object, which
  sole prototype is \lstinline|this|.
\begin{urbiassert}
Object.clone.protos == [Object];
Object.clone.localSlotNames == [];
\end{urbiassert}

\item \lstinline|cloneSlot(\var{from}, \var{to})|\\
  Set the new slot \var{to} using a clone of \var{from}. This can only
  be used into the same object.

\begin{urbiscript}
var foo = Object.new |;
cloneSlot("foo", "bar") |;
assert(!(foo === bar));
\end{urbiscript}

\item \lstinline|copySlot(\var{from}, \var{to})|\\
  Same as \lstinline|cloneSlot|, but the slot aren't cloned, so the
  two slot are the same.

\begin{urbiscript}
var moo = Object.new |;
cloneSlot("moo", "loo") |;
assert(!(moo === loo));
\end{urbiscript}

\item \lstinline|createSlot(\var{name})|\\
  Create an empty slot (which actually means it is bound to
  \lstinline|void|) named \var{name}.  Raise an error if \var{name}
  was already defined.
\begin{urbiscript}
do (Object.new)
{
  assert(!hasSlot("foo"));
  assert(createSlot("foo").isVoid);
  assert(hasSlot("foo"));
}|;
\end{urbiscript}

\item \lstinline|dump(\var{depth})|\\
  Describe \lstinline|this|: its prototypes and slots.  The argument
  \var{depth} specifies how recursive the description is: the greater,
  the more detailed.  This method is mostly useful for debugging
  low-level issues, for a more human-readable interface, see also
  \lstinline|Object.inspect|.
\begin{urbiscript}
do (2) { var this.attr = "foo"; this.attr->prop = "bar" }.dump(0);
[00015137] *** Float_0x240550 {
[00015137] ***   /* Special slots */
[00015137] ***   protos = Float
[00015137] ***   value = 2
[00015137] ***   /* Slots */
[00015137] ***   attr = String_0x23a750 <...>
[00015137] ***     /* Properties */
[00015137] ***     prop = String_0x23a7a0 <...>
[00015137] ***   }
do (2) { var this.attr = "foo"; this.attr->prop = "bar" }.dump(1);
[00020505] *** Float_0x240550 {
[00020505] ***   /* Special slots */
[00020505] ***   protos = Float
[00020505] ***   value = 2
[00020505] ***   /* Slots */
[00020505] ***   attr = String_0x23a750 {
[00020505] ***     /* Special slots */
[00020505] ***     protos = String
[00020505] ***     /* Slots */
[00020505] ***     }
[00020505] ***     /* Properties */
[00020505] ***     prop = String_0x239330 {
[00020505] ***       /* Special slots */
[00020505] ***       protos = String
[00020505] ***       /* Slots */
[00020505] ***       }
[00020505] ***   }
\end{urbiscript}

\item \lstinline|getPeriod|\\
  Deprecated.  Use \lstinline|Syste.period| instead.

\item \lstinline|getProperty(\var{slotName}, \var{propName})|\\
  Return the value of the \var{propName} property associated to the
  slot \var{slotName} if defined, \lstinline|void| otherwise.
\begin{urbiscript}
const var myPi = 3.14|;
assert
{
  getProperty("myPi", "constant");
  getProperty("myPi", "foobar").isVoid;
};
\end{urbiscript}

\item \lstinline|getLocalSlot(\var{name})|\\
  The value associated to \var{name} in \lstinline|this|, excluding
  its ancestors (contrary to \lstinline|getSlot|).

\begin{urbiscript}
var a = Object.new|;

// Local slot.
var a.slot = 21|;
assert
{
  a.locateSlot("slot") === a;
  a.getLocalSlot("slot") == 21;
};

// Inherited slot are not looked-up.
assert { a.locateSlot("init") == Object };
a.getLocalSlot("init");
[00041066:error] !!! lookup failed: init
\end{urbiscript}

\item \lstinline|getSlot(\var{name})|\\
  The value associated to \var{name} in \lstinline|this|, possibly
  after a look-up in its prototypes (contrary to
  \lstinline|getLocalSlot|).

\begin{urbiscript}
var b = Object.new|;
var b.slot = 21|;

assert
{
  // Local slot.
  b.locateSlot("slot") === b;
  b.getSlot("slot") == 21;

  // Inherited slot.
  b.locateSlot("init") === Object;
  b.getSlot("init") == Object.getSlot("init");
};

// Unknown slot.
assert { b.locateSlot("ENOENT") == nil; };
b.getSlot("ENOENT");
[00041066:error] !!! lookup failed: ENOENT
\end{urbiscript}

\item \lstinline|hasLocalSlot(\var{slot})|\\
  Whether \lstinline|this| features a slot \var{slot}, locally, not
  from some ancestor.  See also \lstinline|Object.hasSlot|.

\begin{urbiscript}
class Base         { var this.base = 23; } |;
class Derive: Base { var this.derive = 43 } |;
assert(Derive.hasLocalSlot("derive"));
assert(!Derive.hasLocalSlot("base"));
\end{urbiscript}

\item \lstinline|hasProperty(\var{slotName}, \var{propName})|\\
  Whether the slot \var{slotName} of \lstinline|this| has a property
  \var{propName}.
\begin{urbiscript}
const var halfPi = pi / 2|;
assert
{
  hasProperty("halfPi", "constant");
  !hasProperty("halfPi", "foobar");
};
\end{urbiscript}

\item \lstinline|hasSlot(\var{slot})|\\
  Whether \lstinline|this| has the slot \var{slot}, locally, or from
  some ancestor.  See also \lstinline|Object.hasLocalSlot|.

\begin{urbiassert}
Derive.hasSlot("derive");
Derive.hasSlot("base");
!Base.hasSlot("derive");
\end{urbiassert}

\item \lstinline|'$id'|\\ % fix color $

\item \lstinline|inspect(\var{deep} = false)|\\
  Describe \lstinline|this|: its prototypes and slots, and their
  properties.  If \var{deep}, all the slots are described, not only
  the local slots. See also \lstinline|Object.dump|.
\begin{urbiscript}
do (2) { var this.attr = "foo"; this.attr->prop = "bar"}.inspect;
[00001227] *** Inspecting 2
[00001227] *** ** Prototypes:
[00001227] ***   0
[00001227] *** ** Local Slots:
[00001228] ***   attr : String
[00001228] ***     Properties:
[00001228] ***      prop : String = "bar"
\end{urbiscript}

\item \lstinline|isA(\var{obj})|\\
  Return true if \lstinline|this| has \var{obj} in his parents, false
  otherwise.

\begin{urbiassert}
Float.isA(Orderable);
!(String.isA(Float));
\end{urbiassert}

\item \lstinline|isNil|\\
  Return true if \lstinline|this| is \refObject{nil}, false otherwise.

\begin{urbiassert}
nil.isNil;
!(0.isNil);
\end{urbiassert}

\item \lstinline|isProto|\\
  Return true if \lstinline|this| is a prototype, false otherwise;

\begin{urbiassert}
Float.isProto;
!(42.isProto);
\end{urbiassert}

\item \lstinline|isVoid|\\
  Return true if \lstinline|this| is \lstinline|void|.  See
  \refObject{void}.
\begin{urbiassert}
void.isVoid;
!(42.isVoid);
\end{urbiassert}

\item \lstinline|locateSlot(\var{slot})|\\
  Return \lstinline|nil| if \lstinline|this| don't have the slot
  \lstinline|slot|. Otherwise it returns the first lowest owner of
  \lstinline|slot| of \lstinline|this|.

\begin{urbiassert}
locateSlot("init") == Channel;
locateSlot("doesNotExist").isNil;
\end{urbiassert}

\item \lstinline|print|\\

\item \lstinline|protos|\\
  Return the list of prototypes of \lstinline|this|.
\begin{urbiassert}
12.protos == [0];
\end{urbiassert}

\item \lstinline|properties(\var{slotName})|\\
  Return a dictionary of the properties of slot \var{slotName}.  Raise
  an error if the slot does not exist.
\begin{urbiscript}
2.properties("foo");
[00238495:error] !!! lookup failed: foo
do (2) { var foo = "foo" }.properties("foo");
[00238501] Dictionary {"constant" => false}
do (2) { var foo = "foo" ; foo->bar = "bar" }.properties("foo");
[00238502] Dictionary {"bar" => "bar", "constant" => false}
\end{urbiscript}

\item \lstinline|removeProperty(\var{slotName}, \var{propName})|\\
  Remove the property \var{propName} from the slot \var{slotName}.
  Raise an error if the slot does not exist, do nothing if the
  property does not exist.
\begin{urbiscript}
do (2)
{
  var foo = "foo";
  foo->bar = "bar";
  removeProperty("foo", "bar");
}.properties("foo");
[00238502] Dictionary {"constant" => false}

2.removeProperty("foo", "bar");
[00000072:error] !!! lookup failed: foo

do (2)
{
  var foo = "foo";
  removeProperty("foo", "bar");
}|;
\end{urbiscript}

\item \lstinline|removeProto(\var{proto})|\\
  Remove \var{proto} from the list of prototypes of \lstinline|this|.

\item \lstinline|removeSlot(\var(slot))|\\
  Remove \var{slot} from the list of slots of \lstinline|this|.

\item \lstinline|setConstSlot|\\
  Like \lstinline|setSlot| but the created slot is const.

\begin{urbiscript}
assert(setConstSlot("fortyTwo", 42) == 42);
fortyTwo = 51;
[00000000:error] !!! cannot modify const slot
\end{urbiscript}

\item \lstinline|setProperty|\\

\item \lstinline|setProtos|\\

\item \lstinline|setSlot(\var{name}, \var{value})|\\
  Create a slot \var{name} mapping to \var{value}. Raise an error if
  \var{name} was already defined.

\begin{urbiassert}
Object.setSlot("theObject", Object) === Object;
Object.theObject === Object;
theObject === Object;
\end{urbiassert}

\item \lstinline|slotNames|\\
  Returns the list of slot owned by \lstinline|this|.

\item \lstinline|tasks|\\
  Returns the list of the current running tasks;

\item \lstinline|'$type'|\\ % fix color $

\item \lstinline|uid|\\
  Returns the unique id of \lstinline|this|.

\item \lstinline|unacceptvoid|\\
  Return \lstinline|this|.  See \refObject{void} to know why.
\begin{urbiscript}
{
  var o = Object.new|
  assert(o.unacceptVoid === o);
};
\end{urbiscript}

\item \lstinline|uobject_init|\\

\item \lstinline|updateSlot(\var{name}, \var{value})|\\
  Map the existing slot named \var{name} to \var{value}. Raise an
  error if \var{name} was not defined.
\begin{urbiassert}
Object.setSlot("one", 1)    == 1;
Object.updateSlot("one", 2) == 2;
Object.one                  == 2;
\end{urbiassert}

\item \lstinline|'&&'(\var{that})|\\
  Short-circuiting logical and. If \lstinline|this| evaluates to true
  evaluate and return \var{that}, otherwise return \lstinline|this|
  without evaluating \var{that}.
\begin{urbiassert}
(0 && "foo") == 0;
(2 && "foo") == "foo";

(""    && "foo") == "";
("foo" && "bar") == "bar";
\end{urbiassert}

\item \lstinline/'||'(\var{that})/\\
  Short-circuiting logical or. If \lstinline|this| evaluates to false
  evaluate and return \var{that}, otherwise return \lstinline|this|
  without evaluating \var{that}.
\begin{urbiassert}
(0 || "foo") == "foo";
(2 ||  1/0) ==  2;

(""    || "foo") == "foo";
("foo" || 1/0) ==   "foo";
\end{urbiassert}

\item \lstinline|'!'|\\
  Logical negation. If \lstinline|this| evaluates to false return
  \lstinline|true| and vice-versa.
\begin{urbiassert}
!1 == false;
!0 == true;

!"foo" == false;
!"" ==    true;
\end{urbiassert}
\end{itemize}

%%% Local Variables:
%%% mode: latex
%%% TeX-master: "../urbi-sdk"
%%% ispell-personal-dictionary: "../urbi.dict"
%%% End:
