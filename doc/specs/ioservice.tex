%% Copyright (C) 2010, Gostai S.A.S.
%%
%% This software is provided "as is" without warranty of any kind,
%% either expressed or implied, including but not limited to the
%% implied warranties of fitness for a particular purpose.
%%
%% See the LICENSE file for more information.

\section{IoService}

A \dfn{IoService} is used to manage the various operations of a set of
\refObject{Socket}.

All \refObject{Socket} and \refObject{Server} are by default using the
default \refObject{IoService} which is polled regularly by the system.

\subsection{Example}

Using a different \refObject{IoService} is required if you need to perform
synchronous read operations.

The \refObject{Socket} must be created by the \refObject{IoService} that will
handle it using its \lstinline|makeSocket| function.

\begin{urbiscript}
var io = IoService.new|;
var s = io.makeSocket|;
\end{urbiscript}

You can then use this socket like any other.

\begin{urbiscript}
// Make a simple hello server.
var serverPort = 0|
do(Server.new)
{
  listen("127.0.0.1", "0");
  lobby.serverPort = port;
  at(connection?(var s))
  {
    s.write("hello");
  }
}|;
// Connect to it using our socket.
s.connect("0.0.0.0", serverPort);
at(s.received?(var data))
  echo("received something");
s.write("1;");
\end{urbiscript}

... except that nothing will be read from the socket unless you call one of the
\lstinline|poll| functions of \lstinline|io|.

\begin{urbiscript}
sleep(200ms);
s.isConnected(); // Nothing was received yet
[00000001] true
io.poll();
[00000002] *** received something
sleep(200ms);
\end{urbiscript}

\subsection{Prototypes}
\begin{refObjects}
\item[Object]
\end{refObjects}

\subsection{Construction}

A \lstinline|IoService| is constructed with no argument.

\subsection{Slots}

\begin{urbiscriptapi}
\item[makeServer]
  Create and return a new \refObject{Server} using this \refObject{IoService}.

\item[makeSocket]
  Create and return a new \refObject{Socket} using this \refObject{IoService}.

\item[poll]
  Handle all pending socket operations(read, write, accept) that can be
  performed without waiting.

\item[pollFor](<duration>)%
  Will block for \var{duration} seconds, and handle all ready socket operations
  during this period.

\item[pollOneFor](<duration>)%
  Will block for at most \var{duration}, and handle the first ready socket
  operation and immediately return.

\end{urbiscriptapi}

%%% Local Variables:
%%% mode: latex
%%% TeX-master: "../urbi-sdk"
%%% ispell-dictionary: "american"
%%% ispell-personal-dictionary: "../urbi.dict"
%%% fill-column: 76
%%% End:
