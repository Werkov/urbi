%% Copyright (C) 2011, Gostai S.A.S.
%%
%% This software is provided "as is" without warranty of any kind,
%% either expressed or implied, including but not limited to the
%% implied warranties of fitness for a particular purpose.
%%
%% See the LICENSE file for more information.

\section{Matrix}

\subsection{Prototypes}
\begin{refObjects}
\item[Object]
\end{refObjects}

\subsection{Construction}
\label{sec:specs:matrix:ctor}

The \lstinline|init| function is overloaded, and its behavior depends on the
number of arguments and their types.

When there is a single argument, it can either be a List, or another Matrix.

If it's a List of ``Vectors and/or Lists of Floats'', then they must have
the same sizes and constitute the rows.

\begin{urbiscript}
var listList     = Matrix.new([           [0, 1],             [0, 1] ]);
[00071383] Matrix([
[:]  [0, 1],
[:]  [0, 1]])
var listVector   = Matrix.new([           [0, 1],  Vector.new([0, 1])])|;
var vectorList   = Matrix.new([Vector.new([0, 1]),            [0, 1] ])|;
var vectorVector = Matrix.new([Vector.new([0, 1]), Vector.new([0, 1])])|;

assert
{
  listList == listVector;
  listList == vectorList;
  listList == vectorVector;
};
Matrix.new([           [0],            [1, 2]]);
[00000030:error] !!! new: expecting rows of size 1, got size 2 for row 2
Matrix.new([Vector.new([0]),           [1, 2]]);
[00000056:error] !!! new: expecting rows of size 1, got size 2 for row 2
Matrix.new([           [0], Vector.new([1, 2])]);
[00000071:error] !!! new: expecting rows of size 1, got size 2 for row 2
Matrix.new([Vector.new([0]), Vector.new([1, 2])]);
[00052403:error] !!! new: expecting rows of size 1, got size 2 for row 2
\end{urbiscript}

If it's a Matrix, then there is a deep-copy: they are not aliases.

\begin{urbiscript}
var m1 = Matrix.new([[1, 1], [1, 1]])|;
var m2 = Matrix.new(m1)|;
m2[0, 0] = 0|;
assert
{
  m1 == Matrix.new([[1, 1], [1, 1]]);
  m2 == Matrix.new([[0, 1], [1, 1]]);
};
\end{urbiscript}

When given two Float arguments, they must be the two integers, defining the
size of the null Matrix.
\begin{urbiscript}
Matrix.new(2, 3);
[00051329] Matrix([
[:]  [0, 0, 0],
[:]  [0, 0, 0]])
\end{urbiscript}

In other cases, the arguments are expected to be Lists of Floats and/or
Vectors.

\begin{urbiassert}
Matrix.new([1, 2])                   == Matrix.new([[1, 2]]);
Matrix.new([1, 2],           [3, 4]) == Matrix.new([[1, 2], [3, 4]]);
Matrix.new([1, 2], Vector.new(3, 4)) == Matrix.new([[1, 2], [3, 4]]);
\end{urbiassert}

These rows must have equal sizes.

\begin{urbiscript}
// Lists and Lists.
Matrix.new([0], [1, 2]);
[00000160:error] !!! new: expecting rows of size 1, got size 2 for row 2
Matrix.new([0, 1], [2]);
[00000169:error] !!! new: expecting rows of size 2, got size 1 for row 2

// Lists and Vectors.
Matrix.new([0], Vector.new(1, 2));
[00000178:error] !!! new: expecting rows of size 1, got size 2 for row 2
Matrix.new(Vector.new(0, 1), [2]);
[00000186:error] !!! new: expecting rows of size 2, got size 1 for row 2

// Vectors and Vectors.
Matrix.new(Vector.new(0), Vector.new(1, 2));
[00000195:error] !!! new: expecting rows of size 1, got size 2 for row 2
Matrix.new(Vector.new(0, 1), Vector.new(2));
[00000204:error] !!! new: expecting rows of size 2, got size 1 for row 2
\end{urbiscript}


\subsection{Slots}

\begin{urbiscriptapi}
\item[appendRow](<vector>)%
  Append \var{vector} to \this and return \this.
\begin{urbiassert}
var m2x1 = Matrix.new([0], [1]);
m2x1.appendRow(Vector.new(2)) === m2x1;
m2x1 == Matrix.new([0], [1], [2]);

var m2x2 = Matrix.new([0, 1], [10, 11]);
m2x2.appendRow(Vector.new(20, 21)) == m2x2;
m2x2 == Matrix.new([0, 1], [10, 11], [20, 21]);
\end{urbiassert}

  Sizes must match.
\begin{urbiscript}
Matrix.new([0], [1]).appendRow(Vector.new(10, 11));
[00017936:error] !!! appendRow: incompatible sizes: 2x1, 2

Matrix.new([0, 1]).appendRow(Vector.new(10));
[00050922:error] !!! appendRow: incompatible sizes: 1x2, 1
\end{urbiscript}


\item[asMatrix]%
  \this.
\begin{urbiassert}
Matrix.asMatrix === Matrix;
var m = Matrix.new([1], [2]);
m.asMatrix === m;
\end{urbiassert}


\item[asPrintable]%
  A String that denotes \this.
\begin{urbiassert}
Matrix                    .asPrintable == "Matrix([])";
Matrix.new([1, 2], [3, 4]).asPrintable == "Matrix([[1, 2], [3, 4]])";
\end{urbiassert}


\item[asString]%
  A String that denotes \this.
\begin{urbiassert}
Matrix                    .asString == "<>";
Matrix.new([1, 2], [3, 4]).asString == "<<1, 2>, <3, 4>>";
\end{urbiassert}


\item[asTopLevelPrintable]%
  A String that denotes \this.
\begin{urbiassert}
Matrix                    .asTopLevelPrintable
  == "Matrix([])";
Matrix.new([1, 2], [3, 4]).asTopLevelPrintable
 == "Matrix([\n  [1, 2],\n  [3, 4]])";
\end{urbiassert}


\item[column](<i>)%
  The \var{i}th column as a Vector.  See also \refSlot{row}.
\begin{urbiassert}
var m = Matrix.new([1, 2, 3], [4, 5, 6]);
m.column(0) == Vector.new(1, 4);  m.column(-3) == m.column(0);
m.column(1) == Vector.new(2, 5);  m.column(-2) == m.column(1);
m.column(2) == Vector.new(3, 6);  m.column(-1) == m.column(2);

m.column(3);
[00000232:error] !!! column: invalid column: 3
\end{urbiassert}


\item[createIdentity](<size>)%
  The unit Matrix of dimensions \var{size}.
\begin{urbiassert}
Matrix.createIdentity(0) == Matrix;
Matrix.createIdentity(3) == Matrix.new([1, 0, 0], [0, 1, 0], [0, 0, 1]);

Matrix.createIdentity(-2);
[00000328:error] !!! createIdentity: argument 1: expected non-negative integer: -2
\end{urbiassert}


\item[createOnes](<row>, <col>)%
  Same as \lstinline|createScalar(\var{row}, \var{col}, 1)|.
\begin{urbiassert}
Matrix.createOnes(0, 0) == Matrix;
Matrix.createOnes(2, 3) == Matrix.new([1, 1, 1], [1, 1, 1]);

Matrix.createOnes(-2, 2);
[00000328:error] !!! createOnes: argument 1: expected non-negative integer: -2
\end{urbiassert}


\item[createScalars](<row>, <col>, <scalar>)%
  A Matrix of size \lstinline|\var{row} * \var{col}| filled with
  \var{scalar}.
\begin{urbiassert}
Matrix.createScalars(0, 0, 99) == Matrix;
Matrix.createScalars(2, 3, 99) == Matrix.new([99, 99, 99], [99, 99, 99]);

Matrix.createScalars(-2, 2, 99);
[00000328:error] !!! createScalars: argument 1: expected non-negative integer: -2
\end{urbiassert}


\item[createZeros](<row>, <col>)%
  Same as \lstinline|createScalar(\var{row}, \var{col}, 0)|.
\begin{urbiassert}
Matrix.createZeros(0, 0) == Matrix;
Matrix.createZeros(2, 3) == Matrix.new([0, 0, 0], [0, 0, 0]);

Matrix.createZeros(-2, 2);
[00000328:error] !!! createZeros: argument 1: expected non-negative integer: -2
\end{urbiassert}


\item[distanceMatrix](<that>)%
  Considering that \this and \that are collections of Vectors that denote
  positions in an Euclidean space, produce a Matrix whose value at
  \lstinline|(\var{i}, \var{j})| is the distance between points
  \lstinline|this[\var{i}]| and \lstinline|\var{that}[\var{j}]|.
\begin{urbiassert}
// Left-hand side matrix.
var l0 = Vector.new([0, 0]);  var l1 = Vector.new([2, 1]);
var lhs = Matrix.new([l0, l1]);
// Right-hand side matrix.
var r0 = Vector.new([0, 1]);  var r1 = Vector.new([2, 0]);
var rhs = Matrix.new([r0, r1]);

lhs.distanceMatrix(rhs)
  == Matrix.new([l0.distance(r0), l0.distance(r1)],
                [l1.distance(r0), l1.distance(r1)]);
\end{urbiassert}


\item[init]%
  See \autoref{sec:specs:matrix:ctor}.


\item[inverse]%
  The inverse of \this if it exists, raise an error otherwise.
\begin{urbiassert}
var m = Matrix.new(
  [1, 3, 1],
  [1, 1, 2],
  [2, 3, 4]);

m * m.inverse == Matrix.createIdentity(3);
m.inverse * m == Matrix.createIdentity(3);

m.inverse == Matrix.new(
  [ 2,  9, -5],
  [ 0, -2,  1],
  [-1, -3,  2]);

Matrix.createZeros(2, 2).inverse;
[00000534:error] !!! inverse: non-invertible matrix: <<0, 0>, <0, 0>>
\end{urbiassert}


\item[resize](<row>, <col>)%
  Change the dimensions of \this, using 0 for new members.  Return \this.
\begin{urbiscript}
// Check that <<1, 2><3, 4>> is equal to the Matrix composed of rows,
// when resized to the dimensions of rows.
function resized(var rows[])
{
  var m = Matrix.new([1, 2], [3, 4]);
  var res = Matrix.new(rows);
  // Resize returns this...
  m.resize(res.size.first, res.size.second) === m;
  // ...and does resize.
  m == res;
}|;
assert
{
  // Fewer rows/cols.
  resized([1], [3]);
  resized([1, 2]);
  resized([1]);
  resized([]);

  // As many rows and cols.
  resized([1, 2], [3, 4]);

  // More rows/cols.
  resized([1, 2, 0], [3, 4, 0]);
  resized([1, 2], [3, 4], [0, 0]);
  resized([1, 2, 0], [3, 4, 0], [0, 0, 0]);

  // More rows, less cols, and conversely.
  resized([1], [3], [0]);
  resized([1, 2, 0]);
};
\end{urbiscript}


\item[row](<i>)%
  The \var{i}th row as a Vector.  See also \refSlot{column}.
\begin{urbiassert}
var m = Matrix.new([1, 2, 3], [4, 5, 6]);
m.row(0) == Vector.new(1, 2, 3);  m.row(-2) == m.row(0);
m.row(1) == Vector.new(4, 5, 6);  m.row(-1) == m.row(1);

m.row(2);
[00195645:error] !!! row: invalid row: 2
\end{urbiassert}


\item[rowAdd](<vector>)%
  A Matrix whose rows (vectors) are the sum of each row of \this with
  \var{vector}.
\begin{urbiscript}
do (Matrix)
{
  assert
  {
    rowAdd(Vector) == Matrix;
    new([[1, 2]]).rowAdd(Vector.new(10))     == new([[11, 12]]);
    new([1], [2]).rowAdd(Vector.new(10, 20)) == new([11], [22]);
    new([1, 2], [3, 4]).rowAdd(Vector.new(10, 20)) == new([11, 12], [23, 24]);
  }
}|;
\end{urbiscript}

  The dimensions must be compatible:
  \lstinline|size.first == \var{vector}.size|.
\begin{urbiscript}
Matrix.new([1], [2]).rowAdd(Vector.new(10));
[00000415:error] !!! rowAdd: incompatible sizes: 2x1, 1

Matrix.new([1, 2]).rowAdd(Vector.new(10, 20));
[00000425:error] !!! rowAdd: incompatible sizes: 1x2, 2
\end{urbiscript}


\item[rowDiv](<vector>)%
  A Matrix whose rows (vectors) are the member-wise division of each row of
  \this with \var{vector}.
\begin{urbiscript}
do (Matrix)
{
  assert
  {
    rowDiv(Vector) == Matrix;
    new([[10, 20]]).rowDiv(Vector.new(10))     == new([[1, 2]]);
    new([10], [20]).rowDiv(Vector.new(10, 20)) == new([1], [1]);
    new([10, 30], [20, 40]).rowDiv(Vector.new(10, 20)) == new([1, 3], [1, 2]);
  }
}|;
\end{urbiscript}

  The dimensions must be compatible:
  \lstinline|size.first == \var{vector}.size|.
\begin{urbiscript}
Matrix.new([1], [2]).rowDiv(Vector.new(10));
[00000415:error] !!! rowDiv: incompatible sizes: 2x1, 1

Matrix.new([1, 2]).rowDiv(Vector.new(10, 20));
[00000425:error] !!! rowDiv: incompatible sizes: 1x2, 2
\end{urbiscript}


\item[rowMul](<vector>)%
  A Matrix whose rows (vectors) are the member-wise product of each row of
  \this with \var{vector}.
\begin{urbiscript}
do (Matrix)
{
  assert
  {
    rowMul(Vector) == Matrix;
    new([[10, 20]]).rowMul(Vector.new(10))     == new([[100, 200]]);
    new([10], [20]).rowMul(Vector.new(10, 20)) == new([100], [400]);
    new([1, 2], [3, 4]).rowMul(Vector.new(10, 20)) == new([10, 20], [60, 80]);
  }
}|;
\end{urbiscript}

  The dimensions must be compatible:
  \lstinline|size.first == \var{vector}.size|.
\begin{urbiscript}
Matrix.new([1], [2]).rowMul(Vector.new(10));
[00000415:error] !!! rowMul: incompatible sizes: 2x1, 1

Matrix.new([1, 2]).rowMul(Vector.new(10, 20));
[00000425:error] !!! rowMul: incompatible sizes: 1x2, 2
\end{urbiscript}


\item[rowNorm]%
  A Vector whose values are the (Euclidean) norms of the rows of \this.
\begin{urbiassert}
var m = Matrix.new([1, 2], [3, 3]);
m.rowNorm[ 0] == m.row( 0).norm;
m.rowNorm[-1] == m.row(-1).norm;
\end{urbiassert}


\item[rowSub](<vector>)%
  A Matrix whose rows (vectors) are the difference of each row of \this with
  \var{vector}.
\begin{urbiscript}
do (Matrix)
{
  assert
  {
    rowSub(Vector) == Matrix;
    new([[10, 20]])    .rowSub(Vector.new(1))      == new([[9, 19]]);
    new([11], [22])    .rowSub(Vector.new(10, 20)) == new([1], [2]);
    new([1, 2], [3, 4]).rowSub(Vector.new(1, 2))   == new([0, 1], [1, 2]);
  }
}|;
\end{urbiscript}

  The dimensions must be compatible:
  \lstinline|size.first == \var{vector}.size|.
\begin{urbiscript}
Matrix.new([1], [2]).rowSub(Vector.new(10));
[00000415:error] !!! rowSub: incompatible sizes: 2x1, 1

Matrix.new([1, 2]).rowSub(Vector.new(10, 20));
[00000425:error] !!! rowSub: incompatible sizes: 1x2, 2
\end{urbiscript}


\item[set](<vectors>)%
  Change \this to be equal to the Matrix defined by the list of vectors
  given as argument, and return \this.
\begin{urbiassert}
var m = Matrix.new([]);

var m1 = Matrix.new([0, 1], [0, 1]);
    m  ===    m.set([[0, 1], [0, 1]]);
    m  == m1;

var m2 = Matrix.new([2, 3]);
    m  ===    m.set([[2, 3]]);
    m  == m2;
\end{urbiassert}


\item[setRow](<index>, <vector>)%
  Set the \var{index}th row of \this to \var{vector} and return \this.
\begin{urbiassert}
var m2x1 = Matrix.new([0], [1]);
m2x1.setRow(0, Vector.new(2)) === m2x1;
m2x1 == Matrix.new([2], [1]);

var m2x2 = Matrix.new([0, 1], [10, 11]);
m2x2.setRow(0, Vector.new(20, 21)) == m2x2;
m2x2 == Matrix.new([20, 21], [10, 11]);
\end{urbiassert}

  Sizes and index must match.
\begin{urbiscript}
Matrix.new([0], [1]).setRow(0, Vector.new(10, 11));
[00017936:error] !!! setRow: incompatible sizes: 2x1, 2

Matrix.new([0, 1]).setRow(0, Vector.new(10));
[00050922:error] !!! setRow: incompatible sizes: 1x2, 1

Matrix.new([0], [1]).setRow(2, Vector.new(10));
[00017936:error] !!! setRow: invalid row: 2
\end{urbiscript}


\item[size](<arg>)%
  The dimensions of the \this, as a Pair of Floats.
\begin{urbiassert}
Matrix.size == Pair.new(0, 0);
Matrix.new([1, 2], [3, 4], [5, 6]).size == Pair.new(3, 2);
Matrix.new([1, 2, 3], [4, 5, 6])  .size == Pair.new(2, 3);
\end{urbiassert}

\item[transpose](<arg>)%
  The transposed of \this.
\begin{urbiassert}
Matrix                    .transpose == Matrix;
Matrix.new([1])           .transpose == Matrix.new([1]);
Matrix.new([1], [2])      .transpose == Matrix.new([1, 2]);
Matrix.new([1, 2], [3, 4]).transpose == Matrix.new([1, 3], [2, 4]);
\end{urbiassert}


\item[type]%
  The String \lstinline|Matrix|.
\begin{urbiassert}
Matrix.type         == "Matrix";
Matrix.new([]).type == "Matrix";
\end{urbiassert}


\item['=='](<that>)%
  Whether \this and \that have the same dimensions and members.
\begin{urbiscript}
do (Matrix)
{
  assert
  {
      new([[1], [2], [3]]) == new([[1], [2], [3]]);
    !(new([[1], [2], [3]]) == new([[1], [2]]));
    !(new([[1], [2], [3]]) == new([[3], [2], [1]]));
  };
}|;
\end{urbiscript}


\item['*'](<that>)%
  If \that is a Matrix, matrix product between \this and \that.
\begin{urbiassert}
Matrix.new([1, 2]) * Matrix.new([10], [20])
  == Matrix.new([50]);
Matrix.new([3, 4]) * Matrix.new([10], [20])
  == Matrix.new([110]);
Matrix.new([1, 2], [3, 4]) * Matrix.new([10], [20])
  == Matrix.new([50], [110]);
\end{urbiassert}

The sizes must be compatible ($\texttt{\this.size.second} =
\texttt{\that.size.first}$).
\begin{urbiscript}
Matrix.new([1, 2]) * Matrix.new([3, 4]);
[00081168:error] !!! *: incompatible sizes: 1x2, 1x2
\end{urbiscript}

  If \that is a Float, the scalar product.
\begin{urbiassert}
Matrix.new([1, 2], [3, 4]) * 3 == Matrix.new([3, 6], [9, 12]);
\end{urbiassert}

\item['*='](<that>)%
  In place (\autoref{sec:lang:op:ass}) product (\refSlot{'*'}).  The same
  constraints apply.
\begin{urbiassert}
var lhs1 = Matrix.new([1, 2], [3, 4]);
var lhs2 = Matrix.new(lhs1);
var rhs = Matrix.new([10], [20]);
var res = lhs1 * rhs;

(lhs1.'*='(rhs)) === lhs1;  lhs1 == res;
(lhs2  *=  rhs ) === lhs2;  lhs2 == res;

rhs *= rhs;
[00272182:error] !!! *=: incompatible sizes: 2x1, 2x1
\end{urbiassert}

\begin{urbiassert}
var v = Matrix.new([1, 2], [3, 4]);
var res = v * 3;
(v *= 3) === v; v == res;
\end{urbiassert}


\item['+'](<that>)%
  The sum of \this and \that.  Their sizes must be equal.
\begin{urbiassert}
Matrix.new([1, 2]) + Matrix.new([10, 20])
  == Matrix.new([11, 22]);

Matrix.new([1, 2], [3, 4]) + Matrix.new([10, 20], [30, 40])
  == Matrix.new([11, 22], [33, 44]);

Matrix.new([1, 2]) + Matrix.new([10, 20], [30, 40]);
[00002056:error] !!! +: incompatible sizes: 1x2, 2x2
\end{urbiassert}

  If \that is a Float, the scalar addition.
\begin{urbiassert}
Matrix.new([1, 2], [3, 4]) + 3 == Matrix.new([4, 5], [6, 7]);
\end{urbiassert}


\item['+='](<that>)%
  In place (\autoref{sec:lang:op:ass}) sum (\refSlot{'+'}).  The same
  constraints apply.
\begin{urbiassert}
var lhs1 = Matrix.new([ 1 , 2], [ 3,  4]);
var lhs2 = Matrix.new(lhs1);
var rhs = Matrix.new([10, 20], [30, 40]);
var res = lhs1 + rhs;

(lhs1.'+='(rhs)) === lhs1;  lhs1 == res;
(lhs2  +=  rhs ) === lhs2;  lhs2 == res;

lhs1 += Matrix.new([1, 2]);
[00338194:error] !!! +=: incompatible sizes: 2x2, 1x2
\end{urbiassert}

\begin{urbiassert}
var v = Matrix.new([3, 6], [9, 12]);
var res = v + 3;
(v += 3) === v; v == res;
\end{urbiassert}


\item['-'](<that>)%
  The difference of \this and \that.  Their sizes must be equal.
\begin{urbiassert}
Matrix.new([1, 2]) - Matrix.new([10, 20])
  == Matrix.new([-9, -18]);

Matrix.new([1, 2], [3, 4]) - Matrix.new([10, 20], [30, 40])
  == Matrix.new([-9, -18], [-27, -36]);

Matrix.new([1, 2]) - Matrix.new([10, 20], [30, 40]);
[00002056:error] !!! -: incompatible sizes: 1x2, 2x2
\end{urbiassert}

  If \that is a Float, the scalar difference.
\begin{urbiassert}
Matrix.new([1, 2], [3, 4]) - 3 == Matrix.new([-2, -1], [0, 1]);
\end{urbiassert}



\item['-='](<that>)%
  In place (\autoref{sec:lang:op:ass}) difference (\refSlot{'-'}).  The same
  constraints apply.
\begin{urbiassert}
var lhs1 = Matrix.new([ 1 , 2], [ 3,  4]);
var lhs2 = Matrix.new(lhs1);
var rhs = Matrix.new([10, 20], [30, 40]);
var res = lhs1 - rhs;

(lhs1.'-='(rhs)) === lhs1;  lhs1 == res;
(lhs2  -=  rhs ) === lhs2;  lhs2 == res;

lhs1 -= Matrix.new([1, 2]) ;
[00362383:error] !!! -=: incompatible sizes: 2x2, 1x2
\end{urbiassert}

\begin{urbiassert}
var v = Matrix.new([3, 6], [9, 12]);
var res = v - 3;
(v -= 3) === v; v == res;
\end{urbiassert}


\item['/'](<that>)%
  Same as \lstinline|this * that.inverse|.  \that must be invertible.
\begin{urbiassert}
var lhs = Matrix.new([20, 0], [0, 200]);
var rhs = Matrix.new([10, 0], [0, 100]);
var res = Matrix.new([ 2, 0], [0,   2]);

lhs / rhs == res;
(lhs / rhs) * rhs == lhs;
rhs * (lhs / rhs) == lhs;

lhs / Matrix.createZeros(2, 2);
[00160168:error] !!! /: non-invertible matrix: <<0, 0>, <0, 0>>
\end{urbiassert}

  If \that is a Float, the scalar division.
\begin{urbiassert}
Matrix.new([3, 6], [9, 12]) / 3 == Matrix.new([1, 2], [3, 4]);
\end{urbiassert}



\item['/='](<that>)%
  In place (\autoref{sec:lang:op:ass}) division (\refSlot{'/'}).  The same
  constraints apply.
\begin{urbiassert}
var lhs1 = Matrix.new([20, 0], [0, 200]);
var lhs2 = Matrix.new(lhs1);
var rhs = Matrix.new([10, 0], [0, 100]);
var res = Matrix.new([ 2, 0], [0,   2]);

(lhs1.'/='(rhs)) === lhs1;  lhs1 == res;
(lhs2  /=  rhs ) === lhs2;  lhs2 == res;

lhs1 /= Matrix.createZeros(2, 2);
[00207285:error] !!! /=: non-invertible matrix: <<0, 0>, <0, 0>>
\end{urbiassert}

\begin{urbiassert}
var v = Matrix.new([3, 6], [9, 12]);
var res = v / 3;
(v /= 3) === v; v == res;
\end{urbiassert}


\item|'[]'|(<row>, <col>)%
  The element at \lstinline|\var{row}, \var{col}|.  The index \var{row} must
  verify $0 \le \textrm{\var{row}} < \texttt{\this.size.first}$, or
  $-\texttt{\this.size.first} \le \texttt{\var{row}} < 0$, in which case it
  is equivalent to using index $\textrm{\var{row}} +
  \texttt{\this.size.first}$: it counts ``backward''.  Similarly for
  \var{col}.
\begin{urbiscript}
var m = Matrix.new([1, 2, 3], [10, 20, 30])|;
assert
{
  m[0, 0] == 1;   m[0, -3] == 1;
  m[0, 1] == 2;   m[0, -2] == 2;
  m[0, 2] == 3;   m[0, -1] == 3;

  m[1, 2] == 30;  m[-1, -1] == 30;
};

m[2, 0];
[00127812:error] !!! []: invalid row: 2

m[-3, 0];
[00127824:error] !!! []: invalid row: -3

m[0, 3];
[00127836:error] !!! []: invalid column: 3

m[0, -4];
[00127850:error] !!! []: invalid column: -4
\end{urbiscript}
\begin{urbicomment}
  removeSlots("m");
\end{urbicomment}


\item|'[]='|(<row>, <col>, <val>)%
  Set the element at \lstinline|\var{row}, \var{col}| to \var{val}, and
  return \var{val}.  The index \var{row} must verify $0 \le
  \textrm{\var{row}} < \texttt{\this.size.first}$, or
  $-\texttt{\this.size.first} \le \texttt{\var{row}} < 0$, in which case it
  is equivalent to using index $\textrm{\var{row}} +
  \texttt{\this.size.first}$: it counts ``backward''.  Similarly for
  \var{col}.
\begin{urbiscript}
var m = Matrix.new([1, 2], [10, 20])|;
assert
{
  (m[0, 0]  = -1) == -1;   m[0, 0] == -1;
  (m[-1, -1] = -2) == -2;  m[1, 1] == -2;
};

m[2, 0] = -1;
[00127812:error] !!! []=: invalid row: 2

m[-3, 0] = -1;
[00127824:error] !!! []=: invalid row: -3

m[0, 2] = -1;
[00127836:error] !!! []=: invalid column: 2

m[0, -3] = -1;
[00127850:error] !!! []=: invalid column: -3
\end{urbiscript}
\begin{urbicomment}
  removeSlots("m");
\end{urbicomment}

\end{urbiscriptapi}

%%% Local Variables:
%%% mode: latex
%%% TeX-master: "../urbi-sdk"
%%% ispell-dictionary: "american"
%%% ispell-personal-dictionary: "../urbi.dict"
%%% fill-column: 76
%%% End:
