%% Copyright (C) 2009-2010, Gostai S.A.S.
%%
%% This software is provided "as is" without warranty of any kind,
%% either expressed or implied, including but not limited to the
%% implied warranties of fitness for a particular purpose.
%%
%% See the LICENSE file for more information.

\section{Container}

This object is meant to be used as a prototype for objects that support
\refSlot{has} and \refSlot{hasNot} methods.  Any class using this prototype
must redefine either \refSlot{has}, \refSlot{hasNot} or both.

\subsection{Prototypes}

\begin{itemize}
\item \refObject{Object}
\end{itemize}

\subsection{Slots}

\begin{urbiscriptapi}
\item[has](<e>)%
  \lstinline|!hasNot(\var{e})|.  The indented semantics is
  ``\lstinline|true| when the container has a key (or item) matching
  \var{e}''.  This is what \lstinline|\var{e} in \var{c}| is mapped onto.

\begin{urbiscript}
class NotCell : Container
{
  var val;
  function init(var v)   { val = v };
  function hasNot(var v) { val != v };
}|;
var c = NotCell.new(23)|;
assert
{
  c.has(23);     23 in c;
  c.hasNot(3);    3 not in c;
};
\end{urbiscript}


\item[hasNot](<e>)%
  \lstinline|!has(\var{e})|.  The indented semantics is ``\lstinline|true|
  when the container does not have a key (or item) matching \var{e}''.

\begin{urbiscript}
class Cell : Container
{
  var val;
  function init(var v) { val = v };
  function has(var v)  { val == v };
}|;
var d = Cell.new(23)|;
assert
{
  d.has(23);     23 in d;
  d.hasNot(3);    3 not in d;
};
\end{urbiscript}
\end{urbiscriptapi}

%%% Local Variables:
%%% coding: utf-8
%%% mode: latex
%%% TeX-master: "../urbi-sdk"
%%% ispell-dictionary: "american"
%%% ispell-personal-dictionary: "../urbi.dict"
%%% fill-column: 76
%%% End:
