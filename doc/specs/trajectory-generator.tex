%% Copyright (C) 2010, 2011, Gostai S.A.S.
%%
%% This software is provided "as is" without warranty of any kind,
%% either expressed or implied, including but not limited to the
%% implied warranties of fitness for a particular purpose.
%%
%% See the LICENSE file for more information.

\section{TrajectoryGenerator}

The trajectory generators change the value of a given variable from an
\dfn{initial value} to a \dfn{target value}.  They can be
\dfn{open-loop}, i.e., the intermediate values depend only on the
initial and/or target value of the variable; or \dfn{closed-loop},
i.e., the intermediate values also depend on the current value value
of the variable.

Open-loop trajectories are insensitive to changes made elsewhere to
the variable.  Closed-loop trajectories \emph{are} sensitive to
changes made elsewhere to the variable --- for instance when the human
physically changes the position of a robot's motor.

Trajectory generators are not made to be used directly, rather use the
``continuous assignment'' syntax (\autoref{sec:lang:traj}).


\subsection{Examples}
\label{sec:traj:examples}

%% \subsubsection{Trajectory Name}
%% -------------------------------
\let\subsubsectionOrig\subsubsection
\renewcommand{\subsubsection}[1]
{%
  \subsubsectionOrig{\label{sec:traj:#1}#1}%
  %% It is on purpose that we pass [] to lstinline, because we do in many
  %% other places, and that would result in index entries that makeindex is
  %% unable to merge.
  \index{#1@\lstinline[]{#1}}%
}

\subsubsection{Accel}

The \lstinline{Accel} trajectory reaches a target value at a fixed
acceleration (\lstinline{accel} attribute).

\urbitrajectory{accel}

\subsubsection{Cos}

The \lstinline{Cos} trajectory implements a cosine around the target
value, given an amplitude (\lstinline{ampli} attribute) and period
(\lstinline{cos} attribute).

This trajectory is not ``smooth'': the initial value of the variable
is not taken into account.

\urbitrajectory{cos}

\subsubsection{Sin}

The \lstinline{Sin} trajectory implements a sine around the target
value, given an amplitude (\lstinline{ampli} attribute) and period
(\lstinline{sin} attribute).

This trajectory is not ``smooth'': the initial value of the variable
is not taken into account.

\urbitrajectory{sin}

\subsubsection{Smooth}

The \lstinline{Smooth} trajectory implements a sigmoid.  It changes
the variable from its current value to the target value ``smoothly''
in a given amount of time (\lstinline{smooth} attribute).

\urbitrajectory{smooth}

\subsubsection{Speed}

The \lstinline{Speed} trajectory changes the value of the variable
from its current value to the target value at a fixed speed (the
\lstinline{speed} attribute).

\urbitrajectory{speed}

If the \lstinline{adaptive} attribute is set to true, then the
duration of the trajectory is constantly reevaluated.

\urbitrajectory{speed-adaptive}

\subsubsection{Time}

The \lstinline{Time} trajectory changes the value of the variable from
its current value to the target value within a given duration (the
\lstinline{time} attribute).

\urbitrajectory{time}

If the \lstinline{adaptive} attribute is set to true, then the
duration of the trajectory is constantly reevaluated.

\urbitrajectory{time-adaptive}

%% Restore the definition of \subsubsection.
\let\subsubsection\subsubsectionOrig

\subsubsection{Trajectories and Tags}

Trajectories can be managed using \refObject[Tag]{Tags}.  Stopping or blocking
a tag that manages a trajectory kill the trajectory.

\urbitrajectory{cos-stop}
\urbitrajectory{cos-block}

When a trajectory is frozen, its local time is frozen too, the movement
proceeds from where it was rather than from where it would have been had it
been not frozen.

\urbitrajectory{cos-freeze}


\subsection{Prototypes}
\begin{refObjects}
\item[Object]
\end{refObjects}

\subsection{Construction}

You are not expected to construct trajectory generators by hand, using
modifiers is the recommended way to construct trajectories.  See
\autoref{sec:lang:traj} for details about trajectories, and see
\autoref{sec:traj:examples} for an extensive set of examples.

\subsection{Slots}

\begin{urbiscriptapi}
\item[Accel] This class implements the \lstinline|Accel| trajectory
  (\autoref{sec:traj:Accel}).  It derives from
  \refSlot{OpenLoop}.


\item[ClosedLoop] This class factors the implementation of the
  \dfn{closed-loop} trajectories.  It derives from
  \lstinline|TrajectoryGenerator|.


\item[OpenLoop] This class factors the implementation of the \dfn{open-loop}
  trajectories.  It derives from \lstinline|TrajectoryGenerator|.


\item[Sin] This class implements the \lstinline|Cos| and \lstinline|Sin|
  trajectories (\autoref{sec:traj:Cos}, \autoref{sec:traj:Sin}).  It derives
  from \refSlot{OpenLoop}.


\item[Smooth] This class implements the \lstinline|Smooth| trajectory
  (\autoref{sec:traj:Smooth}).  It derives from
  \refSlot{OpenLoop}.


\item[SpeedAdaptive] This class implements the \lstinline|Speed| trajectory
  when the \lstinline|adaptive| attribute is given
  (\autoref{sec:traj:Speed}).  It derives from
  \refSlot{ClosedLoop}.


\item[Time] This class implements the non-adaptive \lstinline|Speed| and
  \lstinline|Time| trajectories (\autoref{sec:traj:Speed},
  \autoref{sec:traj:Time}).  It derives from
  \refSlot{OpenLoop}.


\item[TimeAdaptive] This class implements the \lstinline|Time| trajectory
  when the \lstinline|adaptive| attribute is given
  (\autoref{sec:traj:Time}).  It derives from \refSlot{ClosedLoop}.
\end{urbiscriptapi}

%%% Local Variables:
%%% coding: utf-8
%%% mode: latex
%%% TeX-master: "../urbi-sdk"
%%% ispell-dictionary: "american"
%%% ispell-personal-dictionary: "../urbi.dict"
%%% fill-column: 76
%%% End:
