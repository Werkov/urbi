%% Copyright (C) 2009-2010, Gostai S.A.S.
%%
%% This software is provided "as is" without warranty of any kind,
%% either expressed or implied, including but not limited to the
%% implied warranties of fitness for a particular purpose.
%%
%% See the LICENSE file for more information.

\section{Directory}

A \dfn{Directory} represents a directory of the file system.

\subsection{Prototypes}
\begin{refObjects}
\item[Object]
\end{refObjects}

\subsection{Construction}

A \dfn{Directory} can be constructed with one argument: the path of
the directory using a \refObject{String} or a \refObject{Path}. It can
also be constructed by the method open of \refObject{Path}.

\begin{urbiscript}[firstnumber=1]
Directory.new(".");
[00000001] Directory(".")
Directory.new(Path.new("."));
[00000002] Directory(".")
\end{urbiscript}

\subsection{Slots}
\begin{urbiscriptapi}
\item[asList]
  The contents of the directory as a \refObject{Path} list.  The
  various paths include the name of the directory \lstinline|this|.

\item[asList]
  \refObject{String} containing the path of the directory.

\begin{urbiscript}
Directory.new(".").asString;
[00000003] "."
\end{urbiscript}


\item[content]
  The contents of the directory as a \refObject{String} list.  The
  strings include only the last component name; they do not contain
  the directory name of \lstinline|this|.

\item \lstinline|fileCreated(\var{name)|\\
  Event launched when a file is created inside the directory.  Be careful,
  this slot may not exists if the \us interpreter does not support for it.

%% firstline is used to separated inotify test from the others.
\begin{urbiscript}[firstnumber=1]
if (Path.new("./dummy.txt").exists)
  File.new("./dummy.txt").remove;

{
  var d = Directory.new(".");
  waituntil(d.fileCreated?(var name));
  echo("%s is created." % name);
  Path.new(d.asString + "/" + name).exists.print;
} & {
  // somewhen, a file is created
  sleep(100ms);
  File.create("./dummy.txt");
}|;
[00000004] *** dummy.txt is created.
[00000005] true
\end{urbiscript}

\item \lstinline|fileDeleted(\var{name)|\\
  Event launched when a file is deleted from the directory.  Be careful,
  this slot may not exists if the \us interpreter does not support for it.

\begin{urbiscript}
if (!Path.new("./dummy.txt").exists)
  File.create("./dummy.txt")|;

{
  var d = Directory.new(".");
  waituntil(d.fileDeleted?(var name));
  echo("%s is deleted." % name);
  Path.new(d.asString + "/" + name).exists.print;
} & {
  // somewhen, a file is deleted
  sleep(100ms);
  File.new("./dummy.txt").remove;
}|;
[00000006] *** dummy.txt is deleted.
[00000007] false
\end{urbiscript}
%% Use firstline after this test if this is not related to inotify.


\end{urbiscriptapi}


%%% Local Variables:
%%% mode: latex
%%% TeX-master: "../urbi-sdk"
%%% ispell-dictionary: "american"
%%% ispell-personal-dictionary: "../urbi.dict"
%%% fill-column: 76
%%% End:
