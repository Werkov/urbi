\section{Lazy}

\dfn{Lazies} are objects that hold a lazy value, that is, a not yet evaluated
value. They provide facilities to evaluate their content only once
(\dfn{memoization}) or several times. Lazy are essentially used in call
messages, to represent lazy arguments, as described in
\autorefObject{CallMessage}.

\subsection{Examples}

\subsubsection{Evaluating once}

One usage of lazy values is to avoid evaluating an expression unless
it's actually needed, because it's expensive or has undesired side
effects. The listing below presents a situation where an
expensive-to-compute value (\lstinline|heavy_computation|) might be
needed zero, one or two times. The objective is to save time by:

\begin{itemize}
\item Not evaluating it if it's not needed.
\item Evaluating it only once if it's needed one or two time.
\end{itemize}

We thus make the wanted expression lazy, and use the \lstinline|value|
method to fetch its value when needed.

\begin{urbiscript}
// This function supposedly performs expensive computations.
function heavy_computation()
{
  echo("Heavy computation");
  return 1 + 1;
}|;

// We want to do the heavy computations only if needed,
// and make it a lazy value to be able to evaluate it "on demand".
var v = Lazy.new(closure () { heavy_computation() });
[00000000] heavy_computation()
/* some code */;
// So far, the value was not needed, and heavy_computation
// was not evaluated.
/* some code */;
// If the value is needed, heavy_computation is evaluated.
v.value;
[00000000] *** Heavy computation
[00000000] 2
// If the value is needed a second time, heavy_computation
// is not reevaluated.
v.value;
[00000000] 2
\end{urbiscript}

\subsubsection{Evaluating several times}

Evaluating a lazy several times only makes sense with lazy arguments
and call messages. See example with call messages in
\autoref{sec:std-callmsg-examples-several}.


\subsection{Caching}

\refObject{Lazy} is meant for functions without argument.  If you need
\dfn{caching} for functions that depend on arguments, it is
straightforward to implement using a \refObject{Dictionary}.  In the
future \us might support dictionaries whose indices are not only
strings, but in the meanwhile, convert the arguments into
strings, as the following sample object demonstrates.

\begin{urbiscript}[firstnumber=last]
class UnaryLazy
{
  function init(f)
  {
    results = Dictionary.new;
    func = f;
  };
  function value(p)
  {
    var sp = p.asString;
    if (results.has(sp))
      return results[sp];
    var res = func(p);
    results[sp] = res |
    res
  };
  var results;
  var func;
} |
// The function to cache.
var inc = function(x) { echo("incing " + x) | x+1 } |
// The function with cache.
// Use "getSlot" to get the unevaluated function.
var p = UnaryLazy.new(getSlot("inc"));
[00062847] UnaryLazy_0x78b750
p.value(1);
[00066758] *** incing 1
[00066759] 2
p.value(1);
[00069058] 2
p.value(2);
[00071558] *** incing 2
[00071559] 3
p.value(2);
[00072762] 3
p.value(1);
[00074562] 2
\end{urbiscript}

\subsection{Prototypes}

\begin{itemize}
\item \refObject{Comparable}
\end{itemize}

\subsection{Construction}

Lazies are seldom instantiated manually. They are mainly created
automatically when a lazy function call is made (see
\autoref{sec:us-fun-callmsg}). One can however create a lazy value with the
standard \lstinline|new| method of \lstinline|Lazy|, giving it an
argument-less function which evaluates to the lazified value.

\begin{urbiscript}[firstnumber=last]
Lazy.new(closure () { /* Value to lazify */ 0 });
[00000000] 0
\end{urbiscript}

\subsection{Slots}

\begin{itemize}
\item \lstinline|==(\var{that})|\\
  Whether \lstinline|this| and \var{that} are the same source code and
  value (an unevaluated Lazy is never equal to an evaluated one).
\begin{urbiassert}[firstnumber=last]
Lazy.new(closure () { 1 + 1 }) == Lazy.new(closure () { 1 + 1 });
Lazy.new(closure () { 1 + 2 }) != Lazy.new(closure () { 2 + 1 });
\end{urbiassert}
\begin{urbiscript}[firstnumber=last]
{
  var l1 = Lazy.new(closure () { 1 + 1 });
  var l2 = Lazy.new(closure () { 1 + 1 });
  assert (l1 == l2);
  l1.eval;
  assert (l1 != l2);
  l2.eval;
  assert (l1 == l2);
};
\end{urbiscript}

\item \lstinline|asString|\\
  The conversion to \refObject{String} of the body of a non-evaluated
  argument.
\begin{urbiassert}[firstnumber=last]
Lazy.new(closure () { echo(1); 1 }).asString == "echo(1);\n1";
\end{urbiassert}

\item \lstinline|eval|\\
  Force the evaluation of the held lazy value. Two calls to
  \lstinline|eval| will systematically evaluate the expression twice,
  which can be useful to duplicate its side effects.

\item \lstinline|value|\\
  Return the held value, potentially evaluating it
  before. \lstinline|value| performs memoization, that is, only the
  first call will actually evaluate the expression, subsequent calls
  will return the cached value. Unless you want to explicitly trigger
  side effects from the expression by evaluating it several time, this
  should be preferred over \lstinline|eval| to avoid evaluating the
  expression several times uselessly.
\end{itemize}


%%% Local Variables:
%%% mode: latex
%%% TeX-master: "../urbi-sdk"
%%% End:
