%% Copyright (C) 2008-2010, Gostai S.A.S.
%%
%% This software is provided "as is" without warranty of any kind,
%% either expressed or implied, including but not limited to the
%% implied warranties of fitness for a particular purpose.
%%
%% See the LICENSE file for more information.

\section{List}

\lstinline|List|s implement potentially-empty ordered (heterogeneous)
collections of elements.

\subsection{Prototypes}

\begin{refObjects}
\item[Object]
 \item[RangeIterable]
 \item[Orderable]
\end{refObjects}

\subsection{Construction}

List can be created with their literal syntax: a possibly empty
sequence of expressions in square brackets, separated by commas.
Non-empty list may actually \emph{terminate} with a comma, rather than
\emph{separate}; in other words, an optional trailing comma is accepted.

\begin{urbiscript}[firstnumber=1]
[]; // The empty list
[00000000] []
[1, "2", [3,],];
[00000000] [1, "2", [3]]
\end{urbiscript}

\subsection{Slots}

\begin{urbiscriptapi}
\item \lstinline|all(\var{fun})|\\
  % FIXME: link to predicate glossary entry
  Return whether all the members of the target verify the predicate
  \var{fun}.

\begin{urbiassert}
// Are all elements positive?
! [-2, 0, 2, 4].all(function (e) { e > 0 });
// Are all elements even?
[-2, 0, 2, 4].all(function (e) { e % 2 == 0 });
\end{urbiassert}

\item \lstinline|any(\var{fun})|\\
  % FIXME: link to predicate glossary entry
  Whether at least one of the members of the target verifies the
  predicate \var{fun}.

\begin{urbiassert}
// Is there any even element?
! [-3, 1, -1].any(function (e) { e % 2 == 0 });
// Is there any positive element?
[-3, 1, -1].any(function (e) { e > 0 });
\end{urbiassert}

\item[asBool]
  Whether not empty.
\begin{urbiassert}
[].asBool == false;
[1].asBool == true;
\end{urbiassert}

\item[asList]
Return the target.

\begin{urbiassert}
[0, 1, 2].asList == [0, 1, 2];
\end{urbiassert}

\item[asString]
  A string describing the list.  Uses \lstinline|asPrintable| on its
  members, so that, for instance, strings are displayed with quotes.

\begin{urbiassert}
[0, [1], "2"].asString == "[0, [1], \"2\"]";
\end{urbiassert}

\item[back]
  The last element of the target. An error if the target is empty.

\begin{urbiscript}
assert([0, 1, 2].back == 2);
[].back;
[00000000:error] !!! back: cannot be applied onto empty list
\end{urbiscript}

\item[clear]
  Empty the target.

\begin{urbiscript}
var x = [0, 1, 2];
[00000000] [0, 1, 2]
assert(x.clear == []);
\end{urbiscript}

\item \lstinline|each(\var{fun})|\\
  Apply the given functional value \var{fun} on all members,
  sequentially.

\begin{urbiscript}
[0, 1, 2].each(function (v) {echo (v * v); echo (v * v)});
[00000000] *** 0
[00000000] *** 0
[00000000] *** 1
[00000000] *** 1
[00000000] *** 4
[00000000] *** 4
\end{urbiscript}

\item \lstinline|eachi(\var{fun})|\\
  Apply the given functional value \var{fun} on all members
  sequentially, additionally passing the current element index.

\begin{urbiscript}
["a", "b", "c"].eachi(function (v, i) {echo ("%s: %s" % [i, v])});
[00000000] *** 0: a
[00000000] *** 1: b
[00000000] *** 2: c
\end{urbiscript}

\item \lstinline|'each&'(\var{fun})|\\
Apply the given functional value on all members simultaneously.

\begin{urbiscript}
[0, 1, 2].'each&'(function (v) {echo (v * v); echo (v * v)});
[00000000] *** 0
[00000000] *** 1
[00000000] *** 4
[00000000] *** 0
[00000000] *** 1
[00000000] *** 4
\end{urbiscript}

\item[empty]
  Whether the target is empty.

\begin{urbiassert}
[].empty;
! [1].empty;
\end{urbiassert}

\item \lstinline|filter(\var{fun})|\\
  The list of all the members of the target that verify the predicate
  \var{fun}.

\begin{urbiscript}
do ([0, 1, 2, 3, 4, 5])
{
  assert
  {
    // Keep only odd numbers.
    filter(function (v) {v % 2 == 1}) == [1, 3, 5];
    // Keep all.
    filter(function (v) { true })     == this;
    // Keep none.
    filter(function (v) { false })    == [];
  };
}|;
\end{urbiscript}

\item \lstinline|foldl(\var{action}, \var{value})|\\
  \wref[Fold_(higher-order_function)]{Fold},
  also known as \dfn{reduce} or \dfn{accumulate}, computes a result
  from a list.  Starting from \var{value} as the initial result, apply
  repeatedly the binary \var{action} to the current result and the
  next member of the list, from left to right.  For instance, if
  \var{action} were the binary addition and \var{value} were 0, then
  folding a list would compute the sum of the list, including for
  empty lists.

\begin{urbiscript}
[].foldl(function (a, b) { a + b }, 0);
[00000000] 0
[1, 2, 3].foldl(function (a, b) { a + b }, 0);
[00000000] 6
[1, 2, 3].foldl(function (a, b) { a - b }, 0);
[00000000] -6
\end{urbiscript}

\item[front]
  Return the first element of the target. An error if the target is
  empty.
\begin{urbiscript}
assert([0, 1, 2].front == 0);
[].front;
[00000000:error] !!! front: cannot be applied onto empty list
\end{urbiscript}

\item \lstinline|has(\var{that})|\\
  Whether one of the members of the target equals the argument.

\begin{urbiassert}
[0, 1, 2].has(1);
! [0, 1, 2].has(5);
\end{urbiassert}

  The infix operators \lstinline|in| and \lstinline|not in| use
  \lstinline|has| (see \autoref{sec:lang:operators:containers}).

\begin{urbiassert}
  1 in     [0, 1];
  2 not in [0, 1];
!(2 in     [0, 1]);
!(1 not in [0, 1]);
\end{urbiassert}

\item \lstinline|hasSame(\var{that})|\\
  Whether one of the members of the target is physically equal to
  \var{that}.
\begin{urbiscript}
var y = 1|;
assert
{
   [0, y, 2].hasSame(y);
  ![0, y, 2].hasSame(1);
};
\end{urbiscript}

\item[head]
  Synonym for \lstinline|front|.
\begin{urbiscript}
assert([0, 1, 2].head == 0);
[].head;
[00000000:error] !!! head: cannot be applied onto empty list
\end{urbiscript}

\item \lstinline|insertBack(\var{that})|\\
  Insert the given element at the end of the target.

\begin{urbiscript}
do ([0, 1])
{
  assert
  {
    insertBack(2) == [0, 1, 2];
    this          == [0, 1, 2];
  };
}|;
\end{urbiscript}

\item \lstinline|insert(\var{where}, \var{what})|\\
  Insert \var{what} before the value at index \var{where}, return
  \lstinline|this|.
\begin{urbiscript}
do ([0, 1])
{
  assert
  {
    insert(0, 10) == [10, 0, 1];
    this          == [10, 0, 1];
    insert(2, 20) == [10, 0, 20, 1];
    this          == [10, 0, 20, 1];
  };
  insert(4, 30);
}|;
[00044239:error] !!! insert: invalid index: 4
\end{urbiscript}

  The index must be valid, to insert past the end, use
  \lstinline|insertBack|.
\begin{urbiscript}
[].insert(0, "foo");
[00044239:error] !!! insert: invalid index: 0
\end{urbiscript}


\item \lstinline|insertFront(\var{that})|\\
  Insert the given element at the beginning of the target.

\begin{urbiscript}
do ([1, 2])
{
  assert
  {
    insertFront(0) == [0, 1, 2];
    this           == [0, 1, 2];
  };
}|;
\end{urbiscript}

\item \lstinline|join(\var{sep} = "", \var{prefix} = "", \var{suffix} = "")|\\
  Bounces to \refSlot{String}{join}, see \refObject{String}.

\begin{urbiassert}
["", "ob", ""].join                == "ob";
["", "ob", ""].join("a")           == "aoba";
["", "ob", ""].join("a", "B", "b") == "Baobab";
\end{urbiassert}

\item \lstinline|keys()|\\
  The list of valid indexes.  This allows uniform iteration over a
  \refObject{Dictionary} or a \refObject{List}.

\begin{urbiscript}
{
  var l = ["a", "b", "c"];
  assert(l.keys == [0, 1, 2]);
  assert({
           var res = [];
           for (var k: l.keys)
             res << l[k];
           res
         }
         == l);
};
\end{urbiscript}

\item \lstinline|map(\var{fun})|\\
Apply the given functional value on every member, and return the list
of results.

\begin{urbiassert}
[0, 1, 2, 3].map(function (v) { v % 2 == 0})
        == [true, false, true, false];
\end{urbiassert}

\item \lstinline|range(\var{begin}, \var{end} = nil)|\\
  Return a sub-range of the list, from the first index included to the
  second index excluded.  An error if out of bounds.  Negative indices
  are valid, and number from the end.

  If \var{end} is \lstinline|nil|, calling \lstinline|range(\var{n})|
  is equivalent to calling \lstinline|range(0, \var{n})|.

\begin{urbiscript}
do ([0, 1, 2, 3])
{
  assert
  {
    range(0, 0)   == [];
    range(0, 1)   == [0];
    range(1)      == [0];
    range(1, 3)   == [1, 2];

    range(-3, -2) == [1];
    range(-3, -1) == [1, 2];
    range(-3, 0)  == [1, 2, 3];
    range(-3, 1)  == [1, 2, 3, 0];
    range(-4, 4)  == [0, 1, 2, 3, 0, 1, 2, 3];
  };
}|;
[].range(1, 3);
[00428697:error] !!! range: invalid index: 1
\end{urbiscript}

\item \lstinline|remove(\var{val})|\\
  Remove all elements from the target that equals \var{val}.

\begin{urbiscript}
var c = [0, 1, 0, 2, 0, 3];
[00000000] [0, 1, 0, 2, 0, 3]
assert(c.remove(0) == [1, 2, 3]);
assert(c == [1, 2, 3]);
\end{urbiscript}

\item[removeBack]
  Remove and return the last element of the target. An error if the
  target is empty.

\begin{urbiscript}
var t = [0, 1, 2];
[00000000] [0, 1, 2]
assert(t.removeBack == 2);
assert(t == [0, 1]);
[].removeBack;
[00000000:error] !!! removeBack: cannot be applied onto empty list
\end{urbiscript}

\item \lstinline|removeById(\var{that})|\\
  Remove all elements from the target that physically equals
  \var{that}.

\begin{urbiscript}
var d = 1;
[00000000] 1
var e = [0, 1, d, 1, 2];
[00000000] [0, 1, 1, 1, 2]
assert(e.removeById(d) == [0, 1, 1, 2]);
assert(e == [0, 1, 1, 2]);
\end{urbiscript}

\item[removeFront]
Remove and return the first element from the target. An error if the
target is empty.

\begin{urbiscript}
var g = [0, 1, 2];
[00000000] [0, 1, 2]
assert(g.removeFront == 0);
assert(g == [1, 2]);
[].removeFront;
[00000000:error] !!! removeFront: cannot be applied onto empty list
\end{urbiscript}

\item[reverse]
Return the target with the order of elements inverted.

\begin{urbiassert}
[0, 1, 2].reverse == [2, 1, 0];
\end{urbiassert}

\item[size]
Return the number of elements in the target.

\begin{urbiassert}
[0, 1, 2].size == 3;
[].size == 0;
\end{urbiassert}

\item[sort]
Return the target, sorted with respect to the \lstinline|<| criteria.

\begin{urbiassert}
[1, 0, 3, 2].sort == [0, 1, 2, 3];
\end{urbiassert}

\item \lstinline|subset(\var{that})|\\
  Whether the members of \lstinline|this| are members of \var{that}.

\begin{urbiassert}
[].subset([]);
[].subset([1, 2, 3]);
[3, 2, 1].subset([1, 2, 3]);
[1, 3].subset([1, 2, 3]);
[1, 1].subset([1, 2, 3]);
![3].subset([]);
![3, 2].subset([1, 2]);
![1, 2, 3].subset([1, 2]);
\end{urbiassert}

\item[tail]
  Return the target, minus the first element. An error if the target
  is empty.

\begin{urbiscript}
assert([0, 1, 2].tail == [1, 2]);
[].tail;
[00000000:error] !!! tail: cannot be applied onto empty list
\end{urbiscript}

\item \lstinline|'=='(\var{that})|\\
Check whether all elements in the target and \var{that}, are
equal two by two.

\begin{urbiassert}
[0, 1, 2] == [0, 1, 2];
!([0, 1, 2] == [0, 0, 2]);
\end{urbiassert}

\item \lstinline|'[]'(\var{n})|\\
  Return the \var{n}th member of the target (indexing is
  zero-based). If \var{n} is negative, start from the end.  An error
  if out of bounds.

\begin{urbiscript}
assert(["0", "1", "2"][0] == "0");
assert(["0", "1", "2"][2] == "2");
["0", "1", "2"][3];
[00007061:error] !!! []: invalid index: 3

assert(["0", "1", "2"][-1] == "2");
assert(["0", "1", "2"][-3] == "0");
["0", "1", "2"][-4];
[00007061:error] !!! []: invalid index: -4
\end{urbiscript}

\item \lstinline|'[]='(\var{index}, \var{value})|\\
  Assign \var{value} to the element of the target at the given
  \var{index}.

\begin{urbiscript}
var f = [0, 1, 2];
[00000000] [0, 1, 2]
f[1] = 42;
[00000000] 42
assert(f == [0, 42, 2]);
\end{urbiscript}

\item \lstinline|'*'(\var{n})|\\
  Return the target, concatenated \var{n} times to itself.
\begin{urbiassert}
[0, 1] * 0 == [];
[0, 1] * 3 == [0, 1, 0, 1, 0, 1];
\end{urbiassert}

  \var{n} must be a non-negative integer.

\begin{urbiscript}
[0, 1] * -2;
[00000063:error] !!! *: expected non-negative integer, got -2
\end{urbiscript}


  Note that since it is the very same list which is repeatedly
  concatenated (the content is not cloned), side-effects on one item
  will reflect on ``all the items''.

\begin{urbiscript}
var l = [[]] * 3;
[00000000] [[], [], []]
l[0] << 1;
[00000000] [1]
l;
[00000000] [[1], [1], [1]]
\end{urbiscript}

\item \lstinline|'+'(\var{other})|\\
  Return the concatenation of the target and the \var{other} list.

\begin{urbiassert}
[0, 1] + [2, 3] == [0, 1, 2, 3];
[] + [2, 3] == [2, 3];
[0, 1] + [] == [0, 1];
[] + [] == [];
\end{urbiassert}

\item \lstinline|'-'(\var{other})|\\
  Return the target without the elements that are equal to any element
  in the \var{other} list.

\begin{urbiassert}
[0, 1, 0, 2, 3] - [1, 2] == [0, 0, 3];
[0, 1, 0, 1, 0] - [1, 2] == [0, 0, 0];
\end{urbiassert}

\item \lstinline|'<<'(\var{that})|\\
  A synonym for \lstinline|insertBack|.

\item \lstinline|'<'(\var{other})|\\
  Return whether the target is inferior to the \var{other} list. A
  list is inferior to another if at least one of its element differs
  from the other, and the first differing element is inferior to the
  other.

\begin{urbiassert}
!([0, 1, 2] < [0, 1, 2]);
!([0, 1, 2] < [0, 0, 2]);
[0, 1, 2] < [0, 2, 2];
\end{urbiassert}

  Since List derives from \refObject{Orderable}, the other order-based
  operators are defined.

\begin{urbiassert}
 [0, 1, 2] <= [0, 1, 2];
 [0, 1, 2] >= [0, 1, 2];
 [0, 1, 2] >  [0, 0, 2];
\end{urbiassert}
\end{urbiscriptapi}

%%% Local Variables:
%%% mode: latex
%%% TeX-master: "../urbi-sdk"
%%% ispell-dictionary: "american"
%%% ispell-personal-dictionary: "../urbi.dict"
%%% fill-column: 76
%%% End:
