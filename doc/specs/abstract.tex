%% Copyright (C) 2010, 2011, Gostai S.A.S.
%%
%% This software is provided "as is" without warranty of any kind,
%% either expressed or implied, including but not limited to the
%% implied warranties of fitness for a particular purpose.
%%
%% See the LICENSE file for more information.

\begin{partDescription}{part:specs}
  {
    %
    This part defines the specifications of the \us language. It defines the
    expected behavior from the \us interpreter, the standard library, and
    the \sdk. It can be used to check whether some code is valid, or browse
    \us or \Cxx \api for a desired feature. Random reading can also provide
    you with advanced knowledge or subtleties about some \us aspects.

    This part is not an \us tutorial; it is not structured in a progressive
    manner and is too detailed.  Think of it as a dictionary: one does not
    learn a foreign language by reading a dictionary. For an \us Tutorial,
    see \autoref{part:tut}.

    This part does not aim at giving advanced programming techniques. Its
    only goal is to define the language and its libraries.
    %
  }
\item[sec:tools] Presentation and usage of the different tools available
  with the \urbi framework related to \us, such as the \urbi server, the
  command line client, \umake, \ldots

\item[sec:lang] Core constructs of the language and their semantics.

\item[sec:stdlib] The classes and methods provided in the standard library.

\item[sec:specs:ros] \urbi provides a set of tools to communicate with ROS
  (Robot Operating System).  For more information about ROS, see
  \url{http://www.ros.org}.

\item[sec:naming] Also known as ``The \urbi Naming Standard'': naming
  conventions in for standard hardware/software devices and components
  implemented as UObject and the corresponding slots/events to access them.
\end{partDescription}


%%% Local Variables:
%%% coding: utf-8
%%% mode: latex
%%% TeX-master: "../urbi-sdk"
%%% ispell-dictionary: "american"
%%% ispell-personal-dictionary: "../urbi.dict"
%%% fill-column: 76
%%% End:
