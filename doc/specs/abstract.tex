\begin{partDescription}{part:specs}
  {
    %
    This part defines the specifications of the \us language version
    2.0. It defines the expected behavior from the \us interpreter,
    the standard library, and the \sdk. It can be used to check
    whether some code is valid, or browse \us or \Cxx \api for a
    desired feature. Random reading can also provide you with advanced
    knowledge or subtleties about some \us aspects.

    This part is not an \us tutorial; it is not structured in a
    progressive manner and is too detailed.  Think of it as a
    dictionary: one does not learn a foreign language by reading a
    dictionary. The \us Tutorial (\autoref{part:tut}), or the live \us
    tutorial built in the interpreter are good introductions to \us.

    This part does not aim at giving advanced programming
    techniques. Its only goal is to define the language and its
    libraries.
    %
  }
\item[sec:tools]
  Presentation and usage of the different tools available with the
  \urbi framework related to \us, such as the \urbi server, the
  command line client, \umake, \ldots

\item[sec:lang]
  Core constructs of the language and their behavior.

\item[sec:stdlib]
  Listing of all classes and methods provided in the standard library.

\item[sec:naming]
  Also known as ``The \urbi Naming Standard'': naming conventions in
  for standard hardware/software devices and components implemented as
  UObject and the corresponding slots/events to access them.

%\item[sec:sdk]
%  The \urbi software development kit that enable to
%  interact with \urbi from \Cxx.
\end{partDescription}


%%% Local Variables:
%%% mode: latex
%%% TeX-master: "../urbi-sdk"
%%% ispell-dictionary: "american"
%%% ispell-personal-dictionary: "../urbi.dict"
%%% End:
