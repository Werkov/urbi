\subsection{Rationale}

Float is an \us primitive to represent floating point number.

\subsection{Prototypes}

\begin{itemize}
\item Comparable (\autoref{sec:std-comparable}).
\item Orderable (\autoref{sec:std-orderable})
\end{itemize}

\subsection{Construction}

The most common way to create fresh floats is by using the literal
syntax presented in \autoref{sec:us-syn-lit-float}. A null float can also be
obtained with \lstinline|Float|'s \lstinline|new| method
(\autoref{lst:new-float}).

\begin{lstlisting}[caption=Creating a new float, label=lst:new-float,
  float=\floatpos]
  Float.new; [00000000] 0
\end{lstlisting}

\subsection{Methods}


\subsubsection{abs}

Return the absolute value of the target (\autoref{lst:float-abs}).

\begin{lstlisting}[caption=Float.abs, label=lst:float-abs,
  float=\floatposh]
  (-5).abs;
  [00000000] 5
  (5).abs;
  [00000000] 5
\end{lstlisting}

\subsubsection{acos}

Return the arcosinus of the target.

\begin{lstlisting}[caption=Float.acos, label=lst:float-acos,
  float=\floatposh]
  1.acos;
  [00000000] 0
\end{lstlisting}

\subsubsection{asin}

Return the arcsinus of the target.

\begin{lstlisting}[caption=Float.asin, label=lst:float-asin,
  float=\floatposh]
  0.asin;
  [00000000] 0
\end{lstlisting}

\subsubsection{asString}

Return a string representing the target (\autoref{lst:float-asString}).

\begin{lstlisting}[caption=Float.asString, label=lst:float-asString,
  float=\floatposh]
  42.asString;
  [00000000] "42"
\end{lstlisting}

\subsubsection{atan}

Return the arctangent of the target (\autoref{lst:float-atan}).

\begin{lstlisting}[caption=Float.atan, label=lst:float-atan,
  float=\floatposh]
  0.atan;
  [00000000] 0
\end{lstlisting}

\subsubsection{bitand}

Return the bitwise and between the target and the argument (\autoref{lst:float-bitand}).

\begin{lstlisting}[caption=Float.bitand, label=lst:float-bitand,
  float=\floatposh]
  3 bitand 6;
  [00000000] 2
\end{lstlisting}

\subsubsection{bitor}

Return the bitwise or between the target and the argument (\autoref{lst:float-bitor}).

\begin{lstlisting}[caption=Float.bitor, label=lst:float-bitor,
  float=\floatposh]
  3 bitand 6;
  [00000000] 7
\end{lstlisting}

\subsubsection{clone}

Return a fresh Float with the same value as the target (\autoref{lst:float-clone}).

\begin{lstlisting}[caption=Float.clone, label=lst:float-clone,
  float=\floatposh]
  var x = 0;
  [00000000] 0
  var y = x.clone;
  [00000000] 0
  x === y;
  [00000000] false
\end{lstlisting}

\subsubsection{compl}

Return the 1-complement of the target (\autoref{lst:float-compl}).

\begin{lstlisting}[caption=Float.compl, label=lst:float-compl,
  float=\floatposh]
  compl 1;
  [00000000] -2
\end{lstlisting}

\subsubsection{cos}

Return the cosinus of the target (\autoref{lst:float-cos}).

\begin{lstlisting}[caption=Float.cos, label=lst:float-cos,
  float=\floatposh]
  0.cos;
  [00000000] 1
\end{lstlisting}

\subsubsection{exp}

Return the exponential of the target (\autoref{lst:float-exp}).

\begin{lstlisting}[caption=Float.exp, label=lst:float-exp,
  float=\floatposh]
  1.exp;
  [00000000] 2.71828
\end{lstlisting}

\subsubsection{inf}

Return the infinity (\autoref{lst:float-inf}).

\begin{lstlisting}[caption=Float.inf, label=lst:float-inf,
  float=\floatposh]
  Float.inf;
  [00000000] inf
\end{lstlisting}

\subsubsection{log}

Return the logarithm of the target (\autoref{lst:float-log}).

\begin{lstlisting}[caption=Float.log, label=lst:float-log,
  float=\floatposh]
  2.71828.log;
  [00000000] 0.999999
\end{lstlisting}

\subsubsection{nan}

Return the ``not a number'' special float value (\autoref{lst:float-nan}).

\begin{lstlisting}[caption=Float.nan, label=lst:float-nan,
  float=\floatposh]
  Float.nan;
  [00000000] nan
\end{lstlisting}

\subsubsection{random}

Return a random number between 0 and the target(\autoref{lst:float-random}).

\begin{lstlisting}[caption=Float.random, label=lst:float-random,
  float=\floatposh]
  5.random;
  [00000000] 3
  5.random;
  [00000000] 1
  5.random;
  [00000000] 2
\end{lstlisting}

\subsubsection{round}

Return the target, rounded (\autoref{lst:float-round}).

\begin{lstlisting}[caption=Float.round, label=lst:float-round,
  float=\floatposh]
  1.6.round;
  [00000000] 2
  1.4.round;
  [00000000] 1
\end{lstlisting}

\subsubsection{seq}

Return the sequence of integer from 0 to target - 1 as a list (\autoref{lst:float-seq}).

\begin{lstlisting}[caption=Float.seq, label=lst:float-seq,
  float=\floatposh]
  3.seq;
  [00000000] [0, 1, 2]
\end{lstlisting}

\subsubsection{sin}

Return the sinus of the target (\autoref{lst:float-sin}).

\begin{lstlisting}[caption=Float.sin, label=lst:float-sin,
  float=\floatposh]
  0.sin;
  [00000000] 0
\end{lstlisting}

\subsubsection{sqrt}

Return the square root of the target (\autoref{lst:float-sqrt}).

\begin{lstlisting}[caption=Float.sqrt, label=lst:float-sqrt,
  float=\floatposh]
  1024.sqrt;
  [00000000] 32
\end{lstlisting}

\subsubsection{tan}

Return the tangent of the target (\autoref{lst:float-tan}).

\begin{lstlisting}[caption=Float.tan, label=lst:float-tan,
  float=\floatposh]
  0.tan;
  [00000000] 0
\end{lstlisting}

\subsubsection{trunc}

Return the target truncated  (\autoref{lst:float-trunc}).

\begin{lstlisting}[caption=Float.trunc, label=lst:float-trunc,
  float=\floatposh]
  1.9.trunc;
  [00000000] 1
\end{lstlisting}

\subsubsection{asFloat}

Return the target (\autoref{lst:float-asFloat}).

\begin{lstlisting}[caption=Float.asFloat, label=lst:float-asFloat,
  float=\floatposh]
  51.asFloat;
  [00000000] 51
\end{lstlisting}

\subsubsection{sqr}

Return the square of the target (\autoref{lst:float-sqr}).

\begin{lstlisting}[caption=Float.sqr, label=lst:float-sqr,
  float=\floatposh]
  32.sqr;
  [00000000] 1024
\end{lstlisting}

\subsubsection{sgn}

Return 1 if the target is positive, -1 otherwise (\autoref{lst:float-sgn}).

\begin{lstlisting}[caption=Float.sgn, label=lst:float-sgn,
  float=\floatposh]
  (-1164).sgn;
  [00000000] -1
  (1164).sgn;
  [00000000] 1
\end{lstlisting}

\subsubsection{times}

Take one functional argument and call it target times (\autoref{lst:float-times}).

\begin{lstlisting}[caption=Float.times, label=lst:float-times,
  float=\floatposh]
  5.times(function (i) { echo("ping")});
  [00000000] *** ping
  [00000000] *** ping
  [00000000] *** ping
  [00000000] *** ping
  [00000000] *** ping
\end{lstlisting}

\subsubsection{each}

Take one functional argument and call it on every integer from 0 to
target - 1 (\autoref{lst:float-each}).

\begin{lstlisting}[caption=Float.each, label=lst:float-each,
  float=\floatposh]
  5.each(function (i) { echo(i * 2)});
  [00000000] *** 0
  [00000000] *** 2
  [00000000] *** 4
  [00000000] *** 6
  [00000000] *** 8
\end{lstlisting}

\subsubsection{\^{}}

Return the bitwise exclusive or between the target and the first argument.

\begin{lstlisting}[caption=Float.'\^{}', label=lst:float-bitxor,
  float=\floatposh]
  3 ^ 6;
  [00000000] 5
\end{lstlisting}

\subsubsection{$>>$}

\lstinline|a >> b| return the \lstinline|a| shifted by \lstinline|b|
bit towards the right (\autoref{lst:float-rshift}).

\begin{lstlisting}[caption=Float.'$>>$', label=lst:float-rshift,
  float=\floatposh]
  4 >> 2;
  [00000000] 1
\end{lstlisting}

\subsubsection{$<$}

\lstinline|a < b| returns whether \lstinline|a| is inferior to
\lstinline|b| (\autoref{lst:float-LT}). Note that other comparison operators
(\lstinline|<=|, \lstinline|>|, \ldots) can thus also be applied on
floats since Float inherits Orderable (see \autoref{sec:std-orderable}).

\begin{lstlisting}[caption=Float.'$<$', label=lst:float-LT,
  float=\floatposh]
  0 < 1;
  [00000000] true
  1 < 0;
  [00000000] false
\end{lstlisting}

\subsubsection{$<<$}

\lstinline|a << b| return the \lstinline|a| shifted by \lstinline|b|
bit towards the left (\autoref{lst:float-lshift}).

\begin{lstlisting}[caption=Float.'$<<$', label=lst:float-lshift,
  float=\floatposh]
  4 << 2;
  [00000000] 16
\end{lstlisting}

\subsubsection{-}

\lstinline|a - b| returns \lstinline|a| minus \lstinline|b| (\autoref{lst:float-minus}).

\begin{lstlisting}[caption=Float.'-', label=lst:float-minus,
  float=\floatposh]
  6 - 3;
  [00000000] 3
\end{lstlisting}

\subsubsection{+}

\lstinline|a + b| returns \lstinline|a| plus \lstinline|b| (\autoref{lst:float-plus}).

\begin{lstlisting}[caption=Float.'+', label=lst:float-plus, float=\floatposh]
  1 + 1;
  [00000000] 2
\end{lstlisting}

\subsubsection{/}

\lstinline|a / b| returns the quotient of \lstinline|a| divided by
\lstinline|b| (\autoref{lst:float-div}).

\begin{lstlisting}[caption=Float.'/', label=lst:float-div, float=\floatposh]
  50 / 11;
  [00000000] 4
\end{lstlisting}

\subsubsection{\%}

\lstinline|a \% b| returns \lstinline|a| modulo \lstinline|b| (\autoref{lst:float-mod}).

\begin{lstlisting}[caption=Float.'\%', label=lst:float-mod, float=\floatposh]
  50 % 11;
  [00000000] 2
\end{lstlisting}

\subsubsection{*}

\lstinline|a * b| returns \lstinline|a| times \lstinline|b| (\autoref{lst:float-STAR}).

\begin{lstlisting}[caption=Float.'*', label=lst:float-STAR, float=\floatposh]
  2 * 3;
  [00000000] 6
\end{lstlisting}

\subsubsection{**}

\lstinline|a ** b| returns \lstinline|a| to the \lstinline|b| power (\autoref{lst:float-STAR-STAR}).

\begin{lstlisting}[caption=Float.'**', label=lst:float-STAR-STAR, float=\floatposh]
  2 ** 10;
  [00000000] 1024
\end{lstlisting}

\subsubsection{==}

\lstinline|a == b| returns whether \lstinline|a| equals \lstinline|b| (\autoref{lst:float-eq}).

\begin{lstlisting}[caption={Float.'=='}, label=lst:float-eq, float=\floatposh]
  1 == 1;
  [00000000] true
  1 == 2;
  [00000000] false
\end{lstlisting}
