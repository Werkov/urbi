\section{Float}

Float is an \us primitive to represent floating point number.

\subsection{Prototypes}

\begin{itemize}
\item \refObject{Comparable}
\item \refObject{Orderable}
\end{itemize}

\subsection{Construction}

The most common way to create fresh floats is by using the literal
syntax presented in \autoref{sec:us-syn-lit-float}. A null float can also be
obtained with \lstinline|Float|'s \lstinline|new| method
(\autoref{lst:new-float}).

\begin{urbiscript}[caption=Creating a new float, label=lst:new-float,
  float=\floatpos]
Float.new;
[00000000] 0
\end{urbiscript}

\subsection{Methods}

\subsubsection{Float.abs}

Return the absolute value of the target (\autoref{lst:float-abs}).

\begin{urbiscript}[caption=Float.abs, label=lst:float-abs]
(-5).abs;
[00000000] 5
(5).abs;
[00000000] 5
\end{urbiscript}

\subsubsection{Float.acos}

Return the arccosine of the target.

\begin{urbiscript}[caption=Float.acos, label=lst:float-acos]
1.acos;
[00000000] 0
\end{urbiscript}

\subsubsection{Float.asin}

Return the arcsine of the target.

\begin{urbiscript}[caption=Float.asin, label=lst:float-asin]
0.asin;
[00000000] 0
\end{urbiscript}

\subsubsection{Float.asString}

Return a string representing the target (\autoref{lst:float-asString}).

\begin{urbiscript}[caption=Float.asString, label=lst:float-asString]
42.asString;
[00000000] "42"
\end{urbiscript}

\subsubsection{Float.atan}

Return the arctangent of the target (\autoref{lst:float-atan}).

\begin{urbiscript}[caption=Float.atan, label=lst:float-atan]
0.atan;
[00000000] 0
\end{urbiscript}

\subsubsection{Float.bitand}

Return the bitwise and between the target and the argument (\autoref{lst:float-bitand}).

\begin{urbiscript}[caption=Float.bitand, label=lst:float-bitand]
3 bitand 6;
[00000000] 2
\end{urbiscript}

\subsubsection{Float.bitor}

Return the bitwise or between the target and the argument
(\autoref{lst:float-bitor}).

\begin{urbiscript}[caption=Float.bitor, label=lst:float-bitor]
3 bitor 6;
[00000000] 7
\end{urbiscript}

\subsubsection{Float.clone}

Return a fresh Float with the same value as the target
(\autoref{lst:float-clone}).

\begin{urbiscript}[caption=Float.clone, label=lst:float-clone]
var x = 0;
[00000000] 0
var y = x.clone;
[00000000] 0
x === y;
[00000000] false
\end{urbiscript}

\subsubsection{Float.compl}

Return the 1-complement of the target (\autoref{lst:float-compl}).

\begin{urbiscript}[caption=Float.compl, label=lst:float-compl]
compl 0;
[00000000] 4294967295
compl 4294967295;
[00000000] 0
\end{urbiscript}

\subsubsection{Float.cos}

Return the cosine of the target (\autoref{lst:float-cos}).

\begin{urbiscript}[caption=Float.cos, label=lst:float-cos]
0.cos;
[00000000] 1
\end{urbiscript}

\subsubsection{Float.exp}

Return the exponential of the target (\autoref{lst:float-exp}).

\begin{urbiscript}[caption=Float.exp, label=lst:float-exp]
1.exp;
[00000000] 2.71828
\end{urbiscript}

\subsubsection{Float.inf}

Return the infinity (\autoref{lst:float-inf}).

\begin{urbiscript}[caption=Float.inf, label=lst:float-inf]
Float.inf;
[00000000] inf
\end{urbiscript}

\subsubsection{Float.log}

Return the logarithm of the target (\autoref{lst:float-log}).

\begin{urbiscript}[caption=Float.log, label=lst:float-log]
2.71828.log;
[00000000] 0.999999
\end{urbiscript}

\subsubsection{Float.nan}

Return the ``not a number'' special float value (\autoref{lst:float-nan}).

\begin{urbiscript}[caption=Float.nan, label=lst:float-nan]
Float.nan;
[00000000] nan
\end{urbiscript}

\subsubsection{Float.random}

Return a random number between 0 and the target(\autoref{lst:float-random}).

\begin{urbiscript}[caption=Float.random, label=lst:float-random]
5.random;
[00000000] 3
5.random;
[00000000] 1
5.random;
[00000000] 2
\end{urbiscript}

\subsubsection{Float.round}

Return the target, rounded (\autoref{lst:float-round}).

\begin{urbiscript}[caption=Float.round, label=lst:float-round]
1.6.round;
[00000000] 2
1.4.round;
[00000000] 1
\end{urbiscript}

\subsubsection{Float.seq}

Return the sequence of integer from 0 to target - 1 as a list (\autoref{lst:float-seq}).

\begin{urbiscript}[caption=Float.seq, label=lst:float-seq]
3.seq;
[00000000] [0, 1, 2]
\end{urbiscript}

\subsubsection{Float.sin}

Return the sinus of the target (\autoref{lst:float-sin}).

\begin{urbiscript}[caption=Float.sin, label=lst:float-sin]
0.sin;
[00000000] 0
\end{urbiscript}

\subsubsection{Float.sqrt}

Return the square root of the target (\autoref{lst:float-sqrt}).

\begin{urbiscript}[caption=Float.sqrt, label=lst:float-sqrt]
1024.sqrt;
[00000000] 32
\end{urbiscript}

\subsubsection{Float.tan}

Return the tangent of the target (\autoref{lst:float-tan}).

\begin{urbiscript}[caption=Float.tan, label=lst:float-tan]
0.tan;
[00000000] 0
\end{urbiscript}

\subsubsection{Float.trunc}

Return the target truncated  (\autoref{lst:float-trunc}).

\begin{urbiscript}[caption=Float.trunc, label=lst:float-trunc]
1.9.trunc;
[00000000] 1
\end{urbiscript}

\subsubsection{Float.asFloat}

Return the target (\autoref{lst:float-asFloat}).

\begin{urbiscript}[caption=Float.asFloat, label=lst:float-asFloat]
51.asFloat;
[00000000] 51
\end{urbiscript}

\subsubsection{Float.sqr}

Return the square of the target (\autoref{lst:float-sqr}).

\begin{urbiscript}[caption=Float.sqr, label=lst:float-sqr]
32.sqr;
[00000000] 1024
\end{urbiscript}

\subsubsection{Float.sgn}

Return 1 if the target is positive, -1 otherwise (\autoref{lst:float-sgn}).

\begin{urbiscript}[caption=Float.sgn, label=lst:float-sgn]
(-1164).sgn;
[00000000] -1
(1164).sgn;
[00000000] 1
\end{urbiscript}

\subsubsection{Float.times}

Take one functional argument and call it target times
(\autoref{lst:float-times}).

\begin{urbiscript}[caption=Float.times, label=lst:float-times]
5.times(function () { echo("ping")});
[00000000] *** ping
[00000000] *** ping
[00000000] *** ping
[00000000] *** ping
[00000000] *** ping
\end{urbiscript}

\subsubsection{Float.each}

Take one functional argument and call it on every integer from 0 to
target - 1 (\autoref{lst:float-each}).

\begin{urbiscript}[caption=Float.each, label=lst:float-each]
5.each(function (i) { echo(i * 2)});
[00000000] *** 0
[00000000] *** 2
[00000000] *** 4
[00000000] *** 6
[00000000] *** 8
\end{urbiscript}

\subsubsection{Float.\^{}}

Return the bitwise exclusive or between the target and the first argument.

\begin{urbiscript}[caption=Float.'\^{}', label=lst:float-bitxor]
3 ^ 6;
[00000000] 5
\end{urbiscript}

\subsubsection{Float.\textgreater\textgreater}

\lstinline|a >> b| return the \lstinline|a| shifted by \lstinline|b|
bit towards the right (\autoref{lst:float-rshift}).

\begin{urbiscript}[caption=Float.'$>>$', label=lst:float-rshift]
4 >> 2;
[00000000] 1
\end{urbiscript}

\subsubsection{Float.\textless}

\lstinline|a < b| returns whether \lstinline|a| is inferior to
\lstinline|b| (\autoref{lst:float-LT}). Note that other comparison operators
(\lstinline|<=|, \lstinline|>|, \ldots) can thus also be applied on
floats since Float inherits \refObject{Orderable}.

\begin{urbiscript}[caption=Float.'$<$', label=lst:float-LT]
0 < 1;
[00000000] true
1 < 0;
[00000000] false
\end{urbiscript}

\subsubsection{Float.\textless\textless}

\lstinline|a << b| return the \lstinline|a| shifted by \lstinline|b|
bit towards the left (\autoref{lst:float-lshift}).

\begin{urbiscript}[caption=Float.'$<<$', label=lst:float-lshift]
4 << 2;
[00000000] 16
\end{urbiscript}

\subsubsection{Float.$-$}

\lstinline|a - b| returns \lstinline|a| minus \lstinline|b|
(\autoref{lst:float-minus}).

\begin{urbiscript}[caption=Float.'-', label=lst:float-minus]
6 - 3;
[00000000] 3
\end{urbiscript}

\subsubsection{Float.+}

\lstinline|a + b| returns \lstinline|a| plus \lstinline|b|
(\autoref{lst:float-plus}).

\begin{urbiscript}[caption=Float.'+', label=lst:float-plus]
1 + 1;
[00000000] 2
\end{urbiscript}

\subsubsection{Float./}

\lstinline|a / b| returns the quotient of \lstinline|a| divided by
\lstinline|b| (\autoref{lst:float-div}).

\begin{urbiscript}[caption=Float.'/', label=lst:float-div]
50 / 10;
[00000000] 5
\end{urbiscript}

\subsubsection{Float.\%}

\lstinline|a % b|
returns \lstinline|a| modulo \lstinline|b| (\autoref{lst:float-mod}).

\begin{urbiscript}[caption=Float.'Percent', label=lst:float-mod]
50 % 11;
[00000000] 6
\end{urbiscript}

\subsubsection{Float.*}

\lstinline|a * b| returns the product of \lstinline|a| by
\lstinline|b| (\autoref{lst:float-STAR}).

\begin{urbiscript}[caption=Float.'*', label=lst:float-STAR]
2 * 3;
[00000000] 6
\end{urbiscript}

\subsubsection{Float.**}

\lstinline|a ** b| returns \lstinline|a| to the \lstinline|b| power
(\autoref{lst:float-STAR-STAR}).

\begin{urbiscript}[caption=Float.'**', label=lst:float-STAR-STAR]
2 ** 10;
[00000000] 1024
\end{urbiscript}

\subsubsection{Float.==}

\lstinline|a == b| returns whether \lstinline|a| equals \lstinline|b|
(\autoref{lst:float-eq}).

\begin{urbiscript}[caption={Float.'=='}, label=lst:float-eq]
1 == 1;
[00000000] true
1 == 2;
[00000000] false
\end{urbiscript}

%%% Local Variables:
%%% mode: latex
%%% TeX-master: "../urbi-sdk"
%%% End:

% LocalWords:  Orderable lst acos arccosine asin arcsine asString atan bitand
% LocalWords:  arctangent bitwise bitor compl sqrt trunc asFloat sqr sgn rshift
% LocalWords:  lshift eq
