%% Copyright (C) 2009-2011, Gostai S.A.S.
%%
%% This software is provided "as is" without warranty of any kind,
%% either expressed or implied, including but not limited to the
%% implied warranties of fitness for a particular purpose.
%%
%% See the LICENSE file for more information.

\section{nil}

The special entity \lstinline|nil| is an object used to denote an empty
value.  Contrary to \refObject{void}, it is a regular value which can be
read.

\subsection{Prototypes}

\begin{refObjects}
\item[Singleton]
\end{refObjects}

\subsection{Construction}

Being a singleton, \lstinline|nil| is not to be constructed, just used.

\begin{urbiassert}[firstnumber=1]
nil == nil;
\end{urbiassert}

\subsection{Slots}

\begin{urbiscriptapi}
\item[isNil] Whether \this is \refObject{nil}.  I.e., true.  See also
  \refSlot[Object]{isNil}.
\begin{urbiassert}
nil.isNil;
!Object.isNil;  !42.isNil;  !(function () { nil }.isNil);
\end{urbiassert}


\item[isVoid] In order to facilitate the transition from older code to
  newer, return true.  In the future, false will be returned.  Therefore, if
  you really need to check whether \var{foo} is \lstinline|void| but not
  \lstinline|nil|, use
  \lstinline|!\var{foo}.acceptVoid.isNil && \var{foo}.isVoid|.
\begin{urbiassert}
nil.isVoid;
[     Logger     ] nil.isVoid will return false eventually, adjust your code.
[     Logger     ]     For instance replace InputStream loops from
[     Logger     ]       while (!(x = i.get.acceptVoid).isVoid)
[     Logger     ]         cout << x;
[     Logger     ]     to
[     Logger     ]       while (!(x = i.get).isNil)
[     Logger     ]         cout << x;
\end{urbiassert}
\end{urbiscriptapi}

%%% Local Variables:
%%% coding: utf-8
%%% mode: latex
%%% TeX-master: "../urbi-sdk"
%%% ispell-dictionary: "american"
%%% ispell-personal-dictionary: "../urbi.dict"
%%% fill-column: 76
%%% End:
