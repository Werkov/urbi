%% Copyright (C) 2009-2011, Gostai S.A.S.
%%
%% This software is provided "as is" without warranty of any kind,
%% either expressed or implied, including but not limited to the
%% implied warranties of fitness for a particular purpose.
%%
%% See the LICENSE file for more information.

\section{Tuple}

A \dfn{tuple} is a container storing a fixed number of objects.  Examples
include \refObject{Pair} and \refObject{Triplet}.

\subsection{Prototypes}
\begin{refObjects}
\item[Object]
\end{refObjects}

\subsection{Construction}

The \lstinline|Tuple| object is not meant to be instantiated, its main
purpose is to share code for its descendants, such as \refObject{Pair}.
Yet it accepts its members as a list.

\begin{urbiscript}[firstnumber=1]
var t = Tuple.new([1, 2, 3]);
[00000000] (1, 2, 3)
\end{urbiscript}

The output generated for a \lstinline|Tuple| can also be used to create a
\lstinline|Tuple|.  Expressions are put inside parentheses and separated by
commas.  One extra comma is allowed after the last element.  To avoid
confusion between a 1 member \lstinline|Tuple| and a parenthesized
expression, the extra comma must be added.  \lstinline|Tuple| with no
expressions are also accepted.

\begin{urbiscript}
// not a Tuple
(1);
[00000000] 1

// Tuples
();
[00000000] ()
(1,);
[00000000] (1,)
(1, 2);
[00000000] (1, 2)
(1, 2, 3, 4,);
[00000000] (1, 2, 3, 4)
\end{urbiscript}


\subsection{Slots}
\begin{urbiscriptapi}
\item[asString]%
  The string \samp{(\var{first}, \var{second}, ..., \var{last})}, using
  \code{asPrintable} to convert members to strings.

\begin{urbiassert}
           ().asString == "()";
         (1,).asString == "(1,)";
       (1, 2).asString == "(1, 2)";
(1, 2, 3, 4,).asString == "(1, 2, 3, 4)";
\end{urbiassert}


\item[matchAgainst](<handler>, <pattern>)%
  Pattern matching on members.  See \refObject{Pattern}.
\begin{urbiscript}
{
  // Match a tuple.
  (var first, var second) = (1, 2);
  assert { first == 1; second == 2 };
};
\end{urbiscript}

\item[size] Number of members.
\begin{urbiassert}
           ().size == 0;
         (1,).size == 1;
 (1, 2, 3, 4).size == 4;
(1, 2, 3, 4,).size == 4;
\end{urbiassert}

\item|'[]'|(<index>)%
  Return the \var{index}-th element.  \var{index} must be in bounds.
\begin{urbiassert}
(1, 2, 3)[0] == 1;
(1, 2, 3)[1] == 2;
\end{urbiassert}

\item|'[]='|(<index>, <value>)%
  Set (and return) the \var{index}-th element to \var{value}.
  \var{index} must be in bounds.
\begin{urbiscript}
{
  var t = (1, 2, 3);
  assert
  {
    (t[0] = 2) == 2;
    t == (2, 2, 3);
  };
};
\end{urbiscript}

\item['<'](<other>)%
  Lexicographic comparison between two tuples.
\begin{urbiassert}
(0, 0) < (0, 1);
(0, 0) < (1, 0);
(0, 1) < (1, 0);

(1, 2, 3) < (2,);
\end{urbiassert}

\item['=='](<other>)%
  Whether \this and \lstinline|other| have the same contents
  (equality-wise).
\begin{urbiassert}
  (1, 2) == (1, 2);
!((1, 1) == (2, 2));
\end{urbiassert}

\item['*'](<value>)%
  Return a \lstinline|Tuple| in which all elements of \this are multiplied
  by a \var{value}.

\begin{urbiassert}
(0, 1, 2, 3) * 3 == (0, 3, 6, 9);
(1, "foo") * 2 == (2, "foofoo");
\end{urbiassert}

\item['+'](<other>)%
  Return a \lstinline|Tuple| in which each element of \this is summed with
  its corresponding element in the \var{other} \lstinline|Tuple|.

\begin{urbiassert}
(0, 1) + (2, 3) == (2, 4);
(1, "foo") + (2, "bar") == (3, "foobar");
\end{urbiassert}


\end{urbiscriptapi}



%%% Local Variables:
%%% mode: latex
%%% TeX-master: "../urbi-sdk"
%%% ispell-dictionary: "american"
%%% ispell-personal-dictionary: "../urbi.dict"
%%% fill-column: 76
%%% End:
