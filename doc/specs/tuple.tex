\section{Tuple}

A \dfn{tuple} is a container storing a fixed number of objects.
Examples include \refObject{Pair} and \refObject{Triplet}.

\subsection{Prototype}
\begin{itemize}
\item \refObject{Object}
\end{itemize}

\subsection{Construction}

The \lstinline|Tuple| object is not meant to be instantiated, its main
purpose is to share code for its descendants, such as \refObject{Pair}.
Yet it accepts its members as a list.

\begin{urbiscript}
var t = Tuple.new([1, 2, 3]);
[00000002] (1, 2, 3)
\end{urbiscript}

\subsection{Slots}
\begin{itemize}
\item \lstinline|asString|\\
  Generate the string \samp{(\var{first}, \var{second}, ...)} using
  \code{asPrintable} to convert members to strings.

\item \lstinline|'[]'(\var{index})|\\
  Return the \var{index}-th element.  \var{index} must be in bounds.
\begin{urbiassert}[firstnumber=last]
Tuple.new([1, 2, 3])[0] == 1;
Tuple.new([1, 2, 3])[1] == 2;
\end{urbiassert}

\item \lstinline|'[]='(\var{index}, \var{value})|\\
  Set (and return) the \var{index}-th element to \var{value}.
  \var{index} must be in bounds.
\begin{urbiscript}[firstnumber=last]
{
  var t = Tuple.new([1, 2, 3]);
  assert
  {
    (t[0] = 2) == 2;
    t == Tuple.new([2, 2, 3]);
  };
};
\end{urbiscript}

\item \lstinline|'<'(\var{other})|\\
  Lexicographic comparison between two tuples.
\begin{urbiassert}[firstnumber=last]
Tuple.new([0, 0]) < Tuple.new([0, 1]);
Tuple.new([0, 0]) < Tuple.new([1, 0]);
Tuple.new([0, 1]) < Tuple.new([1, 0]);
\end{urbiassert}

\item \lstinline|'=='(\var{other})|\\
  Whether \lstinline|this| and \lstinline|other| have the same
  contents (equality-wise).
\begin{urbiassert}[firstnumber=last]
  Tuple.new([1, 2]) == Tuple.new([1, 2]);
!(Tuple.new([1, 1]) == Tuple.new([2, 2]));
\end{urbiassert}
\end{itemize}



%%% Local Variables:
%%% mode: latex
%%% TeX-master: "../urbi-sdk"
%%% ispell-personal-dictionary: "../urbi.dict"
%%% End:
