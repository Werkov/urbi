\section{Math}

This object is actually meant to play the role of a name-space in
which the mathematical functions are defined with a more conventional
notation.  Indeed, in an object-oriented language, writing
\lstinline|pi.cos| makes perfect sense, yet \lstinline|cos(pi)| is
more usual.

\subsection{Prototypes}
\begin{itemize}
\item \refObject{Singleton}
\end{itemize}

\subsection{Construction}

Since it is a \refObject{Singleton}, you are not expected to build
other instances.

\subsection{Slots}

\begin{itemize}
\item \lstinline|abs(\var{float})|\\
  Bounce to \lstinline|\var{float}.abs|.
\begin{urbiassert}
Math.abs(1) == 1;
Math.abs(-1) == 1;
Math.abs(0) == 0;
Math.abs(3.5) == 3.5;
\end{urbiassert}
\item \lstinline|acos(\var{float})|\\
  Bounce to \lstinline|\var{float}.acos|.

\item \lstinline|asin(\var{float})|\\
  Bounce to \lstinline|\var{float}.asin|.

\item \lstinline|atan(\var{float})|\\
  Bounce to \lstinline|\var{float}.atan|.
\begin{urbiassert}
Math.atan(1) ~= pi/4;
\end{urbiassert}

\item \lstinline|atan2(\var{x}, \var{y})|\\
  Bounce to \lstinline|\var{x}.atan2(\var{y})|.
\begin{urbiassert}
Math.atan2(2, 2) ~= pi/4;
Math.atan2(-2, 2) ~= -pi/4;
\end{urbiassert}

\item \lstinline|cos(\var{float})|\\
  Bounce to \lstinline|\var{float}.cos|.
\begin{urbiassert}
Math.cos(0) == 1;
Math.cos(pi) ~= -1;
\end{urbiassert}

\item \lstinline|exp(\var{float})|\\
  Bounce to \lstinline|\var{float}.exp|.

\item \lstinline|inf|\\
  Bounce to \lstinline|Float.inf|.

\item \lstinline|log(\var{float})|\\
  Bounce to \lstinline|\var{float}.log|.
\begin{urbiassert}
Math.log(1) == 0;
\end{urbiassert}

\item \lstinline|max(\var{arg1}, ...)|\\
  Bounce to \lstinline|[\var{arg1}, ...].max|, see \refObject{List}.
\begin{urbiassert}
max( 100,   20,   3 ) == 100;
max("100", "20", "3") == "3";
\end{urbiassert}

\item \lstinline|min(\var{arg1}, ...)|\\
  Bounce to \lstinline|[\var{arg1}, ...].min|, see \refObject{List}.
\begin{urbiassert}
min( 100,   20,   3 ) ==     3;
min("100", "20", "3") == "100";
\end{urbiassert}

\item \lstinline|nan|\\
  Bounce to \lstinline|Float.nan|.

\item \lstinline|pi|\\
  Bounce to \lstinline|Float.pi|.

\item \lstinline|random(\var{float})|\\
  Bounce to \lstinline|\var{float}.random|.

\item \lstinline|round(\var{float})|\\
  Bounce to \lstinline|\var{float}.round|.
\begin{urbiassert}
Math.round(1) == 1;
Math.round(1.1) == 1;
Math.round(1.49) == 1;
Math.round(1.5) == 2;
Math.round(1.51) == 2;
\end{urbiassert}

\item \lstinline|sign(\var{float})|\\
  Bounce to \lstinline|\var{float}.sign|.
\begin{urbiassert}
Math.sign(2)  == 1;
Math.sign(-2) == -1;
Math.sign(0)  == 0;
\end{urbiassert}

\item \lstinline|sin(\var{float})|\\
  Bounce to \lstinline|\var{float}.sin|.
\begin{urbiassert}
Math.sin(0) == 0;
Math.sin(pi) ~= 0;
\end{urbiassert}

\item \lstinline|sqr(\var{float})|\\
  Bounce to \lstinline|\var{float}.sqr|.
\begin{urbiassert}
Math.sqr(2.2) ~= 4.84;
\end{urbiassert}

\item \lstinline|sqrt(\var{float})|\\
  Bounce to \lstinline|\var{float}.sqrt|.
\begin{urbiassert}
Math.sqrt(4) == 2;
\end{urbiassert}

\item \lstinline|tan(\var{float})|\\
  Bounce to \lstinline|\var{float}.tan|.
\begin{urbiassert}
Math.tan(pi/4) ~= 1;
\end{urbiassert}

\item \lstinline|trunc(\var{float})|\\
  Bounce to \lstinline|\var{float}.trunc|.
\begin{urbiassert}
Math.trunc(1) == 1;
Math.trunc(1.1) == 1;
Math.trunc(1.49) == 1;
Math.trunc(1.5) == 1;
Math.trunc(1.51) == 1;
\end{urbiassert}
\end{itemize}


%%% Local Variables:
%%% mode: latex
%%% TeX-master: "../urbi-sdk"
%%% ispell-dictionary: "american"
%%% ispell-personal-dictionary: "../urbi.dict"
%%% End:
