%% Copyright (C) 2009-2010, Gostai S.A.S.
%%
%% This software is provided "as is" without warranty of any kind,
%% either expressed or implied, including but not limited to the
%% implied warranties of fitness for a particular purpose.
%%
%% See the LICENSE file for more information.

\section{Pattern}

\lstinline|Pattern| class is used to make correspondences between a pattern
and another \lstinline|Object|.  The visit is done either on the pattern or
on the element against which the pattern is compared.

\lstinline|Pattern|s are used for the implementation of the pattern matching.
So any class made compatible with the pattern matching implemented by this
class will allow you to use it implicitly in your scripts.

\begin{urbiscript}[firstnumber=1]
[1, var a, var b] = [1, 2, 3];
[00000000] [1, 2, 3]
a;
[00000000] 2
b;
[00000000] 3
\end{urbiscript}

\subsection{Prototypes}

\begin{refObjects}
\item[Object]
\end{refObjects}

\subsection{Construction}

A \lstinline|Pattern| can be created with any object that can be matched.

\begin{urbiscript}
Pattern.new([1]); // create a pattern to match the list [1].
[00000000] Pattern_0x189ea80
Pattern.new(Pattern.Binding.new("a")); // match anything into "a".
[00000000] Pattern_0x18d98b0
\end{urbiscript}

\subsection{Slots}

\begin{urbiscriptapi}
\item[Binding]
  A class used to create pattern variables.

\begin{urbiscript}
Pattern.Binding.new("a");
[00000000] var a
\end{urbiscript}


\item[bindings]

  A \lstinline|Dictionary| filled by the match function for each
  \refSlot{Binding} contained inside the pattern.

\begin{urbiscript}
{
  var p = Pattern.new([Pattern.Binding.new("a"), Pattern.Binding.new("b")]);
  assert (p.match([1, 2]));
  p.bindings
};
[00000000] ["a" => 1, "b" => 2]
\end{urbiscript}


\item[match](<value>)%

  Use \var{value} to unify the current pattern with this value.
  Return the status of the match.
  \begin{itemize}
    \item If the match is correct, then the \var{bindings} member will
      contain the result of every matched values.
    \item If the match is incorrect, then the \var{bindings} member should
      not be used.
  \end{itemize}
  If the pattern contains multiple \refSlot{Binding} with the same name,
  then the behavior is undefined.

\begin{urbiassert}
Pattern.new(1).match(1);
Pattern.new([1, 2]).match([1, 2]);
! Pattern.new([1, 2]).match([1, 3]);
! Pattern.new([1, 2]).match([1, 2, 3]);
Pattern.new(Pattern.Binding.new("a")).match(0);
Pattern.new([1, Pattern.Binding.new("a")]).match([1, 2]);
! Pattern.new([1, Pattern.Binding.new("a")]).match(0);
\end{urbiassert}


\item[matchPattern](<pattern>, <value>)%

  This function is used as a callback function to store all bindings
  in the same place.  This function is useful inside objects that
  implement a \lstinline|match| or \lstinline|matchAgainst| function
  that need to continue the match deeper.  Return the status of the
  match (a Boolean).

  The \var{pattern} should provide a method
  \lstinline|match(\var{handler},\var{value})| otherwise the value method
  \lstinline|matchAgainst(\var{handler}, \var{pattern})| is used.  If none
  are provided the \lstinline|'=='| operator is used.

  To see how to use it, you can have a look at the implementation of
  \refSlot[List]{matchAgainst}.

%% This function is indirectly tested with the match of Pattern.Binding
%% inside lists.
\item[pattern]
  The pattern given at the creation.
\begin{urbiassert}
Pattern.new(1).pattern == 1;
Pattern.new([1, 2]).pattern == [1, 2];
{
  var pattern = [1, Pattern.Binding.new("a")];
  Pattern.new(pattern).pattern === pattern
};
\end{urbiassert}

\end{urbiscriptapi}

%%% Local Variables:
%%% mode: latex
%%% TeX-master: "../urbi-sdk"
%%% ispell-dictionary: "american"
%%% ispell-personal-dictionary: "../urbi.dict"
%%% fill-column: 76
%%% End:
