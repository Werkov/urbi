%% Copyright (C) 2008-2011, Gostai S.A.S.
%%
%% This software is provided "as is" without warranty of any kind,
%% either expressed or implied, including but not limited to the
%% implied warranties of fitness for a particular purpose.
%%
%% See the LICENSE file for more information.

\newcommand{\optionDebug}
  {Set the verbosity level of traces.  See the \env{GD\_LEVEL} documentation
   (\autoref{sec:tools:env}).}

\newcommand{\optionHelp}
  {Display the help message and exit successfully.}

\newcommand{\optionVersion}
  {Display version information and exit successfully.}

\chapter{Programs}
\label{sec:tools}

\section{Environment Variables}
\label{sec:tools:envvars}

There is a number of environment variables that alter the behavior of
the \urbi tools.

\subsection{Search Path Variables}

Some variables define \dfn[search-path]{search-paths}, i.e.,
colon-separated lists of directories in which library files (\us
programs, UObjects and so forth) are looked for.

The tools have predefined values for these variables which are
tailored for your installation --- so that \urbi tools can be run
without any special adjustment.  In order to provide the user with a
means to override or extend these built-in values, the path variables
support a special syntax: a lone colon specifies where the standard
search path must be inserted.  See the following examples about
\env{URBI\_PATH}.

\begin{shell}
# Completely override the system path.  First look for files in
# /home/jessie/urbi, then in /usr/local/urbi.
export URBI_PATH=/home/jessie/urbi:/usr/local/urbi

# Prepend the previous path to the default path.  This is dangerous as
# it may result in some standard files being hidden.
export URBI_PATH=/home/jessie/urbi:/usr/local/urbi:

# First look in Jessie's directory, then the default location, and
# finally in /usr/local/urbi.
export URBI_PATH=/home/jessie/urbi::/usr/local/urbi

# Extend the default path, i.e., files that are not found in the
# default path will be looked for in Jessie's place, and then in
# /usr/local/urbi
export URBI_PATH=:/home/jessie/urbi:/usr/local/urbi
\end{shell}

\begin{windows}
  On Windows too directories are separated by colons, but backslashes
  are used instead of forward-slashes.  For instance
\begin{shell}
URBI_PATH=C:\cygwin\home\jessie\urbi:C:\cygwin\usr\local\urbi
\end{shell}
\end{windows}

\subsection{Environment Variables}
\label{sec:tools:env}

The following variables control the way the log messages are displayed.
They are meant for developers.
\begin{envs}
\item[GD\_COLOR] If set, force the use of colors in the logs, even if the
  output device does not seem to support them.  See also \env{GD\_NO\_COLOR}.
\item[GD\_CATEGORY] If set, a comma-separated list of categories or globs
  (e.g., \code{UValue}, \code{Urbi.*}) with modifiers \code{+} or
  \code{-}. If the modifier is a \code{+} or if there is no modifier the
  category will be displayed. If the modifier is a \code{-} it will be
  hidden. Modifiers can be chained so we get the following behavior:
  \code{-*,+Urbi.*,-Urbi.At} will only display categories beginning with
  \code{Urbi.} except \code{Urbi.At}, \code{-Urbi.*} will display all
  categories except those beginning with \code{Urbi.}.
\item[GD\_ENABLE\_CATEGORY] If set, a comma-separated list of categories or
  globs (e.g., \code{UValue}, \code{Urbi.*}) to display.  All the others
  will be discarded.  See also \env{GD\_DISABLE\_CATEGORY}.
\item[GD\_DISABLE\_CATEGORY] If set, a comma-separated list of categories or
  globs (e.g., \code{UValue}, \code{Urbi.*}) \emph{not} to display.  See
  also \env{GD\_ENABLE\_CATEGORY}.
\item[GD\_LEVEL] Set the verbosity level of traces.  This environment
  variable is meant for the developers of \usdk, yet it is very useful
  when tracking problems such as a UObject that fails to load properly.
  Valid values are, in increasing verbosity order:
  \begin{enumerate}
  \item \code{NONE}, no log messages at all.
  \item \code{LOG}, the default value.
  \item \code{TRACE}
  \item \code{DEBUG}
  \item \code{DUMP}, maximum verbosity.
  \end{enumerate}

\item[GD\_LOC] If set, display the location (file, line, function name) from
  which the log message was issued.

\item[GD\_NO\_COLOR] If set, forbid the use of colors in the logs, even if
  the output device seems to support them, or if \env{GD\_COLOR} is set.

\item[GD\_PID] If set, display the PID (Process Identity).

\item[GD\_THREAD] If set, display the thread identity.

\item[GD\_TIME] If set, display timestamps.

\item[GD\_TIMESTAMP\_US] If set, display timestamps in microseconds.
\end{envs}

The following variables control more high-level features, typically to
override the default behavior.
\begin{envs}
\item[URBI\_PATH] The search-path for \us source files (i.e.,
  \file{*.u} files).

\item[URBI\_ROOT] The \usdk is relocatable: its components know the
  relative location of each other.  Yet they need to ``guess'' the
  \urbi root, i.e., the path to the directory that contains the files.
  This variable also to override that guess.  Do not use it unless you
  know exactly what you are doing.

\item[URBI\_TEXT\_MODE] If set in the environment of a remote urbi-launch,
  disable binary protocol and force using \us messages.

\item[URBI\_UOBJECT\_PATH] The search-path for UObjects files.
  This is used by \command{urbi-launch}, by
  \refSlot[System]{loadModule} and \refSlot[System]{loadLibrary}.

\item[URBI\_UVAR\_HOOK\_CHANGED] If set, set \refSlot[UVar]{hookChanged} to
  false.

\end{envs}

\section{Special Files}
\label{sec:tools:files}

\begin{files}
\item[CLIENT.INI] This is the obsolete name for \file{global.u}.

\item[global.u] If found in the \env{URBI\_PATH} (see
  \autoref{sec:tools:envvars}), this file is loaded by \urbi server upon
  start-up.  It is the appropriate place to install features you
  mean to provide to all the users of the server.  It is will be
  loaded via a special system connection, with its own private lobby.
  Therefore, purely local definitions will not be reachable from
  users; global modifications should be made in globally visible
  objects, say \refObject{Global}.

\item[local.u] If found in the \env{URBI\_PATH} (see
  \autoref{sec:tools:envvars}), this file is loaded by every
  connection established with an \urbi server.  This is the
  appropriate place for enhancements local to a lobby.

\item[URBI.INI] This is the obsolete name for \file{global.u}.
\end{files}

\section{\command{urbi} --- Running an Urbi Server}
\label{sec:tools:urbi}

The \command{urbi} program launches an \urbi server, for either batch,
interactive, or network-based executions.  It is subsumed by, but
simpler to use than, \command{urbi-launch}
(\autoref{sec:tools:urbi-launch}).

\subsection{Options}

\begin{options}[General Options]
\item[h]{help} \optionHelp
\item{version} \optionVersion
\end{options}

\begin{options}[Tuning]
\item[d]{debug=\var{level}} \optionDebug
\item[F]{fast}
  Ignore system time, go as fast as possible.  Do not use this option
  unless you know exactly what you are doing.

  The \option{--fast} flag makes the kernel run the program in
  ``simulated time'', as fast as possible. A \lstinline|sleep| in fast
  mode will not actually wait (from the wall-clock point of view), but
  the kernel will internally increase its simulated time.

  For instance, the following session behaves equally in fast and
  non-fast mode:

\begin{urbiscript}[firstnumber=1]
{ sleep(2s); echo("after") } & { sleep(1s); echo("before") };
[000000463] *** before
[000001463] *** after
\end{urbiscript}

  \noindent
  However, in non fast mode the execution will take two seconds (wall
  clock time), while it be instantaneous in fast mode. This option was
  designed for testing purpose; \emph{it does not preserve the program
    semantics}.

\item[s]{stack-size=\var{size}} Set the coroutine \dfn{stack size}.
  The unit of \var{size} is KB; it defaults to 128.

  This option should not be needed unless you have ``stack exhausted''
  messages from \command{urbi} in which case you should try
  \option{--stack-size=512} or more.

  Alternatively you can define the environment variable
  \env{URBI\_STACK\_SIZE}.  The option \option{--stack-size} has
  precedence over the \env{URBI\_STACK\_SIZE}.

\item[q]{quiet} Do not send the welcome banner to incoming clients.
\end{options}

\begin{options}[Networking]
\item[H]{host=\var{address}} Set the \var{address} on which network
  connections are listened to.  Typical values of \var{address}
  include:
  \begin{sublist}
    \begin{description}
    \item[localhost] only local connections are allowed (no other
      computer can reach this server).
    \item[127.0.0.1] same as \code{localhost}.
    \item[0.0.0.0] any IP v4 connection is allowed, including from
      remote computers.
  \end{description}
  \end{sublist}
  Defaults to \code{0.0.0.0}.
\item[P]{port=\var{port}} Set the port to listen incoming connections to.
  If \var{port} is \code{-1}, no networking.  If \var{port} is \code{0},
  then the system will chose any available port (see \option{--port-file}).
  Defaults to \code{-1}.
\item[w]{port-file=\var{file}} When the system is up and running, and when
  it is ready for network connections, create the file named \var{file}
  which contains the number of the port the server listens to.
\end{options}


\begin{options}[Execution]
\item[e]{expression=\var{exp}} Send the \us expression \var{exp}.
  No separator is added, you have to pass yours.
\item[f]{file=\var{file}} Send the contents of the file \var{file}.
  No separator is added, you have to pass yours.
\item[i]{interactive} Start an interactive session.

  Enabled if no input was provided (i.e., none of
  \option{-e}/\option{--expression}, \option{-f}/\option{--file},
  \option{-P}/\option{--port} was given).
\item[m]{module=\var{module}} Load the \UObject \var{module}.
\end{options}

The options \option{-e}, \option{-f} accumulate, and are run in the
same \refObject{Lobby} as \option{-i} if used.  In other words, the
following session is valid:

\begin{shell}[alsolanguage={[interactive]urbiscript}]
# Create a file "two.u".
$ echo "var two = 2;" >two.u
$ urbi -q -e 'var one = 1;' -f two.u -i
[00000000] 1
[00000000] 2
one + two;
[00000000] 3
\end{shell}%$


\subsection{Quitting}
\label{sec:tools:urbi:quitting}

To exit \command{urbi}, use the command \refSlot[System]{shutdown}.

If you are in interactive mode, you can also use the shortcut sequence
\key{C-d} in the server lobby. The command \refSlot[Lobby]{quit} does not
shutdown the server, it only closes the current lobby which closes the
connection while in remote mode (using \command{telnet} for example), but
only closes the interactive mode if performed on the server lobby.

On Unix systems, \command{urbi} handles the SIGINT signal (action performed
when \key{C-c} is pressed). If you are in interactive mode, a first
\key{C-c} kills the foreground job (or does nothing if no job is running on
foreground).

For example consider the following piece of code:

\begin{urbiunchecked}
every (4s)
  echo("Hello world!");
\end{urbiunchecked}

\noindent
If you try to execute this code, you will notice that you can no longer
execute commands, since the \lstinline{every} job is foreground (because of
the semicolon). Now if you press \key{C-c}, the \lstinline{every} job is
killed, all the pending commands you may have typed are cleared from the
execution queue, and you get back with a working interactive mode.

\begin{urbiunchecked}
every (1s) echo("Hello world!");
[00010181] *** Hello world!
[00011181] *** Hello world!
// This command is entered while the foreground job blocks the evaluation.
// It waits to be executed.
echo("done");
[00012181] *** Hello world!
[00013181] *** Hello world!

// Pressing Control-C here:
[00014100:error] !!! received interruption, killing foreground job.
// Note that the "echo" is not run, the command queue is flushed.

// New commands are honored.
echo("interrupted");
[00019709] *** interrupted
\end{urbiunchecked}

A second \key{C-c} (or SIGINT) within the 1.5s after the first one tries to
execute \refSlot[System]{shutdown}.  This can take a while, in particular if
you have remote UObjects, since a clean shutdown will first take care of
disconnecting them cleanly.

\begin{urbiunchecked}
// Two Control-C in a row.
[00493865:error] !!! received interruption, killing foreground job.
[00494672:error] !!! received interruption, shutting down.
\end{urbiunchecked}

Pressing \key{C-c} a third time triggers the default behavior: killing the
program in emergency, by-passing all the cleanups.

\medskip

In non-interactive mode (after a \refSlot[Lobby]{quit} on the server lobby
for example), the first \key{C-c} executes \refSlot[System]{shutdown}, and
the second one triggers the default behavior.


\section{\command{urbi-image} --- Querying Images from a Server}
\label{sec:tools:urbi-image}

\begin{shell}
urbi-image \var{option}...
\end{shell}

Connect to an \urbi server, and fetch images from it, for instance
from its camera.

\subsection{Options}

\begin{options}[General Options]
\item[h]{help} \optionHelp
\item{version} \optionVersion
\end{options}

\begin{options}[Networking]
\item[H]{host=\var{host}} Address to connect to.
\item[P]{port=\var{port}} Port to connect to.
\item{port-file=\var{file}} Connect to the port contained in the file
  \var{file}.
\end{options}

\begin{options}[Tuning]
\item[p]{period=\var{period}} Specify the period, in millisecond, at
  which images are queried.
\item[F]{format=\var{format}} Select format of the image (\samp{rgb},
  \samp{ycrcb}, \samp{jpeg}, \samp{ppm}).
\item[r]{reconstruct} Use reconstruct mode (for aibo).
\item[j]{jpeg=\var{factor}} JPEG compression factor (from 0 to 100,
  defaults to 70).
\item[d]{device=\var{device}} Query image on \var{device}.val
  (default: \code{camera}).
\item[o]{output=\var{file}} Query and save one image to \var{file}.
\item[R]{resolution=\var{resolution}} Select resolution of the image
  (0=biggest).
\item[s]{scale=\var{factor}} Rescale image with given \var{factor}
  (display only).
\end{options}


\section{\command{urbi-launch} --- Running a UObject}
\label{sec:tools:urbi-launch}

The \command{urbi-launch} program launches an \urbi system.  It is
more general than \command{urbi} (\autoref{sec:tools:urbi}):
everything \command{urbi} can do, \command{urbi-launch} can do it too.

\subsection{Invoking \command{urbi-launch}}

\command{urbi-launch} launches UObjects, either in plugged-in mode, or
in remote mode.  Since UObjects can also accept options, the command
line features two parts, separated by \samp{--}:

\begin{shell}
urbi-launch [\var{urbi-launch-option}...] \var{module}... [-- \var{module-option}...]
\end{shell}

The \var{module}s are looked for in the \env{URBI\_UOBJECT\_PATH}.  The
\var{module} extension (\samp{.so}, or \samp{.dll}) does not need to be
specified.

\begin{options}[Urbi-launch options]
\item[h]{help} \optionHelp
\item{version} \optionVersion
\item[c]{customize=\var{file}} Start the \urbi server in
  \var{file}.  This option is mostly for developers.
\item[d]{debug=\var{level}} \optionDebug
\end{options}

\begin{options}[Mode selection]
\item[p]{plugin} Attach the \var{module} onto a currently running \urbi
  server (identified by \var{host} and \var{port}).  This is equivalent to
  running \lstinline[style=varInString]|loadModule("\var{module}")| on the
  corresponding server.

\item[r]{remote} Run the \var{modules} as separated processes,
  connected to a running Urbi server (identified by \var{host} and
  \var{port}) via network connection.

\item[s]{start} Start an Urbi server with plugged-in
  \var{modules}.  In this case, the \var{module-option} are exactly
  the options supported by \command{urbi}.
\end{options}

\paragraph{Networking}
\command{urbi-launch} supports the same networking options
(\option{--host}, \option{--port}, \option{--port-file}) as
\command{urbi}, see \autoref{sec:tools:urbi}.

\subsection{Examples}

To launch a fresh server in an interactive session with the
\lstinline|UMachine| UObject compiled as the file \file{test/machine.so} (or
\file{test/machine.dll} plugged in, run:

\begin{shell}
urbi-launch --start test/machine -- --interactive
\end{shell}

To start an \urbi server accepting connections on the local port 54000
from any remote host, with \lstinline|UMachine| plugged in, run:

\begin{shell}
urbi-launch --start --host 0.0.0.0 --port 54000 test/machine
\end{shell}

Since \command{urbi-launch} in server mode is basically the same as running
the \command{urbi} program, both programs are quit the same way (see
\autoref{sec:tools:urbi:quitting}).


\section{\command{urbi-ping} --- Checking the Delays with a Server}
\label{sec:tools:urbi-ping}

\begin{shell}
urbi-ping \var{option}... [\var{host}] [\var{interval}] [\var{count}]
\end{shell}

Send ``ping''s to an \urbi server, and report how long it took for ``pong''s
to come back.

\subsection{Options}

\begin{options}[General Options]
\item[h]{help} \optionHelp
\item{version} \optionVersion
\end{options}

\begin{options}[Networking]
\item[H]{host=\var{host}} Address to connect to.
\item[P]{port=\var{port}} Port to connect to.
\item{port-file=\var{file}} Connect to the port contained in the file
  \var{file}.
\end{options}

\begin{options}[Tuning]
\item[c]{count=\var{count}} Number of pings to send.  Pass 0 to send an
  unlimited number of pings.  Defaults to 0.
\item[i]{interval=\var{interval}} Set the delays between pings, in
  milliseconds.  Pass 0 to flood the connection.  Defaults to 1000.
\end{options}

\section{\command{urbi-send} --- Sending \us Commands to a Server}
\label{sec:tools:urbi-send}

\begin{shell}
urbi-send \var{option}...
\end{shell}

Connect to an \urbi server, and send commands or file contents to it.
Stay connected, until server disconnection, or user interruption (such
as \key{C-c} under a Unix terminal).

\begin{options}
\item[e]{expression=\var{script}} Send \var{script} to the server.
\item[f]{file=\var{file}} Send the contents of \var{file} to the
  server.
\item[m]{module=\var{module}} Load the \UObject \var{module}.  This is
  equivalent to \samp[style=varInString]|{-e 'loadModule("\var{module}");'}.
\item[h]{help} \optionHelp
\item[H]{host=\var{host}} Address to connect to.
\item[P]{port=\var{port}} Port to connect to.
\item{port-file=\var{file}} Connect to the port contained in the file
  \var{file}.
\item[Q]{quit} Disconnect from the server immediately after having sent all
  the commands.  This is equivalent to \samp{-e 'quit;'} at the end of the
  options.  This is inappropriate if code running in background is expected
  to deliver its result asynchronously: the connection will be closed before
  the result was sent.

  Without this option, \command{urbi-send} prompts the user to hit
  \key{C-c} to end the connection.
\item{version} \optionVersion
\end{options}


\section{\command{urbi-sound} --- Querying Sounds from a Server}
\label{sec:tools:urbi-sound}

\begin{shell}
urbi-sound \var{option}...
\end{shell}

Connect to an \urbi server, and record sounds from it, and play them on
\file{/dev/dsp}.  This is only supported on GNU/Linux.  If
\option[o]{output} is specified, the recording is saved instead of being
played.

\subsection{Options}

\begin{options}[General Options]
\item[h]{help} \optionHelp
\item{version} \optionVersion
\end{options}

\begin{options}[Networking]
\item[H]{host=\var{host}} Address to connect to.
\item[P]{port=\var{port}} Port to connect to.
\item{port-file=\var{file}} Connect to the port contained in the file
  \var{file}.
\end{options}

\begin{options}[Tuning]
\item[d]{device=\var{device}} Query image on \var{device}.val
  (default: \code{camera}).
\item[D]{duration=\var{duration}} Specify, in seconds, how long the
  recording is performed.
\item[n]{no-header} When saving the sound, do not include the WAV headers in
  the file.
\item[o]{output=\var{file}} Save the sound into \var{file}.
\end{options}


\section{\command{umake} --- Compiling UObject Components}
\label{sec:tools:umake}

The \command{umake} programs builds loadable modules, UObjects, to be
later run using \command{urbi-launch}
(\autoref{sec:tools:urbi-launch}).  Using it is not mandatory: users
familiar with their compilation tools will probably prefer using them
directly.  Yet \command{umake} makes things more uniform and simpler,
at the cost of less control.

\subsection{Invoking \command{umake}}
\label{sec:tools:umake:invoke}

Usage:
\begin{shell}
umake \var{option}... \var{file}...
\end{shell}

Compile the \var{file}.  The \var{files} can be of different kinds:
\begin{itemize}
\item objects files (\file{*.o}, \file{*.obj} and so forth) and linked
  into the result.
\item libraries (\file{*.a}) and linked into the result.
\item source files (\file{*.cc}, \file{*.cpp}, \file{*.c}, \file{*.C})
  are compiled.
\item header files (\file{*.h}, \file{*.hh}, \file{*.hxx},
  \file{*.hpp}) are \emph{not} compiled, but used as dependencies: if
  a header file is changed, the next \command{umake} run will actually
  recompile.
\item directories are recursively traversed, and files of the above
  types are gathered as if they were given on the command line.
\end{itemize}

There are several environment variables that \command{umake} reads:
\begin{envs}
\item[EXTRA\_CPPFLAGS] Options passed to the preprocessor.
\item[EXTRA\_CXXFLAGS] Options passed to the \Cxx compiler.
\item[EXTRA\_LDFLAGS] Options passed to the linker.
\end{envs}

The arguments of \command{umake} may also include assignments, i.e.,
arguments of the type \samp{\var{var}=\var{val}}.  The behavior depends on
\var{var}, the name of the variable:
\begin{envs}
\item[EXTRA\_CPPFLAGS] Appended to the options passed to the preprocessor.
\item[EXTRA\_CXXFLAGS] Appended to the options passed to the \Cxx compiler.
\item[EXTRA\_LDFLAGS]  Appended to the options passed to the linker.
\item[VPATH] Appended to the \env{VPATH}, i.e., the list of directories in
  which the sources are looked for.
\end{envs}
\noindent
Otherwise, the argument is passed as is to \command{make}.

\begin{options}[General options]
\item{debug} Turn on shell debugging (\lstinline|set -x|) to track
  \command{umake} problems.
\item[h]{help} \optionHelp
\item[q]{quiet} Produce no output except errors.
\item[V]{version} \optionVersion
\item[v]{verbose} Report on what is done.
\end{options}

\begin{options}[Compilation options]
\item{deep-clean} Remove all building directories and exit.
\item[c]{clean} Clean building directory before compilation.
\item[j]{jobs=\var{jobs}} Specify the numbers of compilation commands to run
  simultaneously.
\item[l]{library} Produce a library, don't link to a particular core.
\item[s]{shared} Produce a shared library loadable by any core.
\item[o]{output=\var{file}} Set the output file name.
\item[C]{core=\var{core}} Set the build type.
\item[H]{host=\var{host}} Set the destination host.
\item[m]{disable-automain} Do not add the \lstinline|main| function.
\item{package=\var{pkg}} Use \command{pkg-config} to retrieve compilation
  flags (\code{cflags} and \code{libs}).  Fails if \command{pkg-config} does
  not know about the package \var{pkg}.
\end{options}

\begin{options}[Compiler and linker options]
\item[D\var{symbol}]{} Forwarded to the preprocessor (i.e., added to the
  \env{EXTRA\_CPPFLAGS}).
\item[I\var{path}]{} Forwarded to the preprocessor (i.e., added to the
  \env{EXTRA\_CPPFLAGS}).
\item[L\var{path}]{} Forwarded to the linker (i.e., added to the
  \env{EXTRA\_LDFLAGS}).
\item[l\var{lib}]{} Forwarded to the linker (i.e., added to the
  \env{EXTRA\_LDFLAGS}).
\end{options}

\begin{options}[Developer options]
\item[p]{prefix=\var{dir}} Set library files location.
\item[P]{param-mk=\var{file}} Set \file{param.mk} location.
\item[k]{kernel=\var{dir}} Set the kernel location.
\end{options}


\subsection{\command{umake} Wrappers}
\label{sec:tools:umake:wrappers}

As a convenience for common \command{umake} usages, some wrappers are
provided:
\begin{description}
\item[\command{umake-deepclean}] --- Cleaning\\
  Clean the temporary files made by running \command{umake} with the
  same arguments.  Same as \samp{umake --deep-clean}.
\item[\command{umake-shared}] --- Compiling Shared UObjects\\
  Build a shared object to be later run using \command{urbi-launch}
  (\autoref{sec:tools:urbi-launch}).  Same as \samp{umake
    --shared-library}.
\end{description}

%%% Local Variables:
%%% mode: latex
%%% TeX-master: "../urbi-sdk"
%%% ispell-dictionary: "american"
%%% ispell-personal-dictionary: "../urbi.dict"
%%% fill-column: 76
%%% End:
