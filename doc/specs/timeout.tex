\section{Timeout}

Timeout objects can be used as \refObject[s]{Tag} to execute some
code in limited time.

\subsection{Prototypes}
\begin{itemize}
\item \refObject{Tag}
\end{itemize}

\subsection{Construction}
At construction, a Timeout takes a duration, and a \refObject{Boolean}
stating whether an exception should be thrown on timeout (by default,
it does).

\begin{urbiscript}[firstnumber=last]
Timeout.new(300ms);
[00000000] Timeout_0x953c1e0
Timeout.new(300ms, false);
[00000000] Timeout_0x953c1e0
\end{urbiscript}

\subsection{Examples}

Use it as a tag :

\begin{urbiscript}[firstnumber=last]
var t = Timeout.new(300ms);
t:{
  echo("This will be displayed.");
  sleep(500ms);
  echo("This will not.");
};
[00000000] *** This will be displayed.
\end{urbiscript}

Even if exceptions have been disabled, you can check whether the
count-down expired with \lstinline|timedOut|.

\begin{urbiscript}[firstnumber=last]
t:sleep(500ms);
if (t.timedOut)
  echo("The Timeout expired.");
[00000000] *** The Timeout expired.
\end{urbiscript}

\subsection{Slots}
\begin{itemize}
\item \lstinline|launch|\\
  Fire \lstinline|this|.
%%FIXME: there must be more
\end{itemize}

%%% Local Variables:
%%% mode: latex
%%% TeX-master: "../urbi-sdk"
%%% ispell-personal-dictionary: "../urbi.dict"
%%% End:
