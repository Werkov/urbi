\section{Timeout}

\dfn{Timeout} allow to tag some pieces of code which should be
executed in limited time.

\subsection{Prototypes}
\begin{itemize}
\item \refObject{Tag}
\end{itemize}

\subsection{Construction}
A Timeout can be constructed like any other Tag but without name and
with time and optional if you want object Timeout to throw exception
or not in case of timeout. By default, Timeout throws exception.

\begin{urbiscript}
var t = Timeout.new(300ms, false);
[00000000] Timeout_0x953c1e0
\end{urbiscript}

Use it as a tag :

\begin{urbiscript}[firstnumber=last]
t:{
  echo("This code will execute.");
  sleep(350ms);
  echo("But not this one.");
};
[00000000] *** This code will execute.
\end{urbiscript}

Even if exception has been disabled in constructor, you can know if
your piece of code has timed out or not using the member ``timedOut''.

\begin{urbiscript}[firstnumber=last]
t:sleep(350ms);
if (t.timedOut)
  echo("Your code has just timed out !");
[00000000] *** Your code has just timed out !
\end{urbiscript}

%%% Local Variables:
%%% mode: latex
%%% TeX-master: "../urbi-sdk"
%%% End:
