%% Copyright (C) 2009-2010, Gostai S.A.S.
%%
%% This software is provided "as is" without warranty of any kind,
%% either expressed or implied, including but not limited to the
%% implied warranties of fitness for a particular purpose.
%%
%% See the LICENSE file for more information.

\section{Group}
A transparent means to send messages to several objects as if they
were one.

\subsection{Example}

The following session demonstrates the features of the Group
objects.  It first creates the \lstinline|Sample| family of object,
makes a group of such object, and uses that group.

\begin{urbiscript}[firstnumber=1]
class Sample
{
  var value = 0;
  function init(v) { value = v; };
  function asString() { "<" + value.asString + ">"; };
  function timesTen() { new(value * 10); };
  function plusTwo()  { new(value + 2); };
};
[00000000] <0>

var group = Group.new(Sample.new(1), Sample.new(2));
[00000000] Group [<1>, <2>]
group << Sample.new(3);
[00000000] Group [<1>, <2>, <3>]
group.timesTen.plusTwo;
[00000000] Group [<12>, <22>, <32>]

// Bouncing getSlot and updateSlot.
group.value;
[00000000] Group [1, 2, 3]
group.value = 10;
[00000000] Group [10, 10, 10]

// Bouncing to each&.
var sum = 0|
for& (var v : group)
  sum += v.value;
sum;
[00000000] 30
\end{urbiscript}

\subsection{Prototypes}

\begin{refObjects}
\item[Object]
\end{refObjects}

\subsection{Construction}

Groups are created like any other object. The constructor can
take members to add to the group.

\begin{urbiscript}
Group.new;
[00000000] Group []
Group.new(1, "two");
[00000000] Group [1, "two"]
\end{urbiscript}

\subsection{Slots}

\begin{urbiscriptapi}
\item[add](<member>, ...)%
  Add members to \lstinline|this| group, and return \lstinline|this|.

\item[asString]
  Report the members.

\item[each](<action>)%
  Apply \var{action} to all the members, in sequence, then return the
  Group of the results, in the same order.  Allows to iterate over a
  Group via \lstinline|for|.

\item \lstinline|each&(\var{action})|\\
  Apply \var{action} to all the members, concurrently, then return the
  Group of the results.  The order is \emph{not} necessarily the same.
  Allows to iterate over a Group via \lstinline|for&|.

\item[fallback]
  This function is called when a method call on \lstinline|this|
  failed.  It bounces the call to the members of the group, collects
  the results returned as a group.  This allows to chain grouped
  operation in a row.  If the dispatched calls return
  \lstinline|void|, returns a single \lstinline|void|, not a ``group
  of \lstinline|void|''.

\item[getProperty](<slot>, <prop>)%
  Bounced to the members so that
  \lstinline|this.\var{slot}->\var{prop}| actually collects the values
  of the property \var{prop} of the slots \var{slot} of the group
  members.

\item[hasProperty](<name>)%
  Bounced to the members.

\item[remove](<member>, ...)%
  Remove members from \lstinline|this| group, and return
  \lstinline|this|.

\item[setProperty](<slot>, <prop>, <value>)%
  Bounced to the members so that
  \lstinline|this.\var{slot}->\var{prop} = \var{value}|
  actually updates the value of the property \var{prop}
  in the slots \var{slot} of the group members.

\item[updateSlot](<name>, <value>)%
  Bounced to the members so that
  \lstinline|this.\var{name} = \var{value}|
  actually updates the value of the slot \var{name} in
  the group members.

\item \lstinline|'<<'(\var{member})|\\
  Syntactic sugar for \lstinline|add|.
\end{urbiscriptapi}

%%% Local Variables:
%%% mode: latex
%%% TeX-master: "../urbi-sdk"
%%% ispell-dictionary: "american"
%%% ispell-personal-dictionary: "../urbi.dict"
%%% fill-column: 76
%%% End:
