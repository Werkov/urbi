%% Copyright (C) 2011, Gostai S.A.S.
%%
%% This software is provided "as is" without warranty of any kind,
%% either expressed or implied, including but not limited to the
%% implied warranties of fitness for a particular purpose.
%%
%% See the LICENSE file for more information.

\section{Logger}

\lstinline|Logger| is used to report information to the final user or to the
developer. It allows to pretty print warnings, errors, debug messages or
simple logs. \lstinline|Logger| can also be used as \refObject{Tag} objects
for it to handle nested calls indentation. A log message is assigned a
category which is shown between brackets at beginning of lines, and a level
which defines the context in which it has to be shown (see
\autoref{sec:tools:env}). Log level definition and categories filtering can
be changed using environment variables defined in \autoref{sec:tools:env}.

\subsection{Examples}

The proper use of Loggers is to instantiate your own category, and then to
use the operator \lstinline|<<| for log messages, possibly qualified by the
proper level (in increase order of importance: \refSlot{dump},
\refSlot{debug}, \refSlot{trace}, \refSlot{log}, \refSlot{warn},
\refSlot{err}):

\begin{urbiscript}
var logger = Logger.new("Category")|;

logger.dump << "Low level debug message"|;
// Nothing displayed, unless the debug level is set to DUMP.

logger.warn << "something wrong happened, proceeding"|;
[       Category        ] something wrong happened, proceeding

logger.err << "something really bad happened!"|;
[       Category        ] something really bad happened!
\end{urbiscript}

You may also directly use the logger and passing arguments to these slots.

\begin{urbiscript}
Logger.log("message", "Category") |;
[       Category        ] message

Logger.log("message", "Category") :
{
  Logger.log("indented message", "SubCategory")
}|;
[       Category        ] message
[      SubCategory      ]   indented message
\end{urbiscript}

\subsection{Existing Categories}
\label{sec:logger:categories}
\usdk comes with a number of built-in categories that all belong to one of
the four following category families.  Precise categories are provided for
information, but there is no guarantee that these categories will
maintained, that their semantics will not change, or that they are all
listed here.

\begin{description}
\item[Libport] The Libport library.
\item[Sched] The coroutine library.
\item[Serialize] The serialization library.
\item[Urbi] Everything about \usdk.
  \begin{description}
  \item[Urbi.Parser] Bison parser traces.
  \item[Urbi.Scanner] Flex scanner traces.
  \end{description}
\end{description}


\subsection{Prototypes}
\begin{refObjects}
\item[Tag]
\end{refObjects}

\subsection{Construction}

\lstinline|Logger| can be used as is, without being cloned. It is possible
to clone \lstinline|Logger| defining a category and/or a level of debug.

\begin{urbiscript}
Logger.log("foo");
[        Logger         ] foo
[00004702] Logger<Logger>

Logger.log("bar", "Category") |;
[       Category        ] bar

var l = Logger.new("Category2");
[00004703] Logger<Category2>

l.log("foo")|;
[       Category2       ] foo

l.log("foo", "ForcedCategory") |;
[    ForcedCategory     ] foo
\end{urbiscript}

\subsection{Slots}

\begin{urbiscriptapi}
\item['<<'](<object>)%
  Allow to use the \lstinline|Logger| object as a \refObject{Channel}. This
  slot can only be used if both category and level have been defined when
  cloning.

\begin{urbiscript}
l = Logger.new("Category", Logger.Levels.Log);
[00090939] Logger<Category>
l << "abc";
[       Category        ] abc
[00091939] Logger<Category>
\end{urbiscript}


\item[categories]%
  A \refObject{Dictionary} of known categories, mapping their name to their
  activation status as a Boolean.  Note that changing this dictionary has no
  influence over the set of active categories, see \refSlot{enable} and
  \refSlot{disable}.
\begin{urbiassert}[firstnumber=1]
var c = Logger.categories;
c.isA(Dictionary);
c["Urbi.Call"];
// The four category families used by Urbi SDK.
c.keys.map (function (var s) { s.split(".")[0] }).unique
  == ["Libport", "Sched", "Serialize", "Urbi"];
\end{urbiassert}


\item[debug](<message> = "", <category> = category)%
  Report a debug \var{message} of \var{category} to the user. It will be
  shown if the debug level is \lstinline|Debug| or \lstinline|Dump|. Return
  \this to allow chained operations.
\begin{urbiscript}
// None of these are displayed unless the current level is at least DEBUG.
Logger.debug << "debug 1"|;
Logger.debug("debug 2")|;
Logger.debug("debug 3", "Category2")|;
\end{urbiscript}


\item[disable](<pattern>)%
  Disable the categories that match the \var{pattern}, i.e., suppress logs
  about these categories independently of the current \refSlot{level}.
  Affects existing and future categories.  See also \refSlot{enable}.
\begin{urbiassert}
var c = Logger.new("Logger.Foo");
c << "Enabled";
// Enabled by default.
Logger.categories["Logger.Foo"];

// Disable any categories that starts with Logger.Foo, existing or not.
Logger.disable("Logger.Foo*").isVoid;
c << "Disabled";
!Logger.categories["Logger.Foo"];

// A new category, disabled at creation.
var c2 = Logger.new("Logger.Foo.Bar");
c2 << "Disabled";
!Logger.categories["Logger.Foo.Bar"];
\end{urbiassert}


\item[dump](<message> = "", <category> = category)%
  Report a debug \var{message} of \var{category} to the user. It will be
  shown if the debug level is \lstinline|Dump|. Return \this to allow
  chained operations.
\begin{urbiscript}
// None of these are displayed unless the current level is at least DEBUG.
Logger.dump << "dump 1"|;
Logger.dump("dump 2")|;
Logger.dump("dump 3", "Category2")|;
\end{urbiscript}


\item[enable](<pattern>)%
  Enable the categories that match the \var{pattern}, i.e., suppress logs
  about these categories independently of the current \refSlot{level}.
  Affects existing and future categories.  See also \refSlot{disable}.
\begin{urbiassert}
// Disable any categories that starts with Logger.Foo, existing or not.
Logger.disable("Logger.Foo*").isVoid;
Logger.log("disabled", "Logger.Foo.Bar");
Logger.log("disabled", "Logger.Foo.Baz");

// Enable those that start with Logger.Foo.Bar.
Logger.enable("Logger.Foo.Bar*").isVoid;
Logger.log("enabled", "Logger.Foo.Bar");
[    Logger.Foo.Bar     ] enabled
Logger.log("enabled", "Logger.Foo.Bar.Qux");
[  Logger.Foo.Bar.Qux   ] enabled
Logger.log("disabled", "Logger.Foo.Baz");
\end{urbiassert}


\item[err](<message> = "", <category> = category)%
  Report an error \var{message} of \var{category} to the user. Return \this
  to allow chained operations.


\item[init](<category>)%
  Define the \var{category} of the new \lstinline|Logger| object. If no
  category is given the new \lstinline|Logger| will inherit the category of
  its prototype.


\item[level]%
  The current level of verbosity (see \refSlot{Levels}).  Can be read and
  assigned to.
\begin{urbiassert}
// Logger is enabled by default.
Logger.categories["Logger"];
// Log level is enabled by default.
Logger.level == Logger.Levels.Log;
Logger.log("Logger enabled");
[        Logger         ] Logger enabled

// Disable it.
Logger.level = Logger.Levels.None;
Logger.level == Logger.Levels.None;
Logger.log("Logger is disabled");

// Enable it back.
Logger.level = Logger.Levels.Log;
Logger.log("Logger enable again");
[        Logger         ] Logger enable again
\end{urbiassert}


\item[Levels]%
  An \refObject{Enumeration} of the log levels defined in
  \autoref{sec:tools:env}.

\begin{urbiscript}
Logger.Levels.values;
[00000001] [None, Log, Trace, Debug, Dump]
\end{urbiscript}


\item[log](<message> = "", <category> = category)%
  Report a debug \var{message} of \var{category} to the user. It will be
  shown if debug is not disabled. Return \this to allow chained operations.
\begin{urbiscript}
Logger.log << "log 1"|;
[       Logger        ] log 1

Logger.log("log 2")|;
[       Logger        ] log 2

Logger.log("log 3", "Category")|;
[      Category       ] log 3
\end{urbiscript}


\item[onEnter]%
  The primitive called when \lstinline|Logger| is used as a \lstinline|Tag|
  and is entered. This primitive only increments the indentation level.


\item[onLeave]%
  The primitive called when \lstinline|Logger| is used as a \lstinline|Tag|
  and is left. This primitive only decrements the indentation level.


\item[set](<specs>)%
  Enable/disable the categories according to the \var{specs}.  The
  \var{specs} is a comma-separated list of categories or globs (e.g.,
  \samp{Libport}, \samp{Urbi.*}) with optional modifiers \samp{+} or
  \samp{-}.  If the modifier is a \samp{+} or if there is no modifier the
  category will be displayed.  If the modifier is a \samp{-} it will be
  hidden.

  Modifiers can be chained, accumulated from left to right:
  \samp{-*,+Urbi.*,-Urbi.At} will only display categories beginning with
  \samp{Urbi.} except \samp{Urbi.At}, \samp{-Urbi.*} will display all
  categories except those beginning with \samp{Urbi.}.

  Actually, the first character defines the state for unspecified
  categories: \samp{Urbi} (or \samp{+Urbi}) enables only the \samp{Urbi}
  category, while \samp{-Urbi} enables everything but \samp{Urbi}.
  Therefore, \samp{+Urbi.*,-Urbi.At} and \samp{Urbi.*,-Urbi.At} are
  equivalent to \samp{-*,+Urbi.*,-Urbi.At}.

\begin{urbiscript}
var l1 = Logger.new("Logger")|;
var l2 = Logger.new("Logger.Sub")|;
var l3 = Logger.new("Logger.Sub.Sub")|;
function check(specs, s1, s2, s3)
{
  Logger.set(specs);
  assert
  {
    Logger.categories["Logger"]         == s1;
    Logger.categories["Logger.Sub"]     == s2;
    Logger.categories["Logger.Sub.Sub"] == s3;
  };
}|;

check("-L*",      false, false, false);
check("L*",       true, true, true);
check("-L*,+Lo*", true, true, true);
check("-L*,+Logger", true, false, false);

check("+Logger,-Logger.*", true, false, false);
check("-*,+Logger.*,-Logger.Sub.*", false, true, false);
check("+Logger*,-Logger.*,+Logger.Sub.*", true, false, true);
\end{urbiscript}

\begin{urbicomment}
// Restore the rights.
Logger.enable("*");
\end{urbicomment}


\item[trace](<message> = "", <category> = category)%
  Report a debug \var{message} of \var{category} to the user. It will be
  shown if the debug level is \lstinline|Trace|, \lstinline|Debug| or
  \lstinline|Dump|. Return \this to allow chained operations.
\begin{urbiscript}
// None of these are displayed unless the current level is at least TRACE.
Logger.trace << "trace 1"|;
Logger.trace("trace 2")|;
Logger.trace("trace 3", "Category2")|;
\end{urbiscript}


\item[warn](<message> = "", <category> = category)%
  Report a warning \var{message} of \var{category} to the user. Return \this
  to allow chained operations.
\begin{urbiscript}
Logger.warn << "warn 1"|;
[       Logger        ] warn 1

Logger.warn("warn 2")|;
[       Logger        ] warn 2

Logger.warn("warn 3", "Category")|;
[       Category       ] warn 3
\end{urbiscript}
\end{urbiscriptapi}

%%% Local Variables:
%%% mode: latex
%%% TeX-master: "../urbi-sdk"
%%% ispell-dictionary: "american"
%%% ispell-personal-dictionary: "../urbi.dict"
%%% fill-column: 76
%%% End:
