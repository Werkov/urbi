%% Redefine \section is this chapter so that we don't have to
%% call \labelObject each time.  See the bottom of this file for the
%% restoring of \section.
\let\sectionOrig\section
\renewcommand{\section}[1]{\sectionOrig{\labelObject{#1}\index{#1@\lstinline{#1}}#1}}

\chapter{\us Standard Library}
\label{sec:stdlib}

\section{Boolean}
\section{CallMessage}

\subsection{Examples}
\subsubsection{Evaluating an argument several times}
\label{sec:std-callmsg-examples-several}
\section{Channel}
\section{Code}
\section{Comparable}

%% Copyright (C) 2009-2011, Gostai S.A.S.
%%
%% This software is provided "as is" without warranty of any kind,
%% either expressed or implied, including but not limited to the
%% implied warranties of fitness for a particular purpose.
%%
%% See the LICENSE file for more information.

\section{Dictionary}

A \dfn{dictionary} is an \dfn{associative array}, also known as a \dfn{hash}
in some programming languages.  They are arrays whose indexes are arbitrary
objects.

\subsection{Example}

The following session demonstrates the features of the Dictionary objects.

\begin{urbiscript}[firstnumber=1]
var d = ["one" => 1, "two" => 2];
[00000001] ["one" => 1, "two" => 2]

for (var p : d)
  echo (p.first + " => " + p.second);
[00000003] *** one => 1
[00000002] *** two => 2

"three" in d;
[00000004] false
d["three"];
[00000005:error] !!! missing key: three
d["three"] = d["one"] + d["two"]|;
"three" in d;
[00000006] true
d.getWithDefault("four", 4);
[00000007] 4
\end{urbiscript}

\subsection{Hash values}
\label{sec:dictionary:hash}

Arbitrary objects can be used as dictionary keys. To map to the same cell,
two objects used as keys must have equal hashes (retrieved with the
\refSlot[Object]{hash} method) and be equal to each other (in the
\refSlot[Object]{'=='} sense).

This means that two different objects may have the same hash: the equality
operator (\refSlot[Object]{'=='}) is checked in addition to the hash, to
handle such collision.  However a good hash algorithm should avoid this
case, since it hinders performances.

See \refSlot[Object]{hash} for more detail on how to override hash
values. Most standard value-based classes implement a reasonable hash
function: see \refSlot[Float]{hash}, \refSlot[String]{hash},
\refSlot[List]{hash}, \ldots

\subsection{Prototypes}

\begin{refObjects}
\item[Comparable]
\item[Container]
\item[Object]
\item[RangeIterable]
\end{refObjects}

\subsection{Construction}

The Dictionary constructor takes arguments by pair (key, value).

\begin{urbiscript}
Dictionary.new("one", 1, "two", 2);
[00000000] ["one" => 1, "two" => 2]
Dictionary.new;
[00000000] [ => ]
\end{urbiscript}

There must be an even number of arguments.

\begin{urbiscript}
Dictionary.new("1", 2, "3");
[00000001:error] !!! new: odd number of arguments
\end{urbiscript}

You are encouraged to use the specific syntax for Dictionary literals:

\begin{urbiscript}
["one" => 1, "two" => 2];
[00000000] ["one" => 1, "two" => 2]
[=>];
[00000000] [ => ]
\end{urbiscript}

An extra comma can be added at the end of the list.

\begin{urbiscript}
[
  "one" => 1,
  "two" => 2,
];
[00000000] ["one" => 1, "two" => 2]
\end{urbiscript}

It is guaranteed that the pairs to insert are evaluated left-to-write, key
first, the value.

\begin{urbiassert}
   ["a".fresh => "b".fresh, "c".fresh => "d".fresh]
== ["a_5"     => "b_6",     "c_7"     => "d_8"];
\end{urbiassert}

\subsection{Slots}

\begin{urbiscriptapi}
\item['=='](<that>)%
  Whether \this equals \var{that}.  Expects members to be
  \refObject{Comparable}.
\begin{urbiassert}
[ => ] == [ => ];
["a" => 1, "b" => 2] == ["b" => 2, "a" => 1];
\end{urbiassert}


\item|'[]'|(<key>)%
  Syntactic sugar for \lstinline|get(\var{key})|.

\begin{urbiscript}
assert (["one" => 1]["one"] == 1);
["one" => 1]["two"];
[00000012:error] !!! missing key: two
\end{urbiscript}


\item|'[]='|(<key>, <value>)%
  Syntactic sugar for \lstinline|set(\var{key}, \var{value})|, but returns
  \var{value}.

\begin{urbiassert}
var d = ["one" =>"2"];
(d["one"] = 1) == 1;
d["one"] == 1;
\end{urbiassert}


\item[asBool]
  Negation of \refSlot{empty}.
\begin{urbiassert}
[=>].asBool == false;
["key" => "value"].asBool == true;
\end{urbiassert}


\item[asList]%
  The contents of the dictionary as a \refObject{Pair} list (\var{key},
  \var{value}).

\begin{urbiassert}
["one" => 1, "two" => 2].asList == [("one", 1), ("two", 2)];
\end{urbiassert}

  \noindent
  Since Dictionary derives from \refObject{RangeIterable}, it is easy
  to iterate over a Dictionary using a range-\lstinline|for|
  (\autoref{sec:lang:for:each}).  No particular order is ensured.
\begin{urbiscript}
{
  var res = [];
  for| (var entry: ["one" => 1, "two" => 2])
    res << entry.second;
  assert(res == [1, 2]);
};
\end{urbiscript}


\item[asString] A string representing the dictionary.  There is no guarantee
  on the order of the output.
\begin{urbiassert}
                [=>].asString == "[ => ]";
["a" => 1, "b" => 2].asString == "[\"a\" => 1, \"b\" => 2]";
\end{urbiassert}

\item[elementAdded] An event emitted each time a new element is added to
  the Dictionary.

\item[elementChanged] An event emitted each time the value associated to a
  key of the Dictionary is changed.

\item[elementRemoved] An event emitted each time an element is removed from
  the Dictionary.

\begin{urbiscript}
d = [ => ] |;
at(d.elementAdded?) echo ("added");
at(d.elementChanged?) echo ("changed");
at(d.elementRemoved?) echo ("removed");

d["key1"] = "value1";
[00000001] "value1"
[00000001] *** added

d["key2"] = "value2";
[00000001] "value2"
[00000001] *** added

d["key2"] = "value3";
[00000001] "value3"
[00000001] *** changed

d.erase("key2");
[00000002] ["key1" => "value1"]
[00000001] *** removed

d.clear;
[00000003] [ => ]
[00000001] *** removed

d.clear;
[00000003] [ => ]
\end{urbiscript}

\item[clear]
  Empty the dictionary.

\begin{urbiassert}
["one" => 1].clear.empty;
\end{urbiassert}


\item[empty]
  Whether the dictionary is empty.

\begin{urbiassert}
[=>].empty == true;
["key" => "value"].empty == false;
\end{urbiassert}


\item[erase](<key>) Remove the mapping for \var{key}.
\begin{urbicomment}
removeSlot("d")|;
\end{urbicomment}
\begin{urbiscript}
{
  var d = ["one" => 1, "two" => 2];
  assert
  {
    d.erase("two") === d;
    d == ["one" => 1];
  };

  try
  {
    ["one" => 1, "two" => 2].erase("three");
    echo("never reached");
  }
  catch (var e if e.isA(Dictionary.KeyError))
  {
    assert(e.key == "three")
  };
};
\end{urbiscript}

%% commented until a consensus is reached.
%%
%% \item[extend](<ext>)
%%   Extend with the dictionary \var{ext}.
%%   Return the value of the new dictionary.
%% \begin{urbiscript}
%% d = ["one" => 1, "two" => 2];
%% [00000001] ["one" => 1, "two" => 2]
%% d.extend(["one" => 0, "three" => 3]);
%% [00000002] ["one" => 0, "three" => 3, "two" => 2]
%% \end{urbiscript}

\item[get](<key>)%
  The value associated to \var{key}.  A \lstinline|Dictionary.KeyError|
  exception is thrown if the key is missing.
  % FIXME: the following exception test should be rewritten when (if)
  % we introduce the throw assertion.
\begin{urbiscript}
var d = ["one" => 1, "two" => 2]|;

assert(d.get("one") == 1);
["one" => 1, "two" => 2].get("three");
[00000010:error] !!! missing key: three

try
{
  d.get("three");
  echo("never reached");
}
catch (var e if e.isA(Dictionary.KeyError))
{
  assert(e.key == "three")
};
\end{urbiscript}


\item[getWithDefault](<key>, <defaultValue>)%
  The value associated to \var{key} if it exists, \var{defaultValue}
  otherwise.

\begin{urbiassert}
var d = ["one" => 1, "two" => 2];
d.getWithDefault("one",  -1) == 1;
d.getWithDefault("three", 3) == 3;
\end{urbiassert}


\item[has](<key>)%
  Whether the dictionary has a mapping for \var{key}.

\begin{urbiassert}
var d = ["one" => 1];
d.has("one");
!d.has("zero");
\end{urbiassert}

  The infix operators \lstinline|in| and \lstinline|not in| use
  \lstinline|has| (see \autoref{sec:lang:op:containers}).

\begin{urbiassert}
"one" in     ["one" => 1];
"two" not in ["one" => 1];
\end{urbiassert}


\item[init](<key1>, <value1>, ...)%
  Insert the mapping from \var{key1} to \var{value1} and so forth.

\begin{urbiscript}
Dictionary.clone.init("one", 1, "two", 2);
[00000000] ["one" => 1, "two" => 2]
\end{urbiscript}


\item[keys]%
  The list of all the keys.  No particular order is ensured.  Since
  \refObject{List} features the same function, uniform iteration over
  a List or a Dictionary is possible.
\begin{urbiassert}
var d = ["one" => 1, "two" => 2];
d.keys == ["one", "two"];
\end{urbiassert}


\item[matchAgainst](<handler>, <pattern>)
  Pattern matching on members.  See \refObject{Pattern}.

\begin{urbiscript}
{
  // Match a subset of the dictionary.
  ["a" => var a] = ["a" => 1, "b" => 2];
  // get the matched value.
  assert(a == 1);
};
\end{urbiscript}


\item[set](<key>, <value>)%
  Map \var{key} to \var{value} and return \this so that invocations to
  \refSlot{set} can be chained.  The possibly existing previous mapping is
  overridden.

\begin{urbiscript}
[=>].set("one", 2)
    .set("two", 2)
    .set("one", 1);
[00000000] ["one" => 1, "two" => 2]
\end{urbiscript}


\item[size]
  Number of element in the dictionary.

\begin{urbiassert}
var d = [=>];  d.size == 0;
d["a"] = 10;   d.size == 1;
d["b"] = 20;   d.size == 2;
d["a"] = 30;   d.size == 2;
\end{urbiassert}



\end{urbiscriptapi}


%%% Local Variables:
%%% mode: latex
%%% TeX-master: "../urbi-sdk"
%%% ispell-dictionary: "american"
%%% ispell-personal-dictionary: "../urbi.dict"
%%% fill-column: 76
%%% End:


\section{Event}

\section{Float}

A Float is a floating point number.  It is also used, in the current
version of \us, to represent integers.

\subsection{Prototypes}

\begin{refObjects}
\item[Comparable]
\item[Orderable]
\item[RangeIterable]
\end{refObjects}

\subsection{Construction}
\label{sec:float:ctor}

The most common way to create fresh floats is using the literal
syntax.  Numbers are composed of three parts:
\begin{description}
\item[integral] (mandatory) a non empty sequence of (decimal) digits;
\item[fractional] (optional) a period, and a non empty sequence of
  (decimal) digits;
\item[exponent] (optional) either \samp{e} or \samp{E}, an optional
  sign (\samp{+} or \samp{-}), then a non-empty sequence of digits.
\end{description}

In other words, float literals match the
\lstinline|[0-9]+(\.[0-9]+)?([eE][-+]?[0-9]+)?|
regular expression.  For instance:

\begin{urbiassert}
0 == 0000.0000;
// This is actually a call to the unary '+'.
+1 == 1;
0.123456 == 123456 / 1000000;
1e3 == 1000;
1e-3 == 0.001;
1.234e3 == 1234;
\end{urbiassert}

There are also some special numbers, \lstinline|nan|, \lstinline|inf|
(see below).

\begin{urbiassert}
Math.log(0) == -inf;
Math.exp(-inf) == 0;
(inf/inf).asString == "nan";
\end{urbiassert}

A null float can also be obtained with \lstinline|Float|'s
\lstinline|new| method.

\begin{urbiassert}
Float.new == 0;
\end{urbiassert}

\subsection{Slots}

\begin{urbiscriptapi}
\item[abs]
  Absolute value of the target.
\begin{urbiassert}
(-5).abs == 5;
  0 .abs == 0;
  5 .abs == 5;
\end{urbiassert}

\item[acos]
  Arccosine of the target.
\begin{urbiassert}
0.acos == Float.pi/2;
1.acos == 0;
\end{urbiassert}

\item[asBool]
  Whether non null.
\begin{urbiassert}
0.asBool == false;
0.1.asBool == true;
(-0.1).asBool == true;
inf.asBool == true;
nan.asBool == true;
\end{urbiassert}

\item[asFloat]
  Return the target.
\begin{urbiassert}
51.asFloat == 51;
\end{urbiassert}

\item[asList]
  Bounces to \lstinline|seq|.  It is therefore possible to use the
  various flavors of \lstinline|for|-range loops on integers:
\begin{urbiassert}
{
  var res = [];
  for (var i : 3)
    res << i;
  res
}
== [0, 1, 2];

{
  var res = [];
  for|(var i : 3)
    res << i;
  res
}
== [0, 1, 2];

{
  var res = [];
  for&(var i : 3)
    res << i;
  res.sort
}
== [0, 1, 2];
\end{urbiassert}%>>

\item[asin]
  Arcsine of the target.
\begin{urbiassert}
0.asin == 0;
\end{urbiassert}

\item[asString]
  Return a string representing the target.
\begin{urbiassert}
42.asString == "42";
\end{urbiassert}

\item[atan]
  Return the arctangent of the target.
\begin{urbiassert}
0.atan == 0;
1.atan == Float.pi/4;
\end{urbiassert}

\item \lstinline|'bitand'(\var{that})|\\
  The bitwise-and between \lstinline|this| and \var{that}.
\begin{urbiassert}
(3 bitand 6) == 2;
\end{urbiassert}

\item \lstinline|'bitor'(\var{that})|\\
  Bitwise-or between \lstinline|this| and \var{that}.
\begin{urbiassert}
(3 bitor 6) == 7;
\end{urbiassert}

\item[clone]
  Return a fresh Float with the same value as the target.
\begin{urbiscript}
var x = 0;
[00000000] 0
var y = x.clone;
[00000000] 0
x === y;
[00000000] false
\end{urbiscript}

\item[compl]
  The complement to 1 of the target interpreted as a 32 bits integer.
\begin{urbiassert}
compl 0 == 4294967295;
compl 4294967295 == 0;
\end{urbiassert}

\item[cos]
  Cosine of the target.
\begin{urbiassert}
0.cos == 1;
Float.pi.cos == -1;
\end{urbiassert}

\item \lstinline|each(\var{fun})|\\
  Call the functional argument \var{fun} on every integer from 0 to
  target - 1, sequentially.  The number must be non-negative.
\begin{urbiassert}
{
  var res = [];
  3.each(function (i) { res << 100 + i });
  res
}
== [100, 101, 102];

{
  var res = [];
  for(var x : 3) { res << x; sleep(20ms); res << (100 + x); };
  res
}
== [0, 100, 1, 101, 2, 102];

{
  var res = [];
  0.each (function (i) { res << 100 + i });
  res
}
== [];
\end{urbiassert}

\item \lstinline'each|(\var{fun})'\\
  Call the functional argument \var{fun} on every integer from 0 to
  target - 1, with tight sequentiality.  The number must be
  non-negative.
\begin{urbiassert}
{
  var res = [];
  3.'each|'(function (i) { res << 100 + i });
  res
}
== [100, 101, 102];

{
  var res = [];
  for|(var x : 3) { res << x; sleep(20ms); res << (100 + x); };
  res
}
== [0, 100, 1, 101, 2, 102];
\end{urbiassert}%>>>>>>

\item \lstinline|each&(\var{fun})|\\
  Call the functional argument \var{fun} on every integer from 0 to
  target - 1, concurrently.  The number must be non-negative.
\begin{urbiassert}
{
  var res = [];
  for& (var x : 3) { res << x; sleep(30ms); res << (100 + x) };
  res
}
== [0, 1, 2, 100, 101, 102];
\end{urbiassert}%>>>>

\item[exp]
  Exponential of the target.
\begin{urbiscript}
1.exp;
[00000000] 2.71828
\end{urbiscript}

\item \lstinline|format(\var{finfo})|\\
  Format according to the \refObject{FormatInfo} object \var{finfo}.
  The precision, \lstinline|\var{finfo}.precision|, sets the maximum
  number of digits after decimal point when in fixed or scientific
  mode, and in total when in default mode.  Beware that 0 plays a
  special role, as it is not a ``significant'' digit.

  \begin{windows}
    Under Windows the behavior differs slightly.
  \end{windows}
\begin{urbiassert}
"%1.0d" % 0.1 == "0.1";
"%1.0d" % 1.1 == {if (System.Platform.isWindows) "1.1" else "1"};

"%1.0f" % 0.1 == "0";
"%1.0f" % 1.1 == "1";
\end{urbiassert}

\item[inf]
  Return the infinity.
\begin{urbiscript}
Float.inf;
[00000000] inf
\end{urbiscript}

\item[limit_digits]
  Number of digits (in \lstinline|Float.limit_radix| base) in the
  mantissa.
\begin{urbiassert}
Float.limit_digits;
\end{urbiassert}

\item[limit_digits10]
  Number of digits (in decimal base) that can be represented without
  change.
\begin{urbiassert}
Float.limit_digits10;
\end{urbiassert}

\item[limit_epsilon]
  Machine epsilon (the difference between 1 and the least value
  greater than 1 that is representable).
\begin{urbiassert}
1 != 1 + Float.limit_epsilon;
1 == 1 + Float.limit_epsilon / 2;
\end{urbiassert}

\item[limit_max]
  Maximum finite value.
\begin{urbiassert}
Float.limit_max     != Float.inf;
Float.limit_max * 2 == Float.inf;
\end{urbiassert}

\item[limit_max_exponent]
  Maximum integer value for the exponent that generates a normalized
  floating-point number.
\begin{urbiassert}
Float.inf != Float.limit_radix ** (Float.limit_max_exponent - 1);
Float.inf == Float.limit_radix ** Float.limit_max_exponent;
\end{urbiassert}

\item[limit_max_exponent10]
  Maximum integer value such that 10 raised to that power generates a
  normalized finite floating-point number.
\begin{urbiassert}
Float.inf != 10 ** Float.limit_max_exponent10;
Float.inf == 10 ** (Float.limit_max_exponent10 + 1);
\end{urbiassert}

\item[limit_min]
  Minimum positive normalized value.
\begin{urbiassert}
0 != Float.limit_min;
\end{urbiassert}

\item[limit_min_exponent]
  Minimum negative integer value for the exponent that generates a
  normalized floating-point number.
\begin{urbiassert}
0 != Float.limit_radix ** Float.limit_min_exponent;
\end{urbiassert}

\item[limit_min_exponent10]
  Minimum negative integer value such that 10 raised to that power
  generates a normalized floating-point number.
\begin{urbiassert}
0 != 10 ** Float.limit_min_exponent10;
\end{urbiassert}

\item[limit_radix]
  Base of the exponent of the representation.
\begin{urbiassert}
Float.limit_radix == 2;
\end{urbiassert}

\item[log]
  The logarithm of the target.
\begin{urbiassert}
0.log == -inf;
1.log == 0;
1.exp.log == 1;
\end{urbiassert}

\item \lstinline|max(\var{arg1}, ...)|\\
  Bounces to \lstinline|List.max| on \lstinline|[this, \var{arg1}, ...]|.
\begin{urbiassert}
1.max == 1;
1.max(2, 3) == 3;
3.max(1, 2) == 3;
\end{urbiassert}

\item \lstinline|min(\var{arg1}, ...)|\\
  Bounces to \lstinline|List.min| on \lstinline|[this, \var{arg1}, ...]|.
\begin{urbiassert}
1.min == 1;
1.min(2, 3) == 1;
3.min(1, 2) == 1;
\end{urbiassert}

\item[nan]
  The ``not a number'' special float value.  More precisely, this
  returns the ``quiet NaN'', i.e., it is propagated in the various
  computations, it does not raise exceptions.
\begin{urbiscript}
Float.nan;
[00000000] nan
(Float.nan + Float.nan) / (Float.nan - Float.nan);
[00000000] nan
\end{urbiscript}

A {NaN} has one distinctive property over the other Floats: it is
equal to no other float, not even itself.  This behavior is mandated
by the \wref[IEEE_754-2008]{IEEE 754-2008} standard.
\begin{urbiassert}
{ var n = Float.nan; n === n};
{ var n = Float.nan; n  != n};
\end{urbiassert}

\item[pi]
  $\pi$.
\begin{urbiassert}
Float.pi.cos ** 2 + Float.pi.sin ** 2 == 1;
\end{urbiassert}

\item[random]
  A random integer between 0 (included) and the target (excluded).
\begin{urbiscript}
20.map(function (dummy) { 5.random });
[00000000] [1, 2, 1, 3, 2, 3, 2, 2, 4, 4, 4, 1, 0, 0, 0, 3, 2, 4, 3, 2]
\end{urbiscript}

\item[round]
  The target, rounded to the nearest integer.
\begin{urbiassert}
1.6.round == 2;
1.4.round == 1;
\end{urbiassert}

\item[seq]
  The sequence of integers from 0 to \lstinline|this| - 1 as a list.
  The number must be non-negative.
\begin{urbiassert}
3.seq == [0, 1, 2];
0.seq == [];
(-1).seq;
[00004586:error] !!! seq: expected non-negative integer, got -1
\end{urbiassert}

\item[sign]
  Return 1 if \lstinline|this| is positive, 0 if it is null, -1
  otherwise.
\begin{urbiassert}
(-1164).sign == -1;
0.sign       == 0;
(1164).sign  == 1;
\end{urbiassert}

\item[sin]
  The sine of the target.
\begin{urbiassert}
0.sin == 0;
\end{urbiassert}

\item[sqr]
  Square of the target.
\begin{urbiassert}
32.sqr == 1024;
32.sqr == 32 ** 2;
\end{urbiassert}

\item[sqrt]
  The square root of the target.
\begin{urbiassert}
1024.sqrt == 32;
1024.sqrt == 1024 ** 0.5;
\end{urbiassert}

\item[srandom]
  Initialized the seed used by the random function.  As opposed to common
  usage, you should not use
\begin{urbiunchecked}
{
  var now = Date.now.timestamp;
  now.srandom;
  var list1 = 20.map(function (dummy) { 5.random });
  now.srandom;
  var list2 = 20.map(function (dummy) { 5.random });
  assert
  {
    list1 == list2;
  }
};
\end{urbiunchecked}

\item[tan]
  Tangent of the target.
\begin{urbiscript}
assert(0.tan == 0);
(Float.pi/4).tan;
[00000000] 1
\end{urbiscript}

\item \lstinline|times(\var{fun})|\\
  Call the functional argument \var{fun} \lstinline|this| times.

\begin{urbiscript}
3.times(function () { echo("ping")});
[00000000] *** ping
[00000000] *** ping
[00000000] *** ping
\end{urbiscript}

\item[trunc]
  Return the target truncated.
\begin{urbiassert}
1.9.trunc == 1;
(-1.9).trunc == -1;
\end{urbiassert}

\item \lstinline|'^'(\var{that})|\\
  Bitwise exclusive or between \lstinline|this| and \var{that}.
\begin{urbiassert}
(3 ^ 6) == 5;
\end{urbiassert}

\item \lstinline|'>>'(\var{that})|\\%>>
  \lstinline|this| shifted by \var{that} bits towards the right.
\begin{urbiassert}
4 >> 2 == 1;
\end{urbiassert}

\item \lstinline|'<'(\var{that})|\\
  Whether \lstinline|this| is less than \var{b}. The other comparison
  operators (\lstinline|<=|, \lstinline|>|, \ldots) can thus also be
  applied on floats since Float inherits \refObject{Orderable}.
\begin{urbiassert}
  0 < 1;
!(1 < 0);
\end{urbiassert}

\item \lstinline|'<<'(\var{that})|\\
  \lstinline|this| shifted by \var{that} bit towards the left.
\begin{urbiassert}
4 << 2 == 16;
\end{urbiassert}

\item \lstinline|'-'(\var{that})|\\
  \lstinline|this| subtracted by \var{b}.
\begin{urbiassert}
6 - 3 == 3;
\end{urbiassert}

\item \lstinline|'+'(\var{that})|\\
  The sum of \lstinline|this| and \var{that}.
\begin{urbiassert}
1 + 1 == 2;
\end{urbiassert}

\item \lstinline|'/'(\var{that})|\\
  The quotient of \lstinline|this| divided by \var{that}.
\begin{urbiassert}
50 / 10 == 5;
10 / 50 == 0.2;
\end{urbiassert}

\item \lstinline|'%'(\var{that})|\\
  \lstinline|this| modulo \var{b}.
\begin{urbiassert}
50 % 11 == 6;
\end{urbiassert}

\item \lstinline|'*'(\var{that})|\\
  Product of \lstinline|this| by \var{that}.
\begin{urbiassert}
2 * 3 == 6;
\end{urbiassert}

\item \lstinline|'**'(\var{that})|\\
  \lstinline|this| to the \var{that} power (${this}^{that}$).
\begin{urbiassert}
2 ** 10 == 1024;
\end{urbiassert}

\item \lstinline|'=='(\var{that})|\\
  Whether \lstinline|this| equals \var{that}.
\begin{urbiassert}
  1 == 1;
!(1 == 2);
\end{urbiassert}
\end{urbiscriptapi}

%%% Local Variables:
%%% mode: latex
%%% TeX-master: "../urbi-sdk"
%%% ispell-dictionary: "american"
%%% ispell-personal-dictionary: "../urbi.dict"
%%% End:


\section{Group}

%% Copyright (C) 2009-2011, Gostai S.A.S.
%%
%% This software is provided "as is" without warranty of any kind,
%% either expressed or implied, including but not limited to the
%% implied warranties of fitness for a particular purpose.
%%
%% See the LICENSE file for more information.

\section{Lazy}

\dfn{Lazies} are objects that hold a lazy value, that is, a not yet
evaluated value. They provide facilities to evaluate their content only once
(\dfn{memoization}) or several times. Lazy are essentially used in call
messages, to represent lazy arguments, as described in
\autorefObject{CallMessage}.

\subsection{Examples}

\subsubsection{Evaluating once}

One usage of lazy values is to avoid evaluating an expression unless it's
actually needed, because it's expensive or has undesired side effects. The
listing below presents a situation where an expensive-to-compute value
(\lstinline|heavy_computation|) might be needed zero, one or two times. The
objective is to save time by:

\begin{itemize}
\item Not evaluating it if it's not needed.
\item Evaluating it only once if it's needed once or twice.
\end{itemize}

We thus make the wanted expression lazy, and use the \lstinline|value|
method to fetch its value when needed.

\begin{urbiscript}[firstnumber=1]
// This function supposedly performs expensive computations.
function heavy_computation()
{
  echo("Heavy computation");
  return 1 + 1;
}|;

// We want to do the heavy computations only if needed,
// and make it a lazy value to be able to evaluate it "on demand".
var v = Lazy.new(closure () { heavy_computation() });
[00000000] heavy_computation()
/* some code */;
// So far, the value was not needed, and heavy_computation
// was not evaluated.
/* some code */;
// If the value is needed, heavy_computation is evaluated.
v.value();
[00000000] *** Heavy computation
[00000000] 2
// If the value is needed a second time, heavy_computation
// is not reevaluated.
v.value();
[00000000] 2
\end{urbiscript}

\subsubsection{Evaluating several times}

Evaluating a lazy several times only makes sense with lazy arguments and
call messages. See example with call messages in
\autoref{sec:std-callmsg-examples-several}.


\subsection{Caching}

\refObject{Lazy} is meant for functions without argument.  If you need
\dfn{caching} for functions that depend on arguments, it is straightforward
to implement using a \refObject{Dictionary}.  In the future \us might
support dictionaries whose indices are not only strings, but in the
meanwhile, convert the arguments into strings, as the following sample
object demonstrates.

\begin{urbiscript}
class UnaryLazy
{
  function init(f)
  {
    results = [ => ];
    func = f;
  };
  function value(p)
  {
    var sp = p.asString();
    if (results.has(sp))
      return results[sp];
    var res = func(p);
    results[sp] = res |
    res
  };
  var results;
  var func;
} |
// The function to cache.
var inc = function(x) { echo("incing " + x) | x+1 } |
// The function with cache. UnaryLazy simply takes the function as argument.
var p = UnaryLazy.new(inc);
[00062847] UnaryLazy_0x78b750
p.value(1);
[00066758] *** incing 1
[00066759] 2
p.value(1);
[00069058] 2
p.value(2);
[00071558] *** incing 2
[00071559] 3
p.value(2);
[00072762] 3
p.value(1);
[00074562] 2
\end{urbiscript}

\subsection{Prototypes}

\begin{refObjects}
\item[Comparable]
\end{refObjects}

\subsection{Construction}

Lazies are seldom instantiated manually. They are mainly created
automatically when a lazy function call is made (see
\autoref{sec:lang:call}). One can however create a lazy value with the
standard \lstinline|new| method of \lstinline|Lazy|, giving it an
argument-less function which evaluates to the value made lazy.

\begin{urbiscript}
Lazy.new(closure () { /* Value to make lazy */ 0 });
[00000000] 0
\end{urbiscript}

\subsection{Slots}

\begin{urbiscriptapi}
\item['=='](<that>)%
  Whether \this and \var{that} are the same source code and value (an not
  yet evaluated Lazy is never equal to an evaluated one).
\begin{urbiassert}
Lazy.new(closure () { 1 + 1 }) == Lazy.new(closure () { 1 + 1 });
Lazy.new(closure () { 1 + 2 }) != Lazy.new(closure () { 2 + 1 });
\end{urbiassert}
\begin{urbiscript}
{
  var l1 = Lazy.new(closure () { 1 + 1 });
  var l2 = Lazy.new(closure () { 1 + 1 });
  assert (l1 == l2);
  l1.eval();
  assert (l1 != l2);
  l2.eval();
  assert (l1 == l2);
};
\end{urbiscript}


\item[asString]
  The conversion to \refObject{String} of the body of a non-evaluated
  argument.
\begin{urbiassert}
Lazy.new(closure () { echo(1); 1 }).asString() == "echo(1);\n1";
\end{urbiassert}


\item[eval]%
  Force the evaluation of the held lazy value. Two calls to \refSlot{eval}
  will systematically evaluate the expression twice, which can be useful to
  duplicate its side effects.


\item[value]%
  Return the held value, potentially evaluating it before. \refSlot{value}
  performs memoization, that is, only the first call will actually evaluate
  the expression, subsequent calls will return the cached value. Unless you
  want to explicitly trigger side effects from the expression by evaluating
  it several time, this should be preferred over \lstinline|eval| to avoid
  evaluating the expression several times uselessly.
\end{urbiscriptapi}


%%% Local Variables:
%%% coding: utf-8
%%% mode: latex
%%% TeX-master: "../urbi-sdk"
%%% ispell-dictionary: "american"
%%% ispell-personal-dictionary: "../urbi.dict"
%%% fill-column: 76
%%% End:

\section{List}

\lstinline|List|s implement potentially-empty ordered (heterogeneous)
collections of elements.

\subsection{Prototypes}

\begin{itemize}
\item \refObject{Object}
\item \refObject{RangeIterable}
\item \refObject{Orderable}
\end{itemize}

\subsection{Construction}

List can be created with their literal syntax: a possibly empty
sequence of expressions in square brackets, separated by commas.
Non-empty list may actually \emph{terminate} with a comma, rather than
\emph{separate}; in other words, an optional trailing comma is accepted.

\begin{urbiscript}
[]; // The empty list
[00000000] []
[1, "2", [3,],];
[00000000] [1, "2", [3]]
\end{urbiscript}

\subsection{Slots}

\begin{itemize}
\item \lstinline|all(\var{fun})|\\
  % FIXME: link to predicate glossary entry
  Return whether all the members of the target verify the predicate
  \var{fun}.

\begin{urbiassert}[firstnumber=last]
// Are all elements positive?
! [-2, 0, 2, 4].all(function (e) { e > 0 });
// Are all elements even?
[-2, 0, 2, 4].all(function (e) { e % 2 == 0 });
\end{urbiassert}

\item \lstinline|any(\var{fun})|\\
  % FIXME: link to predicate glossary entry
  Whether at least one of the members of the target verifies the
  predicate \var{fun}.

\begin{urbiassert}[firstnumber=last]
// Is there any even element?
! [-3, 1, -1].any(function (e) { e % 2 == 0 });
// Is there any positive element?
[-3, 1, -1].any(function (e) { e > 0 });
\end{urbiassert}

\item \lstinline|asBool|\\
  Whether not empty.
\begin{urbiassert}[firstnumber=last]
[].asBool == false;
[1].asBool == true;
\end{urbiassert}

\item \lstinline|asList|\\
Return the target.

\begin{urbiassert}[firstnumber=last]
[0, 1, 2].asList == [0, 1, 2];
\end{urbiassert}

\item \lstinline|asString|\\
  A string describing the list.  Uses \lstinline|asPrintable| on its
  members, so that, for instance, strings are displayed with quotes.

\begin{urbiassert}[firstnumber=last]
[0, [1], "2"].asString == "[0, [1], \"2\"]";
\end{urbiassert}

\item \lstinline|back|\\
Return the last element of the target. An error if the target is empty.

\begin{urbiscript}[firstnumber=last]
assert([0, 1, 2].back == 2);
[].back;
[00000000:error] !!! back: cannot be applied onto empty list
\end{urbiscript}

\item \lstinline|clear|\\
  Empty the target.

\begin{urbiscript}[firstnumber=last]
var x = [0, 1, 2];
[00000000] [0, 1, 2]
assert(x.clear == []);
\end{urbiscript}

\item \lstinline|each(\var{fun})|\\
  Apply the given functional value \var{fun} on all members,
  sequentially.

\begin{urbiscript}[firstnumber=last]
[0, 1, 2].each(function (v) {echo (v * v); echo (v * v)});
[00000000] *** 0
[00000000] *** 0
[00000000] *** 1
[00000000] *** 1
[00000000] *** 4
[00000000] *** 4
\end{urbiscript}

\item \lstinline|'each&'(\var{fun})|\\
Apply the given functional value on all members simultaneously.

\begin{urbiscript}[firstnumber=last]
[0, 1, 2].'each&'(function (v) {echo (v * v); echo (v * v)});
[00000000] *** 0
[00000000] *** 1
[00000000] *** 4
[00000000] *** 0
[00000000] *** 1
[00000000] *** 4
\end{urbiscript}

\item \lstinline|empty|\\
  Whether the target is empty.

\begin{urbiassert}[firstnumber=last]
[].empty;
! [1].empty;
\end{urbiassert}

\item \lstinline|filter(\var{fun})|\\
  The list of all the members of the target that verify the predicate
  \var{fun}.

\begin{urbiassert}[firstnumber=last]
// Keep only odd numbers.
[0, 1, 2, 3, 4, 5].filter(function (v) {v % 2 == 1}) == [1, 3, 5];
\end{urbiassert}

\item \lstinline|foldl(\var{action}, \var{value})|\\
  \wref[Fold_(higher-order_function)]{Fold},
  also known as \dfn{reduce} or \dfn{accumulate}, computes a result
  from a list.  Starting from \var{value} as the initial result, apply
  repeatedly the binary \var{action} to the current result and the
  next member of the list, from left to right.  For instance, if
  \var{action} were the binary addition and \var{value} were 0, then
  folding a list would compute the sum of the list, including for
  empty lists.

\begin{urbiscript}[firstnumber=last]
[].foldl(function (a, b) { a + b }, 0);
[00000000] 0
[1, 2, 3].foldl(function (a, b) { a + b }, 0);
[00000000] 6
[1, 2, 3].foldl(function (a, b) { a - b }, 0);
[00000000] -6
\end{urbiscript}

\item \lstinline|front|\\
  Return the first element of the target. An error if the target is
  empty.

\begin{urbiscript}[firstnumber=last]
assert([0, 1, 2].front == 0);
[].front;
[00000000:error] !!! front: cannot be applied onto empty list
\end{urbiscript}

\item \lstinline|has(\var{that})|\\
  Whether one of the members of the target equals the argument.

\begin{urbiassert}[firstnumber=last]
[0, 1, 2].has(1);
! [0, 1, 2].has(5);
\end{urbiassert}

\item \lstinline|hasSame(\var{that})|\\
  Return whether one of the member of the target is physically equal
  to the argument.

\begin{urbiscript}[firstnumber=last]
var y = 1;
[00000000:hide] 1
[0, y, 2].hasSame(1);
[00000000] false
[0, y, 2].hasSame(y);
[00000000] true
\end{urbiscript}

\item \lstinline|head|\\
  Synonym for \lstinline|front|.

\item \lstinline|insertBack(\var{that})|\\
  Insert the given element at the end of the target.

\begin{urbiscript}[firstnumber=last]
var z = [0, 1];
[00000000] [0, 1]
assert(z.insertBack(2) == [0, 1, 2]);
assert(z == [0, 1, 2]);
\end{urbiscript}

\item \lstinline|insertFront(\var{that})|\\
  Insert the given element at the beginning of the target.

\begin{urbiscript}[firstnumber=last]
var a = [1, 2];
[00000000] [1, 2]
assert(a.insertFront(0) == [0, 1, 2]);
assert(a == [0, 1, 2]);
\end{urbiscript}

\item \lstinline|join(\var{sep} = "", \var{prefix} = "", \var{suffix} = "")|\\
  Bounces to \lstinline|String.join|, see \refObject{String}.

\begin{urbiassert}[firstnumber=last]
["", "ob", ""].join                == "ob";
["", "ob", ""].join("a")           == "aoba";
["", "ob", ""].join("a", "B", "b") == "Baobab";
\end{urbiassert}

\item \lstinline|keys()|\\
  The list of valid indexes.  This allows uniform iteration over a
  \refObject{Dictionary} or a \refObject{List}.

\begin{urbiscript}[firstnumber=last]
{
  var l = ["a", "b", "c"];
  assert(l.keys == [0, 1, 2]);
  assert({
           var res = [];
           for (var k: l.keys)
             res << l[k];
           res
         }
         == l);
};
\end{urbiscript}

\item \lstinline|map(\var{fun})|\\
Apply the given functional value on every member, and return the list
of results.

\begin{urbiassert}[firstnumber=last]
[0, 1, 2, 3].map(function (v) { v % 2 == 0})
        == [true, false, true, false];
\end{urbiassert}

\item \lstinline|range(\var{begin}, \var{end} = nil)|\\
  Return a sub-range of the list, from the first index included to the
  second index excluded.  An error if out of bounds.  Negative indices
  are valid, and number from the end.

  If \var{end} is \lstinline|nil|, calling \lstinline|range(\var{n})
  is equivalent to calling \lstinline|range(0, \var{n})|.

\begin{urbiscript}[firstnumber=last]
do ([0, 1, 2, 3])
{
  assert
  {
    range(0, 0)   == [];
    range(0, 1)   == [0];
    range(1)      == [0];
    range(1, 3)   == [1, 2];

    range(-3, -2) == [1];
    range(-3, -1) == [1, 2];
    range(-3, 0)  == [1, 2, 3];
    range(-3, 1)  == [1, 2, 3, 0];
    range(-4, 4)  == [0, 1, 2, 3, 0, 1, 2, 3];
  };
}|;
[].range(1, 3);
[00428697:error] !!! range: invalid index: 1
\end{urbiscript}

\item \lstinline|remove(\var{val})|\\
  Remove all elements from the target that equals \var{val}.

\begin{urbiscript}[firstnumber=last]
var c = [0, 1, 0, 2, 0, 3];
[00000000] [0, 1, 0, 2, 0, 3]
assert(c.remove(0) == [1, 2, 3]);
assert(c == [1, 2, 3]);
\end{urbiscript}

\item \lstinline|removeBack|\\
  Remove and return the last element of the target. An error if the
  target is empty.

\begin{urbiscript}[firstnumber=last]
var t = [0, 1, 2];
[00000000] [0, 1, 2]
assert(t.removeBack == 2);
assert(t == [0, 1]);
[].removeBack;
[00000000:error] !!! removeBack: cannot be applied onto empty list
\end{urbiscript}

\item \lstinline|removeById(\var{that})|\\
  Remove all elements from the target that physically equals
  \var{that}.

\begin{urbiscript}[firstnumber=last]
var d = 1;
[00000000] 1
var e = [0, 1, d, 1, 2];
[00000000] [0, 1, 1, 1, 2]
assert(e.removeById(d) == [0, 1, 1, 2]);
assert(e == [0, 1, 1, 2]);
\end{urbiscript}

\item \lstinline|removeFront|\\
Remove and return the first element from the target. An error if the
target is empty.

\begin{urbiscript}[firstnumber=last]
var g = [0, 1, 2];
[00000000] [0, 1, 2]
assert(g.removeFront == 0);
assert(g == [1, 2]);
[].removeFront;
[00000000:error] !!! removeFront: cannot be applied onto empty list
\end{urbiscript}

\item \lstinline|reverse|\\
Return the target with the order of elements inverted.

\begin{urbiassert}[firstnumber=last]
[0, 1, 2].reverse == [2, 1, 0];
\end{urbiassert}

\item \lstinline|size|\\
Return the number of elements in the target.

\begin{urbiassert}[firstnumber=last]
[0, 1, 2].size == 3;
[].size == 0;
\end{urbiassert}

\item \lstinline|sort|\\
Return the target, sorted with respect to the \lstinline|<| criteria.

\begin{urbiassert}[firstnumber=last]
[1, 0, 3, 2].sort == [0, 1, 2, 3];
\end{urbiassert}

\item \lstinline|tail|\\
Return the target, minus the first element. An error if the target is
empty.

\begin{urbiscript}[firstnumber=last]
assert([0, 1, 2].tail == [1, 2]);
[].tail;
[00000000:error] !!! tail: cannot be applied onto empty list
\end{urbiscript}

\item \lstinline|'=='(\var{that})|\\
Check whether all elements in the target and \var{that}, are
equal two by two.

\begin{urbiassert}[firstnumber=last]
[0, 1, 2] == [0, 1, 2];
!([0, 1, 2] == [0, 0, 2]);
\end{urbiassert}

\item \lstinline|'[]'(\var{n})|\\
  Return the \var{n}th member of the target (indexing is
  zero-based). If \var{n} is negative, start from the end.  An error
  if out of bounds.

\begin{urbiscript}[firstnumber=last]
assert(["0", "1", "2"][0] == "0");
assert(["0", "1", "2"][2] == "2");
["0", "1", "2"][3];
[00007061:error] !!! []: invalid index: 3

assert(["0", "1", "2"][-1] == "2");
assert(["0", "1", "2"][-3] == "0");
["0", "1", "2"][-4];
[00007061:error] !!! []: invalid index: -4
\end{urbiscript}

\item \lstinline|'[]='(\var{index}, \var{value})|\\
  Assign \var{value} to the element of the target at the given
  \var{index}.

\begin{urbiscript}[firstnumber=last]
var f = [0, 1, 2];
[00000000] [0, 1, 2]
f[1] = 42;
[00000000] 42
assert(f == [0, 42, 2]);
\end{urbiscript}

\item \lstinline|'*'(\var{n})|\\
  Return the target, concatenated \var{n} times to itself.
\begin{urbiassert}[firstnumber=last]
[0, 1] * 3 == [0, 1, 0, 1, 0, 1];
\end{urbiassert}

  Note that since it is the very same list which is repeatedly
  concatenated (the content is not cloned), side-effects on one item
  will reflect on ``all the items''.

\begin{urbiscript}[firstnumber=last]
var l = [[]] * 3;
[00000000] [[], [], []]
l[0] << 1;
[00000000] [1]
l;
[00000000] [[1], [1], [1]]
\end{urbiscript}

\item \lstinline|'+'(\var{other})|\\
Return the concatenation of the target and the \var{other} list.

\begin{urbiassert}[firstnumber=last]
[0, 1] + [2, 3] == [0, 1, 2, 3];
\end{urbiassert}

\item \lstinline|'-'(\var{other})|\\
Return the target without all element that equals any element in the
\var(other) list.

\begin{urbiassert}[firstnumber=last]
[0, 1, 0, 2, 3] - [1, 2] == [0, 0, 3];
\end{urbiassert}

\item \lstinline|'<<'(\var{that})|\\
  A synonym for \lstinline|insertBack|.

\item \lstinline|'<'(\var{other})|\\
  Return whether the target is inferior to the \var{other} list. A
  list is inferior to another if at least one of its element differs
  from the other, and the first differing element is inferior to the
  other.

\begin{urbiassert}[firstnumber=last]
!([0, 1, 2] < [0, 1, 2]);
!([0, 1, 2] < [0, 0, 2]);
[0, 1, 2] < [0, 2, 2];
\end{urbiassert}

  Since List derives from \refObject{Orderable}, the other order-based
  operators are defined.

\begin{urbiassert}[firstnumber=last]
 [0, 1, 2] <= [0, 1, 2];
 [0, 1, 2] >= [0, 1, 2];
 [0, 1, 2] >  [0, 0, 2];
\end{urbiassert}
\end{itemize}

%%% Local Variables:
%%% mode: latex
%%% TeX-master: "../urbi-sdk"
%%% End:

% LocalWords:  lst asList asString foldl hasSame removeBack popback removeFront
% LocalWords:  popfront insertBack pushback insertFront pushfront urbi sameAs
% LocalWords:  removeById setNth


\section{Lobby}
%% Copyright (C) 2009-2011, Gostai S.A.S.
%%
%% This software is provided "as is" without warranty of any kind,
%% either expressed or implied, including but not limited to the
%% implied warranties of fitness for a particular purpose.
%%
%% See the LICENSE file for more information.

\section{Object}

\refObject{Object} includes the mandatory primitives for all objects in \us.
All objects in \us must inherit (directly or indirectly) from it.

\subsection{Prototypes}

\begin{refObjects}
\item[Comparable]
\item[Global]
\end{refObjects}

\subsection{Construction}

A fresh object can be instantiated by cloning \slot{Object} itself.

\begin{urbiscript}[firstnumber=1]
Object.new;
[00000421] Object_0x00000000
\end{urbiscript}

The keyword \lstindex{class} also allows to define objects intended to serve
as prototype of a family of objects, similarly to classes in traditional
object-oriented programming languages (see \autoref{sec:tut:class}).

\begin{urbiscript}
{
  class Foo
  {
    var attr = 23;
  };
  assert
  {
    Foo.localSlotNames == ["asFoo", "attr", "type"];
    Foo.asFoo === Foo;
    Foo.attr == 23;
    Foo.type == "Foo";
  };
};
\end{urbiscript}


\subsection{Slots}

\begin{urbiscriptapi}
\item[acceptVoid]
  Return \this.  See \refObject{void} to know why.
\begin{urbiscript}
{
  var o = Object.new;
  assert(o.acceptVoid === o);
};
\end{urbiscript}


\item[addProto](<proto>)%
  Add \var{proto} into the list of prototypes of \this.  Return \this.
\begin{urbiscript}
do (Object.new)
{
  assert
  {
    addProto(Orderable) === this;
    protos == [Orderable, Object];
  };
}|;
\end{urbiscript}

\item[allProto]%
  A list with \this, its parents, their parents,\ldots
\begin{urbiassert}
123.allProtos.size == 12;
\end{urbiassert}

\item[allSlotNames]
  Deprecated alias for \refSlot{slotNames}.
\begin{urbiassert}
Object.allSlotNames == Object.slotNames;
\end{urbiassert}

\item[apply](<args>)%
  ``Invoke \this''.  The size of the argument list,
  \var{args}, must be one.  This argument is ignored.  This function
  exists for compatibility with \refSlot[Code]{apply}.
\begin{urbiassert}
Object.apply([this]) === Object;
Object.apply([1])    === Object;
\end{urbiassert}

\item[as](<type>)%
  Convert \this to \var{type}.  This is syntactic sugar for
  \lstinline|as\var{Type}| when \var{Type} is the \lstinline|type| of
  \var{type}.
\begin{urbiassert}
     12.as(Float) == 12;
   "12".as(Float) == 12;
    12.as(String) == "12";
Object.as(Object) === Object;
\end{urbiassert}

\item[asBool]
  Whether \this is ``true'', see \autoref{sec:truth}.
\begin{urbiscript}
assert
{
  Global.asBool == true;
  nil.asBool ==    false;
};
void.asBool;
[00000421:error] !!! unexpected void
\end{urbiscript}

\item[bounce](<name>)%
  Return \lstinline|this.\var{name}| transformed from a method into a
  function that takes its target (its ``\this'') as first
  and only argument.  \lstinline|this.\var{name}| must take no
  argument.
\begin{urbiassert}
{ var myCos = Object.bounce("cos"); myCos(0) }    == 0.cos;
{ var myType = bounce("type"); myType(Object); } == "Object";
{ var myType = bounce("type"); myType(3.14); }   == "Float";
\end{urbiassert}

\item[callMessage](<msg>)%
  Invoke the \refObject{CallMessage} \var{msg} on this.
%%% \begin{urbiscript}
%%% function f(var tgt, var msg, var args)
%%% {
%%%   call.target  = tgt;
%%%   call.message = msg;
%%%   call.code = tgt.getSlot(msg);
%%%   call.args    = args;
%%%   call.inspect;
%%%   tgt.callMessage(call);
%%% }|;
%%% assert
%%% {
%%%   f(Object, "type", []) == "Object.f(1, 2)";
%%%
%%% };
%%% \end{urbiscript}
\item[clone]
  Clone \this, i.e., create a fresh, empty, object, which
  sole prototype is \this.
\begin{urbiassert}
Object.clone.protos == [Object];
Object.clone.localSlotNames == [];
\end{urbiassert}

\item[cloneSlot](<from>, <to>)%
  Set the new slot \var{to} using a clone of \var{from}. This can only
  be used into the same object.

\begin{urbiscript}
var foo = Object.new |;
cloneSlot("foo", "bar") |;
assert(!(foo === bar));
\end{urbiscript}

\item[copySlot](<from>, <to>)%
  Same as \lstinline|cloneSlot|, but the slot aren't cloned, so the
  two slot are the same.
\begin{urbiscript}
var moo = Object.new |;
cloneSlot("moo", "loo") |;
assert(!(moo === loo));
\end{urbiscript}

\item[createSlot](<name>)%
  Create an empty slot (which actually means it is bound to
  \lstinline|void|) named \var{name}.  Raise an error if \var{name}
  was already defined.
\begin{urbiscript}
do (Object.new)
{
  assert(!hasLocalSlot("foo"));
  assert(createSlot("foo").isVoid);
  assert(hasLocalSlot("foo"));
}|;
\end{urbiscript}

\item[dump](<depth>)%
  Describe \this: its prototypes and slots.  The argument
  \var{depth} specifies how recursive the description is: the greater,
  the more detailed.  This method is mostly useful for debugging
  low-level issues, for a more human-readable interface, see also
  \refSlot{inspect}.
\begin{urbiscript}
do (2) { var this.attr = "foo"; this.attr->prop = "bar" }.dump(0);
[00015137] *** Float_0x240550 {
[00015137] ***   /* Special slots */
[00015137] ***   protos = Float
[00015137] ***   value = 2
[00015137] ***   /* Slots */
[00015137] ***   attr = String_0x23a750 <...>
[00015137] ***     /* Properties */
[00015137] ***     prop = String_0x23a7a0 <...>
[00015137] ***   }
do (2) { var this.attr = "foo"; this.attr->prop = "bar" }.dump(1);
[00020505] *** Float_0x240550 {
[00020505] ***   /* Special slots */
[00020505] ***   protos = Float
[00020505] ***   value = 2
[00020505] ***   /* Slots */
[00020505] ***   attr = String_0x23a750 {
[00020505] ***     /* Special slots */
[00020505] ***     protos = String
[00020505] ***     /* Slots */
[00020505] ***     }
[00020505] ***     /* Properties */
[00020505] ***     prop = String_0x239330 {
[00020505] ***       /* Special slots */
[00020505] ***       protos = String
[00020505] ***       /* Slots */
[00020505] ***       }
[00020505] ***   }
\end{urbiscript}

\item[getPeriod]
  Deprecated.  Use \refSlot[System]{period} instead.

\item[getProperty](<slotName>, <propName>)%
  The value of the \var{propName} property associated to the slot
  \var{slotName} if defined.  Raise an error otherwise.
\begin{urbiscript}
const var myPi = 3.14|;
assert (getProperty("myPi", "constant"));

getProperty("myPi", "foobar");
[00000045:error] !!! property lookup failed: myPi->foobar
\end{urbiscript}

\item[getLocalSlot](<name>)%
  The value associated to \var{name} in \this, excluding
  its ancestors (contrary to \lstinline|getSlot|).
\begin{urbiscript}
var a = Object.new|;

// Local slot.
var a.slot = 21|;
assert
{
  a.locateSlot("slot") === a;
  a.getLocalSlot("slot") == 21;
};

// Inherited slot are not looked-up.
assert { a.locateSlot("init") == Object };
a.getLocalSlot("init");
[00041066:error] !!! lookup failed: init
\end{urbiscript}

\item[getSlot](<name>)%
  The value associated to \var{name} in \this, possibly
  after a look-up in its prototypes (contrary to
  \lstinline|getLocalSlot|).
\begin{urbiscript}
var b = Object.new|;
var b.slot = 21|;

assert
{
  // Local slot.
  b.locateSlot("slot") === b;
  b.getSlot("slot") == 21;

  // Inherited slot.
  b.locateSlot("init") === Object;
  b.getSlot("init") == Object.getSlot("init");
};

// Unknown slot.
assert { b.locateSlot("ENOENT") == nil; };
b.getSlot("ENOENT");
[00041066:error] !!! lookup failed: ENOENT
\end{urbiscript}

\item[hash]%
  A \refObject{Hash} object for \this.  This default implementation returns
  a different hash for every object, so every key maps to a different
  cells. Classes that have value semantic should override the hash method so
  as objects that are equal (in the \refSlot[Object]{'=='} sense) have the
  same hash. \refSlot[String]{hash} does so for instance; as a consequence
  different String objects with the same value map to the same cell.

  A hash only makes sense as long as the hashed object exists.

\begin{urbiscript}
var o1 = Object.new|
var o2 = Object.new|
assert
{
  o1.hash == o1.hash;
  o1.hash != o2.hash;
};
\end{urbiscript}

\item[hasLocalSlot](<slot>)%
  Whether \this features a slot \var{slot}, locally (not from some
  ancestor).  See also \refSlot{hasSlot}.
\begin{urbiscript}
class Base         { var this.base = 23; } |;
class Derive: Base { var this.derive = 43 } |;
assert(Derive.hasLocalSlot("derive"));
assert(!Derive.hasLocalSlot("base"));
\end{urbiscript}

\item[hasProperty](<slotName>, <propName>)%
  Whether the slot \var{slotName} of \this has a property
  \var{propName}.
\begin{urbiscript}
const var halfPi = pi / 2|;
assert
{
  hasProperty("halfPi", "constant");
  !hasProperty("halfPi", "foobar");
};
\end{urbiscript}

\item[hasSlot](<slot>)%
  Whether \this has the slot \var{slot}, locally, or from
  some ancestor.  See also \refSlot{hasLocalSlot}.

\begin{urbiassert}
Derive.hasSlot("derive");
Derive.hasSlot("base");
!Base.hasSlot("derive");
\end{urbiassert}

\item['$id']% fix color $

\item[inspect](<deep> = false)%
  Describe \this: its prototypes and slots, and their
  properties.  If \var{deep}, all the slots are described, not only
  the local slots. See also \refSlot{dump}.
\begin{urbiscript}
do (2) { var this.attr = "foo"; this.attr->prop = "bar"}.inspect;
[00001227] *** Inspecting 2
[00001227] *** ** Prototypes:
[00001227] ***   0
[00001227] *** ** Local Slots:
[00001228] ***   attr : String
[00001228] ***     Properties:
[00001228] ***      prop : String = "bar"
\end{urbiscript}

\item[isA](<obj>)%
  Whether \this has \var{obj} in his parents.
\begin{urbiassert}
   Float.isA(Orderable);
! String.isA(Float);
\end{urbiassert}

\item[isNil]%
  Whether \this is \refObject{nil}.
\begin{urbiassert}
 nil.isNil;
!  0.isNil;
\end{urbiassert}

\item[isProto]
  Whether \this is a prototype.
\begin{urbiassert}
 Float.isProto;
!   42.isProto;
\end{urbiassert}

\item[isVoid]%
  Whether \this is \lstinline|void|.  See \refObject{void}.
\begin{urbiassert}
void.isVoid;
! 42.isVoid;
\end{urbiassert}

\item[localSlotNames]%
  A list with the names of the local (i.e., not including those of its
  ancestors) slots of \this.  See also \refSlot{slotNames}.
\begin{urbiscript}
var top = Object.new|;
var top.top1 = 1|;
var top.top2 = 2|;
var bot = top.new|;
var bot.bot1 = 10|;
var bot.bot2 = 20|;
assert
{
  top.localSlotNames == ["top1", "top2"];
  bot.localSlotNames == ["bot1", "bot2"];
};
\end{urbiscript}

\item[locateSlot](<slot>)%
  The \slot{Object} that provides \var{slot} to \this, or \lstinline|nil| if
  \this does not feature \var{slot}.
\begin{urbiassert}
locateSlot("locateSlot") == Object;
locateSlot("doesNotExist").isNil;
\end{urbiassert}

\item[print] Send \this to the \refSlot[Channel]{topLevel} channel.
\begin{urbiscript}
1.print;
[00001228] 1
[1, "12"].print;
[00001228] [1, "12"]
\end{urbiscript}

\item[protos]
  The list of prototypes of \this.
\begin{urbiassert}
12.protos == [Float];
\end{urbiassert}

\item[properties](<slotName>)%
  A dictionary of the properties of slot \var{slotName}.  Raise an error if
  the slot does not exist.
\begin{urbiscript}
2.properties("foo");
[00238495:error] !!! lookup failed: foo
do (2) { var foo = "foo" }.properties("foo");
[00238501] ["constant" => false]
do (2) { var foo = "foo" ; foo->bar = "bar" }.properties("foo");
[00238502] ["bar" => "bar", "constant" => false]
\end{urbiscript}

\item[removeLocalSlot](<slot>)%
  Remove \var{slot} from the (local) list of slots of \this, and return
  \this.  Raise an error if \var{slot} does not exist.  See also
  \refSlot{removeSlot}.
\begin{urbiscript}
var base = Object.new|;
var base.slot = "base"|;

var derive = Base.new|;
var derive.slot = "derive"|;

derive.removeLocalSlot("foo");
[00000080:error] !!! lookup failed: foo

assert
{
  derive.removeLocalSlot("slot") === derive;
  derive.localSlotNames == [];
  base.slot == "base";
};

derive.removeLocalSlot("slot");
[00000090:error] !!! lookup failed: slot

assert
{
  base.slot == "base";
};
\end{urbiscript}


\item[removeProperty](<slotName>, <propName>)%
  Remove the property \var{propName} from the slot \var{slotName}.  Raise an
  error if the slot does not exist.  Warn if \var{propName} does not exist;
  in a future release this will be an error.
\begin{urbiscript}
var r = Object.new|;

// Non-existing slot.
r.removeProperty("slot", "property");
[00000072:error] !!! lookup failed: slot

var r.slot = "slot value"|;
// Non-existing property.
r.removeProperty("slot", "property");
[00000081:warning] !!! no such property: slot->property
[00000081:warning] !!!    called from: removeProperty

r.slot->property = "property value"|;
assert
{
  r.hasProperty("slot", "property");
  // Existing property.
  r.removeProperty("slot", "property") == "property value";
  ! r.hasProperty("slot", "property");
};
\end{urbiscript}

\item[removeProto](<proto>)%
  Remove \var{proto} from the list of prototypes of \this, and return \this.
  Do nothing if \var{proto} is not a prototype of \this.
\begin{urbiscript}
do (Object.new)
{
  assert
  {
    addProto(Orderable);
    removeProto(123) === this;
    protos == [Orderable, Object];
    removeProto(Orderable) === this;
    protos == [Object];
  };
}|;
\end{urbiscript}

\item[removeSlot](<slot>)%
  Remove \var{slot} from the (local) list of slots of \this, and return
  \this.  Warn if \var{slot} does not exist; in a future release this will
  be an error.  See also \refSlot{removeLocalSlot}.
\begin{urbiscript}
{
  var base = Object.new;
  var base.slot = "base";

  var derive = Base.new;
  var derive.slot = "derive";

  assert
  {
    derive.removeSlot("foo") === derive;
[00000080:warning] !!! no such local slot: foo
[00000080:warning] !!!    called from: removeSlot
[00000080:warning] !!!    called from: code
[00000080:warning] !!!    called from: eval
[00000080:warning] !!!    called from: value
[00000080:warning] !!!    called from: assertCall

    derive.removeSlot("slot") === derive;
    derive.localSlotNames == [];
    base.slot == "base";
    derive.removeSlot("slot") === derive;
[00000099:warning] !!! no such local slot: slot
[00000099:warning] !!!    called from: removeSlot
[00000099:warning] !!!    called from: code
[00000099:warning] !!!    called from: eval
[00000099:warning] !!!    called from: value
[00000099:warning] !!!    called from: assertCall

    base.slot == "base";
  };
};
\end{urbiscript}


\item[setConstSlot]%
  Like \refSlot{setSlot} but the created slot is const.
\begin{urbiscript}
assert(setConstSlot("fortyTwo", 42) == 42);
fortyTwo = 51;
[00000000:error] !!! cannot modify const slot
\end{urbiscript}

\item[setProperty](<slotName>, <propName>, <value>)%
  Set the property \var{propName} of slot \var{slotName} to \var{value}.
  Raise an error in \var{slotName} does not exist.  Return \var{value}.
  This is what \lstinline|\var{slotName}->\var{propName} = \var{value}|
  actually performs.
\begin{urbiscript}
do (Object.new)
{
  var slot = "slot";
  var value = "value";
  assert
  {
    setProperty("slot", "prop", value) === value;
    "prop" in properties("slot");
    getProperty("slot", "prop") === value;
    slot->prop === value;
    setProperty("slot", "noSuchProperty", value) === value;
  };
}|;
setProperty("noSuchSlot", "prop", "12");
[00000081:error] !!! lookup failed: noSuchSlot
\end{urbiscript}


\item[setProtos](<protos>)%
  Set the list of prototypes of \this to \var{protos}.  Return
  \lstinline|void|.
\begin{urbiscript}
do (Object.new)
{
  assert
  {
    protos == [Object];
    setProtos([Orderable, Object]).isVoid;
    protos == [Orderable, Object];
  };
}|;
\end{urbiscript}

\item[setSlot](<name>, <value>)%
  Create a slot \var{name} mapping to \var{value}. Raise an error if
  \var{name} was already defined.  This is what
  \lstinline|var \var{name} = \var{value}| actually performs.
\begin{urbiassert}
Object.setSlot("theObject", Object) === Object;
Object.theObject === Object;
theObject === Object;
\end{urbiassert}

  If the current job is in redefinition mode, \lstinline|setSlot| on
  an already defined slot is not an error and overwrites the slot like
  \lstinline|updateSlot| would. See the \lstinline|redefinitionMode|
  method in \refObject{System}.

\item[slotNames]%
  A list with the slot names of \this and its ancestors.
\begin{urbiassert}
Object.localSlotNames
  .subset(Object.slotNames);
Object.protos.foldl(function (var r, var p) { r + p.localSlotNames },
                    [])
  .subset(Object.slotNames);
\end{urbiassert}

\item[type]%
  The name of the type of \this.  The \lstinline|class|
  construct defines this slot to the name of the class
  (\autoref{sec:tut:class}).  This is used to display the name of
  ``instances''.
\begin{urbiscript}
class Example {};
[00000081] Example
assert
{
  Example.type == "Example";
};
Example.new;
[00000081] Example_0x6fb2720
\end{urbiscript}

\item[uid]
  The unique id of \this.
\begin{urbiscript}
{
  var foo = Object.new;
  var bar = Object.new;
  assert
  {
    foo.uid == foo.uid;
    foo.uid != bar.uid;
  };
};
\end{urbiscript}

\item[unacceptVoid]%
  Return \this.  See \refObject{void} to know why.
\begin{urbiscript}
{
  var o = Object.new|
  assert(o.unacceptVoid === o);
};
\end{urbiscript}

%%% FIXME: \item[uobjectInit]
\item[updateSlot](<name>, <value>)%
  Map the existing slot named \var{name} to \var{value}. Raise an
  error if \var{name} was not defined.
\begin{urbiassert}
Object.setSlot("one", 1)    == 1;
Object.updateSlot("one", 2) == 2;
Object.one                  == 2;
\end{urbiassert}

\item['&&'](<that>)%
  Short-circuiting logical and. If \this evaluates to true evaluate and
  return \var{that}, otherwise return \this without evaluating \var{that}.
\begin{urbiassert}
(0 && "foo") == 0;
(2 && "foo") == "foo";

(""    && "foo") == "";
("foo" && "bar") == "bar";
\end{urbiassert}

\item['||'](<that>)%
  Short-circuiting logical or. If \this evaluates to false evaluate and
  return \var{that}, otherwise return \this without evaluating \var{that}.
\begin{urbiassert}
(0 || "foo") == "foo";
(2 ||  1/0)  == 2;

(""    || "foo") == "foo";
("foo" || 1/0)   == "foo";
\end{urbiassert}

\item \lstinline|'!'|\\
  Logical negation.  If \this evaluates to false return \lstinline|true| and
  vice-versa.
\begin{urbiassert}
!1 == false;
!0 == true;

!"foo" == false;
!""    == true;
\end{urbiassert}

\item['+='](<that>)%
  Bounce to \lstinline|this '+' \var{that}|.

\item['-='](<that>)%
  Bounce to \lstinline|this '-' \var{that}|.

\item['*='](<that>)%
  Bounce to \lstinline|this '*' \var{that}|.

\item['/='](<that>)%
  Bounce to \lstinline|this '/' \var{that}|.

\item['^='](<that>)%
  Bounce to \lstinline|this '^' \var{that}|.

\item \lstinline|'%='(\var{that})|\\
  Bounce to \lstinline|this '-' \var{that}|.

\item['=='](<that>)%
  Whether \this and \that are equal.  See also \refObject{Comparable} and
  \autoref{sec:lang:operators:comparison}.  By default, bounces to
  \refSlot{'==='}.  This operator \emph{must} be redefined for objects that
  have a value-semantics; for instance two \refObject{String} objects that
  denotes the same string should be equal according to \lstinline|==|,
  although physically different (i.e., not equal according to
  \lstinline|===|).
\begin{urbiscript}
{
  var o1 = Object.new;
  var o2 = Object.new;
  assert
  {
      o1 == o1;
    !(o1 == o2);
      o1 != o2;
    !(o1 != o1);

      1  ==  1;
     "1" == "1";
     [1] == [1];
  };
};
\end{urbiscript}

\item['==='](<that>)%
  Whether \this and \that are exactly the same object (i.e., \this and \that
  are two different means to denote the very same location in memory).  To
  denote equivalence, use \refSlot{'=='}; for instance two \refObject{Float}
  objects that denote 42 can be different objects (in the sense of
  \lstinline|===|), but will be considered equal by \lstinline|==|.  See
  also \refSlot{'==='} and \autoref{sec:lang:operators:comparison}.
\begin{urbiscript}
{
  var o1 = Object.new;
  var o2 = Object.new;
  assert
  {
      o1 === o1;
    !(o1 === o2);

    !( 1  ===  1 );
    !("1" === "1");
    !([1] === [1]);
  };
};
\end{urbiscript}

\item \lstinline|'!=='(\var{that})|\\%
  The negation of \lstinline|\this === \that|, see \refSlot{'==='}.
\begin{urbiscript}
{
  var o1 = Object.new;
  var o2 = Object.new;
  assert
  {
      o1 !== o2;
    !(o1 !== o1);

      1  !==  1;
     "1" !== "1";
     [1] !== [1];
  };
};
\end{urbiscript}

\end{urbiscriptapi}

%%% Local Variables:
%%% coding: utf-8
%%% mode: latex
%%% TeX-master: "../urbi-sdk"
%%% ispell-dictionary: "american"
%%% ispell-personal-dictionary: "../urbi.dict"
%%% fill-column: 76
%%% End:

\section{Orderable}
\section{Pair}
\section{Pattern}
\section{Primitive}
\section{String}

\section{System}

\section{System.Platform}

A description of the platform (the computer) the server is running on.
\subsection{Slots}

\subsubsection{kind}
Either \code{"POSIX"} or \code{"WIN32"}.

\section{Tag}
\section{Tuple}
\section{void}


%% Restore the definition of \section.
\let\section\sectionOrig

%%% Local Variables:
%%% mode: latex
%%% TeX-master: "urbi-specs"
%%% End:

% LocalWords:  CallMessage memoization callmsg lst eval xADDR createSlot POSIX
% LocalWords:  getSlot setSlot updateSlot Orderable Tuple
