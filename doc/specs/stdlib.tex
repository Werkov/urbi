%% Redefine \section is this chapter so that we don't have to
%% call \labelObject each time.  See the bottom of this file for the
%% restoring of \section.
\let\sectionOrig\section
\renewcommand{\section}[1]{\sectionOrig{\labelObject{#1}\index{#1@\lstinline{#1}}#1}}

\chapter{\us Standard Library}
\label{sec:stdlib}

\section{Boolean}

Booleans does not exist as a real type in \us. On the other hand, all
objects in \us can be evaluated as Boolean value especially the two
specific object \lstinline|true| and \lstinline|false|.

\subsection{Prototypes}

As above, the prototype Boolean does not exists, but \lstinline|true|
and \lstinline|false| have the following prototype.

\begin{itemize}
\item \refObject{Singleton}
\end{itemize}

\subsection{Construction}

Actually you don't want to construct booleans, you just want to know
if a condition is true or false. So the object \lstinline|true| and
\lstinline|false| are the result of all comparison statement.

\begin{urbiscript}
true;
[00000000] true
false;
[00000000] false
2 < 6;
[00000000] true
var x = true.new;
[00000000] true
x === true;
[00000000] true
\end{urbiscript}

As you can see, when you try to clone \lstinline|true| or
\lstinline|false|, you get the object itself and not a copy.

%%% Local Variables:
%%% mode: latex
%%% TeX-master: "../urbi-sdk"
%%% End:

\section{CallMessage}
Capturing a method invocation: its target and arguments.

\subsection{Examples}
\subsubsection{Evaluating an argument several times}
\label{sec:std-callmsg-examples-several}

The following example implements a lazy function which takes an
integer \var{n}, then arguments.  The \var{n}-th argument is evaluated
twice using \lstinline|CallMessage.evalArgAt|.

\begin{urbiscript}
function callTwice
{
  var n = call.evalArgAt(0);
  call.evalArgAt(n);
  call.evalArgAt(n)
} |;

// Call twice echo("foo").
callTwice(1, echo("foo"), echo("bar"));
[00000001] *** foo
[00000002] *** foo

// Call twice echo("bar").
callTwice(2, echo("foo"), echo("bar"));
[00000003] *** bar
[00000004] *** bar
\end{urbiscript}


\subsubsection{Strict Functions}

Strict functions do support \lstinline|call|.

\begin{urbiscript}[firstnumber=last]
function strict(x)
{
  echo("Entering");
  echo("Strict: " + x);
  echo("Lazy:   " + call.evalArgAt(0));
} |;

strict({echo("1"); 1});
[00000011] *** 1
[00000013] *** Entering
[00000012] *** Strict: 1
[00000013] *** 1
[00000014] *** Lazy:   1
\end{urbiscript}


\subsection{Slots}

\begin{itemize}
\item \lstinline|args|\\
  The list of unevaluated arguments.
\begin{urbiscript}[firstnumber=last]
function args { call.args }|
assert
{
  args == [];
  args() == [];
  args({echo(111); 1}) == [Lazy.new(closure() {echo(111); 1})];
  args(1, 2) == [Lazy.new(closure () {1}),
                 Lazy.new(closure () {2})];
};
\end{urbiscript}


\item \lstinline|argsCount|\\
  Return the number of arguments.  Do not evaluate them.
\begin{urbiscript}[firstnumber=last]
function argsCount { call.argsCount }|;
assert
{
  argsCount == 0;
  argsCount() == 0;
  argsCount({echo(1); 1}) == 1;
  argsCount({echo(1); 1}, {echo(2); 2}) == 2;
};
\end{urbiscript}

\item \lstinline|evalArgAt(\var{n})|\\
  Evaluate the \var{n}-th argument, and return its value.  \var{n}
  must evaluate to an nonnegative integer.  Repeated invocations
  repeat the evaluation, see
  \autoref{sec:std-callmsg-examples-several}.
\begin{urbiscript}[firstnumber=last]
function sumTwice
{
  var n = call.evalArgAt(0);
  call.evalArgAt(n) + call.evalArgAt(n)
}|;

function one () { echo("one"); 1 }|;

sumTwice(1, one, one + one);
[00000008] *** one
[00000009] *** one
[00000010] 2
sumTwice(2, one, one + one);
[00000011] *** one
[00000012] *** one
[00000011] *** one
[00000012] *** one
[00000013] 4

sumTwice(3, one, one);
[00000014:error] !!! evalArgAt: invalid index: 3
sumTwice(3.14, one, one);
[00000015:error] !!! evalArgAt: invalid index: 3.14
\end{urbiscript}

\item \lstinline|evalArgs|\\
  Call \lstinline|evalArgAt| for each argument, return the list of
  values.
\begin{urbiscript}[firstnumber=last]
function twice
{
  call.evalArgs + call.evalArgs
}|;
twice({echo(1); 1}, {echo(2); 2});
[00000011] *** 1
[00000012] *** 2
[00000011] *** 1
[00000012] *** 2
[00000013] [1, 2, 1, 2]
\end{urbiscript}
\end{itemize}


%%% Local Variables:
%%% mode: latex
%%% TeX-master: "../urbi-sdk"
%%% End:

%% Copyright (C) 2009-2010, Gostai S.A.S.
%%
%% This software is provided "as is" without warranty of any kind,
%% either expressed or implied, including but not limited to the
%% implied warranties of fitness for a particular purpose.
%%
%% See the LICENSE file for more information.

\section{Channel}
Returning data, typically asynchronous, with a label so that the
``caller'' can find it in the flow.

\subsection{Prototypes}

\begin{itemize}
\item \refObject{Object}
\end{itemize}

\subsection{Construction}

Channels are created like any other object. The constructor must be
called with a string which will be the label.

\begin{urbiscript}[firstnumber=1]
var ch1 = Channel.new("my_label");
[00000201] Channel_0x7985810

ch1 << 1;
[00000201:my_label] 1

var ch2 = ch1;
[00000201] Channel_0x7985810

ch2 << 1/2;
[00000201:my_label] 0.5
\end{urbiscript}

\subsection{Slots}

\begin{urbiscriptapi}
\item \lstinline|'<<'(\var{value})|\\
  Send \var{value} to \lstinline|this| tagged by its label if non-empty.

\begin{urbiscript}
Channel.new("label") << 42;
[00000000:label] 42

Channel.new("") << 51;
[00000000] 51
\end{urbiscript}

\item[echo](<value>)%
  Same as \lstinline|System.echo(\var{value}, name)|.

\begin{urbiscript}
Channel.new("label").echo(42);
[00000000:label] *** 42

Channel.new("").echo("Foo");
[00000000] *** Foo
\end{urbiscript}

\item[enabled] Whether the Channel is enabled.  Disabled Channels
  produce no output.
\begin{urbiscript}
var c = Channel.new("")|;

c << "enabled";
[00000000] "enabled"

c.enabled = false|;
c << "disabled";

c.enabled = true|;
c << "enabled";
[00000000] "enabled"
\end{urbiscript}

\item[quote] Whether the strings are output escaped (the default)
  instead of raw strings.
\begin{urbiscript}
var d = Channel.new("")|;

assert(d.enabled);
d << "A \"String\"";
[00000000] "A \"String\""

d.quote = false|;
d << "A \"String\"";
[00000000] A "String"
\end{urbiscript}

\item[name] The name of the Channel, used to label the output.
\begin{urbiscript}
assert
{
  Channel.new("").name == "";
  Channel.new("foo").name == "foo";
};
\end{urbiscript}

\item[null] A predefined stream whose \lstinline|enabled| is
  \lstinline|false|.
\begin{urbiscript}
Channel.null << "Message";
\end{urbiscript}


\item[topLevel] A predefined stream for regular output.  Strings are
  output escaped.
\begin{urbiscript}
Channel.topLevel << "Message";
[00015895] "Message"
Channel.topLevel << "\"quote\"";
[00015895] "\"quote\""
\end{urbiscript}

\item[warning] A predefined stream for warning messages.  Strings sent
  to it are not escaped.
\begin{urbiscript}
Channel.warning << "Message";
[00015895:warning] Message
Channel.warning << "\"quote\"";
[00015895:warning] "quote"
\end{urbiscript}
\end{urbiscriptapi}

%%% Local Variables:
%%% mode: latex
%%% TeX-master: "../urbi-sdk"
%%% ispell-dictionary: "american"
%%% ispell-personal-dictionary: "../urbi.dict"
%%% fill-column: 76
%%% End:

%% Copyright (C) 2009-2010, Gostai S.A.S.
%%
%% This software is provided "as is" without warranty of any kind,
%% either expressed or implied, including but not limited to the
%% implied warranties of fitness for a particular purpose.
%%
%% See the LICENSE file for more information.

\section{Code}

Functions written in \us.

\subsection{Prototypes}

\begin{refObjects}
\item[Comparable]
\item[Executable]
\end{refObjects}

\subsection{Construction}

The keywords \lstinline|function| and \lstinline|closure| build Code
instances.

\begin{urbiassert}
function(){}.protos[0] === getSlot("Code");
closure(){}.protos[0] === getSlot("Code");
\end{urbiassert}

\subsection{Slots}

\begin{urbiscriptapi}
\item \lstinline|==(\var{that})|\\
  Whether \lstinline|this| and \var{that} are the same source code.
  It actually checks that both have the same \lstinline|asString|.
\begin{urbiassert}
function () { 1 } == function () { 1 };
function () { 1 } != closure  () { 1 };
closure  () { 1 } != function () { 1 };

function () { 1 + 1 } == function () { 1 + 1 };
function () { 1 + 2 } != function () { 2 + 1 };

function () { 1 } != function { 1 };
function () { 1 } != function (ignored) { 1 };
\end{urbiassert}

\item \lstinline|apply(\var{args})|\\
  Invoke the routine, with all the arguments.  The target,
  \lstinline|this|, will be set to \lstinline|\var{args}[0]| and the
  remaining arguments with be given as arguments.
\begin{urbiassert}
function (x, y) { x+y }.apply([nil, 10, 20]) == 30;
function () { this }.apply([123]) == 123;

// There is Object.apply.
1.apply([this]) == 1;
\end{urbiassert}
\begin{urbiscript}
function () {}.apply([]);
[00000001:error] !!! apply: list of arguments must begin with `this'

function () {}.apply([1, 2]);
[00000002:error] !!! apply: expected 0 argument, given 1
\end{urbiscript}

\item[asString]
  Conversion to \refObject{String}.
\begin{urbiassert}
closure  () { 1 }.asString == "closure () {\n  1\n}";
function () { 1 }.asString == "function () {\n  1\n}";
\end{urbiassert}

\item[bodyString]
  Conversion to \refObject{String} of the routine body.
\begin{urbiassert}
closure  () { 1 }.bodyString == "1";
function () { 1 }.bodyString == "1";
\end{urbiassert}

\end{urbiscriptapi}

%%% Local Variables:
%%% mode: latex
%%% TeX-master: "../urbi-sdk"
%%% ispell-dictionary: "american"
%%% ispell-personal-dictionary: "../urbi.dict"
%%% fill-column: 76
%%% End:

%% Copyright (C) 2009-2010, Gostai S.A.S.
%%
%% This software is provided "as is" without warranty of any kind,
%% either expressed or implied, including but not limited to the
%% implied warranties of fitness for a particular purpose.
%%
%% See the LICENSE file for more information.

\section{Comparable}
Objects that can be compared for equality and inequality.  See also
\refObject{Orderable}.

This object, made to serve as prototype, provides a definition of
\lstinline{!=} based on \lstinline{==}.  \lstinline{Object} provides
a default implementation of \lstinline{==} that bounces on the physical
equality \lstinline{===}.

\begin{urbiscript}[firstnumber=1]
class Foo : Comparable
{
  var value = 0;
  function init (v) { value = v; };
  function '==' (lhs) { value == lhs.value; };
};
[00000000] Foo
Foo.new(1) == Foo.new(1);
[00000000] true
Foo.new(1) == Foo.new(2);
[00000000] false
\end{urbiscript}

\subsection{Slots}

\begin{urbiscriptapi}
\item[==](<that>)
  Whether \lstinline|! (this != that)|.
\begin{urbiscript}
class FortyTwo : Comparable
{
  function '!=' (that) { 42 != that };
}|;
assert
{
  FortyTwo != 51;
  FortyTwo == 42;
};
\end{urbiscript}


\item[!=](<that>)
  Whether \lstinline|! (this == that)|.

\begin{urbiscript}
class FiftyOne : Comparable
{
  function '==' (that) { 51 == that };
}|;
assert
{
  FiftyOne == 51;
  FiftyOne != 42;
};
\end{urbiscript}
\end{urbiscriptapi}

%%% Local Variables:
%%% mode: latex
%%% TeX-master: "../urbi-sdk"
%%% ispell-dictionary: "american"
%%% ispell-personal-dictionary: "../urbi.dict"
%%% fill-column: 76
%%% End:

%% Copyright (C) 2009-2011, Gostai S.A.S.
%%
%% This software is provided "as is" without warranty of any kind,
%% either expressed or implied, including but not limited to the
%% implied warranties of fitness for a particular purpose.
%%
%% See the LICENSE file for more information.

\section{Dictionary}

A \dfn{dictionary} is an \dfn{associative array}, also known as a \dfn{hash}
in some programming languages.  They are arrays whose indexes are arbitrary
objects.

\subsection{Example}

The following session demonstrates the features of the Dictionary objects.

\begin{urbiscript}[firstnumber=1]
var d = ["one" => 1, "two" => 2];
[00000001] ["one" => 1, "two" => 2]

for (var p : d)
  echo (p.first + " => " + p.second);
[00000003] *** one => 1
[00000002] *** two => 2

"three" in d;
[00000004] false
d["three"];
[00000005:error] !!! missing key: three
d["three"] = d["one"] + d["two"]|;
"three" in d;
[00000006] true
d.getWithDefault("four", 4);
[00000007] 4
\end{urbiscript}

\subsection{Hash values}
\label{sec:dictionary:hash}

Arbitrary objects can be used as dictionary keys. To map to the same cell,
two objects used as keys must have equal hashes (retrieved with the
\refSlot[Object]{hash} method) and be equal to each other (in the
\refSlot[Object]{'=='} sense).

This means that two different objects may have the same hash: the equality
operator (\refSlot[Object]{'=='}) is checked in addition to the hash, to
handle such collision.  However a good hash algorithm should avoid this
case, since it hinders performances.

See \refSlot[Object]{hash} for more detail on how to override hash
values. Most standard value-based classes implement a reasonable hash
function: see \refSlot[Float]{hash}, \refSlot[String]{hash},
\refSlot[List]{hash}, \ldots

\subsection{Prototypes}

\begin{refObjects}
\item[Comparable]
\item[Container]
\item[Object]
\item[RangeIterable]
\end{refObjects}

\subsection{Construction}

The Dictionary constructor takes arguments by pair (key, value).

\begin{urbiscript}
Dictionary.new("one", 1, "two", 2);
[00000000] ["one" => 1, "two" => 2]
Dictionary.new;
[00000000] [ => ]
\end{urbiscript}

There must be an even number of arguments.

\begin{urbiscript}
Dictionary.new("1", 2, "3");
[00000001:error] !!! new: odd number of arguments
\end{urbiscript}

You are encouraged to use the specific syntax for Dictionary literals:

\begin{urbiscript}
["one" => 1, "two" => 2];
[00000000] ["one" => 1, "two" => 2]
[=>];
[00000000] [ => ]
\end{urbiscript}

An extra comma can be added at the end of the list.

\begin{urbiscript}
[
  "one" => 1,
  "two" => 2,
];
[00000000] ["one" => 1, "two" => 2]
\end{urbiscript}

It is guaranteed that the pairs to insert are evaluated left-to-write, key
first, the value.

\begin{urbiassert}
   ["a".fresh => "b".fresh, "c".fresh => "d".fresh]
== ["a_5"     => "b_6",     "c_7"     => "d_8"];
\end{urbiassert}

\subsection{Slots}

\begin{urbiscriptapi}
\item['=='](<that>)%
  Whether \this equals \var{that}.  Expects members to be
  \refObject{Comparable}.
\begin{urbiassert}
[ => ] == [ => ];
["a" => 1, "b" => 2] == ["b" => 2, "a" => 1];
\end{urbiassert}


\item|'[]'|(<key>)%
  Syntactic sugar for \lstinline|get(\var{key})|.

\begin{urbiscript}
assert (["one" => 1]["one"] == 1);
["one" => 1]["two"];
[00000012:error] !!! missing key: two
\end{urbiscript}


\item|'[]='|(<key>, <value>)%
  Syntactic sugar for \lstinline|set(\var{key}, \var{value})|, but returns
  \var{value}.

\begin{urbiassert}
var d = ["one" =>"2"];
(d["one"] = 1) == 1;
d["one"] == 1;
\end{urbiassert}


\item[asBool]
  Negation of \refSlot{empty}.
\begin{urbiassert}
[=>].asBool == false;
["key" => "value"].asBool == true;
\end{urbiassert}


\item[asList]%
  The contents of the dictionary as a \refObject{Pair} list (\var{key},
  \var{value}).

\begin{urbiassert}
["one" => 1, "two" => 2].asList == [("one", 1), ("two", 2)];
\end{urbiassert}

  \noindent
  Since Dictionary derives from \refObject{RangeIterable}, it is easy
  to iterate over a Dictionary using a range-\lstinline|for|
  (\autoref{sec:lang:for:each}).  No particular order is ensured.
\begin{urbiscript}
{
  var res = [];
  for| (var entry: ["one" => 1, "two" => 2])
    res << entry.second;
  assert(res == [1, 2]);
};
\end{urbiscript}


\item[asString] A string representing the dictionary.  There is no guarantee
  on the order of the output.
\begin{urbiassert}
                [=>].asString == "[ => ]";
["a" => 1, "b" => 2].asString == "[\"a\" => 1, \"b\" => 2]";
\end{urbiassert}

\item[elementAdded] An event emitted each time a new element is added to
  the Dictionary.

\item[elementChanged] An event emitted each time the value associated to a
  key of the Dictionary is changed.

\item[elementRemoved] An event emitted each time an element is removed from
  the Dictionary.

\begin{urbiscript}
d = [ => ] |;
at(d.elementAdded?) echo ("added");
at(d.elementChanged?) echo ("changed");
at(d.elementRemoved?) echo ("removed");

d["key1"] = "value1";
[00000001] "value1"
[00000001] *** added

d["key2"] = "value2";
[00000001] "value2"
[00000001] *** added

d["key2"] = "value3";
[00000001] "value3"
[00000001] *** changed

d.erase("key2");
[00000002] ["key1" => "value1"]
[00000001] *** removed

d.clear;
[00000003] [ => ]
[00000001] *** removed

d.clear;
[00000003] [ => ]
\end{urbiscript}

\item[clear]
  Empty the dictionary.

\begin{urbiassert}
["one" => 1].clear.empty;
\end{urbiassert}


\item[empty]
  Whether the dictionary is empty.

\begin{urbiassert}
[=>].empty == true;
["key" => "value"].empty == false;
\end{urbiassert}


\item[erase](<key>) Remove the mapping for \var{key}.
\begin{urbicomment}
removeSlot("d")|;
\end{urbicomment}
\begin{urbiscript}
{
  var d = ["one" => 1, "two" => 2];
  assert
  {
    d.erase("two") === d;
    d == ["one" => 1];
  };

  try
  {
    ["one" => 1, "two" => 2].erase("three");
    echo("never reached");
  }
  catch (var e if e.isA(Dictionary.KeyError))
  {
    assert(e.key == "three")
  };
};
\end{urbiscript}

%% commented until a consensus is reached.
%%
%% \item[extend](<ext>)
%%   Extend with the dictionary \var{ext}.
%%   Return the value of the new dictionary.
%% \begin{urbiscript}
%% d = ["one" => 1, "two" => 2];
%% [00000001] ["one" => 1, "two" => 2]
%% d.extend(["one" => 0, "three" => 3]);
%% [00000002] ["one" => 0, "three" => 3, "two" => 2]
%% \end{urbiscript}

\item[get](<key>)%
  The value associated to \var{key}.  A \lstinline|Dictionary.KeyError|
  exception is thrown if the key is missing.
  % FIXME: the following exception test should be rewritten when (if)
  % we introduce the throw assertion.
\begin{urbiscript}
var d = ["one" => 1, "two" => 2]|;

assert(d.get("one") == 1);
["one" => 1, "two" => 2].get("three");
[00000010:error] !!! missing key: three

try
{
  d.get("three");
  echo("never reached");
}
catch (var e if e.isA(Dictionary.KeyError))
{
  assert(e.key == "three")
};
\end{urbiscript}


\item[getWithDefault](<key>, <defaultValue>)%
  The value associated to \var{key} if it exists, \var{defaultValue}
  otherwise.

\begin{urbiassert}
var d = ["one" => 1, "two" => 2];
d.getWithDefault("one",  -1) == 1;
d.getWithDefault("three", 3) == 3;
\end{urbiassert}


\item[has](<key>)%
  Whether the dictionary has a mapping for \var{key}.

\begin{urbiassert}
var d = ["one" => 1];
d.has("one");
!d.has("zero");
\end{urbiassert}

  The infix operators \lstinline|in| and \lstinline|not in| use
  \lstinline|has| (see \autoref{sec:lang:op:containers}).

\begin{urbiassert}
"one" in     ["one" => 1];
"two" not in ["one" => 1];
\end{urbiassert}


\item[init](<key1>, <value1>, ...)%
  Insert the mapping from \var{key1} to \var{value1} and so forth.

\begin{urbiscript}
Dictionary.clone.init("one", 1, "two", 2);
[00000000] ["one" => 1, "two" => 2]
\end{urbiscript}


\item[keys]%
  The list of all the keys.  No particular order is ensured.  Since
  \refObject{List} features the same function, uniform iteration over
  a List or a Dictionary is possible.
\begin{urbiassert}
var d = ["one" => 1, "two" => 2];
d.keys == ["one", "two"];
\end{urbiassert}


\item[matchAgainst](<handler>, <pattern>)
  Pattern matching on members.  See \refObject{Pattern}.

\begin{urbiscript}
{
  // Match a subset of the dictionary.
  ["a" => var a] = ["a" => 1, "b" => 2];
  // get the matched value.
  assert(a == 1);
};
\end{urbiscript}


\item[set](<key>, <value>)%
  Map \var{key} to \var{value} and return \this so that invocations to
  \refSlot{set} can be chained.  The possibly existing previous mapping is
  overridden.

\begin{urbiscript}
[=>].set("one", 2)
    .set("two", 2)
    .set("one", 1);
[00000000] ["one" => 1, "two" => 2]
\end{urbiscript}


\item[size]
  Number of element in the dictionary.

\begin{urbiassert}
var d = [=>];  d.size == 0;
d["a"] = 10;   d.size == 1;
d["b"] = 20;   d.size == 2;
d["a"] = 30;   d.size == 2;
\end{urbiassert}



\end{urbiscriptapi}


%%% Local Variables:
%%% mode: latex
%%% TeX-master: "../urbi-sdk"
%%% ispell-dictionary: "american"
%%% ispell-personal-dictionary: "../urbi.dict"
%%% fill-column: 76
%%% End:

%% Copyright (C) 2009-2011, Gostai S.A.S.
%%
%% This software is provided "as is" without warranty of any kind,
%% either expressed or implied, including but not limited to the
%% implied warranties of fitness for a particular purpose.
%%
%% See the LICENSE file for more information.

\section{Event}

An \dfn{event} can be ``emitted'' and ``caught'', or ``sent'' and
``received''.  See also \autoref{sec:tut:events}.

\subsection{Examples}

There are several examples of uses of events in the documentation of
event-based constructs.  See \lstinline{at} (\autoref{sec:lang:at}),
\lstinline{waituntil} (\autoref{sec:lang:waituntil}), \lstinline{whenever}
(\autoref{sec:lang:whenever}), and so forth.  The tutorial chapter about
event-based programming contains other examples, see
\autoref{sec:tut:event-prog}.

\subsection{Synchronicity of Event Handling}
\label{sec:event:sync}
A particular emphasis should be put on the \dfn{synchronicity} of the event
handling, i.e., whether the bodies of the event handlers are run before the
control flow returns from the event emission.  By default, (i.e.,
\lstinline|at (e?...)| and \lstinline|e!(...)|/\lstinline|e.emit(...)|)) the
execution is \dfn{asynchronous}, but if either the emitted or the handler is
marked asynchronous (i.e., \lstinline|at sync (e?...)| or
\lstinline|e.syncEmit(...)|), then the execution is \dfn{synchronous}.

Contrast the following examples:

\begin{multicols}{2}
\paragraph{Asynchronous handlers}~

\begin{urbiscript}[xrightmargin=0mm,xleftmargin=0mm]
var e = Event.new |;

at (e?)
  { echo("a"); sleep(20ms); echo("b") }
onleave
  { echo("c"); sleep(20ms); echo("d") };

e! | echo("done");
[00000001] *** done
sleep(25ms);
[00000002] *** a
[00000003] *** c
[00000101] *** b
[00000102] *** d

e.syncEmit | echo("done");
[00000001] *** a
[00000101] *** b
[00000102] *** c
[00000202] *** d
[00000203] *** done
\end{urbiscript}
\columnbreak

\paragraph{Synchronous handlers}~

\begin{urbicomment}
removeSlots("e");
\end{urbicomment}
\begin{urbiscript}[xrightmargin=0mm,xleftmargin=0mm]
var e = Event.new |;

at sync (e?)
  { echo("a"); sleep(20ms); echo("b") }
onleave
  { echo("c"); sleep(20ms); echo("d") };

e! | echo("done");
// No need to sleep.
[00000011] *** a
[00000031] *** b
[00000031] *** c
[00000052] *** d
[00000052] *** done

e.syncEmit | echo("done");
[00000052] *** a
[00000073] *** b
[00000073] *** c
[00000094] *** d
[00000094] *** done
\end{urbiscript}
\end{multicols}

For more information about the synchronicity of event handlers, see
\autoref{sec:lang:at:sync-async}.

\subsection{Sustained Events}
\label{sec:event:sustain}

Events can also be sustained during a time span starting at
\refSlot{trigger} and ending at \lstinline|handler.stop|.  Note that the
\lstinline|onleave|-clauses of the event handlers is not executed right
after the event was first triggered, but rather when it is stopped.

Synchronicity for sustained events is more complex: the
\lstinline|at|-clause is handled asynchronously iff \emph{both} the emission
and the handler are asynchronous, whereas the \lstinline|onleave|-clause is
handled asynchronously iff the emission was synchronous.  Be warned, but do
not depend on this, as in the future we might change this.

\begin{multicols}{2}
\paragraph{Asynchronous Trigger}~

\begin{urbicomment}
removeSlots("e");
\end{urbicomment}
\begin{urbiscript}[xrightmargin=0mm,xleftmargin=0mm]
var e = Event.new|;
at (e?(var v))
  { echo("a"+v); sleep(20ms); echo("b"+v) }
onleave
  { echo("c"+v); sleep(20ms); echo("d"+v) };

var handler = e.trigger("1") | echo("?");
[00000001] *** ?
[00000002] *** a1
[00000102] *** b1
sleep(200ms);
handler.stop | echo("?");
[00000301] *** ?
sleep(25ms);
[00000302] *** c1
[00000402] *** d1

// at and onleave clauses may overlap.
handler = e.trigger("2") | handler.stop;
sleep(25ms);
[00000001] *** a2
[00000002] *** c2
sleep(25ms);
[00000201] *** b2
[00000202] *** d2

handler = e.syncTrigger("3") | echo("?");
[00000002] *** a3
[00000102] *** b3
[00000001] *** ?
handler.stop | echo("?");
[00000302] *** c3
[00000402] *** d3
[00000301] *** ?
\end{urbiscript}
\columnbreak

\paragraph{Synchronous Trigger}~

\begin{urbicomment}
removeSlots("e", "handler");
\end{urbicomment}
\begin{urbiscript}[xrightmargin=0mm,xleftmargin=0mm]
var e = Event.new|;
at sync (e?(var v))
  { echo("a"+v); sleep(20ms); echo("b"+v) }
onleave
  { echo("c"+v); sleep(20ms); echo("d"+v) };

var handler = e.trigger("1") | echo("?");
// No need to sleep.
[00000002] *** a1
[00000102] *** b1
[00000001] *** ?
handler.stop | echo("?");
[00000301] *** ?
sleep(25ms);
[00000302] *** c1
[00000402] *** d1

// at and onleave clauses don't overlap.
handler = e.trigger("2") | handler.stop;
sleep(25ms);
[00000001] *** a2
[00000201] *** b2
[00000002] *** c2
[00000202] *** d2

handler = e.syncTrigger("3") | echo("?");
[00000002] *** a3
[00000102] *** b3
[00000001] *** ?
handler.stop | echo("?");
[00000302] *** c3
[00000402] *** d3
[00000301] *** ?
\end{urbiscript}
\end{multicols}


\subsection{Prototypes}
\begin{refObjects}
\item[Object]
\end{refObjects}

\subsection{Construction}
\label{sec:stdlib:event:ctor}

An \lstinline{Event} is created like any other object.  The constructor
takes no argument.

\begin{urbiscript}[firstnumber=1]
var e = Event.new;
[00000001] Event_0x9ad8118
\end{urbiscript}

\subsection{Slots}
\begin{urbiscriptapi}
\item[asEvent]
  Return \this.

\item[emit](<args>[])%
  Fire an ``instantaneous'' and ``asynchronous'' \refObject{Event}. This
  function is called by the \lstinline|!| operator.  It takes any number of
  arguments, passed to the receiver when the event is caught.
\begin{urbicomment}
removeSlots("e");
\end{urbicomment}
\begin{urbiscript}
var e = Event.new|;
// No handler, lost message.
e.emit;
at (e?)               echo("e");
at (e?())             echo("e()");
at (e?(var x))        echo("e(%s)" % [x]);
at (e?(var x, var y)) echo("e(%s, %s)" % [x, y]);

// This is what e! does.
e.emit;
[00000135] *** e
[00000135] *** e()

// This is what e!() does: same as e!.
e.emit();
[00000138] *** e
[00000138] *** e()

// This is what e!(1, [2]) does.
e.emit(1, [2]);
[00000141] *** e
[00000141] *** e(1, [2])

// This is what e!(1, [2], "three") does.
e.emit(1, [2], "three");
[00000146] *** e
\end{urbiscript}

To sustain an event, see \refSlot{trigger}.  See \autoref{sec:event:sync}
and \refSlot{syncEmit} for details about the synchronicity of the handling.

  %% An event can also be emitted for a certain duration using
  %% \lstinline|~|.  The execution of \lstinline|at| clauses etc., is
  %% asynchronous: the control flow might be released by the
  %% \lstinline|emit| call before all the watchers have finished their
  %% execution.

\item[onEvent](<guard>, <enter>, <leave>, <sync>)%
  This is the low-level routine used to implement the \lstindex|at|
  construct.  Indeed,
  \lstinline|at (\var{e}? if \var{cond}) \var{enter} onleave \var{leave}|
  is (roughly) translated into
\begin{urbiunchecked}
\var{e}
  .onEvent(
    closure (var '$evt', var '$payload')                 { \var{cond}  },
    closure (var '$evt', var '$payload', var '$pattern') { \var{enter} },
    closure (var '$evt', var '$payload', var '$pattern') { \var{leave} },
    false)
\end{urbiunchecked}

\noindent
where the \lstinline|false| would be \lstinline|true| in case of an
\lstinline|at sync| construct.  The \var{cond} discards the event iff it
returns \lstinline{void}.

\begin{urbicomment}
removeSlots("e");
\end{urbicomment}
\begin{urbiscript}
var e = Event.new|;
e.onEvent(
  function (var args[]) { echo("cond 1") | true },
  function (var args[]) { echo("enter 1") },
  function (var args[]) { echo("leave 1") },
  true);

e.onEvent(
  function (var args[]) { echo("cond 2") },
  function (var args[]) { echo("enter 2") },
  function (var args[]) { echo("leave 2") },
  true);

e.emit(12);
[00001619] *** cond 1
[00001619] *** enter 1
[00001619] *** leave 1
[00001619] *** cond 2

var h = e.trigger|;
[00001620] *** cond 1
[00001620] *** enter 1
[00001620] *** cond 2

h.stop;
[00001621] *** leave 1
\end{urbiscript}

This function is internal and it might change in the future.

\item[onSubscribe]%
  This slot is not set by default. You can optionally assign an event to
  it. In this case, it is triggered each time some code starts watching this
  event (by setting up an \lstinline|at| or a \lstinline|waituntil| on it
  for instance).

  Throw a synchronized event. This call awaits that all functions that have
  to react to this event have returned.  This function can have the same
  arguments as \refSlot{emit}.

\item[syncEmit](<args>[])%
  Same as \refSlot{emit} but require a synchronous handling.  See
  \autoref{sec:event:sync} for details.

\item[syncTrigger](<args>[])%
  Same as \refSlot{trigger} but the call will be synchronous (see
  \autoref{sec:event:sync}). The \lstinline|stop| method of the handler
  object will be synchronous as well.  See \autoref{sec:event:sustain} for
  examples.

\item[trigger](<args>[])%
  Fire a sustained event (for an unknown amount of time) and return a
  handler object whose \lstinline|stop| method stops the event. This method
  is asynchronous and the \lstinline|stop| call will be asynchronous as
  well.  See \autoref{sec:event:sustain} for examples.

\item \lstinline+'||'(\var{that})+%
  Logical ``or'' on events: a new Event that triggers whenever \this or
  \that triggers.

\begin{urbiscript}
var e1 = Event.new|;
var e2 = Event.new|;
var e_or = e1 || e2|;
at (e_or?)
  echo("!");
e1!;
[00000004] *** !
e2!;
[00000005] *** !
\end{urbiscript}

\item['<<'](<that>)%
  Watch a \that event status and reproduce it on itself, return \this.  This
  operator is similar to an optimized \lstinline,||=, operator.  Do not make
  events watch for themselves, directly or indirectly.

\begin{urbiscript}
var e3 = Event.new|;
var e4 = Event.new|;
var e_watch = Event.new << e3 << e4 |;
at (e_watch?)
  echo("!");
e3!;
[00000006] *** !
e4!;
[00000007] *** !
\end{urbiscript}


\end{urbiscriptapi}

%%% Local Variables:
%%% mode: latex
%%% TeX-master: "../urbi-sdk"
%%% ispell-dictionary: "american"
%%% ispell-personal-dictionary: "../urbi.dict"
%%% fill-column: 76
%%% End:

\section{Float}

A Float is a floating point number.  It is also used, in the current
version of \us, to represent integers.

\subsection{Prototypes}

\begin{refObjects}
\item[Comparable]
\item[Orderable]
\item[RangeIterable]
\end{refObjects}

\subsection{Construction}
\label{sec:float:ctor}

The most common way to create fresh floats is using the literal
syntax.  Numbers are composed of three parts:
\begin{description}
\item[integral] (mandatory) a non empty sequence of (decimal) digits;
\item[fractional] (optional) a period, and a non empty sequence of
  (decimal) digits;
\item[exponent] (optional) either \samp{e} or \samp{E}, an optional
  sign (\samp{+} or \samp{-}), then a non-empty sequence of digits.
\end{description}

In other words, float literals match the
\lstinline|[0-9]+(\.[0-9]+)?([eE][-+]?[0-9]+)?|
regular expression.  For instance:

\begin{urbiassert}
0 == 0000.0000;
// This is actually a call to the unary '+'.
+1 == 1;
0.123456 == 123456 / 1000000;
1e3 == 1000;
1e-3 == 0.001;
1.234e3 == 1234;
\end{urbiassert}

There are also some special numbers, \lstinline|nan|, \lstinline|inf|
(see below).

\begin{urbiassert}
Math.log(0) == -inf;
Math.exp(-inf) == 0;
(inf/inf).asString == "nan";
\end{urbiassert}

A null float can also be obtained with \lstinline|Float|'s
\lstinline|new| method.

\begin{urbiassert}
Float.new == 0;
\end{urbiassert}

\subsection{Slots}

\begin{urbiscriptapi}
\item[abs]
  Absolute value of the target.
\begin{urbiassert}
(-5).abs == 5;
  0 .abs == 0;
  5 .abs == 5;
\end{urbiassert}

\item[acos]
  Arccosine of the target.
\begin{urbiassert}
0.acos == Float.pi/2;
1.acos == 0;
\end{urbiassert}

\item[asBool]
  Whether non null.
\begin{urbiassert}
0.asBool == false;
0.1.asBool == true;
(-0.1).asBool == true;
inf.asBool == true;
nan.asBool == true;
\end{urbiassert}

\item[asFloat]
  Return the target.
\begin{urbiassert}
51.asFloat == 51;
\end{urbiassert}

\item[asList]
  Bounces to \lstinline|seq|.  It is therefore possible to use the
  various flavors of \lstinline|for|-range loops on integers:
\begin{urbiassert}
{
  var res = [];
  for (var i : 3)
    res << i;
  res
}
== [0, 1, 2];

{
  var res = [];
  for|(var i : 3)
    res << i;
  res
}
== [0, 1, 2];

{
  var res = [];
  for&(var i : 3)
    res << i;
  res.sort
}
== [0, 1, 2];
\end{urbiassert}%>>

\item[asin]
  Arcsine of the target.
\begin{urbiassert}
0.asin == 0;
\end{urbiassert}

\item[asString]
  Return a string representing the target.
\begin{urbiassert}
42.asString == "42";
\end{urbiassert}

\item[atan]
  Return the arctangent of the target.
\begin{urbiassert}
0.atan == 0;
1.atan == Float.pi/4;
\end{urbiassert}

\item \lstinline|'bitand'(\var{that})|\\
  The bitwise-and between \lstinline|this| and \var{that}.
\begin{urbiassert}
(3 bitand 6) == 2;
\end{urbiassert}

\item \lstinline|'bitor'(\var{that})|\\
  Bitwise-or between \lstinline|this| and \var{that}.
\begin{urbiassert}
(3 bitor 6) == 7;
\end{urbiassert}

\item[clone]
  Return a fresh Float with the same value as the target.
\begin{urbiscript}
var x = 0;
[00000000] 0
var y = x.clone;
[00000000] 0
x === y;
[00000000] false
\end{urbiscript}

\item[compl]
  The complement to 1 of the target interpreted as a 32 bits integer.
\begin{urbiassert}
compl 0 == 4294967295;
compl 4294967295 == 0;
\end{urbiassert}

\item[cos]
  Cosine of the target.
\begin{urbiassert}
0.cos == 1;
Float.pi.cos == -1;
\end{urbiassert}

\item \lstinline|each(\var{fun})|\\
  Call the functional argument \var{fun} on every integer from 0 to
  target - 1, sequentially.  The number must be non-negative.
\begin{urbiassert}
{
  var res = [];
  3.each(function (i) { res << 100 + i });
  res
}
== [100, 101, 102];

{
  var res = [];
  for(var x : 3) { res << x; sleep(20ms); res << (100 + x); };
  res
}
== [0, 100, 1, 101, 2, 102];

{
  var res = [];
  0.each (function (i) { res << 100 + i });
  res
}
== [];
\end{urbiassert}

\item \lstinline'each|(\var{fun})'\\
  Call the functional argument \var{fun} on every integer from 0 to
  target - 1, with tight sequentiality.  The number must be
  non-negative.
\begin{urbiassert}
{
  var res = [];
  3.'each|'(function (i) { res << 100 + i });
  res
}
== [100, 101, 102];

{
  var res = [];
  for|(var x : 3) { res << x; sleep(20ms); res << (100 + x); };
  res
}
== [0, 100, 1, 101, 2, 102];
\end{urbiassert}%>>>>>>

\item \lstinline|each&(\var{fun})|\\
  Call the functional argument \var{fun} on every integer from 0 to
  target - 1, concurrently.  The number must be non-negative.
\begin{urbiassert}
{
  var res = [];
  for& (var x : 3) { res << x; sleep(30ms); res << (100 + x) };
  res
}
== [0, 1, 2, 100, 101, 102];
\end{urbiassert}%>>>>

\item[exp]
  Exponential of the target.
\begin{urbiscript}
1.exp;
[00000000] 2.71828
\end{urbiscript}

\item \lstinline|format(\var{finfo})|\\
  Format according to the \refObject{FormatInfo} object \var{finfo}.
  The precision, \lstinline|\var{finfo}.precision|, sets the maximum
  number of digits after decimal point when in fixed or scientific
  mode, and in total when in default mode.  Beware that 0 plays a
  special role, as it is not a ``significant'' digit.

  \begin{windows}
    Under Windows the behavior differs slightly.
  \end{windows}
\begin{urbiassert}
"%1.0d" % 0.1 == "0.1";
"%1.0d" % 1.1 == {if (System.Platform.isWindows) "1.1" else "1"};

"%1.0f" % 0.1 == "0";
"%1.0f" % 1.1 == "1";
\end{urbiassert}

\item[inf]
  Return the infinity.
\begin{urbiscript}
Float.inf;
[00000000] inf
\end{urbiscript}

\item[limit_digits]
  Number of digits (in \lstinline|Float.limit_radix| base) in the
  mantissa.
\begin{urbiassert}
Float.limit_digits;
\end{urbiassert}

\item[limit_digits10]
  Number of digits (in decimal base) that can be represented without
  change.
\begin{urbiassert}
Float.limit_digits10;
\end{urbiassert}

\item[limit_epsilon]
  Machine epsilon (the difference between 1 and the least value
  greater than 1 that is representable).
\begin{urbiassert}
1 != 1 + Float.limit_epsilon;
1 == 1 + Float.limit_epsilon / 2;
\end{urbiassert}

\item[limit_max]
  Maximum finite value.
\begin{urbiassert}
Float.limit_max     != Float.inf;
Float.limit_max * 2 == Float.inf;
\end{urbiassert}

\item[limit_max_exponent]
  Maximum integer value for the exponent that generates a normalized
  floating-point number.
\begin{urbiassert}
Float.inf != Float.limit_radix ** (Float.limit_max_exponent - 1);
Float.inf == Float.limit_radix ** Float.limit_max_exponent;
\end{urbiassert}

\item[limit_max_exponent10]
  Maximum integer value such that 10 raised to that power generates a
  normalized finite floating-point number.
\begin{urbiassert}
Float.inf != 10 ** Float.limit_max_exponent10;
Float.inf == 10 ** (Float.limit_max_exponent10 + 1);
\end{urbiassert}

\item[limit_min]
  Minimum positive normalized value.
\begin{urbiassert}
0 != Float.limit_min;
\end{urbiassert}

\item[limit_min_exponent]
  Minimum negative integer value for the exponent that generates a
  normalized floating-point number.
\begin{urbiassert}
0 != Float.limit_radix ** Float.limit_min_exponent;
\end{urbiassert}

\item[limit_min_exponent10]
  Minimum negative integer value such that 10 raised to that power
  generates a normalized floating-point number.
\begin{urbiassert}
0 != 10 ** Float.limit_min_exponent10;
\end{urbiassert}

\item[limit_radix]
  Base of the exponent of the representation.
\begin{urbiassert}
Float.limit_radix == 2;
\end{urbiassert}

\item[log]
  The logarithm of the target.
\begin{urbiassert}
0.log == -inf;
1.log == 0;
1.exp.log == 1;
\end{urbiassert}

\item \lstinline|max(\var{arg1}, ...)|\\
  Bounces to \lstinline|List.max| on \lstinline|[this, \var{arg1}, ...]|.
\begin{urbiassert}
1.max == 1;
1.max(2, 3) == 3;
3.max(1, 2) == 3;
\end{urbiassert}

\item \lstinline|min(\var{arg1}, ...)|\\
  Bounces to \lstinline|List.min| on \lstinline|[this, \var{arg1}, ...]|.
\begin{urbiassert}
1.min == 1;
1.min(2, 3) == 1;
3.min(1, 2) == 1;
\end{urbiassert}

\item[nan]
  The ``not a number'' special float value.  More precisely, this
  returns the ``quiet NaN'', i.e., it is propagated in the various
  computations, it does not raise exceptions.
\begin{urbiscript}
Float.nan;
[00000000] nan
(Float.nan + Float.nan) / (Float.nan - Float.nan);
[00000000] nan
\end{urbiscript}

A {NaN} has one distinctive property over the other Floats: it is
equal to no other float, not even itself.  This behavior is mandated
by the \wref[IEEE_754-2008]{IEEE 754-2008} standard.
\begin{urbiassert}
{ var n = Float.nan; n === n};
{ var n = Float.nan; n  != n};
\end{urbiassert}

\item[pi]
  $\pi$.
\begin{urbiassert}
Float.pi.cos ** 2 + Float.pi.sin ** 2 == 1;
\end{urbiassert}

\item[random]
  A random integer between 0 (included) and the target (excluded).
\begin{urbiscript}
20.map(function (dummy) { 5.random });
[00000000] [1, 2, 1, 3, 2, 3, 2, 2, 4, 4, 4, 1, 0, 0, 0, 3, 2, 4, 3, 2]
\end{urbiscript}

\item[round]
  The target, rounded to the nearest integer.
\begin{urbiassert}
1.6.round == 2;
1.4.round == 1;
\end{urbiassert}

\item[seq]
  The sequence of integers from 0 to \lstinline|this| - 1 as a list.
  The number must be non-negative.
\begin{urbiassert}
3.seq == [0, 1, 2];
0.seq == [];
(-1).seq;
[00004586:error] !!! seq: expected non-negative integer, got -1
\end{urbiassert}

\item[sign]
  Return 1 if \lstinline|this| is positive, 0 if it is null, -1
  otherwise.
\begin{urbiassert}
(-1164).sign == -1;
0.sign       == 0;
(1164).sign  == 1;
\end{urbiassert}

\item[sin]
  The sine of the target.
\begin{urbiassert}
0.sin == 0;
\end{urbiassert}

\item[sqr]
  Square of the target.
\begin{urbiassert}
32.sqr == 1024;
32.sqr == 32 ** 2;
\end{urbiassert}

\item[sqrt]
  The square root of the target.
\begin{urbiassert}
1024.sqrt == 32;
1024.sqrt == 1024 ** 0.5;
\end{urbiassert}

\item[srandom]
  Initialized the seed used by the random function.  As opposed to common
  usage, you should not use
\begin{urbiunchecked}
{
  var now = Date.now.timestamp;
  now.srandom;
  var list1 = 20.map(function (dummy) { 5.random });
  now.srandom;
  var list2 = 20.map(function (dummy) { 5.random });
  assert
  {
    list1 == list2;
  }
};
\end{urbiunchecked}

\item[tan]
  Tangent of the target.
\begin{urbiscript}
assert(0.tan == 0);
(Float.pi/4).tan;
[00000000] 1
\end{urbiscript}

\item \lstinline|times(\var{fun})|\\
  Call the functional argument \var{fun} \lstinline|this| times.

\begin{urbiscript}
3.times(function () { echo("ping")});
[00000000] *** ping
[00000000] *** ping
[00000000] *** ping
\end{urbiscript}

\item[trunc]
  Return the target truncated.
\begin{urbiassert}
1.9.trunc == 1;
(-1.9).trunc == -1;
\end{urbiassert}

\item \lstinline|'^'(\var{that})|\\
  Bitwise exclusive or between \lstinline|this| and \var{that}.
\begin{urbiassert}
(3 ^ 6) == 5;
\end{urbiassert}

\item \lstinline|'>>'(\var{that})|\\%>>
  \lstinline|this| shifted by \var{that} bits towards the right.
\begin{urbiassert}
4 >> 2 == 1;
\end{urbiassert}

\item \lstinline|'<'(\var{that})|\\
  Whether \lstinline|this| is less than \var{b}. The other comparison
  operators (\lstinline|<=|, \lstinline|>|, \ldots) can thus also be
  applied on floats since Float inherits \refObject{Orderable}.
\begin{urbiassert}
  0 < 1;
!(1 < 0);
\end{urbiassert}

\item \lstinline|'<<'(\var{that})|\\
  \lstinline|this| shifted by \var{that} bit towards the left.
\begin{urbiassert}
4 << 2 == 16;
\end{urbiassert}

\item \lstinline|'-'(\var{that})|\\
  \lstinline|this| subtracted by \var{b}.
\begin{urbiassert}
6 - 3 == 3;
\end{urbiassert}

\item \lstinline|'+'(\var{that})|\\
  The sum of \lstinline|this| and \var{that}.
\begin{urbiassert}
1 + 1 == 2;
\end{urbiassert}

\item \lstinline|'/'(\var{that})|\\
  The quotient of \lstinline|this| divided by \var{that}.
\begin{urbiassert}
50 / 10 == 5;
10 / 50 == 0.2;
\end{urbiassert}

\item \lstinline|'%'(\var{that})|\\
  \lstinline|this| modulo \var{b}.
\begin{urbiassert}
50 % 11 == 6;
\end{urbiassert}

\item \lstinline|'*'(\var{that})|\\
  Product of \lstinline|this| by \var{that}.
\begin{urbiassert}
2 * 3 == 6;
\end{urbiassert}

\item \lstinline|'**'(\var{that})|\\
  \lstinline|this| to the \var{that} power (${this}^{that}$).
\begin{urbiassert}
2 ** 10 == 1024;
\end{urbiassert}

\item \lstinline|'=='(\var{that})|\\
  Whether \lstinline|this| equals \var{that}.
\begin{urbiassert}
  1 == 1;
!(1 == 2);
\end{urbiassert}
\end{urbiscriptapi}

%%% Local Variables:
%%% mode: latex
%%% TeX-master: "../urbi-sdk"
%%% ispell-dictionary: "american"
%%% ispell-personal-dictionary: "../urbi.dict"
%%% End:

%% Copyright (C) 2009-2011, Gostai S.A.S.
%%
%% This software is provided "as is" without warranty of any kind,
%% either expressed or implied, including but not limited to the
%% implied warranties of fitness for a particular purpose.
%%
%% See the LICENSE file for more information.

\section{Group}
A transparent means to send messages to several objects as if they
were one.

\subsection{Example}

The following session demonstrates the features of the Group
objects.  It first creates the \lstinline|Sample| family of object,
makes a group of such object, and uses that group.

\begin{urbiscript}[firstnumber=1]
class Sample
{
  var value = 0;
  function init(v)    { value = v; };
  function asString() { "<" + value.asString + ">"; };
  function timesTen() { new(value * 10); };
  function plusTwo()  { new(value + 2); };
};
[00000000] <0>

var group = Group.new(Sample.new(1), Sample.new(2));
[00000000] Group [<1>, <2>]
group << Sample.new(3);
[00000000] Group [<1>, <2>, <3>]
group.timesTen.plusTwo;
[00000000] Group [<12>, <22>, <32>]

// Bouncing getSlot and updateSlot.
group.value;
[00000000] Group [1, 2, 3]
group.value = 10;
[00000000] Group [10, 10, 10]

// Bouncing to each&.
var sum = 0|
for& (var v : group)
  sum += v.value;
sum;
[00000000] 30
\end{urbiscript}

\subsection{Prototypes}

\begin{refObjects}
\item[RangeIterable]
\item[Comparable]
\end{refObjects}

\subsection{Construction}

Groups are created like any other object. The constructor can take members
to add to the group.

\begin{urbiscript}
Group.new;
[00000000] Group []
Group.new(1, "two");
[00000000] Group [1, "two"]
\end{urbiscript}

\subsection{Slots}

\begin{urbiscriptapi}
\item['<<'](<member>)%
  Syntactic sugar for \refSlot{add}.


\item['=='](<that>)%
  Whether \lstinline|this.members == \var{that}.members|.
\begin{urbiassert}
               Group.new == Group.new;
Group.new(1, [2], "foo") == Group.new(1, [2], "foo");
         Group.new(1, 2) != Group.new(2, 1);
            Group.new(1) != Group.new(2);
\end{urbiassert}


\item[add](<member>, ...)%
  Add members to \this group, and return \this.
\begin{urbiassert}
var g = Group.new(1, 2);
g.add(3, 4) === g;
g.members == [1, 2, 3, 4];
\end{urbiassert}


\item[asString]
  Report the \lstinline|asString| of the members.
\begin{urbiassert}
Group.new(1, 2).asString == "Group [1, 2]";
\end{urbiassert}


\item[each](<action>)%
  Apply \var{action} to all the members, in sequence, then return the
  Group of the results, in the same order.  Allows to iterate over a
  Group via \lstinline|for|.


\item['each&'](<action>)%
  Apply \var{action} to all the members, concurrently, then return the
  Group of the results.  The order is \emph{not} necessarily the same.
  Allows to iterate over a Group via \lstinline|for&|.


\item[fallback]
  This function is called when a method call on \this
  failed.  It bounces the call to the members of the group, collects
  the results returned as a group.  This allows to chain grouped
  operation in a row.  If the dispatched calls return
  \lstinline|void|, returns a single \lstinline|void|, not a ``group
  of \lstinline|void|''.


\item[getProperty](<slot>, <prop>)%
  Bounced to the members so that \lstinline|this.\var{slot}->\var{prop}|
  actually collects the values of the property \var{prop} of the slots
  \var{slot} of the group members.  See \refSlot[Object]{getProperty}.
\begin{urbiscript}
class C
{
  var val = 0;
}|;

var a = C.new|; var b = C.new|;
var g = Group.new << a << b|;

g.val->prop;
[00010640:error] !!! property lookup failed: val->prop

a.val->prop = 42|;
g.val->prop;
[00010640:error] !!! property lookup failed: val->prop

b.val->prop = 51|;
assert
{
  g.val->prop == Group.new(42, 51);
};
\end{urbiscript}
\begin{urbicomment}
  removeSlots("C", "a", "b", "g");
\end{urbicomment}


\item[hasProperty](<slot>, <prop>)%
  Bounce to the members and return a group for the results.  See
  \refSlot[Object]{hasProperty}.
\begin{urbiscript}
class C
{
  var val = 0;
}|;

var a = C.new|; var b = C.new|;
var g = Group.new << a << b|;

assert
{
  g.hasProperty("val", "prop") == Group.new(false, false);
  a.val->prop = 21;
  g.hasProperty("val", "prop") == Group.new(true, false);
  b.val->prop = 42;
  g.hasProperty("val", "prop") == Group.new(true, true);
};
\end{urbiscript}
\begin{urbicomment}
  removeSlots("C", "a", "b", "g");
\end{urbicomment}


\item[hasSlot](<name>)%
  True if and only if all the members have the slot.

\begin{urbiassert}
var g = Group.new(1, 2);

!g.hasSlot("foo");
 g.hasSlot("+");
 g + 1 == Group.new(2, 3);
\end{urbiassert}


\item[remove](<member>, ...)%
  Remove members from \this group, and return \this.  Non-existing members
  are silently ignored.
\begin{urbiassert}
var g = Group.new(1, 2, 1);
g.remove(1, 3) === g == Group.new(2);
g.remove(2)    === g == Group.new;
\end{urbiassert}


\item[setProperty](<slot>, <prop>, <value>)%
  Bounced to the members so that
  \lstinline|this.\var{slot}->\var{prop} = \var{value}| actually updates the
  value of the property \var{prop} in the slots \var{slot} of the group
  members, and return a group for the collected result.  See
  \refSlot[Object]{setProperty}.

\begin{urbiscript}
class C
{
  var val = 0;
}|;

var g = Group.new << C.new << C.new|;

assert
{
  (g.val->prop = 31) == Group.new(31, 31);
};
\end{urbiscript}
\begin{urbicomment}
  removeSlots("C", "g");
\end{urbicomment}


\item[updateSlot](<name>, <value>)%
  Bounced to the members so that
  \lstinline|this.\var{name} = \var{value}|
  actually updates the value of the slot \var{name} in
  the group members.
\end{urbiscriptapi}

%%% Local Variables:
%%% mode: latex
%%% TeX-master: "../urbi-sdk"
%%% ispell-dictionary: "american"
%%% ispell-personal-dictionary: "../urbi.dict"
%%% fill-column: 76
%%% End:

%% Copyright (C) 2009-2011, Gostai S.A.S.
%%
%% This software is provided "as is" without warranty of any kind,
%% either expressed or implied, including but not limited to the
%% implied warranties of fitness for a particular purpose.
%%
%% See the LICENSE file for more information.

\section{Lazy}

\dfn{Lazies} are objects that hold a lazy value, that is, a not yet
evaluated value. They provide facilities to evaluate their content only once
(\dfn{memoization}) or several times. Lazy are essentially used in call
messages, to represent lazy arguments, as described in
\autorefObject{CallMessage}.

\subsection{Examples}

\subsubsection{Evaluating once}

One usage of lazy values is to avoid evaluating an expression unless it's
actually needed, because it's expensive or has undesired side effects. The
listing below presents a situation where an expensive-to-compute value
(\lstinline|heavy_computation|) might be needed zero, one or two times. The
objective is to save time by:

\begin{itemize}
\item Not evaluating it if it's not needed.
\item Evaluating it only once if it's needed once or twice.
\end{itemize}

We thus make the wanted expression lazy, and use the \lstinline|value|
method to fetch its value when needed.

\begin{urbiscript}[firstnumber=1]
// This function supposedly performs expensive computations.
function heavy_computation()
{
  echo("Heavy computation");
  return 1 + 1;
}|;

// We want to do the heavy computations only if needed,
// and make it a lazy value to be able to evaluate it "on demand".
var v = Lazy.new(closure () { heavy_computation() });
[00000000] heavy_computation()
/* some code */;
// So far, the value was not needed, and heavy_computation
// was not evaluated.
/* some code */;
// If the value is needed, heavy_computation is evaluated.
v.value();
[00000000] *** Heavy computation
[00000000] 2
// If the value is needed a second time, heavy_computation
// is not reevaluated.
v.value();
[00000000] 2
\end{urbiscript}

\subsubsection{Evaluating several times}

Evaluating a lazy several times only makes sense with lazy arguments and
call messages. See example with call messages in
\autoref{sec:std-callmsg-examples-several}.


\subsection{Caching}

\refObject{Lazy} is meant for functions without argument.  If you need
\dfn{caching} for functions that depend on arguments, it is straightforward
to implement using a \refObject{Dictionary}.  In the future \us might
support dictionaries whose indices are not only strings, but in the
meanwhile, convert the arguments into strings, as the following sample
object demonstrates.

\begin{urbiscript}
class UnaryLazy
{
  function init(f)
  {
    results = [ => ];
    func = f;
  };
  function value(p)
  {
    var sp = p.asString();
    if (results.has(sp))
      return results[sp];
    var res = func(p);
    results[sp] = res |
    res
  };
  var results;
  var func;
} |
// The function to cache.
var inc = function(x) { echo("incing " + x) | x+1 } |
// The function with cache. UnaryLazy simply takes the function as argument.
var p = UnaryLazy.new(inc);
[00062847] UnaryLazy_0x78b750
p.value(1);
[00066758] *** incing 1
[00066759] 2
p.value(1);
[00069058] 2
p.value(2);
[00071558] *** incing 2
[00071559] 3
p.value(2);
[00072762] 3
p.value(1);
[00074562] 2
\end{urbiscript}

\subsection{Prototypes}

\begin{refObjects}
\item[Comparable]
\end{refObjects}

\subsection{Construction}

Lazies are seldom instantiated manually. They are mainly created
automatically when a lazy function call is made (see
\autoref{sec:lang:call}). One can however create a lazy value with the
standard \lstinline|new| method of \lstinline|Lazy|, giving it an
argument-less function which evaluates to the value made lazy.

\begin{urbiscript}
Lazy.new(closure () { /* Value to make lazy */ 0 });
[00000000] 0
\end{urbiscript}

\subsection{Slots}

\begin{urbiscriptapi}
\item['=='](<that>)%
  Whether \this and \var{that} are the same source code and value (an not
  yet evaluated Lazy is never equal to an evaluated one).
\begin{urbiassert}
Lazy.new(closure () { 1 + 1 }) == Lazy.new(closure () { 1 + 1 });
Lazy.new(closure () { 1 + 2 }) != Lazy.new(closure () { 2 + 1 });
\end{urbiassert}
\begin{urbiscript}
{
  var l1 = Lazy.new(closure () { 1 + 1 });
  var l2 = Lazy.new(closure () { 1 + 1 });
  assert (l1 == l2);
  l1.eval();
  assert (l1 != l2);
  l2.eval();
  assert (l1 == l2);
};
\end{urbiscript}


\item[asString]
  The conversion to \refObject{String} of the body of a non-evaluated
  argument.
\begin{urbiassert}
Lazy.new(closure () { echo(1); 1 }).asString() == "echo(1);\n1";
\end{urbiassert}


\item[eval]%
  Force the evaluation of the held lazy value. Two calls to \refSlot{eval}
  will systematically evaluate the expression twice, which can be useful to
  duplicate its side effects.


\item[value]%
  Return the held value, potentially evaluating it before. \refSlot{value}
  performs memoization, that is, only the first call will actually evaluate
  the expression, subsequent calls will return the cached value. Unless you
  want to explicitly trigger side effects from the expression by evaluating
  it several time, this should be preferred over \lstinline|eval| to avoid
  evaluating the expression several times uselessly.
\end{urbiscriptapi}


%%% Local Variables:
%%% coding: utf-8
%%% mode: latex
%%% TeX-master: "../urbi-sdk"
%%% ispell-dictionary: "american"
%%% ispell-personal-dictionary: "../urbi.dict"
%%% fill-column: 76
%%% End:

\section{List}

\lstinline|List|s implement potentially-empty ordered (heterogeneous)
collections of elements.

\subsection{Prototypes}

\begin{itemize}
\item \refObject{Object}
\item \refObject{RangeIterable}
\item \refObject{Orderable}
\end{itemize}

\subsection{Construction}

List can be created with their literal syntax: a possibly empty
sequence of expressions in square brackets, separated by commas.
Non-empty list may actually \emph{terminate} with a comma, rather than
\emph{separate}; in other words, an optional trailing comma is accepted.

\begin{urbiscript}
[]; // The empty list
[00000000] []
[1, "2", [3,],];
[00000000] [1, "2", [3]]
\end{urbiscript}

\subsection{Slots}

\begin{itemize}
\item \lstinline|all(\var{fun})|\\
  % FIXME: link to predicate glossary entry
  Return whether all the members of the target verify the predicate
  \var{fun}.

\begin{urbiassert}[firstnumber=last]
// Are all elements positive?
! [-2, 0, 2, 4].all(function (e) { e > 0 });
// Are all elements even?
[-2, 0, 2, 4].all(function (e) { e % 2 == 0 });
\end{urbiassert}

\item \lstinline|any(\var{fun})|\\
  % FIXME: link to predicate glossary entry
  Whether at least one of the members of the target verifies the
  predicate \var{fun}.

\begin{urbiassert}[firstnumber=last]
// Is there any even element?
! [-3, 1, -1].any(function (e) { e % 2 == 0 });
// Is there any positive element?
[-3, 1, -1].any(function (e) { e > 0 });
\end{urbiassert}

\item \lstinline|asBool|\\
  Whether not empty.
\begin{urbiassert}[firstnumber=last]
[].asBool == false;
[1].asBool == true;
\end{urbiassert}

\item \lstinline|asList|\\
Return the target.

\begin{urbiassert}[firstnumber=last]
[0, 1, 2].asList == [0, 1, 2];
\end{urbiassert}

\item \lstinline|asString|\\
  A string describing the list.  Uses \lstinline|asPrintable| on its
  members, so that, for instance, strings are displayed with quotes.

\begin{urbiassert}[firstnumber=last]
[0, [1], "2"].asString == "[0, [1], \"2\"]";
\end{urbiassert}

\item \lstinline|back|\\
Return the last element of the target. An error if the target is empty.

\begin{urbiscript}[firstnumber=last]
assert([0, 1, 2].back == 2);
[].back;
[00000000:error] !!! back: cannot be applied onto empty list
\end{urbiscript}

\item \lstinline|clear|\\
  Empty the target.

\begin{urbiscript}[firstnumber=last]
var x = [0, 1, 2];
[00000000] [0, 1, 2]
assert(x.clear == []);
\end{urbiscript}

\item \lstinline|each(\var{fun})|\\
  Apply the given functional value \var{fun} on all members,
  sequentially.

\begin{urbiscript}[firstnumber=last]
[0, 1, 2].each(function (v) {echo (v * v); echo (v * v)});
[00000000] *** 0
[00000000] *** 0
[00000000] *** 1
[00000000] *** 1
[00000000] *** 4
[00000000] *** 4
\end{urbiscript}

\item \lstinline|'each&'(\var{fun})|\\
Apply the given functional value on all members simultaneously.

\begin{urbiscript}[firstnumber=last]
[0, 1, 2].'each&'(function (v) {echo (v * v); echo (v * v)});
[00000000] *** 0
[00000000] *** 1
[00000000] *** 4
[00000000] *** 0
[00000000] *** 1
[00000000] *** 4
\end{urbiscript}

\item \lstinline|empty|\\
  Whether the target is empty.

\begin{urbiassert}[firstnumber=last]
[].empty;
! [1].empty;
\end{urbiassert}

\item \lstinline|filter(\var{fun})|\\
  The list of all the members of the target that verify the predicate
  \var{fun}.

\begin{urbiassert}[firstnumber=last]
// Keep only odd numbers.
[0, 1, 2, 3, 4, 5].filter(function (v) {v % 2 == 1}) == [1, 3, 5];
\end{urbiassert}

\item \lstinline|foldl(\var{action}, \var{value})|\\
  \wref[Fold_(higher-order_function)]{Fold},
  also known as \dfn{reduce} or \dfn{accumulate}, computes a result
  from a list.  Starting from \var{value} as the initial result, apply
  repeatedly the binary \var{action} to the current result and the
  next member of the list, from left to right.  For instance, if
  \var{action} were the binary addition and \var{value} were 0, then
  folding a list would compute the sum of the list, including for
  empty lists.

\begin{urbiscript}[firstnumber=last]
[].foldl(function (a, b) { a + b }, 0);
[00000000] 0
[1, 2, 3].foldl(function (a, b) { a + b }, 0);
[00000000] 6
[1, 2, 3].foldl(function (a, b) { a - b }, 0);
[00000000] -6
\end{urbiscript}

\item \lstinline|front|\\
  Return the first element of the target. An error if the target is
  empty.

\begin{urbiscript}[firstnumber=last]
assert([0, 1, 2].front == 0);
[].front;
[00000000:error] !!! front: cannot be applied onto empty list
\end{urbiscript}

\item \lstinline|has(\var{that})|\\
  Whether one of the members of the target equals the argument.

\begin{urbiassert}[firstnumber=last]
[0, 1, 2].has(1);
! [0, 1, 2].has(5);
\end{urbiassert}

\item \lstinline|hasSame(\var{that})|\\
  Return whether one of the member of the target is physically equal
  to the argument.

\begin{urbiscript}[firstnumber=last]
var y = 1;
[00000000:hide] 1
[0, y, 2].hasSame(1);
[00000000] false
[0, y, 2].hasSame(y);
[00000000] true
\end{urbiscript}

\item \lstinline|head|\\
  Synonym for \lstinline|front|.

\item \lstinline|insertBack(\var{that})|\\
  Insert the given element at the end of the target.

\begin{urbiscript}[firstnumber=last]
var z = [0, 1];
[00000000] [0, 1]
assert(z.insertBack(2) == [0, 1, 2]);
assert(z == [0, 1, 2]);
\end{urbiscript}

\item \lstinline|insertFront(\var{that})|\\
  Insert the given element at the beginning of the target.

\begin{urbiscript}[firstnumber=last]
var a = [1, 2];
[00000000] [1, 2]
assert(a.insertFront(0) == [0, 1, 2]);
assert(a == [0, 1, 2]);
\end{urbiscript}

\item \lstinline|join(\var{sep} = "", \var{prefix} = "", \var{suffix} = "")|\\
  Bounces to \lstinline|String.join|, see \refObject{String}.

\begin{urbiassert}[firstnumber=last]
["", "ob", ""].join                == "ob";
["", "ob", ""].join("a")           == "aoba";
["", "ob", ""].join("a", "B", "b") == "Baobab";
\end{urbiassert}

\item \lstinline|keys()|\\
  The list of valid indexes.  This allows uniform iteration over a
  \refObject{Dictionary} or a \refObject{List}.

\begin{urbiscript}[firstnumber=last]
{
  var l = ["a", "b", "c"];
  assert(l.keys == [0, 1, 2]);
  assert({
           var res = [];
           for (var k: l.keys)
             res << l[k];
           res
         }
         == l);
};
\end{urbiscript}

\item \lstinline|map(\var{fun})|\\
Apply the given functional value on every member, and return the list
of results.

\begin{urbiassert}[firstnumber=last]
[0, 1, 2, 3].map(function (v) { v % 2 == 0})
        == [true, false, true, false];
\end{urbiassert}

\item \lstinline|range(\var{begin}, \var{end} = nil)|\\
  Return a sub-range of the list, from the first index included to the
  second index excluded.  An error if out of bounds.  Negative indices
  are valid, and number from the end.

  If \var{end} is \lstinline|nil|, calling \lstinline|range(\var{n})
  is equivalent to calling \lstinline|range(0, \var{n})|.

\begin{urbiscript}[firstnumber=last]
do ([0, 1, 2, 3])
{
  assert
  {
    range(0, 0)   == [];
    range(0, 1)   == [0];
    range(1)      == [0];
    range(1, 3)   == [1, 2];

    range(-3, -2) == [1];
    range(-3, -1) == [1, 2];
    range(-3, 0)  == [1, 2, 3];
    range(-3, 1)  == [1, 2, 3, 0];
    range(-4, 4)  == [0, 1, 2, 3, 0, 1, 2, 3];
  };
}|;
[].range(1, 3);
[00428697:error] !!! range: invalid index: 1
\end{urbiscript}

\item \lstinline|remove(\var{val})|\\
  Remove all elements from the target that equals \var{val}.

\begin{urbiscript}[firstnumber=last]
var c = [0, 1, 0, 2, 0, 3];
[00000000] [0, 1, 0, 2, 0, 3]
assert(c.remove(0) == [1, 2, 3]);
assert(c == [1, 2, 3]);
\end{urbiscript}

\item \lstinline|removeBack|\\
  Remove and return the last element of the target. An error if the
  target is empty.

\begin{urbiscript}[firstnumber=last]
var t = [0, 1, 2];
[00000000] [0, 1, 2]
assert(t.removeBack == 2);
assert(t == [0, 1]);
[].removeBack;
[00000000:error] !!! removeBack: cannot be applied onto empty list
\end{urbiscript}

\item \lstinline|removeById(\var{that})|\\
  Remove all elements from the target that physically equals
  \var{that}.

\begin{urbiscript}[firstnumber=last]
var d = 1;
[00000000] 1
var e = [0, 1, d, 1, 2];
[00000000] [0, 1, 1, 1, 2]
assert(e.removeById(d) == [0, 1, 1, 2]);
assert(e == [0, 1, 1, 2]);
\end{urbiscript}

\item \lstinline|removeFront|\\
Remove and return the first element from the target. An error if the
target is empty.

\begin{urbiscript}[firstnumber=last]
var g = [0, 1, 2];
[00000000] [0, 1, 2]
assert(g.removeFront == 0);
assert(g == [1, 2]);
[].removeFront;
[00000000:error] !!! removeFront: cannot be applied onto empty list
\end{urbiscript}

\item \lstinline|reverse|\\
Return the target with the order of elements inverted.

\begin{urbiassert}[firstnumber=last]
[0, 1, 2].reverse == [2, 1, 0];
\end{urbiassert}

\item \lstinline|size|\\
Return the number of elements in the target.

\begin{urbiassert}[firstnumber=last]
[0, 1, 2].size == 3;
[].size == 0;
\end{urbiassert}

\item \lstinline|sort|\\
Return the target, sorted with respect to the \lstinline|<| criteria.

\begin{urbiassert}[firstnumber=last]
[1, 0, 3, 2].sort == [0, 1, 2, 3];
\end{urbiassert}

\item \lstinline|tail|\\
Return the target, minus the first element. An error if the target is
empty.

\begin{urbiscript}[firstnumber=last]
assert([0, 1, 2].tail == [1, 2]);
[].tail;
[00000000:error] !!! tail: cannot be applied onto empty list
\end{urbiscript}

\item \lstinline|'=='(\var{that})|\\
Check whether all elements in the target and \var{that}, are
equal two by two.

\begin{urbiassert}[firstnumber=last]
[0, 1, 2] == [0, 1, 2];
!([0, 1, 2] == [0, 0, 2]);
\end{urbiassert}

\item \lstinline|'[]'(\var{n})|\\
  Return the \var{n}th member of the target (indexing is
  zero-based). If \var{n} is negative, start from the end.  An error
  if out of bounds.

\begin{urbiscript}[firstnumber=last]
assert(["0", "1", "2"][0] == "0");
assert(["0", "1", "2"][2] == "2");
["0", "1", "2"][3];
[00007061:error] !!! []: invalid index: 3

assert(["0", "1", "2"][-1] == "2");
assert(["0", "1", "2"][-3] == "0");
["0", "1", "2"][-4];
[00007061:error] !!! []: invalid index: -4
\end{urbiscript}

\item \lstinline|'[]='(\var{index}, \var{value})|\\
  Assign \var{value} to the element of the target at the given
  \var{index}.

\begin{urbiscript}[firstnumber=last]
var f = [0, 1, 2];
[00000000] [0, 1, 2]
f[1] = 42;
[00000000] 42
assert(f == [0, 42, 2]);
\end{urbiscript}

\item \lstinline|'*'(\var{n})|\\
  Return the target, concatenated \var{n} times to itself.
\begin{urbiassert}[firstnumber=last]
[0, 1] * 3 == [0, 1, 0, 1, 0, 1];
\end{urbiassert}

  Note that since it is the very same list which is repeatedly
  concatenated (the content is not cloned), side-effects on one item
  will reflect on ``all the items''.

\begin{urbiscript}[firstnumber=last]
var l = [[]] * 3;
[00000000] [[], [], []]
l[0] << 1;
[00000000] [1]
l;
[00000000] [[1], [1], [1]]
\end{urbiscript}

\item \lstinline|'+'(\var{other})|\\
Return the concatenation of the target and the \var{other} list.

\begin{urbiassert}[firstnumber=last]
[0, 1] + [2, 3] == [0, 1, 2, 3];
\end{urbiassert}

\item \lstinline|'-'(\var{other})|\\
Return the target without all element that equals any element in the
\var(other) list.

\begin{urbiassert}[firstnumber=last]
[0, 1, 0, 2, 3] - [1, 2] == [0, 0, 3];
\end{urbiassert}

\item \lstinline|'<<'(\var{that})|\\
  A synonym for \lstinline|insertBack|.

\item \lstinline|'<'(\var{other})|\\
  Return whether the target is inferior to the \var{other} list. A
  list is inferior to another if at least one of its element differs
  from the other, and the first differing element is inferior to the
  other.

\begin{urbiassert}[firstnumber=last]
!([0, 1, 2] < [0, 1, 2]);
!([0, 1, 2] < [0, 0, 2]);
[0, 1, 2] < [0, 2, 2];
\end{urbiassert}

  Since List derives from \refObject{Orderable}, the other order-based
  operators are defined.

\begin{urbiassert}[firstnumber=last]
 [0, 1, 2] <= [0, 1, 2];
 [0, 1, 2] >= [0, 1, 2];
 [0, 1, 2] >  [0, 0, 2];
\end{urbiassert}
\end{itemize}

%%% Local Variables:
%%% mode: latex
%%% TeX-master: "../urbi-sdk"
%%% End:

% LocalWords:  lst asList asString foldl hasSame removeBack popback removeFront
% LocalWords:  popfront insertBack pushback insertFront pushfront urbi sameAs
% LocalWords:  removeById setNth

%% Copyright (C) 2009-2011, Gostai S.A.S.
%%
%% This software is provided "as is" without warranty of any kind,
%% either expressed or implied, including but not limited to the
%% implied warranties of fitness for a particular purpose.
%%
%% See the LICENSE file for more information.

\section{Lobby}

A \dfn{lobby} is the local environment for each (remote or local)
connection to an \urbi server.

\subsection{Examples}

Since every lobby is-a \refObject{Channel}, one can use the methods of
Channel.

\begin{urbiscript}
lobby << 123;
[00478679] 123
lobby << "foo";
[00478679] "foo"
\end{urbiscript}

\subsection{Prototypes}
\begin{itemize}
\item \refSlot[Channel]{topLevel}, an instance of \refObject{Channel}
  with an empty Channel name.
\end{itemize}

\subsection{Construction}

A lobby is implicitly created at each connection. At the top level,
\this is a \dfn{Lobby}.

\begin{urbiscript}
this.protos;
[00000001] [Lobby]
this.protos[0].protos;
[00000003] [Channel_0xADDR]
\end{urbiscript}

Lobbies cannot be cloned, they must be created using \refSlot{create}.

\begin{urbiscript}
Lobby.new;
[00000177:error] !!! new: `Lobby' objects cannot be cloned
Lobby.create;
[00000174] Lobby_0x126450
\end{urbiscript}


\subsection{Slots}
\begin{urbiscriptapi}
\item[authors] Credit the authors of \usdk.


\item[banner] Internal.  Display \usdk banner.
\begin{urbiscript}
lobby.banner;
[00005344] *** ********************************************************
[00005344] *** Urbi SDK version 2.7.1 patch 472 revision b876514
[00005344] *** Copyright (C) 2004-2011 Gostai S.A.S.
[00005344] ***
[00005344] *** This program comes with ABSOLUTELY NO WARRANTY.  It can
[00005344] *** be used under certain conditions.  Type `license;',
[00005344] *** `authors;', or `copyright;' for more information.
[00005344] ***
[00005344] *** Check our community site: http://www.urbiforge.org.
[00005344] *** ********************************************************
\end{urbiscript}


\item[bytesReceived] The number of bytes that were ``input'' to \this.  See
  also \refSlot{receive}.
\begin{urbiscript}
var l = Lobby.create|;
assert (l.bytesReceived == 0);

l.receive("123456789;");
[00000022] 123456789
assert (l.bytesReceived == 10);

l.receive("1234;");
[00000023] 1234
assert (l.bytesReceived == 15);
\end{urbiscript}

\begin{urbicomment}
removeSlot("l")|;
\end{urbicomment}


\item[bytesSent] The number of bytes that were ``output'' by \this.  See
  also \refSlot{send} and \refSlot{write}.
\begin{urbiscript}
var l = Lobby.create|;
assert (l.bytesSent == 0);

l.send("0123456789");
[00011988] 0123456789
// 22 = "[00011988] 0123456789\n".size.
assert (l.bytesSent == 22);

l.send("xx", "label");
[00061783:label] xx
// 20 = "[00061783:label] xx\n".size.
assert (l.bytesSent == 42);

l.write("[01234567]\n");
[01234567]
assert (l.bytesSent == 53);
\end{urbiscript}


\item[connected]
  Whether \this is connected.
\begin{urbiassert}
connected;
\end{urbiassert}


\item[connectionTag] The tag of all code executed in the context of \this.
  This tag applies to \this, but the top-level loop is immune to
  \refSlot[Tag]{stop}, therefore \lstinline|connectionTag| controls every
  thing that was launched from this lobby, yet the lobby itself is still
  usable.
\begin{urbiscript}
every (1s) echo(1), sleep(0.5s); every (1s) echo(2),
sleep(1.2s);
connectionTag.stop;
[00000507] *** 1
[00001008] *** 2
[00001507] *** 1
[00002008] *** 2

"We are alive!";
[00002008] "We are alive!"

every (1s) echo(3), sleep(0.5s); every (1s) echo(4),
sleep(1.2s);
connectionTag.stop;
[00003208] *** 3
[00003710] *** 4
[00004208] *** 3
[00004710] *** 4

"and kicking!";
[00002008] "and kicking!"
\end{urbiscript}

  Of course, a background job may stop a foreground one.
\begin{urbiscript}
{ sleep(1.2s); connectionTag.stop; },
// Note the `;', this is a foreground statement.
every (1s) echo(5);
[00005008] *** 5
[00005508] *** 5

"bye!";
[00006008] "bye!"
\end{urbiscript}


\item[copyright](<deep> = true)%
  Display the copyright of \usdk.  Include copyright information
  about sub-components if \var{deep}.
\begin{urbiscript}
lobby.copyright(false);
[00033102] *** Urbi SDK version 2.7.1 patch 472 revision b876514
[00033102] *** Copyright (C) 2004-2011 Gostai S.A.S.

lobby.copyright;
[00088621] *** Urbi SDK version 2.7.1 patch 472 revision b876514
[00088621] *** Copyright (C) 2004-2011 Gostai S.A.S.
[00088621] ***
[00088621] *** Libport version urbi-sdk-2.7.1 patch 117 revision e96cd13
[00088621] *** Copyright (C) 2006-2011 Gostai S.A.S.
\end{urbiscript}


\item[create]
  Instantiate a new Lobby.
\begin{urbiassert}
Lobby.create.isA(Lobby);
\end{urbiassert}


\item[echo](<value>, <channel> = "")%
  Send \lstinline|\var{value}.asString| to \this, prefixed
  by the \refObject{String} \var{channel} name if specified.  This is
  the preferred way to send informative messages (prefixed with
  \samp{***}).
\begin{urbiscript}
lobby.echo("111", "foo");
[00015895:foo] *** 111
lobby.echo(222, "");
[00051909] *** 222
lobby.echo(333);
[00055205] *** 333
\end{urbiscript}


\item[echoEach](<list>, <channel> = "")%
  Apply \lstinline|echo(\var{m}, \var{channel})| for each member \var{m} of
  \var{list}.
\begin{urbiscript}
lobby.echo([1, "2"], "foo");
[00015895:foo] *** [1, "2"]

lobby.echoEach([1, "2"], "foo");
[00015895:foo] *** 1
[00015895:foo] *** 2

lobby.echoEach([], "foo");
\end{urbiscript}


%\item \lstinline|help|\experimental\\
%  Launch the tutorial.


\item[instances]
  A list of the currently alive lobbies.  It contains at least the Lobby
  object itself, and the current \refSlot{lobby}.
\begin{urbiassert}
lobby in Lobby.instances;
Lobby in Lobby.instances;
\end{urbiassert}


\item[license]
  Display the end user license agreement of the \usdk.
\begin{urbiunchecked}
lobby.license;
[00000000] ***                     GNU AFFERO GENERAL PUBLIC LICENSE
[00000000] ***                        Version 3, 19 November 2007
[00000000] ***
[00000000] ***  Copyright (C) 2007 Free Software Foundation, Inc. <http://fsf.org/>
[00000000] ***  Everyone is permitted to copy and distribute verbatim copies
[00000000] ***  of this license document, but changing it is not allowed.
[00000000] ***
[00000000] ***                             Preamble
[00000000] ***
[00000000] ***   The GNU Affero General Public License is a free, copyleft license for
[00000000] *** software and other kinds of works, specifically designed to ensure
[00000000] *** cooperation with the community in the case of network server software.
[00000000] *** [...]
\end{urbiunchecked}


\item[lobby]
  The lobby of the current connection.  This is typically \this.
\begin{urbiassert}
Lobby.lobby === this;
\end{urbiassert}

  But when several connections are active (e.g., when there are remote
  connections), it can be different from the target of the call.

\begin{urbiscript}
Lobby.create| Lobby.create|;
for (var l : lobbies)
  assert (l.lobby == Lobby.lobby);
\end{urbiscript}


\item[onDisconnect](<lobby>)%
  Event launched when \this is disconnected.  There is a single event
  instance for all the lobby, \refSlot[Lobby]{onDisconnect}, the
  disconnected lobby being passed as argument.


\item[quit] Shut this lobby down, i.e., close the connection.  The
  server is still running, see \refSlot[System]{shutdown} to quit the
  server.


\item[receive](<value>)%
  This is low-level routine.  Pretend the \refObject{String}
  \var{value} was received from the connection.  There is no guarantee
  that \var{value} will be the next program block that will be
  processed: for instance, if you load a file which, in its middle,
  uses \lstinline|lobby.receive("foo")|, then \lstinline|"foo"| will
  be appended after the end of the file.
\begin{urbiscript}
Lobby.create.receive("12;");
[00478679] 12
\end{urbiscript}


\item[remoteIP]
  When \this is connected to a remote server, it's Internet address.


\item[send](<value>, <channel> = "")%
  This is low-level routine.  Send the \refObject{String} \var{value}
  to \this, prefixed by the \refObject{String}
  \var{channel} name if specified.
\begin{urbiscript}
lobby.send("111", "foo");
[00015895:foo] 111
lobby.send("222", "");
[00051909] 222
lobby.send("333");
[00055205] 333
\end{urbiscript}


\item[thanks] Credit the contributors of \usdk.


\item[wall](<value>, <channel> = "")%
  Perform \lstinline|echo(\var{value}, \var{channel})| on all the
  existing lobbies (except Lobby itself).
\begin{urbiscript}[firstnumber=1]
Lobby.wall("111", "foo");
[00015895:foo] *** 111
\end{urbiscript}


\item[write](<value>)%
  This is low-level routine.  Send the \refObject{String} \var{value}
  to the connection.  Note that because of buffering, the output might
  not be visible before an end-of-line character is output.
\begin{urbiscript}
lobby.write("[");
lobby.write("999999999:");
lobby.write("myTag] ");
lobby.write("Hello, World!");
lobby.write("\n");
[999999999:myTag] Hello, World!
\end{urbiscript}
\end{urbiscriptapi}

%%% Local Variables:
%%% coding: utf-8
%%% mode: latex
%%% TeX-master: "../urbi-sdk"
%%% ispell-dictionary: "american"
%%% ispell-personal-dictionary: "../urbi.dict"
%%% fill-column: 76
%%% End:

%% Copyright (C) 2009-2010, Gostai S.A.S.
%%
%% This software is provided "as is" without warranty of any kind,
%% either expressed or implied, including but not limited to the
%% implied warranties of fitness for a particular purpose.
%%
%% See the LICENSE file for more information.

\section{Mutex}

\dfn{Mutex} allow to define critical sections.

\subsection{Prototypes}
\begin{itemize}
\item \refObject{Tag}
\end{itemize}

\subsection{Construction}
A Mutex can be constructed like any other Tag but without name.

\begin{urbiscript}[firstnumber=1]
var m = Mutex.new;
[00000000] Mutex_0x964ed40
\end{urbiscript}

You can define critical sections by tagging your code using the Mutex.

\begin{urbiscript}[firstnumber=1]
var m = Mutex.new |
m: echo("this is critical section");
[00000001] *** this is critical section
\end{urbiscript}

As a critical section, two pieces of code tagged by the same ``Mutex''
will never be executed at the same time.

\subsection{Slots}

\begin{urbiscriptapi}
\item[asMutex]  Return \lstinline|this|.
\begin{urbiscript}
var m1 = Mutex.new;
assert
{
  m1.asMutex === m1;
};
\end{urbiscript}
\end{urbiscriptapi}


%%% Local Variables:
%%% mode: latex
%%% TeX-master: "../urbi-sdk"
%%% ispell-dictionary: "american"
%%% ispell-personal-dictionary: "../urbi.dict"
%%% fill-column: 76
%%% End:

%% Copyright (C) 2009-2011, Gostai S.A.S.
%%
%% This software is provided "as is" without warranty of any kind,
%% either expressed or implied, including but not limited to the
%% implied warranties of fitness for a particular purpose.
%%
%% See the LICENSE file for more information.

\section{Object}

\refObject{Object} includes the mandatory primitives for all objects in \us.
All objects in \us must inherit (directly or indirectly) from it.

\subsection{Prototypes}

\begin{refObjects}
\item[Comparable]
\item[Global]
\end{refObjects}

\subsection{Construction}

A fresh object can be instantiated by cloning \slot{Object} itself.

\begin{urbiscript}[firstnumber=1]
Object.new;
[00000421] Object_0x00000000
\end{urbiscript}

The keyword \lstindex{class} also allows to define objects intended to serve
as prototype of a family of objects, similarly to classes in traditional
object-oriented programming languages (see \autoref{sec:tut:class}).

\begin{urbiscript}
{
  class Foo
  {
    var attr = 23;
  };
  assert
  {
    Foo.localSlotNames == ["asFoo", "attr", "type"];
    Foo.asFoo === Foo;
    Foo.attr == 23;
    Foo.type == "Foo";
  };
};
\end{urbiscript}


\subsection{Slots}

\begin{urbiscriptapi}
\item[acceptVoid]
  Return \this.  See \refObject{void} to know why.
\begin{urbiscript}
{
  var o = Object.new;
  assert(o.acceptVoid === o);
};
\end{urbiscript}


\item[addProto](<proto>)%
  Add \var{proto} into the list of prototypes of \this.  Return \this.
\begin{urbiscript}
do (Object.new)
{
  assert
  {
    addProto(Orderable) === this;
    protos == [Orderable, Object];
  };
}|;
\end{urbiscript}

\item[allProto]%
  A list with \this, its parents, their parents,\ldots
\begin{urbiassert}
123.allProtos.size == 12;
\end{urbiassert}

\item[allSlotNames]
  Deprecated alias for \refSlot{slotNames}.
\begin{urbiassert}
Object.allSlotNames == Object.slotNames;
\end{urbiassert}

\item[apply](<args>)%
  ``Invoke \this''.  The size of the argument list,
  \var{args}, must be one.  This argument is ignored.  This function
  exists for compatibility with \refSlot[Code]{apply}.
\begin{urbiassert}
Object.apply([this]) === Object;
Object.apply([1])    === Object;
\end{urbiassert}

\item[as](<type>)%
  Convert \this to \var{type}.  This is syntactic sugar for
  \lstinline|as\var{Type}| when \var{Type} is the \lstinline|type| of
  \var{type}.
\begin{urbiassert}
     12.as(Float) == 12;
   "12".as(Float) == 12;
    12.as(String) == "12";
Object.as(Object) === Object;
\end{urbiassert}

\item[asBool]
  Whether \this is ``true'', see \autoref{sec:truth}.
\begin{urbiscript}
assert
{
  Global.asBool == true;
  nil.asBool ==    false;
};
void.asBool;
[00000421:error] !!! unexpected void
\end{urbiscript}

\item[bounce](<name>)%
  Return \lstinline|this.\var{name}| transformed from a method into a
  function that takes its target (its ``\this'') as first
  and only argument.  \lstinline|this.\var{name}| must take no
  argument.
\begin{urbiassert}
{ var myCos = Object.bounce("cos"); myCos(0) }    == 0.cos;
{ var myType = bounce("type"); myType(Object); } == "Object";
{ var myType = bounce("type"); myType(3.14); }   == "Float";
\end{urbiassert}

\item[callMessage](<msg>)%
  Invoke the \refObject{CallMessage} \var{msg} on this.
%%% \begin{urbiscript}
%%% function f(var tgt, var msg, var args)
%%% {
%%%   call.target  = tgt;
%%%   call.message = msg;
%%%   call.code = tgt.getSlot(msg);
%%%   call.args    = args;
%%%   call.inspect;
%%%   tgt.callMessage(call);
%%% }|;
%%% assert
%%% {
%%%   f(Object, "type", []) == "Object.f(1, 2)";
%%%
%%% };
%%% \end{urbiscript}
\item[clone]
  Clone \this, i.e., create a fresh, empty, object, which
  sole prototype is \this.
\begin{urbiassert}
Object.clone.protos == [Object];
Object.clone.localSlotNames == [];
\end{urbiassert}

\item[cloneSlot](<from>, <to>)%
  Set the new slot \var{to} using a clone of \var{from}. This can only
  be used into the same object.

\begin{urbiscript}
var foo = Object.new |;
cloneSlot("foo", "bar") |;
assert(!(foo === bar));
\end{urbiscript}

\item[copySlot](<from>, <to>)%
  Same as \lstinline|cloneSlot|, but the slot aren't cloned, so the
  two slot are the same.
\begin{urbiscript}
var moo = Object.new |;
cloneSlot("moo", "loo") |;
assert(!(moo === loo));
\end{urbiscript}

\item[createSlot](<name>)%
  Create an empty slot (which actually means it is bound to
  \lstinline|void|) named \var{name}.  Raise an error if \var{name}
  was already defined.
\begin{urbiscript}
do (Object.new)
{
  assert(!hasLocalSlot("foo"));
  assert(createSlot("foo").isVoid);
  assert(hasLocalSlot("foo"));
}|;
\end{urbiscript}

\item[dump](<depth>)%
  Describe \this: its prototypes and slots.  The argument
  \var{depth} specifies how recursive the description is: the greater,
  the more detailed.  This method is mostly useful for debugging
  low-level issues, for a more human-readable interface, see also
  \refSlot{inspect}.
\begin{urbiscript}
do (2) { var this.attr = "foo"; this.attr->prop = "bar" }.dump(0);
[00015137] *** Float_0x240550 {
[00015137] ***   /* Special slots */
[00015137] ***   protos = Float
[00015137] ***   value = 2
[00015137] ***   /* Slots */
[00015137] ***   attr = String_0x23a750 <...>
[00015137] ***     /* Properties */
[00015137] ***     prop = String_0x23a7a0 <...>
[00015137] ***   }
do (2) { var this.attr = "foo"; this.attr->prop = "bar" }.dump(1);
[00020505] *** Float_0x240550 {
[00020505] ***   /* Special slots */
[00020505] ***   protos = Float
[00020505] ***   value = 2
[00020505] ***   /* Slots */
[00020505] ***   attr = String_0x23a750 {
[00020505] ***     /* Special slots */
[00020505] ***     protos = String
[00020505] ***     /* Slots */
[00020505] ***     }
[00020505] ***     /* Properties */
[00020505] ***     prop = String_0x239330 {
[00020505] ***       /* Special slots */
[00020505] ***       protos = String
[00020505] ***       /* Slots */
[00020505] ***       }
[00020505] ***   }
\end{urbiscript}

\item[getPeriod]
  Deprecated.  Use \refSlot[System]{period} instead.

\item[getProperty](<slotName>, <propName>)%
  The value of the \var{propName} property associated to the slot
  \var{slotName} if defined.  Raise an error otherwise.
\begin{urbiscript}
const var myPi = 3.14|;
assert (getProperty("myPi", "constant"));

getProperty("myPi", "foobar");
[00000045:error] !!! property lookup failed: myPi->foobar
\end{urbiscript}

\item[getLocalSlot](<name>)%
  The value associated to \var{name} in \this, excluding
  its ancestors (contrary to \lstinline|getSlot|).
\begin{urbiscript}
var a = Object.new|;

// Local slot.
var a.slot = 21|;
assert
{
  a.locateSlot("slot") === a;
  a.getLocalSlot("slot") == 21;
};

// Inherited slot are not looked-up.
assert { a.locateSlot("init") == Object };
a.getLocalSlot("init");
[00041066:error] !!! lookup failed: init
\end{urbiscript}

\item[getSlot](<name>)%
  The value associated to \var{name} in \this, possibly
  after a look-up in its prototypes (contrary to
  \lstinline|getLocalSlot|).
\begin{urbiscript}
var b = Object.new|;
var b.slot = 21|;

assert
{
  // Local slot.
  b.locateSlot("slot") === b;
  b.getSlot("slot") == 21;

  // Inherited slot.
  b.locateSlot("init") === Object;
  b.getSlot("init") == Object.getSlot("init");
};

// Unknown slot.
assert { b.locateSlot("ENOENT") == nil; };
b.getSlot("ENOENT");
[00041066:error] !!! lookup failed: ENOENT
\end{urbiscript}

\item[hash]%
  A \refObject{Hash} object for \this.  This default implementation returns
  a different hash for every object, so every key maps to a different
  cells. Classes that have value semantic should override the hash method so
  as objects that are equal (in the \refSlot[Object]{'=='} sense) have the
  same hash. \refSlot[String]{hash} does so for instance; as a consequence
  different String objects with the same value map to the same cell.

  A hash only makes sense as long as the hashed object exists.

\begin{urbiscript}
var o1 = Object.new|
var o2 = Object.new|
assert
{
  o1.hash == o1.hash;
  o1.hash != o2.hash;
};
\end{urbiscript}

\item[hasLocalSlot](<slot>)%
  Whether \this features a slot \var{slot}, locally (not from some
  ancestor).  See also \refSlot{hasSlot}.
\begin{urbiscript}
class Base         { var this.base = 23; } |;
class Derive: Base { var this.derive = 43 } |;
assert(Derive.hasLocalSlot("derive"));
assert(!Derive.hasLocalSlot("base"));
\end{urbiscript}

\item[hasProperty](<slotName>, <propName>)%
  Whether the slot \var{slotName} of \this has a property
  \var{propName}.
\begin{urbiscript}
const var halfPi = pi / 2|;
assert
{
  hasProperty("halfPi", "constant");
  !hasProperty("halfPi", "foobar");
};
\end{urbiscript}

\item[hasSlot](<slot>)%
  Whether \this has the slot \var{slot}, locally, or from
  some ancestor.  See also \refSlot{hasLocalSlot}.

\begin{urbiassert}
Derive.hasSlot("derive");
Derive.hasSlot("base");
!Base.hasSlot("derive");
\end{urbiassert}

\item['$id']% fix color $

\item[inspect](<deep> = false)%
  Describe \this: its prototypes and slots, and their
  properties.  If \var{deep}, all the slots are described, not only
  the local slots. See also \refSlot{dump}.
\begin{urbiscript}
do (2) { var this.attr = "foo"; this.attr->prop = "bar"}.inspect;
[00001227] *** Inspecting 2
[00001227] *** ** Prototypes:
[00001227] ***   0
[00001227] *** ** Local Slots:
[00001228] ***   attr : String
[00001228] ***     Properties:
[00001228] ***      prop : String = "bar"
\end{urbiscript}

\item[isA](<obj>)%
  Whether \this has \var{obj} in his parents.
\begin{urbiassert}
   Float.isA(Orderable);
! String.isA(Float);
\end{urbiassert}

\item[isNil]%
  Whether \this is \refObject{nil}.
\begin{urbiassert}
 nil.isNil;
!  0.isNil;
\end{urbiassert}

\item[isProto]
  Whether \this is a prototype.
\begin{urbiassert}
 Float.isProto;
!   42.isProto;
\end{urbiassert}

\item[isVoid]%
  Whether \this is \lstinline|void|.  See \refObject{void}.
\begin{urbiassert}
void.isVoid;
! 42.isVoid;
\end{urbiassert}

\item[localSlotNames]%
  A list with the names of the local (i.e., not including those of its
  ancestors) slots of \this.  See also \refSlot{slotNames}.
\begin{urbiscript}
var top = Object.new|;
var top.top1 = 1|;
var top.top2 = 2|;
var bot = top.new|;
var bot.bot1 = 10|;
var bot.bot2 = 20|;
assert
{
  top.localSlotNames == ["top1", "top2"];
  bot.localSlotNames == ["bot1", "bot2"];
};
\end{urbiscript}

\item[locateSlot](<slot>)%
  The \slot{Object} that provides \var{slot} to \this, or \lstinline|nil| if
  \this does not feature \var{slot}.
\begin{urbiassert}
locateSlot("locateSlot") == Object;
locateSlot("doesNotExist").isNil;
\end{urbiassert}

\item[print] Send \this to the \refSlot[Channel]{topLevel} channel.
\begin{urbiscript}
1.print;
[00001228] 1
[1, "12"].print;
[00001228] [1, "12"]
\end{urbiscript}

\item[protos]
  The list of prototypes of \this.
\begin{urbiassert}
12.protos == [Float];
\end{urbiassert}

\item[properties](<slotName>)%
  A dictionary of the properties of slot \var{slotName}.  Raise an error if
  the slot does not exist.
\begin{urbiscript}
2.properties("foo");
[00238495:error] !!! lookup failed: foo
do (2) { var foo = "foo" }.properties("foo");
[00238501] ["constant" => false]
do (2) { var foo = "foo" ; foo->bar = "bar" }.properties("foo");
[00238502] ["bar" => "bar", "constant" => false]
\end{urbiscript}

\item[removeLocalSlot](<slot>)%
  Remove \var{slot} from the (local) list of slots of \this, and return
  \this.  Raise an error if \var{slot} does not exist.  See also
  \refSlot{removeSlot}.
\begin{urbiscript}
var base = Object.new|;
var base.slot = "base"|;

var derive = Base.new|;
var derive.slot = "derive"|;

derive.removeLocalSlot("foo");
[00000080:error] !!! lookup failed: foo

assert
{
  derive.removeLocalSlot("slot") === derive;
  derive.localSlotNames == [];
  base.slot == "base";
};

derive.removeLocalSlot("slot");
[00000090:error] !!! lookup failed: slot

assert
{
  base.slot == "base";
};
\end{urbiscript}


\item[removeProperty](<slotName>, <propName>)%
  Remove the property \var{propName} from the slot \var{slotName}.  Raise an
  error if the slot does not exist.  Warn if \var{propName} does not exist;
  in a future release this will be an error.
\begin{urbiscript}
var r = Object.new|;

// Non-existing slot.
r.removeProperty("slot", "property");
[00000072:error] !!! lookup failed: slot

var r.slot = "slot value"|;
// Non-existing property.
r.removeProperty("slot", "property");
[00000081:warning] !!! no such property: slot->property
[00000081:warning] !!!    called from: removeProperty

r.slot->property = "property value"|;
assert
{
  r.hasProperty("slot", "property");
  // Existing property.
  r.removeProperty("slot", "property") == "property value";
  ! r.hasProperty("slot", "property");
};
\end{urbiscript}

\item[removeProto](<proto>)%
  Remove \var{proto} from the list of prototypes of \this, and return \this.
  Do nothing if \var{proto} is not a prototype of \this.
\begin{urbiscript}
do (Object.new)
{
  assert
  {
    addProto(Orderable);
    removeProto(123) === this;
    protos == [Orderable, Object];
    removeProto(Orderable) === this;
    protos == [Object];
  };
}|;
\end{urbiscript}

\item[removeSlot](<slot>)%
  Remove \var{slot} from the (local) list of slots of \this, and return
  \this.  Warn if \var{slot} does not exist; in a future release this will
  be an error.  See also \refSlot{removeLocalSlot}.
\begin{urbiscript}
{
  var base = Object.new;
  var base.slot = "base";

  var derive = Base.new;
  var derive.slot = "derive";

  assert
  {
    derive.removeSlot("foo") === derive;
[00000080:warning] !!! no such local slot: foo
[00000080:warning] !!!    called from: removeSlot
[00000080:warning] !!!    called from: code
[00000080:warning] !!!    called from: eval
[00000080:warning] !!!    called from: value
[00000080:warning] !!!    called from: assertCall

    derive.removeSlot("slot") === derive;
    derive.localSlotNames == [];
    base.slot == "base";
    derive.removeSlot("slot") === derive;
[00000099:warning] !!! no such local slot: slot
[00000099:warning] !!!    called from: removeSlot
[00000099:warning] !!!    called from: code
[00000099:warning] !!!    called from: eval
[00000099:warning] !!!    called from: value
[00000099:warning] !!!    called from: assertCall

    base.slot == "base";
  };
};
\end{urbiscript}


\item[setConstSlot]%
  Like \refSlot{setSlot} but the created slot is const.
\begin{urbiscript}
assert(setConstSlot("fortyTwo", 42) == 42);
fortyTwo = 51;
[00000000:error] !!! cannot modify const slot
\end{urbiscript}

\item[setProperty](<slotName>, <propName>, <value>)%
  Set the property \var{propName} of slot \var{slotName} to \var{value}.
  Raise an error in \var{slotName} does not exist.  Return \var{value}.
  This is what \lstinline|\var{slotName}->\var{propName} = \var{value}|
  actually performs.
\begin{urbiscript}
do (Object.new)
{
  var slot = "slot";
  var value = "value";
  assert
  {
    setProperty("slot", "prop", value) === value;
    "prop" in properties("slot");
    getProperty("slot", "prop") === value;
    slot->prop === value;
    setProperty("slot", "noSuchProperty", value) === value;
  };
}|;
setProperty("noSuchSlot", "prop", "12");
[00000081:error] !!! lookup failed: noSuchSlot
\end{urbiscript}


\item[setProtos](<protos>)%
  Set the list of prototypes of \this to \var{protos}.  Return
  \lstinline|void|.
\begin{urbiscript}
do (Object.new)
{
  assert
  {
    protos == [Object];
    setProtos([Orderable, Object]).isVoid;
    protos == [Orderable, Object];
  };
}|;
\end{urbiscript}

\item[setSlot](<name>, <value>)%
  Create a slot \var{name} mapping to \var{value}. Raise an error if
  \var{name} was already defined.  This is what
  \lstinline|var \var{name} = \var{value}| actually performs.
\begin{urbiassert}
Object.setSlot("theObject", Object) === Object;
Object.theObject === Object;
theObject === Object;
\end{urbiassert}

  If the current job is in redefinition mode, \lstinline|setSlot| on
  an already defined slot is not an error and overwrites the slot like
  \lstinline|updateSlot| would. See the \lstinline|redefinitionMode|
  method in \refObject{System}.

\item[slotNames]%
  A list with the slot names of \this and its ancestors.
\begin{urbiassert}
Object.localSlotNames
  .subset(Object.slotNames);
Object.protos.foldl(function (var r, var p) { r + p.localSlotNames },
                    [])
  .subset(Object.slotNames);
\end{urbiassert}

\item[type]%
  The name of the type of \this.  The \lstinline|class|
  construct defines this slot to the name of the class
  (\autoref{sec:tut:class}).  This is used to display the name of
  ``instances''.
\begin{urbiscript}
class Example {};
[00000081] Example
assert
{
  Example.type == "Example";
};
Example.new;
[00000081] Example_0x6fb2720
\end{urbiscript}

\item[uid]
  The unique id of \this.
\begin{urbiscript}
{
  var foo = Object.new;
  var bar = Object.new;
  assert
  {
    foo.uid == foo.uid;
    foo.uid != bar.uid;
  };
};
\end{urbiscript}

\item[unacceptVoid]%
  Return \this.  See \refObject{void} to know why.
\begin{urbiscript}
{
  var o = Object.new|
  assert(o.unacceptVoid === o);
};
\end{urbiscript}

%%% FIXME: \item[uobjectInit]
\item[updateSlot](<name>, <value>)%
  Map the existing slot named \var{name} to \var{value}. Raise an
  error if \var{name} was not defined.
\begin{urbiassert}
Object.setSlot("one", 1)    == 1;
Object.updateSlot("one", 2) == 2;
Object.one                  == 2;
\end{urbiassert}

\item['&&'](<that>)%
  Short-circuiting logical and. If \this evaluates to true evaluate and
  return \var{that}, otherwise return \this without evaluating \var{that}.
\begin{urbiassert}
(0 && "foo") == 0;
(2 && "foo") == "foo";

(""    && "foo") == "";
("foo" && "bar") == "bar";
\end{urbiassert}

\item['||'](<that>)%
  Short-circuiting logical or. If \this evaluates to false evaluate and
  return \var{that}, otherwise return \this without evaluating \var{that}.
\begin{urbiassert}
(0 || "foo") == "foo";
(2 ||  1/0)  == 2;

(""    || "foo") == "foo";
("foo" || 1/0)   == "foo";
\end{urbiassert}

\item \lstinline|'!'|\\
  Logical negation.  If \this evaluates to false return \lstinline|true| and
  vice-versa.
\begin{urbiassert}
!1 == false;
!0 == true;

!"foo" == false;
!""    == true;
\end{urbiassert}

\item['+='](<that>)%
  Bounce to \lstinline|this '+' \var{that}|.

\item['-='](<that>)%
  Bounce to \lstinline|this '-' \var{that}|.

\item['*='](<that>)%
  Bounce to \lstinline|this '*' \var{that}|.

\item['/='](<that>)%
  Bounce to \lstinline|this '/' \var{that}|.

\item['^='](<that>)%
  Bounce to \lstinline|this '^' \var{that}|.

\item \lstinline|'%='(\var{that})|\\
  Bounce to \lstinline|this '-' \var{that}|.

\item['=='](<that>)%
  Whether \this and \that are equal.  See also \refObject{Comparable} and
  \autoref{sec:lang:operators:comparison}.  By default, bounces to
  \refSlot{'==='}.  This operator \emph{must} be redefined for objects that
  have a value-semantics; for instance two \refObject{String} objects that
  denotes the same string should be equal according to \lstinline|==|,
  although physically different (i.e., not equal according to
  \lstinline|===|).
\begin{urbiscript}
{
  var o1 = Object.new;
  var o2 = Object.new;
  assert
  {
      o1 == o1;
    !(o1 == o2);
      o1 != o2;
    !(o1 != o1);

      1  ==  1;
     "1" == "1";
     [1] == [1];
  };
};
\end{urbiscript}

\item['==='](<that>)%
  Whether \this and \that are exactly the same object (i.e., \this and \that
  are two different means to denote the very same location in memory).  To
  denote equivalence, use \refSlot{'=='}; for instance two \refObject{Float}
  objects that denote 42 can be different objects (in the sense of
  \lstinline|===|), but will be considered equal by \lstinline|==|.  See
  also \refSlot{'==='} and \autoref{sec:lang:operators:comparison}.
\begin{urbiscript}
{
  var o1 = Object.new;
  var o2 = Object.new;
  assert
  {
      o1 === o1;
    !(o1 === o2);

    !( 1  ===  1 );
    !("1" === "1");
    !([1] === [1]);
  };
};
\end{urbiscript}

\item \lstinline|'!=='(\var{that})|\\%
  The negation of \lstinline|\this === \that|, see \refSlot{'==='}.
\begin{urbiscript}
{
  var o1 = Object.new;
  var o2 = Object.new;
  assert
  {
      o1 !== o2;
    !(o1 !== o1);

      1  !==  1;
     "1" !== "1";
     [1] !== [1];
  };
};
\end{urbiscript}

\end{urbiscriptapi}

%%% Local Variables:
%%% coding: utf-8
%%% mode: latex
%%% TeX-master: "../urbi-sdk"
%%% ispell-dictionary: "american"
%%% ispell-personal-dictionary: "../urbi.dict"
%%% fill-column: 76
%%% End:

\section{Orderable}
Objects that have a concept of ``smaller than''.  See also
\refObject{Comparable}.

This object, made to serve as prototype, provides a definition of
\lstinline{<} based on \lstinline{>}, and vice versa; and definition
of \lstinline{<=}/\lstinline{>=} based on
\lstinline{<}/\lstinline{>}\lstinline{==}.  You \strong{must} define
either \lstinline{<} or \lstinline{>}, otherwise invoking either
method will result in endless recursions.

\begin{urbiscript}
class Foo : Orderable
{
  var value = 0;
  function init (v) { value = v; };
  function '<' (lhs)  { value < lhs.value; };
  function asString() { "<" + value.asString + ">"; };
};
[00000000] <0>
var one = Foo.new(1);
[00000001] <1>
var two = Foo.new(2);
[00000002] <2>

assert( (one <= one) &&  (one <= two) && !(two <= one));
assert(!(one >  one) && !(one >  two) &&  (two >  one));
assert( (one >= one) && !(one >= two) &&  (two >= one));
\end{urbiscript}


%%% Local Variables:
%%% mode: latex
%%% TeX-master: "../urbi-sdk"
%%% End:

\section{Pair}

A \dfn{pair} is a container storing two objects, similar in spirit to
\lstinline|std::pair| in \Cxx.


\begin{itemize}
\item \lstinline|asString|\\
  Generate the string \samp{(\var{first}, \var{second})} using
  \code{asPrintable} to convert members to strings.

\item \lstinline|first|\\
  Return the first member of the pair.

\item \lstinline|init(\var{first}, \var{second})|~\\
  Instantiate a new \lstinline|Pair| containing \var{first} and
  \var{second}.

\item \lstinline|second|\\
  Return the second member of the pair.

\item \lstinline|'[]'(\var{index})|\\
  Return the \var{index}-th element.  \var{index} must be 0 or 1.

\item \lstinline|'[]='(\var{index}, \var{value})|\\
  Set (and return) the \var{index}-th element to \var{value}.
  \var{index} must be 0 or 1.

\item \lstinline|'<'(\var{other})|\\
  Lexicographic comparison between two pairs.

\item \lstinline|'=='(\var{other})|\\
  Whether \lstinline|this| and \lstinline|other| have the same
  contents (equality-wise).
\end{itemize}

See below for an example.

\begin{urbiscript}
var p = Pair.new(1, 2);
[00000001] (1, 2)
p.first - p.second;
[00000002] -1
p[0] = 10 | p;
[00000002] (10, 2)
assert(Pair.new(0, 0) < Pair.new(0, 1));
assert(Pair.new(0, 0) < Pair.new(1, 0));
assert(Pair.new(0, 1) < Pair.new(1, 0));
assert(Pair.new(1, 2) == Pair.new(1, 2));
assert(!(Pair.new(1, 1) == Pair.new(2, 2)));
\end{urbiscript}


%%% Local Variables:
%%% mode: latex
%%% TeX-master: "../urbi-sdk"
%%% End:

%% Copyright (C) 2009-2010, Gostai S.A.S.
%%
%% This software is provided "as is" without warranty of any kind,
%% either expressed or implied, including but not limited to the
%% implied warranties of fitness for a particular purpose.
%%
%% See the LICENSE file for more information.

\section{Pattern}

\lstinline|Pattern| class is used to make correspondences between a pattern
and another \lstinline|Object|.  The visit is done either on the pattern or
on the element against which the pattern is compared.

\lstinline|Pattern|s are used for the implementation of the pattern matching.
So any class made compatible with the pattern matching implemented by this
class will allow you to use it implicitly in your scripts.

\begin{urbiscript}[firstnumber=1]
[1, var a, var b] = [1, 2, 3];
[00000000] [1, 2, 3]
a;
[00000000] 2
b;
[00000000] 3
\end{urbiscript}

\subsection{Prototypes}

\begin{refObjects}
\item[Object]
\end{refObjects}

\subsection{Construction}

A \lstinline|Pattern| can be created with any object that can be matched.

\begin{urbiscript}
Pattern.new([1]); // create a pattern to match the list [1].
[00000000] Pattern_0x189ea80
Pattern.new(Pattern.Binding.new("a")); // match anything into "a".
[00000000] Pattern_0x18d98b0
\end{urbiscript}

\subsection{Slots}

\begin{urbiscriptapi}
\item[Binding]
  A class used to create pattern variables.

\begin{urbiscript}
Pattern.Binding.new("a");
[00000000] var a
\end{urbiscript}


\item[bindings]

  A \lstinline|Dictionary| filled by the match function for each
  \refSlot{Binding} contained inside the pattern.

\begin{urbiscript}
{
  var p = Pattern.new([Pattern.Binding.new("a"), Pattern.Binding.new("b")]);
  assert (p.match([1, 2]));
  p.bindings
};
[00000000] ["a" => 1, "b" => 2]
\end{urbiscript}


\item[match](<value>)%

  Use \var{value} to unify the current pattern with this value.
  Return the status of the match.
  \begin{itemize}


\item[matchPattern](<pattern>, <value>)%

  This function is used as a callback function to store all bindings
  in the same place.  This function is useful inside objects that
  implement a \lstinline|match| or \lstinline|matchAgainst| function
  that need to continue the match deeper.  Return the status of the
  match (a Boolean).

  The \var{pattern} should provide a method
  \lstinline|match(\var{handler},\var{value})| otherwise the value method
  \lstinline|matchAgainst(\var{handler}, \var{pattern})| is used.  If none
  are provided the \lstinline|'=='| operator is used.

  To see how to use it, you can have a look at the implementation of
  \refSlot[List]{matchAgainst}.

%% This function is indirectly tested with the match of Pattern.Binding
%% inside lists.


\item[pattern]
  The pattern given at the creation.
\begin{urbiassert}
Pattern.new(1).pattern == 1;
Pattern.new([1, 2]).pattern == [1, 2];
{
  var pattern = [1, Pattern.Binding.new("a")];
  Pattern.new(pattern).pattern === pattern
};
\end{urbiassert}


\item If the match is correct, then the \var{bindings} member will
      contain the result of every matched values.
    \item If the match is incorrect, then the \var{bindings} member should
      not be used.
  \end{itemize}
  If the pattern contains multiple \refSlot{Binding} with the same name,
  then the behavior is undefined.

\begin{urbiassert}
Pattern.new(1).match(1);
Pattern.new([1, 2]).match([1, 2]);
! Pattern.new([1, 2]).match([1, 3]);
! Pattern.new([1, 2]).match([1, 2, 3]);
Pattern.new(Pattern.Binding.new("a")).match(0);
Pattern.new([1, Pattern.Binding.new("a")]).match([1, 2]);
! Pattern.new([1, Pattern.Binding.new("a")]).match(0);
\end{urbiassert}
\end{urbiscriptapi}

%%% Local Variables:
%%% mode: latex
%%% TeX-master: "../urbi-sdk"
%%% ispell-dictionary: "american"
%%% ispell-personal-dictionary: "../urbi.dict"
%%% fill-column: 76
%%% End:

\section{Primitive}
\Cxx routine callable from \us.

\subsection{Prototypes}
\begin{itemize}
\item \refObject{Executable}
\item \refObject{Object}
\end{itemize}

\subsection{Construction}

It is not possible to construct a Primitive.

\subsection{Slots}

\begin{urbiscriptapi}
\item \lstinline|apply(\var{args})|\\
  Invoke a primitive.  The argument list, \var{args}, must start with
  the target.
\begin{urbiassert}
Float.getSlot("+").isA(Global.getSlot("Primitive"));
Float.getSlot("+").apply([1, 2]) == 3;

String.getSlot("+").isA(Global.getSlot("Primitive"));
String.getSlot("+").apply(["1", "2"]);
\end{urbiassert}

\item[asPrimitive] Return \lstinline|this|.
\begin{urbiassert}
Float.getSlot("+").asPrimitive === Float.getSlot("+");
\end{urbiassert}
\end{urbiscriptapi}


%%% Local Variables:
%%% mode: latex
%%% TeX-master: "../urbi-sdk"
%%% ispell-personal-dictionary: "../urbi.dict"
%%% End:

%% Copyright (C) 2009-2010, Gostai S.A.S.
%%
%% This software is provided "as is" without warranty of any kind,
%% either expressed or implied, including but not limited to the
%% implied warranties of fitness for a particular purpose.
%%
%% See the LICENSE file for more information.

\section{Singleton}

A \dfn{singleton} is a prototype that cannot be cloned. All prototypes
derived of \lstinline{Singleton} are also singletons.

\subsection{Prototypes}
\begin{refObjects}
\item[Object]
\end{refObjects}

\subsection{Construction}

To be a singleton, the object must have \lstinline{Singleton} as a
prototype. The common way to do this is
%
\lstinline{var s = Singleton.new},
%
but this does not work : \lstinline|s| is not a new singleton, it is
the \lstinline|Singleton| itself since it cannot be cloned. There are
two other ways:

\begin{urbiscript}[firstnumber=1]
// Defining a new class and specifying Singleton as a parent.
class NewSingleton1: Singleton
{
  var asString = "NewSingleton1";
}|
var s1 = NewSingleton1.new;
[00000001] NewSingleton1
assert(s1 === NewSingleton1);
assert(NewSingleton1 !== Singleton);

// Create a new Object and set its prototype by hand.
var NewSingleton2 = Object.new|
var NewSingleton2.asString = "NewSingleton2"|
NewSingleton2.protos = [Singleton]|
var s2 = NewSingleton2.new;
[00000001] NewSingleton2
assert(s2 === NewSingleton2);
assert(NewSingleton2 !== Singleton);
\end{urbiscript}

\subsection{Slots}
\begin{urbiscriptapi}
\item[clone]
  Return \lstinline|this|.

\item \lstinline|'new'|\\
  Return \lstinline|this|.
\end{urbiscriptapi}

%%% Local Variables:
%%% mode: latex
%%% TeX-master: "../urbi-sdk"
%%% ispell-dictionary: "american"
%%% ispell-personal-dictionary: "../urbi.dict"
%%% fill-column: 76
%%% End:

\section{String}

A \dfn{string} is a sequence of characters.

\subsection{Prototypes}
\begin{itemize}
\item \refObject{Comparable}
\item \refObject{Orderable}
\item \refObject{RangeIterable}
\end{itemize}

\subsection{Construction}
Fresh Strings can easily be built using the literal syntax.  Several
escaping sequences (the traditional ones and \us specific ones) allow
to insert special characters.  Consecutive string literals are merged
together.  See \autoref{sec:us-syn-lit-string} for details and
examples.

A null String can also be obtained with \lstinline|String|'s
\lstinline|new| method.

\begin{urbiscript}
assert(String.new == "");
assert(String == "");
assert("123".new == "123");
\end{urbiscript}

\subsection{Slots}
\begin{itemize}
\item \lstinline|asFloat|\\
  If the whole content of \lstinline|this| is an integer, return its
  value, otherwise return an error.
\begin{urbiscript}[firstnumber=last]
assert("23.03".asFloat == 23.03);

"123abc".asFloat;
[00000001:error] !!! asFloat: unable to convert to float: "123abc"
\end{urbiscript}

\item \lstinline|asList|\\
  Return a List of one-letter Strings that, concataneted, equal
  \lstinline|this|.  This allows to use \lstinline|for| to iterate
  over the string.
\begin{urbiscript}[firstnumber=last]
assert("123".asList == ["1", "2", "3"]);
for (var v : "123")
  echo(v);
[00000001] *** 1
[00000001] *** 2
[00000001] *** 3
\end{urbiscript}

\item \lstinline|asPrintable|\\
  Return \lstinline|this| as a literal (escaped) string.
\begin{urbiscript}[firstnumber=last]
assert("foo".asPrintable == "\"foo\"");
assert("foo".asPrintable.asPrintable == "\"\\\"foo\\\"\"");
\end{urbiscript}

\item \lstinline|asString|\\
  Return \lstinline|this|.
\begin{urbiscript}[firstnumber=last]
assert("\"foo\"".asString == "\"foo\"");
\end{urbiscript}

\item \lstinline|distance(\var{other})|\\
  Return the
  \href{http://en.wikipedia.org/wiki/Damerau-Levenshtein_distance}
  {Damerau-Levenshtein distance} between \lstinline|this| and
  \var{other}.  The more alike the strings are, the smaller the
  distance is.
\begin{urbiscript}[firstnumber=last]
assert("foo".distance("foo") == 0);
assert("bar".distance("baz") == 1);
assert("foo".distance("bar") == 3);
\end{urbiscript}

\item \lstinline|fresh|\\
  Return a String that has never been used as an identifier, prefixed
  by \lstinline|this|.  It can safely be used with
  \lstinline|Object.setSlot| and so forth.
\begin{urbiscript}[firstnumber=last]
assert(String.fresh == "_5");
assert("foo".fresh == "foo_6");
\end{urbiscript}

\item Character handling functions\\
  Here is a map of how the original 127-character ASCII set is
  considered by each function (a \textbullet{} indicates that the function
  returns true if all characters of \lstinline|this| are on the
  row).

\begin{tabular}{|l||l||c|c|c|c|c|c|c|c|c|c|c|}
  \hline
  &&&&&&&&&&&&\\
  ASCII values & Characters & \begin{sideways}iscntrl\end{sideways}
    & \begin{sideways}isspace\end{sideways}
    & \begin{sideways}isupper\end{sideways}
    & \begin{sideways}islower\end{sideways}
    & \begin{sideways}isalpha\end{sideways}
    & \begin{sideways}isdigit\end{sideways}
    & \begin{sideways}isxdigit\end{sideways}
    & \begin{sideways}isalnum\end{sideways}
    & \begin{sideways}ispunct\end{sideways}
    & \begin{sideways}isgraph\end{sideways}
    & \begin{sideways}print \end{sideways}\\ \hline \hline
  0x00 .. 0x08 & & \textbullet & & & & & & & & & &\\ \hline
  0x09 .. 0x0D & \textbackslash{}t, \textbackslash{}f,
  \textbackslash{}v, \textbackslash{}n, \textbackslash{}r &
  \textbullet & \textbullet & & & & & & & & &\\ \hline
  0x0E .. 0x1F & & \textbullet & & & & & & & & & &\\ \hline
  0x20 & space (' ') & & \textbullet & & & & & & & & & \textbullet\\ \hline
  0x21 .. 0x2F & \verb|!"#$%&'()*+,-./| & & & & & & & & & \textbullet & \textbullet & \textbullet\\ \hline
  0x30 .. 0x39 & \verb|0-9| & & & & & & \textbullet & \textbullet & \textbullet & & \textbullet & \textbullet\\ \hline
  0x3a .. 0x40 & \verb|:;<=>?@| & & & & & & & & & \textbullet & \textbullet & \textbullet\\ \hline
  0x41 .. 0x46 & \verb|A-F| & & & \textbullet & & \textbullet & & \textbullet & \textbullet & & \textbullet & \textbullet\\ \hline
  0x47 .. 0x5A & \verb|G-Z| & & & \textbullet & & \textbullet & & & \textbullet & & \textbullet & \textbullet\\ \hline
  0x5B .. 0x60 & \verb|[\]^{}_`| & & & & & & & & & \textbullet & \textbullet & \textbullet\\ \hline
  0x61 .. 0x66 & \verb|a-f| & & & & \textbullet & \textbullet & & \textbullet & \textbullet & & \textbullet & \textbullet\\ \hline
  0x67 .. 0x7A & \verb|g-z| & & & & \textbullet & \textbullet & & & \textbullet & & \textbullet & \textbullet\\ \hline
  0x7B .. 0x7E & \verb-{|}~- & & & & & & & & & \textbullet & \textbullet & \textbullet\\ \hline
  0x7F & (DEL) &  \textbullet & & & & & & & & & &\\
  \hline
\end{tabular}

\begin{urbiscript}[firstnumber=last]
assert("".isDigit);
assert("0123456789".isDigit);
assert(!"a".isDigit);

assert("".isLower);
assert("lower".isLower);
assert(! "Not Lower".isLower);

assert("".isUpper);
assert("UPPER".isUpper);
assert(! "Not Upper".isUpper);
\end{urbiscript}

\item \lstinline|join(\var{list}, \var{prefix}, \var{suffix})|\\
  Glue the result of \lstinline|asString| applied to the members of
  \var{list}, separated by \lstinline|this|, and embedded in a pair
  \var{prefix}/var{suffix}.
\begin{urbiscript}[firstnumber=last]
assert("|".join([1, 2, 3], "(", ")")      == "(1|2|3)");
assert(", ".join([1, [2], "3"], "[", "]") == "[1, [2], 3]");
\end{urbiscript}

\item \lstinline|replace(\var{from}, \var{to})|\\
  Replace every occurrence of the string \var{from} in
  \lstinline|this| by \var{to}, and return the result.
  \lstinline|this| is not modified.
\begin{urbiscript}[firstnumber=last]
assert("Hello, World!".replace("Hello", "Bonjour")
                      .replace("World!", "Monde !") ==
       "Bonjour, Monde !");
\end{urbiscript}

\item \lstinline|size|\\
  Return the size of the string.
\begin{urbiscript}[firstnumber=last]
assert("foo".size == 3);
assert("".size == 0);
\end{urbiscript}

\item \lstinline|toLower|\\
  Make lower case every upper case character in \lstinline|this| and
  return the result.  \lstinline|this| is not modified.
\begin{urbiscript}[firstnumber=last]
assert("Hello == World!".toLower, "hello, world!");
\end{urbiscript}

\item \lstinline|toUpper|\\
  Make upper case every lower case character in \lstinline|this| and
  return the result.  \lstinline|this| is not modified.
\begin{urbiscript}[firstnumber=last]
assert("Hello == World!".toUpper, "HELLO, WORLD!");
\end{urbiscript}

\item \lstinline|'%'(\var{args})|\\
  It is an equivalent of \lstinline|Formatter.new(this) % \var{args}|.
  See \refObject{Formatter}.
%  This construct is actually more
%  powerful than this, since it relies on
%  \href{http://www.boost.org/doc/libs/1_39_0/libs/format/doc/format.html,
%    Boost.Format}.  For instance:
\begin{urbiscript}[firstnumber=last]
assert("%s + %s = %s" % [1, 2, 3] == "1 + 2 = 3");
\end{urbiscript}

\item \lstinline|'*'(\var{n})|\\
  Concatenate \lstinline|this| \var{n} times.
\begin{urbiscript}[firstnumber=last]
assert("foo" * 0 == "");
assert("foo" * 1 == "foo");
assert("foo" * 3 == "foofoofoo");
\end{urbiscript}

\item \lstinline|'+'(\var{other})|\\
  Concatenate \lstinline|this| and \lstinline|\var{other}.asString|.
\begin{urbiscript}[firstnumber=last]
assert("foo" + "bar" == "foobar");
assert("foo" + "" == "foo");
assert("foo" + 3 == "foo3");
assert("foo" + [1, 2, 3] == "foo[1, 2, 3]");
\end{urbiscript}

\item \lstinline|'<'(\var{other})|\\
  Whether \lstinline|this| is lexicographically before \var{other},
  which must be a String.
\begin{urbiscript}[firstnumber=last]
assert("" < "a");
assert(!("a" < ""));
assert("a" < "b");
assert(!("a" < "a"));
\end{urbiscript}

\item \lstinline|'[]'(\var{from})|\\
  \lstinline|'[]'(\var{from}, \var{to})|\\
  Return the substring starting at \var{from}, up to and not including
  \var{to} (which defaults to \var{to} + 1).
\begin{urbiscript}[firstnumber=last]
assert("foobar"[0, 3] == "foo");
assert("foobar"[0] == "f");
\end{urbiscript}

\item \lstinline|'[]='(\var{from}, \var{other})|\\
  \lstinline|'[]='(\var{from}, \var{to}, \var{other})|\\
  Replace the substring starting at \var{from}, up to and not including
  \var{to} (which defaults to \var{to} + 1), by \var{other}.  Return
  \var{other}.

  Beware that this routine is imperative: it changes the value of
  \lstinline|this|.
\begin{urbiscript}[firstnumber=last]
var s1 = "foobar" | var s2 = s1 |
assert((s1[0, 3] = "quux") == "quux");
assert(s1 == "quuxbar");
assert(s2 == "quuxbar");
assert((s1[4, 7] = "") == "");
assert(s2 == "quux");
\end{urbiscript}
\end{itemize}

%%% Local Variables:
%%% mode: latex
%%% TeX-master: "../urbi-sdk"
%%% End:

%% Copyright (C) 2009-2012, Gostai S.A.S.
%%
%% This software is provided "as is" without warranty of any kind,
%% either expressed or implied, including but not limited to the
%% implied warranties of fitness for a particular purpose.
%%
%% See the LICENSE file for more information.

\section{System}
Details on the architecture the \urbi server runs on.

\subsection{Prototypes}
\begin{refObjects}
\item[Object]
\end{refObjects}

\subsection{Slots}
\begin{urbiscriptapi}
\item[_exit](<status>)%
  Shut the server down brutally: the connections are not closed, and
  the resources are not explicitly released (the operating system
  reclaims most of them: memory, file descriptors and so forth).
  Architecture dependent.


\item[arguments] The list of the command line arguments passed to the user
  script.  This is especially useful in scripts.
\begin{shell}[alsolanguage={[Interactive]urbiscript}]
$ cat >echo <<EOF
#! /usr/bin/env urbi
System.arguments;
shutdown;
EOF
$ chmod +x echo
$ ./echo 1 2 3
[00000172] ["1", "2", "3"]
$ ./echo -x 12 -v "foo"
[00000172] ["-x", "12", "-v", "foo"]
\end{shell}

\begin{urbicomment}
//#timeout 3
import System.*;
\end{urbicomment}

\item['assert'](<assertion>)%
  Unless \refSlot{ndebug} is true, throw an error if
  \var{assertion} is not verified.  See also the assertion support in
  \us, \autoref{sec:lang:assert}.
\begin{urbiscript}
'assert'(true);
'assert'(42);
'assert'(1 == 1 + 1);
[00000002:error] !!! failed assertion: 1.'=='(1.'+'(1))
\end{urbiscript}


\item[assert_](<assertion>, <message>)%
  If \var{assertion} does not evaluate to true, throw the failure
  \var{message}.
\begin{urbiscript}
assert_(true,       "true failed");
assert_(42,         "42 failed");
assert_(1 == 1 + 1, "one is not two");
[00000001:error] !!! failed assertion: one is not two
\end{urbiscript}


\item[assert_op](<operator>, <lhs>, <rhs>)%
  Deprecated, use \lstinline|assert| instead, see \autoref{sec:lang:assert}.


%% \item[breakpoint]


\item[currentRunner]  An obsolete alias for \refSlot[Job]{current}.


\item[cycle]%
  The number of execution cycles since the beginning. \experimental
\begin{urbiscript}
{
  var first = cycle ; var second = cycle ;
  assert(first + 1 == second);
  first = cycle | second = cycle ;
  assert(first == second);
};
\end{urbiscript}


\item[env]
  A \refObject{Dictionary} containing the current
  environment of \urbi.  See also \refSlot{env.init}.
\begin{urbiassert}
(env["MyVar"] = 12) == "12";
env["MyVar"] == "12";

// A child process that uses the environment variable.
System.system("exit $MyVar") >> 8 ==
       {if (Platform.isWindows) 0 else 12};
(env["MyVar"] = 23) == "23";
System.system("exit $MyVar") >> 8 ==
       {if (Platform.isWindows) 0 else 23};

// Defining to empty is defining, unless you are on Windows.
(env["MyVar"] ="") == "";
env["MyVar"].isNil == Platform.isWindows;

env["UndefinedEnvironmentVariable"].isNil;
!env["PATH"].isNil;

(env["MyVar"] = 12) == "12";
!env["MyVar"].isNil;
env.erase("MyVar") == "12";
env["MyVar"].isNil;
\end{urbiassert}


\item[env.init]%
  Refresh the \urbi environment by fetching all the environment variables.
  Beware that \refSlot{env} is not updated when calling \lstinline{getenv},
  \lstinline{setenv} or \lstinline{unsetenv} from the C library.  Initialize
  it first and then manipulate your environment as a simple
  \refObject{Dictionary}.
\begin{urbiassert}
env.init == env;
!env["USER"].isNil;
\end{urbiassert}


\item[eval](<source>, <target> = this)%
  Evaluate the \us \var{source}, and return its result.  See also
  \refSlot{loadFile}.  The \var{source} must be complete, yet the terminator
  (e.g., \samp{;}) is not required.
\begin{urbiassert}
eval("1+2") == 1+2;
eval("\"x\" * 10") == "x" * 10;
eval("eval(\"1\")") == 1;
eval("{ var x = 1; x + x; }") == 2;
\end{urbiassert}

The evaluation is performed in the context of the current object (\this) or
\var{target} if specified.  In particular, to create local variables, create
scopes.
\begin{urbiassert}
// Create a slot in the current object.
eval("var a = 23;") == 23;
this.a == 23;

eval("var a = 3", Global) == 3;
Global.a == 3;
\end{urbiassert}

  Exceptions are thrown on error (including syntax errors).
\begin{urbiscript}
// Scanner errors.
eval("#");
[00000004:error] !!! 1.1: syntax error: invalid character: `#'
[00000005:error] !!!    called from: eval

// Syntax errors.
eval("3; 1 * * 2");
[00000002:error] !!! 1.8: syntax error: unexpected *
[00000003:error] !!!    called from: eval

// Exceptions.
eval("1/0");
[00008316:error] !!! 1.1-3: /: division by 0
[00008316:error] !!!    called from: eval
try
{
  eval("1/0")
}
catch (var e)
{
  assert
  {
    e.isA(Exception.Primitive);
    e.location.asString  == "1.1-3";
    e.routine            == "/";
    e.message            == "division by 0";
  }
};
\end{urbiscript}

  Warnings are reported.

\begin{urbiscript}
eval("new Object");
[00001388:warning] !!! 1.1-10: `new Obj(x)' is deprecated, use `Obj.new(x)'
[00001388:warning] !!!    called from: eval
[00001388] Object_0x1001b2320
\end{urbiscript}

  Nested calls to \refSlot{eval} behave as expected.  The locations in the
  inner calls refer to the position inside the evaluated string.

\begin{urbiscript}
eval("/");
[00001028:error] !!! 1.1: syntax error: unexpected /
[00001028:error] !!!    called from: eval

eval("eval(\"/\")");
[00001032:error] !!! 1.1: syntax error: unexpected /
[00001032:error] !!!    called from: 1.1-9: eval
[00001032:error] !!!    called from: eval

eval("eval(\"eval(\\\"/\\\")\")");
[00001035:error] !!! 1.1: syntax error: unexpected /
[00001035:error] !!!    called from: 1.1-9: eval
[00001035:error] !!!    called from: 1.1-19: eval
[00001035:error] !!!    called from: eval
\end{urbiscript}


\item[getenv](<name>)%
  Deprecated function use \lstinline|env[\var{name}]| instead.
  The value of the environment variable \var{name} as a \refObject{String}
  if set, \refObject{nil} otherwise.  See also \refSlot{env}, \refSlot{setenv}
  and \refSlot{unsetenv}.
\begin{urbiassert}
getenv("UndefinedEnvironmentVariable").isNil;
[01234567:warning] !!! `System.getenv(that)' is deprecated, use `System.env[that]'
!getenv("PATH").isNil;
[01234567:warning] !!! `System.getenv(that)' is deprecated, use `System.env[that]'
\end{urbiassert}


\item[getLocale](<category>)%
  A \refObject{String} denoting the locale set for \var{category}, or
  raise an error.  See \refSlot{setLocale} for more details.
\begin{urbiscript}
getLocale("LC_IMAGINARY");
[00006328:error] !!! getLocale: invalid category: LC_IMAGINARY
\end{urbiscript}


\item[load](<file>, <target> = this)%
  Look for \var{file} in the \urbi path (\autoref{sec:tools:envvars}), and
  load it in the context of \var{target}.  See also \refSlot{loadFile}.
  Throw a \refSlot[Exception]{FileNotFound} error if the file cannot be
  found.  Return the last value of the file.
\begin{urbiscript}
// Create the file ``123.u'' that contains exactly ``var t = 123;''.
File.save("123.u", "var t = 123;");
assert
{
  load("123.u") == 123;
  this.t == 123;

  load("123.u", Global) == 123;
  Global.t == 123;
};
\end{urbiscript}


\item[loadFile](<file>, <target> = this)%
  Load the \us file \var{file} in the context of \var{target}.  Behaves like
  \refSlot{eval} applied to the content of \var{file}.  Throw a
  \refSlot[Exception]{FileNotFound} error if the file cannot be found.
  Return the last value of the file.
\begin{urbiscript}
// Create the file ``123.u'' that contains exactly ``var y = 123;''.
File.save("123.u", "var y = 123;");
assert
{
  loadFile("123.u") == 123;
  this.y == 123;

  loadFile("123.u", Global) == 123;
  Global.y == 123;
};

\end{urbiscript}


\item[loadLibrary](<library>)%
  Load the library \var{library}, to be found in
  \refSlot[UObject]{searchPath}.  The \var{library} may be a
  \refObject{String} or a \refObject{Path}.  The \Cxx symbols are made
  available to the other \Cxx components.  See also \refSlot{loadModule}.


\item[loadModule](<module>)%
  Load the \UObject \var{module}.  Same as \refSlot{loadLibrary}, except
  that the low-level \Cxx symbols are not made ``global'' (in the sense of
  the shared library loader).


\item[lobbies] Bounce to \refSlot[Lobby]{instances}.


\item[lobby] Bounce to \refSlot[Lobby]{lobby}.


\item[maybeLoad](<file>, <channel> = Channel.null)%
  Look for \var{file} in the \urbi path (\autoref{sec:tools:envvars}).
  If the file is found announce on \var{Channel} that \var{file} is
  about to be loaded, and load it.

\begin{urbiscript}
// Create the file ``123.u'' that contains exactly ``123;''.
File.save("123.u", "123;");
assert
{
  maybeLoad("123.u") == 123;
  maybeLoad("u.123").isVoid;
};
\end{urbiscript}


\item[ndebug] If true, do not evaluate the assertions.  See
  \autoref{sec:lang:assert}.
\begin{urbiscript}
function one() { echo("called!"); 1 }|;
assert(!System.ndebug);

assert(one);
[00000617] *** called!

// Beware of copy-on-write.
System.ndebug = true|;
assert(one);

System.ndebug = false|;
assert(one);
[00000622] *** called!
\end{urbiscript}


%% \item[nonInterruptible]


\item[PackageInfo] See \refObject{System.PackageInfo}.


\item[period] The \dfn{period} of the \urbi kernel.  Influences the
  trajectories (\refObject{TrajectoryGenerator}), and the \UObject
  monitoring.  Defaults to 20ms.
\begin{urbiassert}
System.period == 20ms;
\end{urbiassert}


\item[Platform] See \refObject{System.Platform}.


\item[profile](<function>)%
  Compute some measures during the execution of \var{function} and return
  the results as a \refObject{Profile} object. A \refObject{Profile} details
  information about time, function calls and scheduling.


\item[programName] The path under which the \urbi process was called.
  This is typically \file{.../urbi} (\autoref{sec:tools:urbi}) or
  \file{.../urbi-launch} (\autoref{sec:tools:urbi-launch}).
\begin{urbiassert}
Path.new(System.programName).basename
  in ["urbi", "urbi.exe", "urbi-launch", "urbi-launch.exe"];
\end{urbiassert}


\item[reboot] Restart the \urbi server.  Architecture dependent.


\item[redefinitionMode] Switch the current job in redefinition mode
  until the end of the current scope.  While in redefinition mode,
  setSlot on already existing slots will overwrite the slot instead of
  erring.

\begin{urbiscript}
var Global.x = 0;
[00000001] 0
{
  System.redefinitionMode;
  // Not an error
  var Global.x = 1;
  echo(Global.x);
};
[00000002] *** 1
// redefinitionMode applies only to the scope.
var Global.x = 0;
[00000003:error] !!! slot redefinition: x
\end{urbiscript}


\item[requireFile](<file>, <target>)%
  Load \var{file} in the context of \var{target} if it was not loaded before
  (with \refSlot{load} or \refSlot{requireFile}). Unlike \refSlot{load},
  \lstinline{requireFile} always returns \lstinline|void|. If \var{file} is
  being loaded concurrently \lstinline{requireFile} waits until the loading
  is finished.

\begin{urbiscript}
// Create the file ``test.u'' that echoes a string.
File.save("test1.u", "echo(\"test 1\"); 1;");
requireFile("test1.u");
[00000001] *** test 1
requireFile("test1.u");
// File is not re-loaded

File.save("test2.u", "echo(\"test 2\"); 2;");
load("test2.u");
[00000004] *** test 2
[00000004] 2
requireFile("test2.u");
load("test2.u");
[00000006] *** test 2
[00000006] 2
\end{urbiscript}

  The \var{target} is not taken into account to check whether the file has
  already been loaded: if you require twice the same file with two different
  targets, it will be loaded only for the first.

\begin{urbiscript}
requireFile("test2.u", Global);
\end{urbiscript}


\item[resetStats]%
  Reinitialize the \refSlot{stats} computation.
\begin{urbiassert}
 0  < System.stats["cycles"];
System.resetStats.isVoid;
 1 == System.stats["cycles"];
\end{urbiassert}


\item[scopeTag] Bounce to \refSlot[Tag]{scope}.


\item[searchFile](<file>)%
  Look for \var{file} in the \refSlot{searchPath} and return its
  \refObject{Path}.  Throw a \refSlot[Exception]{FileNotFound} error if the
  file cannot be found.
\begin{urbiscript}
// Create the file ``123.u'' that contains exactly ``123;''.
File.save("123.u", "123;");
assert
{
  searchFile("123.u") == Path.cwd / Path.new("123.u");
};
\end{urbiscript}


\item[searchPath] The \urbi path (i.e., the directories where the \us files
  are looked for, see \autoref{sec:tools:envvars}) as a \refObject{List} of
  \refObject[Path]{Paths}.
\begin{urbiassert}
System.searchPath.isA(List);
System.searchPath[0].isA(Path);
\end{urbiassert}


\item[setenv](<name>, <value>)%
  Deprecated, use \lstinline|env[\var{name}] = \var{value}| instead.  Set
  the environment variable \var{name} to \lstinline|\var{value}.asString|,
  and return this value.  See also \refSlot{env}, \refSlot{getenv} and
  \refSlot{unsetenv}.
  \begin{windows}
    Under Windows, setting to an empty value is equivalent to
    making undefined.
  \end{windows}

\begin{urbiassert}
setenv("MyVar", 12) == "12";
[00000001:warning] !!! `System.setenv(var, value)' is deprecated, \
[:]                       use `System.env[var] = value'
env["MyVar"] == "12";
\end{urbiassert}


\item[setLocale](<category>, <locale> = "")%
  Change the system's \dfn{locale} for the \var{category} to \var{locale}
  and return void.  If \var{locale} is empty, then use the locale specified
  by the user's environment (e.g., the environment variables).  The
  \var{category} can be:
  \begin{sublist}
    \begin{envs}
    \item[LC\_ALL] Overrides all the following categories.
    \item[LC\_COLLATE] Controls how string sorting is performed.
    \item[LC\_CTYPE] Change what characters are considered as letters and so
      on.
    \item[LC\_MESSAGES] The catalog of translated messages.  This category
      is not supported by Microsoft Windows.
    \item[LC\_MONETARY] How to format monetary values.
    \item[LC\_NUMERIC] Set a locale for formatting numbers.
    \item[LC\_TIME] Set a locale for formatting dates and times.
    \end{envs}
  \end{sublist}
  With \command{urbi} is run, it does \emph{not} change its locales: it
  defaults to the ``good old C mode'', which corresponds to the \samp{C} (or
  \samp{POSIX}) locale.  See also \refSlot{getLocale}.
\begin{urbicomment}
// Be sure not to depend on the user environment.
for (var i : ["ALL", "COLLATE", "CTYPE", "MESSAGES", "MONETARY",
              "NUMERIC", "TIME"])
  env.erase("LC_" + i);
\end{urbicomment}
\begin{urbiassert}
// Initially they are all set to "C".
getLocale("LC_ALL")     == "C";
getLocale("LC_CTYPE")   == "C";
getLocale("LC_NUMERIC") == "C";

// Windows does not understand the "fr_FR" locale, it supports "French"
// which actually denotes "French_France.1252".
var fr_FR =
  { if (System.Platform.isWindows) "French_France.1252" else "fr_FR.utf8" };
// Changing one via the environment does not affect the others.
(env["LC_CTYPE"] = fr_FR) == fr_FR;
getLocale("LC_CTYPE")   == "C";
setLocale("LC_CTYPE").isVoid;
getLocale("LC_CTYPE")   == fr_FR;
getLocale("LC_NUMERIC") == "C";

// Changing one via setLocale does not change the others either.
setLocale("LC_CTYPE", "C").isVoid;
getLocale("LC_CTYPE")   == "C";
getLocale("LC_NUMERIC") == "C";

// The environment variable LC_ALL overrides all the others.
env["LC_ALL"] = fr_FR;
setLocale("LC_ALL").isVoid;
getLocale("LC_ALL")     == fr_FR;
getLocale("LC_CTYPE")   == fr_FR;
getLocale("LC_NUMERIC") == fr_FR;

// Explicit changes of LC_ALL overrides all the others.
setLocale("LC_ALL", "C").isVoid;
getLocale("LC_ALL")     == "C";
getLocale("LC_CTYPE")   == "C";
getLocale("LC_NUMERIC") == "C";
\end{urbiassert}

  On invalid requests, raise an error.
\begin{urbiscript}
setLocale("LC_IMAGINARY");
[00006328:error] !!! setLocale: invalid category: LC_IMAGINARY

env["LC_ALL"] = "elfic"|;
setLocale("LC_ALL");
[00024950:error] !!! setLocale: cannot set locale LC_ALL to elfic

setLocale("LC_ALL", "klingon");
[00074958:error] !!! setLocale: cannot set locale LC_ALL to klingon
\end{urbiscript}


\item[shiftedTime] The number of seconds elapsed since the \urbi server was
  launched.  Contrary to \refSlot{time}, time spent in frozen code is not
  counted.
\begin{urbiassert}
{ var t0 = shiftedTime | sleep(1s) | shiftedTime - t0 }.round ~= 1;

  1 ==
  {
    var t = Tag.new|;
    var t0 = time|;
    var res;
    t: { sleep(1s) | res = shiftedTime - t0 },
    t.freeze;
    sleep(1s);
    t.unfreeze;
    sleep(1s);
    res.round;
  };
\end{urbiassert}


\item[shutdown](<exit_status> = 0)%
  Have the \urbi server shut down, with exit status \var{exit\_status}.  All
  the connections are closed, the resources are released.  Architecture
  dependent.


\item[sleep](<duration> = inf)%
  Suspend the execution for \var{duration} seconds.  No CPU cycle is wasted
  during this wait. If no \var{duration} is given the execution is suspended
  indefinitely.

\begin{urbiassert}
(time - {sleep(1s); time}).round == -1;
\end{urbiassert}


\item[spawn](<function>, <clear>)%
  Deprecated internal function.  Bounces to
  \lstinline|\var{function}.spawn(\var{clear})|, see \refSlot[Code]{spawn}.
\begin{urbiassert}
System.spawn(closure () { echo(1) }, true).isA(Job);
[00016657:warning] !!! `System.spawn' is deprecated, use `Code.spawn'
[00016659] *** 1
\end{urbiassert}


\item[stats]%
  A \refObject{Dictionary} containing information about the execution cycles
  of \urbi.  This is an internal feature made for developers, it might be
  changed without notice.  See also \refSlot{resetStats}.  These statistics
  make no sense in \option{--fast} mode (\autoref{sec:tools:urbi:opt}).
\begin{urbicomment}
//#no-fast
\end{urbicomment}
\begin{urbiassert}
var stats = System.stats;

stats.isA(Dictionary);
stats.keys.sort == ["cycles",
                    "cyclesMin", "cyclesMean", "cyclesMax",
                    "cyclesVariance", "cyclesStdDev"].sort;
// Number of cycles.
0 < stats["cycles"];
// Cycles duration.
0 <= stats["cyclesMin"] <= stats["cyclesMean"] <= stats["cyclesMax"];

stats["cyclesVariance"].isA(Float);
stats["cyclesStdDev"].isA(Float);
\end{urbiassert}


%% \item[stopall]


\item[system](<command>)%
  Ask the operating system to run the \var{command}.  This is typically used
  to start new processes.  The exact syntax of \var{command} depends on your
  system.  On Unix systems, this is typically \file{/bin/sh}, while Windows
  uses \file{command.exe}.

  Return the exit status.

  \begin{windows}
    Under Windows, the exit status is always 0.
  \end{windows}

\begin{urbiassert}
System.system("exit 0") == 0;
System.system("exit 23") >> 8
       == { if (System.Platform.isWindows) 0 else 23 };
\end{urbiassert}


\item[time] The number of seconds elapsed since the \urbi server was
  launched.  See also \refObject{Date}.  In presence of a frozen
  \refObject{Tag}, see also \refSlot{shiftedTime}.
\begin{urbiassert}
{ var t0 = time | sleep(1s) | time - t0 }.round ~= 1;

  2 ==
  {
    var t = Tag.new|;
    var t0 = time|;
    var res;
    t: { sleep(1s) | res = time - t0 },
    t.freeze;
    sleep(1s);
    t.unfreeze;
    sleep(1s);
    res.round;
  };
\end{urbiassert}


\item[timeReference]%
  The ``origin of time'' of this run of \urbi, as a \refObject{Date}.  It is
  a constant during the run.  Basically, \lstinline|System.time| is about
  \lstinline|Date.now - System.timeReference|.  See also \refSlot{time} and
  \refSlot[Date]{now}.

\begin{urbiscript}
var t1 = System.timeReference|;
sleep(1s);
var t2 = System.timeReference|;
assert
{
  t1 == t2;
  t1.isA(Date);
  (Date.now - (System.timeReference + System.time)) < 0.5s;
};
\end{urbiscript}
\begin{urbicomment}
removeSlots("t1", "t2");
\end{urbicomment}

\item[unsetenv](<name>)%
  Deprecated use \lstinline|env.erase (\var{name})| instead.
  Undefine the environment variable \var{name}, return its previous value.
  See also \refSlot{env}, \refSlot{getenv} and \refSlot{setenv}.

\begin{urbiassert}
(env["MyVar"] = 12) == "12";
!env["MyVar"].isNil;
unsetenv("MyVar") == "12";
[01234567:warning] !!! `System.unsetenv(var)' is deprecated, use `System.env.erase(var)'
env["MyVar"].isNil;
\end{urbiassert}


\item[version]%
  The version of \usdk.  A string composed of two or more numbers separated
  by periods: \samp{"\packageVersion"}.
\begin{urbiassert}
System.version in Regexp.new("\\d+(\\.\\d+)+");
\end{urbiassert}
\end{urbiscriptapi}

%%% Local Variables:
%%% coding: utf-8
%%% mode: latex
%%% TeX-master: "../urbi-sdk"
%%% ispell-dictionary: "american"
%%% ispell-personal-dictionary: "../urbi.dict"
%%% fill-column: 76
%%% End:

%% Copyright (C) 2009-2010, Gostai S.A.S.
%%
%% This software is provided "as is" without warranty of any kind,
%% either expressed or implied, including but not limited to the
%% implied warranties of fitness for a particular purpose.
%%
%% See the LICENSE file for more information.

\section{System.Platform}
A description of the platform (the computer) the server is running on.

\subsection{Prototypes}
\begin{refObjects}
\item[Object]
\end{refObjects}

\subsection{Slots}
\begin{urbiscriptapi}
\item[host] The type of system \usdk runs on.  Composed of the CPU, the
  vendor, and the OS.
\begin{urbiassert}
System.Platform.host ==
  "%s-%s-%s" % [System.Platform.hostCpu,
                System.Platform.hostVendor,
                System.Platform.hostOs];
\end{urbiassert}

\item[hostAlias] The name of the system \usdk runs on as the person who
  compiled it decided to name it.  Typically empty, it is fragile to depend
  on it.
\begin{urbiassert}
System.Platform.hostAlias.isA(String);
\end{urbiassert}

\item[hostCpu] The CPU type of system \usdk runs on.  The following values
  are those for which Gostai provides binary builds.
\begin{urbiassert}
System.Platform.hostCpu in ["i386", "i686", "x86_64"];
\end{urbiassert}

\item[hostOs] The OS type of system \usdk runs on.  For instance
  \lstinline|darwin9.8.0| or \lstinline|linux-gnu| or \lstinline|mingw32|.

\item[hostVendor] The vendor type of system \usdk runs on.  The following
  values are those for which Gostai provides binary builds.
\begin{urbiassert}
System.Platform.hostVendor in ["apple", "pc", "unknown"];
\end{urbiassert}

\item[isWindows] Whether running under Windows.
\begin{urbiassert}
System.Platform.isWindows in [true, false];
\end{urbiassert}

\item[kind] Either \code{"POSIX"} or \code{"WIN32"}.
\begin{urbiassert}
System.Platform.kind in ["POSIX", "WIN32"];
\end{urbiassert}
\end{urbiscriptapi}

%%% Local Variables:
%%% mode: latex
%%% TeX-master: "../../urbi-sdk"
%%% ispell-dictionary: "american"
%%% ispell-personal-dictionary: "../../urbi.dict"
%%% fill-column: 76
%%% End:

%% Copyright (C) 2009-2011, Gostai S.A.S.
%%
%% This software is provided "as is" without warranty of any kind,
%% either expressed or implied, including but not limited to the
%% implied warranties of fitness for a particular purpose.
%%
%% See the LICENSE file for more information.

\section{Tag}

A \dfn{tag} is an object meant to label blocks of code in order to control
them externally.  Tagged code can be frozen, resumed, stopped\ldots See also
\autoref{sec:tut:tags}.

\subsection{Examples}

\subsubsection{Stop}
\label{sec:specs:tag:stop}

To \dfn{stop} a tag means to kill all the code currently running that it
labels.  It does not affect ``newcomers''.

\begin{urbiscript}[firstnumber=1]
var t = Tag.new|;
var t0 = time|;
t: every(1s) echo("foo"),
sleep(2.2s);
[00000158] *** foo
[00001159] *** foo
[00002159] *** foo

t.stop;
// Nothing runs.
sleep(2.2s);

t: every(1s) echo("bar"),
sleep(2.2s);
[00000158] *** bar
[00001159] *** bar
[00002159] *** bar

t.stop;
\end{urbiscript}

\refSlot[Tag]{stop} can be used to inject a return value to a tagged
expression.

\begin{urbiscript}[firstnumber=1]
var t = Tag.new|;
var res;
detach(res = { t: every(1s) echo("computing") })|;
sleep(2.2s);
[00000001] *** computing
[00000002] *** computing
[00000003] *** computing

t.stop("result");
assert(res == "result");
\end{urbiscript}

Be extremely cautious, the precedence rules can be misleading:
\lstinline|\var{var} = \var{tag}: \var{exp}| is read as
\lstinline|(\var{var} = \var{tag}): \var{exp}| (i.e., defining \var{var} as
an alias to \var{tag} and using it to tag \var{exp}), not as
\lstinline|\var{var} = { \var{tag}: \var{exp} }|.  Contrast the following
example, which is most probably an error from the user, with the previous,
correct, one.

\begin{urbiscript}[firstnumber=1]
var t = Tag.new("t")|;
var res;
res = t: every(1s) echo("computing"),
sleep(2.2s);
[00000001] *** computing
[00000002] *** computing
[00000003] *** computing

t.stop("result");
assert(res == "result");
[00000004:error] !!! failed assertion: res == "result" (Tag<t> != "result")
\end{urbiscript}


\subsubsection{Block/unblock}
\label{sec:specs:tag:block}

To \dfn{block} a tag means:
\begin{itemize}
\item Stop running pieces of code it labels (as with
  \refSlot{stop}).
\item Ignore new pieces of code it labels (this differs from
  \refSlot{stop}).
\end{itemize}

One can \dfn{unblock} the tag.  Contrary to
\refSlot{freeze}/\refSlot{unfreeze}, tagged code does not resume the
execution.

\begin{urbiscript}[firstnumber=1]
var ping = Tag.new("ping")|;
ping:
  every (1s)
    echo("ping"),
assert(!ping.blocked);
sleep(2.1s);
[00000000] *** ping
[00002000] *** ping
[00002000] *** ping

ping.block;
assert(ping.blocked);

ping:
  every (1s)
    echo("pong"),

// Neither new nor old code runs.
ping.unblock;
assert(!ping.blocked);
sleep(2.1s);

// But we can use the tag again.
ping:
  every (1s)
    echo("ping again"),
sleep(2.1s);
[00004000] *** ping again
[00005000] *** ping again
[00006000] *** ping again
\end{urbiscript}

As with \refSlot{stop}, one can force the value of stopped
expressions.

\begin{urbiassert}[firstnumber=1]
{
  var t = Tag.new;
  var res = [];
  for (3)
    detach(res << {t: sleep});
  t.block("foo");
  res;
}
==
["foo", "foo", "foo"];
\end{urbiassert}

\subsubsection{Freeze/unfreeze}
\label{sec:specs:tag:freeze}

To \dfn{freeze} a tag means holding the execution of code it labels.
This applies to code already being run, and ``arriving'' pieces of code.

\begin{urbiscript}[firstnumber=1]
var t = Tag.new|;
var t0 = time|;
t: every(1s) echo("time   : %.0f" % (time - t0)),
sleep(2.2s);
[00000158] *** time   : 0
[00001159] *** time   : 1
[00002159] *** time   : 2

t.freeze;
assert(t.frozen);
t: every(1s) echo("shifted: %.0f" % (shiftedTime - t0)),
sleep(2.2s);
// The tag is frozen, nothing is run.

// Unfreeze the tag: suspended code is resumed.
// Note the difference between "time" and "shiftedTime".
t.unfreeze;
assert(!t.frozen);
sleep(2.2s);
[00004559] *** shifted: 2
[00005361] *** time   : 5
[00005560] *** shifted: 3
[00006362] *** time   : 6
[00006562] *** shifted: 4
\end{urbiscript}


\subsubsection{Scope tags}
\label{sec:specs:tag:scope}

Scopes feature a \lstindex{scopeTag}, i.e., a tag which will be stop
when the execution reaches the end of the current scope.  This is
handy to implement cleanups, how ever the scope was exited from.

\begin{urbiscript}[firstnumber=1]
{
  var t = scopeTag;
  t: every(1s)
      echo("foo"),
  sleep(2.2s);
};
[00006562] *** foo
[00006562] *** foo
[00006562] *** foo

{
  var t = scopeTag;
  t: every(1s)
      echo("bar"),
  sleep(2.2s);
  throw 42;
};
[00006562] *** bar
[00006562] *** bar
[00006562] *** bar
[00006562:error] !!! 42
sleep(2s);
\end{urbiscript}

\subsubsection{Enter/leave events}
\label{sec:specs:tag:enter-leave}

Tags provide two events, \refSlot{enter} and \refSlot{leave}, that trigger
whenever flow control enters or leaves statements the tag.

\begin{urbiscript}[firstnumber=1]
var t = Tag.new("t");
[00000000] Tag<t>

at (t.enter?)
  echo("enter");
at (t.leave?)
  echo("leave");

t: {echo("inside"); 42};
[00000000] *** enter
[00000000] *** inside
[00000000] *** leave
[00000000] 42
\end{urbiscript}

This feature is fundamental; it is a concise and safe way to ensure code
will be executed upon exiting a chunk of code (like \acro{raii} in \Cxx or
\lstinline|finally| in Java). The exit code will be run no matter what the
reason for leaving the block was: natural exit, exceptions, flow control
statements like \lstinline|return| or \lstinline|break|, \ldots

For instance, suppose we want to make sure we turn the gas off when
we're done cooking. Here is the \emph{bad} way to do it:

\begin{urbiscript}
{
  function cook()
  {
    turnGasOn();
    // Cooking code ...
    turnGasOff();
  }|

  enterTheKitchen();
  cook();
  leaveTheKitchen();
};
\end{urbiscript}

This \lstinline|cook| function is wrong because there are several situations
where we could leave the kitchen with gas still turned on. Consider the
following cooking code:

\begin{urbiscript}
{
  function cook()
  {
    turnGasOn();

    if (mealReady)
    {
      echo("The meal is already there, nothing to do!");
      // Oops ...
      return;
    };

    for (var i in recipe)
      if (i in kitchen)
        putIngredient(i)
      else
        // Oops ...
        throw Exception("missing ingredient: %s" % i);

    // ...

    turnGasOff();
  }|
};
\end{urbiscript}

Here, if the meal was already prepared, or if an ingredient is missing, we
will leave the \lstinline|cook| function without executing the
\lstinline|turnGasOff| statement, through the \lstinline|return| statement
or the exception.  One correct way to ensure gas is necessarily turned off
is:

\begin{urbiscript}
{
  function cook()
  {
    var withGas = Tag.new("withGas");

    at (withGas.enter?)
      turnGasOn();
    // Even if exceptions are thrown or return is called,
    // the gas will be turned off.
    at (withGas.leave?)
      turnGasOff();

    withGas: {
      // Cooking code...
    }
  }|
};
\end{urbiscript}

Alternatively, the \lstinline|try|/\lstinline|finally| construct provides an
elegant means to achieve the same result (\autoref{sec:lang:except:finally}).

\begin{urbiscript}
{
  function cook()
  {
    try
    {
      turnGasOn();
      // Cooking code...
    }
    finally
    {
      // Even if exceptions are thrown or return is called,
      // the gas will be turned off.
      turnGasOff();
    }
  }|
};
\end{urbiscript}

\subsubsection{Begin/end}
\label{sec:specs:tag:begin-end}

The \refSlot{begin} and \refSlot{end} methods enable to monitor when code is
executed.  The following example illustrates the proper use of
\refSlot{enter} and \refSlot{leave} events
(\autoref{sec:specs:tag:enter-leave}), which are used to implement this
feature.

\begin{urbiscript}
var myTag = Tag.new("myTag");
[00000000] Tag<myTag>

myTag.begin: echo(1);
[00000000] *** myTag: begin
[00000000] *** 1

myTag.end: echo(2);
[00000000] *** 2
[00000000] *** myTag: end

myTag.begin.end: echo(3);
[00000000] *** myTag: begin
[00000000] *** 3
[00000000] *** myTag: end
\end{urbiscript}

\subsection{Construction}
\label{stdlib:tag:ctor}

As any object, tags are created using \lstinline{new} to create derivatives
of the \lstinline{Tag} object.  The name is optional, it makes easier to
display a tag and remember what it is.

\begin{urbiscript}[firstnumber=1]
// Anonymous tag.
var t1 = Tag.new;
[00000001] Tag<tag_8>

// Named tag.
var t2 = Tag.new("cool name");
[00000001] Tag<cool name>
\end{urbiscript}

\subsection{Slots}

\begin{urbiscriptapi}
\item[begin]
  A sub-tag that prints out "tag\_name: begin" each time flow control
  enters the tagged code. See \autoref{sec:specs:tag:begin-end}.

\item[block](<result> = void)%
  Block any code tagged by \this.  Blocked tags can be
  unblocked using \refSlot{unblock}.  If some \var{result} was
  specified, let stopped code return \var{result} as value.  See
  \autoref{sec:specs:tag:block}.

\item[blocked]
  Whether code tagged by \this is blocked.  See
  \autoref{sec:specs:tag:block}.

\item[end]
  A sub-tag that prints out "tag\_name: end" each time flow control
  leaves the tagged code. See \autoref{sec:specs:tag:begin-end}.

\item[enter] An event triggered each time the flow control enters the
  tagged code.  See \autoref{sec:specs:tag:enter-leave}.

\item[freeze] Suspend code tagged by \this, already running or
  forthcoming.  Frozen code can be later unfrozen using \refSlot{unfreeze}.
  See \autoref{sec:specs:tag:freeze}.

\item[frozen]
  Whether the tag is frozen. See  \autoref{sec:specs:tag:freeze}.

\item[leave] An event triggered each time flow control leaves the
  tagged code.  See \autoref{sec:specs:tag:enter-leave}.

\item[scope] Return a fresh Tag whose \refSlot{stop} will be invoked a the
  end of the current scope.  This function is likely to be removed.  See
  \autoref{sec:specs:tag:scope}.

\item[stop](<result> = void)%
  Stop any code tagged by \this.  If some \var{result} was
  specified, let stopped code return \var{result} as value.
  See \autoref{sec:specs:tag:stop}.

\item[tags] All the undeclared tags are created as slots in this
  object.  Using this feature is discouraged.
\begin{urbiscript}
{
  assert ("brandNewTag" not in Tag.tags.localSlotNames);
  brandNewTag: 1;
  assert ("brandNewTag" in Tag.tags.localSlotNames);
  assert (Tag.tags.brandNewTag.isA(Tag));
};
\end{urbiscript}

\item[unblock]
  Unblock \this.  See \autoref{sec:specs:tag:block}.

\item[unfreeze]
  Unfreeze code tagged by \this.  See
  \autoref{sec:specs:tag:freeze}.
\end{urbiscriptapi}

\subsection{Hierarchical tags}

Tags can be arranged in a parent/child relationship: any operation done on a
tag --- freezing, stopping, \ldots is also performed on its descendants.
Another way to see it is that tagging a piece of code with a child will also
tag it with the parent. To create a child Tag, simply clone its parent.

\begin{urbiscript}
var parent = Tag.new |
var child = parent.clone |

// Stopping parent also stops children.
{
  parent: {sleep(100ms); echo("parent")},
  child:  {sleep(100ms); echo("child")},
  parent.stop;
  sleep(200ms);
  echo("end");
};
[00000001] *** end

// Stopping child has no effect on parent.
{
  parent: {sleep(100ms); echo("parent")},
  child:  {sleep(100ms); echo("child")},
  child.stop;
  sleep(200ms);
  echo("end");
};
[00000002] *** parent
[00000003] *** end
\end{urbiscript}

Hierarchical tags are commonly laid out in slots so as to reflect their tag
hierarchy.

\begin{urbiunchecked}
var a = Tag.new;
var a.b = a.clone;
var a.b.c = a.b.clone;

a:     foo; // Tagged by a
a.b:   bar; // Tagged by a and b
a.b.c: baz; // Tagged by a, b and c
\end{urbiunchecked}

% FIXME: If we ever restore some sugar to create hierarchical tags, document
% it here

%%% Local Variables:
%%% mode: latex
%%% TeX-master: "../urbi-sdk"
%%% ispell-dictionary: "american"
%%% ispell-personal-dictionary: "../urbi.dict"
%%% fill-column: 76
%%% End:

%% Copyright (C) 2009-2011, Gostai S.A.S.
%%
%% This software is provided "as is" without warranty of any kind,
%% either expressed or implied, including but not limited to the
%% implied warranties of fitness for a particular purpose.
%%
%% See the LICENSE file for more information.

\section{Timeout}

Timeout objects can be used as \refObject[Tag]{Tags} to execute some code in
limited time.  See also the \lstinline|timeout| construct
(\autoref{sec:lang:timeout}).

\subsection{Examples}

Use it as a tag:

\begin{urbiscript}
var t = Timeout.new(300ms);
[00000000] Timeout_0x133ec0
t:{
  echo("This will be displayed.");
  sleep(500ms);
  echo("This will not.");
};
[00000000] *** This will be displayed.
[00000007:error] !!! Timeout_0x133ec0 has timed out.
\end{urbiscript}

The same Timeout, \lstinline|t| can be reused.  It is armed again each
time it is used to tag some code.

\begin{urbiscript}
t: { echo("Open"); sleep(1s); echo("Close"); };
[00000007] *** Open
[00000007:error] !!! Timeout_0x133ec0 has timed out.
t: { echo("Open"); sleep(1s); echo("Close"); };
[00000007] *** Open
[00000007:error] !!! Timeout_0x133ec0 has timed out.
\end{urbiscript}

%%% FIXME: Something is wrong here.
%%%
%%%A Timeout cannot be used several times at the same time though.
%%%
%%%\begin{urbiscript}
%%%// Note the commas.
%%%t: { echo("Open"); sleep(1s); echo("Close"); },
%%%[00035424:error] !!! Timeout_0x44e89e0 is already running
%%%t: { echo("Open"); sleep(1s); echo("Close"); },
%%%[00000007] *** Open
%%%[00000007:error] !!! Timeout_0x133ec0 has timed out.
%%%sleep(2s);
%%%\end{urbiscript}
Even if exceptions have been disabled, you can check whether the
count-down expired with \lstinline|timedOut|.

\begin{urbiscript}
t:sleep(500ms);
[00000007:error] !!! Timeout_0x133ec0 has timed out.
if (t.timedOut)
  echo("The Timeout expired.");
[00000000] *** The Timeout expired.
\end{urbiscript}

\subsection{Prototypes}
\begin{refObjects}
\item[Tag]
\end{refObjects}

\subsection{Construction}
At construction, a Timeout takes a duration, and a \refObject{Boolean}
stating whether an exception should be thrown on timeout (by default,
it does).

\begin{urbiscript}
Timeout.new(300ms);
[00000000] Timeout_0x953c1e0
Timeout.new(300ms, false);
[00000000] Timeout_0x953c1e0
\end{urbiscript}

\subsection{Slots}
\begin{urbiscriptapi}
\item[asTimeout] Return \this.
\begin{urbiscript}
{
  var t = Timeout.new(10);
  assert
  {
    Timeout.asTimeout === Timeout;
          t.asTimeout === t;
  };
};
\end{urbiscript}

\item[launch]
  Fire \this.

%% FIXME: there is more.
\end{urbiscriptapi}

%%% Local Variables:
%%% mode: latex
%%% TeX-master: "../urbi-sdk"
%%% ispell-dictionary: "american"
%%% ispell-personal-dictionary: "../urbi.dict"
%%% fill-column: 76
%%% End:

\section{void}

%%% Local Variables:
%%% mode: latex
%%% TeX-master: "../urbi-sdk"
%%% End:


%% Restore the definition of \section.
\let\section\sectionOrig

%%% Local Variables:
%%% mode: latex
%%% TeX-master: "../urbi-sdk"
%%% End:

% LocalWords:  CallMessage memoization callmsg lst eval xADDR createSlot POSIX
% LocalWords:  getSlot setSlot updateSlot Orderable Tuple
