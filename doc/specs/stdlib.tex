%% Redefine \section is this chapter so that we don't have to
%% call \labelObject each time.  See the bottom of this file for the
%% restoring of \section.
\let\sectionOrig\section
\renewcommand{\section}[1]{\sectionOrig{\labelObject{#1}#1}}

\chapter{\us Standard Library}
\label{sec:stdlib}

\section{Boolean}
\section{CallMessage}

\subsection{Examples}
\subsubsection{Evaluating an argument several times}
\label{sec:std-callmsg-examples-several}
\section{Channel}
\section{Code}
\section{Comparable}
\section{Dictionary}

\section{Event}

\section{Float}

A Float is a floating point number.  It is also used, in the current
version of \us, to represent integers.

\subsection{Prototypes}

\begin{refObjects}
\item[Comparable]
\item[Orderable]
\item[RangeIterable]
\end{refObjects}

\subsection{Construction}
\label{sec:float:ctor}

The most common way to create fresh floats is using the literal
syntax.  Numbers are composed of three parts:
\begin{description}
\item[integral] (mandatory) a non empty sequence of (decimal) digits;
\item[fractional] (optional) a period, and a non empty sequence of
  (decimal) digits;
\item[exponent] (optional) either \samp{e} or \samp{E}, an optional
  sign (\samp{+} or \samp{-}), then a non-empty sequence of digits.
\end{description}

In other words, float literals match the
\lstinline|[0-9]+(\.[0-9]+)?([eE][-+]?[0-9]+)?|
regular expression.  For instance:

\begin{urbiassert}
0 == 0000.0000;
// This is actually a call to the unary '+'.
+1 == 1;
0.123456 == 123456 / 1000000;
1e3 == 1000;
1e-3 == 0.001;
1.234e3 == 1234;
\end{urbiassert}

There are also some special numbers, \lstinline|nan|, \lstinline|inf|
(see below).

\begin{urbiassert}
Math.log(0) == -inf;
Math.exp(-inf) == 0;
(inf/inf).asString == "nan";
\end{urbiassert}

A null float can also be obtained with \lstinline|Float|'s
\lstinline|new| method.

\begin{urbiassert}
Float.new == 0;
\end{urbiassert}

\subsection{Slots}

\begin{urbiscriptapi}
\item[abs]
  Absolute value of the target.
\begin{urbiassert}
(-5).abs == 5;
  0 .abs == 0;
  5 .abs == 5;
\end{urbiassert}

\item[acos]
  Arccosine of the target.
\begin{urbiassert}
0.acos == Float.pi/2;
1.acos == 0;
\end{urbiassert}

\item[asBool]
  Whether non null.
\begin{urbiassert}
0.asBool == false;
0.1.asBool == true;
(-0.1).asBool == true;
inf.asBool == true;
nan.asBool == true;
\end{urbiassert}

\item[asFloat]
  Return the target.
\begin{urbiassert}
51.asFloat == 51;
\end{urbiassert}

\item[asList]
  Bounces to \lstinline|seq|.  It is therefore possible to use the
  various flavors of \lstinline|for|-range loops on integers:
\begin{urbiassert}
{
  var res = [];
  for (var i : 3)
    res << i;
  res
}
== [0, 1, 2];

{
  var res = [];
  for|(var i : 3)
    res << i;
  res
}
== [0, 1, 2];

{
  var res = [];
  for&(var i : 3)
    res << i;
  res.sort
}
== [0, 1, 2];
\end{urbiassert}%>>

\item[asin]
  Arcsine of the target.
\begin{urbiassert}
0.asin == 0;
\end{urbiassert}

\item[asString]
  Return a string representing the target.
\begin{urbiassert}
42.asString == "42";
\end{urbiassert}

\item[atan]
  Return the arctangent of the target.
\begin{urbiassert}
0.atan == 0;
1.atan == Float.pi/4;
\end{urbiassert}

\item \lstinline|'bitand'(\var{that})|\\
  The bitwise-and between \lstinline|this| and \var{that}.
\begin{urbiassert}
(3 bitand 6) == 2;
\end{urbiassert}

\item \lstinline|'bitor'(\var{that})|\\
  Bitwise-or between \lstinline|this| and \var{that}.
\begin{urbiassert}
(3 bitor 6) == 7;
\end{urbiassert}

\item[clone]
  Return a fresh Float with the same value as the target.
\begin{urbiscript}
var x = 0;
[00000000] 0
var y = x.clone;
[00000000] 0
x === y;
[00000000] false
\end{urbiscript}

\item[compl]
  The complement to 1 of the target interpreted as a 32 bits integer.
\begin{urbiassert}
compl 0 == 4294967295;
compl 4294967295 == 0;
\end{urbiassert}

\item[cos]
  Cosine of the target.
\begin{urbiassert}
0.cos == 1;
Float.pi.cos == -1;
\end{urbiassert}

\item \lstinline|each(\var{fun})|\\
  Call the functional argument \var{fun} on every integer from 0 to
  target - 1, sequentially.  The number must be non-negative.
\begin{urbiassert}
{
  var res = [];
  3.each(function (i) { res << 100 + i });
  res
}
== [100, 101, 102];

{
  var res = [];
  for(var x : 3) { res << x; sleep(20ms); res << (100 + x); };
  res
}
== [0, 100, 1, 101, 2, 102];

{
  var res = [];
  0.each (function (i) { res << 100 + i });
  res
}
== [];
\end{urbiassert}

\item \lstinline'each|(\var{fun})'\\
  Call the functional argument \var{fun} on every integer from 0 to
  target - 1, with tight sequentiality.  The number must be
  non-negative.
\begin{urbiassert}
{
  var res = [];
  3.'each|'(function (i) { res << 100 + i });
  res
}
== [100, 101, 102];

{
  var res = [];
  for|(var x : 3) { res << x; sleep(20ms); res << (100 + x); };
  res
}
== [0, 100, 1, 101, 2, 102];
\end{urbiassert}%>>>>>>

\item \lstinline|each&(\var{fun})|\\
  Call the functional argument \var{fun} on every integer from 0 to
  target - 1, concurrently.  The number must be non-negative.
\begin{urbiassert}
{
  var res = [];
  for& (var x : 3) { res << x; sleep(30ms); res << (100 + x) };
  res
}
== [0, 1, 2, 100, 101, 102];
\end{urbiassert}%>>>>

\item[exp]
  Exponential of the target.
\begin{urbiscript}
1.exp;
[00000000] 2.71828
\end{urbiscript}

\item \lstinline|format(\var{finfo})|\\
  Format according to the \refObject{FormatInfo} object \var{finfo}.
  The precision, \lstinline|\var{finfo}.precision|, sets the maximum
  number of digits after decimal point when in fixed or scientific
  mode, and in total when in default mode.  Beware that 0 plays a
  special role, as it is not a ``significant'' digit.

  \begin{windows}
    Under Windows the behavior differs slightly.
  \end{windows}
\begin{urbiassert}
"%1.0d" % 0.1 == "0.1";
"%1.0d" % 1.1 == {if (System.Platform.isWindows) "1.1" else "1"};

"%1.0f" % 0.1 == "0";
"%1.0f" % 1.1 == "1";
\end{urbiassert}

\item[inf]
  Return the infinity.
\begin{urbiscript}
Float.inf;
[00000000] inf
\end{urbiscript}

\item[limit_digits]
  Number of digits (in \lstinline|Float.limit_radix| base) in the
  mantissa.
\begin{urbiassert}
Float.limit_digits;
\end{urbiassert}

\item[limit_digits10]
  Number of digits (in decimal base) that can be represented without
  change.
\begin{urbiassert}
Float.limit_digits10;
\end{urbiassert}

\item[limit_epsilon]
  Machine epsilon (the difference between 1 and the least value
  greater than 1 that is representable).
\begin{urbiassert}
1 != 1 + Float.limit_epsilon;
1 == 1 + Float.limit_epsilon / 2;
\end{urbiassert}

\item[limit_max]
  Maximum finite value.
\begin{urbiassert}
Float.limit_max     != Float.inf;
Float.limit_max * 2 == Float.inf;
\end{urbiassert}

\item[limit_max_exponent]
  Maximum integer value for the exponent that generates a normalized
  floating-point number.
\begin{urbiassert}
Float.inf != Float.limit_radix ** (Float.limit_max_exponent - 1);
Float.inf == Float.limit_radix ** Float.limit_max_exponent;
\end{urbiassert}

\item[limit_max_exponent10]
  Maximum integer value such that 10 raised to that power generates a
  normalized finite floating-point number.
\begin{urbiassert}
Float.inf != 10 ** Float.limit_max_exponent10;
Float.inf == 10 ** (Float.limit_max_exponent10 + 1);
\end{urbiassert}

\item[limit_min]
  Minimum positive normalized value.
\begin{urbiassert}
0 != Float.limit_min;
\end{urbiassert}

\item[limit_min_exponent]
  Minimum negative integer value for the exponent that generates a
  normalized floating-point number.
\begin{urbiassert}
0 != Float.limit_radix ** Float.limit_min_exponent;
\end{urbiassert}

\item[limit_min_exponent10]
  Minimum negative integer value such that 10 raised to that power
  generates a normalized floating-point number.
\begin{urbiassert}
0 != 10 ** Float.limit_min_exponent10;
\end{urbiassert}

\item[limit_radix]
  Base of the exponent of the representation.
\begin{urbiassert}
Float.limit_radix == 2;
\end{urbiassert}

\item[log]
  The logarithm of the target.
\begin{urbiassert}
0.log == -inf;
1.log == 0;
1.exp.log == 1;
\end{urbiassert}

\item \lstinline|max(\var{arg1}, ...)|\\
  Bounces to \lstinline|List.max| on \lstinline|[this, \var{arg1}, ...]|.
\begin{urbiassert}
1.max == 1;
1.max(2, 3) == 3;
3.max(1, 2) == 3;
\end{urbiassert}

\item \lstinline|min(\var{arg1}, ...)|\\
  Bounces to \lstinline|List.min| on \lstinline|[this, \var{arg1}, ...]|.
\begin{urbiassert}
1.min == 1;
1.min(2, 3) == 1;
3.min(1, 2) == 1;
\end{urbiassert}

\item[nan]
  The ``not a number'' special float value.  More precisely, this
  returns the ``quiet NaN'', i.e., it is propagated in the various
  computations, it does not raise exceptions.
\begin{urbiscript}
Float.nan;
[00000000] nan
(Float.nan + Float.nan) / (Float.nan - Float.nan);
[00000000] nan
\end{urbiscript}

A {NaN} has one distinctive property over the other Floats: it is
equal to no other float, not even itself.  This behavior is mandated
by the \wref[IEEE_754-2008]{IEEE 754-2008} standard.
\begin{urbiassert}
{ var n = Float.nan; n === n};
{ var n = Float.nan; n  != n};
\end{urbiassert}

\item[pi]
  $\pi$.
\begin{urbiassert}
Float.pi.cos ** 2 + Float.pi.sin ** 2 == 1;
\end{urbiassert}

\item[random]
  A random integer between 0 (included) and the target (excluded).
\begin{urbiscript}
20.map(function (dummy) { 5.random });
[00000000] [1, 2, 1, 3, 2, 3, 2, 2, 4, 4, 4, 1, 0, 0, 0, 3, 2, 4, 3, 2]
\end{urbiscript}

\item[round]
  The target, rounded to the nearest integer.
\begin{urbiassert}
1.6.round == 2;
1.4.round == 1;
\end{urbiassert}

\item[seq]
  The sequence of integers from 0 to \lstinline|this| - 1 as a list.
  The number must be non-negative.
\begin{urbiassert}
3.seq == [0, 1, 2];
0.seq == [];
(-1).seq;
[00004586:error] !!! seq: expected non-negative integer, got -1
\end{urbiassert}

\item[sign]
  Return 1 if \lstinline|this| is positive, 0 if it is null, -1
  otherwise.
\begin{urbiassert}
(-1164).sign == -1;
0.sign       == 0;
(1164).sign  == 1;
\end{urbiassert}

\item[sin]
  The sine of the target.
\begin{urbiassert}
0.sin == 0;
\end{urbiassert}

\item[sqr]
  Square of the target.
\begin{urbiassert}
32.sqr == 1024;
32.sqr == 32 ** 2;
\end{urbiassert}

\item[sqrt]
  The square root of the target.
\begin{urbiassert}
1024.sqrt == 32;
1024.sqrt == 1024 ** 0.5;
\end{urbiassert}

\item[srandom]
  Initialized the seed used by the random function.  As opposed to common
  usage, you should not use
\begin{urbiunchecked}
{
  var now = Date.now.timestamp;
  now.srandom;
  var list1 = 20.map(function (dummy) { 5.random });
  now.srandom;
  var list2 = 20.map(function (dummy) { 5.random });
  assert
  {
    list1 == list2;
  }
};
\end{urbiunchecked}

\item[tan]
  Tangent of the target.
\begin{urbiscript}
assert(0.tan == 0);
(Float.pi/4).tan;
[00000000] 1
\end{urbiscript}

\item \lstinline|times(\var{fun})|\\
  Call the functional argument \var{fun} \lstinline|this| times.

\begin{urbiscript}
3.times(function () { echo("ping")});
[00000000] *** ping
[00000000] *** ping
[00000000] *** ping
\end{urbiscript}

\item[trunc]
  Return the target truncated.
\begin{urbiassert}
1.9.trunc == 1;
(-1.9).trunc == -1;
\end{urbiassert}

\item \lstinline|'^'(\var{that})|\\
  Bitwise exclusive or between \lstinline|this| and \var{that}.
\begin{urbiassert}
(3 ^ 6) == 5;
\end{urbiassert}

\item \lstinline|'>>'(\var{that})|\\%>>
  \lstinline|this| shifted by \var{that} bits towards the right.
\begin{urbiassert}
4 >> 2 == 1;
\end{urbiassert}

\item \lstinline|'<'(\var{that})|\\
  Whether \lstinline|this| is less than \var{b}. The other comparison
  operators (\lstinline|<=|, \lstinline|>|, \ldots) can thus also be
  applied on floats since Float inherits \refObject{Orderable}.
\begin{urbiassert}
  0 < 1;
!(1 < 0);
\end{urbiassert}

\item \lstinline|'<<'(\var{that})|\\
  \lstinline|this| shifted by \var{that} bit towards the left.
\begin{urbiassert}
4 << 2 == 16;
\end{urbiassert}

\item \lstinline|'-'(\var{that})|\\
  \lstinline|this| subtracted by \var{b}.
\begin{urbiassert}
6 - 3 == 3;
\end{urbiassert}

\item \lstinline|'+'(\var{that})|\\
  The sum of \lstinline|this| and \var{that}.
\begin{urbiassert}
1 + 1 == 2;
\end{urbiassert}

\item \lstinline|'/'(\var{that})|\\
  The quotient of \lstinline|this| divided by \var{that}.
\begin{urbiassert}
50 / 10 == 5;
10 / 50 == 0.2;
\end{urbiassert}

\item \lstinline|'%'(\var{that})|\\
  \lstinline|this| modulo \var{b}.
\begin{urbiassert}
50 % 11 == 6;
\end{urbiassert}

\item \lstinline|'*'(\var{that})|\\
  Product of \lstinline|this| by \var{that}.
\begin{urbiassert}
2 * 3 == 6;
\end{urbiassert}

\item \lstinline|'**'(\var{that})|\\
  \lstinline|this| to the \var{that} power (${this}^{that}$).
\begin{urbiassert}
2 ** 10 == 1024;
\end{urbiassert}

\item \lstinline|'=='(\var{that})|\\
  Whether \lstinline|this| equals \var{that}.
\begin{urbiassert}
  1 == 1;
!(1 == 2);
\end{urbiassert}
\end{urbiscriptapi}

%%% Local Variables:
%%% mode: latex
%%% TeX-master: "../urbi-sdk"
%%% ispell-dictionary: "american"
%%% ispell-personal-dictionary: "../urbi.dict"
%%% End:


\section{Group}

\section{Lazy}

\subsection{Rationale}

\dfn{Lazies} are objects that hold a lazy value, that is, a not yet evaluated
value. They provide facilities to evaluate their content only once
(\dfn{memoization}) or several times. Lazy are essentially used in call
messages, to represent lazy arguments, as described in
\autorefObject{CallMessage}.

\subsection{Construction}

Lazies are seldom instantiated manually. They are mainly created
automatically when a lazy function call is made (see
\autoref{sec:us-fun-callmsg}). One can however create a lazy value with the
standard \lstinline|new| method of \lstinline|Lazy|, giving it an
argument-less function which evaluates to the lazified value
(\autoref{lst:new-lazy}).

\begin{urbiscript}[caption=Creating a lazy value, label=lst:new-lazy,
  float=\floatpos]
Lazy.new(closure () { /* Value to lazify */ });
[00000000:hide] Lazy_0xADDR
\end{urbiscript}

\subsection{Methods}
\subsubsection{eval}

The \lstinline|eval| method forces evaluation of the held lazy
value. Two calls to \lstinline|eval| will systematically evaluate the
expression twice, which can be useful to duplicates its side effects.

\subsubsection{value}

The \lstinline|value| method returns the held value, potentially
evaluating it before. \lstinline|value| performs memoization, that is,
only the first call will actually evaluate the expression, subsequent
calls will return the cached value. Unless you want to explicitly
trigger side effects from the expression by evaluating it several
time, this should be preferred over \lstinline|eval| to avoid
evaluating the expression several times uselessly.

\subsection{Examples}

\subsubsection{Evaluating once}

One usage of lazy values is to avoid evaluating an expression unless
it's actually needed, because it's expensive or has undesired side
effects. \autoref{lst:lazy-once} presents a situation where an
expensive-to-compute value (\lstinline|heavy_computation|) might be
needed zero, one or two times. The objective is to save time by:

\begin{itemize}
\item Not evaluating it if it's not needed.
\item Evaluating it only once if it's needed one or two time.
\end{itemize}

We thus make the wanted expression lazy, and use the \lstinline|value|
method to fetch its value when needed.

\begin{urbiscript}[caption=, label=lst:lazy-once, float=\floatpos]
// This function supposedly performs expensive computations.
function heavy_computation()
{
  echo("Heavy computation");
  return 1 + 1;
};
[00000000:hide] function () {
[:]  echo("Heavy computation");
[:]  return 1 . '+'(1);
[:]}

// We want to do the heavy computations only if needed,
// and make it a lazy value to be able to evaluate it "on demand".
var v = Lazy.new(closure () { heavy_computation() });
[00000000] Lazy_0xADDR
/* some code */;
// So far, the value was not needed, and heavy_computation
// was not evaluated.
/* some code */;
// If the value is needed, heavy_computation is evaluated.
v.value;
[00000000] *** Heavy computation
[00000000] 2
// If the value is needed a second time, heavy_computation
// is not reevaluated.
v.value;
[00000000] 2
\end{urbiscript}

\subsubsection{Evaluating several times}

Evaluating a lazy several times only makes sense with lazy arguments
and call messages. See example with call messages in
\autoref{sec:std-callmsg-examples-several}.

\section{List}

\lstinline|List|s implement potentially-empty ordered (heterogeneous)
collections of elements.

\subsection{Prototypes}

\begin{itemize}
\item \refObject{Object}
\item \refObject{RangeIterable}
\item \refObject{Orderable}
\end{itemize}

\subsection{Construction}

List can be created with their literal syntax: a possibly empty
sequence of expressions in square brackets, separated by commas.
Non-empty list may actually \emph{terminate} with a comma, rather than
\emph{separate}; in other words, an optional trailing comma is accepted.

\begin{urbiscript}
[]; // The empty list
[00000000] []
[1, "2", [3,],];
[00000000] [1, "2", [3]]
\end{urbiscript}

\subsection{Slots}

\begin{itemize}
\item \lstinline|all(\var{fun})|\\
  % FIXME: link to predicate glossary entry
  Return whether all the members of the target verify the predicate
  \var{fun}.

\begin{urbiassert}[firstnumber=last]
// Are all elements positive?
! [-2, 0, 2, 4].all(function (e) { e > 0 });
// Are all elements even?
[-2, 0, 2, 4].all(function (e) { e % 2 == 0 });
\end{urbiassert}

\item \lstinline|any(\var{fun})|\\
  % FIXME: link to predicate glossary entry
  Whether at least one of the members of the target verifies the
  predicate \var{fun}.

\begin{urbiassert}[firstnumber=last]
// Is there any even element?
! [-3, 1, -1].any(function (e) { e % 2 == 0 });
// Is there any positive element?
[-3, 1, -1].any(function (e) { e > 0 });
\end{urbiassert}

\item \lstinline|asBool|\\
  Whether not empty.
\begin{urbiassert}[firstnumber=last]
[].asBool == false;
[1].asBool == true;
\end{urbiassert}

\item \lstinline|asList|\\
Return the target.

\begin{urbiassert}[firstnumber=last]
[0, 1, 2].asList == [0, 1, 2];
\end{urbiassert}

\item \lstinline|asString|\\
  A string describing the list.  Uses \lstinline|asPrintable| on its
  members, so that, for instance, strings are displayed with quotes.

\begin{urbiassert}[firstnumber=last]
[0, [1], "2"].asString == "[0, [1], \"2\"]";
\end{urbiassert}

\item \lstinline|back|\\
Return the last element of the target. An error if the target is empty.

\begin{urbiscript}[firstnumber=last]
assert([0, 1, 2].back == 2);
[].back;
[00000000:error] !!! back: cannot be applied onto empty list
\end{urbiscript}

\item \lstinline|clear|\\
  Empty the target.

\begin{urbiscript}[firstnumber=last]
var x = [0, 1, 2];
[00000000] [0, 1, 2]
assert(x.clear == []);
\end{urbiscript}

\item \lstinline|each(\var{fun})|\\
  Apply the given functional value \var{fun} on all members,
  sequentially.

\begin{urbiscript}[firstnumber=last]
[0, 1, 2].each(function (v) {echo (v * v); echo (v * v)});
[00000000] *** 0
[00000000] *** 0
[00000000] *** 1
[00000000] *** 1
[00000000] *** 4
[00000000] *** 4
\end{urbiscript}

\item \lstinline|'each&'(\var{fun})|\\
Apply the given functional value on all members simultaneously.

\begin{urbiscript}[firstnumber=last]
[0, 1, 2].'each&'(function (v) {echo (v * v); echo (v * v)});
[00000000] *** 0
[00000000] *** 1
[00000000] *** 4
[00000000] *** 0
[00000000] *** 1
[00000000] *** 4
\end{urbiscript}

\item \lstinline|empty|\\
  Whether the target is empty.

\begin{urbiassert}[firstnumber=last]
[].empty;
! [1].empty;
\end{urbiassert}

\item \lstinline|filter(\var{fun})|\\
  The list of all the members of the target that verify the predicate
  \var{fun}.

\begin{urbiassert}[firstnumber=last]
// Keep only odd numbers.
[0, 1, 2, 3, 4, 5].filter(function (v) {v % 2 == 1}) == [1, 3, 5];
\end{urbiassert}

\item \lstinline|foldl(\var{action}, \var{value})|\\
  \wref[Fold_(higher-order_function)]{Fold},
  also known as \dfn{reduce} or \dfn{accumulate}, computes a result
  from a list.  Starting from \var{value} as the initial result, apply
  repeatedly the binary \var{action} to the current result and the
  next member of the list, from left to right.  For instance, if
  \var{action} were the binary addition and \var{value} were 0, then
  folding a list would compute the sum of the list, including for
  empty lists.

\begin{urbiscript}[firstnumber=last]
[].foldl(function (a, b) { a + b }, 0);
[00000000] 0
[1, 2, 3].foldl(function (a, b) { a + b }, 0);
[00000000] 6
[1, 2, 3].foldl(function (a, b) { a - b }, 0);
[00000000] -6
\end{urbiscript}

\item \lstinline|front|\\
  Return the first element of the target. An error if the target is
  empty.

\begin{urbiscript}[firstnumber=last]
assert([0, 1, 2].front == 0);
[].front;
[00000000:error] !!! front: cannot be applied onto empty list
\end{urbiscript}

\item \lstinline|has(\var{that})|\\
  Whether one of the members of the target equals the argument.

\begin{urbiassert}[firstnumber=last]
[0, 1, 2].has(1);
! [0, 1, 2].has(5);
\end{urbiassert}

\item \lstinline|hasSame(\var{that})|\\
  Return whether one of the member of the target is physically equal
  to the argument.

\begin{urbiscript}[firstnumber=last]
var y = 1;
[00000000:hide] 1
[0, y, 2].hasSame(1);
[00000000] false
[0, y, 2].hasSame(y);
[00000000] true
\end{urbiscript}

\item \lstinline|head|\\
  Synonym for \lstinline|front|.

\item \lstinline|insertBack(\var{that})|\\
  Insert the given element at the end of the target.

\begin{urbiscript}[firstnumber=last]
var z = [0, 1];
[00000000] [0, 1]
assert(z.insertBack(2) == [0, 1, 2]);
assert(z == [0, 1, 2]);
\end{urbiscript}

\item \lstinline|insertFront(\var{that})|\\
  Insert the given element at the beginning of the target.

\begin{urbiscript}[firstnumber=last]
var a = [1, 2];
[00000000] [1, 2]
assert(a.insertFront(0) == [0, 1, 2]);
assert(a == [0, 1, 2]);
\end{urbiscript}

\item \lstinline|join(\var{sep} = "", \var{prefix} = "", \var{suffix} = "")|\\
  Bounces to \lstinline|String.join|, see \refObject{String}.

\begin{urbiassert}[firstnumber=last]
["", "ob", ""].join                == "ob";
["", "ob", ""].join("a")           == "aoba";
["", "ob", ""].join("a", "B", "b") == "Baobab";
\end{urbiassert}

\item \lstinline|keys()|\\
  The list of valid indexes.  This allows uniform iteration over a
  \refObject{Dictionary} or a \refObject{List}.

\begin{urbiscript}[firstnumber=last]
{
  var l = ["a", "b", "c"];
  assert(l.keys == [0, 1, 2]);
  assert({
           var res = [];
           for (var k: l.keys)
             res << l[k];
           res
         }
         == l);
};
\end{urbiscript}

\item \lstinline|map(\var{fun})|\\
Apply the given functional value on every member, and return the list
of results.

\begin{urbiassert}[firstnumber=last]
[0, 1, 2, 3].map(function (v) { v % 2 == 0})
        == [true, false, true, false];
\end{urbiassert}

\item \lstinline|range(\var{begin}, \var{end} = nil)|\\
  Return a sub-range of the list, from the first index included to the
  second index excluded.  An error if out of bounds.  Negative indices
  are valid, and number from the end.

  If \var{end} is \lstinline|nil|, calling \lstinline|range(\var{n})
  is equivalent to calling \lstinline|range(0, \var{n})|.

\begin{urbiscript}[firstnumber=last]
do ([0, 1, 2, 3])
{
  assert
  {
    range(0, 0)   == [];
    range(0, 1)   == [0];
    range(1)      == [0];
    range(1, 3)   == [1, 2];

    range(-3, -2) == [1];
    range(-3, -1) == [1, 2];
    range(-3, 0)  == [1, 2, 3];
    range(-3, 1)  == [1, 2, 3, 0];
    range(-4, 4)  == [0, 1, 2, 3, 0, 1, 2, 3];
  };
}|;
[].range(1, 3);
[00428697:error] !!! range: invalid index: 1
\end{urbiscript}

\item \lstinline|remove(\var{val})|\\
  Remove all elements from the target that equals \var{val}.

\begin{urbiscript}[firstnumber=last]
var c = [0, 1, 0, 2, 0, 3];
[00000000] [0, 1, 0, 2, 0, 3]
assert(c.remove(0) == [1, 2, 3]);
assert(c == [1, 2, 3]);
\end{urbiscript}

\item \lstinline|removeBack|\\
  Remove and return the last element of the target. An error if the
  target is empty.

\begin{urbiscript}[firstnumber=last]
var t = [0, 1, 2];
[00000000] [0, 1, 2]
assert(t.removeBack == 2);
assert(t == [0, 1]);
[].removeBack;
[00000000:error] !!! removeBack: cannot be applied onto empty list
\end{urbiscript}

\item \lstinline|removeById(\var{that})|\\
  Remove all elements from the target that physically equals
  \var{that}.

\begin{urbiscript}[firstnumber=last]
var d = 1;
[00000000] 1
var e = [0, 1, d, 1, 2];
[00000000] [0, 1, 1, 1, 2]
assert(e.removeById(d) == [0, 1, 1, 2]);
assert(e == [0, 1, 1, 2]);
\end{urbiscript}

\item \lstinline|removeFront|\\
Remove and return the first element from the target. An error if the
target is empty.

\begin{urbiscript}[firstnumber=last]
var g = [0, 1, 2];
[00000000] [0, 1, 2]
assert(g.removeFront == 0);
assert(g == [1, 2]);
[].removeFront;
[00000000:error] !!! removeFront: cannot be applied onto empty list
\end{urbiscript}

\item \lstinline|reverse|\\
Return the target with the order of elements inverted.

\begin{urbiassert}[firstnumber=last]
[0, 1, 2].reverse == [2, 1, 0];
\end{urbiassert}

\item \lstinline|size|\\
Return the number of elements in the target.

\begin{urbiassert}[firstnumber=last]
[0, 1, 2].size == 3;
[].size == 0;
\end{urbiassert}

\item \lstinline|sort|\\
Return the target, sorted with respect to the \lstinline|<| criteria.

\begin{urbiassert}[firstnumber=last]
[1, 0, 3, 2].sort == [0, 1, 2, 3];
\end{urbiassert}

\item \lstinline|tail|\\
Return the target, minus the first element. An error if the target is
empty.

\begin{urbiscript}[firstnumber=last]
assert([0, 1, 2].tail == [1, 2]);
[].tail;
[00000000:error] !!! tail: cannot be applied onto empty list
\end{urbiscript}

\item \lstinline|'=='(\var{that})|\\
Check whether all elements in the target and \var{that}, are
equal two by two.

\begin{urbiassert}[firstnumber=last]
[0, 1, 2] == [0, 1, 2];
!([0, 1, 2] == [0, 0, 2]);
\end{urbiassert}

\item \lstinline|'[]'(\var{n})|\\
  Return the \var{n}th member of the target (indexing is
  zero-based). If \var{n} is negative, start from the end.  An error
  if out of bounds.

\begin{urbiscript}[firstnumber=last]
assert(["0", "1", "2"][0] == "0");
assert(["0", "1", "2"][2] == "2");
["0", "1", "2"][3];
[00007061:error] !!! []: invalid index: 3

assert(["0", "1", "2"][-1] == "2");
assert(["0", "1", "2"][-3] == "0");
["0", "1", "2"][-4];
[00007061:error] !!! []: invalid index: -4
\end{urbiscript}

\item \lstinline|'[]='(\var{index}, \var{value})|\\
  Assign \var{value} to the element of the target at the given
  \var{index}.

\begin{urbiscript}[firstnumber=last]
var f = [0, 1, 2];
[00000000] [0, 1, 2]
f[1] = 42;
[00000000] 42
assert(f == [0, 42, 2]);
\end{urbiscript}

\item \lstinline|'*'(\var{n})|\\
  Return the target, concatenated \var{n} times to itself.
\begin{urbiassert}[firstnumber=last]
[0, 1] * 3 == [0, 1, 0, 1, 0, 1];
\end{urbiassert}

  Note that since it is the very same list which is repeatedly
  concatenated (the content is not cloned), side-effects on one item
  will reflect on ``all the items''.

\begin{urbiscript}[firstnumber=last]
var l = [[]] * 3;
[00000000] [[], [], []]
l[0] << 1;
[00000000] [1]
l;
[00000000] [[1], [1], [1]]
\end{urbiscript}

\item \lstinline|'+'(\var{other})|\\
Return the concatenation of the target and the \var{other} list.

\begin{urbiassert}[firstnumber=last]
[0, 1] + [2, 3] == [0, 1, 2, 3];
\end{urbiassert}

\item \lstinline|'-'(\var{other})|\\
Return the target without all element that equals any element in the
\var(other) list.

\begin{urbiassert}[firstnumber=last]
[0, 1, 0, 2, 3] - [1, 2] == [0, 0, 3];
\end{urbiassert}

\item \lstinline|'<<'(\var{that})|\\
  A synonym for \lstinline|insertBack|.

\item \lstinline|'<'(\var{other})|\\
  Return whether the target is inferior to the \var{other} list. A
  list is inferior to another if at least one of its element differs
  from the other, and the first differing element is inferior to the
  other.

\begin{urbiassert}[firstnumber=last]
!([0, 1, 2] < [0, 1, 2]);
!([0, 1, 2] < [0, 0, 2]);
[0, 1, 2] < [0, 2, 2];
\end{urbiassert}

  Since List derives from \refObject{Orderable}, the other order-based
  operators are defined.

\begin{urbiassert}[firstnumber=last]
 [0, 1, 2] <= [0, 1, 2];
 [0, 1, 2] >= [0, 1, 2];
 [0, 1, 2] >  [0, 0, 2];
\end{urbiassert}
\end{itemize}

%%% Local Variables:
%%% mode: latex
%%% TeX-master: "../urbi-sdk"
%%% End:

% LocalWords:  lst asList asString foldl hasSame removeBack popback removeFront
% LocalWords:  popfront insertBack pushback insertFront pushfront urbi sameAs
% LocalWords:  removeById setNth


\section{Lobby}
\section{Object}

\subsection{Methods}

\subsubsection{createSlot}
\label{sec:std-object-createslot}

\subsubsection{getSlot}
\label{sec:std-object-getslot}

\subsubsection{setSlot}
\label{sec:std-object-setslot}

\subsubsection{updateSlot}
\label{sec:std-object-updateslot}

\section{Orderable}
\section{Pattern}
\section{Primitive}
\section{String}

\section{System}

\section{System.Platform}

A description of the platform (the computer) the server is running on.
\subsection{Slots}

\subsubsection{kind}
Either \code{"POSIX"} or \code{"WIN32"}.

\section{Tag}
\section{Tuple}
\section{void}


%% Restore the definition of \section.
\let\section\sectionOrig
%%% Local Variables:
%%% mode: latex
%%% TeX-master: "urbi-specs"
%%% End:
