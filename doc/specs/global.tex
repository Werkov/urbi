\section{Global}

\dfn{Global} is designed for the purpose of being global
namespace. Since \dfn{Global} is a prototype of \dfn{Object} and all
objects are an \dfn{Object}, all slots of \dfn{Global} are accessible from
anywhere.

\subsection{Prototypes}
\begin{itemize}
\item \lstinline|Global.uobjects|, see below.
\item \lstinline|Tag.tags| (see \refObject{Tag})
\item \refObject{Math}
\item \refObject{System}
\item \refObject{Object}
\end{itemize}

\subsection{Slots}
\begin{urbiscriptapi}
\item[Barrier] See \refObject{Barrier}.
\item[Binary] See \refObject{Binary}.
\item[CallMessage] See \refObject{CallMessage}.

\item[cerr] A predefined stream for error messages.  Strings sent to
  it are not escaped, contrary to regular streams (see
  \lstinline|output| for instance).
\begin{urbiscript}
cerr << "Message";
[00015895:error] Message
cerr << "\"quote\"";
[00015895:error] "quote"
\end{urbiscript}

\item[Channel] See \refObject{Channel}.

\item[clog] A predefined stream for log messages.  Strings are output
  escaped.
\begin{urbiscript}
clog << "Message";
[00015895:clog] "Message"
\end{urbiscript}

\item[Code] See \refObject{Code}.
\item[Comparable] See \refObject{Comparable}.

\item[cout] A predefined stream for output messages.  Strings are
  output escaped.
\begin{urbiscript}
cout << "Message";
[00015895:output] "Message"
cout << "\"quote\"";
[00015895:output] "\"quote\""
\end{urbiscript}

\item[Date] See \refObject{Date}.

\item \lstinline|detach(\var{exp})|\\
  Bounce to \lstinline|Control.detach|, see \refObject{Control}.

\item[Dictionary] See \refObject{Dictionary}.
\item[Directory] See \refObject{Directory}.

\item \lstinline|disown(\var{exp})|\\
  Bounce to \lstinline|Control.disown|, see \refObject{Control}.

\item[Duration] See \refObject{Duration}.

\item \lstinline|echo(\var{value}, \var{channel} = "")|\\
  Bounce to \lstinline|lobby.echo|, see \refObject{Lobby}.
\begin{urbiscript}
echo("111", "foo");
[00015895:foo] *** 111
echo(222, "");
[00051909] *** 222
echo(333);
[00055205] *** 333
\end{urbiscript}

\item[evaluate] This \refObject{UVar} provides a synchronous interface
  to the \urbi engine: write to it to ``send'' an expression to
  compute it, and ``read'' it to get the result.  This UVar is
  designed to be used from the \Cxx; it makes little sense in \us, use
  \lstinline|System.eval| instead, if it is really required (see
  \autoref{sec:k122:dollar}).  Since the semantics of the assignment
  requires that it evaluates to the right-hand side argument, reading
  \lstinline|evaluate| after the assignment is needed, which makes
  race conditions likely.  To avoid this, use \lstinline{|} (or better
  yet, do not use \lstinline|Global.evaluate| at all in \us).

\begin{urbiassert}
(evaluate = "1+2;") == "1+2;";
 evaluate == 3;
{ evaluate = "1+2;" | evaluate } == 3;
{ evaluate = "var x = 1;"; x } == 1;
\end{urbiassert}

  Errors raise an exception.

\begin{urbiscript}
evaluate = "1/0;";
[00087671:error] !!! Exception caught while processing notify on Global.evaluate:
[00087671:error] !!! 1.1-3: /: division by 0
[00087671] "1/0;"
\end{urbiscript}

\item[Event] See \refObject{Event}.
\item[Exception] See \refObject{Exception}.
\item[Executable] See \refObject{Executable}.
\item[external] An system object used to implement UObject support in
  \us.
\item[false]  See \autoref{sec:truth}.
\item[File] See \refObject{File}.
\item[Finalizable] See \refObject{Finalizable}.
\item[Float] See \refObject{Float}.
\item[FormatInfo] See \refObject{FormatInfo}.
\item[Formatter] See \refObject{Formatter}.

\item \lstinline|getProperty(\var{slotName}, \var{propName})| This
  wrapper around \lstinline|Object.getProperty| is actually a
  by-product of the existence of the \lstinline|Global.evaluate|
  \refObject{UVar}.

\item[Global] See \refObject{Global}.
\item[Group] See \refObject{Group}.
\item[InputStream] See \refObject{InputStream}.
%% FIXME: [00008768] ***   installUpdateHookStack : Code
\item \lstinline|isdef(\var{qualified-identifier})|\\
  Whether the \var{qualified-identifier} is defined.  It features some
  (fragile) magic to support an argument passed as a literal
  (\lstinline|isdef(foo)|), not a string (\lstinline|isdef("foo")|).
  It is not recommended to use this feature, which is provided for \us
  compatibility.  See \lstinline|Object.hasSlot| for a safer
  alternative.
\begin{urbiscript}
assert
{
  !isdef(a);
  !isdef(a.b);
  !isdef(a.b.c);
};

var a = Object.new|;
assert
{
   isdef(a);
  !isdef(a.b);
  !isdef(a.b.c);
};

var a.b = Object.new|;
assert
{
   isdef(a);
   isdef(a.b);
  !isdef(a.b.c);
};

var a.b.c = Object.new|;
assert
{
   isdef(a);
   isdef(a.b);
   isdef(a.b.c);
};
\end{urbiscript}


\item[Job] See \refObject{Job}.
\item[Kernel1] See \refObject{Kernel1}.
\item[Lazy] See \refObject{Lazy}.
\item[List] See \refObject{List}.
\item[Loadable] See \refObject{Loadable}.
\item[Lobby] See \refObject{Lobby}.
\item[Math] See \refObject{Math}.
\item \lstinline|methodToFunction(\var{name})|\\
  Create a function from the method \var{name} so that calling the
  function which arguments \lstinline|(\var{a}, \var{b}, ...)| is that
  same as calling \lstinline|\var{a}.\var{name}(\var{b}, ...)|.
\begin{urbiscript}
var uid_of = methodToFunction("uid")|;
assert
{
  uid_of(Object) == Object.uid;
  uid_of(Global) == Global.uid;
};
var '+_of' = methodToFunction("+")|;
assert
{
  '+_of'( 1,   2)  ==  1  + 2;
  '+_of'("1", "2") == "1" + "2";
  '+_of'([1], [2]) == [1] + [2];
};
\end{urbiscript}

\item[Mutex] See \refObject{Mutex}.
\item[nil] See \refObject{nil}.
\item[Object] See \refObject{Object}.
\item[Orderable] See \refObject{Orderable}.
\item[OutputStream] See \refObject{OutputStream}.
\item[Pair] See \refObject{Pair}.
\item[Path] See \refObject{Path}.
\item[Pattern] See \refObject{Pattern}.

\item \lstinline|persist(\var{exp})|\\
  Bounce to \lstinline|Control.persist|, see \refObject{Control}.

\item[Position] See \refObject{Position}.
\item[Primitive] See \refObject{Primitive}.
\item[Process] See \refObject{Process}.
\item[Profiling] See \refObject{Profiling}.
\item[PseudoLazy] See \refObject{PseudoLazy}.
\item[PubSub] See \refObject{PubSub}.
\item[RangeIterable] See \refObject{RangeIterable}.
\item[Regexp] See \refObject{Regexp}.
\item[Semaphore] See \refObject{Semaphore}.
\item[Server] See \refObject{Server}.
\item[Singleton] See \refObject{Singleton}.
\item[Socket] See \refObject{Socket}.
\item[String] See \refObject{String}.
\item[System] See \refObject{System}.
\item[Tag] See \refObject{Tag}.
\item[Timeout] See \refObject{Timeout}.
\item[TrajectoryGenerator] See \refObject{TrajectoryGenerator}.
\item[Triplet] See \refObject{Triplet}.
\item[true]  See \autoref{sec:truth}.
\item[Tuple] See \refObject{Tuple}.
\item[UObject] See \refObject{UObject}.
\item[uobjects] An object whose slots are all the \refObject{UObject}
  bound into the system.

%% FIXME: [00008801] ***   uobjects_handle : Finalizable
%% FIXME: [00008802] ***   UpdateHookStack : Code
\item[UValue] See \refObject{UValue}.

\item[UVar] See \refObject{UVar}.
\item[void] See \refObject{void}.
\item \lstinline|wall(\var{value}, \var{channel} = "")|\\
  Bounce to \lstinline|lobby.wall|, see \refObject{Lobby}.
\begin{urbiscript}
wall("111", "foo");
[00015895:foo] *** 111
wall(222, "");
[00051909] *** 222
wall(333);
[00055205] *** 333
\end{urbiscript}

\item \lstinline|warn(\var{message})|\\
  Issue \var{message} on \lstinline|Channel.warning|.
\begin{urbiscript}
warn("cave canem");
[00015895:warning] *** cave canem
\end{urbiscript}

%% FIXME: [00008806] ***   WeakDictionary : Finalizable
%% FIXME: [00008806] ***   WeakPointer : Finalizable
\end{urbiscriptapi}

%%% Local Variables:
%%% mode: latex
%%% TeX-master: "../urbi-sdk"
%%% ispell-dictionary: "american"
%%% ispell-personal-dictionary: "../urbi.dict"
%%% fill-column: 76
%%% End:
