\section{Binary}

A Binary object, sometimes called a \dfn{blob}, is raw memory,
decorated with a user defined header.

\subsection{Prototypes}
\begin{itemize}
\item \refObject{Object}
\end{itemize}

\subsection{Construction}

Binaries are usually not made by users, but they are heavily used by
the internal machinery when exchanging Binary UValues.  A binary
features some \lstinline|content| and some \lstinline|keywords|, both
simple \refObject[s]{String}.

\begin{urbiscript}[firstnumber=1]
Binary.new("my header", "my content");
[00000001] BIN 10 my header
[:]my content
\end{urbiscript}

Beware that the third line above (\samp{my content}), was output by
the system, although not preceded by a timestamp.

\subsection{Slots}

\begin{itemize}
\item \lstinline|'+'(\var{that})|\\
  Return a new Binary whose keywords are those of \lstinline|this| if
  not empty, otherwise those of \var{that}, and whose data is the
  concatenation of both.
\begin{urbiassert}
Binary.new("0", "0") + Binary.new("1", "1")
       == Binary.new("0", "01");
Binary.new("", "0") + Binary.new("1", "1")
       == Binary.new("1", "01");
\end{urbiassert}

\item \lstinline|'=='(\var{other})|\\
  Whether \lstinline|keywords| and \lstinline|data| are equal.
\begin{urbiassert}
Binary.new("0", "0") == Binary.new("0", "0");
Binary.new("0", "0") != Binary.new("0", "1");
Binary.new("0", "0") != Binary.new("1", "0");
\end{urbiassert}

\item \lstinline|asString|\\
  Display using the syntactic rules of the UObject/UValue protocol.
  Incoming binaries must use a semicolon to separate the header part
  from the content, while outgoing binaries use a carriage-return.
\begin{urbiscript}
assert(Binary.new("head", "content").asString
       == "BIN 7 head\ncontent");
var b = BIN 7 header;content;
[00000002] BIN 7 header
[:]content
assert(b == Binary.new("header", "content"));
\end{urbiscript}

This syntax (\lstinline|BIN \var{size} \var{header}; \var{content}|)
is \emph{partially} supported in \us, but it is strongly discouraged.
Rather, use the \lstinline|\B(\var{size})(\var{data})| special escape
(see \autoref{sec:us-syn-lit-string}):

\begin{urbiassert}
Binary.new("head", "\B(7)(content)").asString
       == "BIN 7 head\ncontent";
\end{urbiassert}


\item \lstinline|data|\\
  The data carried by the Binary.
\begin{urbiassert}
Binary.new("head", "content").data == "content";
\end{urbiassert}

\item \lstinline|empty|\\
  Whether the data is empty.
\begin{urbiassert}
Binary.new("head", "").empty;
!Binary.new("head", "content").empty;
\end{urbiassert}

\item \lstinline|keywords|\\
  The headers carried by the Binary.
\begin{urbiassert}
Binary.new("head", "content").keywords == "head";
\end{urbiassert}
\end{itemize}


%%% Local Variables:
%%% mode: latex
%%% TeX-master: "../urbi-sdk"
%%% ispell-personal-dictionary: "../urbi.dict"
%%% End:
