%% Copyright (C) 2009-2010, Gostai S.A.S.
%%
%% This software is provided "as is" without warranty of any kind,
%% either expressed or implied, including but not limited to the
%% implied warranties of fitness for a particular purpose.
%%
%% See the LICENSE file for more information.

\section{String}

A \dfn{string} is a sequence of characters.

\subsection{Prototypes}
\begin{refObjects}
\item[Comparable]
\item[Orderable]
\item[RangeIterable]
\end{refObjects}

\subsection{Construction}
Fresh Strings can easily be built using the literal syntax.  Several
escaping sequences (the traditional ones and \us specific ones) allow
to insert special characters.  Consecutive string literals are merged
together.  See \autoref{sec:us-syn-lit-string} for details and
examples.

A null String can also be obtained with \lstinline|String|'s
\lstinline|new| method.

\begin{urbiassert}[firstnumber=1]
String.new == "";
String == "";
"123".new == "123";
\end{urbiassert}

\subsection{Slots}
\begin{urbiscriptapi}
\item[asFloat]
  If the whole content of \this is an integer, return its
  value, otherwise return an error.
\begin{urbiscript}
assert("23.03".asFloat == 23.03);

"123abc".asFloat;
[00000001:error] !!! asFloat: unable to convert to float: "123abc"
\end{urbiscript}

\item[asList]
  Return a List of one-letter Strings that, concatenated, equal
  \this.  This allows to use \lstinline|for| to iterate
  over the string.
\begin{urbiscript}
assert("123".asList == ["1", "2", "3"]);
for (var v : "123")
  echo(v);
[00000001] *** 1
[00000001] *** 2
[00000001] *** 3
\end{urbiscript}

\item[asPrintable]
  Return \this as a literal (escaped) string.
\begin{urbiassert}
"foo".asPrintable == "\"foo\"";
"foo".asPrintable.asPrintable == "\"\\\"foo\\\"\"";
\end{urbiassert}

\item[asString]
  Return \this.
\begin{urbiassert}
"\"foo\"".asString == "\"foo\"";
\end{urbiassert}

\item[closest](<set>)%
  Return the string in \var{set} that is the closest (in the sense of
  \refSlot{distance}) to \this.  If there is no convincing match,
  return \lstinline|nil|.
\begin{urbiassert}
"foo".closest(["foo", "baz", "qux", "quux"]) == "foo";
"bar".closest(["foo", "baz", "qux", "quux"]) == "baz";
"FOO".closest(["foo", "bar", "baz"])         == "foo";
"qux".closest(["foo", "bar", "baz"])         == nil;
\end{urbiassert}

\item[distance](<other>)%
  Return the
  \href{http://en.wikipedia.org/wiki/Damerau-Levenshtein_distance}
  {Damerau-Levenshtein distance} between \this and
  \var{other}.  The more alike the strings are, the smaller the
  distance is.
\begin{urbiassert}
"foo".distance("foo") == 0;
"bar".distance("baz") == 1;
"foo".distance("bar") == 3;
\end{urbiassert}

\item[fresh] Return a String that has never been used as an identifier,
  prefixed by \this.  It can safely be used with
  \refSlot[Object]{setSlot} and so forth.
\begin{urbiassert}
String.fresh == "_5";
"foo".fresh == "foo_6";
\end{urbiassert}

\item[fromAscii(<v>)] The character corresponding to the integer \var{v}
  according to the \acro{ASCII} coding.  See also \refSlot{toAscii}.
\begin{urbiassert}
String.fromAscii(  97) == "a";
String.fromAscii(  98) == "b";
String.fromAscii(0xFF) == "\xff";
[0, 1, 2, 254, 255]
  .map(function (var v) { String.fromAscii(v) })
  .map(function (var v) { v.toAscii })
  == [0, 1, 2, 254, 255];
\end{urbiassert}


\item Character handling functions\\
  Here is a map of how the original 127-character ASCII set is
  considered by each function (a \textbullet{} indicates that the function
  returns true if all characters of \this are on the
  row).

\begin{tabular}{|l||l||c|c|c|c|c|c|c|c|c|c|c|}
  \hline
  &&&&&&&&&&&&\\
  ASCII values & Characters & \begin{sideways}iscntrl\end{sideways}
    & \begin{sideways}isspace\end{sideways}
    & \begin{sideways}isupper\end{sideways}
    & \begin{sideways}islower\end{sideways}
    & \begin{sideways}isalpha\end{sideways}
    & \begin{sideways}isdigit\end{sideways}
    & \begin{sideways}isxdigit\end{sideways}
    & \begin{sideways}isalnum\end{sideways}
    & \begin{sideways}ispunct\end{sideways}
    & \begin{sideways}isgraph\end{sideways}
    & \begin{sideways}print \end{sideways}\\ \hline \hline
  0x00 .. 0x08 & & \textbullet & & & & & & & & & &\\ \hline
  0x09 .. 0x0D & \textbackslash{}t, \textbackslash{}f,
  \textbackslash{}v, \textbackslash{}n, \textbackslash{}r &
  \textbullet & \textbullet & & & & & & & & &\\ \hline
  0x0E .. 0x1F & & \textbullet & & & & & & & & & &\\ \hline
  0x20 & space (' ') & & \textbullet & & & & & & & & & \textbullet\\ \hline
  0x21 .. 0x2F & \verb|!"#$%&'()*+,-./| & & & & & & & & & \textbullet & \textbullet & \textbullet\\ \hline
  0x30 .. 0x39 & \verb|0-9| & & & & & & \textbullet & \textbullet & \textbullet & & \textbullet & \textbullet\\ \hline
  0x3a .. 0x40 & \verb|:;<=>?@| & & & & & & & & & \textbullet & \textbullet & \textbullet\\ \hline
  0x41 .. 0x46 & \verb|A-F| & & & \textbullet & & \textbullet & & \textbullet & \textbullet & & \textbullet & \textbullet\\ \hline
  0x47 .. 0x5A & \verb|G-Z| & & & \textbullet & & \textbullet & & & \textbullet & & \textbullet & \textbullet\\ \hline
  0x5B .. 0x60 & \verb|[\]^{}_`| & & & & & & & & & \textbullet & \textbullet & \textbullet\\ \hline
  0x61 .. 0x66 & \verb|a-f| & & & & \textbullet & \textbullet & & \textbullet & \textbullet & & \textbullet & \textbullet\\ \hline
  0x67 .. 0x7A & \verb|g-z| & & & & \textbullet & \textbullet & & & \textbullet & & \textbullet & \textbullet\\ \hline
  0x7B .. 0x7E & \verb-{|}~- & & & & & & & & & \textbullet & \textbullet & \textbullet\\ \hline
  0x7F & (DEL) &  \textbullet & & & & & & & & & &\\
  \hline
\end{tabular}

\begin{urbiassert}
"".isDigit;
"0123456789".isDigit;
!"a".isDigit;

"".isLower;
"lower".isLower;
! "Not Lower".isLower;

"".isUpper;
"UPPER".isUpper;
! "Not Upper".isUpper;
\end{urbiassert}

%%% FIXME: Optional arguments.
\item[join](<list>, <prefix>, <suffix>)%
  Glue the result of \refSlot{asString} applied to the members of
  \var{list}, separated by \this, and embedded in a pair
  \var{prefix}/\var{suffix}.
\begin{urbiassert}
"|".join([1, 2, 3], "(", ")")      == "(1|2|3)";
", ".join([1, [2], "3"], "[", "]") == "[1, [2], 3]";
\end{urbiassert}

\item[replace](<from>, <to>)%
  Replace every occurrence of the string \var{from} in
  \this by \var{to}, and return the result.
  \this is not modified.
\begin{urbiassert}
"Hello, World!".replace("Hello", "Bonjour")
                      .replace("World!", "Monde !") ==
       "Bonjour, Monde !";
\end{urbiassert}

\item[size]
  Return the size of the string.
\begin{urbiassert}
"foo".size == 3;
"".size == 0;
\end{urbiassert}

%%% FIXME: incorrect rendering: we lose the backslashes from the
%%% separators.
%\item[split](<sep> = [" ", "\t", "\n", "\r"], <lim> = -1, <keepSep> = false, <keepEmpty> = true)%
\item \hypertarget{slot:String.split}{}%
  \lstinline|split(\var{sep} = [" ", "\t", "\n", "\r"], \var{lim} = -1, \var{keepSep} = false, \var{keepEmpty} = true)|\\
  Split \this on the separator \var{sep}, in at most \var{lim}
  components, which include the separator if \var{keepSep}, and the
  empty components of \var{keepEmpty}.  Return a list of strings.

  The separator, \var{sep}, can be a string.

\begin{urbiassert}
       "a,b;c".split(",") == ["a", "b;c"];
       "a,b;c".split(";") == ["a,b", "c"];
      "foobar".split("x") == ["foobar"];
     "foobar".split("ob") == ["fo", "ar"];
\end{urbiassert}

\noindent
It can also be a list of strings.

\begin{urbiassert}
"a,b;c".split([",", ";"]) == ["a", "b", "c"];
\end{urbiassert}

\noindent
By default splitting is performed on white-spaces:

\begin{urbiassert}
"  abc  def\tghi\n".split == ["abc", "def", "ghi"];
\end{urbiassert}

\noindent
Splitting on the empty string stands for splitting between each character:

\begin{urbiassert}
"foobar".split("") == ["f", "o", "o", "b", "a", "r"];
\end{urbiassert}

The limit \var{lim} indicates a maximum number of splits that can occur. A
negative number corresponds to no limit:

\begin{urbiassert}
"a:b:c".split(":",  1) == ["a", "b:c"];
"a:b:c".split(":", -1) == ["a", "b", "c"];
\end{urbiassert}

\var{keepSep} indicates whether to keep delimiters in the result:

\begin{urbiassert}
"aaa:bbb;ccc".split([":", ";"], -1, false) == ["aaa",      "bbb",      "ccc"];
"aaa:bbb;ccc".split([":", ";"], -1, true)  == ["aaa", ":", "bbb", ";", "ccc"];
\end{urbiassert}

\var{keepEmpty} indicates whether to keep empty elements:

\begin{urbiassert}
"foobar".split("o")                   == ["f", "", "bar"];
"foobar".split("o", -1, false, true)  == ["f", "", "bar"];
"foobar".split("o", -1, false, false) == ["f",     "bar"];
\end{urbiassert}


\item[toAscii] Convert the first character of \this to its integer value in
  the \acro{ASCII} coding.  See also \refSlot{fromAscii}.
\begin{urbiassert}
   "a".toAscii == 97;
   "b".toAscii == 98;
"\xff".toAscii == 0xff;
"Hello, World!\n"
  .asList
  .map(function (var v) { v.toAscii })
  .map(function (var v) { String.fromAscii(v) })
  .join
  == "Hello, World!\n";
\end{urbiassert}

\item[toLower]
  Make lower case every upper case character in \this and
  return the result.  \this is not modified.
\begin{urbiassert}
"Hello, World!".toLower == "hello, world!";
\end{urbiassert}

\item[toUpper]
  Make upper case every lower case character in \this and
  return the result.  \this is not modified.
\begin{urbiassert}
"Hello, World!".toUpper == "HELLO, WORLD!";
\end{urbiassert}

\item['=='](<that>)%
  Whether \this and \var{that} are the same string.
\begin{urbiassert}
  "" == "";        !("" != "");
!("" == "\0");       "" != "\0";

  "0" == "0";      !("0" != "0");
!("0" == "1");       "0" != "1";
!("1" == "0");       "1" != "0";
\end{urbiassert}

\item \lstinline|'%'(\var{args})|\\
  It is an equivalent of \lstinline|Formatter.new(this) % \var{args}|.
  See \refObject{Formatter}.
%  This construct is actually more
%  powerful than this, since it relies on
%  \href{http://www.boost.org/doc/libs/1_39_0/libs/format/doc/format.html,
%    Boost.Format}.  For instance:
\begin{urbiassert}
"%s + %s = %s" % [1, 2, 3] == "1 + 2 = 3";
\end{urbiassert}

\item['*'](<n>)%
  Concatenate \this \var{n} times.
\begin{urbiassert}
"foo" * 0 == "";
"foo" * 1 == "foo";
"foo" * 3 == "foofoofoo";
\end{urbiassert}

\item['+'](<other>)%
  Concatenate \this and \lstinline|\var{other}.asString|.
\begin{urbiassert}
"foo" + "bar" == "foobar";
"foo" + "" == "foo";
"foo" + 3 == "foo3";
"foo" + [1, 2, 3] == "foo[1, 2, 3]";
\end{urbiassert}

\item['<'](<other>)%
  Whether \this is lexicographically before \var{other},
  which must be a String.
\begin{urbiassert}
"" < "a";
!("a" < "");
"a" < "b";
!("a" < "a");
\end{urbiassert}

\item \lstinline|'[]'(\var{from})|\\
  \lstinline|'[]'(\var{from}, \var{to})|\\
  Return the sub-string starting at \var{from}, up to and not including
  \var{to} (which defaults to \var{to} + 1).
\begin{urbiassert}
"foobar"[0, 3] == "foo";
"foobar"[0] == "f";
\end{urbiassert}

\item \lstinline|'[]='(\var{from}, \var{other})|\\
  \lstinline|'[]='(\var{from}, \var{to}, \var{other})|\\
  Replace the sub-string starting at \var{from}, up to and not including
  \var{to} (which defaults to \var{to} + 1), by \var{other}.  Return
  \var{other}.

  Beware that this routine is imperative: it changes the value of
  \this.
\begin{urbiscript}
var s1 = "foobar" | var s2 = s1 |
assert((s1[0, 3] = "quux") == "quux");
assert(s1 == "quuxbar");
assert(s2 == "quuxbar");
assert((s1[4, 7] = "") == "");
assert(s2 == "quux");
\end{urbiscript}
\end{urbiscriptapi}

%%% Local Variables:
%%% mode: latex
%%% TeX-master: "../urbi-sdk"
%%% ispell-dictionary: "american"
%%% ispell-personal-dictionary: "../urbi.dict"
%%% fill-column: 76
%%% End:
