\section{String}

A \dfn{string} is a sequence of characters.

\subsection{Prototypes}
\begin{itemize}
\item \refObject{Comparable}
\item \refObject{Orderable}
\end{itemize}

\subsection{Construction}
The simplest way to create fresh String is by using the literal
syntax. A null String can also be obtained with \lstinline|String|'s
\lstinline|new| method.

\begin{urbiscript}
String.new;
[00000000] ""
String;
[00000000] ""
"123";
[00000000] "123"
\end{urbiscript}

\subsection{Methods}
\begin{itemize}
\item \lstinline|asFloat|\\
  If the whole content of \lstinline|this| is an integer, return its
  value, otherwise return an error.
\begin{urbiscript}[firstnumber=last]
"23.03".asFloat;
[00000000] 23.03

"123abc".asFloat;
[00000001:error] !!! asFloat: unable to convert to float: "123abc"
\end{urbiscript}

\item \lstinline|asList|\\
  Return a List of one-letter Strings that, concataneted, equal
  \lstinline|this|.  This allows to use \lstinline|for| to iterate
  over the string.
\begin{urbiscript}[firstnumber=last]
assert("123".asList == ["1", "2", "3"]);
for (var v : "123")
  echo(v);
[00000001] *** 1
[00000001] *** 2
[00000001] *** 3
\end{urbiscript}

\item \lstinline|asPrintable|\\
  Return \lstinline|this| as a literal (escaped) string.
\begin{urbiscript}[firstnumber=last]
assert("foo".asPrintable == "\"foo\"");
assert("foo".asPrintable.asPrintable == "\"\\\"foo\\\"\"");
\end{urbiscript}

\item \lstinline|asString|\\
  Return \lstinline|this|.
\begin{urbiscript}[firstnumber=last]
assert("\"foo\"".asString == "\"foo\"");
\end{urbiscript}

\item \lstinline|distance(\var{other})|\\
  Return the
  \href{http://en.wikipedia.org/wiki/Damerau-Levenshtein_distance}
  {Damerau-Levenshtein distance} between \lstinline|this| and
  \var{other}.  The more alike the strings are, the smaller the
  distance is.
\begin{urbiscript}[firstnumber=last]
assert("foo".distance("foo") == 0);
assert("bar".distance("baz") == 1);
assert("foo".distance("bar") == 3);
\end{urbiscript}

\item \lstinline|fresh|\\
  Return a String that has never been used as an identifier, prefixed
  by \lstinline|this|.  It can safely be used with
  \lstinline|Object.setSlot| and so forth.
\begin{urbiscript}[firstnumber=last]
assert(String.fresh == "_5");
assert("foo".fresh == "foo_6");
\end{urbiscript}

\item Character handling functions\\
  Here is a map of how the original 127-character ASCII set is
  considered by each function (a \textbullet{} indicates that the function
  returns true if all characters of \lstinline|this| are on the
  row).

\begin{tabular}{|l||l||c|c|c|c|c|c|c|c|c|c|c|}
  \hline
  &&&&&&&&&&&&\\
  ASCII values & Characters & \begin{sideways}iscntrl\end{sideways}
    & \begin{sideways}isspace\end{sideways}
    & \begin{sideways}isupper\end{sideways}
    & \begin{sideways}islower\end{sideways}
    & \begin{sideways}isalpha\end{sideways}
    & \begin{sideways}isdigit\end{sideways}
    & \begin{sideways}isxdigit\end{sideways}
    & \begin{sideways}isalnum\end{sideways}
    & \begin{sideways}ispunct\end{sideways}
    & \begin{sideways}isgraph\end{sideways}
    & \begin{sideways}print \end{sideways}\\ \hline \hline
  0x00 .. 0x08 & & \textbullet & & & & & & & & & &\\ \hline
  0x09 .. 0x0D & \textbackslash{}t, \textbackslash{}f,
  \textbackslash{}v, \textbackslash{}n, \textbackslash{}r &
  \textbullet & \textbullet & & & & & & & & &\\ \hline
  0x0E .. 0x1F & & \textbullet & & & & & & & & & &\\ \hline
  0x20 & space (' ') & & \textbullet & & & & & & & & & \textbullet\\ \hline
  0x21 .. 0x2F & \verb|!"#$%&'()*+,-./| & & & & & & & & & \textbullet & \textbullet & \textbullet\\ \hline
  0x30 .. 0x39 & \verb|0-9| & & & & & & \textbullet & \textbullet & \textbullet & & \textbullet & \textbullet\\ \hline
  0x3a .. 0x40 & \verb|:;<=>?@| & & & & & & & & & \textbullet & \textbullet & \textbullet\\ \hline
  0x41 .. 0x46 & \verb|A-F| & & & \textbullet & & \textbullet & & \textbullet & \textbullet & & \textbullet & \textbullet\\ \hline
  0x47 .. 0x5A & \verb|G-Z| & & & \textbullet & & \textbullet & & & \textbullet & & \textbullet & \textbullet\\ \hline
  0x5B .. 0x60 & \verb|[\]^{}_`| & & & & & & & & & \textbullet & \textbullet & \textbullet\\ \hline
  0x61 .. 0x66 & \verb|a-f| & & & & \textbullet & \textbullet & & \textbullet & \textbullet & & \textbullet & \textbullet\\ \hline
  0x67 .. 0x7A & \verb|g-z| & & & & \textbullet & \textbullet & & & \textbullet & & \textbullet & \textbullet\\ \hline
  0x7B .. 0x7E & \verb-{|}~- & & & & & & & & & \textbullet & \textbullet & \textbullet\\ \hline
  0x7F & (DEL) &  \textbullet & & & & & & & & & &\\
  \hline
\end{tabular}

\begin{urbiscript}[firstnumber=last]
assert("".isLower);
assert("lower".isLower);
assert(! "Not Lower".isLower);
assert("".isUpper);
assert("UPPER".isUpper);
assert(! "Not Upper".isUpper);
\end{urbiscript}

\item \lstinline|replace(\var{from}, \var{to})|\\
  Replace every occurrence of the string \var{from} in
  \lstinline|this| by \var{to}, and return the result.
  \lstinline|this| is not modified.
\begin{urbiscript}[firstnumber=last]
assert("Hello == World!".replace("Hello", "Bonjour")
                      .replace("World!", "Monde !"),
       "Bonjour, Monde !");
\end{urbiscript}

\item \lstinline|size|\\
  Return the size of the string.
\begin{urbiscript}[firstnumber=last]
assert("foo".size == 3);
assert("".size == 0);
\end{urbiscript}

\item \lstinline|toLower|\\
  Make lower case every upper case character in \lstinline|this| and
  return the result.  \lstinline|this| is not modified.
\begin{urbiscript}[firstnumber=last]
assert("Hello == World!".toLower, "hello, world!");
\end{urbiscript}

\item \lstinline|toUpper|\\
  Make upper case every lower case character in \lstinline|this| and
  return the result.  \lstinline|this| is not modified.
\begin{urbiscript}[firstnumber=last]
assert("Hello == World!".toUpper, "HELLO, WORLD!");
\end{urbiscript}

\item \lstinline|'%'(\var{args})|\\
  Use \lstinline|this| as format string, and convert occurrences of
  \lstinline|%s| in \lstinline|this| by the \var{args} converted as
  strings via \lstinline|asString|.
%  This construct is actually more
%  powerful than this, since it relies on
%  \href{http://www.boost.org/doc/libs/1_39_0/libs/format/doc/format.html,
%    Boost.Format}.  For instance:
\begin{urbiscript}[firstnumber=last]
assert("%s + %s = %s" % [1, 2, 3] == "1 + 2 = 3");
\end{urbiscript}

\item \lstinline|'*'(\var{n})|\\
  Return the concatentation of \lstinline|this| \var{n} times.
\begin{urbiscript}[firstnumber=last]
assert("foo" * 0 == "");
assert("foo" * 1 == "foo");
assert("foo" * 3 == "foofoofoo");
\end{urbiscript}

\item \lstinline|'+'(\var{other})|\\
  Return the concatenation of \lstinline|this| and
  \lstinline|\var{other}.asString|.
\begin{urbiscript}[firstnumber=last]
assert("foo" + "bar" == "foobar");
assert("foo" + "" == "foo");
assert("foo" + 3 == "foo3");
assert("foo" + [1 == 2, 3], "foo[1, 2, 3]");
\end{urbiscript}

\item \lstinline|'<'(\var{other})|\\
  Whether \lstinline|this| is lexicographically before \var{other},
  which must be a String.
\begin{urbiscript}[firstnumber=last]
assert("" < "a");
assert(!("a" < ""));
assert("a" < "b");
assert(!("a" < "a"));
\end{urbiscript}

\item \lstinline|'[]'(\var{from})|\\
  \lstinline|'[]'(\var{from}, \var{to})|\\
  Return the substring starting at \var{from}, up to and not including
  \var{to} (which defaults to \var{to} + 1).
\begin{urbiscript}[firstnumber=last]
assert("foobar"[0 == 3], "foo");
assert("foobar"[0] == "f");
\end{urbiscript}

\item \lstinline|'[]='(\var{from}, \var{other})|\\
  \lstinline|'[]='(\var{from}, \var{to}, \var{other})|\\
  Replace the substring starting at \var{from}, up to and not including
  \var{to} (which defaults to \var{to} + 1), by \var{other}.  Return
  \var{other}.

  Beware that this routine is imperative: it changes the value of
  \lstinline|this|.
\begin{urbiscript}[firstnumber=last]
var s1 = "foobar" | var s2 = s1 |
assert(s1[0 == 3] = "quux", "quux");
assert(s1 == "quuxbar");
assert(s2 == "quuxbar");
assert(s1[4 == 7] = "", "");
assert(s2 == "quux");
\end{urbiscript}
\end{itemize}

%%% Local Variables:
%%% mode: latex
%%% TeX-master: "../urbi-sdk"
%%% End:
