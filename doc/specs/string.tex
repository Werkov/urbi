\section{String}

A \dfn{string} is a sequence of characters.

\begin{itemize}
\item \lstinline|asFloat|\\
  If the whole content of \lstinline|this| is an integer, return its
  value, otherwise return an error.
\begin{urbiscript}
"23.03".asFloat;
[00000000] 23.03

"123abc".asFloat;
[00000001:error] !!! asFloat: unable to convert to float: "123abc"
\end{urbiscript}

\item \lstinline|asString|\\
  Return \lstinline|this|.

\item \lstinline|distance(\var{other})|\\
  Return the
  \href{http://en.wikipedia.org/wiki/Damerau-Levenshtein_distance,
    Damerau-Levenshtein} distance between \lstinline|this| and
  \var{other}.  The more alike the strings are, the smaller the
  distance is.
\begin{urbiscript}
assert("foo".distance("foo") == 0);
assert("bar".distance("baz") == 1);
assert("foo".distance("bar") == 3);
\end{urbiscript}

\item \lstinline|size|\\
  Return the size of the string.
\begin{urbiscript}
assert("foo".size == 3);
assert("".size == 0);
\end{urbiscript}

\item \lstinline|toLower|\\
  Make lower case every upper case character in \lstinline|this| and
  return the result.  \lstinline|this| is not modified.

\item \lstinline|toUpper|\\
  Make upper case every lower case character in \lstinline|this| and
  return the result.  \lstinline|this| is not modified.

\item \lstinline|% \var{args}|\\
  Use \lstinline|this| as format string, and convert occurrences of
  \lstinline|%s| in \lstinline|this| by the \var{args} converted as
  strings via \lstinline|asString|.
%  This construct is actually more
%  powerful than this, since it relies on
%  \href{http://www.boost.org/doc/libs/1_39_0/libs/format/doc/format.html,
%    Boost.Format}.  For instance:
\begin{urbiscript}
"%s + %s = %s" % [1, 2, 3];
[00000000] "1 + 2 = 3"
\end{urbiscript}

\item \lstinline|* \var{n}|\\
  Return the concatentation of \lstinline|this| \var{n} times.
\begin{urbiscript}
assert("foo" * 0 == "");
assert("foo" * 1 == "foo");
assert("foo" * 3 == "foofoofoo");
\end{urbiscript}
\end{itemize}

\begin{urbiscript}
var s = "Hello, World!\n"|;
assert(s.replace("Hello", "Bonjour")
        .replace("World!", "Monde !")
       == "Bonjour, Monde !\n");
assert(s.toLower == "hello, world!\n");
assert(s.toUpper == "HELLO, WORLD!\n");
\end{urbiscript}

%%% Local Variables:
%%% mode: latex
%%% TeX-master: "../urbi-sdk"
%%% End:
