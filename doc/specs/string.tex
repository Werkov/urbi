%% Copyright (C) 2009-2011, Gostai S.A.S.
%%
%% This software is provided "as is" without warranty of any kind,
%% either expressed or implied, including but not limited to the
%% implied warranties of fitness for a particular purpose.
%%
%% See the LICENSE file for more information.

\section{String}

A \dfn{string} is a sequence of characters.

\subsection{Prototypes}
\begin{refObjects}
\item[Comparable]
\item[Orderable]
\item[RangeIterable]
\end{refObjects}

\subsection{Construction}
Fresh Strings can easily be built using the literal syntax.  Several
escaping sequences (the traditional ones and \us specific ones) allow to
insert special characters.  Consecutive string literals are merged together.
See \autoref{sec:lang:string} for details and examples.

A null String can also be obtained with \slot[String]{new}.

\begin{urbiassert}
String.new == "";
String == "";
"123".new == "123";
\end{urbiassert}

\subsection{Slots}
\begin{urbiscriptapi}
\item \lstinline|'%'(\var{args})|\\
  It is an equivalent of \lstinline|Formatter.new(this) % \var{args}|.
  See \refObject{Formatter}.
%  This construct is actually more
%  powerful than this, since it relies on
%  \href{http://www.boost.org/doc/libs/1_39_0/libs/format/doc/format.html,
%    Boost.Format}.  For instance:
\begin{urbiassert}
"%s + %s = %s" % [1, 2, 3] == "1 + 2 = 3";
\end{urbiassert}


\item['*'](<n>)%
  Concatenate \this \var{n} times.
\begin{urbiassert}
"foo" * 0 == "";
"foo" * 1 == "foo";
"foo" * 3 == "foofoofoo";
\end{urbiassert}


\item['+'](<other>)%
  Concatenate \this and \lstinline|\var{other}.asString|.
\begin{urbiassert}
"foo" + "bar" == "foobar";
"foo" + "" == "foo";
"foo" + 3 == "foo3";
"foo" + [1, 2, 3] == "foo[1, 2, 3]";
\end{urbiassert}


\item['<'](<other>)%
  Whether \this is lexicographically before \var{other},
  which must be a String.
\begin{urbiassert}
"" < "a";
!("a" < "");
"a" < "b";
!("a" < "a");
\end{urbiassert}


\item['=='](<that>)%
  Whether \this and \var{that} are the same string.
\begin{urbiassert}
  "" == "";        !("" != "");
!("" == "\0");       "" != "\0";

  "0" == "0";      !("0" != "0");
!("0" == "1");       "0" != "1";
!("1" == "0");       "1" != "0";
\end{urbiassert}


\item|'[]'|(<from>, <to> = <from> + 1)%
  The sub-string starting at \var{from}, up to and not including \var{to}.
\begin{urbiassert}
"foobar"[0, 3] == "foo";
"foobar"[0] == "f";
\end{urbiassert}

The indexes must be integers from -\refSlot{size} up to \refSlot{size} + 1
(inclusive).  Independently of their order as integers, \var{from} must be
\emph{before} or equal to \var{to}.

\begin{urbiscript}
"foobar"[1.1];
[00051825:error] !!! []: invalid index: 1.1
"foobar"[-7];
[00051841:error] !!! []: invalid index: -7
"foobar"[6];
[00051853:error] !!! []: invalid index: 6

"foobar"[3, 1];
[00051953:error] !!! []: range starting after its end: 3, 1
\end{urbiscript}

Negative indexes means ``from the end''.

\begin{urbiassert}
"foobar"[-1] == "r";
"foobar"[-6] == "f";
"foobar"[-1, 0] == "r";
"foobar"[-6, -3] == "foo";
// Valid since position 2 is before position -1.
"foobar"[2, -1] == "oba"
\end{urbiassert}


\item|'[]='|(<from>, <other>)%
  \lstinline|'[]='(\var{from}, \var{to}, \var{other})|\\
  Replace the sub-string starting at \var{from}, up to and not including
  \var{to} (which defaults to \var{to} + 1), by \var{other}.  The condition
  on \var{from} and \var{to} are the same of for \refSlot{'[]'}.  Return
  \var{other}.

  Beware that this routine is imperative: it changes the value of
  \this.
\begin{urbiscript}
var s1 = "foobar" | var s2 = s1 |
assert
{
  (s1[0, 3] = "quux") == "quux";
  s1 == "quuxbar";
  s2 == "quuxbar";
  (s1[4, 7] = "") == "";
  s2 == "quux";

  (s1[-3, -1] = "UU") == "UU";
  s1 == "qUUx";
  (s1[-1] = "X") == "X";
  s1 == "qUUX";
};
\end{urbiscript}


\item[asFloat]%
  The value of \this as a \refObject{Float}.  See \autoref{sec:lang:float}
  for their syntax.
\begin{urbiassert}
            "23".asFloat == 23;
         "23.03".asFloat == 23.03;
   "123_456_789".asFloat == 123_456_789;
"12_34_56_78_90".asFloat == 12_34_56_78_90;
   "1_2__3___45".asFloat == 1_2__3___45;
      "1_2.3__4".asFloat == 1_2.3__4;
   "0xFFFF_FFFF".asFloat == 0xFFFF_FFFF;
         "1e1_0".asFloat == 1e1_0;
\end{urbiassert}

  Raise an error on invalid numbers.

\begin{urbiscript}
"123abc".asFloat;
[00000001:error] !!! asFloat: invalid number: "123abc"
"0xabcdefg".asFloat;
[00061848:error] !!! asFloat: invalid number: "0xabcdefg"
"1.2_".asFloat;
[00048342:error] !!! asFloat: invalid number: "1.2_"
\end{urbiscript}


\item[asList]%
  A List of one-letter Strings that, concatenated, equal \this.  This allows
  to use \lstinline|for| to iterate over the string.
\begin{urbiscript}
assert("123".asList == ["1", "2", "3"]);
for (var v : "123")
  echo(v);
[00000001] *** 1
[00000001] *** 2
[00000001] *** 3
\end{urbiscript}


\item[asPrintable]
  \this as a literal (escaped) string.
\begin{urbiassert}
"foo".asPrintable == "\"foo\"";
"foo".asPrintable.asPrintable == "\"\\\"foo\\\"\"";
\end{urbiassert}


\item[asString]
  \this.
\begin{urbiassert}
var s = "\"foo\"";
s.asString === s;
\end{urbiassert}


\item[closest](<set>)%
  The closest (in the sense of \refSlot{distance}) string in \var{set} to
  \this.  If there is no convincing match, return \lstinline|nil|.
\begin{urbiassert}
"foo".closest(["foo", "baz", "qux", "quux"]) == "foo";
"bar".closest(["foo", "baz", "qux", "quux"]) == "baz";
"FOO".closest(["foo", "bar", "baz"])         == "foo";
"qux".closest(["foo", "bar", "baz"])         == nil;
\end{urbiassert}


\item[distance](<other>)%
  The \href{http://en.wikipedia.org/wiki/Damerau-Levenshtein_distance}
  {Damerau-Levenshtein distance} between \this and \var{other}.  The more
  alike the strings are, the smaller the distance is.
\begin{urbiassert}
"foo".distance("foo") == 0;
"bar".distance("baz") == 1;
"foo".distance("bar") == 3;

  "foo".distance("fozo") == 1; // Deletion.
 "fozo".distance("foo")  == 1; // Insertion.
 "ofzo".distance("fozo") == 1; // Transposition.
 "fpzo".distance("fozo") == 1; // Substitution.

"fpzzo".distance("fozo") == 2; // Substitution and insertion.
\end{urbiassert}


\item[empty] Whether this is the empty string.
\begin{urbiassert}
  "".empty;
!"x".empty;
\end{urbiassert}


\item[find](<pattern>, <position> = 0)%
  Search for the \var{pattern} string in \this starting from the
  \var{position}.  When the \var{pattern} is not found \lstinline|-1| is
  returned.
\begin{urbiassert}
"Hello, World!".find("o")    == 4;
"Hello, World!".find("o", 4) == 4;
"Hello, World!".find("o", 5) == 8;
"Hello, World!".find("o", 9) == -1;
\end{urbiassert}


\item[fresh] A String that has never been used as an identifier, prefixed by
  \this.  It can safely be used with \refSlot[Object]{setSlot} and so forth.
\begin{urbiassert}
String.fresh == "_5";
"foo".fresh == "foo_6";
\end{urbiassert}


\item[fromAscii](<v>) The character corresponding to the integer \var{v}
  according to the \acro{ASCII} coding.  See also \refSlot{toAscii}.
\begin{urbiassert}
String.fromAscii(  97) == "a";
String.fromAscii(  98) == "b";
String.fromAscii(0xFF) == "\xff";
[0, 1, 2, 254, 255]
  .map(function (var v) { String.fromAscii(v) })
  .map(function (var v) { v.toAscii })
  == [0, 1, 2, 254, 255];
\end{urbiassert}


\item[hash]%
  A \refObject{Hash} object corresponding to this string value.  Equal
  strings (in the sense of \refSlot{'=='}) have equal hashes.  See
  \refSlot[Object]{hash}.

\begin{urbiassert}
"".hash.isA(Hash);
"foo".hash == "foo".hash;
"foo".hash != "bar".hash;
\end{urbiassert}


\item[isAlnum] Whether all the characters of \this are digits or/and
  characters (see \autoref{tab:string:ischar}).
  \begin{table}[tp]
    %% Not factoring the common "is" to avoid adding too many words to our
    %% dictionary (e.g., adding Cntrl in addition to isCntrl).  Add a small
    %% space (~) afterwards, otherwise the hline touches it.
    \newcommand{\is}[1]{\begin{sideways}\refSlot{#1}~~\end{sideways}}
    \newcommand{\X}{\textbullet}
    \newcommand{\B}[1]{\textbackslash{}#1}
    \centering
    \begin{tabular}{|l||l||*{12}{c|}}
      \hline
      ASCII values & Characters
      & \is{isCntrl} & \is{isSpace} & \is{isBlank} & \is{isUpper} & \is{isLower}
      & \is{isAlpha} & \is{isDigit} & \is{isXdigit} & \is{isAlnum}
      & \is{isPunct} & \is{isGraph} & \is{isPrint}
      \\ \hline \hline
      0x00 .. 0x08 &                            & \X &    &    &    &    &    &    &    &    &    &    &    \\\hline
      0x09         & \B{t}                      & \X & \X & \X &    &    &    &    &    &    &    &    &    \\\hline
      0x0A .. 0x0D & \B{f}, \B{v}, \B{n}, \B{r} & \X &    & \X &    &    &    &    &    &    &    &    &    \\\hline
      0x0E .. 0x1F &                            & \X &    &    &    &    &    &    &    &    &    &    &    \\\hline
      0x20         & space (' ')                &    & \X & \X &    &    &    &    &    &    &    &    & \X \\\hline
      0x21 .. 0x2F & \verb|!"#$%&'()*+,-./|     &    &    &    &    &    &    &    &    &    & \X & \X & \X \\\hline
      0x30 .. 0x39 & \verb|0-9|                 &    &    &    &    &    &    & \X & \X & \X &    & \X & \X \\\hline
      0x3a .. 0x40 & \verb|:;<=>?@|             &    &    &    &    &    &    &    &    &    & \X & \X & \X \\\hline
      0x41 .. 0x46 & \verb|A-F|                 &    &    &    & \X &    & \X &    & \X & \X &    & \X & \X \\\hline
      0x47 .. 0x5A & \verb|G-Z|                 &    &    &    & \X &    & \X &    &    & \X &    & \X & \X \\\hline
      0x5B .. 0x60 & \verb|[\]^{}_`|            &    &    &    &    &    &    &    &    &    & \X & \X & \X \\\hline
      0x61 .. 0x66 & \verb|a-f|                 &    &    &    &    & \X & \X &    & \X & \X &    & \X & \X \\\hline
      0x67 .. 0x7A & \verb|g-z|                 &    &    &    &    & \X & \X &    &    & \X &    & \X & \X \\\hline
      0x7B .. 0x7E & \verb-{|}~-                &    &    &    &    &    &    &    &    &    & \X & \X & \X \\\hline
      0x7F         & (DEL)                      & \X &    &    &    &    &    &    &    &    &    &    &    \\\hline
    \end{tabular}
    \begin{legend}
      Here is a map of how the original 127-character ASCII set is
      considered by each function (a \X{} indicates that the function
      returns true if all characters of \this are on the row).

      Note the following equivalences:

      \begin{tabular}{r@{~$\equiv$~}l}
        \lstinline'isAlnum'
        & \lstinline'isAlpha || isDigit'\\
%%        & \lstinline'isDigit || isLower || isUpper'\\
        \lstinline'isAlpha'
        & \lstinline'isLower || isUpper'\\
        \lstinline'isGraph'
        & \lstinline'isAlpha || isDigit || isPunct'\\
%%        & \lstinline'isDigit || isLower || isPunct || isUpper'\\
% Not true, there is not \t in isPrint.
%        \lstinline'isPrint'
%        & \lstinline'isBlank || isGraph'\\
%        & \lstinline'isBlank || isDigit || isLower || isPunct || isUpper'\\
      \end{tabular}
    \end{legend}
    \caption{Character handling functions}
    \label{tab:string:ischar}
  \end{table}
\begin{urbiassert}
           "".isAlnum;
        "123".isAlnum;
        "abc".isAlnum; "1b".isAlnum;
\end{urbiassert}


\item[isAlpha] Whether all the characters of \this are letters, upper or
  lower case (see \autoref{tab:string:ischar}).
\begin{urbiassert}
           "".isAlpha;
     "abcABC".isAlpha;
       !"123".isAlpha;  "b".isAlpha;
\end{urbiassert}

\item[isAscii] Whether all the characters of \this are ascii characters
  (see \autoref{tab:string:ischar}).
\begin{urbiassert}[escapeinside=<>]
           "".isAscii;
     "abc123".isAscii;
         !"<é>".isAscii; !"<è>".isAscii;
\end{urbiassert}

\item[isBlank] Whether all the characters of \this are spaces or tabulations
  (see \autoref{tab:string:ischar}).
\begin{urbiassert}
           "".isBlank;
     "\t \t ".isBlank;
       !"123".isBlank;  !"\v".isBlank;
\end{urbiassert}


\item[isCntrl] Whether all the characters of \this are control characters
  (non-printable) (see \autoref{tab:string:ischar}).
\begin{urbiassert}
           "".isCntrl;
     "\t\n\f".isCntrl;
        "\10".isCntrl; !"abc".isCntrl;
\end{urbiassert}


\item[isDigit] Whether all the characters of \this are decimal digits (see
  \autoref{tab:string:ischar}).
\begin{urbiassert}
           "".isDigit;
 "0123456789".isDigit;
         !"a".isDigit;  !"0x10".isDigit;
\end{urbiassert}


\item[isGraph] Whether all the characters of \this are printable characters
  (not a space) (see \autoref{tab:string:ischar}).
\begin{urbiassert}
          "".isGraph;
    "abc123".isGraph;
       "{|}".isGraph;
   !"\t\n\r".isGraph; !" ".isGraph;
\end{urbiassert}


\item[isLower] Whether all the characters of \this are lower characters (see
  \autoref{tab:string:ischar}).
\begin{urbiassert}
          "".isLower;
       "abc".isLower;
      !"123".isLower;  !"A".isLower;
\end{urbiassert}


\item[isPrint] Whether all the characters of \this are printable characters
  (see \autoref{tab:string:ischar}).
\begin{urbiassert}
          "".isPrint;
  "abcd1234".isPrint;
       "{|}".isPrint; !"\r".isPrint;
\end{urbiassert}


\item[isPunct] Whether all the characters of \this are punctuation marks
  (see \autoref{tab:string:ischar}).
\begin{urbiassert}
         "".isPunct;
     "!_[]".isPunct;
     !"abc".isPunct;  !"a".isPunct;
\end{urbiassert}


\item[isSpace] Whether all the characters of \this are spaces (see
  \autoref{tab:string:ischar}).
\begin{urbiassert}
          "".isSpace;
       "   ".isSpace;
      !" a ".isSpace;
\end{urbiassert}


\item[isUpper] Whether all the characters of \this are upper characters (see
  \autoref{tab:string:ischar}).
\begin{urbiassert}
          "".isUpper;
       "ABC".isUpper;
      !"123".isUpper; !"a".isUpper;
\end{urbiassert}


\item[isXdigit] Whether all the characters of \this are hexadecimal digits
  (see \autoref{tab:string:ischar}).
\begin{urbiassert}
          "".isXdigit;
    "abcdef".isXdigit;
"0123456789".isXdigit;
        !"g".isXdigit; "123abc".isXdigit;
\end{urbiassert}


\item[join](<list>, <prefix>, <suffix>)%
  %%% FIXME: Optional arguments.
  Glue the result of \refSlot{asString} applied to the members of
  \var{list}, separated by \this, and embedded in a pair
  \var{prefix}/\var{suffix}.
\begin{urbiassert}
"|".join([1, 2, 3], "(", ")")      == "(1|2|3)";
", ".join([1, [2], "3"], "[", "]") == "[1, [2], 3]";
\end{urbiassert}


\item[length] The number of characters in the string.  Currently, this is a
  synonym of \refSlot{size}.
\begin{urbiassert}
"foo".length == 3;
   "".length == 0;
\end{urbiassert}


\item[replace](<from>, <to>)%
  Replace every occurrence of the string \var{from} in \this by \var{to},
  and return the result.  \this is not modified.
\begin{urbiassert}
"Hello, World!".replace("Hello", "Bonjour")
                      .replace("World!", "Monde !") ==
       "Bonjour, Monde !";
\end{urbiassert}


\item[rfind](<pattern>, <position> = -1)%
  Search backward for the \var{pattern} string in \this starting from the
  \var{position}.  To denote the end of the string, use -1 as
  \var{position}.  When the \var{pattern} is not found \lstinline|-1| is
  returned.
\begin{urbiassert}
"Hello, World!".rfind("o")     == 8;
"Hello, World!".rfind("o", 8)  == 8;
"Hello, World!".rfind("o", 7)  == 4;
"Hello, World!".rfind("o", 3)  == -1;
"Hello, World!".rfind("o", -1) == 8;
\end{urbiassert}


\item[size]
  The size of the string.
\begin{urbiassert}
"foo".size == 3;
   "".size == 0;
\end{urbiassert}


\item \labelSlot{split}%
  \lstinline|(\var{sep} = [" ", "\t", "\n", "\r"], \var{lim} = -1, \var{keepSep} = false, \var{keepEmpty} = true)|\\
  %%% FIXME: incorrect rendering: we lose the backslashes from the
  %%% separators: \item[split](<sep> = [" ", "\t", "\n", "\r"], <lim> = -1,
  %%% <keepSep> = false, <keepEmpty> = true)%
  Split \this on the separator \var{sep}, in at most \var{lim}
  components, which include the separator if \var{keepSep}, and the
  empty components of \var{keepEmpty}.  Return a list of strings.

  The separator, \var{sep}, can be a string.

\begin{urbiassert}
       "a,b;c".split(",") == ["a", "b;c"];
       "a,b;c".split(";") == ["a,b", "c"];
      "foobar".split("x") == ["foobar"];
     "foobar".split("ob") == ["fo", "ar"];
\end{urbiassert}

\noindent
It can also be a list of strings.

\begin{urbiassert}
"a,b;c".split([",", ";"]) == ["a", "b", "c"];
\end{urbiassert}

\noindent
By default splitting is performed on white-spaces:

\begin{urbiassert}
"  abc  def\tghi\n".split == ["abc", "def", "ghi"];
\end{urbiassert}

\noindent
Splitting on the empty string stands for splitting between each character:

\begin{urbiassert}
"foobar".split("") == ["f", "o", "o", "b", "a", "r"];
\end{urbiassert}

The limit \var{lim} indicates a maximum number of splits that can occur. A
negative number corresponds to no limit:

\begin{urbiassert}
"a:b:c".split(":",  1) == ["a", "b:c"];
"a:b:c".split(":", -1) == ["a", "b", "c"];
\end{urbiassert}

\var{keepSep} indicates whether to keep delimiters in the result:

\begin{urbiassert}
"aaa:bbb;ccc".split([":", ";"], -1, false) == ["aaa",      "bbb",      "ccc"];
"aaa:bbb;ccc".split([":", ";"], -1, true)  == ["aaa", ":", "bbb", ";", "ccc"];
\end{urbiassert}

\var{keepEmpty} indicates whether to keep empty elements:

\begin{urbiassert}
"foobar".split("o")                   == ["f", "", "bar"];
"foobar".split("o", -1, false, true)  == ["f", "", "bar"];
"foobar".split("o", -1, false, false) == ["f",     "bar"];
\end{urbiassert}


\item[toAscii] Convert the first character of \this to its integer value in
  the \acro{ASCII} coding.  See also \refSlot{fromAscii}.
\begin{urbiassert}
   "a".toAscii == 97;
   "b".toAscii == 98;
"\xff".toAscii == 0xff;
"Hello, World!\n"
  .asList
  .map(function (var v) { v.toAscii })
  .map(function (var v) { String.fromAscii(v) })
  .join
  == "Hello, World!\n";
\end{urbiassert}


\item[toLower]%
  A String which is \this with upper case letters converted to lower case.
  See also \refSlot{toLower}.
\begin{urbiassert}
    var hello =  "Hello, World!";
hello.toLower == "hello, world!";
        hello == "Hello, World!";
\end{urbiassert}


\item[toUpper]
  A String which is \this with lower case letters converted to upper case.
  See also \refSlot{toLower}.
\begin{urbiassert}
    var hello =  "Hello, World!";
hello.toUpper == "HELLO, WORLD!";
        hello == "Hello, World!";
\end{urbiassert}
\end{urbiscriptapi}

%%% Local Variables:
%%% coding: utf-8
%%% mode: latex
%%% TeX-master: "../urbi-sdk"
%%% ispell-dictionary: "american"
%%% ispell-personal-dictionary: "../urbi.dict"
%%% fill-column: 76
%%% End:
