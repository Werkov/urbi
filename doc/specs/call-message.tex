\section{CallMessage}
Capturing a method invocation: its target and arguments.

\subsection{Examples}
\subsubsection{Evaluating an argument several times}
\label{sec:std-callmsg-examples-several}

The following example implements a lazy function which takes an
integer \var{n}, then arguments.  The \var{n}-th argument is evaluated
twice using \lstinline|CallMessage.evalArgAt|.

\begin{urbiscript}
function callTwice
{
  var n = call.evalArgAt(0);
  call.evalArgAt(n);
  call.evalArgAt(n)
} |;

// Call twice echo("foo").
callTwice(1, echo("foo"), echo("bar"));
[00000001] *** foo
[00000002] *** foo

// Call twice echo("bar").
callTwice(2, echo("foo"), echo("bar"));
[00000003] *** bar
[00000004] *** bar
\end{urbiscript}


\subsubsection{Strict Functions}

Strict functions do support \lstinline|call|.

\begin{urbiscript}[firstnumber=last]
function strict(x)
{
  echo("Entering");
  echo("Strict: " + x);
  echo("Lazy:   " + call.evalArgAt(0));
} |;

strict({echo("1"); 1});
[00000011] *** 1
[00000013] *** Entering
[00000012] *** Strict: 1
[00000013] *** 1
[00000014] *** Lazy:   1
\end{urbiscript}


\subsection{Slots}

\begin{itemize}
\item \lstinline|args|\\
  The list of unevaluated arguments.
\begin{urbiscript}[firstnumber=last]
function args { call.args }|
assert
{
  args == [];
  args() == [];
  args({echo(111); 1}) == [Lazy.new(closure() {echo(111); 1})];
  args(1, 2) == [Lazy.new(closure () {1}),
                 Lazy.new(closure () {2})];
};
\end{urbiscript}


\item \lstinline|argsCount|\\
  Return the number of arguments.  Do not evaluate them.
\begin{urbiscript}[firstnumber=last]
function argsCount { call.argsCount }|;
assert
{
  argsCount == 0;
  argsCount() == 0;
  argsCount({echo(1); 1}) == 1;
  argsCount({echo(1); 1}, {echo(2); 2}) == 2;
};
\end{urbiscript}

\item \lstinline|evalArgAt(\var{n})|\\
  Evaluate the \var{n}-th argument, and return its value.  \var{n}
  must evaluate to an nonnegative integer.  Repeated invocations
  repeat the evaluation, see
  \autoref{sec:std-callmsg-examples-several}.
\begin{urbiscript}[firstnumber=last]
function sumTwice
{
  var n = call.evalArgAt(0);
  call.evalArgAt(n) + call.evalArgAt(n)
}|;

function one () { echo("one"); 1 }|;

sumTwice(1, one, one + one);
[00000008] *** one
[00000009] *** one
[00000010] 2
sumTwice(2, one, one + one);
[00000011] *** one
[00000012] *** one
[00000011] *** one
[00000012] *** one
[00000013] 4

sumTwice(3, one, one);
[00000014:error] !!! evalArgAt: invalid index: 3
sumTwice(3.14, one, one);
[00000015:error] !!! evalArgAt: invalid index: 3.14
\end{urbiscript}

\item \lstinline|evalArgs|\\
  Call \lstinline|evalArgAt| for each argument, return the list of
  values.
\begin{urbiscript}[firstnumber=last]
function twice
{
  call.evalArgs + call.evalArgs
}|;
twice({echo(1); 1}, {echo(2); 2});
[00000011] *** 1
[00000012] *** 2
[00000011] *** 1
[00000012] *** 2
[00000013] [1, 2, 1, 2]
\end{urbiscript}
\end{itemize}


%%% Local Variables:
%%% mode: latex
%%% TeX-master: "../urbi-sdk"
%%% End:
