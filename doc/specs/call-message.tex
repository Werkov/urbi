%% Copyright (C) 2009-2011, Gostai S.A.S.
%%
%% This software is provided "as is" without warranty of any kind,
%% either expressed or implied, including but not limited to the
%% implied warranties of fitness for a particular purpose.
%%
%% See the LICENSE file for more information.

\section{CallMessage}
Capturing a method invocation: its target and arguments.

\subsection{Examples}
\subsubsection{Evaluating an argument several times}
\label{sec:std-callmsg-examples-several}

The following example implements a lazy function which takes an integer
\var{n}, then arguments.  The \var{n}-th argument is evaluated twice using
\refSlot{evalArgAt}.

\begin{urbiscript}[firstnumber=1]
function callTwice
{
  var n = call.evalArgAt(0);
  call.evalArgAt(n);
  call.evalArgAt(n)
} |;

// Call twice echo("foo").
callTwice(1, echo("foo"), echo("bar"));
[00000001] *** foo
[00000002] *** foo

// Call twice echo("bar").
callTwice(2, echo("foo"), echo("bar"));
[00000003] *** bar
[00000004] *** bar
\end{urbiscript}


\subsubsection{Strict Functions}

Strict functions do support \lstinline|call|.

\begin{urbiscript}
function strict(x)
{
  echo("Entering");
  echo("Strict: " + x);
  echo("Lazy:   " + call.evalArgAt(0));
} |;

strict({echo("1"); 1});
[00000011] *** 1
[00000013] *** Entering
[00000012] *** Strict: 1
[00000013] *** 1
[00000014] *** Lazy:   1
\end{urbiscript}


\subsection{Prototypes}
\begin{refObjects}
\item[Object]
\end{refObjects}

\subsection{Slots}

\begin{urbiscriptapi}
\item[args]%
  The list of not yet evaluated arguments.
\begin{urbiscript}
function args { call.args }|
assert
{
  args == [];
  args() == [];
  args({echo(111); 1}) == [Lazy.new(closure() {echo(111); 1})];
  args(1, 2) == [Lazy.new(closure () {1}),
                 Lazy.new(closure () {2})];
};
\end{urbiscript}


\item[argsCount]
  The number of arguments.  Do not evaluate them.
\begin{urbiscript}
function argsCount { call.argsCount }|;
assert
{
  argsCount == 0;
  argsCount() == 0;
  argsCount({echo(1); 1}) == 1;
  argsCount({echo(1); 1}, {echo(2); 2}) == 2;
};
\end{urbiscript}


\item[code]
  The body of the called function as a \refObject{Code}.
\begin{urbiscript}
function code { call.getSlotValue("code") }|
assert (code == getSlotValue("code"));
\end{urbiscript}


\item[eval] Evaluate \this, and return the result.
\begin{urbiscript}
var c1 = do (CallMessage.new)
{
  var this.target = 1;
  var this.message = "+";
  var this.args = [Lazy.new(function () {2})];
}|
assert { c1.eval == 3 };

// A lazy function that returns the sum of this and the second argument,
// regardless of the first argument.
function Float.addSecond
{
  this + call.evalArgAt(1);
}|
var c2 = do (CallMessage.new)
{
  var this.target = 2;
  var this.message = "addSecond";
  var this.args = [Lazy.new(function (){ assert (false) }),
                   Lazy.new(function (){ echo (5); 5 })];
}|
assert { c2.eval == 7 };
[00000454] *** 5
\end{urbiscript}


\item[evalArgAt](<n>)%
  Evaluate the \var{n}-th argument, and return its value.  \var{n}
  must evaluate to an non-negative integer.  Repeated invocations
  repeat the evaluation, see
  \autoref{sec:std-callmsg-examples-several}.
\begin{urbiscript}
function sumTwice
{
  var n = call.evalArgAt(0);
  call.evalArgAt(n) + call.evalArgAt(n)
}|;

function one () { echo("one"); 1 }|;

sumTwice(1, one, one + one);
[00000008] *** one
[00000009] *** one
[00000010] 2
sumTwice(2, one, one + one);
[00000011] *** one
[00000012] *** one
[00000011] *** one
[00000012] *** one
[00000013] 4

sumTwice(3, one, one);
[00000014:error] !!! evalArgAt: invalid index: 3
[00000014:error] !!!    called from: sumTwice
sumTwice(3.14, one, one);
[00000015:error] !!! evalArgAt: invalid index: 3.14
[00000015:error] !!!    called from: sumTwice
\end{urbiscript}


\item[evalArgs]%
  Call \refSlot{evalArgAt} for each argument, return the list of values.
\begin{urbiscript}
function twice
{
  call.evalArgs + call.evalArgs
}|;
twice({echo(1); 1}, {echo(2); 2});
[00000011] *** 1
[00000012] *** 2
[00000011] *** 1
[00000012] *** 2
[00000013] [1, 2, 1, 2]
\end{urbiscript}


\item[message]
  The name under which the function was called.
\begin{urbiscript}
function myself { call.message }|
assert(myself == "myself");
\end{urbiscript}


\item[sender]%
  The object \emph{from which} the invocation was made (the \dfn{caller} in
  other languages).  Not to be confused with \refSlot{target}.
\begin{urbiscript}
function Global.getSender { call.sender } |
function Global.callGetSender { getSender } |

assert
{
  // Call from the current Lobby, with the Lobby as target.
  getSender === lobby;
  // Call from the current Lobby, with Global as the target.
  Global.getSender === lobby;
  // Ask Lobby to call getSender.
  callGetSender === lobby;
  // Ask Global to call getSender.
  Global.callGetSender === Global;
};
\end{urbiscript}


\item[target]%
  The object \emph{on which} the invocation is made.  In other words, the
  object that will be bound to \this during the evaluation.  Not to be
  confused with \refSlot{sender}.
\begin{urbiscript}
function Global.getTarget { call.target } |
function Global.callGetTarget { getTarget } |

assert
{
  // Call from the current Lobby, with the Lobby as target.
  getTarget === lobby;
  // Call from the current Lobby, with Global as the target.
  Global.getTarget === Global;
  // Ask Lobby to call getTarget.
  callGetTarget === lobby;
  // Ask Global to call getTarget.
  Global.callGetTarget === Global;
};
\end{urbiscript}
\end{urbiscriptapi}


%%% Local Variables:
%%% coding: utf-8
%%% mode: latex
%%% TeX-master: "../urbi-sdk"
%%% ispell-dictionary: "american"
%%% ispell-personal-dictionary: "../urbi.dict"
%%% fill-column: 76
%%% End:
