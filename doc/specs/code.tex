%% Copyright (C) 2009-2010, Gostai S.A.S.
%%
%% This software is provided "as is" without warranty of any kind,
%% either expressed or implied, including but not limited to the
%% implied warranties of fitness for a particular purpose.
%%
%% See the LICENSE file for more information.

\section{Code}

Functions written in \us.

\subsection{Prototypes}

\begin{refObjects}
\item[Comparable]
\item[Executable]
\end{refObjects}

\subsection{Construction}

The keywords \lstinline|function| and \lstinline|closure| build Code
instances.

\begin{urbiassert}
function(){}.protos[0] === getSlot("Code");
closure(){}.protos[0] === getSlot("Code");
\end{urbiassert}

\subsection{Slots}

\begin{urbiscriptapi}
\item[==](<that>)%
  Whether \this and \var{that} are the same source code (actually checks
  that both have the same \refSlot{asString}), and same closed values.

  Closures and functions are different, even if the body is the same.
\begin{urbiassert}
function () { 1 } == function () { 1 };
function () { 1 } != closure  () { 1 };
closure  () { 1 } != function () { 1 };
closure  () { 1 } == closure  () { 1 };
\end{urbiassert}

No form of equivalence is applied on the body, it must be the same.
\begin{urbiassert}
function () { 1 + 1 } == function () { 1 + 1 };
function () { 1 + 2 } != function () { 2 + 1 };
\end{urbiassert}

Arguments do matter, even if in practice the functions are the same.
\begin{urbiassert}
function (var ignored) {} != function () {};
function (var x) { x }    != function (y) { y };
\end{urbiassert}

A lazy function cannot be equal to a strict one.
\begin{urbiassert}
function () { 1 } != function { 1 };
\end{urbiassert}

If the functions capture different variables, they are different.
\begin{urbiscript}
{
  var x;
  function Global.capture_x() { x };
  function Global.capture_x_again () { x };
  {
    var x;
    function Global.capture_another_x() { x };
  }
}|;
assert
{
  getSlot("capture_x") == getSlot("capture_x_again");
  getSlot("capture_x") != getSlot("capture_another_x");
};
\end{urbiscript}

If the functions capture different targets, they are different.
\begin{urbiscript}
class Foo
{
  function makeFunction() { function () {} };
  function makeClosure()  { closure () {} };
}|;

class Bar
{
  function makeFunction() { function () {} };
  function makeClosure()  { closure () {} };
}|;

assert
{
  Foo.makeFunction() == Bar.makeFunction();
  Foo.makeClosure()  != Bar.makeClosure();
};
\end{urbiscript}

\item[apply](<args>)%
  Invoke the routine, with all the arguments.  The target, \this, will be
  set to \lstinline|\var{args}[0]| and the remaining arguments with be given
  as arguments.
\begin{urbiassert}
function (x, y) { x+y }.apply([nil, 10, 20]) == 30;
function () { this }.apply([123]) == 123;

// There is Object.apply.
1.apply([this]) == 1;
\end{urbiassert}
\begin{urbiscript}
function () {}.apply([]);
[00000001:error] !!! apply: argument list must begin with `this'

function () {}.apply([1, 2]);
[00000002:error] !!! apply: expected 0 argument, given 1
\end{urbiscript}

\item[asString]
  Conversion to \refObject{String}.
\begin{urbiassert}
closure  () { 1 }.asString == "closure () {\n  1\n}";
function () { 1 }.asString == "function () {\n  1\n}";
\end{urbiassert}

\item[bodyString]
  Conversion to \refObject{String} of the routine body.
\begin{urbiassert}
closure  () { 1 }.bodyString == "1";
function () { 1 }.bodyString == "1";
\end{urbiassert}

\end{urbiscriptapi}

%%% Local Variables:
%%% mode: latex
%%% TeX-master: "../urbi-sdk"
%%% ispell-dictionary: "american"
%%% ispell-personal-dictionary: "../urbi.dict"
%%% fill-column: 76
%%% End:
