%% Copyright (C) 2009-2011, Gostai S.A.S.
%%
%% This software is provided "as is" without warranty of any kind,
%% either expressed or implied, including but not limited to the
%% implied warranties of fitness for a particular purpose.
%%
%% See the LICENSE file for more information.

\section{Regexp}

A Regexp is an object which allow you to match strings with a regular
expression.

\subsection{Prototypes}
\begin{refObjects}
\item[Container]
\item[Object]
\end{refObjects}

\subsection{Construction}
\label{stdlib:regexp:ctor}

A \lstinline{Regexp} is created from a regular expression once and for all;
it can be used several times to match with other strings.

\begin{urbiscript}
Regexp.new(".");
[00000001] Regexp(".")

var num = Regexp.new("\\d+\\.\\d+");
[00000004] Regexp("\\d+\\.\\d+")
"1.3" in num;
[00019618] true
"1." in num;
[00023113] false
\end{urbiscript}

\us supports Perl regular expressions, see
\href{http://perldoc.perl.org/perlre.html}{the perlre man page}.

Expressions cannot be empty, and must be syntactically correct.

%% This is likely to fails on some other arch because the error messages
%% might be different.  This is to handle in neutralize, not here.
\begin{urbiscript}
Regexp.new("");
[00000001:error] !!! new: invalid regular expression: empty expression: `'

Regexp.new("(");
[00003237:error] !!! new: invalid regular expression: \
[:]    unmatched marking parenthesis ( or \(: `(>>>HERE>>>'

Regexp.new("*");
[00004372:error] !!! new: invalid regular expression: \
[:]    the repeat operator "*" cannot start a regular expression: `>>>HERE>>>*'
\end{urbiscript}



\subsection{Slots}
\begin{urbiscriptapi}
\item|'[]'|(<n>)%
  Same as \lstinline|this.matches[\var{n}]|.
\begin{urbiscript}
var d = Regexp.new("(1+)(2+)(3+)")|;
assert
{
  "01223334" in d;
  d[0] == "122333";
  d[1] == "1";
  d[2] == "22";
  d[3] == "333";
};
d[4];
[00000009:error] !!! []: out of bound index: 4
\end{urbiscript}


\item[asPrintable] A string that shows that \this is a Regexp, and its
  value.
\begin{urbiassert}
           Regexp.new("abc").asPrintable() == "Regexp(\"abc\")";
Regexp.new("\\d+(\\.\\d+)?").asPrintable() == "Regexp(\"\\\\d+(\\\\.\\\\d+)?\")";
\end{urbiassert}


\item[asString] The regular expression that was compiled.
\begin{urbiassert}
           Regexp.new("abc").asString() == "abc";
Regexp.new("\\d+(\\.\\d+)?").asString() == "\\d+(\\.\\d+)?";
\end{urbiassert}


\item[has](<str>)%
  An experimental alias to \refSlot{match}, so that the infix operators
  \lstinline|in| and \lstinline|not in| can be used (see
  \autoref{sec:lang:op:containers}).
\begin{urbiassert}
"23.03"     in Regexp.new("^\\d+\\.\\d+$");
"-3.14" not in Regexp.new("^\\d+\\.\\d+$");
\end{urbiassert}


\item[match](<str>)%
  Whether \this matches \var{str}.
\begin{urbiscript}
// Ordinary characters
var r = Regexp.new("oo")|
assert
{
  r.match("oo");
  r.match("foobar");
  !r.match("bazquux");
};

// ^, anchoring at the beginning of line.
r = Regexp.new("^oo")|
assert
{
  r.match("oops");
  !r.match("woot");
};

// $, anchoring at the end of line.
r = Regexp.new("oo$")|
assert
{
  r.match("foo");
  !r.match("mooh");
};

// *, greedy repetition, 0 or more.
r = Regexp.new("fo*bar")|
assert
{
  r.match("fbar");
  r.match("fooooobar");
  !r.match("far");
};

// (), grouping.
r = Regexp.new("f(oo)*bar")|
assert
{
  r.match("foooobar");
  !r.match("fooobar");
};
\end{urbiscript}


\item[matches]%
  If the latest \refSlot{match} was successful, the matched groups, as
  delimited by parentheses in the regular expression; the first element
  being the whole match.  Otherwise, the empty list.  See also
  \refSlot{'[]'}.

\begin{urbiscript}
var re = Regexp.new("([a-zA-Z0-9._]+)@([a-zA-Z0-9._]+)")|;
assert
{
  re.match("Someone <someone@somewhere.com>");
  re.matches == ["someone@somewhere.com", "someone", "somewhere.com"];

  "does not match" not in re;
  re.matches == [];
};
\end{urbiscript}
\end{urbiscriptapi}

%%% Local Variables:
%%% coding: utf-8
%%% mode: latex
%%% TeX-master: "../urbi-sdk"
%%% ispell-dictionary: "american"
%%% ispell-personal-dictionary: "../urbi.dict"
%%% fill-column: 76
%%% End:
