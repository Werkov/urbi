\section{Regexp}

A Regexp is an object which allow you to match strings with a regular
expression.

\subsection{Prototypes}
\begin{itemize}
\item \refObject{Object}
\end{itemize}

\subsection{Construction}
\label{stdlib:regexp:ctor}

A \lstinline{Regexp} is created with the regular expression once and
for all, and it can be used many times to match with other strings.

\begin{urbiscript}[firstnumber=1]
Regexp.new(".");
[00000001] Regexp(".")
\end{urbiscript}

\us supports Perl regular expressions, see
\href{http://perldoc.perl.org/perlre.html}{the perlre man page}.
Expressions cannot be empty.

\subsection{Slots}
\begin{itemize}
\item \lstinline|match(\var{str})|\\
  Whether \lstinline|this| matches \var{str}.
\begin{urbiscript}
// Ordinary characters
var r = Regexp.new("oo")|
assert(r.match("oo"));
assert(r.match("foobar"));
assert(!r.match("bazquux"));

// ^, anchoring at the beginning of line.
r = Regexp.new("^oo")|
assert(r.match("oops"));
assert(!r.match("woot"));

// $, anchoring at the end of line.
r = Regexp.new("oo$")|
assert(r.match("foo"));
assert(!r.match("mooh"));

// *, greedy repetition, 0 or more.
r = Regexp.new("fo*bar")|
assert(r.match("fbar"));
assert(r.match("fooooobar"));
assert(!r.match("far"));

// (), grouping.
r = Regexp.new("f(oo)*bar")|
assert(r.match("foooobar"));
assert(!r.match("fooobar"));
\end{urbiscript}

\end{itemize}

%%% Local Variables:
%%% mode: latex
%%% TeX-master: "../urbi-sdk"
%%% ispell-dictionary: "american"
%%% ispell-personal-dictionary: "../urbi.dict"
%%% fill-column: 76
%%% End:
