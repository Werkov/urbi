\section{Event}
Entities that can be ``emited'' and ``caught'', or ``sent'' and
``received''.  See also \autoref{sec:tut:events}.

\subsection{Prototypes}
\begin{itemize}
\item \refObject{Object}
\end{itemize}

\subsection{Construction}

An \lstinline{Event} is created like any other object, without arguments.

\begin{urbiscript}
var e = Event.new;
[00000001] Event_0x9ad8118
\end{urbiscript}

\subsection{Methods}
\begin{itemize}
\item \lstinline|asEvent|\\
  Return \lstinline|this|.

\item \lstinline|'emit'|\\
  Throw an \lstinline|Event|. The operator bang can also be used. This
  function can take zero or more argument of same or different type.
  These argument are passed when the throwed event are caugth. An
  event can also be throwed for a certain time using \lstinline|~|.

\item \lstinline|syncEmit|\\
  Throw a synchronized event. This call waits that all functions that
  have to react to this event have returned. This function can have
  the same argument as \lstinline|emit|.

\item \lstinline|trigger|\\
  This function is used to launch an event during a not already known
  time. Calling this function launchs and keeps the event trigerred
  and returns an object that can be stopped with a method named stop
  to stop launching the event.
\end{itemize}

%%% Local Variables:
%%% mode: latex
%%% TeX-master: "../urbi-sdk"
%%% End:
