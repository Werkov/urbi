\section{Event}

An \dfn{event} can be ``emited'' and ``caught'', or ``sent'' and
``received''.  See also \autoref{sec:tut:events}.

\subsection{Prototypes}
\begin{itemize}
\item \refObject{Object}
\end{itemize}

\subsection{Construction}

An \lstinline{Event} is created like any other object, without
arguments.

\begin{urbiscript}
var e = Event.new;
[00000001] Event_0x9ad8118
\end{urbiscript}

\subsection{Slots}
\begin{itemize}
\item \lstinline|asEvent|\\
  Return \lstinline|this|.

\item \lstinline|'emit'|\\
  Throw an \lstinline|Event|. This function is called by the bang
  operator.  It takes any number of arguments, passed to the receiver
  when the event is caught. An event can also be emitted for a certain
  duration using \lstinline|~|.

\item \lstinline|syncEmit|\\
  Throw a synchronized event. This call awaits that all functions that
  have to react to this event have returned.  This function can have
  the same arguments as \lstinline|emit|.

\item \lstinline|trigger|\\
  This function is used to launch an event during an unknown amount of
  time. Calling this function launchs and keeps the event trigerred
  and returns an object whose \lstinline|stop| method stops launching
  the event.
\end{itemize}

%%% Local Variables:
%%% mode: latex
%%% TeX-master: "../urbi-sdk"
%%% ispell-personal-dictionary: "../urbi.dict"
%%% End:
