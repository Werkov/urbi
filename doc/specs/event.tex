%% Copyright (C) 2009-2011, Gostai S.A.S.
%%
%% This software is provided "as is" without warranty of any kind,
%% either expressed or implied, including but not limited to the
%% implied warranties of fitness for a particular purpose.
%%
%% See the LICENSE file for more information.

\section{Event}

An \dfn{event} can be ``emitted'' and ``caught'', or ``sent'' and
``received''.  See also \autoref{sec:tut:events}.

\subsection{Examples}

There are several examples of uses of events in the documentation of
event-based constructs.  See \lstinline{at} (\autoref{sec:lang:at}),
\lstinline{waituntil} (\autoref{sec:lang:waituntil}), \lstinline{whenever}
(\autoref{sec:lang:whenever}), and so forth.  The tutorial chapter about
event-based programming contains other examples, see
\autoref{sec:tut:event-prog}.

\subsection{Synchronicity of Event Handling}
\label{sec:event:sync}
A particular emphasis should be put on the \dfn{synchronicity} of the event
handling, i.e., whether the bodies of the event handlers are run before the
control flow returns from the event emission.  By default, (i.e.,
\lstinline|at (e?...)| and \lstinline|e!(...)|/\lstinline|e.emit(...)|)) the
execution is \dfn{asynchronous}, but if either the emitted or the handler is
marked asynchronous (i.e., \lstinline|at sync (e?...)| or
\lstinline|e.syncEmit(...)|), then the execution is \dfn{synchronous}.

Contrast the following examples:

\begin{multicols}{2}
\paragraph{Asynchronous handlers}~

\begin{urbiscript}[xrightmargin=0mm,xleftmargin=0mm]
var e = Event.new |;

at (e?)
  { echo("a"); sleep(20ms); echo("b") }
onleave
  { echo("c"); sleep(20ms); echo("d") };

e! | echo("done");
[00000001] *** done
sleep(25ms);
[00000002] *** a
[00000003] *** c
[00000101] *** b
[00000102] *** d

e.syncEmit | echo("done");
[00000001] *** a
[00000101] *** b
[00000102] *** c
[00000202] *** d
[00000203] *** done
\end{urbiscript}
\columnbreak

\paragraph{Synchronous handlers}~

\begin{urbicomment}
removeSlots("e");
\end{urbicomment}
\begin{urbiscript}[xrightmargin=0mm,xleftmargin=0mm]
var e = Event.new |;

at sync (e?)
  { echo("a"); sleep(20ms); echo("b") }
onleave
  { echo("c"); sleep(20ms); echo("d") };

e! | echo("done");
// No need to sleep.
[00000011] *** a
[00000031] *** b
[00000031] *** c
[00000052] *** d
[00000052] *** done

e.syncEmit | echo("done");
[00000052] *** a
[00000073] *** b
[00000073] *** c
[00000094] *** d
[00000094] *** done
\end{urbiscript}
\end{multicols}

For more information about the synchronicity of event handlers, see
\autoref{sec:lang:at:sync-async}.

\subsection{Sustained Events}
\label{sec:event:sustain}

Events can also be sustained during a time span starting at
\refSlot{trigger} and ending at \lstinline|handler.stop|.  Note that the
\lstinline|onleave|-clauses of the event handlers is not executed right
after the event was first triggered, but rather when it is stopped.

Synchronicity for sustained events is more complex: the
\lstinline|at|-clause is handled asynchronously iff \emph{both} the emission
and the handler are asynchronous, whereas the \lstinline|onleave|-clause is
handled asynchronously iff the emission was synchronous.  Be warned, but do
not depend on this, as in the future we might change this.

\begin{multicols}{2}
\paragraph{Asynchronous Trigger}~

\begin{urbicomment}
removeSlots("e");
\end{urbicomment}
\begin{urbiscript}[xrightmargin=0mm,xleftmargin=0mm]
var e = Event.new|;
at (e?(var v))
  { echo("a"+v); sleep(20ms); echo("b"+v) }
onleave
  { echo("c"+v); sleep(20ms); echo("d"+v) };

var handler = e.trigger("1") | echo("?");
[00000001] *** ?
[00000002] *** a1
[00000102] *** b1
sleep(200ms);
handler.stop | echo("?");
[00000301] *** ?
sleep(25ms);
[00000302] *** c1
[00000402] *** d1

// at and onleave clauses may overlap.
handler = e.trigger("2") | handler.stop;
sleep(25ms);
[00000001] *** a2
[00000002] *** c2
sleep(25ms);
[00000201] *** b2
[00000202] *** d2

handler = e.syncTrigger("3") | echo("?");
[00000002] *** a3
[00000102] *** b3
[00000001] *** ?
handler.stop | echo("?");
[00000302] *** c3
[00000402] *** d3
[00000301] *** ?
\end{urbiscript}
\columnbreak

\paragraph{Synchronous Trigger}~

\begin{urbicomment}
removeSlots("e", "handler");
\end{urbicomment}
\begin{urbiscript}[xrightmargin=0mm,xleftmargin=0mm]
var e = Event.new|;
at sync (e?(var v))
  { echo("a"+v); sleep(20ms); echo("b"+v) }
onleave
  { echo("c"+v); sleep(20ms); echo("d"+v) };

var handler = e.trigger("1") | echo("?");
// No need to sleep.
[00000002] *** a1
[00000102] *** b1
[00000001] *** ?
handler.stop | echo("?");
[00000301] *** ?
sleep(25ms);
[00000302] *** c1
[00000402] *** d1

// at and onleave clauses don't overlap.
handler = e.trigger("2") | handler.stop;
sleep(25ms);
[00000001] *** a2
[00000201] *** b2
[00000002] *** c2
[00000202] *** d2

handler = e.syncTrigger("3") | echo("?");
[00000002] *** a3
[00000102] *** b3
[00000001] *** ?
handler.stop | echo("?");
[00000302] *** c3
[00000402] *** d3
[00000301] *** ?
\end{urbiscript}
\end{multicols}


\subsection{Prototypes}
\begin{refObjects}
\item[Object]
\end{refObjects}

\subsection{Construction}
\label{sec:stdlib:event:ctor}

An \lstinline{Event} is created like any other object.  The constructor
takes no argument.

\begin{urbiscript}[firstnumber=1]
var e = Event.new;
[00000001] Event_0x9ad8118
\end{urbiscript}

\subsection{Slots}
\begin{urbiscriptapi}
\item[asEvent]
  Return \this.

\item[emit](<args>[])%
  Fire an ``instantaneous'' and ``asynchronous'' \refObject{Event}. This
  function is called by the \lstinline|!| operator.  It takes any number of
  arguments, passed to the receiver when the event is caught.
\begin{urbicomment}
removeSlots("e");
\end{urbicomment}
\begin{urbiscript}
var e = Event.new|;
// No handler, lost message.
e.emit;
at (e?)               echo("e");
at (e?())             echo("e()");
at (e?(var x))        echo("e(%s)" % [x]);
at (e?(var x, var y)) echo("e(%s, %s)" % [x, y]);

// This is what e! does.
e.emit;
[00000135] *** e
[00000135] *** e()

// This is what e!() does: same as e!.
e.emit();
[00000138] *** e
[00000138] *** e()

// This is what e!(1, [2]) does.
e.emit(1, [2]);
[00000141] *** e
[00000141] *** e(1, [2])

// This is what e!(1, [2], "three") does.
e.emit(1, [2], "three");
[00000146] *** e
\end{urbiscript}

To sustain an event, see \refSlot{trigger}.  See \autoref{sec:event:sync}
and \refSlot{syncEmit} for details about the synchronicity of the handling.

  %% An event can also be emitted for a certain duration using
  %% \lstinline|~|.  The execution of \lstinline|at| clauses etc., is
  %% asynchronous: the control flow might be released by the
  %% \lstinline|emit| call before all the watchers have finished their
  %% execution.

\item[onEvent](<guard>, <enter>, <leave>, <sync>)%
  This is the low-level routine used to implement the \lstindex|at|
  construct.  Indeed,
  \lstinline|at (\var{e}? if \var{cond}) \var{enter} onleave \var{leave}|
  is (roughly) translated into
\begin{urbiunchecked}
\var{e}
  .onEvent(
    closure (var '$evt', var '$payload')                 { \var{cond}  },
    closure (var '$evt', var '$payload', var '$pattern') { \var{enter} },
    closure (var '$evt', var '$payload', var '$pattern') { \var{leave} },
    false)
\end{urbiunchecked}

\noindent
where the \lstinline|false| would be \lstinline|true| in case of an
\lstinline|at sync| construct.  The \var{cond} discards the event iff it
returns \lstinline{void}.

\begin{urbicomment}
removeSlots("e");
\end{urbicomment}
\begin{urbiscript}
var e = Event.new|;
e.onEvent(
  function (var args[]) { echo("cond 1") | true },
  function (var args[]) { echo("enter 1") },
  function (var args[]) { echo("leave 1") },
  true);

e.onEvent(
  function (var args[]) { echo("cond 2") },
  function (var args[]) { echo("enter 2") },
  function (var args[]) { echo("leave 2") },
  true);

e.emit(12);
[00001619] *** cond 1
[00001619] *** enter 1
[00001619] *** leave 1
[00001619] *** cond 2

var h = e.trigger|;
[00001620] *** cond 1
[00001620] *** enter 1
[00001620] *** cond 2

h.stop;
[00001621] *** leave 1
\end{urbiscript}

This function is internal and it might change in the future.

\item[onSubscribe]%
  This slot is not set by default. You can optionally assign an event to
  it. In this case, it is triggered each time some code starts watching this
  event (by setting up an \lstinline|at| or a \lstinline|waituntil| on it
  for instance).

  Throw a synchronized event. This call awaits that all functions that have
  to react to this event have returned.  This function can have the same
  arguments as \refSlot{emit}.

\item[syncEmit](<args>[])%
  Same as \refSlot{emit} but require a synchronous handling.  See
  \autoref{sec:event:sync} for details.

\item[syncTrigger](<args>[])%
  Same as \refSlot{trigger} but the call will be synchronous (see
  \autoref{sec:event:sync}). The \lstinline|stop| method of the handler
  object will be synchronous as well.  See \autoref{sec:event:sustain} for
  examples.

\item[trigger](<args>[])%
  Fire a sustained event (for an unknown amount of time) and return a
  handler object whose \lstinline|stop| method stops the event. This method
  is asynchronous and the \lstinline|stop| call will be asynchronous as
  well.  See \autoref{sec:event:sustain} for examples.

\item \lstinline+'||'(\var{that})+%
  Logical ``or'' on events: a new Event that triggers whenever \this or
  \that triggers.

\begin{urbiscript}
var e1 = Event.new|;
var e2 = Event.new|;
var e_or = e1 || e2|;
at (e_or?)
  echo("!");
e1!;
[00000004] *** !
e2!;
[00000005] *** !
\end{urbiscript}

\item['<<'](<that>)%
  Watch a \that event status and reproduce it on itself, return \this.  This
  operator is similar to an optimized \lstinline,||=, operator.  Do not make
  events watch for themselves, directly or indirectly.

\begin{urbiscript}
var e3 = Event.new|;
var e4 = Event.new|;
var e_watch = Event.new << e3 << e4 |;
at (e_watch?)
  echo("!");
e3!;
[00000006] *** !
e4!;
[00000007] *** !
\end{urbiscript}


\end{urbiscriptapi}

%%% Local Variables:
%%% mode: latex
%%% TeX-master: "../urbi-sdk"
%%% ispell-dictionary: "american"
%%% ispell-personal-dictionary: "../urbi.dict"
%%% fill-column: 76
%%% End:
