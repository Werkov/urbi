\section{FormatInfo}

A \dfn{format info} is used when formatting a la \code{printf}. It
store the formatting pattern itself and all the format information it
can extract from the pattern.

\subsection{Prototypes}

\begin{itemize}
\item \refObject{Object}
\end{itemize}

\subsection{Construction}

The constructor expects a string as argument, whose syntax is similar
to \code{printf}'s.  It is detailed below.

\begin{urbiscript}
var f = FormatInfo.new("%+2.3d");
[00000001] %+2.3d
\end{urbiscript}

\subsection{Syntax}

A formatting pattern must one of the following (brackets denote
optional arguments):
\begin{itemize}
\item \verb&%&\var{options} \var{spec}
\item \verb&%|&\var{options}[\var{spec}]\verb&|&
\end{itemize}

\noindent
\var{options} is a sequence of 0 or several of the following
characters:

\begin{tabular}{|c|l|}
  \hline
  \samp{-} & Left alignment.\\
  \samp{=} & Centered alignment.\\
  \samp{+} & Show sign even for positive number.\\
  \samp{ } & If the string does not begin with \samp{+} or \samp{-}, insert
  a space before the converted string.\\
  \samp{0} & Pad with 0's (inserted after sign or base indicator).\\
  \samp{\#} & Show numerical base, and decimal point.\\
  \samp{'} & Split thousands (\samp{1 000}).\\
  \hline
\end{tabular}

\noindent
\var{spec} is the convertion character and must be one of the
following:

\begin{tabular}{|c|l|}
  \hline
  \samp{s} & Default character, prints normally\\
  \samp{d} & Case modifier: lowercase \\
  \samp{D} & Case modifier: uppercase \\
  \samp{x} & Prints in hexadecimal lowercase \\
  \samp{X} & Prints in hexadecimal uppercase \\
  \samp{o} & Prints in octal\\
  \samp{b} & Prints in binary\\
  \hline
\end{tabular}
