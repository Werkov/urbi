%% Copyright (C) 2009-2010, Gostai S.A.S.
%%
%% This software is provided "as is" without warranty of any kind,
%% either expressed or implied, including but not limited to the
%% implied warranties of fitness for a particular purpose.
%%
%% See the LICENSE file for more information.

\section{Exception}

Exceptions are used to handle errors.  More generally, they are a
means to escape from the normal control-flow to handle exceptional
situations.

The language support for throwing and catching exceptions (using
\lstinline|try|/\lstinline|catch| and \lstinline|throw|, see
\autoref{sec:lang:except}) work perfectly well with any kind of
object, yet it is a good idea to throw only objects that derive from
\lstinline|Exception|.

\subsection{Prototypes}

\begin{refObjects}
\item[Object]
\item[Traceable]
\end{refObjects}

\subsection{Construction}

There are several types of exceptions, each of which corresponding to
a particular kind of error.  The top-level object,
\lstinline|Exception|, takes a single argument: an error message.

\begin{urbiscript}[firstnumber=1]
Exception.new("something bad has happened!");
[00000001] Exception `something bad has happened!'
Exception.Arity.new("myRoutine", 1, 10, 23);
[00000002] Exception.Arity `myRoutine: expected between 10 and 23 arguments, given 1'
\end{urbiscript}


\subsection{Slots}

Exception has many slots which are specific exceptions.  See
\autoref{sec:specs:except:sub} for their documentation.

\begin{urbiscriptapi}
\item[backtrace] The call stack at the moment the exception was thrown (not
  created), as a \refObject{List} of \refObject[s]{StackFrame}, from the
  innermost to the outermost call.  (see \refSlot[backtrace]{Traceable})
\begin{urbiscript}
//#push 1 "file.u"
try
{
  function innermost () { throw Exception.new("Ouch") };
  function inner     () { innermost() };
  function outer     () { inner() };
  function outermost () { outer() };

  outermost();
}
catch (var e)
{
  assert
  {
    e.backtrace[0].location.asString == "file.u:4.27-37";
    e.backtrace[0].name == "innermost";

    e.backtrace[1].location.asString == "file.u:5.27-33";
    e.backtrace[1].name == "inner";

    e.backtrace[2].location.asString == "file.u:6.27-33";
    e.backtrace[2].name == "outer";

    e.backtrace[3].location.asString == "file.u:8.3-13";
    e.backtrace[3].name == "outermost";
  };
};
//#pop
\end{urbiscript}

\item[location] The location from which the exception was thrown (not
  created).
\begin{urbiscript}
eval("1/0");
[00090441:error] !!! 1.1-3: /: division by 0
try
{
  eval("1/0");
}
catch (var e)
{
  assert (e.location.asString == "1.1-3");
};
\end{urbiscript}

\item[message] The error message provided at construction.
\begin{urbiassert}
Exception.new("Ouch").message == "Ouch";
\end{urbiassert}
\end{urbiscriptapi}

\subsection{Specific Exceptions}
\label{sec:specs:except:sub}

\begin{itemize}
\item \lstinline|ArgumentType.new(\var{routine}, \var{index}, \var{effective}, \var{expected})|\\
  Derives from \refSlot{Type}.  The \var{routine} was
  called with a \var{index}-nth argument of type \var{effective}
  instead of \var{expected}.
\begin{urbiscript}
Exception.ArgumentType.new("myRoutine", 1, "hisResult", "myExceptation");
[00000003] Exception.ArgumentType `myRoutine: unexpected "hisResult" for argument 1, expected a String'
\end{urbiscript}

\item \lstinline|Arity.new(\var{routine}, \var{effective}, \var{min}, \var{max} = void)|\\
  The \var{routine} was called with an incorrect number of arguments
  (\var{effective}).  It requires at least \var{min} arguments, and,
  if specified, at most \var{max}.
\begin{urbiscript}
Exception.Arity.new("myRoutine", 1, 10, 23);
[00000004] Exception.Arity `myRoutine: expected between 10 and 23 arguments, given 1'
\end{urbiscript}
%% try
%% {
%%   Math.cos(1, 2);
%% }
%% catch (var e)
%% {
%%   assert(e == Exception.Arity.new("cos", 2, 1));
%% };
\item \lstinline|BadInteger.new(\var{routine}, \var{fmt}, \var{effective})|\\
  The \var{routine} was called with an inappropriate integer
  (\var{effective}).  Use the format \var{fmt} to create an error
  message from \var{effective}.  Derives from
  \refSlot{BadNumber}.
\begin{urbiscript}
Exception.BadInteger.new("myRoutine", "bad integer: %s", 12);
[00000005] Exception.BadInteger `myRoutine: bad integer: 12'
\end{urbiscript}

\item \lstinline|BadNumber.new(\var{routine}, \var{fmt}, \var{effective})|\\
  The \var{routine} was called with an inappropriate number
  (\var{effective}).  Use the format \var{fmt} to create an error
  message from \var{effective}.
\begin{urbiscript}
Exception.BadNumber.new("myRoutine", "bad number: %s", 12.34);
[00000005] Exception.BadNumber `myRoutine: bad number: 12.34'
\end{urbiscript}

\item \lstinline|Constness.new(\var{msg})|\\
  An attempt was made to change a constant value.
\begin{urbiscript}
Exception.Constness.new;
[00000006] Exception.Constness `cannot modify const slot'
\end{urbiscript}

\item \lstinline|FileNotFound.new(\var{name})|\\
  The file named \var{name} cannot be found.
\begin{urbiscript}
Exception.FileNotFound.new("foo");
[00000007] Exception.FileNotFound `file not found: foo'
\end{urbiscript}

\item \lstinline|ImplicitTagComponent.new(\var{msg})|\\
  An attempt was made to create an implicit tag, a component of which
  being undefined.
\begin{urbiscript}
Exception.ImplicitTagComponent.new;
[00000008] Exception.ImplicitTagComponent `invalid component in implicit tag'
\end{urbiscript}

\item \lstinline|Lookup.new(\var{object}, \var{name})|\\
  A failed name lookup was performed om \var{object} to find a slot
  named \var{name}.  If \lstinline|Exception.Lookup.fixSpelling| is
  true (which is the default), suggest what the user might have meant
  to use.
\begin{urbiscript}
Exception.Lookup.new(Object, "GetSlot");
[00000009] Exception.Lookup `lookup failed: Object'
\end{urbiscript}

\item \lstinline|MatchFailure.new|\\
  A pattern matching failed.
\begin{urbiscript}
Exception.MatchFailure.new;
[00000010] Exception.MatchFailure `pattern did not match'
\end{urbiscript}

\item \lstinline|NegativeNumber.new(\var{routine}, \var{effective})|\\
  The \var{routine} was called with a negative number
  (\var{effective}).  Derives from \refSlot{BadNumber}.
\begin{urbiscript}
Exception.NegativeNumber.new("myRoutine", -12);
[00000005] Exception.NegativeNumber `myRoutine: expected non-negative number, got -12'
\end{urbiscript}

\item \lstinline|NonPositiveNumber.new(\var{routine}, \var{effective})|\\
  The \var{routine} was called with a non-positive number
  (\var{effective}).  Derives from \refSlot{BadNumber}.
\begin{urbiscript}
Exception.NonPositiveNumber.new("myRoutine", -12);
[00000005] Exception.NonPositiveNumber `myRoutine: expected positive number, got -12'
\end{urbiscript}

\item \lstinline|Primitive.new(\var{routine}, \var{msg})|\\
  The built-in \var{routine} encountered an error described by
  \var{msg}.
\begin{urbiscript}
Exception.Primitive.new("myRoutine", "cannot do that");
[00000011] Exception.Primitive `myRoutine: cannot do that'
\end{urbiscript}

\item \lstinline|Redefinition.new(\var{name})|\\
  An attempt was made to refine a slot named \var{name}.
\begin{urbiscript}
Exception.Redefinition.new("foo");
[00000012] Exception.Redefinition `slot redefinition: foo'
\end{urbiscript}

\item \lstinline|Scheduling.new(\var{msg})|\\
  Something really bad has happened with the \urbi task scheduler.
\begin{urbiscript}
Exception.Scheduling.new("cannot schedule");
[00000013] Exception.Scheduling `cannot schedule'
\end{urbiscript}

\item \lstinline|Syntax.new(\var{loc}, \var{message}, \var{input})|\\
  Declare a syntax error in \var{input}, at location \var{loc},
  described by \var{message}.  \var{loc} is the location of the syntax
  error, \var{location} is the place the error was thrown.  They are
  usually equal, except when the errors are caught while using
  \refSlot[System]{eval} or \refSlot[System]{load}.  In that case
  \var{loc} is really the position of the syntax error, while
  \var{location} refers to the location of the \refSlot[System]{eval}
  or \refSlot[System]{load} invocation.
\begin{urbiscript}
Exception.Syntax.new(Location.new(Position.new("file.u", 14, 25)),
                     "unexpected pouCharque", "file.u");
[00000013] Exception.Syntax `file.u:14.25: syntax error: unexpected pouCharque'

try
{
  eval("1 / / 0");
}
catch (var e)
{
  assert
  {
    e.isA(Exception.Syntax);
    e.loc.asString == "1.5";
    e.input == "1 / / 0";
    e.message == "unexpected /";
  }
};
\end{urbiscript}


\item \lstinline|Type.new(\var{effective}, \var{expected})|\\
  A value of type \var{effective} was received, while a value of type
  \var{expected} was expected.
\begin{urbiscript}
Exception.Type.new("hisResult", "myExceptation");
[00000014] Exception.Type `unexpected "hisResult", expected a String'
\end{urbiscript}

\item \refSlot[UnexpectedVoid]{new}\\
  An attempt was made to read the value of \lstinline|void|.
\begin{urbiscript}
Exception.UnexpectedVoid.new;
[00000015] Exception.UnexpectedVoid `unexpected void'
var a = void;
a;
[00000016:error] !!! unexpected void
[00000017:error] !!! lookup failed: a
\end{urbiscript}

\end{itemize}


%%% Local Variables:
%%% mode: latex
%%% TeX-master: "../urbi-sdk"
%%% ispell-dictionary: "american"
%%% ispell-personal-dictionary: "../urbi.dict"
%%% fill-column: 76
%%% End:
