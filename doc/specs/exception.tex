%% Copyright (C) 2009-2011, Gostai S.A.S.
%%
%% This software is provided "as is" without warranty of any kind,
%% either expressed or implied, including but not limited to the
%% implied warranties of fitness for a particular purpose.
%%
%% See the LICENSE file for more information.

\section{Exception}

Exceptions are used to handle errors.  More generally, they are a
means to escape from the normal control-flow to handle exceptional
situations.

The language support for throwing and catching exceptions (using
\lstinline|try|/\lstinline|catch| and \lstinline|throw|, see
\autoref{sec:lang:except}) work perfectly well with any kind of
object, yet it is a good idea to throw only objects that derive from
\lstinline|Exception|.

\subsection{Prototypes}

\begin{refObjects}
\item[Object]
\item[Traceable]
\end{refObjects}

\subsection{Construction}

There are several types of exceptions, each of which corresponding to
a particular kind of error.  The top-level object,
\lstinline|Exception|, takes a single argument: an error message.

\begin{urbiscript}[firstnumber=1]
Exception.new("something bad has happened!");
[00000001] Exception `something bad has happened!'
Exception.Arity.new("myRoutine", 1, 10, 23);
[00000002] Exception.Arity `myRoutine: expected between 10 and 23 arguments, given 1'
\end{urbiscript}


\subsection{Slots}

Exception has many slots which are specific exceptions.  See
\autoref{sec:specs:except:sub} for their documentation.

\subsubsection{General Features}
\begin{urbiscriptapi}
\item[backtrace] The call stack at the moment the exception was thrown (not
  created), as a \refObject{List} of \refObject[StackFrame]{StackFrames}, from the
  innermost to the outermost call.  Uses \refSlot[Traceable]{backtrace}.
\begin{urbiscript}
//#push 1 "file.u"
try
{
  function innermost () { throw Exception.new("Ouch") };
  function inner     () { innermost() };
  function outer     () { inner() };
  function outermost () { outer() };

  outermost();
}
catch (var e)
{
  assert
  {
    e.backtrace[0].location().asString() == "file.u:4.27-37";
    e.backtrace[0].name == "innermost";

    e.backtrace[1].location().asString() == "file.u:5.27-33";
    e.backtrace[1].name == "inner";

    e.backtrace[2].location().asString() == "file.u:6.27-33";
    e.backtrace[2].name == "outer";

    e.backtrace[3].location().asString() == "file.u:8.3-13";
    e.backtrace[3].name == "outermost";
  };
};
//#pop
\end{urbiscript}


\item[location] The location from which the exception was thrown (not
  created).
\begin{urbiscript}
System.eval("1/0");
[00090441:error] !!! 1.1-3: /: division by 0
[00090441:error] !!!    called from: eval
try
{
  System.eval("1/0");
}
catch (var e)
{
  assert (e.location().asString() == "1.1-3");
};
\end{urbiscript}


\item[message] The error message provided at construction.
\begin{urbiassert}
Exception.new("Ouch").message == "Ouch";
\end{urbiassert}
\end{urbiscriptapi}

\subsubsection{Specific Exceptions}
\label{sec:specs:except:sub}

In the following, since these slots are actually Objects, what is presented
as arguments to the slots are actually arguments to pass to the constructor
of the corresponding exception type.
\begin{urbiscriptapi}
\item[Argument](<routine>, <index>, <exception>)
  During the call of \var{routine}, the instantiation of the \var{index}-nth
  argument has thrown an \var{exception}.
\begin{urbiassert}
Exception.Argument
  .new("myRoutine", 3, Exception.Type.new("19/11/2010", Date))
  .asString()
== "myRoutine: argument 3: unexpected \"19/11/2010\", expected a Date";
\end{urbiassert}


\item[ArgumentType](<routine>, <index>, <effective>, <expected>)%
  Deprecated exception that derives from \refSlot{Type}.  The \var{routine}
  was called with a \var{index}-nth argument of type \var{effective} instead
  of \var{expected}.
\begin{urbiassert}
Exception.ArgumentType
  .new("myRoutine", 1, "hisResult", "Expectation")
  .asString()
== "myRoutine: argument 1: unexpected \"hisResult\", expected a String";
[00000003:warning] !!! `Exception.ArgumentType' is deprecated
\end{urbiassert}


\item[Arity](<routine>, <effective>, <min>, <max> = void)%
  The \var{routine} was called with an incorrect number of arguments
  (\var{effective}).  It requires at least \var{min} arguments, and,
  if specified, at most \var{max}.
\begin{urbiassert}
Exception.Arity.new("myRoutine", 1, 10, 23).asString()
  == "myRoutine: expected between 10 and 23 arguments, given 1";
\end{urbiassert}
%% try
%% {
%%   Math.cos(1, 2);
%% }
%% catch (var e)
%% {
%%   assert(e == Exception.Arity.new("cos", 2, 1));
%% };


\item[BadInteger](<routine>, <fmt>, <effective>)%
  The \var{routine} was called with an inappropriate integer
  (\var{effective}).  Use the format \var{fmt} to create an error
  message from \var{effective}.  Derives from
  \refSlot{BadNumber}.
\begin{urbiassert}
Exception.BadInteger.new("myRoutine", "bad integer: %s", 12).asString()
  == "myRoutine: bad integer: 12";
\end{urbiassert}


\item[BadNumber](<routine>, <fmt>, <effective>)%
  The \var{routine} was called with an inappropriate number
  (\var{effective}).  Use the format \var{fmt} to create an error
  message from \var{effective}.
\begin{urbiassert}
Exception.BadNumber.new("myRoutine", "bad number: %s", 12.34).asString()
  == "myRoutine: bad number: 12.34";
\end{urbiassert}


\item[Constness]%
  An attempt was made to change a constant value.
\begin{urbiassert}
Exception.Constness.new().asString()
  == "cannot modify const slot";
\end{urbiassert}


\item[Duplicate](<fmt>, <name>)%
  A duplication was made. Use the format \var{fmt} to
  create an error message from \var{name}.
\begin{urbiassert}
Exception.Duplicate.new("duplicate dictionary key", "key").asString()
  == "duplicate dictionary key: \"key\"";
\end{urbiassert}


\item[FileNotFound](<name>)%
  The file named \var{name} cannot be found.
\begin{urbiassert}
Exception.FileNotFound.new("foo").asString()
  == "file not found: foo";
\end{urbiassert}


\item[Lookup](<object>, <name>)%
  A failed name lookup was performed on \var{object} to find a slot named
  \var{name}.  Suggest what the user might have meant if
  \lstinline|Exception.Lookup.fixSpelling| is true (which is the default).
\begin{urbiassert}
Exception.Lookup.new(Object, "GetSlot").asString()
  == "lookup failed: Object";
\end{urbiassert}


\item[MatchFailure]%
  A pattern matching failed.
\begin{urbiassert}
Exception.MatchFailure.new().asString()
  == "pattern did not match";
\end{urbiassert}


\item[NegativeNumber](<routine>, <effective>)%
  The \var{routine} was called with a negative number
  (\var{effective}).  Derives from \refSlot{BadNumber}.
\begin{urbiassert}
Exception.NegativeNumber.new("myRoutine", -12).asString()
== "myRoutine: unexpected -12, expected non-negative number";
\end{urbiassert}


\item[NonPositiveNumber](<routine>, <effective>)
  The \var{routine} was called with a non-positive number
  (\var{effective}).  Derives from \refSlot{BadNumber}.
\begin{urbiassert}
Exception.NonPositiveNumber.new("myRoutine", -12).asString()
== "myRoutine: unexpected -12, expected positive number";
\end{urbiassert}


\item[Primitive](<routine>, <msg>)
  The built-in \var{routine} encountered an error described by
  \var{msg}.
\begin{urbiassert}
Exception.Primitive.new("myRoutine", "cannot do that").asString()
  == "myRoutine: cannot do that";
\end{urbiassert}


\item[Redefinition](<name>)
  An attempt was made to refine a slot named \var{name}.
\begin{urbiassert}
Exception.Redefinition.new("foo").asString()
  == "slot redefinition: foo";
\end{urbiassert}


\item[Scheduling](<msg>)
  Something really bad has happened with the \urbi task scheduler.
\begin{urbiassert}
Exception.Scheduling.new("cannot schedule").asString()
  == "cannot schedule";
\end{urbiassert}


\item[Syntax](<loc>, <message>, <input>)%
  Declare a syntax error in \var{input}, at location \var{loc}, described by
  \var{message}.  \var{loc} is the location of the syntax error,
  \var{location} is the place the error was thrown.  They are usually equal,
  except when the errors are caught while using \refSlot[System]{eval} or
  \refSlot[System]{load}.  In that case \var{loc} is really the position of
  the syntax error, while \var{location} refers to the location of the
  \refSlot[System]{eval} or \refSlot[System]{load} invocation.
\begin{urbiassert}
Exception.Syntax
  .new(Location.new(Position.new("file.u", 14, 25)),
       "unexpected pouCharque", "file.u")
  .asString()
== "file.u:14.25: syntax error: unexpected pouCharque";
\end{urbiassert}

\begin{urbiscript}
try
{
  System.eval("1 / / 0");
}
catch (var e)
{
  assert
  {
    e.isA(Exception.Syntax);
    e.loc.asString() == "1.5";
    e.input == "1 / / 0";
    e.message == "unexpected /";
  }
};
\end{urbiscript}


\item[TimeOut]%
  Used internally to implement the \lstinline|timeout| construct
  (\autoref{sec:lang:timeout}).
\begin{urbiassert}
Exception.TimeOut.new().asString()
  == "timeout expired";
\end{urbiassert}


\item[Type](<effective>, <expected>)%
  A value of type \var{effective} was received, while a value of type
  \var{expected} was expected.
\begin{urbiassert}
Exception.Type.new("hisResult", "Expectation").asString()
  == "unexpected \"hisResult\", expected a String";
\end{urbiassert}


\item[UnexpectedVoid]%
  An attempt was made to read the value of \lstinline|void|.
\begin{urbiassert}
Exception.UnexpectedVoid.new().asString()
  == "unexpected void";
\end{urbiassert}
\begin{urbiscript}
var a = void;
a;
[00000016:error] !!! unexpected void
[00000017:error] !!! lookup failed: a
\end{urbiscript}
\end{urbiscriptapi}


%%% Local Variables:
%%% coding: utf-8
%%% mode: latex
%%% TeX-master: "../urbi-sdk"
%%% ispell-dictionary: "american"
%%% ispell-personal-dictionary: "../urbi.dict"
%%% fill-column: 76
%%% End:
