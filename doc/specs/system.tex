\section{System}
Details on the architecture the \urbi server runs on.

\subsection{Prototypes}
\begin{itemize}
\item \refObject{Object}
\end{itemize}

\subsection{Slots}
\begin{itemize}
%% \item \lstinline|breakpoint|
%% \item \lstinline|currentRunner|
%% \item \lstinline|jobs|
%% \item \lstinline|lobbies|
%% \item \lstinline|lobby|
%% \item \lstinline|nonInterruptible|
%% \item \lstinline|programName|
%% \item \lstinline|registerAtJob|
%% \item \lstinline|resetStats|
%% \item \lstinline|searchPath|
%% \item \lstinline|spawn|
%% \item \lstinline|stats|
%% \item \lstinline|stopall|

\item \lstinline|_exit(\var{status})|\\
  Shut the server down brutally: the connections are not closed, and
  the resources are not explicitly released (the operating system
  reclaims most of them: memory, file descriptors and so forth).
  Architecture dependent.

\item \lstinline|aliveJobs|\\
  The number of detached routines currently running.
\begin{urbiscript}[firstnumber=last]
{
  var nJobs = aliveJobs;
  for (var i: [1s, 2s, 3s])
    detach({sleep(i)});
  sleep(0.5s);
  assert(aliveJobs - nJobs == 3);
  sleep(1s);
  assert(aliveJobs - nJobs == 2);
  sleep(1s);
  assert(aliveJobs - nJobs == 1);
  sleep(1s);
  assert(aliveJobs - nJobs == 0);
};
\end{urbiscript}

\item \lstinline|arguments|\\
  The list of the command line arguments passed to the user script.
  This is especially useful in scripts.
\begin{shell}[alsolanguage={[Interactive]Urbi}]
$ cat >echo <<EOF
#! /usr/bin/env urbi
System.arguments;
shutdown;
EOF
$ chmod +x echo
$ ./echo 1 2 3
[00000172] ["1", "2", "3"]
$ ./echo -x 12 -v "foo"
[00000172] ["-x", "12", "-v", "foo"]
\end{shell}

\item \lstinline|assert_(\var{assertion}, \var{message})|\\
  If \var{assertion} does not evaluate to true, throw the failure
  \var{message}.
\begin{urbiscript}[firstnumber=last]
assert_(true,       "true failed");
assert_(42,         "42 failed");
assert_(1 == 1 + 1, "one is not two");
[00000001:error] !!! failed assertion: one is not two
\end{urbiscript}

\item \lstinline|'assert'(\var{assertion})|\\
  Unless \lstinline|System.ndebug| is true, throw an error if
  \var{assertion} is not verified.  See also the assertion support in
  \us, \autoref{sec:assertions}.
\begin{urbiscript}[firstnumber=last]
'assert'(true);
'assert'(42);
'assert'(1 == 1 + 1);
[00000002:error] !!! failed assertion: 1 . '=='(1 . '+'(1))
\end{urbiscript}

\item \lstinline|assert_op(\var{operator}, \var{lhs}, \var{rhs})|\\
  Unless \lstinline|System.ndebug| is true, throw an error if
  \lstinline|\var{lhs} \var{operator} \var{rhs}| is not verified.
\begin{urbiscript}[firstnumber=last]
assert_op("<",  1, 1 + 1);
assert_op("<=", 1, 1 + 1);
assert_op(">",  1, 1 + 1);
[00000003:error] !!! failed assertion: 1 > 1 . '+'(1) (1 <= 2)
\end{urbiscript}

\item \lstinline|backtrace|\experimental\\
  Display the call stack on the channel \code{backtrace}.
\begin{urbiscript}[firstnumber=last]
//#push 100 "foo.u"
function innermost () { backtrace }|;
function inner ()     { innermost }|;
function outer ()     { inner }|;
function outermost () { outer }|;
outermost;
[00000013:backtrace] innermost (foo.u:101.25-33)
[00000014:backtrace] inner (foo.u:102.25-29)
[00000015:backtrace] outer (foo.u:103.25-29)
[00000016:backtrace] outermost (foo.u:104.1-9)
//#pop
\end{urbiscript}

\item \lstinline|cycle|\experimental\\
  The number of execution cycles since the beginning.
\begin{urbiscript}[firstnumber=last]
{
  var first = cycle ; var second = cycle ;
  assert(first + 1 == second);
  first = cycle | second = cycle ;
  assert(first == second);
};
\end{urbiscript}

\item \lstinline|eval(\var{source})|\\
  Evaluate the \us \var{source}, and return its result.  The
  \var{source} must be complete, yet the terminator (e.g., \samp{;})
  is not required.  The evaluation is performed in the context of the
  current object (\lstinline|this|), in particular, to create local
  variables, create scopes.
\begin{urbiassert}[firstnumber=last]
eval("1+2") == 1+2;
eval("\"x\" * 10") == "x" * 10;
eval("eval(\"1\")") ==  1;
eval("{ var x = 1; x + x; }") ==  2;

// This creates a slot in the current object.
eval("var x = 23;") == 23;
x == 23;
\end{urbiassert}

\item \lstinline|getenv(\var{name})|\\
  Return the value of the environment variable \var{name} as a
  \refObject{String} if set, \lstinline|nil| otherwise.  See also
  \lstinline|System.setenv| and \lstinline|System.unsetenv|.
\begin{urbiassert}[firstnumber=last]
getenv("UndefinedEnvironmentVariable").isNil;
!getenv("PATH").isNil;
\end{urbiassert}

\item \lstinline|loadFile(\var{file})|\\
  Load the \us file \var{file}.  Throw a \lstinline|FileNotFound|
  error if the file cannot be found.  Return the last value of the
  file.
\begin{urbiassert}[firstnumber=last]
// Create the file ``123.u'' that contains exactly ``123;''.
System.system("echo '123;' >123.u") == 0;
loadFile("123.u") == 123;
\end{urbiassert}

\item \lstinline|load(\var{file})|\\
  Look for \var{file} in the \urbi path (\autoref{sec:tools:envvars}),
  and load it.  Throw a \lstinline|FileNotFound| error if the file
  cannot be found.  Return the last value of the file.
\begin{urbiassert}[firstnumber=last]
// Create the file ``123.u'' that contains exactly ``123;''.
System.system("echo '123;' >123.u") == 0;
load("123.u") == 123;
\end{urbiassert}

\item \lstinline|loadLibrary(\var{library})|\\
  Load the library \var{library}, to be found in the
  \env{URBI\_UOBJECT\_PATH} search-path (see
  \autoref{sec:tools:envvars}).  The \var{library} may be a
  \refObject{String} or a \refObject{Path}.  The \Cxx symbols are made
  available to the other \Cxx components.  See also
  \lstinline|loadModule|.

\item \lstinline|loadModule(\var{module})|\\
  Load the UObject \var{module}.  Same as \lstinline|loadLibrary|,
  except that the low-level \Cxx symbols are not made ``global'' (in
  the sense of the shared library loader).

\item \lstinline|maybeLoad(\var{file}, \var{channelName} = "")|\\
  Look for \var{file} in the \urbi path
  (\autoref{sec:tools:envvars}).  If the file is found and
  \var{channelName} is non-empty, announce on it that \var{file} is
  about to be loaded, and load it.

\begin{urbiassert}[firstnumber=last]
// Create the file ``123.u'' that contains exactly ``123;''.
System.system("echo '123;' >123.u") == 0;
maybeLoad("123.u") == 123;
maybeLoad("u.123").isVoid;
\end{urbiassert}

\item \lstinline|Platform|\\
  See \refObject{System.Platform}

\item \lstinline|reboot|\\
  Restart the \urbi server.  Architecture dependent.

\item \lstinline|searchFile(\var{file})|\\
  Look for \var{file} in the \urbi path (\autoref{sec:tools:envvars})
  and return its \refObject{Path}.  Throw a \lstinline|FileNotFound|
  error if the file cannot be found.
\begin{urbiassert}[firstnumber=last]
System.system("echo '123;' >123.u") == 0;
searchFile("123.u") == Path.cwd / Path.new("123.u");
\end{urbiassert}

\item \lstinline|setenv(\var{name}, \var{value})|\\
  Set the environment variable \var{name} to
  \lstinline|\var{value}.asString|, and return this value.  See also
  \lstinline|System.getenv| and \lstinline|System.unsetenv|.
  \begin{windows}
    Under Windows, setting to an empty value is equivalent to
    undefining.
  \end{windows}

\begin{urbiassert}[firstnumber=last]
setenv("MyVar", 12) == "12";
getenv("MyVar") == "12";

// A child process that uses the environment variable.
System.system("exit $MyVar") >> 8 ==
       {if (Platform.isWindows) 0 else 12};
setenv("MyVar", 23) == "23";
System.system("exit $MyVar") >> 8 ==
       {if (Platform.isWindows) 0 else 23};

// Defining to empty is not undefining, unless you are on Windows.
setenv("MyVar", "") == "";
getenv("MyVar").isNil == Platform.isWindows;
\end{urbiassert}

\item \lstinline|scopeTag|\\
  Return a fresh Tag whose \lstinline|stop| will be invoked a the end
  of the current scope.  This function is likely to be removed, or
  maybe just moved into \refObject{Tag}.  See
  \autoref{sec:specs:tag:scope}.

\item \lstinline|sleep(\var{duration})|\\
  Suspend the execution for \var{duration} seconds.  No CPU cycle is
  wasted during this wait.

\begin{urbiassert}[firstnumber=last]
(time - {sleep(1s); time}).round == -1;
\end{urbiassert}

\item \lstinline|shiftedTime|\\
  Return the number of seconds elapsed since the \urbi server was
  launched.  Contrary to \lstinline|System.time|, time spent in frozen
  code is not counted.
\begin{urbiassert}[firstnumber=last]
{ var t0 = shiftedTime | sleep(1s) | shiftedTime - t0 }.round ~= 1;

  1 ==
  {
    var t = Tag.new|;
    var t0 = time|;
    var res;
    t: { sleep(1s) | res = shiftedTime - t0 },
    t.freeze;
    sleep(1s);
    t.unfreeze;
    sleep(1s);
    res.round;
  };
\end{urbiassert}

\item \lstinline|shutdown|\\
  Have the \urbi server shut down.  All the connections are closed,
  the resources are released.  Architecture dependent.

\item \lstinline|system(\var{command})|\\
  Ask the operating system to run the \var{command}.  This is
  typically used to start new processes.  The exact syntax of
  \var{command} depends on your system.  On Unix systems, this is
  typically \file{/bin/sh}, while Windows uses \file{command.exe}.

  Return the exit status.

  \begin{windows}
    Under Windows, the exit status is always 0.
  \end{windows}

\begin{urbiassert}[firstnumber=last]
System.system("exit 0") == 0;
System.system("exit 23") >> 8
       == { if (System.Platform.isWindows) 0 else 23 };
\end{urbiassert}


\item \lstinline|time|\\
  Return the number of seconds elapsed since the \urbi server was
  launched.  In presence of a frozen \refObject{Tag}, see also
  \lstinline|System.shiftedTime|.
\begin{urbiassert}[firstnumber=last]
{ var t0 = time | sleep(1s) | time - t0 }.round ~= 1;

  2 ==
  {
    var t = Tag.new|;
    var t0 = time|;
    var res;
    t: { sleep(1s) | res = time - t0 },
    t.freeze;
    sleep(1s);
    t.unfreeze;
    sleep(1s);
    res.round;
  };
\end{urbiassert}

\item \lstinline|unsetenv(\var{name})|\\
  Undefine the environment variable \var{name}, return its previous
  value.  See also \lstinline|System.getenv| and
  \lstinline|System.setenv|.

\begin{urbiassert}[firstnumber=last]
setenv("MyVar", 12) == "12";
!getenv("MyVar").isNil;
unsetenv("MyVar") == "12";
getenv("MyVar").isNil;
\end{urbiassert}


\end{itemize}

%%% Local Variables:
%%% mode: latex
%%% TeX-master: "../urbi-sdk"
%%% ispell-personal-dictionary: "../urbi.dict"
%%% End:
