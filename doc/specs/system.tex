%% Copyright (C) 2009-2010, Gostai S.A.S.
%%
%% This software is provided "as is" without warranty of any kind,
%% either expressed or implied, including but not limited to the
%% implied warranties of fitness for a particular purpose.
%%
%% See the LICENSE file for more information.

\section{System}
Details on the architecture the \urbi server runs on.

\subsection{Prototypes}
\begin{itemize}
\item \refObject{Object}
\end{itemize}

\subsection{Slots}
\begin{urbiscriptapi}
\item[_exit](<status>)%
  Shut the server down brutally: the connections are not closed, and
  the resources are not explicitly released (the operating system
  reclaims most of them: memory, file descriptors and so forth).
  Architecture dependent.


\item[aliveJobs] The number of detached routines currently running.
\begin{urbiscript}
{
  var nJobs = aliveJobs;
  for (var i: [1s, 2s, 3s])
    detach({sleep(i)});
  sleep(0.5s);
  assert(aliveJobs - nJobs == 3);
  sleep(1s);
  assert(aliveJobs - nJobs == 2);
  sleep(1s);
  assert(aliveJobs - nJobs == 1);
  sleep(1s);
  assert(aliveJobs - nJobs == 0);
};
\end{urbiscript}


\item[arguments] The list of the command line arguments passed to the user
  script.  This is especially useful in scripts.
\begin{shell}[alsolanguage={[Interactive]urbiscript}]
$ cat >echo <<EOF
#! /usr/bin/env urbi
System.arguments;
shutdown;
EOF
$ chmod +x echo
$ ./echo 1 2 3
[00000172] ["1", "2", "3"]
$ ./echo -x 12 -v "foo"
[00000172] ["-x", "12", "-v", "foo"]
\end{shell}


\item['assert'](<assertion>)%
  Unless \refSlot{ndebug} is true, throw an error if
  \var{assertion} is not verified.  See also the assertion support in
  \us, \autoref{sec:assertions}.
\begin{urbiscript}
'assert'(true);
'assert'(42);
'assert'(1 == 1 + 1);
[00000002:error] !!! failed assertion: 1.'=='(1.'+'(1))
\end{urbiscript}


\item[assert_](<assertion>, <message>)%
  If \var{assertion} does not evaluate to true, throw the failure
  \var{message}.
\begin{urbiscript}
assert_(true,       "true failed");
assert_(42,         "42 failed");
assert_(1 == 1 + 1, "one is not two");
[00000001:error] !!! failed assertion: one is not two
\end{urbiscript}


\item[assert_op](<operator>, <lhs>, <rhs>)%
  Deprecated, use \lstinline|assert| instead, see \autoref{sec:assertions}.


\item[backtrace]%
  Display the call stack on the channel \code{backtrace}. \experimental
\begin{urbiscript}
//#push 100 "foo.u"
function innermost () { backtrace }|;
function inner ()     { innermost }|;
function outer ()     { inner }|;
function outermost () { outer }|;
outermost;
[00000013:backtrace] innermost (foo.u:101.25-33)
[00000014:backtrace] inner (foo.u:102.25-29)
[00000015:backtrace] outer (foo.u:103.25-29)
[00000016:backtrace] outermost (foo.u:104.1-9)
//#pop
\end{urbiscript}


%% \item[breakpoint]


%% \item[currentRunner]


\item[cycle]%
  The number of execution cycles since the beginning. \experimental
\begin{urbiscript}
{
  var first = cycle ; var second = cycle ;
  assert(first + 1 == second);
  first = cycle | second = cycle ;
  assert(first == second);
};
\end{urbiscript}


\item[eval](<source>, <target> = this)%
  Evaluate the \us \var{source}, and return its result.  See also
  \refSlot{loadFile}.  The \var{source} must be complete, yet the terminator
  (e.g., \samp{;}) is not required.
\begin{urbiassert}
eval("1+2") == 1+2;
eval("\"x\" * 10") == "x" * 10;
eval("eval(\"1\")") == 1;
eval("{ var x = 1; x + x; }") == 2;
\end{urbiassert}

The evaluation is performed in the context of the current object (\this) or
\var{target} if specified.  In particular, to create local variables, create
scopes.
\begin{urbiassert}
// Create a slot in the current object.
eval("var a = 23;") == 23;
this.a == 23;

eval("var a = 3", Global) == 3;
Global.a == 3;
\end{urbiassert}

  Exceptions are thrown on error (including syntax errors).
\begin{urbiscript}
eval("1/0");
[00008316:error] !!! 1.1-3: /: division by 0
[00008316:error] !!!    called from: eval
try
{
  eval("1/0")
}
catch (var e)
{
  assert
  {
    e.isA(Exception.Primitive);
    e.location.asString  == "1.1-3";
    e.routine            == "/";
    e.message            == "division by 0";
  }
};
\end{urbiscript}

  Warnings are reported.

\begin{urbiscript}
eval("new Object");
[00001388:warning] !!! 1.1-10: `new Obj(x)' is deprecated, use `Obj.new(x)'
[00001388:warning] !!!    called from: eval
[00001388] Object_0x1001b2320
\end{urbiscript}

  Nested calls to \refSlot{eval} behave as expected.  The locations in the
  inner calls refer to the position inside the evaluated string.

\begin{urbiscript}
eval("/");
[00001028:error] !!! 1.1: syntax error: unexpected /
[00001028:error] !!!    called from: eval

eval("eval(\"/\")");
[00001032:error] !!! 1.1: syntax error: unexpected /
[00001032:error] !!!    called from: 1.1-9: eval
[00001032:error] !!!    called from: eval

eval("eval(\"eval(\\\"/\\\")\")");
[00001035:error] !!! 1.1: syntax error: unexpected /
[00001035:error] !!!    called from: 1.1-9: eval
[00001035:error] !!!    called from: 1.1-19: eval
[00001035:error] !!!    called from: eval
\end{urbiscript}

\item[getenv](<name>)%
  The value of the environment variable \var{name} as a \refObject{String}
  if set, \refObject{nil} otherwise.  See also \refSlot{setenv} and
  \refSlot{unsetenv}.
\begin{urbiassert}
getenv("UndefinedEnvironmentVariable").isNil;
!getenv("PATH").isNil;
\end{urbiassert}


%% \item[jobs]


\item[load](<file>, <target> = this)%
  Look for \var{file} in the \urbi path (\autoref{sec:tools:envvars}), and
  load it in the context of \var{target}.  See also \refSlot{loadFile}.
  Throw a \refSlot[Exception]{FileNotFound} error if the file cannot be
  found.  Return the last value of the file.
\begin{urbiscript}
// Create the file ``123.u'' that contains exactly ``var t = 123;''.
File.save("123.u", "var t = 123;");
assert
{
  load("123.u") == 123;
  this.t == 123;

  load("123.u", Global) == 123;
  Global.t == 123;
};
\end{urbiscript}


\item[loadFile](<file>, <target> = this)%
  Load the \us file \var{file} in the context of \var{target}.  Behaves like
  \refSlot{eval} applied to the content of \var{file}.  Throw a
  \refSlot[Exception]{FileNotFound} error if the file cannot be found.
  Return the last value of the file.
\begin{urbiscript}
// Create the file ``123.u'' that contains exactly ``var y = 123;''.
File.save("123.u", "var y = 123;");
assert
{
  loadFile("123.u") == 123;
  this.y == 123;

  loadFile("123.u", Global) == 123;
  Global.y == 123;
};

\end{urbiscript}


\item[loadLibrary](<library>)%
  Load the library \var{library}, to be found in
  \refSlot[UObject]{searchPath}.  The \var{library} may be a
  \refObject{String} or a \refObject{Path}.  The \Cxx symbols are made
  available to the other \Cxx components.  See also \refSlot{loadModule}.


\item[loadModule](<module>)%
  Load the \UObject \var{module}.  Same as \refSlot{loadLibrary}, except
  that the low-level \Cxx symbols are not made ``global'' (in the sense of
  the shared library loader).


\item[lobbies] Bounce to \refSlot[Lobby]{instances}.


\item[lobby] Bounce to \refSlot[Lobby]{lobby}.


\item[maybeLoad](<file>, <channel> = Channel.null)%
  Look for \var{file} in the \urbi path (\autoref{sec:tools:envvars}).
  If the file is found announce on \var{Channel} that \var{file} is
  about to be loaded, and load it.

\begin{urbiscript}
// Create the file ``123.u'' that contains exactly ``123;''.
File.save("123.u", "123;");
assert
{
  maybeLoad("123.u") == 123;
  maybeLoad("u.123").isVoid;
};
\end{urbiscript}


\item[ndebug] If true, do not evaluate the assertions.  See
  \autoref{sec:assertions}.
\begin{urbiscript}
function one() { echo("called!"); 1 }|;
assert(!System.ndebug);

assert(one);
[00000617] *** called!

// Beware of copy-on-write.
System.ndebug = true|;
assert(one);

System.ndebug = false|;
assert(one);
[00000622] *** called!
\end{urbiscript}


%% \item[nonInterruptible]


\item[PackageInfo] See \refObject{System.PackageInfo}.


\item[period] The \dfn{period} of the \urbi kernel.  Influences the
  trajectories (\refObject{TrajectoryGenerator}), and the \UObject
  monitoring.  Defaults to 20ms.
\begin{urbiassert}
System.period == 20ms;
\end{urbiassert}


\item[Platform] See \refObject{System.Platform}.


\item[programName] The path under which the \urbi process was called.
  This is typically \file{.../urbi} (\autoref{sec:tools:urbi}) or
  \file{.../urbi-launch} (\autoref{sec:tools:urbi-launch}).
\begin{urbiassert}
Path.new(System.programName).basename
  in ["urbi", "urbi.exe", "urbi-launch", "urbi-launch.exe"];
\end{urbiassert}


\item[reboot] Restart the \urbi server.  Architecture dependent.


\item[redefinitionMode] Switch the current job in redefinition mode
  until the end of the current scope.  While in redefinition mode,
  setSlot on already existing slots will overwrite the slot instead of
  erring.

\begin{urbiscript}
var Global.x = 0;
[00000001] 0
{
  System.redefinitionMode;
  // Not an error
  var Global.x = 1;
  echo(Global.x);
};
[00000002] *** 1
// redefinitionMode applies only to the scope.
var Global.x = 0;
[00000003:error] !!! slot redefinition: x
\end{urbiscript}


\item[requireFile](<file>)%
  Load \var{file} if it was not loaded before (with \refSlot{load} or
  \refSlot{requireFile}). Unlike \refSlot{load}, \lstinline{requireFile}
  always returns \lstinline|void|.

\begin{urbiscript}
// Create the file ``test.u'' that echoes a string.
File.save("test1.u", "echo(\"test 1\"); 1;");
requireFile("test1.u");
[00000001] *** test 1
requireFile("test1.u");
// File is not re-loaded

File.save("test2.u", "echo(\"test 2\"); 2;");
load("test2.u");
[00000004] *** test 2
[00000004] 2
requireFile("test2.u");
load("test2.u");
[00000006] *** test 2
[00000006] 2
\end{urbiscript}

%% \item[resetStats]


\item[scopeTag] Bounce to \refSlot[Tag]{scope}.


\item[searchFile](<file>)%
  Look for \var{file} in the \refSlot{searchPath} and return its
  \refObject{Path}.  Throw a \refSlot[Exception]{FileNotFound} error if the
  file cannot be found.
\begin{urbiscript}
// Create the file ``123.u'' that contains exactly ``123;''.
File.save("123.u", "123;");
assert
{
  searchFile("123.u") == Path.cwd / Path.new("123.u");
};
\end{urbiscript}


\item[searchPath] The \urbi path (i.e., the directories where the \us files
  are looked for, see \autoref{sec:tools:envvars}) as a \refObject{List} of
  \refObject[s]{Path}.
\begin{urbiassert}
System.searchPath.isA(List);
System.searchPath[0].isA(Path);
\end{urbiassert}


\item[setenv](<name>, <value>)%
  Set the environment variable \var{name} to
  \lstinline|\var{value}.asString|, and return this value.  See also
  \refSlot{getenv} and \refSlot{unsetenv}.
  \begin{windows}
    Under Windows, setting to an empty value is equivalent to
    making undefined.
  \end{windows}

\begin{urbiassert}
setenv("MyVar", 12) == "12";
getenv("MyVar") == "12";

// A child process that uses the environment variable.
System.system("exit $MyVar") >> 8 ==
       {if (Platform.isWindows) 0 else 12};
setenv("MyVar", 23) == "23";
System.system("exit $MyVar") >> 8 ==
       {if (Platform.isWindows) 0 else 23};

// Defining to empty is defining, unless you are on Windows.
setenv("MyVar", "") == "";
getenv("MyVar").isNil == Platform.isWindows;
\end{urbiassert}


\item[shiftedTime] Return the number of seconds elapsed since the
  \urbi server was launched.  Contrary to \refSlot{time},
  time spent in frozen code is not counted.
\begin{urbiassert}
{ var t0 = shiftedTime | sleep(1s) | shiftedTime - t0 }.round ~= 1;

  1 ==
  {
    var t = Tag.new|;
    var t0 = time|;
    var res;
    t: { sleep(1s) | res = shiftedTime - t0 },
    t.freeze;
    sleep(1s);
    t.unfreeze;
    sleep(1s);
    res.round;
  };
\end{urbiassert}


\item[shutdown] Have the \urbi server shut down.  All the connections
  are closed, the resources are released.  Architecture dependent.


\item[sleep](<duration>)%
  Suspend the execution for \var{duration} seconds.  No CPU cycle is
  wasted during this wait.

\begin{urbiassert}
(time - {sleep(1s); time}).round == -1;
\end{urbiassert}


\item[spawn](<function>, <clear>)%
  Run the \var{function}, with fresh tags if \var{clear} is true, otherwise
  under the control of the current tags.  Return the spawn \refObject{Job}.
\begin{urbiscript}
var jobs = []|;
var res = []|;
for (var i : [0, 1, 2])
{
  jobs << System.spawn(closure () { res << i; res << i },
                       true) |
  if (i == 2)
    break
}|
jobs;
[00009120] [Job<shell_11>, Job<shell_12>, Job<shell_13>]
// Wait for the jobs to be done.
jobs.each (function (var j) { j.waitForTermination });
assert (res == [0, 1, 0, 2, 1, 2]);
\end{urbiscript}

\begin{urbiscript}
jobs = []|;
res = []|;
for (var i : [0, 1, 2])
{
  jobs << System.spawn(closure () { res << i; res << i },
                       false) |
  if (i == 2)
    break
}|
jobs;
[00009120] [Job<shell_14>, Job<shell_15>, Job<shell_16>]
// Give some time to get the output of the detached expressions.
sleep(100ms);
assert (res == [0, 1, 0]);
\end{urbiscript}

%% \item[stats]


%% \item[stopall]


\item[system](<command>)%
  Ask the operating system to run the \var{command}.  This is
  typically used to start new processes.  The exact syntax of
  \var{command} depends on your system.  On Unix systems, this is
  typically \file{/bin/sh}, while Windows uses \file{command.exe}.

  Return the exit status.

  \begin{windows}
    Under Windows, the exit status is always 0.
  \end{windows}

\begin{urbiassert}
System.system("exit 0") == 0;
System.system("exit 23") >> 8
       == { if (System.Platform.isWindows) 0 else 23 };
\end{urbiassert}


\item[time] Return the number of seconds elapsed since the \urbi
  server was launched.  In presence of a frozen \refObject{Tag}, see
  also \refSlot{shiftedTime}.
\begin{urbiassert}
{ var t0 = time | sleep(1s) | time - t0 }.round ~= 1;

  2 ==
  {
    var t = Tag.new|;
    var t0 = time|;
    var res;
    t: { sleep(1s) | res = time - t0 },
    t.freeze;
    sleep(1s);
    t.unfreeze;
    sleep(1s);
    res.round;
  };
\end{urbiassert}


\item[unsetenv](<name>)%
  Undefine the environment variable \var{name}, return its previous value.
  See also \refSlot{getenv} and \refSlot{setenv}.

\begin{urbiassert}
setenv("MyVar", 12) == "12";
!getenv("MyVar").isNil;
unsetenv("MyVar") == "12";
getenv("MyVar").isNil;
\end{urbiassert}


\item[version]%
  The version of \usdk.  A string composed of two or more numbers separated
  by periods: \samp{"\packageVersion"}.
\begin{urbiassert}
System.version in Regexp.new("\\d+(\\.\\d+)+");
\end{urbiassert}
\end{urbiscriptapi}

%%% Local Variables:
%%% mode: latex
%%% TeX-master: "../urbi-sdk"
%%% ispell-dictionary: "american"
%%% ispell-personal-dictionary: "../urbi.dict"
%%% fill-column: 76
%%% End:
