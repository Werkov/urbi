%% Copyright (C) 2009-2010, Gostai S.A.S.
%%
%% This software is provided "as is" without warranty of any kind,
%% either expressed or implied, including but not limited to the
%% implied warranties of fitness for a particular purpose.
%%
%% See the LICENSE file for more information.

\section{System}
Details on the architecture the \urbi server runs on.

\subsection{Prototypes}
\begin{itemize}
\item \refObject{Object}
\end{itemize}

\subsection{Slots}
\begin{urbiscriptapi}
%% \item[breakpoint]
%% \item[currentRunner]
%% \item[jobs]
%% \item[lobbies]
%% \item[lobby]
%% \item[nonInterruptible]
%% \item[registerAtJob]
%% \item[resetStats]
%% \item[searchPath]
%% \item[spawn]
%% \item[stats]
%% \item[stopall]
\item[_exit](<status>)%
  Shut the server down brutally: the connections are not closed, and
  the resources are not explicitly released (the operating system
  reclaims most of them: memory, file descriptors and so forth).
  Architecture dependent.

\item[aliveJobs] The number of detached routines currently running.
\begin{urbiscript}
{
  var nJobs = aliveJobs;
  for (var i: [1s, 2s, 3s])
    detach({sleep(i)});
  sleep(0.5s);
  assert(aliveJobs - nJobs == 3);
  sleep(1s);
  assert(aliveJobs - nJobs == 2);
  sleep(1s);
  assert(aliveJobs - nJobs == 1);
  sleep(1s);
  assert(aliveJobs - nJobs == 0);
};
\end{urbiscript}

\item[arguments] The list of the command line arguments passed to the user script.
  This is especially useful in scripts.
\begin{shell}[alsolanguage={[Interactive]Urbi}]
$ cat >echo <<EOF
#! /usr/bin/env urbi
System.arguments;
shutdown;
EOF
$ chmod +x echo
$ ./echo 1 2 3
[00000172] ["1", "2", "3"]
$ ./echo -x 12 -v "foo"
[00000172] ["-x", "12", "-v", "foo"]
\end{shell}

\item[assert_](<assertion>, <message>)%
  If \var{assertion} does not evaluate to true, throw the failure
  \var{message}.
\begin{urbiscript}
assert_(true,       "true failed");
assert_(42,         "42 failed");
assert_(1 == 1 + 1, "one is not two");
[00000001:error] !!! failed assertion: one is not two
\end{urbiscript}

\item['assert'](<assertion>)%
  Unless \refSlot{ndebug} is true, throw an error if
  \var{assertion} is not verified.  See also the assertion support in
  \us, \autoref{sec:assertions}.
\begin{urbiscript}
'assert'(true);
'assert'(42);
'assert'(1 == 1 + 1);
[00000002:error] !!! failed assertion: 1.'=='(1.'+'(1))
\end{urbiscript}

\item[assert_op](<operator>, <lhs>, <rhs>)%
  Deprecated, use \lstinline|assert| instead, see \autoref{sec:assertions}.

\item[backtrace]%
  Display the call stack on the channel \code{backtrace}. \experimental
\begin{urbiscript}
//#push 100 "foo.u"
function innermost () { backtrace }|;
function inner ()     { innermost }|;
function outer ()     { inner }|;
function outermost () { outer }|;
outermost;
[00000013:backtrace] innermost (foo.u:101.25-33)
[00000014:backtrace] inner (foo.u:102.25-29)
[00000015:backtrace] outer (foo.u:103.25-29)
[00000016:backtrace] outermost (foo.u:104.1-9)
//#pop
\end{urbiscript}

\item[cycle]%
  The number of execution cycles since the beginning. \experimental
\begin{urbiscript}
{
  var first = cycle ; var second = cycle ;
  assert(first + 1 == second);
  first = cycle | second = cycle ;
  assert(first == second);
};
\end{urbiscript}

\item[eval](<source>)%
  Evaluate the \us \var{source}, and return its result.  The
  \var{source} must be complete, yet the terminator (e.g., \samp{;})
  is not required.
\begin{urbiassert}
eval("1+2") == 1+2;
eval("\"x\" * 10") == "x" * 10;
eval("eval(\"1\")") ==  1;
eval("{ var x = 1; x + x; }") ==  2;
\end{urbiassert}

  The evaluation is performed in the context of the current object
  (\this), in particular, to create local variables, create
  scopes.
\begin{urbiassert}
// This creates a slot in the current object.
eval("var x = 23;") == 23;
x == 23;
\end{urbiassert}

  Exceptions are thrown on error.
\begin{urbiscript}
eval("1/0");
[00008316:error] !!! 1.1-3: /: division by 0
try
{
  eval ("1/0")
}
catch (var e)
{
  assert
  {
    e.isA(Exception.Primitive);
    e.location.asString  == "1.1-3";
    e.routine            == "/";
    e.message            == "division by 0";
  }
};
\end{urbiscript}

\item[getenv](<name>)%
  Return the value of the environment variable \var{name} as a
  \refObject{String} if set, \lstinline|nil| otherwise.  See also
  \refSlot{setenv} and \refSlot{unsetenv}.
\begin{urbiassert}
getenv("UndefinedEnvironmentVariable").isNil;
!getenv("PATH").isNil;
\end{urbiassert}

\item[loadFile](<file>)%
  Load the \us file \var{file}.  Throw a \lstinline|FileNotFound|
  error if the file cannot be found.  Return the last value of the
  file.
\begin{urbiassert}
// Create the file ``123.u'' that contains exactly ``123;''.
System.system("echo '123;' >123.u") == 0;
loadFile("123.u") == 123;
\end{urbiassert}

\item[load](<file>)%
  Look for \var{file} in the \urbi path (\autoref{sec:tools:envvars}),
  and load it.  Throw a \lstinline|FileNotFound| error if the file
  cannot be found.  Return the last value of the file.
\begin{urbiassert}
// Create the file ``123.u'' that contains exactly ``123;''.
System.system("echo '123;' >123.u") == 0;
load("123.u") == 123;
\end{urbiassert}

\item[loadLibrary](<library>)%
  Load the library \var{library}, to be found in the
  \env{URBI\_UOBJECT\_PATH} search-path (see
  \autoref{sec:tools:envvars}), or the default UObject path.
  The \var{library} may be a \refObject{String} or a \refObject{Path}.
  The \Cxx symbols are made available to the other \Cxx components.  See also
  \lstinline|loadModule|.

\item[loadModule](<module>)%
  Load the \UObject \var{module}.  Same as \lstinline|loadLibrary|,
  except that the low-level \Cxx symbols are not made ``global'' (in
  the sense of the shared library loader).

\item[maybeLoad](<file>, <channel> = Channel.null)%
  Look for \var{file} in the \urbi path (\autoref{sec:tools:envvars}).
  If the file is found announce on \var{Channel} that \var{file} is
  about to be loaded, and load it.

\begin{urbiassert}
// Create the file ``123.u'' that contains exactly ``123;''.
System.system("echo '123;' >123.u") == 0;
maybeLoad("123.u") == 123;
maybeLoad("u.123").isVoid;
\end{urbiassert}

\item[period] The \dfn{period} of the \urbi kernel.  Influences the
  trajectories (\refObject{TrajectoryGenerator}), and the \UObject
  monitoring.  Defaults to 20ms.
\begin{urbiassert}
System.period == 20ms;
\end{urbiassert}

\item[Platform] See \refObject{System.Platform}

\item[programName] The path under which the \urbi process was called.
  This is typically \file{.../urbi} (\autoref{sec:tools:urbi}) or
  \file{.../urbi-launch} (\autoref{sec:tools:urbi-launch}).
\begin{urbiassert}
Path.new(System.programName).basename
  in ["urbi", "urbi.exe", "urbi-launch", "urbi-launch.exe"];
\end{urbiassert}

\item[reboot] Restart the \urbi server.  Architecture dependent.

\item[searchFile](<file>)%
  Look for \var{file} in the \urbi path (\autoref{sec:tools:envvars})
  and return its \refObject{Path}.  Throw a \lstinline|FileNotFound|
  error if the file cannot be found.
\begin{urbiassert}
System.system("echo '123;' >123.u") == 0;
searchFile("123.u") == Path.cwd / Path.new("123.u");
\end{urbiassert}

\item[setenv](<name>, <value>)%
  Set the environment variable \var{name} to
  \lstinline|\var{value}.asString|, and return this value.  See also
  \refSlot{getenv} and \refSlot{unsetenv}.
  \begin{windows}
    Under Windows, setting to an empty value is equivalent to
    making undefined.
  \end{windows}

\begin{urbiassert}
setenv("MyVar", 12) == "12";
getenv("MyVar") == "12";

// A child process that uses the environment variable.
System.system("exit $MyVar") >> 8 ==
       {if (Platform.isWindows) 0 else 12};
setenv("MyVar", 23) == "23";
System.system("exit $MyVar") >> 8 ==
       {if (Platform.isWindows) 0 else 23};

// Defining to empty is defining, unless you are on Windows.
setenv("MyVar", "") == "";
getenv("MyVar").isNil == Platform.isWindows;
\end{urbiassert}

\item[redefinitionMode] Switch the current job in redefinition mode
  until the end of the current scope.  While in redefinition mode,
  setSlot on already existing slots will overwrite the slot instead of
  erring.

\begin{urbiscript}
var Global.x = 0;
[00000001] 0
{
  System.redefinitionMode;
  // Not an error
  var Global.x = 1;
  echo(Global.x);
};
[00000002] *** 1
// redefinitionMode applies only to the scope.
var Global.x = 0;
[00000003:error] !!! slot redefinition: x
\end{urbiscript}

\item[scopeTag] Return a fresh Tag whose \lstinline|stop| will be
  invoked a the end of the current scope.  This function is likely to
  be removed, or maybe just moved into \refObject{Tag}.  See
  \autoref{sec:specs:tag:scope}.

\item[sleep](<duration>)%
  Suspend the execution for \var{duration} seconds.  No CPU cycle is
  wasted during this wait.

\begin{urbiassert}
(time - {sleep(1s); time}).round == -1;
\end{urbiassert}

\item[shiftedTime] Return the number of seconds elapsed since the
  \urbi server was launched.  Contrary to \refSlot{time},
  time spent in frozen code is not counted.
\begin{urbiassert}
{ var t0 = shiftedTime | sleep(1s) | shiftedTime - t0 }.round ~= 1;

  1 ==
  {
    var t = Tag.new|;
    var t0 = time|;
    var res;
    t: { sleep(1s) | res = shiftedTime - t0 },
    t.freeze;
    sleep(1s);
    t.unfreeze;
    sleep(1s);
    res.round;
  };
\end{urbiassert}

\item[shutdown] Have the \urbi server shut down.  All the connections
  are closed, the resources are released.  Architecture dependent.

\item[system](<command>)%
  Ask the operating system to run the \var{command}.  This is
  typically used to start new processes.  The exact syntax of
  \var{command} depends on your system.  On Unix systems, this is
  typically \file{/bin/sh}, while Windows uses \file{command.exe}.

  Return the exit status.

  \begin{windows}
    Under Windows, the exit status is always 0.
  \end{windows}

\begin{urbiassert}
System.system("exit 0") == 0;
System.system("exit 23") >> 8
       == { if (System.Platform.isWindows) 0 else 23 };
\end{urbiassert}


\item[time] Return the number of seconds elapsed since the \urbi
  server was launched.  In presence of a frozen \refObject{Tag}, see
  also \refSlot{shiftedTime}.
\begin{urbiassert}
{ var t0 = time | sleep(1s) | time - t0 }.round ~= 1;

  2 ==
  {
    var t = Tag.new|;
    var t0 = time|;
    var res;
    t: { sleep(1s) | res = time - t0 },
    t.freeze;
    sleep(1s);
    t.unfreeze;
    sleep(1s);
    res.round;
  };
\end{urbiassert}

\item[unsetenv](<name>)%
  Undefine the environment variable \var{name}, return its previous value.
  See also \refSlot{getenv} and \refSlot{setenv}.

\begin{urbiassert}
setenv("MyVar", 12) == "12";
!getenv("MyVar").isNil;
unsetenv("MyVar") == "12";
getenv("MyVar").isNil;
\end{urbiassert}


\end{urbiscriptapi}

%%% Local Variables:
%%% mode: latex
%%% TeX-master: "../urbi-sdk"
%%% ispell-dictionary: "american"
%%% ispell-personal-dictionary: "../urbi.dict"
%%% fill-column: 76
%%% End:
