%% Copyright (C) 2009-2011, Gostai S.A.S.
%%
%% This software is provided "as is" without warranty of any kind,
%% either expressed or implied, including but not limited to the
%% implied warranties of fitness for a particular purpose.
%%
%% See the LICENSE file for more information.

\section{System}
Details on the architecture the \urbi server runs on.

\subsection{Prototypes}
\begin{refObjects}
\item[Object]
\end{refObjects}

\subsection{Slots}
\begin{urbiscriptapi}
\item[_exit](<status>)%
  Shut the server down brutally: the connections are not closed, and
  the resources are not explicitly released (the operating system
  reclaims most of them: memory, file descriptors and so forth).
  Architecture dependent.


\item[arguments] The list of the command line arguments passed to the user
  script.  This is especially useful in scripts.
\begin{shell}[alsolanguage={[Interactive]urbiscript}]
$ cat >echo <<EOF
#! /usr/bin/env urbi
System.arguments;
shutdown;
EOF
$ chmod +x echo
$ ./echo 1 2 3
[00000172] ["1", "2", "3"]
$ ./echo -x 12 -v "foo"
[00000172] ["-x", "12", "-v", "foo"]
\end{shell}


\item['assert'](<assertion>)%
  Unless \refSlot{ndebug} is true, throw an error if
  \var{assertion} is not verified.  See also the assertion support in
  \us, \autoref{sec:lang:assert}.
\begin{urbiscript}
'assert'(true);
'assert'(42);
'assert'(1 == 1 + 1);
[00000002:error] !!! failed assertion: 1.'=='(1.'+'(1))
\end{urbiscript}


\item[assert_](<assertion>, <message>)%
  If \var{assertion} does not evaluate to true, throw the failure
  \var{message}.
\begin{urbiscript}
assert_(true,       "true failed");
assert_(42,         "42 failed");
assert_(1 == 1 + 1, "one is not two");
[00000001:error] !!! failed assertion: one is not two
\end{urbiscript}


\item[assert_op](<operator>, <lhs>, <rhs>)%
  Deprecated, use \lstinline|assert| instead, see \autoref{sec:lang:assert}.


%% \item[breakpoint]


\item[currentRunner]  An obsolete alias for \refSlot[Job]{current}.

\item[cycle]%
  The number of execution cycles since the beginning. \experimental
\begin{urbiscript}
{
  var first = cycle ; var second = cycle ;
  assert(first + 1 == second);
  first = cycle | second = cycle ;
  assert(first == second);
};
\end{urbiscript}


\item[eval](<source>, <target> = this)%
  Evaluate the \us \var{source}, and return its result.  See also
  \refSlot{loadFile}.  The \var{source} must be complete, yet the terminator
  (e.g., \samp{;}) is not required.
\begin{urbiassert}
eval("1+2") == 1+2;
eval("\"x\" * 10") == "x" * 10;
eval("eval(\"1\")") == 1;
eval("{ var x = 1; x + x; }") == 2;
\end{urbiassert}

The evaluation is performed in the context of the current object (\this) or
\var{target} if specified.  In particular, to create local variables, create
scopes.
\begin{urbiassert}
// Create a slot in the current object.
eval("var a = 23;") == 23;
this.a == 23;

eval("var a = 3", Global) == 3;
Global.a == 3;
\end{urbiassert}

  Exceptions are thrown on error (including syntax errors).
\begin{urbiscript}
eval("1/0");
[00008316:error] !!! 1.1-3: /: division by 0
[00008316:error] !!!    called from: eval
try
{
  eval("1/0")
}
catch (var e)
{
  assert
  {
    e.isA(Exception.Primitive);
    e.location.asString  == "1.1-3";
    e.routine            == "/";
    e.message            == "division by 0";
  }
};
\end{urbiscript}

  Warnings are reported.

\begin{urbiscript}
eval("new Object");
[00001388:warning] !!! 1.1-10: `new Obj(x)' is deprecated, use `Obj.new(x)'
[00001388:warning] !!!    called from: eval
[00001388] Object_0x1001b2320
\end{urbiscript}

  Nested calls to \refSlot{eval} behave as expected.  The locations in the
  inner calls refer to the position inside the evaluated string.

\begin{urbiscript}
eval("/");
[00001028:error] !!! 1.1: syntax error: unexpected /
[00001028:error] !!!    called from: eval

eval("eval(\"/\")");
[00001032:error] !!! 1.1: syntax error: unexpected /
[00001032:error] !!!    called from: 1.1-9: eval
[00001032:error] !!!    called from: eval

eval("eval(\"eval(\\\"/\\\")\")");
[00001035:error] !!! 1.1: syntax error: unexpected /
[00001035:error] !!!    called from: 1.1-9: eval
[00001035:error] !!!    called from: 1.1-19: eval
[00001035:error] !!!    called from: eval
\end{urbiscript}

\item[getenv](<name>)%
  The value of the environment variable \var{name} as a \refObject{String}
  if set, \refObject{nil} otherwise.  See also \refSlot{setenv} and
  \refSlot{unsetenv}.
\begin{urbiassert}
getenv("UndefinedEnvironmentVariable").isNil;
!getenv("PATH").isNil;
\end{urbiassert}


\item[getLocale](<category>)%
  Return a \refObject{String} denoting the locale set for \var{category}, or
  raise an error.  See \refSlot{setLocale} for more details.
\begin{urbiscript}
getLocale("LC_IMAGINARY");
[00006328:error] !!! getLocale: invalid category: LC_IMAGINARY
\end{urbiscript}


\item[load](<file>, <target> = this)%
  Look for \var{file} in the \urbi path (\autoref{sec:tools:envvars}), and
  load it in the context of \var{target}.  See also \refSlot{loadFile}.
  Throw a \refSlot[Exception]{FileNotFound} error if the file cannot be
  found.  Return the last value of the file.
\begin{urbiscript}
// Create the file ``123.u'' that contains exactly ``var t = 123;''.
File.save("123.u", "var t = 123;");
assert
{
  load("123.u") == 123;
  this.t == 123;

  load("123.u", Global) == 123;
  Global.t == 123;
};
\end{urbiscript}


\item[loadFile](<file>, <target> = this)%
  Load the \us file \var{file} in the context of \var{target}.  Behaves like
  \refSlot{eval} applied to the content of \var{file}.  Throw a
  \refSlot[Exception]{FileNotFound} error if the file cannot be found.
  Return the last value of the file.
\begin{urbiscript}
// Create the file ``123.u'' that contains exactly ``var y = 123;''.
File.save("123.u", "var y = 123;");
assert
{
  loadFile("123.u") == 123;
  this.y == 123;

  loadFile("123.u", Global) == 123;
  Global.y == 123;
};

\end{urbiscript}


\item[loadLibrary](<library>)%
  Load the library \var{library}, to be found in
  \refSlot[UObject]{searchPath}.  The \var{library} may be a
  \refObject{String} or a \refObject{Path}.  The \Cxx symbols are made
  available to the other \Cxx components.  See also \refSlot{loadModule}.


\item[loadModule](<module>)%
  Load the \UObject \var{module}.  Same as \refSlot{loadLibrary}, except
  that the low-level \Cxx symbols are not made ``global'' (in the sense of
  the shared library loader).


\item[lobbies] Bounce to \refSlot[Lobby]{instances}.


\item[lobby] Bounce to \refSlot[Lobby]{lobby}.


\item[maybeLoad](<file>, <channel> = Channel.null)%
  Look for \var{file} in the \urbi path (\autoref{sec:tools:envvars}).
  If the file is found announce on \var{Channel} that \var{file} is
  about to be loaded, and load it.

\begin{urbiscript}
// Create the file ``123.u'' that contains exactly ``123;''.
File.save("123.u", "123;");
assert
{
  maybeLoad("123.u") == 123;
  maybeLoad("u.123").isVoid;
};
\end{urbiscript}


\item[ndebug] If true, do not evaluate the assertions.  See
  \autoref{sec:lang:assert}.
\begin{urbiscript}
function one() { echo("called!"); 1 }|;
assert(!System.ndebug);

assert(one);
[00000617] *** called!

// Beware of copy-on-write.
System.ndebug = true|;
assert(one);

System.ndebug = false|;
assert(one);
[00000622] *** called!
\end{urbiscript}


%% \item[nonInterruptible]


\item[PackageInfo] See \refObject{System.PackageInfo}.


\item[period] The \dfn{period} of the \urbi kernel.  Influences the
  trajectories (\refObject{TrajectoryGenerator}), and the \UObject
  monitoring.  Defaults to 20ms.
\begin{urbiassert}
System.period == 20ms;
\end{urbiassert}


\item[Platform] See \refObject{System.Platform}.

\item[profile](<function>)%
  Compute some measures during the execution of \var{function} and return
  the results as a \refObject{Profile} object. A \refObject{Profile} details
  information about time, function calls and scheduling.


\item[programName] The path under which the \urbi process was called.
  This is typically \file{.../urbi} (\autoref{sec:tools:urbi}) or
  \file{.../urbi-launch} (\autoref{sec:tools:urbi-launch}).
\begin{urbiassert}
Path.new(System.programName).basename
  in ["urbi", "urbi.exe", "urbi-launch", "urbi-launch.exe"];
\end{urbiassert}


\item[reboot] Restart the \urbi server.  Architecture dependent.


\item[redefinitionMode] Switch the current job in redefinition mode
  until the end of the current scope.  While in redefinition mode,
  setSlot on already existing slots will overwrite the slot instead of
  erring.

\begin{urbiscript}
var Global.x = 0;
[00000001] 0
{
  System.redefinitionMode;
  // Not an error
  var Global.x = 1;
  echo(Global.x);
};
[00000002] *** 1
// redefinitionMode applies only to the scope.
var Global.x = 0;
[00000003:error] !!! slot redefinition: x
\end{urbiscript}


\item[requireFile](<file>, <target>)%
  Load \var{file} in the context of \var{target} if it was not loaded before
  (with \refSlot{load} or \refSlot{requireFile}). Unlike \refSlot{load},
  \lstinline{requireFile} always returns \lstinline|void|. If \var{file} is
  being loaded concurrently \lstinline{requireFile} waits until the loading
  is finished.

\begin{urbiscript}
// Create the file ``test.u'' that echoes a string.
File.save("test1.u", "echo(\"test 1\"); 1;");
requireFile("test1.u");
[00000001] *** test 1
requireFile("test1.u");
// File is not re-loaded

File.save("test2.u", "echo(\"test 2\"); 2;");
load("test2.u");
[00000004] *** test 2
[00000004] 2
requireFile("test2.u");
load("test2.u");
[00000006] *** test 2
[00000006] 2
\end{urbiscript}

  The \var{target} is not taken into account to check whether the file has
  already been loaded: if you require twice the same file with two different
  targets, it will be loaded only for the first.

\begin{urbiscript}
requireFile("test2.u", Global);
\end{urbiscript}

\item[resetStats]%
  Reinitialize the \refSlot{stats} computation.
\begin{urbiassert}
 0  < System.stats["cycles"];
System.resetStats.isVoid;
 1 == System.stats["cycles"];
\end{urbiassert}


\item[scopeTag] Bounce to \refSlot[Tag]{scope}.


\item[searchFile](<file>)%
  Look for \var{file} in the \refSlot{searchPath} and return its
  \refObject{Path}.  Throw a \refSlot[Exception]{FileNotFound} error if the
  file cannot be found.
\begin{urbiscript}
// Create the file ``123.u'' that contains exactly ``123;''.
File.save("123.u", "123;");
assert
{
  searchFile("123.u") == Path.cwd / Path.new("123.u");
};
\end{urbiscript}


\item[searchPath] The \urbi path (i.e., the directories where the \us files
  are looked for, see \autoref{sec:tools:envvars}) as a \refObject{List} of
  \refObject[s]{Path}.
\begin{urbiassert}
System.searchPath.isA(List);
System.searchPath[0].isA(Path);
\end{urbiassert}


\item[setenv](<name>, <value>)%
  Set the environment variable \var{name} to
  \lstinline|\var{value}.asString|, and return this value.  See also
  \refSlot{getenv} and \refSlot{unsetenv}.
  \begin{windows}
    Under Windows, setting to an empty value is equivalent to
    making undefined.
  \end{windows}

\begin{urbiassert}
setenv("MyVar", 12) == "12";
getenv("MyVar") == "12";

// A child process that uses the environment variable.
System.system("exit $MyVar") >> 8 ==
       {if (Platform.isWindows) 0 else 12};
setenv("MyVar", 23) == "23";
System.system("exit $MyVar") >> 8 ==
       {if (Platform.isWindows) 0 else 23};

// Defining to empty is defining, unless you are on Windows.
setenv("MyVar", "") == "";
getenv("MyVar").isNil == Platform.isWindows;
\end{urbiassert}


\item[setLocale](<category>, <locale> = "")%
  Change the system's \dfn{locale} for the \var{category} to \var{locale}
  and return void.  If \var{locale} is empty, then use the locale specified
  by the user's environment (e.g., the environment variables).  The
  \var{category} can be:
  \begin{sublist}
    \begin{envs}
    \item[LC\_ALL] Overrides all the following categories.
    \item[LC\_COLLATE] Controls how string sorting is performed.
    \item[LC\_CTYPE] Change what characters are considered as letters and so
      on.
    \item[LC\_MESSAGES] The catalog of translated messages.  This category
      is not supported by Microsoft Windows.
    \item[LC\_MONETARY] How to format monetary values.
    \item[LC\_NUMERIC] Set a locale for formatting numbers.
    \item[LC\_TIME] Set a locale for formatting dates and times.
    \end{envs}
  \end{sublist}
  With \command{urbi} is run, it does \emph{not} change its locales: it
  defaults to the ``good old C mode'', which corresponds to the \samp{C} (or
  \samp{POSIX}) locale.  See also \refSlot{getLocale}.
\begin{urbicomment}
// Be sure not to depend on the user environment.
for (var i : ["ALL", "COLLATE", "CTYPE", "MESSAGES", "MONETARY",
              "NUMERIC", "TIME"])
  unsetenv("LC_" + i);
\end{urbicomment}
\begin{urbiassert}
// Initially they are all set to "C".
getLocale("LC_ALL")     == "C";
getLocale("LC_CTYPE")   == "C";
getLocale("LC_NUMERIC") == "C";

// Windows does not under the "fr_FR" locale, it supports "French"
// which actually denotes "French_France.1252".
var fr_FR =
  { if (System.Platform.isWindows) "French_France.1252" else "fr_FR" };
// Changing one via the environment does not affect the others.
setenv("LC_CTYPE", fr_FR) == fr_FR;
getLocale("LC_CTYPE")   == "C";
setLocale("LC_CTYPE").isVoid;
getLocale("LC_CTYPE")   == fr_FR;
getLocale("LC_NUMERIC") == "C";

// Changing one via setLocale does not change the others either.
setLocale("LC_CTYPE", "C").isVoid;
getLocale("LC_CTYPE")   == "C";
getLocale("LC_NUMERIC") == "C";

// The environment variable LC_ALL overrides all the others.
setenv("LC_ALL", fr_FR);
setLocale("LC_ALL").isVoid;
getLocale("LC_ALL")     == fr_FR;
getLocale("LC_CTYPE")   == fr_FR;
getLocale("LC_NUMERIC") == fr_FR;

// Explicit changes of LC_ALL overrides all the others.
setLocale("LC_ALL", "C").isVoid;
getLocale("LC_ALL")     == "C";
getLocale("LC_CTYPE")   == "C";
getLocale("LC_NUMERIC") == "C";
\end{urbiassert}

  On invalid requests, raise an error.
\begin{urbiscript}
setLocale("LC_IMAGINARY");
[00006328:error] !!! setLocale: invalid category: LC_IMAGINARY

setenv("LC_ALL", "elfic")|;
setLocale("LC_ALL");
[00024950:error] !!! setLocale: cannot set locale LC_ALL to elfic

setLocale("LC_ALL", "klingon");
[00074958:error] !!! setLocale: cannot set locale LC_ALL to klingon
\end{urbiscript}


\item[shiftedTime] The number of seconds elapsed since the \urbi server was
  launched.  Contrary to \refSlot{time}, time spent in frozen code is not
  counted.
\begin{urbiassert}
{ var t0 = shiftedTime | sleep(1s) | shiftedTime - t0 }.round ~= 1;

  1 ==
  {
    var t = Tag.new|;
    var t0 = time|;
    var res;
    t: { sleep(1s) | res = shiftedTime - t0 },
    t.freeze;
    sleep(1s);
    t.unfreeze;
    sleep(1s);
    res.round;
  };
\end{urbiassert}


\item[shutdown](<exit_status> = 0)%
  Have the \urbi server shut down, with exit status \var{exit\_status}.  All
  the connections are closed, the resources are released.  Architecture
  dependent.


\item[sleep](<duration> = inf)%
  Suspend the execution for \var{duration} seconds.  No CPU cycle is wasted
  during this wait. If no \var{duration} is given the execution is suspended
  indefinitely.

\begin{urbiassert}
(time - {sleep(1s); time}).round == -1;
\end{urbiassert}


\item[spawn](<function>, <clear>)%
  Run the \var{function}, with fresh tags if \var{clear} is true, otherwise
  under the control of the current tags.  Return the spawn \refObject{Job}.
\begin{urbiscript}
var jobs = []|;
var res = []|;
for (var i : [0, 1, 2])
{
  jobs << System.spawn(closure () { res << i; res << i },
                       true) |
  if (i == 2)
    break
}|
jobs;
[00009120] [Job<shell_11>, Job<shell_12>, Job<shell_13>]
// Wait for the jobs to be done.
jobs.each (function (var j) { j.waitForTermination });
assert (res == [0, 1, 0, 2, 1, 2]);
\end{urbiscript}

\begin{urbiscript}
jobs = []|;
res = []|;
for (var i : [0, 1, 2])
{
  jobs << System.spawn(closure () { res << i; res << i },
                       false) |
  if (i == 2)
    break
}|
jobs;
[00009120] [Job<shell_14>, Job<shell_15>, Job<shell_16>]
// Give some time to get the output of the detached expressions.
sleep(100ms);
assert (res == [0, 1, 0]);
\end{urbiscript}

\item[stats]%
  Return a \refObject{Dictionary} containing information about the execution
  cycles of \urbi.  This is an internal feature made for developers, it
  might be changed without notice.  See also \refSlot{resetStats}.
\begin{urbiassert}
var stats = System.stats;

stats.isA(Dictionary);
stats.keys.sort == ["cycles",
                    "cyclesMin", "cyclesMean", "cyclesMax",
                    "cyclesVariance", "cyclesStdDev"].sort;
// Number of cycles.
0 < stats["cycles"];
// Cycles duration.
0 < stats["cyclesMin"] <= stats["cyclesMean"] <= stats["cyclesMax"];

stats["cyclesVariance"].isA(Float);
stats["cyclesStdDev"].isA(Float);
\end{urbiassert}


%% \item[stopall]


\item[system](<command>)%
  Ask the operating system to run the \var{command}.  This is typically used
  to start new processes.  The exact syntax of \var{command} depends on your
  system.  On Unix systems, this is typically \file{/bin/sh}, while Windows
  uses \file{command.exe}.

  Return the exit status.

  \begin{windows}
    Under Windows, the exit status is always 0.
  \end{windows}

\begin{urbiassert}
System.system("exit 0") == 0;
System.system("exit 23") >> 8
       == { if (System.Platform.isWindows) 0 else 23 };
\end{urbiassert}


\item[time] The number of seconds elapsed since the \urbi server was
  launched.  See also \refObject{Date}.  In presence of a frozen
  \refObject{Tag}, see also \refSlot{shiftedTime}.
\begin{urbiassert}
{ var t0 = time | sleep(1s) | time - t0 }.round ~= 1;

  2 ==
  {
    var t = Tag.new|;
    var t0 = time|;
    var res;
    t: { sleep(1s) | res = time - t0 },
    t.freeze;
    sleep(1s);
    t.unfreeze;
    sleep(1s);
    res.round;
  };
\end{urbiassert}


\item[unsetenv](<name>)%
  Undefine the environment variable \var{name}, return its previous value.
  See also \refSlot{getenv} and \refSlot{setenv}.

\begin{urbiassert}
setenv("MyVar", 12) == "12";
!getenv("MyVar").isNil;
unsetenv("MyVar") == "12";
getenv("MyVar").isNil;
\end{urbiassert}


\item[version]%
  The version of \usdk.  A string composed of two or more numbers separated
  by periods: \samp{"\packageVersion"}.
\begin{urbiassert}
System.version in Regexp.new("\\d+(\\.\\d+)+");
\end{urbiassert}
\end{urbiscriptapi}

%%% Local Variables:
%%% mode: latex
%%% TeX-master: "../urbi-sdk"
%%% ispell-dictionary: "american"
%%% ispell-personal-dictionary: "../urbi.dict"
%%% fill-column: 76
%%% End:
