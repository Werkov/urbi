%% Copyright (C) 2009-2011, Gostai S.A.S.
%%
%% This software is provided "as is" without warranty of any kind,
%% either expressed or implied, including but not limited to the
%% implied warranties of fitness for a particular purpose.
%%
%% See the LICENSE file for more information.

\section{Date}

This class is meant to record dates in time, with microsecond resolution.
See also \refSlot[System]{time}.
\experimental{}

\subsection{Prototypes}
\begin{refObjects}
\item[Orderable]
\item[Comparable]
\end{refObjects}

\subsection{Construction}

Without argument, newly constructed Dates refer to the current date.

\begin{urbiunchecked}[firstnumber=1]
Date.new;
[00000001] 2010-08-17 14:40:52.549726
\end{urbiunchecked}

With a string argument \var{d}, refers to the date contained in \var{d}.
The string should be formatted as \samp{\var{yyyy}-\var{mm}-\var{dd}
  \var{hh}:\var{mn}:\var{ss}} (see \refSlot{asString}). \var{mn} and
\var{ss} are optional. If the block \samp{\var{hh}:\var{mn}:\var{ss}} is
absent, the behavior is undefined.


\begin{urbiscript}
Date.new("2003-10-10 20:10:50");
[00000001] 2003-10-10 20:10:50

Date.new("2003-Oct-10 20:10");
[00000002] 2003-10-10 20:10:00

Date.new("2003-10-10 20");
[00000003] 2003-10-10 20:00:00
\end{urbiscript}

Pay attention that the format is rather strict; for instance too many spaces
between day and time result in an error.

\begin{urbiscript}
Date.new("2003-10-10  20:10:50");
[00001968:error] !!! new: cannot convert to date: 2003-10-10  20:10:50
\end{urbiscript}

\subsection{Slots}

\begin{urbiscriptapi}
\item['+'](<that>)%
  The date which corresponds to waiting \refObject{Duration} \var{that}
  after \this.
\begin{urbiassert}
Date.new("2010-08-17 12:00") + 60s == Date.new("2010-08-17 12:01");
\end{urbiassert}

\item['-'](<that>)%
  If \var{that} is a Date, the difference between \this and \var{that} as a
  \refObject{Duration}.
\begin{urbiassert}
Date.new("2010-08-17 12:01") - Date.new("2010-08-17 12:00") ==  60s;
Date.new("2010-08-17 12:00") - Date.new("2010-08-17 12:01") == -60s;
\end{urbiassert}

If \var{that} is a Duration or a Float, the corresponding Date.

\begin{urbiassert}
Date.new("2010-08-17 12:01") - 60s == Date.new("2010-08-17 12:00");
Date.new("2010-08-17 12:01") - 60s
  == Date.new("2010-08-17 12:01") - Duration.new(60s);
\end{urbiassert}

\item['=='](<that>)%
  Equality test.
\begin{urbiassert}
Date.new("2010-08-17 12:00") == Date.new("2010-08-17 12:00");
Date.new("2010-08-17 12:00") != Date.new("2010-08-17 12:01");
\end{urbiassert}

\item['<'](<that>)%
  Order comparison.
\begin{urbiassert}
   Date.new("2010-08-17 12:00") < Date.new("2010-08-17 12:01");
! (Date.new("2010-08-17 12:01") < Date.new("2010-08-17 12:00"));
\end{urbiassert}

\item[asFloat] The duration since the \refSlot{epoch}, as a Float.
\begin{urbiscript}
var d = Date.new("2002-01-20 23:59:59")|;
assert
{
  d.asFloat == d - d.epoch;
  d.asFloat.isA(Float);
};
\end{urbiscript}

\item[asString] Present as \samp{\var{yyyy}-\var{mm}-\var{dd}
    \var{hh}:\var{mn}:\var{ss}.\var{us}} where \var{yyyy} is the four-digit
  year, \var{mm} the three letters name of the month (Jan, Feb, ...),
  \var{dd} the two-digit day in the month (from 1 to 31), \var{hh} the
  two-digit hour (from 0 to 23), \var{mn} the two-digit number of minutes in
  the hour (from 0 to 59), and \var{ss} the two-digit number of seconds in
  the minute (from 0 to 59).
\begin{urbiassert}
Date.new("2009-02-14 00:31:30").asString == "2009-02-14 00:31:30";
\end{urbiassert}

\item[day]
  The day as a \refObject{Float}.
\begin{urbiscript}
{
  var d = Date.new("2010-09-29 17:32:53");
  assert(d.day == 29);
  d.day = 1;
  assert(d == Date.new("2010-09-01 17:32:53"));
};
\end{urbiscript}
\begin{urbiscript}
Date.new("2010-02-01 17:32:53").day = 29;
[00000001:error] !!! updateSlot: Day of month is not valid for year
\end{urbiscript}

\item[epoch]
  A fixed value, the ``origin of times'': January 1st 1970, at
  midnight.
\begin{urbiunchecked}
Date.epoch == Date.new("1970-01-01 00:00");
\end{urbiunchecked}

\item[hour]
  The hour as a \refObject{Float}.
\begin{urbiscript}
{
  var d = Date.new("2010-09-29 17:32:53");
  assert(d.hour == 17);
  d.hour = 8;
  assert(d == Date.new("2010-09-29 08:32:53"));
};
\end{urbiscript}

\item[minute]
  The minute as a \refObject{Float}.
\begin{urbiscript}
{
  var d = Date.new("2010-09-29 17:32:53");
  assert(d.minute == 32);
  d.minute = 12;
  assert(d == Date.new("2010-09-29 17:12:53"));
};
\end{urbiscript}

\item[month]
  The month as a \refObject{Float}.
\begin{urbiscript}
{
  var d = Date.new("2010-09-29 17:32:53");
  assert(d.month == 9);
  d.month = 3;
  assert(d == Date.new("2010-03-29 17:32:53"));
};
\end{urbiscript}

\item[now] The current date. Equivalent to Date.new.
\begin{urbiunchecked}
Date.now;
[00000000] 2012-03-02 15:31:42
\end{urbiunchecked}

\item[second]
  The second as a \refObject{Float}.
\begin{urbiscript}
{
  var d = Date.new("2010-09-29 17:32:53");
  assert(d.second == 53);
  d.second = 37;
  assert(d == Date.new("2010-09-29 17:32:37"));
};
\end{urbiscript}

\item[timestamp] Synonym for \refSlot{asFloat}.

\item[year]
  The year as a \refObject{Float}.
\begin{urbiscript}
{
  var d = Date.new("2010-09-29 17:32:53");
  assert(d.year == 2010);
  d.year = 2000;
  assert(d == Date.new("2000-09-29 17:32:53"));
};
\end{urbiscript}

\end{urbiscriptapi}


%%% Local Variables:
%%% mode: latex
%%% TeX-master: "../urbi-sdk"
%%% ispell-dictionary: "american"
%%% ispell-personal-dictionary: "../urbi.dict"
%%% fill-column: 76
%%% End:
