\chapter{\us language specifications}
\label{sec:lang}

\section{Syntax}
\subsection{Comments}

Comments are used to document the code, they are ignored by the
\us interpreter. \Cxx-style and \C-style comments are supported.

\begin{itemize}
\item A \lstinline|//| introduces a comment that lasts until the end
  of the line.
\item A \lstinline|/*| introduces a comment that lasts until
  \lstinline|*/| is encountered. Comments nest, contrary to \C:
  if two \lstinline|/*| are encountered, the
  comment will end after two \lstinline|*/|, not one.
\end{itemize}

\autoref{lst:comments} illustrates this.

\begin{lstlisting}[caption=Comments,label=lst:comments,float=\floatpos]
// C++ style comment
/* C style comment */
/* These comments /* do */ nest */
\end{lstlisting}

\subsection{Identifiers}
\label{sec:us-syn-id}

Identifiers in \us are composed of one or more alphanumeric or
underscore (\lstinline|_|) characters, given that they must not start
with a digit. That is, identifiers must match the
\lstinline|[a-zA-Z_][a-zA-Z0-9_]*| regular expression (see
\autoref{lst:identifiers}).  Additionally, identifiers must not match any of
the \us reserved word which are documented in \autoref{sec:syn-key}. The only
exception to this rule is \lstinline|new|, which can be used as the
method identifier in a method call, as shown in
\autoref{lst:identifiers-keywords}.  Finally, identifiers can also be written
between simple quotes (\lstinline|'|), in which case they may contain
any characters, as in \autoref{lst:escape-identifiers}.

\begin{lstlisting}[caption=Identifiers,label=lst:identifiers,float=\floatpos]
var x;
var foobar51;
obj.met_hod();
// Those are invalid.
// var 3x;
// obj.3x();
\end{lstlisting}

\begin{lstlisting}[caption=Identifiers cannot be keywords,
  label=lst:identifiers-keywords,float=\floatpos]
// Those are invalid because "if" is a keyword.
// var if;
// obj.if();
// However, keywords can be escaped with simple quotes
var 'if';
obj.'if'();
\end{lstlisting}

\begin{lstlisting}[caption=Escaping identifiers with simple quotes,
  label=lst:escape-identifiers,float=\floatpos]
// Identifiers can be escaped with simple quotes
var '%x';
var '1 2 3';
obj.'[]';
\end{lstlisting}

\subsection{Keywords}
\label{sec:syn-key}

Keywords are reserved words that cannot be used as identifiers, for
instance.  They are listed in \autoref{tab:keywords}.  Keywords can be
escaped, and thus used as identifier by surrounding them with simple
quotes (\lstinline|'|), as shown in \autoref{lst:escape-keyword}.

\renewcommand{\baselinestretch}{.85}
\begin{table}[\floatpos]
  \caption{Keywords}
  \label{tab:keywords}
  \centering
  \begin{tabular}{|c|c||c|c|}
    \hline
    Keyword                       & Remark                                  &
    Keyword                       & Remark                                  \\
    \hline
    \lstinline"and"               & Synonym for  operator                   &
    \lstinline"long"              & Unused                                  \\
    \lstinline"and_eq"            & Synonym for  operator                   &
    \lstinline"loop"              & \lstinline|loop&| and
                                    \lstinline-loop|- flavors               \\
    \lstinline"asm"               & Unused                                  &
    \lstinline"loopn"             & Deprecated in favor of for              \\
    \lstinline"at"                &                                         &
    \lstinline"mutable"           & Unused                                  \\
    \lstinline"auto"              & Unused                                  &
    \lstinline"namespace"         & Unused                                  \\
    \lstinline"bitand"            & Synonym for \lstinline|&| operator      &
    \lstinline"new"               &                                         \\
    \lstinline"bitor"             & Synonym for \lstinline-|- operator      &
    \lstinline"not"               & Synonym for \lstinline|!| operator      \\
    \lstinline"bool"              & Unused                                  &
    \lstinline"not_eq"            & Synonym for \lstinline|!=| operator     \\
    \lstinline"break"             &                                         &
    \lstinline"object"            &                                         \\
    \lstinline"call"              &                                         &
    \lstinline"onleave"           &                                         \\
    \lstinline"case"              &                                         &
    \lstinline"or"                & Synonym for \lstinline-||- operator     \\
    \lstinline"catch"             & Unused                                  &
    \lstinline"or_eq"             & Synonym for \lstinline-|=- operator     \\
    \lstinline"char"              & Unused                                  &
    \lstinline"private"           & Ignored                                 \\
    \lstinline"class"             &                                         &
    \lstinline"protected"         & Ignored                                 \\
    \lstinline"closure"           &                                         &
    \lstinline"public"            & Ignored                                 \\
    \lstinline"compl"             & Synonym for \lstinline|~|               &
    \lstinline"register"          & Unused                                  \\
    \lstinline"const"             & Unused                                  &
    \lstinline"reinterpret_cast"  & Unnused                                 \\
    \lstinline"const_cast"        & Unused                                  &
    \lstinline"return"            &                                         \\
    \lstinline"continue"          &                                         &
    \lstinline"short"             & Unused                                  \\
    \lstinline"default"           & Unused                                  &
    \lstinline"signed"            & Unused                                  \\
    \lstinline"delete"            &                                         &
    \lstinline"sizeof"            & Unused                                  \\
    \lstinline"do"                &                                         &
    \lstinline"static"            & Deprecated                              \\
    \lstinline"double"            & Unused                                  &
    \lstinline"static_cast"       & Unused                                  \\
    \lstinline"dynamic_cast"      & Unused                                  &
    \lstinline"stopif"            &                                         \\
    \lstinline"else"              &                                         &
    \lstinline"struct"            & Unused                                  \\
    \lstinline"emit"              &                                         &
    \lstinline"switch"            &                                         \\
    \lstinline"enum"              & Unused                                  &
    \lstinline"template"          & Unused                                  \\
    \lstinline"event"             &                                         &
    \lstinline"this"              &                                         \\
    \lstinline"every"             &                                         &
    \lstinline"throw"             & Unused                                  \\
    \lstinline"explicit"          & Unused                                  &
    \lstinline"timeout"           &                                         \\
    \lstinline"export"            & Unused                                  &
    \lstinline"try"               & Unused                                  \\
    \lstinline"extern"            & Unused                                  &
    \lstinline"typedef"           & Unused                                  \\
    \lstinline"external"          &                                         &
    \lstinline"typeid"            & Unused                                  \\
    \lstinline"float"             & Unused                                  &
    \lstinline"typename"          & Unused                                  \\
    \lstinline"for"               & \lstinline|for&| and                    &
                                    \lstinline-for|- flavors
    \lstinline"union"             & Unused                                  \\
    \lstinline"foreach"           & Deprecated in favor of for              &
    \lstinline"unsigned"          & Unused                                  \\
    \lstinline"freezeif"          &                                         &
    \lstinline"using"             & Unused                                  \\
    \lstinline"friend"            & Unused                                  &
    \lstinline"var"               &                                         \\
    \lstinline"from"              &                                         &
    \lstinline"virtual"           & Unused                                  \\
    \lstinline"function"          &                                         &
    \lstinline"volatile"          & Unused                                  \\
    \lstinline"goto"              & Unused                                  &
    \lstinline"waituntil"         &                                         \\
    \lstinline"if"                &                                         &
    \lstinline"wchar_t"           & Unused                                  \\
    \lstinline"in"                &                                         &
    \lstinline"whenever"          &                                         \\
    \lstinline"inline"            & Unused                                  &
    \lstinline"while"             & \lstinline|while&| and
                                    \lstinline-while|- flavors              \\
    \lstinline"int"               & Unused                                  &
    \lstinline"xor"               & Synonym for \lstinline|^| operator      \\
    \lstinline"internal"          & Deprecated                              &
    \lstinline"xor_eq"            & Synonym \lstinline|^=| operator         \\
    \hline
  \end{tabular}
\end{table}
\renewcommand{\baselinestretch}{1}

\begin{lstlisting}[caption=Escaping keywords,label=lst:escape-keyword,float=\floatpos]
  var 'if' = 21;
  [00000000] 21
  'if' * 2;
  [00000000] 42
\end{lstlisting}

\subsection{Literals}
\subsubsection{Durations}

Durations are floats (see \autoref{sec:us-syn-lit-float}) followed by a
% FIXME: formulation
time unit. They are simply equivalent to the same float, expressed in
millisecond. For instance, \lstinline|1s 1ms|, which stands for ``one
second and 1 millisecond'', is strictly equivalent to
\lstinline|1.0001|. Available units
and their equivalent are shown in \autoref{tab:durations}.

\begin{table}[\floatpos]
  \caption{Duration units}
  \label{tab:durations}
  \centering
  \begin{tabular}{|c|c|c|}
    \hline
    unit        & abbreviation & equivalence for n         \\
    \hline
    millisecond & ms           & $n$                       \\
    second      & s            & $n * 1000$                \\
    minute      & m            & $n * 1000 * 60$           \\
    hour        & h            & $n * 1000 * 60 * 60$      \\
    day         & d            & $n * 1000 * 60 * 60 * 24$ \\
    \hline
  \end{tabular}
\end{table}

\subsubsection{Floats}
\label{sec:us-syn-lit-float}

Literal floats consist in a succession of digit, representing the
integral part of the number in decimal base, possibly followed by a
dot (\lstinline|.|) and another succession of digit representing the
decimal part. Briefly, float literals must verify the
\lstinline|[0-9]+(\.[0-9]+)?| regular expression. \autoref{lst:literal-floats}
is an example of literal floats.

\begin{lstlisting}[caption=Literal floats,label=lst:literal-floats,float=\floatpos]
  1;
  [00000000] 1
  1.1;
  [00000000] 1.1
\end{lstlisting}

\subsubsection{Lists}
\label{sec:us-syn-lit-list}

Literal lists are represented with a comma-separated, potentially
empty list of arbitrary expressions enclosed in square brackets
(\lstinline|[]|), as shown in \autoref{lst:literal-lists}.

\begin{lstlisting}[caption=Literal lists,label=lst:literal-lists,float=\floatpos]
  []; // The empty list
  [00000000] []
  [1, 2, 3];
  [00000000] [1, 2, 3]
\end{lstlisting}

\subsubsection{Strings}

String are enclosed in double quotes (\lstinline|"|) and can contain
arbitrary characters, which stand for themselves, with the exception
of the escapement character, backslash (\lstinline|\|), as we can see
in \autoref{lst:literal-strings}. Backslash can introduce the following escapements:

\begin{itemize}
  \item \lstinline |\\|: an actual backslash
  \item \lstinline|\a|: a bell ring
  \item \lstinline|\b|: a backspace
  \item \lstinline|\f|: a form feed
  \item \lstinline|\n|: a line feed
  \item \lstinline|\r|: a carriage return
  \item \lstinline|\t|: a tabulation
  \item \lstinline|\v|: a vertical tabulation
\end{itemize}

Note that the interpreter print escaped strings. That is, line feed
will be printed out as \lstinline|\n| when a string result is
dumped. An actual line feed will of course be output if a string
content is printed with echo for instance.

\begin{lstlisting}[caption=Literal strings,label=lst:literal-strings,float=\floatpos]
  "";
  [00000000] ""
  "foo";
  [00000000] ""
  "a\nb"; // UrbiScript escapes string when dumping them
  [00000000] "a\nb"
  echo("a\nb"); // We can see there is an actual line feed
  [00000000] *** a
  ..b
  echo("a\\nb");
  [00000000] a\nb
\end{lstlisting}

\subsection{Operators}

\us supports many operators, most of which are inspired from
\Cxx. Their syntax is presented here, and they are sorted and
described with their original semantic - that is, \lstinline|+| is an
arithmetic operator that sums two numeric values. However, as in \Cxx,
these operators might be use for any other purpose - that is,
\lstinline|+| can also be used as the concatenation operator on lists
and strings. Their semantic is thus not limited to what is presented
here.

Tables in this section sort operators top-down, by precedence order.
Group of rows (not separated by horizontal lines) describe operators
that have the same precedence. Many operators are syntactic sugar that
bounce on a method. In this case, the equivalent desugared expression
is shown in the ``Equivalence'' column. This can help you determine
what method to override to define an operator for an object (see
\fixme{autoref{sec:fixme}}).

Although code examples are present, this section does not intend do
define those operators behaviors; it only present their syntax and
grammatical rules, such as precedence and associativity. The behavior
of those operators is described in the documentation of the classes on
which they are applicable, in \autoref{sec:stdlib}.

% Operator generators
\newcommand{\operatorhead}{Operator & Use & Associativity & Original semantic
  & Equivalence\\}


\newcommand{\operator}[6][ ]{\lstinline@#2@&\lstinline@#3@&#4&#5&\lstinline@#6@#1\\}
\newcommand{\boperator}[3]{\operator{#1}{a #1 b}{#2}{#3}{a.'#1'(b)}}
\newcommand{\poperator}[3]{\operator{#1}{#1a}{#2}{#3}{a.'#1'()}}

\newcommand{\operatordot}    {\operator  {.}    {a.b}              {-}     {Message sending}          {Not redefinable}       }
\newcommand{\operatordota}   {\operator  {.}    {a.b(args)}        {-}     {Message sending}          {Not redefinable}       }
\newcommand{\operatorsub}    {\operator  {[]}   {a[args]}          {-}     {Subscript}                {a.'[]'(args)}          }
\newcommand{\operatorsubass} {\operator  {[] =} {a[args] = v}      {-}     {Subscript assignment}     {a.'[]='(args, v)}      }
\newcommand{\operatorass}[2][ ]    {\operator[#1]
                                         {=}    {a = b}            {Right} {Assignment}               {updateSlot("a", b)}    }

\newcommand{\operatoriass}[1]{\operator  {#1=}  {a #1= b}          {Right} {In place assignment}      {a = a #1 b}            }
\newcommand{\operatorsiass}  {
    \operatoriass{+}
    \operatoriass{-}
    \operatoriass{*}
    \operatoriass{/}
    \operatoriass{\%}
    \operatoriass{\^}
    \operatoriass{\~}
}
\newcommand{\operatorinc}    {\operator  {++}   {a++}              {-}     {Incrementation}           {(a = a + 1) - 1}       }
\newcommand{\operatordec}    {\operator  {--}   {a--}              {-}     {Incrementation}           {(a = a - 1) + 1}       }

\newcommand{\operatoruplus}  {\poperator {+}    {-}                {Identity}               }
\newcommand{\operatorumin}   {\poperator {-}    {-}                {Opposite}               }
\newcommand{\operatorexp}    {\boperator {**}   {Right}            {Exponentiation}         }
\newcommand{\operatormult}   {\boperator {*}    {Left}             {Multiplication}         }
\newcommand{\operatordiv}    {\boperator {/}    {Left}             {Division}               }
\newcommand{\operatormod}    {\boperator {\%}   {Left}             {Modulo}                 }
\newcommand{\operatorplus}   {\boperator {+}    {Left}             {Sum}                    }
\newcommand{\operatorminus}  {\boperator {-}    {Left}             {Difference}             }
\newcommand{\operatorlshift} {\boperator {<<}   {Left}             {Left bit shift}         }
\newcommand{\operatorrshift} {\boperator {>>}   {Left}             {Right bit shift}        }
\newcommand{\operatoreq}     {\boperator {==}   {Non Associative}  {Equality}               }
\newcommand{\operatorneq}    {\boperator {!=}   {Non Associative}  {Inequality}             }
\newcommand{\operatorpeq}    {\boperator {===}  {Non Associative}  {Physical equality}      }
\newcommand{\operatorpneq}   {\boperator {!==}  {Non Associative}  {Physical Inequality}    }
\newcommand{\operatoraeq}    {\boperator {=~=}  {Non Associative}  {Approximative equality} }
\newcommand{\operatorinf}    {\boperator {<}    {Non Associative}  {Inferior}               }
\newcommand{\operatorinfeq}  {\boperator {<=}   {Non Associative}  {Inferior or equal}      }
\newcommand{\operatorsup}    {\boperator {>}    {Non Associative}  {Superior}               }
\newcommand{\operatorsupeq}  {\boperator {>=}   {Non Associative}  {Superior or equal}      }
\newcommand{\operatorbxor}   {\boperator {^}    {Left}             {Bitwise exclusive or}   }
\newcommand{\operatorneg}    {\poperator {!}    {Left}             {Logical negation}       }
\newcommand{\operatorand}    {\boperator {\&\&} {Left}             {Logical and}            }
\newcommand{\operatoror}     {\boperator {||}   {Left}             {Logical or}             }

\subsubsection{Arithmetic operators}

\us supports classic arithmetic operators, with the classic semantic
on numeric values. See \autoref{tab:arithmetic-operators} and
\autoref{lst:arithmetic-operators}.

\begin{table}[\floatposh]
  \caption{Arithmetic operators}
  \label{tab:arithmetic-operators}
  \centering
  \begin{tabular}{|c|c|c|c|c|c|}
    \hline
    \operatorhead
    \hline
    \operatoruplus
    \operatorumin
    \hline
    \operatorexp
    \hline
    \operatormult
    \operatordiv
    \operatormod
    \hline
    \operatorplus
    \operatorminus
    \hline
  \end{tabular}
\end{table}

\begin{lstlisting}[caption=Arithmetic operators,
  label=lst:arithmetic-operators,float=\floatposh]
  1 + 1;
  [00000000] 2
  1 - 2;
  [00000000] -1
  2 * 3;
  [00000000] 6
  10 / 2;
  [00000000] 5
  2 ** 10;
  [00000000] 1024
  -(1 + 2);
  [00000000] -3
\end{lstlisting}

\subsubsection{Assignment operators}

Assignment in \us can be performed with the \lstinline|=| operator. A
large variety of shorthands such as \lstinline|+=| exist. Those are
not directly redefinable since they are equivalent to a normal
affectation combined with another operator. See
\autoref{tab:assignment-operators} and \autoref{lst:assignment-operators}.


\begin{table}[\floatposh]
  \caption{Assignment operators}
  \label{tab:assignment-operators}
  \centering
  \begin{tabular}{|c|c|c|c|c|c|}
    \hline
    \operatorhead
    \hline
    \operatorass[\footnotemark]{}
    \operatorsiass
    \hline
  \end{tabular}
\end{table}
\footnotetext{For object fields only. Assignment to local variables
  cannot be redefined. }

\begin{lstlisting}[caption=Assignment operators,
  label=lst:assignment-operators,float=\floatposh]
  var x = 0;
  [00000000] 0
  x = 10;
  [00000000] 10
  x += 10;
  [00000000] 20
  x /= 5;
  [00000000] 4
  x %= 3;
  [00000000] 1
  x++;
  [00000000] 1
  x;
  [00000000] 2
\end{lstlisting}

\subsubsection{Bitwise operators}

\us supports some bitwise operators, although they are often use for
other purpose than bit-related operations. See \autoref{tab:bitwise-operators}
and \autoref{lst:bitwise-operators}.

\begin{table}[\floatposh]
  \caption{Bitwise operators}
  \label{tab:bitwise-operators}
  \centering
  \begin{tabular}{|c|c|c|c|c|c|}
    \hline
    \operatorhead
    \hline
    \operatorlshift
    \operatorrshift
    \hline
    \operatorbxor
    \hline
  \end{tabular}
\end{table}

\begin{lstlisting}[caption=Bitwise operators,
  label=lst:bitwise-operators,float=\floatposh]
  4 << 2;
  [00000000] 16
  4 >> 2;
  [00000000] 1
\end{lstlisting}

\subsubsection{Logic operators}

\us support classic logic operators. See \autoref{tab:logic-operators} and
\autoref{lst:logic-operators}.

\begin{table}[\floatposh]
  \caption{Logic operators}
  \label{tab:logic-operators}
  \centering
  \begin{tabular}{|c|c|c|c|c|c|}
    \hline
    \operatorhead
    \hline
    \operatorneg
    \hline
    \operatorand
    \hline
    \operatoror
    \hline
  \end{tabular}
\end{table}

\begin{lstlisting}[caption=Logic operators,
  label=lst:logic-operators,float=\floatposh]
  true && true;
  [00000000] true
  true || false;
  [00000000] true
  !true;
  [00000000] false
\end{lstlisting}

\subsubsection{Comparison operators}

\us support classic comparison operators. See \autoref{tab:comparison-operators} and
\autoref{lst:comparison-operators}.

\begin{table}[\floatposh]
  \caption{Comparison operators}
  \label{tab:comparison-operators}
  \centering
  \begin{tabular}{|c|c|c|c|c|c|}
    \hline
    \operatorhead
    \hline
    \operatoreq
    \operatorneq
    \operatorpeq
    \operatorpneq
    \operatoraeq
    \operatorinf
    \operatorinfeq
    \operatorsup
    \operatorsupeq
    \hline
  \end{tabular}
\end{table}

\begin{lstlisting}[caption=Comparison operators,
  label=lst:comparison-operators,float=\floatposh]
  0 < 0;
  [00000000] false
  0 <= 0;
  [00000000] true
  0 == 0;
  [00000000] true
  0 === 0;
  [00000000] false
  var x = 0;
  [00000000] 0
  x === x;
  [00000000] true
  x !== x;
  [00000000] false
\end{lstlisting}

\subsubsection{Miscellaneous operators}

Those operators cannot be classified in other categories. See
\autoref{tab:miscellaneous-operators} and \autoref{lst:miscellaneous-operators}. Note
that the subscript (square bracket) operator is variadic: it can take
any number of arguments, that will be passed to the \lstinline|'[]'|
method of the targeted object.

\begin{table}[\floatposh]
  \caption{Miscellaneous operators}
  \label{tab:miscellaneous-operators}
  \centering
  \begin{tabular}{|c|c|c|c|c|c|}
    \hline
    \operatorhead
    \hline
    \operatordot
    \operatordota
    \hline
    \operatorsub
    \operatorsubass
    \hline
  \end{tabular}
\end{table}

\begin{lstlisting}[caption=Miscellaneous operators,
  label=lst:miscellaneous-operators,float=\floatposh]
  0 < 0;
  [00000000] false
  0 <= 0;
  [00000000] true
  0 == 0;
  [00000000] true
  0 === 0;
  [00000000] false
  var x = 0;
  [00000000] 0
  x === x;
  [00000000] true
  x !== x;
  [00000000] false
\end{lstlisting}

\clearpage
\subsubsection{All operators summary}

\autoref{tab:operators-summary} is a summary of all operators, to highlight
the overall precedences. Operators are sorted vertically by decreasing
precedence. Groups of rows represent operators with the same
precedence.

\begin{table}[\floatposh]
  \caption{Operators summary}
  \label{tab:operators-summary}
  \centering
  \begin{tabular}{|c|c|c|c|c|c|}
    \hline
    Operator               & Use                                    & Associativity
    & Original semantic    & Equivalence                            \\
    \hline
    \operatordot
    \operatordota
    \hline
    \operatorsub
    \operatorsubass
    \hline
    \operatoruplus
    \operatorumin
    \hline
    \operatorexp
    \hline
    \operatormult
    \operatordiv
    \operatormod
    \hline
    \operatorplus
    \operatorminus
    \hline
    \operatorlshift
    \operatorrshift
    \hline
    \operatoreq
    \operatorneq
    \operatorpeq
    \operatorpneq
    \operatoraeq
    \operatorinf
    \operatorinfeq
    \operatorsup
    \operatorsupeq
    \hline
    \operatorbxor
    \hline
    \operatorneg
    \hline
    \operatorand
    \hline
    \operatoror
    \hline
    \operatorass
    \operatorsiass
    \hline
    \operatorinc
    \operatordec
    \hline
  \end{tabular}

\end{table}
\FloatBarrier

\FloatBarrier
\section{Scopes and local variables}

\subsection{Scopes}

% FIXME: This is wrong: the last separator isn't ignored; a pipe voids
% the result, for instance.
Scopes in \us are sequences of expressions, enclosed in curly brackets
(\lstinline|{}|). Expressions are separated with the four expression
separators (see \fixme{autoref{sec:fixme}}). The last separator is optional and
ignored if present. Scopes are themselves expressions, and can thus be
used in composite expressions, nested, \ldots

\begin{lstlisting}[caption=Scopes,label=lst:scopes,float=\floatpos]
  // Scopes evaluate to their last expression
  {
    1;
    2;
    3;
  };
  [00000000] 3
  // Scopes can be used as expressions
  {1; 2; 3} + 1;
  [00000000] 4
\end{lstlisting}

\subsection{Local variables}

Local variables are introduced with the \lstinline|var| keyword,
followed by a valid identifier (see \autoref{sec:us-syn-id}) and an optional
initialization value assignment. If the initial value is omitted, it
defaults to void (\autoref{sec:std-void}). Variable declarations evaluate to
the initialization value. They can later be referred to by their
name. Their value can be changed with the assignment operator; such an
assignment expression returns the new value. The use of local
variables is illustrated in \autoref{lst:local-variables}.

\begin{lstlisting}[caption=Using local variables,
  label=lst:local-variables,float=\floatpos]
  // This declare variable x with value 42, and evaluates to 42.
  var x = 42;
  [00000000] 42
  // x equals 42
  x;
  [00000000] 42
  // We can assign it a new value
  x = 51;
  [00000000] 51
  x;
  [00000000] 51
  // Initialization defaults to void
  var y;
  y.isVoid;
  [00000000] true
\end{lstlisting}

Local variables lifespan is the same as their enclosing scope. They
are thus only accessible from their scope and its
subscopes\footnote{Local variables can actually escape their scope
  with lexical closures, as explained in \autoref{sec:us-fun-closures}}. Two
variables with the same name cannot be defined in the same scope. A
variable with the same name can be redefined in a subscope, in which
case references refer to the innermost variable, as shown in
\autoref{lst:local-variables-scoping}.

\begin{lstlisting}[caption=Local variables scoping,
  label=lst:local-variables-scoping,float=\floatpos]
  {
    var x = 0;
    var y = 1;
    {
      var y = 2;
      var z = 3;
      // We can access variables of parent scopes.
      echo(x);
      // This refers to the inner y.
      echo(y);
      echo(z);
    };
    // This refers to the outer y.
    echo(y);
    // This would be invalid: z does not exist anymore.
    // echo(z);
    // This would be invalid: x is already declared in this scope.
    // var x;
  };
  [00000000] *** 0
  [00000000] *** 2
  [00000000] *** 3
  [00000000] *** 1
\end{lstlisting}

\FloatBarrier
\section{Functions}

\subsection{Functions creation}

Functions in \us are first class citizen. Which mean, unlike \C and
related languages, a function is a normal value, just like an integer
or a string is. One can create a functional value thanks to the
\lstinline|function| keyword, followed by the list of formal arguments
and a scope representing the body of the function. Formal arguments
are a comma separated list of identifiers. \autoref{lst:function} illustrates
this.

\begin{lstlisting}[caption=Functional
  value,label=lst:function,float=\floatpos]
  function (arg1, arg2) { echo(0) };
  [00000000] function(arg1, arg2) {
  ..  echo(0)
  ..}
\end{lstlisting}

The whole point of a function is to be called later, so storing it in
a variable is the most common usage. \us provide a simple sugar to
initialize a new variable with a given function, presented in
\autoref{lst:function-sugar}.

\begin{lstlisting}[caption=Storing functions in variables,
  label=lst:function-sugar,float=\floatpos]
  // Functions are often stored in variables to call them later
  var f1 = function () {
    echo("hello")
  };
  f1();
  [00000000] *** hello
  // This form is strictly equivalent, yet simpler
  function f2()
  {
    echo("hello");
  }
  f2();
  [00000000] *** hello
\end{lstlisting}

\subsection{Arguments}

The list of formal arguments defines the number of argument the
function requires. They are accessible by their name from within the
body, as shown in \autoref{lst:function-args}. The list of formal arguments
can be omitted, in which case the number of given arguments isn't
check, and arguments themselves are not evaluated. Arguments can then
be manipulated with the call message, explained in
\autoref{sec:us-fun-callmsg}.

\begin{lstlisting}[caption=Storing and calling
  functions,label=lst:function-args,float=\floatpos]
  var f = function(a, b) {
    echo(b);
    echo(a);
  };
  f(1, 0);
  [00000000] *** 0
  [00000000] *** 1
  // Calling a function with the wrong number of argument is an error
  f(0);
  [00000000:error] !!! f: expected 2 arguments, given 1
  f(0, 1, 2);
  [00000000:error] !!! f: expected 2 arguments, given 3
\end{lstlisting}

\subsection{Return value}

The return value of the function is the evaluation of its body - that
is, since the body is a scope, the last evaluated expression in the
scope. However, value can be returned manually with the
\lstinline|return| keyword followed by the value, in which case the
evaluation of the function is stopped. If \lstinline|return| is used
with no value, the functions returns void. See \autoref{lst:function-return}.

\begin{lstlisting}[caption=Returning values from functions,
  label=lst:function-return,float=\floatpos]
  function f(a, b)
  {
    echo(a);
    echo(b);
    a // Return value is a
  };
  f(1, 2)
  [00000000] *** 1
  [00000000] *** 2
  [00000000] 1
  function g(a, b)
  {
    echo(a);
    return a; // Stop execution at this point and return a
    echo(b); // This is not executed
  };
  g(1, 2);
  [00000000] *** 1
  [00000000] 1
  function h()
  {
    return; // Stop execution at this point and return void
    echo(0); // This is not executed
  };
  h(); // Returns void, so nothing is printed.
\end{lstlisting}

\subsection{Call messages}
\label{sec:us-fun-callmsg}

Functions can access meta-information about how they were called,
through a \lstinline|CallMessage| object. The call message associated
with a function can be accessed with the \lstinline|call| keyword. It
contains several information such as not-yet evaluated arguments, the
name of the function, the target \ldots

One of the main usage of call message is to create lazy function, by
not specifying the formal arguments so as arguments are not evaluated
at the call site, and evaluate them by hand. A quick example is given
in \autoref{lst:function-lazy}. More information about this can be found in
the CallMessage class documentation (\autoref{sec:std-callmsg}).

\begin{lstlisting}[caption=Lazy
  function,label=lst:function-lazy,float=\floatposh]
  // Do not specify formal arguments, so as
  // effective arguments are not evaluated at call site
  function lazy
  {
    // Evaluate only the first argument, twice
    call.evalArgAt(0);
    call.evalArgAt(0);
  };
  lazy(echo("a"), echo("b"));
  [00000000] *** a
  [00000000] *** a
\end{lstlisting}

\subsection{Lexical closures}
\label{sec:us-fun-closures}

Lexical closure are an additional scoping rule, with which a function
can refer to a local variable located outside the function - but still
in the current context. \us supports read/write lexical closures,
meaning that the variable is shared between the function and the outer
environment, as shown in \autoref{lst:function-closure}.

\begin{lstlisting}[caption=Lexical
  closure,label=lst:function-closure,float=\floatpos]
  var x = 0;

  function cl()
  {
    // x refers to a variable outside the function
    x++;
    echo(x);
  };
  cl();
  [00000000] *** 1
  x;
  [00000000] 1
  x++;
  [00000000] 1
  cl();
  [00000000] *** 3
\end{lstlisting}

\autoref{lst:function-closure-escape} illustrate that local variables can even
escape their declaration scope by lexical closure.

\begin{lstlisting}[caption=Local variable escaping its scope by
  lexical closure.,label=lst:function-closure-escape,float=\floatpos]
  function wrapper()
  {
    // Normally, x is local to 'wrapper', and is limited to this scope.
    var x = 0;
    at (x > 1)
      echo("ping");
    // Here we make it escape the scope by returning a closure on it.
    return function() { x++ };
  };
  var f = wrapper();
  f();
  [00000000] 1
  f();
  [00000000] 2
  [00000000] *** ping
\end{lstlisting}

\FloatBarrier
\section{Objects}

Any value in \us is an object. Objects contain:

\begin{itemize}
\item A list of prototypes, which are also objects.
\item A list of slots, which to a name associate an object.
\end{itemize}

\subsection{Slots}

\subsubsection{Manipulation}

Objects can contain any number of slots, every slot has a name and a
value. They are often call ``fields'' or ``members'' in other object
oriented language.

The \lstinline|createSlot| is used to add a slot to an object with the
void (\autoref{sec:std-void}) value. The \lstinline|updateSlot| can be later
used to change the value of the slot, while the \lstinline|getSlot|
method can be used to read it. The \lstinline|setSlot| method can be
used to create a slot with a given value instead of void. Finally, the
\lstinline|slotNames| method can be used to get the list of the object
slot's name. \autoref{lst:slots} shows how to manipulate slots. More
documentation about these method can be found in \autoref{sec:std-object}.

\begin{lstlisting}[caption=Manipulating slots, label=lst:slots,
  float=\floatpos]
  var o = Object.new;
  o.slotNames;
  [00000000] []
  o.createSlot("test");
  [00000000] ["test"]
  o.slotNames;
  [00000000] ["test"]
  o.getSlot("test").asString;
  [00000000] "void"
  o.updateSlot("test", 42);
  [00000000] 42
  o.getSlot("test");
  [00000000] 42
\end{lstlisting}

\subsubsection{Sugars}

Several sugars exist to ease the use of slot methods:

\begin{itemize}
\item \lstinline|var o.name| is equivalent to
  \lstinline|o.createSlot("name")|.
\item \lstinline|var o.name = value| is equivalent to
  \lstinline|o.setSlot("name", value)|.
\item \lstinline|o.name = value| is equivalent to
  \lstinline|o.updateSlot("name", value)|.
\item \lstinline|delete o.name| is equivalent to
  \lstinline|o.removeSlot("name")|.
\end{itemize}


\subsection{Prototypes}

\subsubsection{Manipulation}

\us is a prototype-based language, unlike most classical object
oriented language, which are class-based. In prototype-based
languages, objects have no type, only prototypes, from which they
inherits behavior.

In \us, an object can have several prototypes. The list of prototypes
can be read with the \lstinline|protos| method. Prototypes can be add
or removed with the \lstinline|addProto| and \lstinline|removeProto|
methods respectively, as illustrated by \autoref{lst:prototypes}. More
documentation about these methods can be found in \autoref{sec:std-object}.

\begin{lstlisting}[caption=Manipulating prototypes, label=lst:prototypes,
  float=\floatpos]
  var o = Object.new;
  o.protos;
  [00000000] [<Object>]
  o.addProto(Pair);
  o.protos;
  [00000000] [<Object>, <Pair>]
  o.removeProto(Object);
  o.protos;
  [00000000] [<Pair>]
\end{lstlisting}

\subsubsection{Inheritance}

Objects inherit their prototypes' slots: \lstinline|getSlot| will also
look in an object prototypes' slots. \lstinline|getSlot| performs a
depth-first traversal of the prototypes hierarchy to find slots. That
is, when looking for a slot in an object:

\begin{itemize}
\item \lstinline|getSlot| checks first if the object itself has the
  requested slot. If so, it returns its value.
\item Otherwise, it applies the same research on every prototype, in
  the order of the prototype list (since addProto inserts in the front
  of the prototype list, the last prototype added has priority). This
  search is recursive: \lstinline|getSlot| will also look in the first
  prototype's prototype, etc before looking in the second
  prototype. If the slot is found in a prototype, it is returned.
\item Finally, if no prototype had the slot, an error is raised.
\end{itemize}

\autoref{lst:inheritance} shows how slots are inherited.

\begin{lstlisting}[caption=Slots inheritance, label=lst:inheritance,
  float=\floatpos]
  var a = Object.new;
  var b = Object.new;
  var c = Object.new;
  a.setSlot("x", "slot in a");
  b.setSlot("x", "slot in b");
  // c has no "x" slot
  c.getSlot("x");
  [00000000:error] !!! lookup failed: x
  // c can inherit the "x" slot from a
  c.addProto(a);
  c.getSlot("x");
  [00000000] "slot in a"
  // b is prepended to the prototype list, and has thus priority
  c.addProto(b);
  c.getSlot("x");
  [00000000] "slot in b"
  // a local slot in c has priority over prototypes
  c.setSlot("x", "slot in c");
  c.getSlot("x");
  [00000000] "slot in c"
\end{lstlisting}

\subsubsection{Copy on write}

The \lstinline|updateSlot| method has a particular behavior with
respect to prototypes. Although performing an \lstinline|updateSlot|
on a non existent slot is an error, it is valid if the slot is
inherited from a prototype. In this case, the slot is however not
updated in the prototype, but rather created in the object itself,
with the new value. This process is called \emph{copy on write}; thanks
to it, prototypes are not altered when the slot is overridden in a
child object (see \autoref{lst:copy-on-write}).

\begin{lstlisting}[caption=Copy on write, label=lst:copy-on-write,
  float=\floatpos]
  var p = Object.new;
  var p.slot = 0;
  var c = Object.new;
  c.addProto(p);
  c.slot;
  [00000000] 0
  c.slot = 1;
  [00000000] 1
  // p's slot was not altered
  p.slot;
  [00000000] 0
  // It was copied in c
  c.slot;
  [00000000] 1
\end{lstlisting}

\subsection{Sending messages}

A message in \us consists in a message name and arguments. One can
send a message to an object with the dot (\lstinline|.|) operator,
followed by the message name (which can be any valid identifier) and
the arguments, as shown in \autoref{lst:messages}. When there are no
arguments, the empty parentheses can be omitted. As you might see,
sending messages is very similar to calling methods in classical
languages.

\begin{lstlisting}[caption=Sending messages, label=lst:messages,
  float=\floatpos]
  // Send the message msg to object obj, with arguments arg1 and arg2
  obj.msg(arg1, arg2);
  // Send the message msg to object obj, with no arguments
  obj.msg();
  // This is strictly equivalent
  obj.msg;
\end{lstlisting}

When a message ``msg'' is sent to object \lstinline|obj|:

\begin{itemize}
\item The ``msg'' slot of \lstinline|obj| is retrieved (i.e.,
  \lstinline|obj.getSlot("msg")|). If the slot is not found, the
  classic lookup error is raised.
\item If the object is not a \lstinline|Routine| (\fixme{autoref{sec:fixme}}), it's
  the result of the message sending. In this case, there must be no
  argument, otherwise an error is raised.
\item If the object is a \lstinline|Routine|, it is invoked with the
  message sending arguments, and the returned value is the result. As
  a consequence, the number of arguments in the message sending must
  match the one required by the \lstinline|Routine|.
\end{itemize}

Such message sending is illustrated by \autoref{lst:send-message}.

\begin{lstlisting}[caption=Sending messages, label=lst:send-message,
  float=\floatpos]
  var obj = Object.new;
  var obj.a = 42;
  var obj.b = function (x) { x + 1 };
  obj.a;
  [00000000] 42
  obj.a();
  [00000000] 42
  obj.a(50);
  [00000000:error] !!! a: expected 0 arguments, given 1
  obj.b;
  [00000000:error] !!! b: expected 1 arguments, given 0
  obj.b();
  [00000000:error] !!! b: expected 1 arguments, given 0
  obj.b(50);
  [00000000] 51
\end{lstlisting}

\FloatBarrier
\section{Imperative flow control}

\subsection{break}

When encountered within a \lstinline|for| or a \lstinline|while| loop,
\lstinline|break| makes execution jump after the loop. See
\autoref{lst:break}.

\begin{lstlisting}[caption=Using break, label=lst:break,
  float=\floatpos]
  var i = 5;
  for (; true; echo(i))
  {
    if (i > 8)
      break;
    i++;
  }
  [00000000] *** 6
  [00000000] *** 7
  [00000000] *** 8
  [00000000] *** 9
\end{lstlisting}

\subsection{continue}

When encountered within a \lstinline|for| or a \lstinline|while| loop,
\lstinline|break| makes execution jump at the loop's body end. See
\autoref{lst:continue}.

\begin{lstlisting}[caption=Using continue, label=lst:continue,
  float=\floatpos]
  for (var i = 0; i < 8; i++)
  {
    if (i % 2 != 0)
      continue;
    echo(i);
  }
  [00000000] *** 0
  [00000000] *** 2
  [00000000] *** 4
  [00000000] *** 6
\end{lstlisting}

\subsection{do}

The \lstinline|do| construct enables do evaluate an expression with a
different ``\lstinline|this|''.

\begin{lstlisting}[frame=, backgroundcolor=, ]
  do target
  {
    body
  }
\end{lstlisting}

It evaluates \lstinline|body|, with \lstinline|this| being
\lstinline|target|, as shown in \autoref{lst:do}.  The \lstinline|do| itself
evaluates to \lstinline|body|.

\begin{lstlisting}[caption=Using \lstinline|do|, label=lst:do,
  float=\floatpos]
  var x = 42;
  do x
  {
    echo(this);
    echo(sqrt);
    setSlot("y", 0);
    42;
  };
  [00000000] 42
\end{lstlisting}

\subsection{for}

\us support the classical \C-like \lstinline|for| construct.

\begin{lstlisting}[frame=, backgroundcolor=, ]
  for (initialization; condition; increment)
    body;
\end{lstlisting}

It has the exact same behavior as \C's \lstinline|for|:

\begin{enumerate}
\item The \lstinline|initialization| is evaluated.
\item The \lstinline|condition| is evaluated. If the result is false,
  executions jumps after the \lstinline|for|.
\item The \lstinline|body| is evaluated. If \lstinline|continue| is
  encountered, execution jumps to point 4. If \lstinline|break| is
  encountered, executions jumps after the \lstinline|for|.
\item The \lstinline|increment| is evaluated.
\item Execution jumps to point 2.
\end{enumerate}

\subsection{for in}

\us support iteration over a collection with another form of the
\lstinline|for| loop.

\begin{lstlisting}[frame=, backgroundcolor=, ]
  for (var name in collection)
    action;
\end{lstlisting}

It evaluates \lstinline|action| for each element in
\lstinline|collection|. Inside \lstinline|action|, the current element
is accessible via the \lstinline|name| local variable. \autoref{lst:foreach}
illustrates this.


\begin{lstlisting}[caption=Iterating over a collection with for, label=lst:foreach,
  float=\floatpos]
  for (var x in [0, 1, 2, 3, 4])
    echo(0.sqr);
  [00000000] *** 0
  [00000000] *** 1
  [00000000] *** 4
  [00000000] *** 9
  [00000000] *** 16
\end{lstlisting}

This form of \lstinline|for| simply sends the ``each'' message to
\lstinline|collection| with one argument: the function that takes the
current element and performs \lstinline|action| over it. Thus, you can
make any object acceptable in a \lstinline|for| by defining an
adequate \lstinline|each| method.

\begin{lstlisting}
  var Hobbits = Object.new;
  function Hobbits.each (action)
  {
    action("Frodo");
    action("Merry");
    action("Pippin");
    action("Sam");
  };

  for (var name in Hobbits)
    echo("%s is a hobbit." % name);
  [00000000] Frodo is a hobbit
  [00000000] Merry is a hobbit
  [00000000] Pipin is a hobbit
  [00000000] Sam is a hobbit
  // This for statement is equivalent to:
  Hobbits.each(function (name) { echo("%s is a hobbit." % name) });
\end{lstlisting}

\subsection{if}

\us support classical \lstinline|if| construct.

\begin{lstlisting}[frame=, backgroundcolor=, ]
  if (condition)
    action;

  if (condition)
    action
  else
    otherwise;
\end{lstlisting}

If the \lstinline|condition| evaluation is true, \lstinline|action| is
evaluated. Otherwise, in the latter version, \lstinline|otherwise| is
executed.

\subsection{switch}

The \lstinline|switch| statement in \us is similiar to \C's one.

\begin{lstlisting}[frame=, backgroundcolor=, ]
  switch (value)
  {
    case value_one:
      action_one;
    case value_two:
      action_two;
  //case ...:
  //  ...
    default:
      default_action;
  };
\end{lstlisting}

It might contain an arbitrary number of cases, and optionally a
default case. The \lstinline|value| is evaluated first, and then the
result is compared sequentially with the evaluation of all cases
values, with the \lstinline|==| operator, until one comparison is
true. If such a match is found, the corresponding action is executed,
and execution jumps after the \lstinline|switch|. Otherwise, the
default case - if any - is executed, and execution jumps after the
switch. The switch itself evaluates to case that was evaluated, or to
void if no match was found and there's no default case. \autoref{lst:switch}
illustrates \lstinline|switch| usage.

Note that, unlike in \C, no \lstinline|break| are required inside the
\lstinline|switch|: execution will never span over several cases. Also
note that, since the comparison is done with the generic '=='
operator, switch can be performed on any comparable data type, unlike
in \C where it need be integral.

\begin{lstlisting}[caption=The \lstinline|switch| construct,
  label=lst:switch, float=\floatpos]
  function sw(v)
  {
    switch (v)
    {
      case "":
        echo("Empty string");
      case "foo":
        "bar";
      default:
        v[0];
    }
  };
  sw("");
  [00000000] *** Empty string
  sw("foo");
  [00000000] "bar"
  sw("foobar");
  [00000000] "f"
\end{lstlisting}

\subsection{while}

The \lstinline|while| loop is similar to \C's one.

\begin{lstlisting}[frame=, backgroundcolor=, ]
  while (condition)
    body;
\end{lstlisting}

If \lstinline|condition| evaluation, is true, \lstinline|body| is
evaluated and execution jumps before the \lstinline|while|, otherwise execution
jumps after the \lstinline|while|. See \autoref{lst:while}.

\begin{lstlisting}[caption=The \lstinline|while| construct, label=lst:while,
  float=\floatpos]
  var i = 5;
  while (i > 0)
  {
    echo(i);
    i--;
  };
  [00000000] *** 5
  [00000000] *** 4
  [00000000] *** 3
  [00000000] *** 2
  [00000000] *** 1
\end{lstlisting}

\FloatBarrier
\section{Parallel and event-based flow control}

\subsection{at}
\subsection{every}
\subsection{for\& (:)}
\subsection{for\& (n)}
\subsection{waituntil}
\subsection{whenever}

\FloatBarrier
\section{Pattern matching}
\FloatBarrier
\section{Trajectories}

% Local Variables:
%%% mode: latex
%%% TeX-master: "urbi-specs"
%%% End:
