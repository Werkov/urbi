%% Copyright (C) 2008-2011, Gostai S.A.S.
%%
%% This software is provided "as is" without warranty of any kind,
%% either expressed or implied, including but not limited to the
%% implied warranties of fitness for a particular purpose.
%%
%% See the LICENSE file for more information.

\chapter{\us Language Reference Manual}
\label{sec:lang}

\section{Syntax}

\subsection{Characters, encoding}
\index{encoding}
\index{ASCII}
\index{UTF-8}

Currently \us makes no assumptions about the encoding used in the
programs, but the streams are handled as 8-bit characters.

While you are allowed to use whatever character you want in the string
literals (especially using the binary escapes,
\autoref{sec:us-syn-lit-string}), only plain ASCII characters are
allowed in the program body.  Invalid characters are reported,
possibly escaped if they are not ``printable''.  If you enter UTF-8
characters, since they possibly span over several 8-bit characters, a
single (UTF-8) character may be reported as several invalid (8-bit)
characters.

%% UTF-8 is not supported by lstlisting, we need to escape to TeX.
%% TeX4ht produces ugly results when using lstnewenvironment.  Worse,
%% here it creates a new <pre> on each side of the escape characters.
%% So let's hope Eitan fixes this some day.
\begin{urbiscript}[firstnumber=1,escapeinside=<>]
#<Été>;
[00048238:error] !!! syntax error: invalid character: `#'
[00048239:error] !!! syntax error: invalid character: `\xc3'
[00048239:error] !!! syntax error: invalid character: `\x89'
[00048239:error] !!! syntax error: invalid character: `\xc3'
[00048239:error] !!! syntax error: invalid character: `\xa9'
\end{urbiscript}

\subsection{Comments}

\dfn{Comments} are used to document the code, they are ignored by the
\us interpreter. Both \Cxx comment types are supported.

\begin{itemize}
\item A \lstinline|//| introduces a comment that lasts until the end
  of the line.
\item A \lstinline|/*| introduces a comment that lasts until \lstinline|*/|
  is encountered. Comments nest, contrary to \C/\Cxx: if two \lstinline|/*|
  are encountered, the comment will end after two \lstinline|*/|, not one.
\end{itemize}

\begin{urbiscript}
1; // This is a one line comment.
[00000001] 1

2; /* an inner comment */ 3;
[00000002] 2
[00000003] 3

4; /* nested /* comments */ 5; */ 6;
[00000004] 4
[00000005] 6

7
  /*
    /*
       Multi-line.
    */
  */
;
[00000006] 7
\end{urbiscript}

\subsection{Synclines}
\label{sec:specs:synclines}

While the interaction with an \us kernel is usually performed via a
network connection, programmers are used to work with files which have
names, line numbers and so forth.  This is most important in error
messages.  Since even loading a file actually means sending its
content as if it were typed in the network session, in order to
provide the user with meaningful locations in error messages, \us
features \dfn[syncline]{synclines}, a means to change the ``current
location'', similarly to \lstinline[language=C]|#line| in \C-like
languages.  This is achieved using special \lstinline|//#| comments.

The following special comments are recognized only as a whole line.
If some component does not match exactly the expected syntax, or if
there are trailing items, the whole line is treated as a comment.
\begin{itemize}
%% I failed to use \var for line and file here.  -- AD.
\item \lstinline|//#line line "file"|\\
  Specify that the \emph{next} line is from the file named \var{file},
  and which line number is \var{line}.  The current location (i.e.,
  current file and line) is lost.

\item \lstinline|//#push line "file"|\\
  Save the current location, and then behave as if \lstinline|//#line|
  was used.

\item \lstinline|//#pop|\\
  Restore the current location.  \lstinline|//#push| and
  \lstinline|//#pop| must match.
\end{itemize}


\subsection{Identifiers}
\label{sec:us-syn-id}

\dfn{Identifiers} in \us are composed of one or more alphanumeric or
underscore (\lstinline|_|) characters, not starting by a digit, i.e.,
identifiers match the \lstinline|[a-zA-Z_][a-zA-Z0-9_]*| regular
expression.  Additionally, identifiers must not match any of the \us
reserved words\footnote{
%%
  The only exception to this rule is \lstinline|new|, which can be
  used as the method identifier in a method call.
%%
} documented in \autoref{sec:syn-key}. Identifiers can also be written
between simple quotes (\lstinline|'|), in which case they may contain
any character.

\begin{urbiscript}
var x;
var foobar51;
var this.a_name_with_underscores;
// Invalid.
// var 3x;
// obj.3x();

// Invalid because "if" is a keyword.
// var if;
// obj.if();
// However, keywords can be escaped with simple quotes.
var 'if';
var this.'else';

// Identifiers can be escaped with simple quotes
var '%x';
var '1 2 3';
var this.'[]';
\end{urbiscript}

\subsection{Keywords}
\label{sec:syn-key}

\dfn{Keywords} are reserved words that cannot be used as identifiers,
for instance.  They are listed in \autoref{tab:keywords}.

\renewcommand{\baselinestretch}{.85}
\begin{table}[\floatpos]
  \centering
  \ifx\ifHtml\undefined
\begin{tabular}{|c|l||c|l|}
  \hline
  Keyword & Remark & Keyword & Remark \\
  \hline
     \lstinline"and" & Synonym for \lstinline|&&|               &
  \lstinline"and_eq" & Synonym for \lstinline|&=|               \\
     \lstinline"asm" & Reserved                                 &
  \lstinline"assert" & \autoref{sec:lang:assert}                \\
      \lstinline"at" & \autoref{sec:lang:at}                    &
    \lstinline"auto" & Reserved                                 \\
  \lstinline"bitand" & \autoref{sec:lang:op:bitwise}            &
   \lstinline"bitor" & \autoref{sec:lang:op:bitwise}            \\
    \lstinline"bool" & Reserved                                 &
   \lstinline"break" & \autoref{sec:lang:break}                 \\
    \lstinline"call" & \autoref{sec:lang:rv}                    &
    \lstinline"case" & \autoref{sec:lang:switch}                \\
   \lstinline"catch" & \autoref{sec:lang:catch}, \autoref{sec:lang:timeout} &
    \lstinline"char" & Reserved                                 \\
   \lstinline"class" & \autoref{sec:lang:class}                 &
 \lstinline"closure" & \autoref{sec:lang:closure}               \\
   \lstinline"compl" & \autoref{sec:lang:op:bitwise}            &
   \lstinline"const" & \autoref{sec:lang:prop:std}              \\
\lstinline"const_cast" & Reserved                                 &
\lstinline"continue" & \autoref{sec:lang:continue}              \\
 \lstinline"default" & \autoref{sec:lang:switch}                &
  \lstinline"delete" & Reserved                                 \\
  \lstinline"detach" &                                          &
  \lstinline"disown" &                                          \\
      \lstinline"do" & \autoref{sec:lang:do}                    &
  \lstinline"double" & Reserved                                 \\
\lstinline"dynamic_cast" & Reserved                                 &
    \lstinline"else" & \autoref{sec:lang:if}                    \\
    \lstinline"enum" & \autoref{sec:lang:continue}              &
   \lstinline"every" & \autoref{sec:lang:every}                 \\
\lstinline"explicit" & Reserved                                 &
  \lstinline"export" & Reserved                                 \\
  \lstinline"extern" & Reserved                                 &
\lstinline"external" & Reserved                                 \\
   \lstinline"float" & Reserved                                 &
     \lstinline"for" & \lstinline|for&| and \lstinline-for|- flavors \\
 \lstinline"foreach" & Deprecated, use \lstinline|for|          &
\lstinline"freezeif" &                                          \\
  \lstinline"friend" & Reserved                                 &
\lstinline"function" & \autoref{sec:lang:function}, \autoref{sec:lang:closure} \\
    \lstinline"goto" & Reserved                                 &
      \lstinline"if" & \autoref{sec:lang:if}                    \\
      \lstinline"in" & \autoref{sec:lang:foreach}               &
  \lstinline"inline" & Reserved                                 \\
     \lstinline"int" & Reserved                                 &
\lstinline"internal" & Deprecated                               \\
    \lstinline"long" & Reserved                                 &
    \lstinline"loop" & \lstinline|loop&|, \lstinline-loop|-, \autoref{sec:lang:loop} \\
 \lstinline"mutable" & Reserved                                 &
\lstinline"namespace" & Reserved                                 \\
     \lstinline"new" & \autoref{sec:tut:ctor}                   &
     \lstinline"not" & Synonym for \lstinline|!|                \\
  \lstinline"not_eq" & Synonym for \lstinline|!=|               &
 \lstinline"onleave" & \autoref{sec:lang:at}                    \\
      \lstinline"or" & Synonym for \lstinline-||-               &
   \lstinline"or_eq" & Synonym for \lstinline-|=-               \\
 \lstinline"private" & Ignored                                  &
\lstinline"protected" & Ignored                                  \\
  \lstinline"public" & Ignored                                  &
\lstinline"register" & Reserved                                 \\
\lstinline"reinterpret_cast" & Reserved                                 &
  \lstinline"return" & \autoref{sec:lang:rv}                    \\
   \lstinline"short" & Reserved                                 &
  \lstinline"signed" & Reserved                                 \\
  \lstinline"sizeof" & Reserved                                 &
  \lstinline"static" & Deprecated                               \\
\lstinline"static_cast" & Reserved                                 &
  \lstinline"stopif" &                                          \\
  \lstinline"struct" & Reserved                                 &
  \lstinline"switch" & \autoref{sec:lang:switch}                \\
\lstinline"template" & Reserved                                 &
    \lstinline"this" &                                          \\
   \lstinline"throw" & \autoref{sec:lang:throw}                 &
 \lstinline"timeout" & \autoref{sec:lang:timeout}               \\
     \lstinline"try" & \autoref{sec:lang:catch}                 &
 \lstinline"typedef" & Reserved                                 \\
  \lstinline"typeid" & Reserved                                 &
\lstinline"typename" & Reserved                                 \\
   \lstinline"union" & Reserved                                 &
\lstinline"unsigned" & Reserved                                 \\
   \lstinline"using" & Reserved                                 &
     \lstinline"var" & \autoref{sec:lang:var}, \autoref{sec:lang:slots} \\
 \lstinline"virtual" & Reserved                                 &
\lstinline"volatile" & Reserved                                 \\
\lstinline"waituntil" & \autoref{sec:lang:waituntil}             &
   \lstinline"watch" & \autoref{sec:lang:watch}                 \\
 \lstinline"wchar_t" & Reserved                                 &
\lstinline"whenever" & \autoref{sec:lang:whenever}              \\
   \lstinline"while" & \lstinline|while&| and \lstinline-while|- flavors &
     \lstinline"xor" & Synonym for \lstinline|^|                \\
  \lstinline"xor_eq" & Synonym \lstinline|^=|                   &
\\
  \hline
\end{tabular}
\else
\begin{tabular}{|c|l|}
  \hline
  Keyword & Remark \\
  \hline
     \lstinline"and" & Synonym for \verb|&&|                    \\
  \lstinline"and_eq" & Synonym for \verb|&=|                    \\
     \lstinline"asm" & Reserved                                 \\
  \lstinline"assert" & \autoref{sec:lang:assert}                \\
      \lstinline"at" & \autoref{sec:lang:at}                    \\
    \lstinline"auto" & Reserved                                 \\
  \lstinline"bitand" & \autoref{sec:lang:op:bitwise}            \\
   \lstinline"bitor" & \autoref{sec:lang:op:bitwise}            \\
    \lstinline"bool" & Reserved                                 \\
   \lstinline"break" & \autoref{sec:lang:break}                 \\
    \lstinline"call" & \autoref{sec:lang:rv}                    \\
    \lstinline"case" & \autoref{sec:lang:switch}                \\
   \lstinline"catch" & \autoref{sec:lang:catch}, \autoref{sec:lang:timeout} \\
    \lstinline"char" & Reserved                                 \\
   \lstinline"class" & \autoref{sec:lang:class}                 \\
 \lstinline"closure" & \autoref{sec:lang:closure}               \\
   \lstinline"compl" & \autoref{sec:lang:op:bitwise}            \\
   \lstinline"const" & \autoref{sec:lang:prop:std}              \\
\lstinline"const_cast" & Reserved                                 \\
\lstinline"continue" & \autoref{sec:lang:continue}              \\
 \lstinline"default" & \autoref{sec:lang:switch}                \\
  \lstinline"delete" & Reserved                                 \\
  \lstinline"detach" &                                          \\
  \lstinline"disown" &                                          \\
      \lstinline"do" & \autoref{sec:lang:do}                    \\
  \lstinline"double" & Reserved                                 \\
\lstinline"dynamic_cast" & Reserved                                 \\
    \lstinline"else" & \autoref{sec:lang:if}                    \\
    \lstinline"enum" & \autoref{sec:lang:continue}              \\
   \lstinline"every" & \autoref{sec:lang:every}                 \\
\lstinline"explicit" & Reserved                                 \\
  \lstinline"export" & Reserved                                 \\
  \lstinline"extern" & Reserved                                 \\
\lstinline"external" & Reserved                                 \\
   \lstinline"float" & Reserved                                 \\
     \lstinline"for" & \verb|for&| and \verb-for|- flavors      \\
 \lstinline"foreach" & Deprecated, use \verb|for|               \\
\lstinline"freezeif" &                                          \\
  \lstinline"friend" & Reserved                                 \\
\lstinline"function" & \autoref{sec:lang:function}, \autoref{sec:lang:closure} \\
    \lstinline"goto" & Reserved                                 \\
      \lstinline"if" & \autoref{sec:lang:if}                    \\
      \lstinline"in" & \autoref{sec:lang:foreach}               \\
  \lstinline"inline" & Reserved                                 \\
     \lstinline"int" & Reserved                                 \\
\lstinline"internal" & Deprecated                               \\
    \lstinline"long" & Reserved                                 \\
    \lstinline"loop" & \verb|loop&|, \verb-loop|-, \autoref{sec:lang:loop} \\
 \lstinline"mutable" & Reserved                                 \\
\lstinline"namespace" & Reserved                                 \\
     \lstinline"new" & \autoref{sec:tut:ctor}                   \\
     \lstinline"not" & Synonym for \verb|!|                     \\
  \lstinline"not_eq" & Synonym for \verb|!=|                    \\
 \lstinline"onleave" & \autoref{sec:lang:at}                    \\
      \lstinline"or" & Synonym for \verb-||-                    \\
   \lstinline"or_eq" & Synonym for \verb-|=-                    \\
 \lstinline"private" & Ignored                                  \\
\lstinline"protected" & Ignored                                  \\
  \lstinline"public" & Ignored                                  \\
\lstinline"register" & Reserved                                 \\
\lstinline"reinterpret_cast" & Reserved                                 \\
  \lstinline"return" & \autoref{sec:lang:rv}                    \\
   \lstinline"short" & Reserved                                 \\
  \lstinline"signed" & Reserved                                 \\
  \lstinline"sizeof" & Reserved                                 \\
  \lstinline"static" & Deprecated                               \\
\lstinline"static_cast" & Reserved                                 \\
  \lstinline"stopif" &                                          \\
  \lstinline"struct" & Reserved                                 \\
  \lstinline"switch" & \autoref{sec:lang:switch}                \\
\lstinline"template" & Reserved                                 \\
    \lstinline"this" &                                          \\
   \lstinline"throw" & \autoref{sec:lang:throw}                 \\
 \lstinline"timeout" & \autoref{sec:lang:timeout}               \\
     \lstinline"try" & \autoref{sec:lang:catch}                 \\
 \lstinline"typedef" & Reserved                                 \\
  \lstinline"typeid" & Reserved                                 \\
\lstinline"typename" & Reserved                                 \\
   \lstinline"union" & Reserved                                 \\
\lstinline"unsigned" & Reserved                                 \\
   \lstinline"using" & Reserved                                 \\
     \lstinline"var" & \autoref{sec:lang:var}, \autoref{sec:lang:slots} \\
 \lstinline"virtual" & Reserved                                 \\
\lstinline"volatile" & Reserved                                 \\
\lstinline"waituntil" & \autoref{sec:lang:waituntil}             \\
   \lstinline"watch" & \autoref{sec:lang:watch}                 \\
 \lstinline"wchar_t" & Reserved                                 \\
\lstinline"whenever" & \autoref{sec:lang:whenever}              \\
   \lstinline"while" & \verb|while&| and \verb-while|- flavors  \\
     \lstinline"xor" & Synonym for \verb|^|                     \\
  \lstinline"xor_eq" & Synonym \verb|^=|                        \\
  \hline
\end{tabular}
\fi

  \caption{Keywords}
  \label{tab:keywords}
\end{table}
\renewcommand{\baselinestretch}{1}

\subsection{Literals}

\subsubsection{Angles}

\dfn[angle]{Angles} are floats (see \autoref{sec:us-syn-lit-float})
followed by an angle unit. They are simply equivalent to the same
float, expressed in radians. For instance, \lstinline|180deg| (180
degrees) is equal to \lstinline|pi|. Available units and their
equivalent are presented in \autoref{tab:angle}.

\begin{table}[\floatposh]
  \centering
  \begin{tabular}{|c|c|c|}
    \hline
    unit        & abbreviation & equivalence for $n$  \\
    \hline
    radian      & rad          & $n$         \\
    degree      & deg          & $n / 180 * \pi$        \\
    grad        & grad         & $n / 200 * \pi$        \\
    \hline
  \end{tabular}
  \caption{Angle units}
  \label{tab:angle}
\end{table}

\begin{urbiassert}
pi == 180deg;
pi == 200grad;
\end{urbiassert}

\subsubsection{Dictionaries}
\label{sec:us-syn-lit-dictionary}

Literal \dfn{dictionaries} are represented with a comma-separated,
potentially empty list of arbitrary associations enclosed in square brackets
(\lstinline|[]|), as shown in the listing below.  Empty dictionaries are
represented with an association arrow between the brackets to avoid
confusion with empty lists.  See \refObject{Dictionary} for more details.

Each association is composed of a key, which is represented by a string, an
arrow (\lstinline|=>|) and an expression.

\begin{urbiscript}
[ => ]; // The empty dictionary
[00000000] [ => ]
["a" => 1, "b" => 2, "c" => 3];
[00000000] ["a" => 1, "b" => 2, "c" => 3]
\end{urbiscript}

\subsubsection{Durations}

\dfn{Durations} are floats (see \autoref{sec:us-syn-lit-float})
followed by a time unit. They are simply equivalent to the same float,
expressed in seconds. For instance, \lstinline|1s 1ms|, which stands
for ``one second and one millisecond'', is strictly equivalent to
\lstinline|1.0001|. Available units and their equivalent are presented
in \autoref{tab:duration}.

\begin{table}
  \centering
  \begin{tabular}{|c|c|c|}
    \hline
    unit        & abbreviation & equivalence for $n$  \\
    \hline
    millisecond & ms           & $n / 1000$         \\
    second      & s            & $n$                \\
    minute      & min          & $n \times 60$           \\
    hour        & h            & $n \times 60 \times 60$      \\
    day         & d            & $n \times 60 \times 60 \times 24$ \\
    \hline
  \end{tabular}
  \caption{Duration units}
  \label{tab:duration}
\end{table}

\begin{urbiassert}
1d   == 24h;
0.5d == 12h;
1h   == 60min;
1min == 60s;
1s   == 1000ms;


1s == 1;
1s 2s 3s == 6;
1s 1ms == 1.001;
1ms 1s == 1.001;
\end{urbiassert}

\subsubsection{Floats}
\label{sec:us-syn-lit-float}

\us supports the \dfn{scientific notation} for floating-point
literals.  See \refObject{Float} for more details.  Examples include:

%% FIXME: On windows, does not pass.      0.000001 == 1e-06;
%% FIXME: On windows, does not pass.     0.0000001 == 1e-07;
%% FIXME: On windows, does not pass. 0.00000000001 == 1e-11;

\begin{urbiassert}
            1 == 1;
            1 == 1.0;
          1.2 == 1.2000;
      1.234e6 == 1234000;
        1e+11 == 1E+11;
         1e10 == 10000000000;
         1e30 == 1e10 * 1e10 * 1e10;
\end{urbiassert}

Numbers are displayed rounded by the top level, but internally, as
seen above, they keep their accurate value.

%% Don't try 1E-4: it is not portably output equally.  On OS X it
%% is output as 0.0001, but other architectures output 1e-4.  And
%% they seem to be right according to
%% http://www.opengroup.org/onlinepubs/007908799/xsh/fprintf.html
%%
%%  g, G
%%
%%  The double argument is converted in the style f or e (or in the style
%%  E in the case of a G conversion character), with the precision
%%  specifying the number of significant digits. If an explicit precision
%%  is 0, it is taken as 1. The style used depends on the value converted;
%%  style e (or E) will be used only if the exponent resulting from such a
%%  conversion is less than -4 or greater than or equal to the
%%  precision. Trailing zeros are removed from the fractional portion of
%%  the result; a radix character appears only if it is followed by a
%%  digit. The fprintf() family of functions may make available character
%%  string representations for infinity and NaN.
%%
\begin{urbiscript}
0.000001;
[00000011] 1e-06

0.0000001;
[00000012] 1e-07

0.00000000001;
[00000013] 1e-11

1e+3;
[00000014] 1000

1E-5;
[00000014] 1e-05
\end{urbiscript}

In order to tell apart numbers with units (\samp{1min}) and calling a method
on a number (\samp{1.min}), numbers that include a period must have a
fractional part.  In other words, \samp{1.}, if not followed by digits, is
always read as \samp{1 .}:

\begin{urbiscript}
1.;
[00004701:error] !!! syntax error: unexpected ;
\end{urbiscript}

Hexadecimal notation is supported for integers: \lstinline|0x| followed by
one or more hexadecimal digits, whose case is irrelevant.

\begin{urbiassert}
      0x2a == 42;
      0x2A == 42;
  0xabcdef == 11259375;
  0xABCDEF == 11259375;
0xFFFFFFFF == 4294967295;
\end{urbiassert}

Numbers with unknown suffixes are invalid tokens:

\begin{urbiscript}
123foo;
[00005658:error] !!! syntax error: invalid token: `123foo'
12.3foo;
[00018827:error] !!! syntax error: invalid token: `12.3foo'
0xabcdef;
[00060432] 11259375
0xabcdefg;
[00061848:error] !!! syntax error: invalid token: `0xabcdefg'
\end{urbiscript}

In order to augment readability, you may separate digits with underscores.
\begin{urbiassert}
   123_456_789 == 123456789;
12_34_56_78_90 == 1234567890;
   1_2__3___45 == 12345;
      1_2.3__4 == 12.34;
   0xFFFF_FFFF == 0xFFFFFFFF;
         1e1_0 == 1e10;
\end{urbiassert}

The underscores must always be \emph{between} digits.
\begin{urbiscript}
// This is actually an identifier.
_0__2;
[00000112:error] !!! lookup failed: _0__2

20_;
[00029616:error] !!! syntax error: invalid token: `20_'

// This is actually a call to the method `_2' in 1.
1._2;
[00029653:error] !!! lookup failed: _2

1.2_;
[00029654:error] !!! syntax error: invalid token: `1.2_'

0x_12;
[00116580:error] !!! syntax error: invalid token: `0x_12'
0_x12;
[00116582:error] !!! syntax error: invalid token: `0_x12'

1e_10;
[00253832:error] !!! syntax error: invalid token: `1e_10'
1e10_;
[00257727:error] !!! syntax error: invalid token: `1e10_'
\end{urbiscript}

\subsubsection{Lists}
\label{sec:us-syn-lit-list}

Literal \dfn{lists} are represented with a comma-separated, potentially
empty list of arbitrary expressions enclosed in square brackets
(\lstinline|[]|), as shown in the listing below.  See
\refObject{List} for more details.

\begin{urbiscript}
[]; // The empty list
[00000000] []
[1, 2, 3];
[00000000] [1, 2, 3]
\end{urbiscript}

\subsubsection{Strings}
\label{sec:us-syn-lit-string}

\dfn{String} literals are enclosed in double quotes (\lstinline|"|)
and can contain arbitrary characters, which stand for themselves, with
the exception of the escape character, backslash (\lstinline|\|), see
below.  The escapes sequences are defined in \autoref{tab:escapes}.

\begin{table}[\floatposh]
  \centering
  \begin{tabular}{|c|p{.6\linewidth}|}
    \hline
    \lstinline|\\| & backslash             \\
    \lstinline|\"| & double-quote          \\
    \lstinline|\a| & bell ring             \\
    \lstinline|\b| & backspace             \\
    \lstinline|\f| & form feed             \\
    \lstinline|\n| & line feed             \\
    \lstinline|\r| & carriage return       \\
    \lstinline|\t| & tabulation            \\
    \lstinline|\v| & vertical tabulation   \\

    \lstinline|\[0-7]{1,3}|
    & eight-bit character corresponding to a one-, two- or three-digit
    octal number.  For instance, \lstinline|\0|, \lstinline|\000| and
    \lstinline|177|.  The matching is greedy: as many digits as
    possible are taken: \lstinline|\0|, \lstinline|\000| are both
    resolved in the null character.
    \\

    \lstinline|\x[0-9a-fA-F]{2}|
    & eight-bit character corresponding to a two-digit hexadecimal
    number.  For instance, \lstinline|0xfF|. \\

    \lstinline|\B(\var{length})(\var{content})|
    & binary blob.  A \var{length}-long sequence of verbatim
    \var{content}.  \var{length} is expressed in decimal.  \var{content}
    is not interpreted in any way.  The parentheses are part of the syntax,
    they are mandatory.  For instance \lstinline|\B(2)(\B)|\\
    \hline
  \end{tabular}
  \caption{String escapes}
  \label{tab:escapes}
\end{table}

\begin{urbiassert}
// Special characters.
"\"" == "\"";
"\\" == "\\";

// ASCII characters.
"\a" == "\007"; "\a" == "\x07";
"\b" == "\010"; "\b" == "\x08";
"\f" == "\014"; "\f" == "\x0c";
"\n" == "\012"; "\n" == "\x0a";
"\r" == "\015"; "\r" == "\x0d";
"\t" == "\011"; "\t" == "\x09";
"\v" == "\013"; "\v" == "\x0b";

// Octal escapes.
"\0" == "\00"; "\0" == "\000";
"\0000" == "\0""0";
"\062\063" == "23";

// Hexadecimal escapes.
"\x00" == "\0";
"\x32\x33" == "23";

// Binary blob escape.
"\B(3)("\")" == "\"\\\"";
\end{urbiassert}

Consecutive string literals are glued together into a single string.
This is useful to split large strings into chunks that fit usual
programming widths.

\begin{urbiassert}
"foo" "bar" "baz" == "foobarbaz";
\end{urbiassert}

The interpreter prints the strings escaped; for instance, line feed
will be printed out as \lstinline|\n| when a string result is dumped
and so forth. An actual line feed will of course be output if a string
content is printed with echo for instance.

\begin{urbiscript}
"";
[00000000] ""
"foo";
[00000000] "foo"
"a\nb"; // urbiscript escapes string when dumping them
[00000000] "a\nb"
echo("a\nb"); // We can see there is an actual line feed
[00000000] *** a
[:]b
echo("a\\nb");
[00000000] *** a\nb
\end{urbiscript}

See \refObject{String} for more details.

\subsubsection{Tuples}
\label{sec:us-syn-lit-tuples}

Literal \dfn{tuples} are represented with a comma-separated, potentially
empty list of arbitrary elements enclosed in parenthesis (\lstinline|()|),
as shown in the listing below.  One extra comma can be added after the last
element.  To avoid confusion between a 1 member \lstinline|Tuple| and a
parenthesized expression, the extra comma must be added.  See
\refObject{Tuple} for more details.

\begin{urbiscript}
();
[00000000] ()
(1,);
[00000000] (1,)
(1, 2);
[00000000] (1, 2)
(1, 2, 3, 4,);
[00000000] (1, 2, 3, 4)
\end{urbiscript}


\subsubsection{Pseudo classes}
\label{sec:lang:class}

Objects meant to serve as prototypes are best defined using the
\lstindex{class} construct.  See also the tutorial, \autoref{sec:tut:class}.

\begin{bnf}
<class-statement>
  ::= "class" <lvalue> (":" <prototypes>)? <block>

<lvalue>
  ::= (<expression> ".")* "identifier"

<prototypes>
  ::= (<expression> ",")* expression

<block>
  ::= "{" <statement>* "}"
\end{bnf}

This results in the (constant) definition of the name \var{lvalue} in the
current context (\lstinline|class| construct can be used inside a scope or
in an object) with:
\begin{itemize}
\item a slot named \lstinline|type| which is a the trailing component of
  \var{lvalue} as a \refObject{String};
\item a slot named \lstinline|as\var{type}| that returns \this.
\item the list of prototypes is equal to the list \var{prototypes} that
  served as parent, defaulting to \refObject{Object};
\item the \var{block} is evaluated in the context of this object, as with a
  \lstinline|do|-block;
\item the value of the whole statement is the newly defined object.
\end{itemize}

\begin{urbiscript}
class Base
{
  var slot = 12;
}|;

assert
{
  hasLocalSlot("Base");
  Base.type   == "Base";
  Base.protos == [Object];
  Base.slot   == 12;
  Base.asBase === Base;
};

class Global.Derive : Base
{
  var slot = 34;
}|;

assert
{
  Global.hasLocalSlot("Derive");
  Global.Derive.type     == "Derive";
  Global.Derive.protos   == [Base];
  Global.Derive.slot     == 34;
  Global.Derive.asDerive === Global.Derive;
  Global.Derive.asBase   === Global.Derive;
};

class Base2 {}|;

class Derive2 : Base, Base2 {}|;

assert
{
  Derive2.type      == "Derive2";
  Derive2.protos    == [Base, Base2];
  Derive2.slot      == 12;
  Derive2.asDerive2 === Derive2;
  Derive2.asBase    === Derive2;
  Derive2.asBase2   === Derive2;
};
\end{urbiscript}

It is guaranteed that the expressions that define the class name and its
parents are evaluated only once.

\begin{urbiscript}
function Global.verboseId(var x)
{
  echo(x) | x
}|;
class verboseId(Global).math : verboseId(Math)
{
};
[00000686] *** Global
[00000686] *** Math
[00000686] math
\end{urbiscript}


\subsection{Statement Separators}
\label{sec:lang:separators}

Sequential languages such as \Cxx support a single way to compose two
statements: the sequential composition, ``denoted'' by \samp{;}.  To
support concurrency and more fined tuned sequentiality, \us features
four different statement-separators (or connectors):
\begin{description}
\item[\samp{;}] sequentiality
\item[\samp{|}] tight sequentiality
\item[\samp{,}] background concurrency
\item[\samp{\&}] fair-start concurrency
\end{description}

\subsubsection{\samp{;}}

The \samp{;}-connector waits for the first statement to finish before
starting the second statement.  When used in the top-level interactive
session, both results are displayed.

\begin{urbiscript}
1; 2; 3;
[00000000] 1
[00000000] 2
[00000000] 3
\end{urbiscript}

\subsubsection{\samp{,}}

The \samp{,}-connector sends the first statement in background for
concurrent execution, and starts the second statement when possible.
When used in interactive sessions, the value of back-grounded
statements are \emph{not} printed --- the time of their arrival being
unpredictable, such results would clutter the output randomly.  Use
\refObject[s]{Channel} or \refObject[s]{Event} to return results
asynchronously.

\begin{urbiscript}
{
  for (3)
  {
    sleep(1s);
    echo("ping");
  },
  sleep(0.5s);
  for (3)
  {
    sleep(1s);
    echo("pong");
  },
};
[00000316] *** ping
[00000316] *** pong
[00000316] *** ping
[00000316] *** pong
[00000316] *** ping
[00000316] *** pong
\end{urbiscript}

Both \samp{;} and \samp{,} have equal precedence.  They are scoped too: the
execution follow ``waits'' for the end of the jobs back-grounded with
\samp{,} before proceeding.  Compare the two following executions.

\begin{urbiscript}
{
  sleep(100ms) | echo("1"),
  sleep(400ms) | echo("2"),
  echo("done");
};
[00000316] *** done
[00000316] *** 1
[00000316] *** 2
\end{urbiscript}

\begin{urbiscript}
{
  sleep(100ms) | echo("1"),
  sleep(400ms) | echo("2"),
};
echo("done");
[00000316] *** 1
[00000316] *** 2
[00000316] *** done
\end{urbiscript}


\subsubsection{\samp{|}}
When using the \samp{;} connector, the scheduler is allowed to run
other commands between the first and the second statement.  The
\samp{|} does not yield between both statements.  It is therefore more
efficient, and, in a way, provides some atomicity for concurrent tasks.

\begin{urbiscript}
{
  { echo("11") ; sleep(100ms) ; echo("12") },
  { echo("21") ; sleep(400ms) ; echo("22") },
};
[00000316] *** 11
[00000316] *** 21
[00000316] *** 12
[00000316] *** 22
\end{urbiscript}

%% Cannot use sleep here, as it yields, which makes the point moot.
\begin{urbiscript}
{
  { echo("11") | echo("12") },
  { echo("21") | echo("22") },
};
[00000316] *** 11
[00000316] *** 12
[00000316] *** 21
[00000316] *** 22
\end{urbiscript}

In an interactive session, both statements must be ``known'' before
launching the sequence.  The value of the composed statement is the value of
the second statement.

\subsubsection{\samp{\&}}

The \samp{\&} is very similar to the \samp{,} connector, but for its
precedence.  \urbi expects to process the whole statement before launching
the connected statements.  This is especially handy in interactive sessions,
as a means to fire a set of tasks concurrently.


\subsection{Operators}

\us supports many \dfn{operators}, most of which are inspired from
\Cxx. Their syntax is presented here, and they are sorted and described with
their original semantics --- that is, \lstinline|+| is an arithmetic
operator that sums two numeric values. However, as in \Cxx, these operators
might be used for any other purpose --- for instance, \lstinline|+| is also
used to concatenate lists and strings.  Their semantics is thus not limited
to what is presented here.

Tables in this section sort operators by decreasing precedence order.  Group
of rows (not separated by horizontal lines) describe operators that have the
same precedence. Many operators are syntactic sugar that bounce on a method
call. In this case, the equivalent desugared expression is shown in the
``Equivalence'' column.  To define an operator for an object, override the
corresponding method (see \autoref{sec:tut:operators}).

This section defines the syntax (what they look like), precedence and
associativity of the operators. Their semantics (what they actually do) is
described in \autoref{sec:stdlib} in the documentation of the classes that
provide them.


% Operator generators.
% \operator {SYMBOL} {SYNTAX} {ASSOCIATIVITY} {NAME} {SEMANTICS}
\newcommand{\operator}[6][]{%
  \lstinline@#2@ & \lstinline@#3@ & #4 & #5 & \lstinline@#6@#1%
}

% \boperator {SYMBOL} {ASSOCIATIVITY} {NAME}
% Binary infix operator.
\newcommand{\boperator}[3]{%
  \operator{#1}{a #1 b}{#2}{#3}{a.'#1'(b)}%
}

% \poperator {SYMBOL} {ASSOCIATIVITY} {NAME}
% Unary postfix operator.
\newcommand{\poperator}[3]{%
  \operator{#1}{#1a}{#2}{#3}{a.'#1'()}%
}

% Container operators.
\newcommand{\operatorin}     {\operator  {in}     {a in b}       {None}     {Membership}            {b.has(a)}          }
\newcommand{\operatornotin}  {\operator  {not in} {a not in b}   {None}     {Non-membership}        {b.hasNot(a)}      }
\newcommand{\operatorsub}    {\operator  {[]}   {a[args]}        {Left}     {Subscript}             {a.'[]'(args)}          }
\newcommand{\operatorsubass} {\operator  {[] =} {a[args] = v}    {Right}     {Subscript assignment}  {a.'[]='(args, v)}      }

% Object operators.
\newcommand{\operatorprop}   {\operator  {->}   {a->b}             {Left}     {Property access}        {getProperty("a", "b")} }
\newcommand{\operatorpropass}{\operator  {->}   {a->b = v}         {Right}     {Property assignment}    {setProperty("a", "b", v)}}
\newcommand{\operatordot}    {\operator  {.}    {a.b}              {Left}     {Message sending}        {Not redefinable}       }
\newcommand{\operatordota}   {\operator  {.}    {a.b(args)}        {Left}     {Message sending}        {Not redefinable}       }
\newcommand{\operatordotand} {\operator  {.\&}   {a.\&b}             {Left}     {Slot access}            {a.getSlot("b")}        }
\newcommand{\operatoruand} {\operator  {\&}   {\&a}             {-}     {Slot access}            {getSlot("a")}        }
\newcommand{\operatorass}[1][]{\operator[#1]
                                         {=}    {a = b}            {Right} {Assignment}             {updateSlot("a", b)}    }

\newcommand{\operatoriass}[1]{\operator  {#1=}  {a #1= b}          {Right} {In place #1}    {a = a.'#1='(b)}            }
\newcommand{\operatorsiass}  {
  \operatoriass{+}\\
  \operatoriass{-}\\
  \operatoriass{*}\\
  \operatoriass{/}\\
  \operatoriass{\%}\\
  \operatoriass{\^}%
  % \operatoriass{\~}
}
\newcommand{\operatorinc}    {\operator  {++}   {a++}  {-}     {Incrementation}           {\{var '$a' = a | a = a.'++' | '$a'\}       }}
\newcommand{\operatordec}    {\operator  {--}   {a--}  {-}     {Decrementation}           {\{var '$a' = a | a = a.'--' | '$a'\}       }}

\newcommand{\operatoruplus}  {\poperator {+}    {-}    {Identity}               }
\newcommand{\operatorumin}   {\poperator {-}    {-}    {Opposite}               }
\newcommand{\operatorexp}    {\boperator {**}   {Right}{Exponentiation}         }
\newcommand{\operatormult}   {\boperator {*}    {Left} {Multiplication}         }
\newcommand{\operatordiv}    {\boperator {/}    {Left} {Division}               }
\newcommand{\operatormod}    {\boperator {\%}   {Left} {Modulo}                 }
\newcommand{\operatorplus}   {\boperator {+}    {Left} {Sum}                    }
\newcommand{\operatorminus}  {\boperator {-}    {Left} {Difference}             }
\newcommand{\operatorlshift} {\boperator {<<}   {Left} {Left bit shift}         }
\newcommand{\operatorrshift} {\boperator {>>}   {Left} {Right bit shift}        }
\newcommand{\operatoreq}     {\boperator {==}   {None} {Equality}               }
\newcommand{\operatorneq}    {\boperator {!=}   {None} {Inequality}             }
\newcommand{\operatorpeq}    {\boperator {===}  {None} {Physical equality}      }
\newcommand{\operatorpneq}   {\boperator {!==}  {None} {Physical inequality}    }
\newcommand{\operatoraeq}    {\boperator {\~=}  {None} {Relative approximate equality} }
\newcommand{\operatoreqaeq}  {\boperator {=~=}  {None} {Absolute approximate equality} }
\newcommand{\operatorinf}    {\boperator {<}    {None} {Less than}              }
\newcommand{\operatorinfeq}  {\boperator {<=}   {None} {Less than or equal to}  }
\newcommand{\operatorsup}    {\boperator {>}    {None} {Greater than}           }
\newcommand{\operatorsupeq}  {\boperator {>=}   {None} {Greater than or equal to}}
% Need a space bw compl and a, cannot use \poperator.
%\newcommand{\operatorbcompl} {\poperator {compl}    {-} {Bitwise complement}   }
\newcommand{\operatorbcompl} {\operator{compl}{compl a}{-}{Bitwise complement}{a.'compl'()}}
\newcommand{\operatorband}   {\boperator {bitand}    {Left} {Bitwise and}   }
\newcommand{\operatorbor}    {\boperator {bitor}    {Left} {Bitwise or}   }
\newcommand{\operatorbxor}   {\boperator {^}    {Left} {Bitwise exclusive or}   }
\newcommand{\operatorneg}    {\poperator {!}    {Left} {Logical negation}       }
\newcommand{\operatorand}    {\boperator {\&\&} {Left} {Logical and}            }
\newcommand{\operatoror}     {\boperator {||}   {Left} {Logical or}             }

\subsubsection{Arithmetic operators}

\us supports classic \dfn{arithmetic operators}, with the classic semantics
on numeric values. See \autoref{tab:arith} and the listing below.

% \whetherHtml{HTML OUTPUT}{NOT-HTML OUTPUT}
% ------------------------------------------
\newcommand{\whetherHtml}[2]{%
  \ifx\ifHtml\undefined%
    #2%
  \else%
    #1%
  \fi%
}

%% \begin{operatorTabular}
%%   1 & 2 & 3 & 4 & 5 \\
%%   \hline
%% \end{operatorTabular}
\newenvironment{operatorTabular}
{
  \begin{tabular}{|*{5}{c|}}
    \hline
    \strong{\whetherHtml{Operator}{Oper.}}
    & \strong{Syntax}
    & \strong{\whetherHtml{Associativity}{Assoc.}}
    & \strong{Semantics}
    & \strong{Equivalence}
    \\
    \hline
  }{%
    \\
    \hline
  \end{tabular}
}

%% \begin{operatorTable}{tab:foo}{Foo Operators}
%%   1 & 2 & 3 & 4 & 5 \\
%%   \hline
%% \end{operatorTable}
\newenvironment{operatorTable}[3][]
{
  \def\operatorTableLabel{#2}%
  \def\operatorTableCaption{#3}%
  \begin{table}[\floatposh]
    \centering#1%
    \begin{operatorTabular}
    }{%
    \end{operatorTabular}
    \caption{\operatorTableCaption}
    \label{\operatorTableLabel}
  \end{table}
}

\begin{operatorTable}{tab:arith}{Arithmetic operators}
  \operatoruplus\\
  \operatorumin
  \\\hline
  \operatorexp
  \\\hline
  \operatormult\\
  \operatordiv\\
  \operatormod
  \\\hline
  \operatorplus\\
  \operatorminus
\end{operatorTable}

\begin{urbiassert}
       1 + 1 ==    2;
       1 - 2 ==   -1;
       2 * 3 ==    6;
      10 / 2 ==    5;
     2 ** 10 == 1024;
    -(1 + 2) ==   -3;
   1 + 2 * 3 ==    7;
 (1 + 2) * 3 ==    9;
     -2 ** 2 ==   -4;
   - - - - 1 ==    1;
\end{urbiassert}

\subsubsection{Assignment operators}
\label{sec:lang:iass}
\dfn{Assignment} in \us can be performed with the \lstinline|=| operator.
Assignment operators, such as \lstinline|+=|, are supported too, see
\autoref{tab:assignment} and the examples below.


\begin{operatorTable}{tab:assignment}{Assignment operators}
  \operatorass[\footnotemark]\\
  \operatorsiass
\end{operatorTable}
\footnotetext{For object fields only. Assignment to local variables
  cannot be redefined. }

% FIXME: this in place modulo example was removed
%         because %= is a lame Urbi operator.
%  x %= 3;

The following example demonstrates that \lstinline|\var{a} += \var{b}|
behaves as \lstinline|\var{a} = \var{a} + \var{b}| for Floats.
\begin{urbiscript}
var y = 0;
[00000000] 0
y = 10;
[00000000] 10
y += 10;
[00000000] 20
y /= 5;
[00000000] 4
\end{urbiscript}

These operators are redefinable.  Indeed, \lstinline|\var{a} += \var{b}| is
actually processed as \lstinline|\var{a} = \var{a}.'+='(\var{b})|.  This
definition, which is neither that of \C
(\lstinline|\var{a} = \var{a}.'+'(\var{b})|) nor that of \Cxx
(\lstinline|\var{a}.'+='(\var{b})|), provides support for both
\dfn{immutable} and \dfn{mutable} values.

\paragraph{Immutable Values}
Small objects such as Floats should typically be immutable, i.e., the value
of a Float cannot change:

\begin{urbiscript}
var value = 0|;
var valueAlias = value|;
value += 10;
[00002275] 10
valueAlias;
[00002301] 0
\end{urbiscript}

It would be traitorous for most users that \lstinline|valueAlias| be equal
to 10 too.  That's why \slot[Float]{'+='} (which is actually
\refSlot[Object]{'+='}) simply bounces to \slot[Float]{'+'}.  The ``net
result'' of \lstinline|value += 10| is therefore
\lstinline|value = value.'+'(10)|, i.e., a \emph{new} Float is computed from
\lstinline|0.'+'(10)|, and \lstinline|value| is rebound to it.  The binding
from \lstinline|valueAlias| to \lstinline|0| is left as is.

\paragraph{Mutable Values}
On the contrary, large, ``updatable'' objects should provide an
implementation of \lstinline|'+='| that mutates them.  For instance,
implementing \lstinline|\var{a}.'+='(\var{b})| as
\lstinline|\var{a}.'+'(\var{b})| would be too costly for
\refObject[s]{List}.  Each time \lstinline|+=| is used, we need to create a
new List (whose content is that of \var{a}), then to append the contents of
\var{b}, and finally throw away the former value of \var{a}.

Not only is this inefficient, this is also wrong (at least from a certain
point of view).  Indeed, since we no longer update the List pointed to by
\var{a}, but rather store a new List, everything that was special to the
original List (its uid or whatever special slot the user may have defined)
is lost.  The proper implementation of \refSlot[List]{'+='} is therefore to
\emph{modify} \this by appending the added members.

\begin{urbiscript}
var myList = []|;
var myList.specialFeature = 42|;
myList += [1, 2, 3];
[00848865] [1, 2, 3]
myList.specialFeature;
[00848869] 42
var myOtherList = myList + [4, 5];
[00848873] [1, 2, 3, 4, 5]
myOtherList.specialFeature;
[00848926:error] !!! lookup failed: specialFeature
\end{urbiscript}

Note however that this means that because \lstinline|\var{a} += \var{b}| is
\emph{not} processed as \lstinline|\var{a} = \var{a} + \var{b}|, aliases to
\var{a} are possibly modified.

\begin{urbiscript}
var something = []|;
var somethingElse = something|;
something += [1, 2];
[00008557] [1, 2]
somethingElse += [3, 4];
[00008562] [1, 2, 3, 4]
something;
[00008566] [1, 2, 3, 4]
\end{urbiscript}

\paragraph{Example}
So basically, the rules to redefine these operators are:
\begin{description}
\item[Immutable (small) objects] \lstinline|'+='| should redirect to
  \lstinline|'+'| (which of course should \emph{not} modify its target).
\item[Mutable (large) objects] \lstinline|'+='| should update \this and
  return it.
\end{description}

The following examples contrasts both approaches.
\begin{urbiscript}
class Counter
{
  var count = 0;
  function init (n)   { var this.count = n };
  // Display the value, and the identity.
  function asString() { "%s @ %s" % [count, uid ] };
  function '+'(var n) { new(count + n) };
  function '-'(var n) { new(count - n) };
}|;
\end{urbiscript}

\begin{multicols}{2}
\begin{urbiscript}[xrightmargin=0mm,xleftmargin=0mm]
class ImmutableCounter : Counter
{
  function '+='(var n) { this + n };
  function '-='(var n) { this - n };
}|;

var ic1 = ImmutableCounter.new(0);
[00010566] 0 @ 0x100354b70
var ic2 = ic1;
[00010574] 0 @ 0x100354b70

ic1 += 1;
[00010588] 1 @ 0x10875bee0

// ic1 points to a new object.
ic1;
[00010592] 1 @ 0x10875bee0
// ic2 still points to its original value.
ic2;
[00010594] 0 @ 0x100354b70
\end{urbiscript}
\columnbreak
\begin{urbiscript}[xrightmargin=0mm,xleftmargin=0mm]
class MutableCounter : Counter
{
  function '+='(var n) { count += n | this };
  function '-='(var n) { count -= n | this };
}|;

var mc1 = MutableCounter.new(0);
[00029902] 0 @ 0x100364e00
var mc2 = mc1;
[00029911] 0 @ 0x100364e00

mc1 += 1;
[00029925] 1 @ 0x100364e00

// mc1 points to the same, updated, object.
mc1;
[00029930] 1 @ 0x100364e00
// mc2 too.
mc2;
[00029936] 1 @ 0x100364e00
\end{urbiscript}
\end{multicols}

\subsubsection{Postfix Operators}
\label{sec:lang:postop}

In the tradition of \C, \us provides postfix operators (see
\autoref{tab:postop}), e.g., \lstinline|\var{b} = \var{a}++|.  Prefix
operators, however, are not supported.  Rather than \lstinline|++\var{a}|,
write \lstinline|\var{a} += 1|.

\begin{operatorTable}{tab:postop}{Postfix operators}
  \operatorinc\\
  \operatordec
\end{operatorTable}

These operators \emph{modify} the variable/slot they are applied to, and
return the \emph{former} value of the variable/slot.

\begin{urbiassert}
var count = 0;
var countAlias = count;

count++    == 0;
count      == 1;
countAlias == 0;
count++    == 1;
count      == 2;
countAlias == 0;
count--    == 2;
count      == 1;
\end{urbiassert}

Similarly to assignment operators, these operators are redefinable.
Indeed, \lstinline|\var{a}++| is actually processed like
% beware of wrapping.
\lstinline-{ var '$save' = \var{a} | \var{a} = \var{a}.'++' | '$save' }-.
In other words, you are entitled to redefine the operator \lstinline|'++'|
whose semantics is ``return the successor of \this''.

Beware that the operator \lstinline|'++'| should \emph{not} modify its
target, but rather return a fresh value.  Indeed, if it alters \this, the
copy made in \lstinline|'$save'| will also have its value updated.
In % $ Pacify emacs
other words, the value of \lstinline|\var{a}++| would be its new one, not
its former one.

Redefining \lstinline|'++'| is still an experimental feature which might be
changed in future releases of \usdk, do not rely on it.

\subsubsection{Bitwise operators}
\label{sec:lang:bitwise}

\us features \dfn{bitwise operators}.  They are also used for other
purpose than bit-related operations. See \autoref{tab:bitwise} and the
listing below.

\begin{operatorTable}{tab:bitwise}{Bitwise operators}
  \operatorbcompl
  \\\hline
  \operatorlshift\\\operatorrshift
  \\\hline
  \operatorband
  \\\hline
  \operatorbxor
  \\\hline
  \operatorbor
\end{operatorTable}

\begin{urbiassert}
4 << 2 == 16;
4 >> 2 ==  1;

compl 0x0F0F_F0F0 == 0xF0F0_0F0F;

0x0000 bitand 0xF0F0 == 0x0000;
0x0F0F bitand 0xF0F0 == 0x0000;
0xF0F0 bitand 0xF0F0 == 0xF0F0;
0xFFFF bitand 0xF0F0 == 0xF0F0;

0x0000 bitor  0xF0F0 == 0xF0F0;
0x0F0F bitor  0xF0F0 == 0xFFFF;
0xF0F0 bitor  0xF0F0 == 0xF0F0;
0xFFFF bitor  0xF0F0 == 0xFFFF;

0x0000 ^ 0xF0F0 == 0xF0F0;
0x0F0F ^ 0xF0F0 == 0xFFFF;
0xF0F0 ^ 0xF0F0 == 0x0000;
0xFFFF ^ 0xF0F0 == 0x0F0F;
\end{urbiassert}

\subsubsection{Logical operators}

\us supports the usual \dfn{Boolean operators}. See the table and the
listing below. The operators \lstinline|&&| and \lstinline-||- are
short-circuiting: their right-hand side is evaluated only if needed.

\begin{operatorTable}{tab:Boolean}{Boolean operators}
  \operatorneg
  \\\hline
  \operatorand
  \\\hline
  \operatoror
\end{operatorTable}

\begin{urbiassert}
true && true;  !(true && false); !(false && true); !(false && false);
true || false; true || true; false || true; !(false || false);

!true == false; !false == true;

// Right hand-side is not evaluated when not needed.
true || (1 / 0);
(false && (1 / 0)) == false;
\end{urbiassert}

\subsubsection{Comparison operators}
\label{sec:lang:operators:comparison}

\us supports classical \dfn{comparison operators}. See
\autoref{tab:comparison} and the listing below.

\begin{operatorTable}{tab:comparison}{Comparison operators}
  \operatoreq\\
  \operatorneq\\
  \operatorpeq\\
  \operatorpneq\\
  \operatoraeq\\
  \operatoreqaeq\\
  \operatorinf\\
  \operatorinfeq\\
  \operatorsup\\
  \operatorsupeq
\end{operatorTable}

\begin{urbiassert}
 ! (0 < 0);
    0 <= 0;
    0 == 0;
   0 !== 0;

var z = 0;
     z === z;
  ! (z !== z);
\end{urbiassert}

\subsubsection{Container operators}
\label{sec:lang:operators:containers}

These operators work on containers and their members. See
\autoref{tab:operators:containers}.
\begin{operatorTable}{tab:operators:containers}{Container operators}
  \operatorsub\\\operatorsubass
  \\\hline
  \operatorin\\\operatornotin
\end{operatorTable}

The \dfn{in} and \dfn{not in} operators test the membership of an element in
a container.  They bounce to the container's \lstinline|has| and
\lstinline|hasNot| methods (see \refObject{Container}.  They are
non-associative.

\begin{urbiassert}
1     in [0, 1, 2];
3 not in [0, 1, 2];

"one"   in     ["zero" => 0, "one" => 1, "two" => 2];
"three" not in ["zero" => 0, "one" => 1, "two" => 2];
\end{urbiassert}

The following operators use an index. Note that the
\dfn[operator!subscript]{subscript} (square bracket) operator is
\dfn{variadic}: it takes any number of arguments that will be passed
to the \lstinline|'[]'| method of the targeted object.

\begin{urbiassert}
// On lists.
var l = [1, 2, 3, 4, 5];
  l[0] == 1;
  l[-1] == 5;
  (l[0] = 10) == 10;
  l == [10, 2, 3, 4, 5];

// On strings.
var s = "abcdef";
  s[0] == "a";
  s[1,3] == "bc";
  (s[1,3] = "foo") == "foo";
  s == "afoodef";
\end{urbiassert}


\subsubsection{Object operators}

These core operators provide access to slots and their properties. See
\autoref{tab:operators:object}.

\begin{operatorTable}{tab:operators:object}{Object operators}
  \operatordot\\\operatordota
  \\\hline
  \operatorprop\\\operatorpropass
  \\\hline
  \operatoruand\\\operatordotand
\end{operatorTable}

\begin{urbiassert}
var obj = Object.new;
var obj.f = function() { 24 };

  obj.f == 24;
  obj.&f != 24;
  obj.&f.isA(&Code);
  obj.&f === obj.getSlot("f");
\end{urbiassert}

\subsubsection{All operators summary}

\autoref{tab:operators-summary} is a summary of all operators, to highlight
the overall precedences. Operators are sorted by decreasing
precedence. Groups of rows represent operators with the same precedence.

\begin{operatorTable}[\ifx\ifHtml\undefined\small\fi]%
  {tab:operators-summary}%
  {Operators sorted by decreasing precedence groups.}
  \operatordot\\\operatordota
  \\\hline
  \operatorprop\\\operatorpropass
  \\\hline
  \operatoruand\\\operatordotand
  \\\hline
  \operatorsub\\\operatorsubass
  \\\hline
  \operatorexp\\\operatorbcompl\\\operatoruplus\\\operatorumin\\\operatorneg
  \\\hline
  \operatormult\\\operatordiv\\\operatormod
  \\\hline
  \operatorplus\\\operatorminus
  \\\hline
  \operatorlshift\\\operatorrshift
  \\\hline
  \operatorband
  \\\hline
  \operatorbxor
  \\\hline
  \operatorbor
  \\\hline
  \operatoreq\\\operatorneq\\\operatorpeq\\\operatorpneq\\\operatoreqaeq\\
  \operatoraeq\\\operatorinf\\\operatorinfeq\\\operatorsup\\\operatorsupeq
  \\\hline
  \operatorin\\\operatornotin
  \\\hline
  \operatorand
  \\\hline
  \operatoror
  \\\hline
  \operatorass\\\operatorsiass
  \\\hline
  \operatorinc\\\operatordec
\end{operatorTable}


\section{Scopes and local variables}

\subsection{Scopes}

\dfn{Scopes} are sequences of statements, enclosed in curly brackets
(\lstinline|{}|). Statements are separated with the four statements
separators (see \autoref{sec:lang:separators}).  A trailing \samp{;} or
\samp{,} is ignored.  A trailing \samp{\&} or \samp{|} behaves as if
\lstinline|& {}| or \lstinline'| {}' was used.  This particular case is
heavily used by \us programmers to discard the value of an expression:

\begin{urbiscript}
// Return value is 1.  Displayed.
1;
[00000000] 1
// Return value is that of {}, i.e., void.  Nothing displayed.
1 | {};
// Same as "1 | {}", a valueless expression.
1|;
\end{urbiscript}

Scopes are themselves expressions, and can thus be used in composite
expressions, nested, and so forth.

\begin{urbiscript}
// Scopes evaluate to their last expression
{
  1;
  2;
  3; // This last separator is optional.
};
[00000000] 3
// Scopes can be used as expressions
{1; 2; 3} + 1;
[00000000] 4
\end{urbiscript}

\subsection{Local variables}

\dfn{Local variables} are introduced with the \lstinline|var| keyword,
followed by an identifier (see \autoref{sec:us-syn-id}) and an optional
initialization value assignment. If the initial value is omitted, it
defaults to \refObject{void}. Variable declarations evaluate to the
initialization value. They can later be referred to by their name. Their
value can be changed with the assignment operator; such an assignment
expression returns the new value. The use of local variables is illustrated
below.

\begin{urbiscript}
// This declare variable x with value 42, and evaluates to 42.
var t = 42;
[00000000] 42
// x equals 42
t;
[00000000] 42
// We can assign it a new value
t = 51;
[00000000] 51
t;
[00000000] 51
// Initialization defaults to void
var u;
u.isVoid;
[00000000] true
\end{urbiscript}

The lifespan of local variables is the same as their enclosing scope. They
are thus only accessible from their scope and its
sub-scopes\footnote{Local variables can actually escape their scope
  with lexical closures, see \autoref{sec:us-fun-closures}.}. Two
variables with the same name cannot be defined in the same scope. A
variable with the same name can be defined in an inner scope, in which
case references refer to the innermost variable, as shown below.

\begin{urbiscript}
{
  var x = "x";
  var y = "outer y";
  {
    var y = "inner y";
    var z = "z";
    // We can access variables of parent scopes.
    echo(x);
    // This refers to the inner y.
    echo(y);
    echo(z);
  };
  // This refers to the outer y.
  echo(y);
  // This would be invalid: z does not exist anymore.
  // echo(z);
  // This would be invalid: x is already declared in this scope.
  // var x;
};
[00000000] *** x
[00000000] *** inner y
[00000000] *** z
[00000000] *** outer y
\end{urbiscript}


\section{Functions}

\subsection{Function Definition}
\label{sec:lang:function}
\dfn{Functions} in \us are first class citizens: a function is a value, like
floats and strings, and can be handled as such.  This is different from most
\C-like languages.  One can create a functional value thanks to the
\lstindex{function} keyword, followed by the list of formal arguments and a
compound statement representing the body of the function. Formal arguments
are a possibly-empty comma-separated list of identifiers.  Non-empty lists
of formal arguments may optionally end with a trailing comma. The listing
below illustrates this.

\begin{urbiscript}
function () { echo(0) };
[00000000] function () { echo(0) }

function (arg1, arg2) { echo(0) };
[00000000] function (var arg1, var arg2) { echo(0) }

function (
           arg1, // Ignored argument.
           arg2, // Also ignored.
          )
{
  echo(0)
};
[00000000] function (var arg1, var arg2) { echo(0) }
\end{urbiscript}

Usually functions are bound to an identifier to be invoked later.  The
listing below shows a short-hand to define a named function.

\begin{urbiscript}
// Functions are often stored in variables to call them later.
var f1 = function () {
  echo("hello")
}|
f1();
[00000000] *** hello

// This form is strictly equivalent, yet simpler.
function f2()
{
  echo("hello")
}|
f2();
[00000000] *** hello
\end{urbiscript}


\subsection{Arguments}

The list of formal arguments defines the number of argument the
function requires. They are accessible by their name from within the
body. If the list of formal arguments is omitted, the number of
effective arguments is not checked, and arguments themselves are not
evaluated. Arguments can then be manipulated with the call message,
explained below.

\begin{urbiscript}
var f = function(a, b) {
  echo(b + a);
}|
f(1, 0);
[00000000] *** 1
// Calling a function with the wrong number of argument is an error.
f(0);
[00000000:error] !!! f: expected 2 arguments, given 1
f(0, 1, 2);
[00000000:error] !!! f: expected 2 arguments, given 3
\end{urbiscript}

Non-empty lists of effective arguments may end with an optional comma.
\begin{urbiscript}
f(
  "bar",
  "foo",
 );
[00000000] *** foobar
\end{urbiscript}


\subsection{Return value}
\label{sec:lang:rv}
The \dfn[function!return value]{return value} of the function is the
evaluation of its body --- that is, since the body is a scope, the last
evaluated expression in the scope.

The execution of the function can be interrupted by a \lstinline{return}
statement.  The control flow then resumes to the caller.  The value of the
function call is the argument provided to the \lstinline{return} statement,
or \lstinline{void} if no argument was provided.

\begin{urbiscript}
function g1(a, b)
{
  echo(a);
  echo(b);
  a // Return value is a
}|
g1(1, 2);
[00000000] *** 1
[00000000] *** 2
[00000000] 1

function g2(a, b)
{
  echo(a);
  return a; // Stop execution at this point and return a
  echo(b); // This is not executed
}|
g2(1, 2);
[00000000] *** 1
[00000000] 1

function g3()
{
  return; // Stop execution at this point and return void
  echo(0); // This is not executed
}|
g3(); // Returns void, so nothing is printed.
\end{urbiscript}

You are encouraged no to use \lstinline{return} when the control flow needs
not be interrupted, see \autoref{sec:guideline:return}.

\subsection{Call messages}
\label{sec:lang:call}

Functions can access meta-information about how they were called, through a
\lstinline|CallMessage| object. The \dfn{call message} associated with a
function can be accessed with the \lstindex{all} keyword. It contains
several information such as not-yet evaluated arguments, the name of the
function, the target \ldots

\subsection{Strictness}

\us features two different function calls:
\dfn[function!strict]{strict} function calls, effective arguments are
evaluated before invoking the function, and \dfn[function!lazy]{lazy}
function calls, arguments are passed as-is to the function.  As a
matter of fact, the difference is rather that there are strict
functions and lazy functions.

Functions defined with a (possibly empty) list of formal arguments in
parentheses are strict: the effective arguments are first evaluated,
and then their value is given to the called function.

\begin{urbiscript}
function first1(a, b) {
  echo(a); echo(b)
}|
first1({echo("Arg1"); 1},
       {echo("Arg2"); 2});
[00000000] *** Arg1
[00000000] *** Arg2
[00000000] *** 1
[00000000] *** 2
\end{urbiscript}

A function declared with no formal argument list is lazy.  Use its call
message to manipulate its arguments \emph{not} evaluated.  The listing below
gives an example.  More information about this can be found in the
\refObject{CallMessage} class documentation.

\begin{urbiscript}
function first2
{
  echo(call.evalArgAt(0));
  echo(call.evalArgAt(1));
}|
first2({echo("Arg1"); 1},
       {echo("Arg2"); 2});
[00000000] *** Arg1
[00000000] *** 1
[00000000] *** Arg2
[00000000] *** 2
\end{urbiscript}

A lazy function may implement a strict interface by evaluating its
arguments and storing them as local variables, see below.  This is
less efficient than defining a strict function.

\begin{urbiscript}
function first3
{
  var a = call.evalArgAt(0);
  var b = call.evalArgAt(1);
  echo(a); echo(b);
}|
first3({echo("Arg1"); 1},
       {echo("Arg2"); 2});
[00000000] *** Arg1
[00000000] *** Arg2
[00000000] *** 1
[00000000] *** 2
\end{urbiscript}

\subsection{Lexical closures}
\label{sec:us-fun-closures}

\dfn{Lexical closures} are an additional scoping rule, with which a function
can refer to a local variable located outside the function --- but still
in the current context. \us supports read/write lexical closures,
meaning that the variable is shared between the function and the outer
environment, as shown below.

\begin{urbiscript}
var n = 0|
function cl()
{
  // x refers to a variable outside the function
  n++;
  echo(n);
}|
cl();
[00000000] *** 1
n;
[00000000] 1
n++;
[00000000] 1
cl();
[00000000] *** 3
\end{urbiscript}

The following listing illustrates that local variables can even escape their
declaration scope by lexical closure.

\begin{urbiscript}
function wrapper()
{
  // Normally, x is local to 'wrapper', and is limited to this scope.
  var x = 0;
  at (x > 1)
    echo("ping");
  // Here we make it escape the scope by returning a closure on it.
  return function() { x++ };
} |

var w = wrapper()|
w();
[00000000] 0
w();
[00000000] 1
[00000000] *** ping
\end{urbiscript}


\subsection{Variadic functions}
\label{sec:us-fun-variadic}

\dfn{Variadic} functions are functions that take a variable number of
arguments. They are created by using the \lstinline|[]| tag after a
formal argument: the function will accept any number of arguments, and
they will be assigned to the variadic formal argument as a list.

\begin{urbiscript}
function variadic(var args[])
{
  echo(args)
} |

variadic();
[00000000] *** []
variadic(1, 2, 3);
[00000000] *** [1, 2, 3]
\end{urbiscript}

There can be other formal arguments, as long as the variadic argument is at
the last position. If \var{n} is the number of non variadic arguments, the
function will request as least \var{n} effective arguments, which will be
assigned to the non variadic arguments in order like a classical function
call. All remaining arguments will be passed in list as the variadic
argument.

\begin{comment}
\begin{urbiscript}
removeSlot("variadic")|;
\end{urbiscript}
\end{comment}

\begin{urbiscript}
function invalid(var args[], var last)
{} |;
[00000000:error] !!! syntax error: argument after list-argument

function variadic(var a1, var a2, var a3, var args[])
{
  echo(a1);
  echo(a2);
  echo(a3);
  echo(args)
} |

// Not enough arguments.
variadic();
[00000000:error] !!! variadic: expected at least 3 arguments, given 0

// No variadic arguments.
variadic(1, 2, 3);
[00000000] *** 1
[00000000] *** 2
[00000000] *** 3
[00000000] *** []

// Two variadic arguments.
variadic(1, 2, 3, 4, 5);
[00000000] *** 1
[00000000] *** 2
[00000000] *** 3
[00000000] *** [4, 5]
\end{urbiscript}



\section{Objects}

Any \us value is an object. Objects contain:

\begin{itemize}
\item A set of slots, which associate an object to a name.
\item A list of prototypes, which are also objects.
\end{itemize}

\subsection{Slots}

\dfn{Objects} can contain any number of \dfn{slots}, every slot has a name
and a value. Slots are often called ``fields'', ``attributes'' or
``members'' in other object-oriented languages.

\subsubsection{Manipulation}

The \refSlot[Object]{createSlot} function adds a slot to an object with the
\refObject{void} value. The \refSlot[Object]{updateSlot} function changes
the value of a slot; \refSlot[Object]{getSlot} reads it. The
\refSlot[Object]{setSlot} method creates a slot with a given value. Finally,
the \refSlot[Object]{localSlotNames} method returns the list of the object
slot's name. The listing below shows how to manipulate slots. More
documentation about these methods can be found in \autorefObject{Object}.

\begin{urbiscript}
var o = Object.new|
assert (o.localSlotNames == []);

o.createSlot("test");
assert
{
  o.localSlotNames == ["test"];
  o.&test.isVoid;
};

o.updateSlot("test", 42);
[00000000] 42
assert
{
  o.&test == 42;
};
\end{urbiscript}

\subsubsection{Syntactic Sugar}

There is some syntactic sugar for slot methods:
\begin{itemize}
\item \lstinline|var \var{o}.\var{name}| is equivalent to
  \lstinline[style=varInString]|\var{o}.createSlot("\var{name}")|.
\item \lstinline|var \var{o}.\var{name} = \var{value}| is equivalent to
  \lstinline[style=varInString]|\var{o}.setSlot("\var{name}", \var{value})|.
\item \lstinline|\var{o}.\var{name} = \var{value}| is equivalent to
  \lstinline[style=varInString]|\var{o}.updateSlot("\var{name}", \var{value})|.
\item \lstinline|\var{o}.&\var{name}| is equivalent to
  \lstinline[style=varInString]|\var{o}.getSlot("\var{name}")| (\var{o} can
  be omitted, like for regular method invocations: \lstinline|&\var{name}|
  is equivalent to \lstinline[style=varInString]|getSlot("\var{name}")|).
\end{itemize}


\subsection{Properties}
\label{sec:lang:prop}
Slots can have properties, see \autoref{sec:tut:prop} for an introduction to
properties.

\subsubsection{Manipulation}

There is a number of functions to manipulate properties:
\begin{itemize}
\item \refSlot[Object]{setProperty}, to define/set a property.
\item \refSlot[Object]{getProperty}, to get a property.
\item \refSlot[Object]{removeProperty}, to delete a property.
\item \refSlot[Object]{hasProperty}, to test for the existence of a
  property.
\item \refSlot[Object]{properties}, to get all the properties of a slot.
\end{itemize}

There is also syntactic sugar for some of them:

\begin{itemize}
\item \lstinline|\var{slot}->\var{name}| is equivalent to
  \lstinline[style=varInString]|getProperty("\var{slot}", "\var{name}")|.
\item \lstinline|\var{slot}->\var{name} = \var{value}| is equivalent to
  \lstinline[style=varInString]|setProperty("\var{slot}", "\var{name}", \var{value})|.
\end{itemize}

\subsubsection{Standard Properties}
\label{sec:lang:prop:std}

Some properties are handled by the system itself.


\paragraph{The \lstinline|changed| property}

It allows to monitor when a slot is bound to new values: each time a new
value is assigned to the monitored slot, an event \lstinline|changed| is
emitted:

\begin{urbiscript}[firstnumber=1]
var x = []|;

at (x->changed?)
  echo("x->changed");

x = [1]|;
[00092656] *** x->changed
x = [1, 2]|;
[00092756] *** x->changed
\end{urbiscript}

Even if the slot is assigned to the very same value, the
\lstinline|x->changed| event is emitted.

\begin{urbiscript}
x = x|;
[00092856] *** x->changed
\end{urbiscript}

This is different from checking updates to the value a slot is bound to:

\begin{urbiscript}
x << 3;
[00092866] [1, 2, 3]
\end{urbiscript}

One can monitor updates using the \lstinline|changed| \emph{slot} (not
property).


\paragraph{The \lstinline|constant| property}
It defines whether a slot can be assigned a new value.

\begin{urbiscript}
var c = 0;
[00000000] 0
c = 1;
[00000000] 1

c->constant = true;
[00000000] true
c = 2;
[00000000:error] !!! cannot modify const slot

c->constant = false;
[00000000] false
c = 3;
[00000000] 3
\end{urbiscript}

A new slot can be declared constant when first defined, in which case the
initial value is required.

\begin{urbiscript}
const var two;
[00000030:error] !!! syntax error: const declaration without a value
const var two = 2;
[00000036] 2
two = 3;
[00000037:error] !!! cannot modify const slot
two->constant;
[00000038] true
\end{urbiscript}

When a slot is declared constant, copy-on-write does not apply (see
\refSlot[Object]{updateSlot}).

\begin{urbiscript}
class Top
{
  const var x = 1;
}|;
var t = Top.new|;
t.x = 2;
[00000047:error] !!! cannot modify const slot
\end{urbiscript}



\subsection{Prototypes}

\subsubsection{Manipulation}

\us is a prototype-based language, unlike most classical object oriented
language, which are class-based. In prototype-based languages, objects have
no type, only \dfn{prototypes}, from which they inherit behavior.

\us objects can have several prototypes. The list of prototypes is given by
the \refSlot[Object]{protos} method; they can be added or removed with
\refSlot[Object]{addProto} and \refSlot[Object]{removeProto}.  See
\autorefObject{Object} for more documentation.

\begin{urbiscript}
var ob = Object.new|
assert (ob.protos == [Object]);

ob.addProto(Pair);
[00000000] (nil, nil)
assert (ob.protos == [(nil, nil), Object]);

ob.removeProto(Object);
[00000000] (nil, nil)
assert (ob.protos == [(nil, nil)]);
\end{urbiscript}

\subsubsection{Inheritance}

Objects inherit their prototypes' slots: \refSlot[Object]{getSlot} will also
look in an object prototypes' slots. \refSlot[Object]{getSlot} performs a
depth-first traversal of the prototypes hierarchy to find slots. That is,
when looking for a slot in an object:

\begin{itemize}
\item \refSlot[Object]{getSlot} checks first if the object itself has the
  requested slot. If so, it returns its value.
\item Otherwise, it applies the same research on every prototype, in the
  order of the prototype list (since \refSlot[Object]{addProto} inserts in
  the front of the prototype list, the last prototype added has
  priority). This search is recursive: \refSlot[Object]{getSlot} will also
  look in the first prototype's prototype, etc. before looking in the second
  prototype. If the slot is found in a prototype, it is returned.
\item Finally, if no prototype had the slot, an error is raised.
\end{itemize}

The following example shows how slots are inherited.

\begin{urbiscript}[firstnumber=1]
var a = Object.new|
var b = Object.new|
var c = Object.new|
a.setSlot("x", "slot in a")|
b.setSlot("x", "slot in b")|
// c has no "x" slot
c.getSlot("x");
[00000000:error] !!! lookup failed: x
// c can inherit the "x" slot from a.
c.addProto(a)|
c.getSlot("x");
[00000000] "slot in a"
// b is prepended to the prototype list, and has thus priority.
c.addProto(b)|
c.getSlot("x");
[00000000] "slot in b"
// A local slot in c has priority over prototypes.
c.setSlot("x", "slot in c")|
c.getSlot("x");
[00000000] "slot in c"
\end{urbiscript}

\subsubsection{Copy on write}

The \refSlot[Object]{updateSlot} method has a particular behavior with
respect to prototypes. Although performing an \refSlot[Object]{updateSlot}
on a non existent slot is an error, it is valid if the slot is inherited
from a prototype. In this case, the slot is however not updated in the
prototype, but rather created in the object itself, with the new value. This
process is called \dfn{copy on write}; thanks to it, prototypes are not
altered when the slot is overridden in a child object.

\begin{urbiscript}
var p = Object.new|
var p.slot = 0|
var d = Object.new|
d.addProto(p)|
d.slot;
[00000000] 0
d.slot = 1;
[00000000] 1
// p's slot was not altered
p.slot;
[00000000] 0
// It was copied in d
d.slot;
[00000000] 1
\end{urbiscript}

\subsection{Sending messages}

A \dfn{message} in \us consists in a message name and arguments. One can
send a message to an object with the dot (\lstinline|.|) operator, followed
by the message name (which can be any valid identifier) and the arguments,
as shown below. When there are no arguments, the parentheses can be
omitted. As you might see, sending messages is very similar to calling
methods in classical languages.

\begin{urbiunchecked}
// Send the message msg to object obj, with arguments arg1 and arg2.
obj.msg(arg1, arg2);
// Send the message msg to object obj, with no arguments.
obj.msg();
// This is strictly equivalent.
obj.msg;
\end{urbiunchecked}

When a message \var{msg} is sent to object \var{obj}:

\begin{itemize}
\item The \var{msg} slot of \var{obj} is retrieved (i.e.,
  \lstinline[style=varInString]|\var{obj}.getSlot("\var{msg}")|). If the
  slot is not found, a lookup error is raised.
\item If the object is a \dfn{routine} (either a primitive, written in \Cxx
  for instance, or a function implemented in \us), it is invoked with the
  message arguments, and the returned value is the result. As a consequence,
  the number of arguments in the message sending must match the one required
  by the routine.
\item Otherwise (the object is not a routine), this object is the result of
  the message sending. There must be no argument.
\end{itemize}

Such message sending is illustrated below.

\begin{urbiscript}
var obj = Object.new|
var obj.a = 42|
var obj.b = function (x) { x + 1 }|
obj.a;
[00000000] 42
obj.a();
[00000000] 42
obj.a(50);
[00000000:error] !!! a: expected 0 argument, given 1
obj.b;
[00000000:error] !!! b: expected 1 argument, given 0
obj.b();
[00000000:error] !!! b: expected 1 argument, given 0
obj.b(50);
[00000000] 51
\end{urbiscript}

\section{Enumeration types}
\label{sec:lang:enum}

Enumeration types enable to create types represented by a finite set of
values, like the \lstinline{enum} declaration in \C.

\begin{urbiscript}
enum Suit
{
  hearts,
  diamonds,
  clubs,
  spades, // Last comma is optional
};
[00000001] Suit
\end{urbiscript}

Since everything is an object in \us, enums are too, with
\refObject{Enumeration} as prototype.

\begin{urbiscript}
Suit;
[00000001] Suit
Suit.protos;
[00000002] [Enumeration]
\end{urbiscript}

The possible enum values are stored inside the enum object. They inherit the
enum object, so you can easily test whether an object is a Suit or not.

\begin{urbiscript}
Suit.hearts;
[00000001] hearts
Suit.diamonds;
[00000002] diamonds
Suit.clubs.isA(Suit);
[00000003] true
42.isA(Suit);
[00000003] false
\end{urbiscript}

Enumeration values support comparison and pattern matching. You can iterate
on the enum object to cycle through all possible values.

\begin{urbiscript}
function find_ace(var suit)
{
  switch (suit)
  {
    case Suit.spades: "The only card I need is";
    default:          "I have";
  }
}|;

for (var suit in Suit)
  echo("%s the ace of %s." % [find_ace(suit), suit]);
[00000001] *** I have the ace of hearts.
[00000002] *** I have the ace of diamonds.
[00000003] *** I have the ace of clubs.
[00000004] *** The only card I need is the ace of spades.
\end{urbiscript}

\section{Structural Pattern Matching}
\label{sec:lang:pattern}

Structural \dfn{pattern matching} is useful to deconstruct tuples, lists and
dictionaries with a small and readable syntax.

These patterns can be used in the following clauses:
\begin{itemize}
\item The left hand side of an assignment.
\item \lstinline{case}
\item \lstinline{catch}
\item \lstinline{at}
\item \lstinline{waituntil}
\item \lstinline{whenever}
\end{itemize}

The following examples illustrate the possibilities of \dfn{structural
  pattern matching} inside \lstinline{case} clauses:

\begin{urbiscript}
switch ( ("foo", [1, 2]) )
{
  // The pattern does not match the values of the list.
  case ("foo", [2, 1]):
    echo("fail");

  // The pattern does not match the tuple.
  case ["foo", [1, 2]]:
    echo("fail");

  // The pattern matches and binds the variable "l"
  // but the condition is not verified.
  case ("foo", var l) if l.size == 0:
    echo("fail");

  // The pattern matches.
  case ("foo", [var a, var b]):
    echo("foo(%s, %s)" % [a, b]);
};
[00000000] *** foo(1, 2)
\end{urbiscript}

\subsection{Basic Pattern Matching}

Matching is used in many locations and allows to match literal values (e.g.,
\refObject{List}, \refObject{Tuple}, \refObject{Dictionary},
\refObject{Float}, \refObject{String}).  In the following expressions each
pattern (on the left hand side) matches the value (on the right hand side).

\begin{urbiscript}
(1, "foo") = (1, "foo");
[00000000] (1, "foo")
[1, "foo"] = [1, "foo"];
[00000000] [1, "foo"]
["b" => "foo", "a" => 1] = ["a" => 1, "b" => "foo"];
[00000000] ["a" => 1, "b" => "foo"]
\end{urbiscript}

A \refSlot[Exception]{MatchFailure} exception is thrown when a pattern does
not match.

\begin{urbiscript}
try
{
  (1, 2) = (3, 4)
}
catch (var e if e.isA(Exception.MatchFailure))
{
  e.message
};
[00000000] "pattern did not match"
\end{urbiscript}

\subsection{Variable}

Patterns can contain variable declarations, to match any value and to bind
it to a new variable.

\begin{urbiscript}
{
  (var a, var b) = (1, 2);
  echo("a = %d, b = %d" % [a, b]);
};
[00000000] *** a = 1, b = 2
{
  [var a, var b] = [1, 2];
  echo("a = %d, b = %d" % [a, b]);
};
[00000000] *** a = 1, b = 2
{
  ["b" => var b, "a" => var a] = ["a" => 1, "b" => 2, "c" => 3];
  echo("a = %d, b = %d" % [a, b]);
};
[00000000] *** a = 1, b = 2
\end{urbiscript}

\subsection{Guard}

Patterns used inside a \lstinline{switch}, a \lstinline{catch} or an event
catching construct accept guards.

\dfn[guard]{Guard} are used by appending a \lstinline{if} after a pattern or
after a matched event.

The following example is inspired from the \refObject{TrajectoryGenerator}
where a \refObject{Dictionary} is used to set the trajectory type.

\begin{urbiscript}
switch (["speed" => 2, "time" => 6s])
{
  case ["speed" => var s] if s > 3:
    echo("Too fast");
  case ["speed" => var s, "time" => var t] if s * t > 10:
    echo("Too far");
};
[00000000] *** Too far
\end{urbiscript}

The same guard are available for \lstinline{catch} statement.

\begin{urbiscript}
try
{
  throw ("message", 0)
}
catch (var e if e.isA(Exception))
{
  echo(e.message)
}
catch ((var msg, var value) if value.isA(Float))
{
  echo("%s: %d" % [msg, value])
};
[00000000] *** message: 0
\end{urbiscript}

Events catchers can have guards on the pattern arguments.  You can add these
inside \lstinline{at}, \lstinline{whenever} and \lstinline{waituntil}
statements.

\begin{urbiscript}
{
  var e = Event.new;
  at (e?(var msg, var value) if value % 2 == 0)
    echo("%s: %d" % [msg, value]);

  // Does not trigger the "at" because the guard is not verified.
  e!("message", 1);

  // Trigger the "at".
  e!("message", 2);
};
[00000000] *** message: 2
\end{urbiscript}

%% \subsection{Extension}
%% \refObject{Pattern}
%% \refSlot[Pattern]{Binding}
%% \refSlot[List]{matchAgainst}

\section{Imperative flow control}

\subsection{break}
\label{sec:lang:break}

When encountered within a loop (\lstinline|for| or a \lstinline|while|),
\lstinline|break| makes the execution jump after the loop.

\begin{urbiscript}
{
  var i = 5;
  for (; true; echo(i))
  {
    if (i > 8)
      break;
    i += 1;
  };
};
[00000000] *** 6
[00000000] *** 7
[00000000] *** 8
[00000000] *** 9
\end{urbiscript}

\begin{urbiscript}
function breakInWhile ()
{
  var i = 0;
  while (i < 10)
  {
    echo(i);
    if (i == 2)
      break;
    i += 1;
  };
  echo(i);
}|;
breakInWhile;
[00000004] *** 0
[00000005] *** 1
[00000006] *** 2
[00000007] *** 2
\end{urbiscript}

Occurrences of \lstinline|break| outside any loop is a syntax error.

\begin{urbiscript}
break;
[00000011:error] !!! syntax error: `break' not within a loop
\end{urbiscript}

It is also invalid within function bodies inside loops.

\begin{urbiscript}
for (var i : 10)
{
  function f() { break };
  f;
};
[00000011:error] !!! syntax error: `break' not within a loop
\end{urbiscript}


\subsection{continue}
\label{sec:lang:continue}

When encountered within a \lstinline|for| or a \lstinline|while| loop,
\lstinline|continue| short-circuits the rest of the loop-body, and
runs the next iteration (if there remains one).

\begin{urbiscript}
for (var i = 0; i < 8; i++)
{
  if (i % 2 != 0)
    continue;
  echo(i);
};
[00000000] *** 0
[00000002] *** 2
[00000004] *** 4
[00000006] *** 6
\end{urbiscript}

\begin{urbiscript}
function continueInWhile ()
{
  var i = 0|;
  while (i < 6)
  {
    i += 1;
    if (i % 2 != 0)
      continue;
    echo(i);
  };
  echo(i);
}|;
continueInWhile;
[00000002] *** 2
[00000004] *** 4
[00000006] *** 6
[00000008] *** 6
\end{urbiscript}

Occurrences of \lstinline|continue| outside any loop is a syntax error.

\begin{urbiscript}
continue;
[00000011:error] !!! syntax error: `continue' not within a loop
\end{urbiscript}

It is also invalid within function bodies inside loops.

\begin{urbiscript}
for (var i : 10)
{
  function f() { continue };
  f;
};
[00000011:error] !!! syntax error: `continue' not within a loop
\end{urbiscript}

\subsection{do}
\label{sec:lang:do}

The \lstinline|do| construct changes the target (\this) when evaluating an
expression.  In some programming languages it is also named
\lstinline|with|.  It is a convenient means to avoid repeating the same
target several times.

\begin{bnf}
<do-statement> ::= "do" "(" <expression ")" <block>
\end{bnf}

It evaluates the \var{block}, with \this denoting the value of the
\var{expression}, as shown below.  The whole construct evaluates to the
value of \var{body}.

\begin{urbiscript}
do (1024)
{
  assert(this == 1024);
  assert(sqrt == 32);
  setSlot("y", 23);
}.y;
[00000000] 23
\end{urbiscript}




\subsection{for}
\label{sec:lang:for}
There are several kinds of \lstinline|for| loops:
\begin{description}
\item[C-like for] such as
  \lstinline|for (var i = 0; i < n; i += 1) echo(i);|, see
  \autoref{sec:lang:for:c}.
\item[Range-for] also known as \dfn{foreach}-loops,
  \lstinline|for (var i : [0, 1, 2, 4]) echo(i);|, see
  \autoref{sec:lang:for:each}.
\item[for $n$-times] for instance \lstinline|for (10) echo(10);|, see
  \autoref{sec:lang:for:n}.
\end{description}

\subsubsection{C-like for}
\label{sec:lang:for:c}

\us support the classical \C-like \lstinline|for| construct.

\begin{urbiunchecked}
for (\var{initialization}; \var{condition}; \var{increment})
  \var{body}
\end{urbiunchecked}

It has the exact same behavior as \C's \lstinline|for|:

\begin{enumerate}
\item The \var{initialization} is evaluated.
\item \var{condition} is evaluated. If the result is false, executions
  jump after \lstinline|for|.
\item \var{body} is evaluated. If \lstinline|continue| is encountered,
  execution jumps to point 4. If \lstinline|break| is encountered,
  executions jumps after the \lstinline|for|.
\item The \var{increment} is evaluated.
\item Execution jumps to point 2.
\item The loop evaluates to \lstinline|void|.
\end{enumerate}

\subsubsection{Range-for}
\label{sec:lang:for:each}

\us supports iteration over a collection with another form of the
\lstinline|for| loop.

\begin{urbiunchecked}
for (var \var{name} : \var{collection})
   \var{body};
\end{urbiunchecked}

It evaluates \var{body} for each element in \var{collection}. The loop
evaluates to \lstinline|void|.  Inside \var{body}, the current element
is accessible via the \var{name} local variable. The listing below
illustrates this.

\begin{urbiscript}
for (var x : [0, 1, 2, 3, 4])
  echo(x.sqr);
[00000000] *** 0
[00000000] *** 1
[00000000] *** 4
[00000000] *** 9
[00000000] *** 16
\end{urbiscript}

This form of \lstinline|for| simply sends the \lstinline|each| message
to \var{collection} with one argument: the function that takes the
current element and performs \lstinline|action| over it. Thus, you can
make any object acceptable in a \lstinline|for| by defining an
adequate \lstinline|each| method.

\begin{urbiscript}
var Hobbits = Object.new|
function Hobbits.each (action)
{
  action("Frodo");
  action("Merry");
  action("Pippin");
  action("Sam");
}|
for (var name in Hobbits)
  echo("%s is a hobbit." % [name]);
[00000000] *** Frodo is a hobbit.
[00000000] *** Merry is a hobbit.
[00000000] *** Pippin is a hobbit.
[00000000] *** Sam is a hobbit.
// This for statement is equivalent to:
Hobbits.each(function (name) { echo("%s is a hobbit." % [name]) });
[00000000] *** Frodo is a hobbit.
[00000000] *** Merry is a hobbit.
[00000000] *** Pippin is a hobbit.
[00000000] *** Sam is a hobbit.
\end{urbiscript}

\subsubsection{for $n$-times}
\label{sec:lang:for:n}

\us provides some support for simple replication of computations: it allows
to repeat a loop body $n$-times.  With the exception that the loop index is
not available within the body, \lstinline|for (n)| is equivalent to
\lstinline|for (var i: n)|.  It supports the same flavors: \lstinline|for;|,
\lstinline{for|}, and \lstinline|for&|.

\begin{bnf}
<for-statement>
   ::= "for" "(" <expression ")" <block>
     | "for;" "(" <expression ")" <block>
     | "for|" "(" <expression ")" <block>
\end{bnf}

The loop evaluates to \lstinline|void|.

\begin{urbiassert}
{ var res = []; for (3) { res << 1; res << 2 } ; res }
        == [1, 2, 1, 2, 1, 2];

{ var res = []; for|(3) { res << 1; res << 2 } ; res }
        == [1, 2, 1, 2, 1, 2];

{ var res = []; for&(3) { res << 1; res << 2 } ; res }
        == [1, 1, 1, 2, 2, 2];
\end{urbiassert}

Note that since these \lstinline|for| loops are merely anonymous
foreach-style loops, the argument needs not being an integer, any iterable
value can be used.

\begin{urbiassert}
3 == { var r = 0; for ([1, 2, 3]) r += 1; r};
3 == { var r = 0; for ("123")     r += 1; r};
\end{urbiassert}


\subsection{if}
\label{sec:lang:if}
As in most programming languages, conditionals are expressed with
\lstindex{if}.

\begin{bnf}
<statement-if>
  ::= "if" "(" <statements> ")" <statement>
    | "if" "(" <statements> ")" <statement> "else" <statement>
\end{bnf}

First the condition (\var{statements}) is evaluated; if it evaluates to a
value which is true (\autoref{sec:truth}), evaluate the \dfn{then-clause}
(the first \var{statements}), otherwise, if applicable, evaluate
\dfn{else-clause} (the optional statement introduced by \lstinline|else|).

\begin{urbiscript}
if (true) assert(true) else assert(false);
if (false) assert(false) else assert(true);
if (true) assert(true);
\end{urbiscript}

Beware that contrary to most programming languages \emph{there must not be a
  terminator after the then-clause}:

\begin{urbiscript}
if (true)
  assert(true);
else
  assert(false);
[00000002:error] !!! syntax error: unexpected else
\end{urbiscript}

Contrary to \C/\Cxx, it has value: it also implements the
\lstinline|\var{condition} ? \var{then-clause} : \var{else-clause}|
construct.  Unfortunately, due to syntactic constraints inherited from \C,
it is a \emph{statement}: it cannot be used directly as an expression.  But
as everywhere else in \us, to use a statement where an expression is
expected, use braces:

\begin{urbiscript}
assert(1 + if (true) 3 else 4 == 4);
[00000003:error] !!! syntax error: unexpected if
assert(1 + { if (true) 3 else 4 } == 4);
\end{urbiscript}

The condition can be any statement list.  Variables which it declares are
visible in both the then-clause and the else-clause, but do not escape the
\lstinline|if| construct.

\begin{urbiassert}
{if (false) 10 else 20} == 20;
{if (true)  10 else 20} == 10;

{if (true) 10         } == 10;

{if (var x = 10) x + 2 else x - 2} == 12;
{if (var x = 0)  x + 2 else x - 2} == -2;

{if (var xx = 123) xx | xx};
[00000005:error] !!! lookup failed: xx
\end{urbiassert}


\subsection{loop}
\label{sec:lang:loop}

Endless loops can be created with \lstindex{loop}, which is equivalent to
\lstinline|while (true)|.

\begin{bnf}
<loop-statement>
  ::= "loop" <statement>
    | "loop;" <statement>
    | "loop|" <statement>
\end{bnf}

The loop evaluates to \lstinline|void|.  Both sequential flavors,
\lstinline|loop;| and \lstinline'loop|', are supported.  The default flavor
is \lstinline|loop;|.

\begin{urbiassert}
{
  var n = 10|;
  var res = []|;
  loop;
  {
    n--;
    res << n;
    if (n == 0)
      break
  };
  res
}
==
[9, 8, 7, 6, 5, 4, 3, 2, 1, 0];
\end{urbiassert}

\begin{urbiassert}
{
  var n = 10|;
  var res = []|;
  loop|
  {
    n--;
    res << n;
    if (n == 0)
      break
  };
  res
}
==
[9, 8, 7, 6, 5, 4, 3, 2, 1, 0];
\end{urbiassert}

\subsection{switch}
\label{sec:lang:switch}
The \lstinline|switch| statement in \us is similar to \C's one.

\begin{urbiunchecked}
switch (\var{value})
{
  case \var{value_one}:
    \var{action_one};
  case \var{value_two}:
    \var{action_two};
//case ...:
//  ...
  default:
    \var{default_action};
};
\end{urbiunchecked}

It might contain an arbitrary number of cases, and optionally a default
case. The \var{value} is evaluated first, and then the result is compared
sequentially with the evaluation of all cases values, with the
\lstinline|==| operator, until one comparison is true.  If such a match is
found, the corresponding action is executed, and execution jumps after the
\lstinline|switch|. Otherwise, the default case --- if any --- is executed,
and execution jumps after the switch. The switch itself evaluates to case
that was evaluated, or to void if no match was found and there's no default
case. The listing below illustrates \lstinline|switch| usage.

Unlike \C, there is no \lstinline|break| to end \lstinline|case| clauses:
execution will never span over several cases.  Since the comparisons are
performed with the generic \lstinline|==| operator, \lstinline|switch| can
be performed on any comparable data type.  Actually, the comparison process
is richer than simply using \lstinline|==|: it support pattern-matching
(\autoref{sec:lang:pattern}).

\begin{urbiscript}
function sw(v)
{
  switch (v)
  {
    case "":
      echo("Empty string");
    case "foo":
      "bar";
    default:
      v[0];
  }
}|;
sw("");
[00000000] *** Empty string
sw("foo");
[00000000] "bar"
sw("foobar");
[00000000] "f"
\end{urbiscript}
% $ Pacify emacs math mode.

\subsection{while}
\label{sec:lang:while}

The \lstinline|while| loop is similar to \C's one.

\begin{urbiunchecked}
while (\var{condition})
  \var{body};
\end{urbiunchecked}

If \var{condition} evaluation, is true, \var{body} is evaluated and
execution jumps before the \lstinline|while|, otherwise execution
jumps after the \lstinline|while|.

\begin{urbiscript}
var j = 3|
while (0 < j)
{
  echo(j);
  j--;
};
[00000000] *** 3
[00000000] *** 2
[00000000] *** 1
\end{urbiscript}

The default flavor for \lstinline|while| is \lstinline|while;|.

\subsubsection{while;}

The semantics of

\begin{urbiunchecked}
while; (\var{condition})
  \var{body};
\end{urbiunchecked}

\noindent
is the same as

\begin{urbiunchecked}
\var{condition} | \var{body} ; \var{condition} | \var{body} ; ...
\end{urbiunchecked}

\noindent
as long as \var{cond} evaluates to true, or until \lstinline|break| is
invoked.  If \lstinline|continue| is evaluated, the rest of the body
is skipped, and the next iteration is started.

\begin{urbiscript}
{
  var i = 4|
  while (true)
  {
    i -= 1;
    echo ("in: " + i);
    if (i == 1)
      break
    else if (i == 2)
      continue;
    echo ("out: " + i);
  };
};
[00000000] *** in: 3
[00000000] *** out: 3
[00000000] *** in: 2
[00000000] *** in: 1
\end{urbiscript}


\subsubsection{while|}

The semantics of

\begin{urbiunchecked}
while| (\var{condition})
  \var{body};
\end{urbiunchecked}

\noindent
is the same as

\begin{urbiunchecked}
\var{condition} | \var{body} | \var{condition} | \var{body} | ...
\end{urbiunchecked}

The execution is can be controlled by \lstinline|break| and
\lstinline|continue|.

\begin{urbiscript}
{
  var i = 4|
  while| (true)
  {
    i -= 1;
    echo ("in: " + i);
    if (i == 1)
      break
    else if (i == 2)
      continue;
    echo ("out: " + i);
  };
};
[00000000] *** in: 3
[00000000] *** out: 3
[00000000] *** in: 2
[00000000] *** in: 1
\end{urbiscript}


\section{Exceptions}
\label{sec:lang:except}
\subsection{Throwing exceptions}
\label{sec:lang:throw}

Use the \lstinline|throw| keyword to \dfn[exception!throwing]{throw
  exceptions}, as shown below. Thrown exceptions will break the execution
upward until they are caught, or until they reach the top-level --- as in
\Cxx.  Contrary to \Cxx, exceptions reaching the top-level are printed, and
won't abort the kernel --- other and new connections will continue to
execute normally.

\begin{urbiscript}
throw 42;
[00000000:error] !!! 42
function inner() { throw "exn" } |
function outer() { inner() }|
// Exceptions propagate to parent call up to the top-level
outer();
[00000000:error] !!! exn
[00000000:error] !!!    called from: inner
[00000000:error] !!!    called from: outer
\end{urbiscript}

\subsection{Catching exceptions}
\label{sec:lang:catch}

Exceptions are \dfn[exception!catching]{caught} with the
\lstinline|try|/\lstinline|catch| construct. Its syntax is as follows:

\begin{bnf}
<try-statement>
  ::= "try" "{" <statement>* "}" <catch-clause>+ <else-clause>? <finally-clause>?
    | "try" "{" <statement>* "}" <finally-clause>

<catch-clause>
  ::= "catch" "(" <pattern> ")" "{" <statement>* "}"
    | "catch" "{" <statement>* "}"

<else-clause>
  ::= "else" "{" <statement>* "}"

<finally-clause>
  ::= "finally" "{" <statement>* "}"
\end{bnf}

It consists of a first block of statements (the \dfn{try-block}), from which
we want to catch exceptions, and one or more catch clauses to stop the
exception (\dfn{catch-blocks}).

Each \lstinline|catch| clause defines a pattern against which the thrown
exception is matched. If no pattern is specified, the catch clause matches
systematically (equivalent to \lstinline|catch (...)| in \Cxx).  It is a
syntax error if this catch-all clause is followed by a catch-clause with a
pattern:

\begin{urbiscript}
try {} catch {} catch (var e) {};
[00000701:error] !!! syntax error: catch: exception already caught by a previous clause
\end{urbiscript}

\noindent
The catch-all clause, if present, must be last:

\begin{urbiscript}
try {} catch (var e) {} catch {};
\end{urbiscript}

Exceptions thrown from the \lstinline|try| block are matched sequentially
against all catch clauses. The first matching clause is executed, and
control jumps after the whole try/catch block. If no catch clause matches,
the exceptions isn't stopped and continues upward.

\begin{urbiscript}
function test(e)
{
  try
  { throw e;  }
  catch (0)
  { echo("zero") }
  catch ([var x, var y])
  { echo(x + y) }
}|;
test(0);
[00002126] *** zero
test([22, 20]);
[00002131] *** 42
test(51);
[00002143:error] !!! 51
[00002143:error] !!!    called from: test
\end{urbiscript}

If an \lstinline|else|-clause is specified, it is executed if the
\lstinline|try| block did not raise an exception.

\begin{urbiscript}
try   { echo("try") }
catch { echo("catch")}
else  { echo("else")};
[00002855] *** try
[00002855] *** else

try   { echo("try"); echo("throw"); throw 0 }
catch { echo("catch")}
else  { echo("else")};
[00002855] *** try
[00002855] *** throw
[00002855] *** catch
\end{urbiscript}

The value of the whole construct is:
\begin{itemize}
\item if the \lstinline|try| block raised an exception
  \begin{itemize}
  \item if the exception is not caught (or an exception is thrown from the
    catch-clause), then there is no value, as the control flow is broken;
  \item if the exception is caught; then it's the value of the corresponding
    catch clause.
\begin{urbiscript}
try { throw 0; "try" } catch (0) { "catch" } else { "else" };
[00467080] "catch"
\end{urbiscript}
  \end{itemize}
\item if the \lstinline|try| block finish properly, then
  \begin{itemize}
  \item if there is an \lstinline|else|-clause, its value.
\begin{urbiscript}
try { "try" } catch (0) { "catch" } else { "else" };
[00467080] "else"
\end{urbiscript}
  \item otherwise the value of the \lstinline|try|-block.
\begin{urbiscript}
try { "try" } catch (0) { "catch" };
[00467080] "try"
\end{urbiscript}
  \end{itemize}
\end{itemize}


\subsection{Inspecting exceptions}

An \refObject{Exception} is a regular object, on which introspection
can be performed.

\begin{urbiscript}
try
{
  Math.cos(3,1415);
}
catch (var e)
{
  echo ("Exception type: %s" % e.type);
  if (e.isA(Exception.Arity))
  {
    echo("Routine: %s" % e.routine);
    echo("Number of effective arguments: %s" % e.effective);
  };
};
[00000132] *** Exception type: Arity
[00000133] *** Routine: cos
[00000134] *** Number of effective arguments: 2
\end{urbiscript}

\subsection{Finally}
\label{sec:lang:except:finally}

Using the finally-clause construct, you can ensure some code is executed
upon exiting a try-clause, be it naturally or through an exception,
\lstinline|return|, \lstinline|continue|, \ldots

\subsubsection{Regular execution}

The finally-clause is executed when the try-clause exits normally.

\begin{urbiscript}
try
{
  echo("inside");
}
finally
{
  echo("finally");
};
[00000001] *** inside
[00000002] *** finally
\end{urbiscript}

Because it is meant to be used to reclaim resources (``clean up'', the
finally-clause never contributes to the value of the whole statement.

\begin{urbiscript}
try { "try" } catch { "catch" } else { "else" } finally { "finally" };
[00000001] "else"
try { throw "throw" } catch { "catch" } else { "else" } finally { "finally" };
[00000001] "catch"
\end{urbiscript}


\subsubsection{Control-flow}

The finally clause is executed even if \lstinline|return| is run.

\begin{urbiscript}
function with_return(var enable)
{
  try
  {
    echo("before return");
    if (enable)
      return;
    echo("after return");
  }
  finally
  {
    echo("finally");
  };
  echo("after try-block")
}|

with_return(false);
[00001983] *** before return
[00001985] *** after return
[00001985] *** finally
[00001986] *** after try-block

with_return(true);
[00001991] *** before return
[00001992] *** finally
\end{urbiscript}

It is also the case when the control flow is disrupted by
\lstinline|continue| or \lstinline|break|.

\begin{urbiscript}
for (var i : ["1", "continue", "2", "break", "3"])
  try
  {
    echo("before:  " + i);
    switch (i)
    {
      case "break":    break;
      case "continue": continue;
    };
    echo("after:   " + i);
  }
  finally
  {
    echo("finally: " + i);
  };
[00000663] *** before:  1
[00000671] *** after:   1
[00000671] *** finally: 1
[00000673] *** before:  continue
[00000675] *** finally: continue
[00000682] *** before:  2
[00000703] *** after:   2
[00000703] *** finally: 2
[00000704] *** before:  break
[00000705] *** finally: break
\end{urbiscript}


\subsubsection{Exceptions}

Exceptions caught in the try-catch clause are much like a regular execution
flow.  In particular, the value of the construct is that of the try-catch
clause regardless of the execution of the \lstinline|finally| clause.

\begin{urbiscript}
try           { echo("try");     "try" }
catch (var e) { echo("catch");   "catch" }
finally       { echo("finally"); "finally" };
[00000614] *** try
[00000615] *** finally
[00000616] "try"

try           { echo("try");     "try" }
catch (var e) { echo("catch");   "catch" }
else          { echo("else");    "else" }
finally       { echo("finally"); "finally" };
[00000614] *** try
[00000615] *** else
[00000615] *** finally
[00000616] "else"

try                      { echo("throw 42"); throw 42; "try" }
catch (var e if e == 42) { echo("caught " + e);        "catch" }
finally                  { echo("finally");            "finally" };
[00000626] *** throw 42
[00000626] *** caught 42
[00000631] *** finally
[00000631] "catch"
\end{urbiscript}


Uncaught exceptions (i.e., exceptions for which there were no handlers) are
propagated after the exception of the finally-clause.

\begin{urbiscript}
try                      { echo("throw"); throw 51; "try" }
catch (var e if e == 42) { echo("caught " + e);     "catch" }
finally                  { echo("finally");         "finally" };
[00000616] *** throw
[00000617] *** finally
[00000625:error] !!! 51
\end{urbiscript}

Exceptions launched in the finally-clause override previous exceptions.

\begin{urbiscript}
try     { throw "throw" }
catch   { throw "catch" }
finally { throw "finally" };
[00005200:error] !!! finally
\end{urbiscript}


%% FIXME: \subsection{Exceptions and parallelism}

\section{Assertions}
\label{sec:lang:assert}

\dfn[assertion]{Assertions} allow to embed consistency checks in the code.
They are particularly useful when developing a program since they allow
early catching of errors.  Yet, they can be costly in production mode: the
run-time cost of verifying every single assertion might be prohibitive.
Therefore, as in \C-like languages, assertions are disabled when
\refSlot[System]{ndebug} is true, see \refObject{System}.

\us supports assertions in two different ways: with a function-like
syntax, which is adequate for single claims, and a block-like syntax,
to group claims together.

\subsection{Asserting an Expression}

\begin{urbiscript}
assert (true);
assert (42);
\end{urbiscript}

Failed assertions are displayed in a user friendly fashion: first the
assertion is displayed before evaluation, then the effective values are
reported.

\begin{urbiscript}
function fail () { false }|;
assert (fail);
[00010239:error] !!! failed assertion: fail (fail == false)

function lazyFail { call.evalArgAt(0); false }|;
assert (lazyFail(1+2, "+" * 2));
[00010241:error] !!! failed assertion: lazyFail(1.'+'(2), "+".'*'(2)) (lazyFail(3, ?) == false)
\end{urbiscript}

The following example is more realistic.

\begin{urbiscript}
function areEqual(var args[])
{
  var res = true;
  if (!args.empty)
  {
    var a = args[0];
    for (var b : args.tail)
      if (a != b)
      {
        res = false;
        break;
      }
  };
  res
}|;
assert (areEqual);
assert (areEqual(1));
assert (areEqual(1, 0 + 1));
assert (areEqual(1, 1, 1+1));
[00001388:error] !!! failed assertion: areEqual(1, 1, 1.'+'(1)) (areEqual(1, 1, 2) == false)
assert (areEqual(1*2, 1+2));
[00001393:error] !!! failed assertion: areEqual(1.'*'(2), 1.'+'(2)) (areEqual(2, 3) == false)
\end{urbiscript}

Comparison operators are recognized, and displayed specially:
\begin{urbiscript}
assert (1 == 1 + 1);
[00000002:error] !!! failed assertion: 1 == 1.'+'(1) (1 != 2)
\end{urbiscript}

Note however that if opposite comparison operators are absurd (i.e., if for
instance \lstinline|a == b| is not true, but \lstinline|a != b| is not true
either), them the message is unlikely to make sense.

\subsection{Assertion Blocks}

Groups of assertions are more readable when used with the
\lstinline|assert{\var{exp1}; \var{exp2}; ...}| construct.  The (possibly
empty) list of claims may be ended with a semicolon.

\begin{urbiscript}
assert
{
  true;
  42;
  1 == 1 + 1;
};
[00000002:error] !!! failed assertion: 1 == 1.'+'(1) (1 != 2)
\end{urbiscript}

Variable declarations are valid in assertion blocks, in which case their
scope is, of course, limited to the block.  If the variable is initialized,
the initial value is \emph{not} asserted.  Contrast the following two
similar examples.

\begin{urbiscript}
assert
{
  var x = 0;
  ! x;
};

assert
{
  var x;
  x = 0;
  ! x;
};
[00000576:error] !!! failed assertion: x = 0
\end{urbiscript}

For sake of readability and compactness, this documentation shows assertion
blocks as follows (see \autoref{sec:notations} for a description of the
notations).

\begin{urbiassert}
true;
42;
1 == 1 + 1;
[00000002:error] !!! failed assertion: 1 == 1.'+'(1) (1 != 2)
\end{urbiassert}


\section{Parallel Flow Control}

This section defines the syntactic constructs that allows concurrent
executions.  Some of these constructs, loop-like, are concurrent alternative
of traditional imperative constructs.

\subsection{\lstinline'every'}
\label{sec:lang:every}
The \lstindex{every} statement enables to execute a block of code
repeatedly, with the given period.

\begin{urbiscript}
// Print out a message every second.
timeout (2.1s)
  every (1s)
    echo("Are you still there?");
[00000000] *** Are you still there?
[00001000] *** Are you still there?
[00002000] *** Are you still there?
\end{urbiscript}

It exists in several flavors, with the same syntax:

\begin{bnf}
<every-statement>
  ::= "every"  "(" <expression> ")" <statement>
    | "every|" "(" <expression> ")" <statement>
    | "every," "(" <expression> ")" <statement>
\end{bnf}

\subsubsection{\lstinline'every|'}
The whole \lstinline'every|' statement itself remains in foreground:
statements attached after it with \lstinline';' or \lstinline'|' will
not be reached unless you \lstinline'break' out of it.  You may use
\lstinline|continue| to finish one iteration.  In that case, the
following iteration is not immediately started, it will be launched as
expected, at the given period.

% We used to use 100ms instead of 1s, but severely loaded machines
% (the Mac Mini) fail way too often.
\begin{urbiscript}
{
  var count = 4;
  var start = time;
  echo("before");
  every| (1s)
  {
    count -= 1;
    echo("begin: %s @ %1.0fs" % [count, time - start]);
    if (count == 2)
      continue;
    if (count == 0)
      break;
    echo("end:   " + count);
  };
  echo("after");
};
[00000597] *** before
[00000598] *** begin: 3 @ 0s
[00000599] *** end:   3
[00000698] *** begin: 2 @ 1s
[00000798] *** begin: 1 @ 2s
[00000799] *** end:   1
[00000898] *** begin: 0 @ 3s
[00000899] *** after
\end{urbiscript}

The \lstinline{every|} flavor does not let iterations overlap. If an
iteration takes too long, the following iterations are delayed. That
is, the next iterations will start immediately after the end of the
current one, and next iterations will occur normally from this point.

\begin{urbiscript}
{
  var too_long = true|;

  var count = 5;
  // Every other iteration exceeds the period, and will delay the
  // following one.
  every| (1s)
  {
    if (! count -=1)
      break;

    if (too_long)
    {
      too_long = false;
      echo("Long in");
      sleep(1.5s);
      echo("Long out");
    }
    else
    {
      too_long = true;
      echo("Short");
    };
  };
};
[00000000] *** Long in
[00001500] *** Long out
[00001500] *** Short
[00002500] *** Long in
[00004000] *** Long out
[00004000] *** Short
\end{urbiscript}

The flow-control constructs \lstinline|break| and \lstinline|continue|
are supported.

\begin{urbiscript}
{
  var count = 0;
  every| (250ms)
  {
    count += 1;
    if (count == 2)
      continue;
    if (count == 4)
      break;
    echo(count);
  }
};
/*(*/sleep(2s);/*)*/
[00000000] *** 1
[00001500] *** 3
\end{urbiscript}



\subsubsection{\lstinline'every,'}
The default flavor, \lstinline|every,|, launches the execution of the
block in the background every given period. Iterations may overlap.

% Cut the previous every, no
\begin{urbiscript}[firstnumber=1]
// If an iteration is longer than the given period, it will overlap
// with the next one.
timeout (2.8s)
  every (1s)
  {
    echo("In");
    sleep(1.5s);
    echo("Out");
  };
[00000000] *** In
[00001000] *** In
[00001500] *** Out
[00002000] *** In
[00002500] *** Out
\end{urbiscript}

\subsection{for}

The \lstinline|for| loops come into several flavors, depending one the
actual kind of \lstinline|for| loop.

\subsubsection{C-for,}
\experimentalremoved{}

\lstinline|for,| is syntactic sugar for \lstinline|while,|, see
\autoref{sec:lang:while:comma}.

\begin{urbiscript}
for, (var i = 3; 0 < i; i -= 1)
{
  var j = i |
  echo ("in: i = %s, j = %s" % [i, j]);
  sleep(j/10);
  echo ("out: i = %s, j = %s" % [i, j]);
};
echo ("done");
[00000144] *** in: i = 3, j = 3
[00000145] *** in: i = 2, j = 2
[00000145] *** in: i = 1, j = 1
[00000246] *** out: i = 0, j = 1
[00000346] *** out: i = 0, j = 2
[00000445] *** out: i = 0, j = 3
[00000446] *** done
\end{urbiscript}

\begin{urbiscript}
for, (var i = 9; 0 < i; i -= 1)
{
  var j = i;
  if (j % 2)
    continue
  else if (j == 4)
    break
  else
    echo("%s: done" % j)
};
echo("done");
[00000146] *** 8: done
[00000148] *** 6: done
[00000150] *** done
\end{urbiscript}


\subsubsection{range-for\&}
\label{sec:lang:for:each:and}

One can iterate concurrently over the members of a collection.

\begin{urbiscript}
for& (var i: [0, 1, 2])
{
  echo (i * i);
  echo (i * i);
};
[00000000] *** 0
[00000000] *** 1
[00000000] *** 4
[00000000] *** 0
[00000000] *** 1
[00000000] *** 4
\end{urbiscript}

If an iteration executes \lstinline|continue|, it is stopped; the
other iterations are not affected.

\begin{urbiscript}
for& (var i: [0, 1, 2])
{
  var j = i;
  if (j == 1)
    continue;
  echo (j);
};
[00020653] *** 0
[00021054] *** 2
\end{urbiscript}

If an iteration executes \lstinline|break|, all the iterations
including this one, are stopped.

\begin{urbiscript}
for& (var i: [0, 1, 2])
{
  var j = i;
  echo (j);
  if (j == 1)
   { echo ("break");
    break;};
  sleep(1s);
  echo (j);
};
[00000001] *** 0
[00000001] *** 1
[00000001] *** 2
[00000002] *** break
\end{urbiscript}

\subsubsection{for\& (n)}

Since \lstinline|for& (\var{n}) \var{body}| is processed as
\lstinline|for& (var \var{tmp}: \var{n}) \var{body}|, which \var{tmp}
a hidden variable, see \autoref{sec:lang:for:each:and} for details.


\subsection{loop,}
\experimentalremoved{}

This is syntactic sugar for \lstinline|while,(true)|.  In the
following example, care must be taken that concurrent executions don't
modify \lstinline|n| simultaneously.  This would happen had
\lstinline|;| been used instead of \lstinline'|'.

\begin{urbiassert}
{
  var n = 10|;
  var res = []|;
  loop,
  {
    n-- |
    res << n |
    if (n == 0)
      break
  };
  res.sort
}
==
[0, 1, 2, 3, 4, 5, 6, 7, 8, 9];
\end{urbiassert}

\subsection{timeout}
\label{sec:lang:timeout}

The \lstinline|timeout| construct allows to interrupt a piece of code that
took too long to run.  Its syntax is based on the
\lstinline|try|/\lstinline|catch| construct (\autoref{sec:lang:catch}).

\begin{bnf}
<timeout-statement>
::= "timeout" "(" <float> ")" <statement> <catch-all-clause>? <else-clause>? <finally-clause>?

<catch-all-clause>
::= "catch" "{" <statement>* "}"

<else-clause>
::= "else" "{" <statement>* "}"

<finally-clause>
::= "finally" "{" <statement>* "}"
\end{bnf}

If the statement finishes its expiration before \var{duration} expires, then
the optional \lstinline|else|-clause is run.  Otherwise, if \var{duration}
expires, the optional \lstinline|catch|-clause is run.  In both cases, then
the optional \lstinline|finally|-clause is executed.

The value of the \lstinline|timeout|-statement is that of the last
completed non-\lstinline|finally| block that was run; \lstinline|void| if
no block was completed.  The value of the \lstinline|finally| is always
ignored, because this clause is meant to release resources, and in typical
uses it does not contribute to the computation.

Because there are too many cases to demonstrate, only some of them are
exemplified below.
\begin{urbiscript}
// timeout alone.
timeout (1s)  { echo("body"); sleep(0.5s); "body-value" };
[00000002] *** body
[00000004] "body-value"

timeout (1s)  { echo("body"); sleep(2s); "body-value" };
[00000002] *** body

// timeout and catch.
timeout (1s)  { echo("body"); sleep(0.5s); "body-value" }
catch         { echo("catch"); "catch-value" };
[00000002] *** body
[00000002] "body-value"

timeout (1s)  { echo("body"); sleep(2s); "body-value" }
catch         { echo("catch"); "catch-value" };
[00000002] *** body
[00000002] *** catch
[00000002] "catch-value"

// timeout and else.
timeout (1s)  { echo("body"); sleep(0.5s); "body-value" }
else          { echo("else"); "else-value" };
[00000002] *** body
[00000002] *** else
[00000002] "else-value"

timeout (1s)  { echo("body"); sleep(2s); "body-value" }
else          { echo("else"); "else-value" };
[00000002] *** body

// timeout, catch, and else.
timeout (1s)  { echo("body"); sleep(0.5s); "body-value" }
catch         { echo("catch"); "catch-value" }
else          { echo("else"); "else-value" };
[00000002] *** body
[00000002] *** else
[00000002] "else-value"

timeout (1s)  { echo("body"); sleep(2s); "body-value" }
catch         { echo("catch"); "catch-value" }
else          { echo("else"); "else-value" };
[00000002] *** body
[00000002] *** catch
[00000002] "catch-value"

// timeout, catch, else and finally.
timeout (1s)  { echo("body"); sleep(0.5s); "body-value" }
catch         { echo("catch"); "catch-value" }
else          { echo("else"); "else-value" }
finally       { echo("finally"); "finally-value" };
[00000002] *** body
[00000002] *** else
[00000002] *** finally
[00000002] "else-value"

timeout (1s)  { echo("body"); sleep(2s); "body-value" }
catch         { echo("catch"); "catch-value" }
else          { echo("else"); "else-value" }
finally       { echo("finally"); "finally-value" };
[00000002] *** body
[00000002] *** catch
[00000002] *** finally
[00000002] "catch-value"
\end{urbiscript}

If an exception is thrown from the \lstinline|timeout|-clause, it is
\emph{not} caught by the \lstinline|catch|-clause (nor the
\lstinline|else|-clause), rather it continues its propagation to the outer
statements.

\begin{urbiscript}
timeout (1s)  { echo("body"); throw Exception.new("Ouch") }
catch         { echo("catch"); "catch-value" }
else          { echo("else"); "else-value" }
finally       { echo("finally"); "finally-value" };
[00000566] *** body
[00000569] *** finally
[00000585:error] !!! Ouch
\end{urbiscript}

Alternatively, you may use the \refObject{Timeout} object.

\subsection{While}
\subsubsection{\lstinline|while,|}
\label{sec:lang:while:comma}
\experimentalremoved{}

This construct provides a means to run concurrently multiple instances
of statements.  The semantics of

\begin{urbiunchecked}
while, (\var{condition})
  \var{body};
\end{urbiunchecked}

\noindent
is the same as

\begin{urbiunchecked}
\var{condition} | \var{body} , \var{condition} | \var{body} , ...
\end{urbiunchecked}

Attention must be paid to the fact that the (concurrent) iterations
share a common access to the environment, therefore if, for instance,
you want to keep the value of some index variable, use a local
variable inside the loop body:

% Cut the previous every, no
\begin{urbiscript}[firstnumber=1]
{
  var i = 4|
  while, (i)
  {
    var j = i -= 1;
    echo ("in: i = %s, j = %s" % [i, j]);
    sleep(j/10);
    echo ("out: i = %s, j = %s" % [i, j]);
  }|
  echo ("done");
}|
[00000144] *** in: i = 2, j = 3
[00000145] *** in: i = 1, j = 2
[00000145] *** in: i = 0, j = 1
[00000146] *** in: i = 0, j = 0
[00000146] *** out: i = 0, j = 0
[00000246] *** out: i = 0, j = 1
[00000346] *** out: i = 0, j = 2
[00000445] *** out: i = 0, j = 3
[00000446] *** done
\end{urbiscript}

As for the other flavors, \lstinline|continue| skips the current iteration,
and \lstinline|break| ends the loop.  Note that \lstinline|break| stops all
the running iterations.  This semantics is likely to be changed to
``\lstinline|break| ends the current iteration and stops the generation of
others, but lets the other concurrent iterations finish'', so do not rely on
this feature.

Control flow is passed to the following statement when all the iterations
are done.

\begin{urbiscript}
{
  var i = 10|
  while, (i)
  {
    var j = i -= 1;
    if (j % 2)
      continue
    else if (j == 4)
      break
    else
      echo("%s: done" % j)
  }|
  echo("done");
};
[00000146] *** 8: done
[00000148] *** 6: done
[00000150] *** done
\end{urbiscript}


\section{Event Handling}

\subsection{\lstinline'at'}
\label{sec:lang:at}
Using the \lstindex{at} construct, one can arm code that will be
triggered each time some condition is true.

The \lstinline'at' construct is as follows:

\begin{urbiunchecked}
at (\var{condition})
  \var{statement1}
onleave
  \var{statement2}
\end{urbiunchecked}

The \var{condition} can be of two different kinds:
\lstinline|\var{e}?(\var{args})| to catch when events are sent, or
\lstinline|\var{exp}| to catch each time a Boolean \var{exp} becomes true.

The \lstindex{onleave}-clause is optional.  Note that, as is the case for
the \lstinline|if| statement, there must \emph{not} be a semicolon after
\var{statement1} if there is an \lstinline|onleave| clause.

\subsubsection{\lstinline'at' on Events}
%% FIXME: More details are needed.  Don't bounce elsewhere.
See \autoref{sec:tut:events} for an example of using \lstinline|at|
statements to watch events.

\paragraph{Durations}

Since events may last for a given duration
%% Beware of wrapping.
(\lstinline|e! ~ \var{duration}|),
%%
event handlers may also require an event to be sustained for a given amount
of time before being ``accepted'' (\lstinline|at (e? ~ \var{duration})|).

\begin{urbiscript}[firstnumber=1]
var e = Event.new|;
at (e?(var start) ~ 1s)
  echo("in : %s" % (time - start).round)
onleave
  echo("out: %s" % (time - start).round);

// This emission is too short to trigger the at.
e!(time);

// This one is long enough.
// The body triggers 1s after the emission started.
e!(time) ~ 2s;
[00001000] *** in : 1
[00002000] *** out: 2
\end{urbiscript}


\subsubsection{\lstinline'at' on Boolean Expressions}

The \lstinline|at| construct can be used to watch a given Boolean
expression.

\begin{urbiscript}[firstnumber=1]
var x = 0 |
var x_is_two = false |
at (x == 2)
  x_is_two = true
onleave
  x_is_two = false;

x = 3|;  assert(!x_is_two);
x = 2|;  assert( x_is_two);
x = 2|;  assert( x_is_two);
x = 3|;  assert(!x_is_two);
\end{urbiscript}

It can also wait for some condition to hold long enough:
\lstinline|\var{exp} ~ \var{duration}|, as a condition, denotes the
fact that \var{exp} was true for \var{duration} seconds.

\begin{urbiscript}[firstnumber=1]
var x = 0 |
var x_was_two_for_two_seconds = false |
at (x == 2 ~ 2s)
  x_was_two_for_two_seconds = true
onleave
  x_was_two_for_two_seconds = false;

x = 2       | assert(!x_was_two_for_two_seconds);
sleep(1.5s) | assert(!x_was_two_for_two_seconds);
sleep(1.5s) | assert( x_was_two_for_two_seconds);

x = 3|; sleep(0.1s);  assert(!x_was_two_for_two_seconds);

x = 2       | assert(!x_was_two_for_two_seconds);
sleep(1.5s) | assert(!x_was_two_for_two_seconds);
x = 3|; x = 2|; sleep (1s) | assert(!x_was_two_for_two_seconds);
\end{urbiscript}

\subsubsection{Synchronous and asynchronous \lstinline'at'}
\label{sec:lang:at:sync-async}

By default, \lstinline{at} is asynchronous: the enter and leave actions are
executed in detached jobs and won't interfere with the execution flow of the
job that triggered it.

\begin{urbiscript}[firstnumber=1]
var e = Event.new;
[00000001] Event_0xADDR

at (e?)
{
  sleep(1s);
  echo("in");
}
onleave
{
  sleep(2s);
  echo("out");
};

e!;
// Actions are triggered in the background and won't block
// the execution flow.
sleep(500ms);
echo("Not blocked");
[00000002] *** Not blocked
sleep(1s);
[00000003] *** in
echo("Not blocked");
[00000004] *** Not blocked
sleep(500ms);
[00000003] *** out
\end{urbiscript}

When using the \lstinline{sync} keyword after \lstinline{at}, it becomes
synchronous: when a job triggers it, all enter and leave actions are
executed synchronously before the triggering statement returns.

\begin{urbiscript}[firstnumber=1]
var e = Event.new;
[00000001] Event_0xADDR

at sync (e?)
{
  sleep(1s);
  echo("in");
}
onleave
{
  sleep(1s);
  echo("out");
};

e!;
// Actions are triggered synchronously, the next line will be executed
// when they're done.
echo("Blocked");
[00000002] *** in
[00000003] *** out
[00000004] *** Blocked
\end{urbiscript}

Alternatively, the event emission may request a synchronous handling, see
\autoref{sec:event:sync}.

\subsubsection{Scoping at \lstinline'at'}

\lstinline'at' statements are not scoped.  But, using a \refObject{Tag}
object, one can control them.  In the following example,
\refSlot[Tag]{scope} is used to label the \lstinline|at| statement.  When
the function ends, the \lstinline|at| is no longer active.

\begin{urbiscript}[firstnumber=1]
var x = 0 |
var x_is_two = false |;

{
  Tag.scope:
    at (x == 2)
      x_is_two = true
    onleave
      x_is_two = false;
  sleep(2s);
},
x = 2 |; assert(x_is_two);
x = 1 |; assert(!x_is_two);
sleep(3s);
x = 2 | assert(!x_is_two);
\end{urbiscript}

\subsection{\lstinline|waituntil|}
\label{sec:lang:waituntil}

The \lstinline|waituntil| construct is used to hold the execution
until some condition is verified.  Similarly to \lstinline|at|
(\autoref{sec:lang:at}) and the other event-based constructs,
\lstinline|waituntil| may work on events, or on Boolean expressions.

\subsubsection{\lstinline'waituntil' on Events}

When the execution flow enters a \lstinline|waituntil|, the execution
flow is held until the event is fired.  Once caught, the event is
consumed, another \lstinline|waituntil| will require another event
emission.

\begin{urbiscript}
{
  var e = Event.new;
  {
    waituntil (e?);
    echo ("caught e");
  },
  e!;
[00021054] *** caught e
  e!;
};
\end{urbiscript}

In the case of lasting events (see \refSlot[Event]{trigger}), the
condition remains verified as long as the event is ``on''.

\begin{urbiscript}
{
  var e = Event.new;
  e.trigger;
  {
    waituntil (e?);
    echo ("caught e");
  };
[00021054] *** caught e
  {
    waituntil (e?);
    echo ("caught e");
  };
[00021054] *** caught e
  {
    waituntil (e?);
    echo ("caught e");
  };
[00021054] *** caught e
};
\end{urbiscript}

The event specification may use pattern-matching to specify the
accepted events.

\begin{urbiscript}
{
  var e = Event.new;
  {
    waituntil (e?(1, var b));
    echo ("caught e(1, %s)" % b);
  },
  e!;
  e!(1);
  e!(2, 2);
  e!(1, 2);
[00021054] *** caught e(1, 2)
  e!(1, 2);
};
\end{urbiscript}

Events sent before do not release the construct.

\begin{urbiscript}
{
  var e = Event.new;
  e!;
  {
    waituntil (e?);
    echo ("caught e");
  },
  e!;
[00021054] *** caught e
};
\end{urbiscript}

\subsubsection{\lstinline'waituntil' on Boolean Expressions}

You may use any expression that evaluates to a truth value as argument
to \lstinline'waituntil'.

\begin{urbiscript}
{
  var foo = Object.new;
  {
    waituntil (foo.hasLocalSlot("bar"));
    echo(foo.getLocalSlot("bar"));
  },
  var foo.bar = 123|;
};
[00021054] *** 123
\end{urbiscript}

\subsection{watch}
\label{sec:lang:watch}

The \lstindex{watch} construct is similar in spirit to using the
\lstinline|at| construct to monitor expressions, except it enables you to be
notified when an arbitrary expression changed, not only when it becomes true
or false. This makes \lstinline|watch| a more primitive tool than
\lstinline|at| on expressions.  Actually, \lstinline|at| on expressions uses
\lstinline|watch| to determine when to reevaluate its condition.

\lstinline|watch(\var{expression})| evaluates to an \refObject{Event} that
triggers every time \var{expression} changes, with its new value as payload.

\begin{urbiscript}[firstnumber=1]
var x = 0;
[00000000] 0
var y = 0;
[00000000] 0
var e = watch(x + y);
[00000000] Event_0x103a1e978
at (e?(var value))
  echo("x + y = %s" % value);
x = 1;
[00000000] 1
[00000000] *** x + y = 1
y = 2;
[00000000] 2
[00000000] *** x + y = 3
\end{urbiscript}

Note that ``the expression changed'' might be ambiguous: Urbi considers the
expression to have changed when any component involved in its evaluation
changed.  If a \refObject{Float} is replaced with another Float of the same
value, the expression has changed, since the new Float may have different
slots.

\begin{urbiscript}[firstnumber=1]
var x = 0;
[00000000] 0
at (watch(x)?(var value))
  echo("x = %s" % value);
// This is considered as a change, although the new float value is also 0.
x = 0;
[00000000] 0
[00000000] *** x = 0
\end{urbiscript}

Also, some modification may modify the evaluation, but still yield the same
result.

\begin{urbiscript}[firstnumber=1]
var x = 1;
[00000000] 1
at (watch(x % 2)?(var value))
  echo("x %% 2 = %s" % value);

// This is considered as a change, although the computation yields the
// same result.
x = 3;
[00000000] 3
[00000000] *** x % 2 = 1
\end{urbiscript}


\subsection{\lstinline|whenever|}
\label{sec:lang:whenever}

The \lstindex{whenever} construct really behaves like a never-ending
\lstinline|loop if| construct.  It also works on events and Boolean
expressions, and triggers each time the condition \emph{becomes} verified.

\begin{urbiunchecked}
whenever (\var{condition})
  \var{statement1}
\end{urbiunchecked}

It supports an optional \lstinline|else| clause, which is run whenever
the condition changes ``from true to false''.

\begin{urbiunchecked}
whenever (\var{condition})
  \var{statement1}
else
  \var{statement2}
\end{urbiunchecked}

The execution of a \lstinline|whenever| clause is ``instantaneous'',
there is no mean to use \samp{,} to put it in background.  It is also
asynchronous with respect to the condition: the emission of an event
is not held until all its watchers have completed their job.

\subsubsection{\lstinline'whenever' on Events}

A \lstinline'whenever' clause can be used to catch events with or
without payloads.

\begin{urbiunchecked}[firstnumber=1]
var e = Event.new|;
whenever (e?)
  echo("e on")
else
  echo("e off");
[00000001] *** e off
[00000002] *** e off
[00000003] *** ...
e!;
[00000004] *** e on
[00000005] *** e off
[00000006] *** e off
[00000007] *** ...
e!(1) & e!(2);
[00000008] *** e on
[00000009] *** e on
[00000010] *** e off
[00000011] *** e off
[00000012] *** ...
\end{urbiunchecked}

The pattern-matching and guard on the payload is available.

\begin{urbiunchecked}[firstnumber=1]
var e = Event.new|;
whenever (e?("arg", var arg) if arg % 2)
  echo("e (%s) on" % arg)
else
  echo("e off");
e!("param", 23);
e!("arg", 52);
e!("arg", 23);
[00000001] *** e (23) on
[00000002] *** e off
[00000003] *** e off
[00000004] *** ...
e!("arg", 52);
e!("arg", 17);
[00000005] *** e (17) on
[00000006] *** e off
[00000007] *** e off
[00000008] *** ...
\end{urbiunchecked}


If the body of the \lstinline|whenever| lasts for a long time, it is
possible that two executions be run concurrently.

\begin{urbiscript}[firstnumber=1]
var e = Event.new|;
whenever (e?(var d))
{
  echo("e (%s) on begin" % d);
  sleep(d);
  echo("e (%s) on end" % d);
};

e!(0.3s) & e!(1s);
sleep(3s);
[00000202] *** e (1) on begin
[00000202] *** e (0.3) on begin
[00000508] *** e (0.3) on end
[00001208] *** e (1) on end
\end{urbiscript}

\subsubsection{\lstinline'whenever' on Boolean Expressions}

A \lstindex{whenever} construct will repeatedly evaluate its body as long as
its condition holds.  The number of evaluation of the bodies is typically
non-deterministic, as not only does it depend on how long the condition
holds, but also ``how fast'' the \urbi kernel runs.

\begin{urbiscript}[firstnumber=1]
var x = 0|;
var count = 0|;
var t = Tag.new|;
t:
  whenever (x % 2)
  {
    if (!count)
      echo("x is now odd (%s)" % x);
    count++;
  }
  else
  {
    if (!count)
      echo("x is now even (%s)" % x);
    count++;
  };

t:
  whenever (100 < count)
  {
    count = 0 |
    x++;
  };
waituntil(x == 4);
[00000769] *** x is now even (0)
[00000809] *** x is now odd (1)
[00000846] *** x is now even (2)
[00000886] *** x is now odd (3)
[00000924] *** x is now even (4)
t.stop;
\end{urbiscript}



\section{Trajectories}
\label{sec:lang:traj}

In robotics, \dfn[trajectory]{trajectories} are often used: they are a
means to change the value of a variable (actually, a slot) over time.
This can be done using detached executions, for instance using a
combination of \lstinline|every| and \lstinline|detach|, but \us
provides syntactic sugar to this end.

For instance the following drawing shows how the \lstinline|y|
variable is moved smoothly from its \dfn{initial value}
(\lstinline|0|) to its \dfn{target value} (\lstinline|100|) in 3
seconds (the value given to the \lstinline|smooth| \dfn{attribute}.

\urbitrajectory{smooth}

Trajectories can be frozen and unfrozen, using tags
(\autoref{sec:tut:tags}).  In that case, ``time is suspended'', and
the trajectory resumes as if the trajectory was never interrupted.

\urbitrajectory{smooth-frozen}

When the target value is reached, the trajectory generator is detached
from the variables: changes to the value of the variable no longer
trigger the trajectory generator.

\urbitrajectory{smooth-continued}

See the specifications of \refObject{TrajectoryGenerator} for the list
of supported trajectories.

\section{Garbage Collection and Limitations}
\label{sec:lang:gc}

\us provides automatic garbage collection. That is, you can create new
objects and don't have to worry about reclaiming the memory when you're done
with them. We use a reference counting algorithm for garbage collection:
every object has a counter indicating how many references to it exist. When
that counter drops to zero, nobody has a reference to the object anymore,
and it is thus deleted.

\begin{urbiunchecked}
{
  var x = List.new; // A new list is allocated
  x << 42;
};
// The list will be automatically freed, since there are no references to it left.
\end{urbiunchecked}

This is not part of the language interface, and we might change the garbage
collecting system in the future. Therefore, do not rely on the current
garbage collecting behavior, and especially not on the determinism of the
destruction time of objects.

However, this implementation has a limitation you should be aware of: cycle
of object references won't be properly reclaimed. Indeed if object A has a
reference to B, and B has a reference to A, none of them will ever be
reclaimed since they both have a reference pointing to them. As a
consequence, avoid creating cycles in object references, or if you really
have to, break the cycle manually before releasing your last reference to
the object.

\begin{urbiunchecked}
// Create a reference cycle
var A = Object.new;
var A.B = Object.new; // A refers to B
var A.B.A = A; // B refers back to A

removeLocalSlot("A"); // delete our last reference to A
// Although we have no reference left to A or B,
// they won't be deleted since they refer to each other.
\end{urbiunchecked}

If you really need the cycle, this is how you could break it manually:

\begin{urbiunchecked}
A.B.removeLocalSlot("A"); // Break the cycle
removeLocalSlot("A"); // Delete our last reference to A
// A will be deleted since it's not referred from anywhere.
// Since A held the last reference to B, B will be deleted too.
\end{urbiunchecked}

%%% Local Variables:
%%% mode: latex
%%% coding: utf-8
%%% TeX-master: "../urbi-sdk"
%%% ispell-dictionary: "american"
%%% ispell-personal-dictionary: "../urbi.dict"
%%% fill-column: 76
%%% End:
