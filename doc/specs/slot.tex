%% Copyright (C) 2010, 2011, Gostai S.A.S.
%%
%% This software is provided "as is" without warranty of any kind,
%% either expressed or implied, including but not limited to the
%% implied warranties of fitness for a particular purpose.
%%
%% See the LICENSE file for more information.

\section{Slot}

A slot is an intermediate object that embodies the concept of ``variable''
or ``field'' in \us.  It contains an underlying value, meta-information
about this value, and slots to alter the behavior of read and write
operations.

\subsection{Accessing the slot object}
Section \autoref{sec:lang:slots} describes how to access a slot object.

\subsection{Key features}

The contained value returned by default when \refSlot[Object]{getSlotValue}
is called is stored in the \refSlot{value} slot.

The \refSlot{changed} slot is an \refObject{Event} that is triggered each
time the slot is written to.

Setters and getters to modify the slot behaviors can be installed by writing
to \refSlot{set}, \refSlot{get}, \refSlot{oset} and \refSlot{oget}.

Two slots can be linked together to build dataflows using operator
\refSlot{>>}

\subsection{Split mode}
\label{sec:lang:slot:split}

It can sometimes be convenient to store two values in one slot, one which is
read, and the other written. For instance, the \lstinline|val| slot of
a rotational motor Object can be the current motor position when reading, and
a target position to reach when writing.

This behavior is controlled by the \refSlot{split} slot.

\subsection{Construction}

Slots are automatically created when \refSlot[Object]{setSlot} is called.

\subsection{Prototypes}
\begin{refObjects}
\item[Object]
\end{refObjects}

\subsection{Slots}

\begin{urbiscriptapi}
\item[<<](<slot>)%
  Bounces to \refSlot{>>} reversing the two Slots.


\item[>>](<slot>)%
  The \refSlot{>>} operator connects two \refSlot{Slot} together through a
  \refSlot{Subscription}. Each time the source Slot is updated, its new value
  is written to the target Slot.
  This function should be used to bridge a component producing an output value
  to a component expecting an input value.
  It returns a \refSlot{Subscription} that can be used to configure the link,
  and gather statistics.

\begin{urbiscript}
var tick = 0|;
var tack = 0|;
var sub = &tick >> &tack;
[00000001] Slot_0x42389d88 >> Slot_0x42387c88
tack->set = function(v) { echo("tack " + v)}|;
timeout(10.5s) every|(1s) tick++,
sleep(1.5s) | sub.callCount;
[00000002] *** tack 1
[00000003] *** tack 2
[00000004] 2
sub.enabled = false| sleep(1s)| sub.enabled = true|;
sleep(2s);
[00000002] *** tack 4
[00000003] *** tack 5
sub.disconnect();
sleep(2s);
\end{urbiscript}


\item[changed]%
  Contains an \refObject{Event} which is emitted each time the slot value
  is written to. It is used by the system to implement the
  \lstinline{watch}-expression (\autoref{sec:lang:watch}).

\begin{urbiscript}[firstnumber=1]
var x = []|;

at (x->changed?) // Same thing as &x.changed
  echo("x->changed");

x = [1]|;
[00092656] *** x->changed
x = [1, 2]|;
[00092756] *** x->changed
\end{urbiscript}

Even if the slot is assigned to the very same value, the
\lstinline|x->changed| event is emitted.

\begin{urbiscript}
x = x|;
[00092856] *** x->changed
\end{urbiscript}

This is different from checking value \emph{updates}.  In the following
example, \lstinline{x} is not rebound to another list, it is the content of
the list that changes, therefore the \lstinline{changed} event is \emph{not}
fired:

\begin{urbiscript}
x << 3;
[00092866] [1, 2, 3]
\end{urbiscript}

To monitor changes of value, use the \lstinline{watch}-construct
(\autoref{sec:lang:watch}).


\item[constant]

Defines whether a slot can be assigned a new value.

\begin{urbiscript}
var c = 0;
[00000000] 0
c = 1;
[00000000] 1

c->constant = true;
[00000000] true
c = 2;
[00000000:error] !!! cannot modify const slot

c->constant = false;
[00000000] false
c = 3;
[00000000] 3
\end{urbiscript}

A new slot can be declared constant when first defined, in which case the
initial value is required.

\begin{urbiscript}
const var two;
[00000030:error] !!! syntax error: const declaration without a value
const var two = 2;
[00000036] 2
two = 3;
[00000037:error] !!! cannot modify const slot
two->constant;
[00000038] true
\end{urbiscript}

When a slot is declared constant, copy-on-write does not apply (see
\refSlot[Object]{updateSlot}).

\begin{urbiscript}
class Top
{
  const var x = 1;
}|;
var t = Top.new|;
t.x = 2;
[00000047:error] !!! cannot modify const slot
\end{urbiscript}


\item[copy](<targetObject>, <targetName>)%
  Duplicate the \lstinline|Slot| to the slot \var{targetName} of object
  \var{targetObject}.


\item[copyOnWrite]%
  If set to false, disables copy on write behavior for this slot
  (see \autoref{sec:lang:cow}).

\begin{urbiscript}
class a {
  var x = 0;
  var y = 0;
}|;
var b = a.new|;
a.&x.copyOnWrite = false|;
b.x = 1| b.y = 1|;
assert
{
  a.x == 1;
  a.y == 0;
};
\end{urbiscript}


\item[get]%
  Together with \refSlot{oset}, this slot can be set with a function that will
  be called each time the slot
  is accessed. If one exists, the \refSlot{value} slot is ignored, and the
  value returned from \refSlot{get} or \refSlot{oget} is used instead.
  Only one of them can be set. \refSlot{get} is called on the slot itself with
  no other argument, whereas \refSlot{oget} is called on the object who first
  owned the slot, with the slot as argument.

  Use \refSlot{get} when all the information needed to compute the value are
  in the slot itself.

\begin{urbiscript}
var counter = 0|;
var &counter.increment = 2|;
counter->get = function()
{ value += increment | value}|;
counter;
[00000001] 2
counter;
[00000002] 4
counter->increment = 3|;
counter;
[00000003] 7
\end{urbiscript}

Use \refSlot{oget} when the computation needs information in the object
owning the slot. Using \refSlot{oget} is better than using \refSlot{get} with
a closure.

\begin{urbiscript}
class Vector
{
  var x;
  var y;
  function init(x_, y_)
  {
    x = x_;
    y = y_;
  };
  var length;
  length->oget = function(slot)
  {
    ((x*x)+(y*y)).sqrt
  };
}|;
var v = Vector.new(2, 0)|;
v.length;
[00000001] 2
\end{urbiscript}

\begin{urbicomment}
removeSlot("Vector")|;
removeSlot("v")|;
\end{urbicomment}

\item[oget]%
  Similar to \refSlot{get}, but with a different signature: the callback
  function is called on the object owning the slot, instead of the slot
  itself.

\item[set]%
  Together with \refSlot{oset}, this slot can be set with a function that will
  be called each time the
  slot is written to. If one exists, the default behavior that simply writes
  the value to \refSlot{value} is disabled.
  \refSlot{set} is called on the slot itself, with the value as its sole
  argument. \refSlot{oset} is called on the object owning the slot, with
  the value and the slot as arguments.

  Use \refSlot{set} when the operation performed by your function needs no
  information outside the slot.

  Only the object owning the slot should use \refSlot{set} and \refSlot{oset}.
  Other objects should use \refSlot{changed}, watch constructs
  ((\autoref{sec:lang:watch})) or \refSlot{>>}.

\begin{urbiscript}
var integer = 0|;
integer->set = function(v) { value = v.round}|;
integer = 1.6 | integer;
[00000001] 2
\end{urbiscript}

  Use \refSlot{oset} when you need access to the object from your function.

\begin{urbiscript}
class Vector
{
  var x;
  var y;
  function init(x_, y_)
  {
    x = x_;
    y = y_;
  };
  var length;
  length->oget = function(slot) { ((x*x)+(y*y)).sqrt };
  /// Change the vector length, keeping the same direction
  length->oset = function(value, slot)
  {
    var ratio = value / length;
    x *= ratio;
    y *= ratio;
  };
}|;
var v = Vector.new(2, 0)|;
v.length *= 2|;
v.x;
[00000001] 4
v.y;
[00000002] 0
\end{urbiscript}

\item[oset]%
  Similar to \refSlot{set} but with a different signature: the function is
  called on the object owning the slot instead of the slot itself.


\item[notifyAccess](<onAccess>)%
  Deprecated, use \refSlot{get} or \refSlot{oget}.
  Similar to the \Cxx \lstinline|UNotifyAccess|, calls \var{onAccess} each
  time the \lstinline|Slot| is accessed (read).

\begin{urbiscript}
var Global.counter = 0|
var Global.access = 0|
var accessHandle = &access.notifyAccess(closure() {
  Global.access = ++Global.counter
})|
assert
{
  access == 1;
  access == 2;
  access == 3;
};
&access.removeNotifyAccess(accessHandle)|;
assert
{
  access == 3;
  access == 3;
};
\end{urbiscript}


\item[notifyChange](<onChange>)%
  Deprecated, use \refSlot{>>} or \refSlot{changed}.
  Similar to the \Cxx \lstinline|UNotifyChange| (see
  \autoref{sec:uobject:uvar-notify}), register \var{onChange} and call it
  each time this \lstinline|Slot| is written to.  Return an identifier that
  can be passed to \refSlot{removeNotifyChange} to unregister the callback.
  Subscribing to the \refSlot{changed} event has a similar effect.

\begin{urbiscript}
var Global.y = 0|
var handle = Global.&y.notifyChange(closure() {
  echo("The value is now " + Global.y)
})|
Global.y = 12;
[00000001] *** The value is now 12
[00000002] 12
Global.&y.removeNotifyChange(handle)|;
Global.y = 13;
[00000003] 13
\end{urbiscript}


\item[notifyChangeOwned](<onChangeOwned>)%
  Deprecated, this call now just set the \refSlot{set} slot.
  Similar to the \Cxx \lstinline|UNotifyChange| for a split Slot (see
  \autoref{sec:uobject:uvar-notify}), register \var{onChange} and call it
  each time this \lstinline|UVar| is written to.  Return an identifier that
  can be passed to \refSlot{removeNotifyChangeOwned} to unregister the
  callback.


\item[outputValue]%
  The value that is returned when a read occurs, if in \refSlot{split} mode.


\item[owned]%
  True if the \lstinline|Slot| is in \dfn{split mode}, that is if it
  contains both a sensor and a command value. This name is for backward
  compatibility.


\item[removeNotifyAccess](<id>)%
  Disable the notification installed as \var{id} by \refSlot{notifyAccess}.


\item[removeNotifyChange](<id>)%
  Disable the notification installed as \var{id} by \refSlot{notifyChange}.


\item[removeNotifyChangeOwned](<id>)%
  Disable the notification installed as \var{id} by
  \refSlot{notifyChangeOwned}.

\item[setOutputValue](<val>)%
  In \refSlot{split} mode, update output value slot \refSlot{outputValue} and
  trigger \refSlot{changed}.


\item[split]%
  Indicates that the slot has both an input value and an output value.  The
  input value is written to by external code, and the slot will act on it.
  The output value is updated by the slot itself and made visible to
  external code.

  This feature's intended use is to have both a sensor value and an actuator
  command accessible on the same slot.

  More formally, once activated by setting the \refSlot{split} slot to true:
\begin{itemize}
\item Writing to the slot no longer triggers \refSlot{changed}, but still
  updates \refSlot{value} and calls \refSlot{set}.
\item Reading the slot returns \refSlot{outputValue} instead of
  \refSlot{value}
\end{itemize}

Practically, code external to the slot owner continue to use the slot as
usual.  The object owning the slot must:
\begin{itemize}
\item Directly access \refSlot{value}, or use a setter to read the value
  written by external code.
\item Call \refSlot{setOutputValue} to update the value visible to external
  call.  This call will trigger the \refSlot{changed} event.
\end{itemize}

\begin{urbiscript}
class Motor
{
  function init()
  {
    // our val is both motor command, and motor current position
    var this.val = 0;
    var this.running = false;
    var this.runTag = Tag.new;
    &val.owned = true;
    // Initialize current position
    &val.outputValue = 0;
    // Install a setter, that will be called when 'val' is written to.
    // Use oset, so that the 'this' passed to oset is the motor.
    &val.oset = getSlotValue("setCommand");
  };
  function setCommand(command, slot)
  { // This function is called when someone writes 'command' to 'val'.
    // Since a setter is present, value was not updated, so do it
    &val.value = command;
    var same = (command == &val.outputValue);
    echo("Motor command is now " + &val.value);
    if (same && running)
    { // Target reached, stop motor control loop
      runTag.stop;
      running = false;
    }
    else if (!same && !running)
    { // Start motor control loop
        detach({runMotor()});
    }
  };
  function runMotor()
  { // Move current motor position toward target, one unit per second.
    running = true;
    runTag: while(&val.outputValue != &val.value)
    {
      &val.setOutputValue(
        &val.outputValue + 1 * (&val.value -&val.outputValue).sign);
      sleep(1s);
    };
    running = false;
  }
}|;
var m = Motor.new;
[00000001] Motor_0x42347788
// Send a command to motor by writing to val
m.val = 5|;
[00000002] *** Motor command is now 5
sleep(1.5s);
// Read motor current position by reading val.
m.val;
[00000003] 2
// Changed triggers when current position is updated by the Motor object.
at(m.val->changed?) echo("Motor position is " + m.val);
sleep(10s);
[00000004] *** Motor position is 3
[00000005] *** Motor position is 4
[00000006] *** Motor position is 5
\end{urbiscript}


\item[value]%
  The underlying value.
\begin{urbiscript}
var z = 2;
[00000001] 2
&z.value;
[00000001] 2
&z.value = 3;
[00000002] 3
z;
[00000002] 3
\end{urbiscript}
\end{urbiscriptapi}

%%% Local Variables:
%%% coding: utf-8
%%% mode: latex
%%% TeX-master: "../urbi-sdk"
%%% ispell-dictionary: "american"
%%% ispell-personal-dictionary: "../urbi.dict"
%%% fill-column: 76
%%% End:
