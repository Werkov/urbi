%% Copyright (C) 2010, Gostai S.A.S.
%%
%% This software is provided "as is" without warranty of any kind,
%% either expressed or implied, including but not limited to the
%% implied warranties of fitness for a particular purpose.
%%
%% See the LICENSE file for more information.

\section{Finalizable}

Objects that derive from this object will execute their \refSlot{finalize}
routine right before being destroyed (reclaimed) by the system.  It is
comparable to a \dfn{destructor}.

\subsection{Example}

The following object is set up to die verbosely.

\begin{urbiscript}
var obj =
  do (Finalizable.new)
  {
    function finalize ()
    {
      echo ("Ouch");
    }
  }|;
\end{urbiscript}

\noindent
It is reclaimed by the system when it is no longer referenced by any
other object.

\begin{urbiscript}
var alias = obj|;
obj = nil|;
\end{urbiscript}

\noindent
Here, the object is still alive, since \lstinline|alias| references
it.   Once it no longer does, the object dies.

\begin{urbiscript}
alias = nil|;
[00000004] *** Ouch
\end{urbiscript}

\subsection{Prototypes}

\begin{refObjects}
\item[Object]
\end{refObjects}

\subsection{Construction}

The constructor takes no argument.

\begin{urbiscript}
Finalizable.new;
[00000527] Finalizable_0x135360
\end{urbiscript}

\bigskip

Because of specific constraints of Finalizable, you cannot change the
prototype of an object to make it ``finalizable'': it \emph{must} be an
instance of Finalizable from its inception.

There, instead of these two invalid constructs,

\begin{multicols}{2}
\begin{urbiscript}
class obj : Finalizable.new
{
  function finalize()
  {
    echo("Ouch");
  }
}|;
[00000008:error] !!! <empty>: cannot inherit from a Finalizable without being a Finalizable too
\end{urbiscript}

\begin{urbiscript}
class o2
{
  protos = [Finalizable];
  function finalize()
  {
    echo("Ouch");
  }
}|;
[00000009:error] !!! <empty>: cannot inherit from a Finalizable without being a Finalizable too
\end{urbiscript}
\end{multicols}

write:

\begin{urbiscript}
var o3 =
  do (Finalizable.new)
  {
    function finalize()
    {
      echo("Ouch");
    }
  }|;
\end{urbiscript}

\bigskip

If you need multiple prototypes, do as follows.

\begin{urbiscript}
class Global.Foo
{
  function init()
  {
    echo("1");
  };
}|;

class Global.FinalizableFoo
{
  addProto(Foo.new);

  function 'new'()
  {
    var r = clone |
    r.init |
    Finalizable.new.addProto(r);
  };

  function init()
  {
    echo("2");
  };

  function finalize()
  {
    echo("3");
  };

}|;

var i = FinalizableFoo.new|;
[00000117] *** 1
[00000117] *** 2

i = nil;
[00000117] *** 3
\end{urbiscript}



\subsection{Slots}

\begin{urbiscriptapi}
\item[finalize] a simple function that takes no argument that will be
  evaluated when the object is reclaimed.  Its return value is
  ignored.
\begin{urbiscript}
Finalizable.new.setSlot("finalize", function() { echo("Ouch") })|;
[00033240] *** Ouch
\end{urbiscript}
\end{urbiscriptapi}

%%% Local Variables:
%%% mode: latex
%%% TeX-master: "../urbi-sdk"
%%% ispell-dictionary: "american"
%%% ispell-personal-dictionary: "../urbi.dict"
%%% fill-column: 76
%%% End:
