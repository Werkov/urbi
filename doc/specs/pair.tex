\section{Pair}

A \dfn{pair} is a container storing two objects, similar in spirit to
\lstinline|std::pair| in \Cxx.

\subsection{Prototype}
\begin{itemize}
\item \refObject{Tuple}
\end{itemize}

\subsection{Construction}

A \lstinline|Pair| is contructed with two arguments.

\begin{urbiscript}[firstnumber=1]
Pair.new(1, 2);
[00000001] (1, 2)

Pair.new;
[00000003:error] !!! Pair.init: expected 2 arguments, given 0

Pair.new(1, 2, 3, 4);
[00000003:error] !!! Pair.init: expected 2 arguments, given 4
\end{urbiscript}

\subsection{Slots}
\begin{itemize}
\item \lstinline|first|\\
  Return the first member of the pair.
\begin{urbiassert}[firstnumber=last]
Pair.new(1, 2).first == 1;
Pair[0] === Pair.first;
\end{urbiassert}

\item \lstinline|second|\\
  Return the second member of the pair.
\begin{urbiassert}[firstnumber=last]
Pair.new(1, 2).second == 2;
Pair[1] === Pair.second;
\end{urbiassert}
\end{itemize}



%%% Local Variables:
%%% mode: latex
%%% TeX-master: "../urbi-sdk"
%%% ispell-personal-dictionary: "../urbi.dict"
%%% End:
