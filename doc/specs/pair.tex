\section{Pair}

A \dfn{pair} is a container storing two objects, similar in spirit to
\lstinline|std::pair| in \Cxx.

\subsection{Prototype}
\begin{itemize}
\item \refObject{Object}
\end{itemize}

\subsection{Construction}

A \dfn{Pair} is contructed with one or two arguments. In case on one
argument used, the constructor uses it twice.

\begin{urbiscript}
Pair.new(1, 2);
[00000001] (1, 2)
Pair.new(3);
[00000002] (3, 3)
\end{urbiscript}

\subsection{Methods}
\begin{itemize}
\item \lstinline|asString|\\
  Generate the string \samp{(\var{first}, \var{second})} using
  \code{asPrintable} to convert members to strings.

\item \lstinline|first|\\
  Return the first member of the pair.
\begin{urbiscript}[firstnumber=last]
assert(Pair.new(1, 2).first == 1);
\end{urbiscript}

\item \lstinline|second|\\
  Return the second member of the pair.
\begin{urbiscript}[firstnumber=last]
assert(Pair.new(1, 2).second == 2);
\end{urbiscript}

\item \lstinline|'[]'(\var{index})|\\
  Return the \var{index}-th element.  \var{index} must be 0 or 1.
\begin{urbiscript}[firstnumber=last]
assert(Pair[0] === Pair.first);
assert(Pair[1] === Pair.second);
\end{urbiscript}

\item \lstinline|'[]='(\var{index}, \var{value})|\\
  Set (and return) the \var{index}-th element to \var{value}.
  \var{index} must be 0 or 1.

\item \lstinline|'<'(\var{other})|\\
  Lexicographic comparison between two pairs.
\begin{urbiscript}[firstnumber=last]
assert(Pair.new(0, 0) < Pair.new(0, 1));
assert(Pair.new(0, 0) < Pair.new(1, 0));
assert(Pair.new(0, 1) < Pair.new(1, 0));
\end{urbiscript}

\item \lstinline|'=='(\var{other})|\\
  Whether \lstinline|this| and \lstinline|other| have the same
  contents (equality-wise).
\begin{urbiscript}[firstnumber=last]
assert(Pair.new(1, 2) == Pair.new(1, 2));
assert(!(Pair.new(1, 1) == Pair.new(2, 2)));
\end{urbiscript}
\end{itemize}



%%% Local Variables:
%%% mode: latex
%%% TeX-master: "../urbi-sdk"
%%% End:
