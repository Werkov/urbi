\section{Socket}

A \dfn{Socket} can manage asynchronous input/output network
connections.

\subsection{Example}

The following example demonstrates how both the \refObject{Server} and
\refObject{Socket} object work.

This simple example will establish a dialogue between
\lstinline|server| and \lstinline|client|.  The following object,
\lstinline|Dialogue|, contains the script of this exchange.  It is put
into \lstinline|Global| so that both the server and client can read
it.  \lstinline|Dialogue.reply(var \var{s})| returns the reply to a
message \var{s}.

\begin{urbiscript}[firstnumber=last]
class Global.Dialogue
{
  var lines =
  [
    "Hi!",
    "Hey!",
    "Hey you doin'?",
    "Whazaaa!",
    "See ya.",
  ]|;

  function reply(var s)
  {
    for (var i: lines.size - 1)
      if (s == lines[i])
        return lines[i + 1];
    "off";
  }
}|;
\end{urbiscript}

The server, an instance of \refObject{Server}, expects incoming
connections, notified by the socket's \lstinline|connection?| event.
Once the connection establish, it listens to the \lstinline|socket|
for incoming messages, notified by the \lstinline|received?| event.
Its reaction to this event is to send the following line of the
dialogue.  At the end of the dialogue, the socket is disconnected.

\begin{urbiscript}[firstnumber=last]
var server =
  do (Server.new)
  {
    at (connection?(var socket))
      at (socket.received?(var data))
      {
        var reply = Dialogue.reply(data);
        socket.write(reply);
        echo("server: " + reply);
        if (reply == "off")
          socket.disconnect;
      };
  }|;
\end{urbiscript}

The client, an instance of \refObject{Socket} expects incoming
messages, notified by the \lstinline|received?| event.  Its reaction
is to send the following line of the dialogue.

\begin{urbiscript}[firstnumber=last]
var client =
  do (Socket.new)
  {
    at (received?(var data))
    {
      var reply = Dialogue.reply(data);
      write(reply);
      echo("client: " + reply);
    };
  }|;
\end{urbiscript}

As of today, \us's socket machinery requires to be regularly polled.

\begin{urbiscript}[firstnumber=last]
every (100ms)
  Socket.poll,
\end{urbiscript}

The server is then activated, listening to incoming connections on a
port that will be chosen by the system amongst the free ones.

\begin{urbiscript}[firstnumber=last]
server.listen("localhost", "0");
clog << "connecting to %s:%s" % [server.host, server.port];
\end{urbiscript}

The client connects to the server, and initiates the dialogue.

\begin{urbiscript}[firstnumber=last]
client.connect(server.host, server.port);
echo("client: " + Dialogue.lines[0]);
client.write(Dialogue.lines[0]);
[00000003] *** client: Hi!
\end{urbiscript}

Because this dialogue is asynchronous, the easiest way to wait for the
dialogue to finish is to wait for the \lstinline|disconnected?| event.

\begin{urbiscript}[firstnumber=last]
waituntil(client.disconnected?);
[00000004] *** server: Hey!
[00000005] *** client: Hey you doin'?
[00000006] *** server: Whazaaa!
[00000007] *** client: See ya.
[00000008] *** server: off
\end{urbiscript}

\subsection{Prototypes}
\begin{itemize}
\item \refObject{Object}
\end{itemize}

\subsection{Construction}

A \lstinline|Socket| is constructed with no argument. At creation, a
new \lstinline|Socket| has four own slots: \lstinline|connected|,
\lstinline|disconnected|, \lstinline|error| and \lstinline|received|.

\begin{urbiscript}[firstnumber=last]
var s = Socket.new|
\end{urbiscript}

\subsection{Slots}
\begin{itemize}

\item \lstinline|connect(\var{host}, \var{port})|\\
  Connect \lstinline|this| to \var{host} and \var{port}.  The
  \var{port} can be either an integer, or a string that denotes
  symbolic ports, such as \lstinline|"smtp"|, or \lstinline|"ftp"| and
  so forth.

\item \lstinline|connected|\\
  Event launched when the connection is established.

\item \lstinline|connectSerial(\var{device}, \var{baudrate})|\\
  Connect \lstinline|this| to the serial port \var{device}, with given
  \var{baudrate}.

\item \lstinline|disconnect|\\
  Close the connection.

\item \lstinline|disconnected|\\
  Event launched when a disconnection happens.

\item \lstinline|error|\\
  Event launched when an error happens. This event is launched with
  the error message in argument. The event \lstinline|disconnected| is
  also always launched.

\item \lstinline|host|\\
  The host of the connection.

\item \lstinline|isConnected|\\
  Whether \lstinline|this| is connected.

\item \lstinline|poll|\\
  Runs the event processing loop to execute ready handlers. By
  default, the loop is executed every \lstinline|pollInterval|.

\item \lstinline|pollInterval|\\
  Each \lstinline|pollInterval| amout of time, \lstinline|Socket.poll|
  is called. If \lstinline|pollInterval| equals zero,
  \lstinline|Socket.poll| is not called.

\item \lstinline|port|\\
  The port of the connection.

\item \lstinline|received|\\
  Event launched when \lstinline|this| has received data. The data is
  given by argument to the event.

\item \lstinline|write(\var{data})|\\
  Sends \var{data} trough the connection.

\end{itemize}

%%% Local Variables:
%%% mode: latex
%%% TeX-master: "../urbi-sdk"
%%% End:
