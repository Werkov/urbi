%% Copyright (C) 2009-2011, Gostai S.A.S.
%%
%% This software is provided "as is" without warranty of any kind,
%% either expressed or implied, including but not limited to the
%% implied warranties of fitness for a particular purpose.
%%
%% See the LICENSE file for more information.

\section{Lobby}

A \dfn{lobby} is the local environment for each (remote or local)
connection to an \urbi server.

\subsection{Examples}

Since every lobby is-a \refObject{Channel}, one can use the methods of
Channel.

\begin{urbiscript}
lobby << 123;
[00478679] 123
lobby << "foo";
[00478679] "foo"
\end{urbiscript}

\subsection{Prototypes}
\begin{itemize}
\item \refSlot[Channel]{topLevel}, an instance of \refObject{Channel}
  with an empty Channel name.
\end{itemize}

\subsection{Construction}

A lobby is implicitly created at each connection. At the top level,
\this is a \dfn{Lobby}.

\begin{urbiscript}
this.protos;
[00000001] [Lobby]
this.protos[0].protos;
[00000003] [Channel_0x4233b248]
\end{urbiscript}

Lobbies cannot be cloned, they must be created using \refSlot{create}.

\begin{urbiscript}
Lobby.new;
[00000177:error] !!! new: `Lobby' objects cannot be cloned
Lobby.create;
[00000174] Lobby_0x126450
\end{urbiscript}


\subsection{Slots}
\begin{urbiscriptapi}
\item[authors] Credit the authors of \usdk.  See also \refSlot{thanks} and
  \autoref{sec:genesis}.
\begin{urbiscript}
lobby.authors;
[00478679] *** Authors of Urbi:
[00478679] ***  - Jean-Christophe Baillie (original designer)
[00478679] ***
[00478679] *** Urbi 2 and subsequent versions:
[00478679] ***  - Akim Demaille
[00478679] ***  - Matthieu Nottale
[00478679] ***  - Quentin Hocquet
[00478679] ***
[00478679] *** See also `thanks;'.
\end{urbiscript}


\item[banner] Internal.  Display \usdk banner.
\begin{urbiscript}
lobby.banner;
[00005344] *** ********************************************************
[00005344] *** Urbi version 2.7.1 patch 472 revision b876514
[00005344] *** Copyright (C) 2004-2011 Gostai S.A.S.
[00005344] ***
[00005344] *** This program comes with ABSOLUTELY NO WARRANTY.  It can
[00005344] *** be used under certain conditions.  Type `license;',
[00005344] *** `authors;', or `copyright;' for more information.
[00005344] ***
[00005344] *** Check our community site: http://www.urbiforge.org.
[00005344] *** ********************************************************
\end{urbiscript}


\item[bytesReceived] The number of bytes that were ``input'' to \this.  See
  also \refSlot{receive}.
\begin{urbiscript}
var l = Lobby.create|;
assert (l.bytesReceived == 0);

l.receive("123456789;");
[00000022] 123456789
assert (l.bytesReceived == 10);

l.receive("1234;");
[00000023] 1234
assert (l.bytesReceived == 15);
\end{urbiscript}
\begin{urbicomment}
removeSlot("l")|;
\end{urbicomment}


\item[bytesSent] The number of bytes that were ``output'' by \this.  See
  also \refSlot{send} and \refSlot{write}.
\begin{urbicomment}
// We used to have a non deterministic bug in Urbi 2 in the assert below.
// It does not seem to exist with Urbi 3.  It is probably related to
// some issues with Lazy (as it reported an error about "code"
// being an already defined slot).
\end{urbicomment}
\begin{urbiscript}
var l = Lobby.create|;
assert (l.bytesSent == 0);

l.send("0123456789");
[00011988] 0123456789
// 22 = "[00011988] 0123456789\n".size.
assert (l.bytesSent == 22);

l.send("xx", "label");
[00061783:label] xx
// 20 = "[00061783:label] xx\n".size.
assert (l.bytesSent == 42);

l.write("[01234567]\n");
[01234567]
assert (l.bytesSent == 53);
\end{urbiscript}
\begin{urbicomment}
removeSlots("l");
\end{urbicomment}


\item[connected]
  Whether \this is connected.
\begin{urbiassert}
connected;
\end{urbiassert}


\item[connectionTag] The tag of all code executed in the context of \this.
  This tag applies to \this, but the top-level loop is immune to
  \refSlot[Tag]{stop}, therefore \lstinline|connectionTag| controls every
  thing that was launched from this lobby, yet the lobby itself is still
  usable.
\begin{urbiscript}
every (1s) echo(1), sleep(0.5s); every (1s) echo(2),
sleep(1.2s);
connectionTag.stop;
[00000507] *** 1
[00001008] *** 2
[00001507] *** 1
[00002008] *** 2

"We are alive!";
[00002008] "We are alive!"

every (1s) echo(3), sleep(0.5s); every (1s) echo(4),
sleep(1.2s);
connectionTag.stop;
[00003208] *** 3
[00003710] *** 4
[00004208] *** 3
[00004710] *** 4

"and kicking!";
[00002008] "and kicking!"
\end{urbiscript}

  Of course, a background job may stop a foreground one.
\begin{urbiscript}
{ sleep(1.2s); connectionTag.stop; },
// Note the `;', this is a foreground statement.
every (1s) echo(5);
[00005008] *** 5
[00005508] *** 5

"bye!";
[00006008] "bye!"
\end{urbiscript}


\item[copyright](<deep> = true)%
  Display the copyright of \usdk.  Include copyright information
  about sub-components if \var{deep}.
\begin{urbiscript}
lobby.copyright(false);
[00033102] *** Urbi version 2.7.1 patch 472 revision b876514
[00033102] *** Copyright (C) 2004-2011 Gostai S.A.S.

lobby.copyright;
[00088621] *** Urbi version 2.7.1 patch 472 revision b876514
[00088621] *** Copyright (C) 2004-2011 Gostai S.A.S.
[00088621] ***
[00088621] *** Libport version urbi-sdk-2.7.1 patch 117 revision e96cd13
[00088621] *** Copyright (C) 2006-2011 Gostai S.A.S.
\end{urbiscript}


\item[create]
  Instantiate a new Lobby.
\begin{urbiassert}
Lobby.create.isA(Lobby);
\end{urbiassert}


\item[echo](<value>, <channel> = "")%
  Send \lstinline|\var{value}.asString| to \this, prefixed
  by the \refObject{String} \var{channel} name if specified.  This is
  the preferred way to send informative messages (prefixed with
  \samp{***}).
\begin{urbiscript}
lobby.echo("111", "foo");
[00015895:foo] *** 111
lobby.echo(222, "");
[00051909] *** 222
lobby.echo(333);
[00055205] *** 333
\end{urbiscript}


\item[echoEach](<list>, <channel> = "")%
  Apply \lstinline|echo(\var{m}, \var{channel})| for each member \var{m} of
  \var{list}.
\begin{urbiscript}
lobby.echo([1, "2"], "foo");
[00015895:foo] *** [1, "2"]

lobby.echoEach([1, "2"], "foo");
[00015895:foo] *** 1
[00015895:foo] *** 2

lobby.echoEach([], "foo");
\end{urbiscript}


%\item \lstinline|help|\experimental\\
%  Launch the tutorial.


\item[instances]
  A list of the currently alive lobbies.  It contains at least the Lobby
  object itself, and the current \refSlot{lobby}.
\begin{urbiassert}
lobby in Lobby.instances;
Lobby in Lobby.instances;
\end{urbiassert}


\item[license]
  Display the end user license agreement of the \usdk.
\begin{urbiscript}
lobby.license;
[00000000] ***                     GNU AFFERO GENERAL PUBLIC LICENSE
[00000000] ***                        Version 3, 19 November 2007
[00000000] ***
[00000000] ***  Copyright (C) 2007 Free Software Foundation, Inc. <http://fsf.org/>
[00000000] ***  Everyone is permitted to copy and distribute verbatim copies
[00000000] ***  of this license document, but changing it is not allowed.
[00000000] ***
[:][...]
[00000000] *** For more information on this, and how to apply and follow the GNU AGPL, see
[00000000] *** <http://www.gnu.org/licenses/>.
\end{urbiscript}


\item[lobby]
  The lobby of the current connection.  This is typically \this.
\begin{urbiassert}
Lobby.lobby === this;
\end{urbiassert}

  But when several connections are active (e.g., when there are remote
  connections), it can be different from the target of the call.

\begin{urbiscript}
Lobby.create| Lobby.create|;
for (var l : lobbies)
  assert (l.lobby == Lobby.lobby);
\end{urbiscript}


\item[onDisconnect](<lobby>)%
  Event launched when \this is disconnected.  There is a single event
  instance for all the lobby, \refSlot[Lobby]{onDisconnect}, the
  disconnected lobby being passed as argument.


\item[quit] Shut this lobby down, i.e., close the connection.  The
  server is still running, see \refSlot[System]{shutdown} to quit the
  server.


\item[receive](<value>)%
  This is low-level routine.  Pretend the \refObject{String}
  \var{value} was received from the connection.  There is no guarantee
  that \var{value} will be the next program block that will be
  processed: for instance, if you load a file which, in its middle,
  uses \lstinline|lobby.receive("foo")|, then \lstinline|"foo"| will
  be appended after the end of the file.
\begin{urbiscript}
Lobby.create.receive("12;");
[00478679] 12
\end{urbiscript}


\item[remoteIP]
  When \this is connected to a remote server, it's Internet address.


\item[send](<value>, <channel> = "")%
  This is low-level routine.  Send the \refObject{String} \var{value}
  to \this, prefixed by the \refObject{String}
  \var{channel} name if specified.
\begin{urbiscript}
lobby.send("111", "foo");
[00015895:foo] 111
lobby.send("222", "");
[00051909] 222
lobby.send("333");
[00055205] 333
\end{urbiscript}


\item[thanks] Credit the contributors of \usdk.  See also \refSlot{authors}
  and \autoref{sec:genesis}.
%% FIXME: Can't credit everyone because listings does not support UTF-8 in
%% these environments... And Benoît Sigoure, for instance, has some.
\begin{urbiscript}
lobby.thanks;
[00478679] *** Contributors to Urbi:
[00478679] ***
[00478679] ***  - Adam Oleksy <adam.oleksy@pwr.wroc.pl>
[00478679] ***  - Alexandre Morgand
[:][...]
[00478679] ***
[00478679] *** See also `authors;'.
\end{urbiscript}

\item[wall](<value>, <channel> = "")%
  Perform \lstinline|echo(\var{value}, \var{channel})| on all the
  existing lobbies (except Lobby itself).
\begin{urbiscript}[firstnumber=1]
Lobby.wall("111", "foo");
[00015895:foo] *** 111
\end{urbiscript}


\item[write](<value>)%
  This is low-level routine.  Send the \refObject{String} \var{value}
  to the connection.  Note that because of buffering, the output might
  not be visible before an end-of-line character is output.
\begin{urbiscript}
lobby.write("[");
lobby.write("999999999:");
lobby.write("myTag] ");
lobby.write("Hello, World!");
lobby.write("\n");
[999999999:myTag] Hello, World!
\end{urbiscript}
\end{urbiscriptapi}

%%% Local Variables:
%%% coding: utf-8
%%% mode: latex
%%% TeX-master: "../urbi-sdk"
%%% ispell-dictionary: "american"
%%% ispell-personal-dictionary: "../urbi.dict"
%%% fill-column: 76
%%% End:
