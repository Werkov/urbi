%% Copyright (C) 2010, Gostai S.A.S.
%%
%% This software is provided "as is" without warranty of any kind,
%% either expressed or implied, including but not limited to the
%% implied warranties of fitness for a particular purpose.
%%
%% See the LICENSE file for more information.

\section{Job}

Jobs are independent threads of executions.  Jobs can run concurrently.
They can also be managed using \refObject[s]{Tag}.

\subsection{Prototypes}

\begin{refObjects}
\item[Object]
\end{refObjects}

\subsection{Construction}

A Job is typically constructed via \refSlot[Control]{detach},
\refSlot[Control]{disown}, or \refSlot[System]{spawn}.

\begin{urbiscript}
detach(sleep(10));
[00202654] Job<shell_4>

disown(sleep(10));
[00204195] Job<shell_5>

spawn (function () { sleep(10) }, false);
[00274160] Job<shell_6>
\end{urbiscript}

\subsection{Slots}

\begin{urbiscriptapi}
\item[asJob]
  Return \lstinline|this|.

\item[asString] The string \lstinline|Job<\var{name}>| where \var{name} is
  the name of the job.

\item[backtrace] The current backtrace of the job as a list of
  \refObject[s]{StackFrame}.

\begin{urbiscript}
//#push 1 "file.u"
var s = detach(sleep(10))|;
// Leave some time for s to be started.
sleep(1);
assert
{
  s.backtrace[0].asString == "file.u:1.16-24: sleep";
  s.backtrace[1].asString == "file.u:1.9-25: detach, indirect call of code";
};
//#pop
\end{urbiscript}

\item[clone]
  Cloning a job is impossible since Job is considered as being an atom.

\item[dumpState]
  Pretty-print the state of the job.

\begin{urbiscript}
//#push 1 "file.u"
var t = detach(sleep(10))|;
// Leave some time for s to be started.
sleep(1);
t.dumpState;
[00004295] *** Job: shell_10
[00004295] ***   State: sleeping
[00004295] ***   Tags:
[00004295] ***     Tag<Lobby_1>
[00004297] ***   Backtrace:
[00004297] ***     file.u:1.16-24: sleep
[00004297] ***     file.u:1.9-25: detach, indirect call of code
//#pop
\end{urbiscript}

\item[name]  The name of the job.
\begin{urbiscript}
detach(sleep(10)).name;
[00004297] "shell_5"
\end{urbiscript}

\item[setSideEffectFree](<value>)
  If value is true, mark the current job as side-effect free. It indicates
  whether the current state may influence other parts of the system. This is
  used by the scheduler to choose whether other jobs need scheduling or not.
  The default value is false.

\item[status]
  The current status of the job (starting, running, \ldots), and its
  properties (frozen, \ldots).

\item[tags] The list of \refObject[s]{Tag} that manage this job.

\item[terminate]  Kill this job.
\begin{urbiscript}
var r = detach({ sleep(1s); echo("done") })|;
assert (r in jobs);
r.terminate;
assert (r not in jobs);
sleep(2s);
\end{urbiscript}

\item[timeShift]
  Get the total amount of time during which we were frozen.
\begin{urbiscript}
tag: r = detach({ sleep(3); echo("done") })|;
tag.freeze();
sleep(2);
tag.unfreeze();
Math.round(r.timeShift);
[00000001] 2
\end{urbiscript}

\item[waitForChanges] Resume the scheduler, putting the current Job in a
  waiting status.  The scheduler may reschedule the job immediately.

\item[waitForTermination] Wait for the job to terminate before resuming
  execution of the current one.  If the job has already terminated, return
  immediately.
\end{urbiscriptapi}


%%% Local Variables:
%%% mode: latex
%%% TeX-master: "../urbi-sdk"
%%% ispell-dictionary: "american"
%%% ispell-personal-dictionary: "../urbi.dict"
%%% fill-column: 76
%%% End:
