\section{Barrier}

\lstinline|Barrier| is used to wait until another job raise a signal.
This can be used to implements blocking calls which are waiting until
a resource is made available.

\subsection{Prototypes}

\begin{refObjects}
\item[Object]
\end{refObjects}

\subsection{Construction}

A \lstinline|Barrier| can be created with no argument.  Signals and wait
calls done on this instance are restricted to this instance.

\begin{urbiscript}[firstnumber=1]
Barrier.new;
[00000000] Barrier_0x25d2280
\end{urbiscript}

\subsection{Slots}

\begin{urbiscriptapi}

\item \lstinline|signal(\var{payload})|
  Wake up one of the job waiting for a signal.  The \var{payload} is sent to
  the \var{wait} method.  This method returns the number of job woken up.

\begin{urbiscript}
do (Barrier.new)
{
  echo(wait) &
  echo(wait) &
  assert
  {
    signal(1) == 1;
    signal(2) == 1
  }
};
[00000000] *** 1
[00000000] *** 2
\end{urbiscript}


\item \lstinline|signalAll(\var{payload})|
  Wake up all of the job waiting for a signal.  The \var{payload} is sent to
  all \var{wait} methods.  This method returns the number of job woken up.

\begin{urbiscript}
do (Barrier.new)
{
  echo(wait) &
  echo(wait) &
  assert
  {
    signalAll(1) == 2;
    signalAll(2) == 0
  }
};
[00000000] *** 1
[00000000] *** 1
\end{urbiscript}


\item[wait]
  Block until a signal is received.  The \var{payload} sent with the signal
  function is returned by the \lstinline|wait| method.

\begin{urbiscript}
do (Barrier.new)
{
  echo(wait) &
  signal(1)
};
[00000000] *** 1
\end{urbiscript}

\end{urbiscriptapi}

%%% Local Variables:
%%% mode: latex
%%% TeX-master: "../urbi-sdk"
%%% ispell-dictionary: "american"
%%% ispell-personal-dictionary: "../urbi.dict"
%%% End:
