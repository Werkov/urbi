%% Copyright (C) 2010, Gostai S.A.S.
%%
%% This software is provided "as is" without warranty of any kind,
%% either expressed or implied, including but not limited to the
%% implied warranties of fitness for a particular purpose.
%%
%% See the LICENSE file for more information.

\section{Barrier}

\lstinline|Barrier| is used to wait until another job raises a signal.
This can be used to implement blocking calls waiting until a resource
is available.

\subsection{Prototypes}

\begin{refObjects}
\item[Object]
\end{refObjects}

\subsection{Construction}

A \lstinline|Barrier| can be created with no argument.  Signals and wait
calls done on this instance are restricted to this instance.

\begin{urbiscript}[firstnumber=1]
Barrier.new;
[00000000] Barrier_0x25d2280
\end{urbiscript}

\subsection{Slots}

\begin{urbiscriptapi}

\item[signal](<payload>)%
  Wake up one of the job waiting for a signal.  The \var{payload} is sent to
  the \var{wait} method.  This method returns the number of job woken up.

\begin{urbiscript}
do (Barrier.new)
{
  echo(wait) &
  echo(wait) &
  assert
  {
    signal(1) == 1;
    signal(2) == 1;
  }
}|;
[00000000] *** 1
[00000000] *** 2
\end{urbiscript}


\item[signalAll](<payload>)%
  Wake up all the jobs waiting for a signal.  The \var{payload} is
  sent to all \var{wait} methods.  Return the number of jobs woken up.

\begin{urbiscript}
do (Barrier.new)
{
  echo(wait) &
  echo(wait) &
  assert
  {
    signalAll(1) == 2;
    signalAll(2) == 0;
  }
}|;
[00000000] *** 1
[00000000] *** 1
\end{urbiscript}


\item[wait]
  Block until a signal is received.  The \var{payload} sent with the signal
  function is returned by the \lstinline|wait| method.

\begin{urbiscript}
do (Barrier.new)
{
  echo(wait) &
  signal(1)
}|;
[00000000] *** 1
\end{urbiscript}

\end{urbiscriptapi}

%%% Local Variables:
%%% mode: latex
%%% TeX-master: "../urbi-sdk"
%%% ispell-dictionary: "american"
%%% ispell-personal-dictionary: "../urbi.dict"
%%% fill-column: 76
%%% End:
