\documentclass[openright,twoside,11pt]{book}
  % Index.  Must be before loading hyperref.
  \usepackage{myindex}
  \usepackage{index}
  \makeindex
  \usepackage{urbi-report}
  \usepackage{indexing}

% CONFIGURATION
\title{\us 2.0 specifications}
\author{Kernel Team}

% DOCUMENT
\begin{document}

\maketitle
\tableofcontents

\chapter*{Introduction}

\section*{Status of this document}

This document defines the specifications of the \us language version
2.0. It defines the expected behavior from the \us interpreter, the
standard library, and the \sdk. It can be used to check whether some code
is valid, or browse \us or \Cxx \api for a desired feature. Random reading
can also provide you with
advanced knowledge or subtleties about some \us aspects.

This document is not an \us tutorial; it is not structured in a
progressive manner and is too
detailed. Think of this document as a dictionary: one does not
learn a foreign language by reading a dictionary. The \us Tutorial, or
the live \us tutorial built in the interpreter are good introductions
to \us.

This document does not aim at giving advanced programming
techniques. Its only goal is to define the language and its
libraries.

\section*{\us}

\dfn[urbiscript@\us]{\us} is an interpreted language designed for highly
interactive and parallel behavior programming. %FILLME

\section*{Structure of this document}

This document structured as follows:

\begin{description}
  \newcommand{\xitem}[2]{\item[\autoref{#1} --- #2]~\\}
\xitem{sec:tools}{Tools specifications}%
  Presentation and usage of the different tools available with the
  \urbi framework related to \us, such as the \urbi server, the
  command line client, \umake, \ldots

\xitem{sec:lang}{\us language specifications}%
  Core constructs of the language and their behavior.

\xitem{sec:stdlib}{\us standard library specifications}%
  Listing of all classes and methods provided in the standard library.

\xitem{sec:sdk}{\urbi \sdk specifications}%
  The \urbi software development kit that enable to
  interact with \urbi from \Cxx.
\end{description}

\newcommand{\optionDebug}{
  Set the verbosity level of traces.
  This option is mostly for developers, but it is very useful when
  tracking problems such as a UObject that fails to load properly.
  Valid values for \var{level} are, in increasing verbosity order:
  \begin{sublist}
    \begin{enumerate}
    \item \code{NONE}, no log messages at all.
    \item \code{LOG}, the default value.
    \item \code{TRACE}
    \item \code{DEBUG}
    \item \code{DUMP}, maximum verbosity.
    \end{enumerate}
  \end{sublist}
}
\newcommand{\optionHelp}
  {Display the help message and exit successfully.}

\newcommand{\optionVersion}
  {Display version information and exit successfully.}

\chapter{Programs}
\label{sec:tools}

\section{Environment Variables}
\label{sec:tools:envvars}

There is a number of environment variables that alter the behavior of
the \urbi tools.

\subsection{Search Path Variables}

Some variables define \dfn[search-path]{search-paths}, i.e.,
colon-separated lists of directories in which library files (\us
programs, UObjects and so forth) are looked for.

The tools have predefined values for these variables which are
tailored for your installation --- so that \urbi tools can be run
without any special adjustment.  In order to provide the user with a
means to override or extend these built-in values, the path variables
support a special syntax: a lone colon specifies where the standard
search path must be inserted.  See the following examples about
\env{URBI\_PATH}.

\begin{shell}
# Completely override the system path.  First look for files in
# /home/jessie/urbi, then in /usr/local/urbi.
export URBI_PATH=/home/jessie/urbi:/usr/local/urbi

# Prepend the previous path to the default path.  This is dangerous as
# it may result in some standard files being hidden.
export URBI_PATH=/home/jessie/urbi:/usr/local/urbi:

# First look in Jessie's directory, then the default location, and
# finally in /usr/local/urbi.
export URBI_PATH=/home/jessie/urbi::/usr/local/urbi

# Extend the default path, i.e., files that are not found in the
# default path will be looked for in Jessie's place, and then in
# /usr/local/urbi
export URBI_PATH=:/home/jessie/urbi:/usr/local/urbi
\end{shell}

\begin{windows}
  On Windows too directories are separated by colons, but backslashes
  are used instead of forward-slashes.  For instance
\begin{shell}
URBI_PATH=C:\cygwin\home\jessie\urbi:C:\cygwin\usr\local\urbi
\end{shell}
\end{windows}

\subsection{Environment Variables}
\begin{envs}
\item[URBI\_PATH] The search-path for \us source files (i.e.,
  \file{*.u} files).

\item[URBI\_ROOT] The \urbi SDK is relocatable: its components know the
  relative location of each other.  Yet they need to ``guess'' the
  \urbi root, i.e., the path to the directory that contains the files.
  This variable also to override that guess.  Do not use it unless you
  know exactly what you are doing.

\item[URBI\_UOBJECT\_PATH] The search-path for UObjects files.
  This is used by \command{urbi-launch}, by
  \lstinline|System.loadModule| and \lstinline|System.loadLibrary|.
\end{envs}

\section{Special Files}
\label{sec:tools:files}

\begin{files}
\item[CLIENT.INI] This is the obsolete name for \file{global.u}.

\item[global.u] If found in the \env{URBI\_PATH} (see
  \autoref{sec:tools:envvars}), this file is loaded by \urbi server upon
  start-up.  It is the appropriate place to install features you
  mean to provide to all the users of the server.  It is will be
  loaded via a special system connection, with its own private lobby.
  Therefore, purely local definitions will not be reachable from
  users; global modifications should be made in globally visible
  objects, say \refObject{Global}.

\item[local.u] If found in the \env{URBI\_PATH} (see
  \autoref{sec:tools:envvars}), this file is loaded by every
  connection established with an \urbi server.  This is the
  appropriate place for enhancements local to a lobby.

\item[URBI.INI] This is the obsolete name for \file{global.u}.
\end{files}

\section{\command{urbi} --- Running an Urbi Server}
\label{sec:tools:urbi}

The \command{urbi} program launches an \urbi server, for either batch,
interactive, or network-based executions.  It is subsumed by, but
simpler to use than, \command{urbi-launch}
(\autoref{sec:tools:urbi-launch}).

\subsection{Options}

\begin{options}[General Options]
\item[h]{help} \optionHelp
\item{version} \optionVersion
\end{options}

\begin{options}[Tuning]
\item[d]{debug=\var{level}} \optionDebug
\item[F]{fast}
  Ignore system time, go as fast as possible.  Do not use this option
  unless you know exactly what you are doing.

  The \option{--fast} flag makes the kernel run the program in
  ``simulated time'', as fast as possible. A \lstinline|sleep| in fast
  mode will not actually wait (from the wall-clock point of view), but
  the kernel will internally increase its simulated time.

  For instance, the following session behaves equally in fast and
  non-fast mode:

\begin{urbiscript}[firstnumber=1]
{ sleep(2s); echo("after") } & { sleep(1s); echo("before") };
[000000463] *** before
[000001463] *** after
\end{urbiscript}

  \noindent
  However, in non fast mode the execution will take two seconds (wall
  clock time), while it be instantaneous in fast mode. This option was
  designed for testing purpose; \emph{it does not preserve the program
    semantics}.

\item[s]{stack-size=\var{size}} Set the coroutine \dfn{stack size}.
  The unit of \var{size} is KB; it defaults to 128.

  This option should not be needed unless you have ``stack exhausted''
  messages from \command{urbi} in which case you should try
  \option{--stack-size=512} or more.

  Alternatively you can define the environment variable
  \env{URBI\_STACK\_SIZE}.  The option \option{--stack-size} has
  precedence over the \env{URBI\_STACK\_SIZE}.

\item[q]{quiet} Do not send the welcome banner to incoming clients.
\end{options}

\begin{options}[Networking]
\item[H]{host=\var{address}} Set the \var{address} on which network
  connections are listened to.  Typical values of \var{address}
  include:
  \begin{sublist}
    \begin{description}
    \item[localhost] only local connections are allowed (no other
      computer can reach this server).
    \item[127.0.0.1] same as \code{localhost}.
    \item[0.0.0.0] any IP v4 connection is allowed, including from
      remote computers.
  \end{description}
  \end{sublist}
  Defaults to \code{0.0.0.0}.
\item[P]{port=\var{port}} Set the port to listen incoming
  connections to.  If \var{port} is \code{-1}, no networking.  If
  \var{port} is \code{0}, then the system will chose any available
  port (see \option{--port-file}).  Defaults to \code{-1}.
\item[w]{port-file=\var{file}} When the system is up and running,
  and when it is ready for network connections, create the file named
  \var{file} which contains the number of the port the server listens
  to.
\end{options}


\begin{options}[Execution]
\item[e]{expression=\var{exp}} Send the \us expression \var{exp}.
  No separator is added, you have to pass yours.
\item[f]{file=\var{file}} Send the contents of the file \var{file}.
  No separator is added, you have to pass yours.
\item[i]{interactive} Start an interactive session.
\end{options}

The options \option{-e}, \option{-f} accumulate, and are run in the
same \refObject{Lobby} as \option{-i} if used.  In other words, the
following session is valid:

\begin{shell}[alsolanguage={[interactive]Urbi}]
# Create a file "two.u".
$ echo "var two = 2;" >two.u
# urbi -e 'var one = 1;' -f two.u -i
[00000000] 1
[00000000] 2
one + two;
[00000000] 3
\end{shell}%$

\section{\command{urbi-image} --- Querying Images from a Server}
\label{sec:tools:urbi-image}

\begin{shell}
urbi-image \var{option}...
\end{shell}

Connect to an \urbi server, and fetch images from it, for instance
from its camera.

\subsection{Options}

\begin{options}[General Options]
\item[h]{help} \optionHelp
\item{version} \optionVersion
\end{options}

\begin{options}[Networking]
\item[H]{host=\var{host}} Address to connect to.
\item[P]{port=\var{port}} Port to connect to.
\end{options}

\begin{options}[Tuning]
\item[p]{period=\var{period}} Specify the period, in millisecond, at
  which images are queried.
\item[F]{format=\var{format}} Select format of the image (rgb, ycrcb,
  jpeg, ppm).
\item[r]{reconstruct} Use reconstruct mode (for aibo).
\item[j]{jpeg=\var{factor}} JPEG compression factor (from 0 to 100,
  defaults to 70).
\item[d]{device=\var{device}} Query image on \var{device}.val
  (default: \code{camera}).
\item[o]{output=\var{file}} Query and save one image to \var{file}.
\item[R]{resolution=\var{resolution}} Select resolution of the image
  (0=biggest).
\item[s]{scale=\var{factor}} Rescale image with given \var{factor}
  (display only).
\end{options}


\section{\command{urbi-launch} --- Running a UObject}
\label{sec:tools:urbi-launch}

The \command{urbi-launch} program launches an \urbi system.  It is
more general than \command{urbi} (\autoref{sec:tools:urbi}):
everything \command{urbi} can do, \command{urbi-launch} can do it too.

\subsection{Invoking \command{urbi-launch}}

\command{urbi-launch} launches UObjects, either in plugged-in mode, or
in remote mode.  Since UObjects can also accept options, the command
line features two parts, separated by \samp{--}:

\begin{shell}
urbi-launch [\var{urbi-launch-option}...] \var{module}... [-- \var{module-option}...]
\end{shell}

The \var{module}s are looked for in the \env{URBI\_UOBJECT\_PATH}.

\begin{options}[Urbi-launch options]
\item[h]{help} \optionHelp
\item{version} \optionVersion
\item[c]{customize=\var{file}} Start the \urbi server in
  \var{file}.  This option is mostly for developers.
\item[d]{debug=\var{level}} \optionDebug
\end{options}

\begin{options}[Mode selection]
% !!! \lstinline|loadModule("\var{module}")| does not escape the \var
% because it is inside a string.
\item[p]{plugin} Attach the \var{module} onto a currently running
  \urbi server (identified by \var{host} and \var{port}).  This is
  equivalent to running \lstinline|loadModule("module")| on the
  corresponding server.

\item[r]{remote} Run the \var{modules} as separated processes,
  connected to a running Urbi server (identified by \var{host} and
  \var{port}) via network connection.

\item[s]{start} Start an Urbi server with plugged-in
  \var{modules}.  In this case, the \var{module-option} are exactly
  the options supported by \command{urbi}.
\end{options}

\paragraph{Networking}
\command{urbi-launch} supports the same networking options
(\option{--host}, \option{--port}, \option{--port-file}) as
\command{urbi}, see \autoref{sec:tools:urbi}.

\subsection{Examples}

To launch a fresh server in an interactive session with the
\lstinline|UFactory| UObject compiled as the file \file{factory.so}
(or \file{factory.dll} plugged in, run:

\begin{shell}
urbi-launch --start ufactory -- --interactive
\end{shell}

To start an \urbi server accepting connections on the local port 54000
from any remote host, with \lstinline|UFactory| plugged in, run:

\begin{shell}
urbi-launch --start --host 0.0.0.0 --port 54000 ufactory
\end{shell}


\section{\command{urbi-send} --- Sending \us Commands to a Server}
\label{sec:tools:urbi-send}

\begin{shell}
urbi-send \var{option}...
\end{shell}

Connect to an \urbi server, and send commands or file contents to it.
Stay connected, until server disconnection, or user interruption (such
as \key{C-c} under a Unix terminal).

\begin{options}
\item[b]{banner} Do not hide the banner from the server.
\item[e]{expression=\var{script}} Send \var{script} to the server.
\item[f]{file=\var{file}} Send the contents of \var{file} to the
  server.
\item[h]{help} \optionHelp
\item[H]{host=\var{host}} Address to listen to.
\item[P]{port=\var{port}} Port to listen to, 0 for automatic
  selection.
\item{port-file=\var{file}} Listen to the port contained in the file
  \var{file}.
\item[q]{quit} Disconnect from the server immediately after having
  sent all the commands.  This is equivalent to \samp{-e 'quit;'}.
  This is inappropriate if code running in background is expected to
  deliver its result asynchronously: the connection will be closed
  before the result was sent.

  Without this option, \command{urbi-send} prompts the user to hit
  \key{C-c} to end the connection.
\item{version} \optionVersion
\end{options}


\section{\command{umake} --- Compiling UObject Components}
\label{sec:tools:umake}

The \command{umake} programs builds loadable modules, UObjects, to be
later run using \command{urbi-launch}
(\autoref{sec:tools:urbi-launch}).  Using it is not mandatory: users
familiar with their compilation tools will probably prefer using them
directly.  Yet \command{umake} makes things more uniform and simpler,
at the cost of less control.

\subsection{Invoking \command{umake}}
\label{sec:tools:umake:invoke}

Usage:
\begin{shell}
umake \var{option}... \var{file}...
\end{shell}

Compile the \var{file}.  The \var{files} can be of different kinds:
\begin{itemize}
\item objects files (\file{*.o}, \file{*.obj} and so forth) and linked
  into the result.
\item libraries (\file{*.a}) and linked into the result.
\item source files (\file{*.cc}, \file{*.cpp}, \file{*.c}, \file{*.C})
  are compiled.
\item header files (\file{*.h}, \file{*.hh}, \file{*.hxx},
  \file{*.hpp}) are \emph{not} compiled, but used as dependencies: if
  a header file is changed, the next \command{umake} run will actually
  recompile.
\item directories are recursively traversed, and files of the above
  types are gathered as if they were given on the command line.
\end{itemize}

\begin{options}[General options]
\item[D]{debug} Turn on shell debugging (\lstinline|set -x|) to
  track \command{umake} problems.
\item[h]{help} \optionHelp
\item[q]{quiet} Produce no output except errors.
\item[v]{version} \optionVersion
\item[V]{verbose} Report on what is done.
\end{options}

\begin{options}[Compilation options]
\item{deep-clean} Remove all building directories and exit.
\item[c]{clean} Clean building directory before compilation.
\item[j]{jobs=\var{jobs}} Specify the numbers of compilation
  commands to run simultaneously.
\item[l]{library} Produce a library, don't link to a particular
  core.
\item[s]{shared-library} Produce a shared library loadable by any
  core.
\item[o]{output=\var{output}} Set the output file name.
\item[C]{core=\var{core}} Set the build type.
\item[H]{host=\var{host}} Set the destination host.
\item[m]{disable-automain} Do not add the \lstinline|main| function.
\end{options}

\begin{options}[Developer options]
\item[p]{prefix=\var{dir}} Set library files location.
\item[P]{param-mk=\var{file}} Set \file{param.mk} location.
\item[k]{kernel=\var{dir}} Set the kernel location.
\end{options}


\subsection{\command{umake} Wrappers}
\label{sec:tools:umake:wrappers}

As a convenience for common \command{umake} usages, some wrappers are
provided:
\begin{description}
\item[\command{umake-deepclean}] --- Cleaning\\
  Clean the temporary files made by running \command{umake} with the
  same arguments.  Same as \samp{umake --deep-clean}.
\item[\command{umake-shared}] --- Compiling Shared UObjects\\
  Build a shared object to be later run using \command{urbi-launch}
  (\autoref{sec:tools:urbi-launch}).  Same as \samp{umake
    --shared-library}.
\end{description}

%%% Local Variables:
%%% mode: latex
%%% TeX-master: "../urbi-sdk"
%%% ispell-dictionary: "american"
%%% ispell-personal-dictionary: "../urbi.dict"
%%% End:

\FloatBarrier
\chapter{\us Language Specifications}
\label{sec:lang}

\section{Syntax}

\subsection{Characters, encoding}
\index{encoding}
\index{ASCII}
\index{UTF-8}

Currently \us makes no assumptions about the encoding used in the
programs, but the streams are handled as 8-bit characters.

While you are allowed to use whatever character you want in the string
literals (especially using the binary escapes,
\autoref{sec:us-syn-lit-string}), only plain ASCII characters are
allowed in the program body.  Invalid characters are reported,
possibly escaped if they are not ``printable''.  If you enter UTF-8
characters, since they possibly span over several 8-bit characters, a
single (UTF-8) character may be reported as several invalid (8-bit)
characters.

%% UTF-8 is not supported by lstlisting, we need to escape to TeX.
%% TeX4ht produces ugly results when using lstnewenvironment.  Worse,
%% here it creates a new <pre> on each side of the escape characters.
%% So let's hope Eitan fixes this some day.
\begin{urbiscript}[firstnumber=1,escapeinside=<>]
#<Été>;
[00048238:error] !!! invalid character: `#'
[00048239:error] !!! invalid character: `\xc3'
[00048239:error] !!! invalid character: `\x89'
[00048239:error] !!! invalid character: `\xc3'
[00048239:error] !!! invalid character: `\xa9'
\end{urbiscript}

\subsection{Comments}

\dfn{Comments} are used to document the code, they are ignored by the
\us interpreter. Both \Cxx comment types are supported.

\begin{itemize}
\item A \lstinline|//| introduces a comment that lasts until the end
  of the line.
\item A \lstinline|/*| introduces a comment that lasts until
  \lstinline|*/| is encountered. Comments nest, contrary to \C/\Cxx:
  if two \lstinline|/*| are encountered, the
  comment will end after two \lstinline|*/|, not one.
\end{itemize}

\begin{urbiscript}[firstnumber=last]
// C++ style comment
/* C style comment */
/* These comments /* do */ nest */
\end{urbiscript}

\subsection{Synclines}
\label{sec:specs:synclines}

While the interaction with an \us kernel is usually performed via a
network connection, programmers are used to work with files which have
names, line numbers and so forth.  This is most important in error
messages.  Since even loading a file actually means sending its
content as if it were typed in the network session, in order to
provide the user with meaningful locations in error messages, \us
features \dfn[syncline]{synclines}, a means to change the ``current
location'', similarly to \lstinline[language=C]|#line| in \C-like
languages.  This is achieved using special \lstinline|//#| comments.

The following special comments are recognized only as a whole line.
If some component does not match exactly the expected syntax, or if
there are trailing items, the whole line is treated as a comment.
\begin{itemize}
%% I failed to use \var for line and file here.  -- AD.
\item \lstinline|//#line line "file"|\\
  Specify that the \emph{next} line is from the file named \var{file},
  and which line number is \var{line}.  The current location (i.e.,
  current file and line) is lost.

\item \lstinline|//#push line "file"|\\
  Save the current location, and then behave as if \lstinline|//#line|
  was used.

\item \lstinline|//#pop|\\
  Restore the current location.  \lstinline|//#push| and
  \lstinline|//#pop| must match.
\end{itemize}


\subsection{Identifiers}
\label{sec:us-syn-id}

\dfn{Identifiers} in \us are composed of one or more alphanumeric or
underscore (\lstinline|_|) characters, not starting by a digit, i.e.,
identifiers match the \lstinline|[a-zA-Z_][a-zA-Z0-9_]*| regular
expression.  Additionally, identifiers must not match any of the \us
reserved words\footnote{
%%
  The only exception to this rule is \lstinline|new|, which can be
  used as the method identifier in a method call.
%%
} documented in \autoref{sec:syn-key}. Identifiers can also be written
between simple quotes (\lstinline|'|), in which case they may contain
any character.

\begin{urbiscript}[firstnumber=last]
var x;
var foobar51;
var this.a_name_with_underscores;
// Invalid.
// var 3x;
// obj.3x();

// Invalid because "if" is a keyword.
// var if;
// obj.if();
// However, keywords can be escaped with simple quotes.
var 'if';
var this.'else';

// Identifiers can be escaped with simple quotes
var '%x';
var '1 2 3';
var this.'[]';
\end{urbiscript}

\subsection{Keywords}
\label{sec:syn-key}

\dfn{Keywords} are reserved words that cannot be used as identifiers,
for instance.  They are listed in \autoref{tab:keywords}.

\renewcommand{\baselinestretch}{.85}
\begin{table}[\floatpos]
  \centering
  \begin{tabular}{|c|c||c|c|}
    \hline
    Keyword                       & Remark                                  &
    Keyword                       & Remark                                  \\
    \hline
    \lstinline"and"               & Synonym for \lstinline|&&|              &
    \lstinline"long"              & Reserved                                \\
    \lstinline"and_eq"            & Synonym for \lstinline|&=|              &
    \lstinline"loop"              & \lstinline|loop&| and
                                    \lstinline-loop|- flavors               \\
    \lstinline"asm"               & Reserved                                &
    \lstinline"loopn"             & Deprecated, use \lstinline|for|         \\
    \lstinline"at"                &                                         &
    \lstinline"mutable"           & Reserved                                \\
    \lstinline"auto"              & Reserved                                &
    \lstinline"namespace"         & Reserved                                \\
    \lstinline"bitand"            & Synonym for \lstinline|&| operator      &
    \lstinline"new"               &                                         \\
    \lstinline"bitor"             & Synonym for \lstinline-|- operator      &
    \lstinline"not"               & Synonym for \lstinline|!| operator      \\
    \lstinline"bool"              & Reserved                                &
    \lstinline"not_eq"            & Synonym for \lstinline|!=| operator     \\
    \lstinline"break"             &                                         &
    \lstinline"object"            &                                         \\
    \lstinline"call"              &                                         &
    \lstinline"onleave"           &                                         \\
    \lstinline"case"              &                                         &
    \lstinline"or"                & Synonym for \lstinline-||- operator     \\
    \lstinline"catch"             & Reserved                                &
    \lstinline"or_eq"             & Synonym for \lstinline-|=- operator     \\
    \lstinline"char"              & Reserved                                &
    \lstinline"private"           & Ignored                                 \\
    \lstinline"class"             &                                         &
    \lstinline"protected"         & Ignored                                 \\
    \lstinline"closure"           &                                         &
    \lstinline"public"            & Ignored                                 \\
    \lstinline"compl"             & Synonym for \lstinline|~|               &
    \lstinline"register"          & Reserved                                \\
    \lstinline"const"             & Reserved                                &
    \lstinline"reinterpret_cast"  & Reserved                               \\
    \lstinline"const_cast"        & Reserved                                &
    \lstinline"return"            &                                         \\
    \lstinline"continue"          &                                         &
    \lstinline"short"             & Reserved                                \\
    \lstinline"default"           & Reserved                                &
    \lstinline"signed"            & Reserved                                \\
    \lstinline"delete"            &                                         &
    \lstinline"sizeof"            & Reserved                                \\
    \lstinline"do"                &                                         &
    \lstinline"static"            & Deprecated                              \\
    \lstinline"double"            & Reserved                                &
    \lstinline"static_cast"       & Reserved                                \\
    \lstinline"dynamic_cast"      & Reserved                                &
    \lstinline"stopif"            &                                         \\
    \lstinline"else"              &                                         &
    \lstinline"struct"            & Reserved                                \\
    \lstinline"emit"              & Deprecated                              &
    \lstinline"switch"            &                                         \\
    \lstinline"enum"              & Reserved                                &
    \lstinline"template"          & Reserved                                \\
                                  &                                         &
    \lstinline"this"              &                                         \\
    \lstinline"every"             &                                         &
    \lstinline"throw"             & Reserved                                \\
    \lstinline"explicit"          & Reserved                                &
    \lstinline"timeout"           &                                         \\
    \lstinline"export"            & Reserved                                &
    \lstinline"try"               & Reserved                                \\
    \lstinline"extern"            & Reserved                                &
    \lstinline"typedef"           & Reserved                                \\
    \lstinline"external"          &                                         &
    \lstinline"typeid"            & Reserved                                \\
    \lstinline"float"             & Reserved                                &
    \lstinline"typename"          & Reserved                                \\
    \lstinline"for"               & \lstinline|for&| and \lstinline-for|- flavors&
    \lstinline"union"             & Reserved                                \\
    \lstinline"foreach"           & Deprecated, use \lstinline|for|    &
    \lstinline"unsigned"          & Reserved                                \\
    \lstinline"freezeif"          &                                         &
    \lstinline"using"             & Reserved                                \\
    \lstinline"friend"            & Reserved                                &
    \lstinline"var"               &                                         \\
    \lstinline"from"              &                                         &
    \lstinline"virtual"           & Reserved                                \\
    \lstinline"function"          &                                         &
    \lstinline"volatile"          & Reserved                                \\
    \lstinline"goto"              & Reserved                                &
    \lstinline"waituntil"         &                                         \\
    \lstinline"if"                &                                         &
    \lstinline"wchar_t"           & Reserved                                \\
    \lstinline"in"                &                                         &
    \lstinline"whenever"          &                                         \\
    \lstinline"inline"            & Reserved                                &
    \lstinline"while"             & \lstinline|while&| and
                                    \lstinline-while|- flavors              \\
    \lstinline"int"               & Reserved                                &
    \lstinline"xor"               & Synonym for \lstinline|^| operator      \\
    \lstinline"internal"          & Deprecated                              &
    \lstinline"xor_eq"            & Synonym \lstinline|^=| operator         \\
    \hline
  \end{tabular}
  \caption{Keywords}
  \label{tab:keywords}
\end{table}
\renewcommand{\baselinestretch}{1}

\subsection{Literals}

\subsubsection{Angles}

\dfn[angle]{Angles} are floats (see \autoref{sec:us-syn-lit-float})
followed by an angle unit. They are simply equivalent to the same
float, expressed in radians. For instance, \lstinline|180deg| (180
degrees) is equal to \lstinline|pi|. Available units and their
equivalent are presented in \autoref{tab:angle}.

\begin{table}[\floatposh]
  \centering
  \begin{tabular}{|c|c|c|}
    \hline
    unit        & abbreviation & equivalence for $n$  \\
    \hline
    radian      & rad          & $n$         \\
    degree      & deg          & $n / 180 * \pi$        \\
    grad        & grad         & $n / 200 * \pi$        \\
    \hline
  \end{tabular}
  \caption{Angle units}
  \label{tab:angle}
\end{table}

\begin{urbiassert}[firstnumber=last]
pi == 180deg;
pi == 200grad;
\end{urbiassert}

\subsubsection{Durations}

\dfn{Durations} are floats (see \autoref{sec:us-syn-lit-float})
followed by a time unit. They are simply equivalent to the same float,
expressed in seconds. For instance, \lstinline|1s 1ms|, which stands
for ``one second and one millisecond'', is strictly equivalent to
\lstinline|1.0001|. Available units and their equivalent are presented
in \autoref{tab:duration}.

\begin{table}
  \centering
  \begin{tabular}{|c|c|c|}
    \hline
    unit        & abbreviation & equivalence for $n$  \\
    \hline
    millisecond & ms           & $n / 1000$         \\
    second      & s            & $n$                \\
    minute      & min          & $n \times 60$           \\
    hour        & h            & $n \times 60 \times 60$      \\
    day         & d            & $n \times 60 \times 60 \times 24$ \\
    \hline
  \end{tabular}
  \caption{Duration units}
  \label{tab:duration}
\end{table}

\begin{urbiassert}[firstnumber=last]
1d == 24h;
1h == 60min;
1min == 60s;
1s == 1000ms;

1s == 1;
1s2s3s == 6;
1s 2s 3s == 6;
1s 1ms == 1.001;
1ms 1s == 1.001;
\end{urbiassert}

\subsubsection{Floats}
\label{sec:us-syn-lit-float}

\us supports the so called \dfn{scientific notation} for
floating-point literals.  See \refObject{Float} for more details.
Examples include:

\begin{urbiscript}[firstnumber=last]
1;
[00000000] 1
1.234e6;
[00000000] 1234000
\end{urbiscript}

\subsubsection{Lists}
\label{sec:us-syn-lit-list}

Literal \dfn{lists} are represented with a comma-separated, potentially
empty list of arbitrary expressions enclosed in square brackets
(\lstinline|[]|), as shown in the listing below.  See
\refObject{List} for more details.

\begin{urbiscript}[firstnumber=last]
[]; // The empty list
[00000000] []
[1, 2, 3];
[00000000] [1, 2, 3]
\end{urbiscript}

\subsubsection{Strings}
\label{sec:us-syn-lit-string}

\dfn{String} literals are enclosed in double quotes (\lstinline|"|)
and can contain arbitrary characters, which stand for themselves, with
the exception of the escape character, backslash (\lstinline|\|), see
below.  The escapes sequences are defined in \autoref{tab:escapes}.

\begin{table}[\floatposh]
  \centering
  \begin{tabular}{|c|p{.6\linewidth}|}
    \hline
    \lstinline|\\| & backslash             \\
    \lstinline|\a| & bell ring             \\
    \lstinline|\b| & backspace             \\
    \lstinline|\f| & form feed             \\
    \lstinline|\n| & line feed             \\
    \lstinline|\r| & carriage return       \\
    \lstinline|\t| & tabulation            \\
    \lstinline|\v| & vertical tabulation   \\

    \lstinline|\[0-7]{3}|
    & eight-bit character corresponding to a three-digit octal number.
    For instance, \lstinline|\000| and \lstinline|177|. \\

    \lstinline|\x[0-9a-fA-F]{2}|
    & eight-bit character corresponding to a two-digit hexadecimal
    number.  For instance, \lstinline|0xfF|. \\

    \lstinline|\B(\var{length})(\var{content})|
    & binary blob.  A \var{length}-long sequence of verbatim
    \var{content}.  \var{length} is expressed in decimal.  \var{content}
    is not interpreted in any way.  The parentheses are part of the syntax,
    they are mandatory.  For instance \lstinline|\B(2)(\B)|\\
    \hline
  \end{tabular}
  \caption{String escapes}
  \label{tab:escapes}
\end{table}

Consecutive string literals are glued together into a unique string.
This is useful to split large strings into chunks that fit usual
programming widths.

\begin{urbiassert}[firstnumber=last]
"foo" "bar" "baz" == "foobarbaz";
"\B(3)("\")" == "\"\\\"";
\end{urbiassert}

The interpreter prints the strings escaped; for instance, line feed
will be printed out as \lstinline|\n| when a string result is dumped
and so forth. An actual line feed will of course be output if a string
content is printed with echo for instance.

\begin{urbiscript}[firstnumber=last]
"";
[00000000] ""
"foo";
[00000000] "foo"
"a\nb"; // urbiscript escapes string when dumping them
[00000000] "a\nb"
echo("a\nb"); // We can see there is an actual line feed
[00000000] *** a
[:]b
echo("a\\nb");
[00000000] *** a\nb
\end{urbiscript}

See \refObject{String} for more details.

\subsection{Statement Separators}
\label{sec:lang:separators}

Sequential languages such as \Cxx support a single way to compose two
statements: the sequential composition, ``denoted'' by \samp{;}.  To
support concurrency and more fined tuned sequentiality, \us features
four different statement-separators (or connectors):
\begin{description}
\item[\samp{;}] sequentiality
\item[\samp{|}] tight sequentiality
\item[\samp{,}] background concurrency
\item[\samp{\&}] fair-start concurrency
\end{description}

\subsubsection{\samp{;}}

The \samp{;}-connector waits for the first statement to finish before
starting the second statement.  When used in the top-level interactive
session, both results are displayed.

\begin{urbiscript}[firstnumber=last]
1; 2; 3;
[00000000] 1
[00000000] 2
[00000000] 3
\end{urbiscript}

\subsubsection{\samp{,}}

The \samp{,}-connector sends the first statement in background for
concurrent execution, and starts the second statement when possible.
When used in interactive sessions, the value of back-grounded
statements are \emph{not} printed --- the time of their arrival being
unpredictable, such results would clutter the output randomly.  Use
\refObject[s]{Channel} or \refObject[s]{Event} to return results
asynchronously.

\begin{urbiscript}[firstnumber=last]
{
  for (3)
  {
    sleep(1s);
    echo("ping");
  },
  sleep(0.5s);
  for (3)
  {
    sleep(1s);
    echo("pong");
  },
};
[00000316] *** ping
[00000316] *** pong
[00000316] *** ping
[00000316] *** pong
[00000316] *** ping
[00000316] *** pong
\end{urbiscript}

Both \samp{;} and \samp{,} have equal precedence.  They are scoped
too: the execution follow ``waits'' for the end of the jobs
back-grounded with \samp{,} before proceeding.  Compare the two
following executions.

\begin{urbiscript}[firstnumber=last]
{
  sleep(100ms) | echo("1"),
  sleep(400ms) | echo("2"),
  echo("done");
};
[00000316] *** done
[00000316] *** 1
[00000316] *** 2
\end{urbiscript}

\begin{urbiscript}[firstnumber=last]
{
  sleep(100ms) | echo("1"),
  sleep(400ms) | echo("2"),
};
echo("done");
[00000316] *** 1
[00000316] *** 2
[00000316] *** done
\end{urbiscript}


\subsubsection{\samp{|}}
When using the \samp{;} connector, the scheduler is allowed to run
other commands between the first and the second statement.  The
\samp{|} does not yield between both statements.  It is therefore more
efficient, and, in a way, provides some atomicity for concurrent tasks.

\begin{urbiscript}[firstnumber=last]
{
  { echo("11") ; sleep(100ms) ; echo("12") },
  { echo("21") ; sleep(400ms) ; echo("22") },
};
[00000316] *** 11
[00000316] *** 21
[00000316] *** 12
[00000316] *** 22
\end{urbiscript}

%% Cannot use sleep here, as it yields, which makes the point moot.
\begin{urbiscript}[firstnumber=last]
{
  { echo("11") | echo("12") },
  { echo("21") | echo("22") },
};
[00000316] *** 11
[00000316] *** 12
[00000316] *** 21
[00000316] *** 22
\end{urbiscript}

In an interactive session, both statements must be ``known'' before
launching the sequence.  The value of the composed statement is the
value of the second statement.

\subsubsection{\samp{\&}}

The \samp{\&} is very similar to the \samp{,} connector, but for its
precedence.  \urbi expects to process the whole statement before
launching the connected statements.   This is especially handy in
interactive sessions, as a means to fire a set of tasks concurrently.


\subsection{Operators}

\us supports many \dfn{operators}, most of which are inspired from
\Cxx. Their syntax is presented here, and they are sorted and
described with their original semantics --- that is, \lstinline|+| is
an arithmetic operator that sums two numeric values. However, as in
\Cxx, these operators might be use for any other purpose --- that is,
\lstinline|+| can also be used as the concatenation operator on lists
and strings. Their semantics is thus not limited to what is presented
here.

Tables in this section sort operators top-down, by precedence order.
Group of rows (not separated by horizontal lines) describe operators
that have the same precedence. Many operators are syntactic sugar that
bounce on a method. In this case, the equivalent desugared expression
is shown in the ``Equivalence'' column. This can help you determine
what method to override to define an operator for an object (see
\autoref{sec:tut:operators}).

This section defines the syntax, precedence and associativity of the
operators. Their semantics is described in \autoref{sec:stdlib} in the
documentation of the classes that provide them.

% Operator generators
\newcommand{\operatorhead}{Operator & Use & Associativity & Original semantic
  & Equivalence\\}


\newcommand{\operator}[6][ ]{\lstinline@#2@&\lstinline@#3@&#4&#5&\lstinline@#6@#1\\}
\newcommand{\boperator}[3]{\operator{#1}{a #1 b}{#2}{#3}{a.'#1'(b)}}
\newcommand{\poperator}[3]{\operator{#1}{#1a}{#2}{#3}{a.'#1'()}}

\newcommand{\operatordot}    {\operator  {.}    {a.b}              {-}     {Message sending}          {Not redefinable}       }
\newcommand{\operatordota}   {\operator  {.}    {a.b(args)}        {-}     {Message sending}          {Not redefinable}       }
\newcommand{\operatorsub}    {\operator  {[]}   {a[args]}          {-}     {Subscript}                {a.'[]'(args)}          }
\newcommand{\operatorsubass} {\operator  {[] =} {a[args] = v}      {-}     {Subscript assignment}     {a.'[]='(args, v)}      }
\newcommand{\operatorass}[2][ ]    {\operator[#1]
                                         {=}    {a = b}            {Right} {Assignment}               {updateSlot("a", b)}    }

\newcommand{\operatoriass}[1]{\operator  {#1=}  {a #1= b}          {Right} {In place assignment}      {a = a #1 b}            }
\newcommand{\operatorsiass}  {
    \operatoriass{+}
    \operatoriass{-}
    \operatoriass{*}
    \operatoriass{/}
    \operatoriass{\%}
    \operatoriass{\^}
%    \operatoriass{\~}
}
\newcommand{\operatorinc}    {\operator  {++}   {a++}              {-}     {Incrementation}           {(a = a + 1) - 1}       }
\newcommand{\operatordec}    {\operator  {--}   {a--}              {-}     {Incrementation}           {(a = a - 1) + 1}       }

\newcommand{\operatoruplus}  {\poperator {+}    {-}                {Identity}               }
\newcommand{\operatorumin}   {\poperator {-}    {-}                {Opposite}               }
\newcommand{\operatorexp}    {\boperator {**}   {Right}            {Exponentiation}         }
\newcommand{\operatormult}   {\boperator {*}    {Left}             {Multiplication}         }
\newcommand{\operatordiv}    {\boperator {/}    {Left}             {Division}               }
\newcommand{\operatormod}    {\boperator {\%}   {Left}             {Modulo}                 }
\newcommand{\operatorplus}   {\boperator {+}    {Left}             {Sum}                    }
\newcommand{\operatorminus}  {\boperator {-}    {Left}             {Difference}             }
\newcommand{\operatorlshift} {\boperator {<<}   {Left}             {Left bit shift}         }
\newcommand{\operatorrshift} {\boperator {>>}   {Left}             {Right bit shift}        }
\newcommand{\operatoreq}     {\boperator {==}   {Non Associative}  {Equality}               }
\newcommand{\operatorneq}    {\boperator {!=}   {Non Associative}  {Inequality}             }
\newcommand{\operatorpeq}    {\boperator {===}  {Non Associative}  {Physical equality}      }
\newcommand{\operatorpneq}   {\boperator {!==}  {Non Associative}  {Physical Inequality}    }
\newcommand{\operatoraeq}    {\boperator {\~=}  {Non Associative}  {Relative Approximate equality} }
\newcommand{\operatoreqaeq}  {\boperator {=~=}  {Non Associative}  {Absolute Approximate equality} }
\newcommand{\operatorinf}    {\boperator {<}    {Non Associative}  {Less than}              }
\newcommand{\operatorinfeq}  {\boperator {<=}   {Non Associative}  {Less than or equal to}  }
\newcommand{\operatorsup}    {\boperator {>}    {Non Associative}  {Greater than}           }
\newcommand{\operatorsupeq}  {\boperator {>=}   {Non Associative}  {Greater than or equal to}}
\newcommand{\operatorbxor}   {\boperator {^}    {Left}             {Bitwise exclusive or}   }
\newcommand{\operatorneg}    {\poperator {!}    {Left}             {Logical negation}       }
\newcommand{\operatorand}    {\boperator {\&\&} {Left}             {Logical and}            }
\newcommand{\operatoror}     {\boperator {||}   {Left}             {Logical or}             }

\subsubsection{Arithmetic operators}

\us supports classic \dfn{arithmetic operators}, with the classic
semantics on numeric values. See \autoref{tab:arith} and the listing
below.

\begin{table}[\floatposh]
  \centering
  \begin{tabular}{|c|c|c|c|c|c|}
    \hline
    \operatorhead
    \hline
    \operatoruplus
    \operatorumin
    \hline
    \operatorexp
    \hline
    \operatormult
    \operatordiv
    \operatormod
    \hline
    \operatorplus
    \operatorminus
    \hline
  \end{tabular}
  \caption{Arithmetic operators}
  \label{tab:arith}
\end{table}

\begin{urbiassert}[firstnumber=last]
   1 + 1 ==    2;
   1 - 2 ==   -1;
   2 * 3 ==    6;
  10 / 2 ==    5;
 2 ** 10 == 1024;
-(1 + 2) ==   -3;
\end{urbiassert}

\subsubsection{Assignment operators}

\dfn{Assignment} in \us can be performed with the \lstinline|=|
operator.  Shorthands such as \lstinline|+=| exist; they are not
redefinable since they are equivalent to a regular assignment combined
with another operator. See \autoref{tab:assignment} and the listing
below.


\begin{table}[\floatposh]
  \centering
  \begin{tabular}{|c|c|c|c|c|c|}
    \hline
    \operatorhead
    \hline
    \operatorass[\footnotemark]{}
    \operatorsiass
    \hline
  \end{tabular}
  \caption{Assignment operators}
  \label{tab:assignment}
\end{table}
\footnotetext{For object fields only. Assignment to local variables
  cannot be redefined. }

% FIXME: this in place modulo example was removed
%         because %= is a lame Urbi operator.
%  x %= 3;


\begin{urbiscript}[firstnumber=last]
var y = 0;
[00000000] 0
y = 10;
[00000000] 10
y += 10;
[00000000] 20
y /= 5;
[00000000] 4
y++;
[00000000] 4
y;
[00000000] 5
\end{urbiscript}

\subsubsection{Bitwise operators}

\us features \dfn{bitwise operators}.  They are also used for other
purpose than bit-related operations. See \autoref{tab:bitwise} and the
listing below.

\begin{table}[\floatposh]
  \centering
  \begin{tabular}{|c|c|c|c|c|c|}
    \hline
    \operatorhead
    \hline
    \operatorlshift
    \operatorrshift
    \hline
    \operatorbxor
    \hline
  \end{tabular}
  \caption{Bitwise operators}
  \label{tab:bitwise}
\end{table}

\begin{urbiassert}[firstnumber=last]
4 << 2 == 16;
4 >> 2 ==  1;
\end{urbiassert}

\subsubsection{Logical operators}

\us supports the usual \dfn{Boolean operators}. See the table and the
listing below. The operators \lstinline|&&| and \lstinline-||- are
short-circuiting: their right-hand side is evaluated only if needed.

\begin{table}[\floatposh]
  \centering
  \begin{tabular}{|c|c|c|c|c|c|}
    \hline
    \operatorhead
    \hline
    \operatorneg
    \hline
    \operatorand
    \hline
    \operatoror
    \hline
  \end{tabular}
  \caption{Boolean operators}
  \label{tab:boolean}
\end{table}

\begin{urbiassert}[firstnumber=last]
true && true;
true || false;
!true == false;
true || (1 / 0);
(false && (1 / 0)) == false;
\end{urbiassert}

\subsubsection{Comparison operators}

\us supports classical \dfn{comparison operators}. See
\autoref{tab:comparison} and the listing below.

\begin{table}[\floatposh]
  \centering
  \begin{tabular}{|c|c|c|c|c|c|}
    \hline
    \operatorhead
    \hline
    \operatoreq
    \operatorneq
    \operatorpeq
    \operatorpneq
    \operatoraeq
    \operatoreqaeq
    \operatorinf
    \operatorinfeq
    \operatorsup
    \operatorsupeq
    \hline
  \end{tabular}
  \caption{Comparison operators}
  \label{tab:comparison}
\end{table}

\begin{urbiscript}[firstnumber=last]
assert
{
 ! (0 < 0);
    0 <= 0;
    0 == 0;
   0 !== 0;
};
var z = 0;
[00000000] 0
assert
{
  z === z;
  ! (z !== z);
};
\end{urbiscript}

\subsubsection{Miscellaneous operators}

These operators do not fit the previous categories. See the table and
the listing below. Note that the \dfn[operator!subscript]{subscript}
(square bracket) operator is \dfn{variadic}: it takes any number of
arguments that will be passed to the \lstinline|'[]'| method of the
targeted object.

\begin{table}[\floatposh]
  \begin{tabular}{|c|c|c|c|c|c|}
    \hline
    \operatorhead
    \hline
    \operatordot
    \operatordota
    \hline
    \operatorsub
    \operatorsubass
    \hline
  \end{tabular}
  \caption{Miscellaneous operators}
\end{table}

\begin{urbiscript}[firstnumber=last]
// On lists.
var l = [1, 2, 3, 4, 5];
[00000000] [1, 2, 3, 4, 5]
assert
{
  l[0] == 1;
  l[-1] == 5;
  (l[0] = 10) == 10;
  l == [10, 2, 3, 4, 5];
};

// On strings.
var s = "abcdef";
[00000005] "abcdef"
assert
{
  s[0] == "a";
  s[1,3] == "bc";
  (s[1,3] = "foo") == "foo";
  s == "afoodef";
};
\end{urbiscript}

% \clearpage
\subsubsection{All operators summary}

\autoref{tab:operators-summary} is a summary of all operators, to
highlight the overall precedences. Operators are sorted by decreasing
precedence. Groups of rows represent operators with the same
precedence.

\begin{table}[\floatposh]
  \begin{tabular}{|c|c|c|c|c|c|}
    \hline
    Operator               & Use                                    & Associativity
    & Original semantic    & Equivalence                            \\
    \hline
    \operatordot
    \operatordota
    \hline
    \operatorsub
    \operatorsubass
    \hline
    \operatoruplus
    \operatorumin
    \hline
    \operatorexp
    \hline
    \operatormult
    \operatordiv
    \operatormod
    \hline
    \operatorplus
    \operatorminus
    \hline
    \operatorlshift
    \operatorrshift
    \hline
    \operatoreq
    \operatorneq
    \operatorpeq
    \operatorpneq
    \operatoreqaeq
    \operatoraeq
    \operatorinf
    \operatorinfeq
    \operatorsup
    \operatorsupeq
    \hline
    \operatorbxor
    \hline
    \operatorneg
    \hline
    \operatorand
    \hline
    \operatoror
    \hline
    \operatorass
    \operatorsiass
    \hline
    \operatorinc
    \operatordec
    \hline
  \end{tabular}
  \caption{Operators summary}
  \label{tab:operators-summary}
\end{table}


\section{Scopes and local variables}

\subsection{Scopes}

\dfn{Scopes} are sequences of statements, enclosed in curly brackets
(\lstinline|{}|). Statements are separated with the four statements
separators (see \autoref{sec:lang:separators}).  A trailing \samp{;}
or \samp{,} is ignored.  A trailing \samp{\&} or \samp{|} behaves as
if \lstinline|& {}| or \lstinline'| {}' was used.  This particular
case is heavily used by \us programmers to discard the value of an
expression:

\begin{urbiscript}[firstnumber=last]
// Return value is 1.  Displayed.
1;
[00000000] 1
// Return value is that of {}, i.e., void.  Nothing displayed.
1 | {};
// Same as "1 | {}", a valueless expression.
1|;
\end{urbiscript}

Scopes are themselves expressions, and can thus be used in composite
expressions, nested, and so forth.

\begin{urbiscript}[firstnumber=last]
// Scopes evaluate to their last expression
{
  1;
  2;
  3; // This last separator is optional.
};
[00000000] 3
// Scopes can be used as expressions
{1; 2; 3} + 1;
[00000000] 4
\end{urbiscript}

\subsection{Local variables}

\dfn{Local variables} are introduced with the \lstinline|var| keyword,
followed by an identifier (see \autoref{sec:us-syn-id}) and an optional
initialization value assignment. If the initial value is omitted, it
defaults to \refObject{void}. Variable
declarations evaluate to
the initialization value. They can later be referred to by their
name. Their value can be changed with the assignment operator; such an
assignment expression returns the new value. The use of local
variables is illustrated below.

\begin{urbiscript}[firstnumber=last]
// This declare variable x with value 42, and evaluates to 42.
var t = 42;
[00000000] 42
// x equals 42
t;
[00000000] 42
// We can assign it a new value
t = 51;
[00000000] 51
t;
[00000000] 51
// Initialization defaults to void
var u;
u.isVoid;
[00000000] true
\end{urbiscript}

The lifespan of local variables is the same as their enclosing scope. They
are thus only accessible from their scope and its
sub-scopes\footnote{Local variables can actually escape their scope
  with lexical closures, see \autoref{sec:us-fun-closures}.}. Two
variables with the same name cannot be defined in the same scope. A
variable with the same name can be defined in an inner scope, in which
case references refer to the innermost variable, as shown below.

\begin{urbiscript}[firstnumber=last]
{
  var x = "x";
  var y = "outer y";
  {
    var y = "inner y";
    var z = "z";
    // We can access variables of parent scopes.
    echo(x);
    // This refers to the inner y.
    echo(y);
    echo(z);
  };
  // This refers to the outer y.
  echo(y);
  // This would be invalid: z does not exist anymore.
  // echo(z);
  // This would be invalid: x is already declared in this scope.
  // var x;
};
[00000000] *** x
[00000000] *** inner y
[00000000] *** z
[00000000] *** outer y
\end{urbiscript}


\section{Functions}

\subsection{Function Definition}

\dfn{Functions} in \us are first class citizens: a function is a
value, like floats and strings, and can be handled as such.  This is
different from most \C-like languages.  One can create a functional
value thanks to the \lstinline|function| keyword, followed by the list
of formal arguments and a compound statement representing the body of
the function. Formal arguments are a possibly-empty comma-separated
list of identifiers.  Non-empty lists of formal arguments may
optionally end with a trailing comma. The listing below illustrates
this.

\begin{urbiscript}[firstnumber=last]
function () { echo(0) };
[00000000] function () {
[:]  echo(0)
[:]}

function (arg1, arg2) { echo(0) };
[00000000] function (arg1, arg2) {
[:]  echo(0)
[:]}

function (
           arg1, // Ignored argument.
           arg2, // Also ignored.
          )
{
  echo(0)
};
[00000000] function (arg1, arg2) {
[:]  echo(0)
[:]}
\end{urbiscript}

Usually functions are bound to an identifier to be invoked later.
The listing below shows a short-hand to define a named
function.

\begin{urbiscript}[firstnumber=last]
// Functions are often stored in variables to call them later.
var f1 = function () {
  echo("hello")
}|
f1();
[00000000] *** hello

// This form is strictly equivalent, yet simpler.
function f2()
{
  echo("hello")
}|
f2();
[00000000] *** hello
\end{urbiscript}


\subsection{Arguments}

The list of formal arguments defines the number of argument the
function requires. They are accessible by their name from within the
body. If the list of formal arguments is omitted, the number of
effective arguments is not checked, and arguments themselves are not
evaluated. Arguments can then be manipulated with the call message,
explained below.

\begin{urbiscript}[firstnumber=last]
var f = function(a, b) {
  echo(b + a);
}|
f(1, 0);
[00000000] *** 1
// Calling a function with the wrong number of argument is an error.
f(0);
[00000000:error] !!! f: expected 2 arguments, given 1
f(0, 1, 2);
[00000000:error] !!! f: expected 2 arguments, given 3
\end{urbiscript}

Non-empty lists of effective arguments may end with an optional comma.
\begin{urbiscript}[firstnumber=last]
f(
  "bar",
  "foo",
 );
[00000000] *** foobar
\end{urbiscript}


\subsection{Return value}

The \dfn[function!return value]{return value} of the function is the
evaluation of its body --- that is, since the body is a scope, the
last evaluated expression in the scope.  Values can be returned
manually with the \lstinline|return| keyword followed by the value, in
which case the evaluation of the function is stopped. If
\lstinline|return| is used with no value, the function returns
\lstinline|void|.

\begin{urbiscript}[firstnumber=last]
function g1(a, b)
{
  echo(a);
  echo(b);
  a // Return value is a
}|
g1(1, 2);
[00000000] *** 1
[00000000] *** 2
[00000000] 1

function g2(a, b)
{
  echo(a);
  return a; // Stop execution at this point and return a
  echo(b); // This is not executed
}|
g2(1, 2);
[00000000] *** 1
[00000000] 1

function g3()
{
  return; // Stop execution at this point and return void
  echo(0); // This is not executed
}|
g3(); // Returns void, so nothing is printed.
\end{urbiscript}

\subsection{Call messages}
\label{sec:us-fun-callmsg}

Functions can access meta-information about how they were called,
through a \lstinline|CallMessage| object. The \dfn{call message}
associated with a function can be accessed with the \lstinline|call|
keyword. It contains several information such as not-yet evaluated
arguments, the name of the function, the target \ldots

\subsection{Strictness}

\us features two different function calls:
\dfn[function!strict]{strict} function calls, effective arguments are
evaluated before invoking the function, and \dfn[function!lazy]{lazy}
function calls, arguments are passed as-is to the function.  As a
matter of fact, the difference is rather that there are strict
functions and lazy function.

Functions defined with a (possibly empty) list of formal arguments in
parentheses are strict: the effective arguments are first evaluated,
and then their value is given to the called function.

\begin{urbiscript}[firstnumber=last]
function first1(a, b) {
  echo(a); echo(b)
}|
first1({echo("Arg1"); 1},
       {echo("Arg2"); 2});
[00000000] *** Arg1
[00000000] *** Arg2
[00000000] *** 1
[00000000] *** 2
\end{urbiscript}

A function declared with no formal argument list is lazy.  Use its
call message to manipulate its \emph{unevaluated} arguments.
The listing below gives an example.  More information about
this can be found in the \refObject{CallMessage} class documentation.

\begin{urbiscript}[firstnumber=last]
function first2
{
  echo(call.evalArgAt(0));
  echo(call.evalArgAt(1));
}|
first2({echo("Arg1"); 1},
       {echo("Arg2"); 2});
[00000000] *** Arg1
[00000000] *** 1
[00000000] *** Arg2
[00000000] *** 2
\end{urbiscript}

A lazy function may implement a strict interface by evaluating its
arguments and storing them as local variables, see below.  This is
less efficient than defining a strict function.

\begin{urbiscript}[firstnumber=last]
function first3
{
  var a = call.evalArgAt(0);
  var b = call.evalArgAt(1);
  echo(a); echo(b);
}|
first3({echo("Arg1"); 1},
       {echo("Arg2"); 2});
[00000000] *** Arg1
[00000000] *** Arg2
[00000000] *** 1
[00000000] *** 2
\end{urbiscript}

\subsection{Lexical closures}
\label{sec:us-fun-closures}

\dfn{Lexical closures} are an additional scoping rule, with which a function
can refer to a local variable located outside the function --- but still
in the current context. \us supports read/write lexical closures,
meaning that the variable is shared between the function and the outer
environment, as shown below.

\begin{urbiscript}[firstnumber=last]
var n = 0|
function cl()
{
  // x refers to a variable outside the function
  n++;
  echo(n);
}|
cl();
[00000000] *** 1
n;
[00000000] 1
n++;
[00000000] 1
cl();
[00000000] *** 3
\end{urbiscript}

The following listing illustrate that local variables can even
escape their declaration scope by lexical closure.

\begin{urbiscript}[firstnumber=last]
function wrapper()
{
  // Normally, x is local to 'wrapper', and is limited to this scope.
  var x = 0;
  at (x > 1)
    echo("ping");
  // Here we make it escape the scope by returning a closure on it.
  return function() { x++ };
} |

var w = wrapper()|
w();
[00000000] 0
w();
[00000000] 1
[00000000] *** ping
\end{urbiscript}

See \autoref{sec:faq:atexp} for more details about the warning.


\section{Objects}

Any value in \us is an object. Objects contain:

\begin{itemize}
\item A list of prototypes, which are also objects.
\item A list of slots, which to a name associate an object.
\end{itemize}

\subsection{Slots}

\subsubsection{Manipulation}

\dfn{Objects} can contain any number of \dfn{slots}, every slot has a
name and a value. Slots are often called ``fields'', ``attributes'' or
``members'' in other object-oriented languages.

The \lstinline|createSlot| function adds a slot to an object with the
void (\autoref{sec:std-void}) value. The \lstinline|updateSlot|
function changes the value of a slot; \lstinline|getSlot| reads
it. The \lstinline|setSlot| method creates a slot with a given
value. Finally, the \lstinline|localSlotNames| method returns the list of
the object slot's name. The listing below shows how to manipulate
slots. More documentation about these methods can be found in
\autorefObject{Object}.

\begin{urbiscript}[firstnumber=last]
var o = Object.new|
o.localSlotNames;
[00000000] []
o.createSlot("test");
o.localSlotNames;
[00000000] ["test"]
o.getSlot("test").asString;
[00000000] "void"
o.updateSlot("test", 42);
[00000000] 42
o.getSlot("test");
[00000000] 42
\end{urbiscript}

\subsubsection{Syntactic Sugar}

There is some syntactic sugar for slot methods:
\begin{itemize}
\item \lstinline|var o.name| is equivalent to
  \lstinline|o.createSlot("name")|.
\item \lstinline|var o.name = value| is equivalent to
  \lstinline|o.setSlot("name", value)|.
\item \lstinline|o.name = value| is equivalent to
  \lstinline|o.updateSlot("name", value)|.
\end{itemize}


\subsection{Prototypes}

\subsubsection{Manipulation}

\us is a prototype-based language, unlike most classical object
oriented language, which are class-based. In prototype-based
languages, objects have no type, only \dfn{prototypes}, from which they
inherit behavior.

\us objects can have several prototypes. The list of prototypes is
given by the \lstinline|protos| method; they can be added or removed
with \lstinline|addProto| and \lstinline|removeProto|.  See
\autorefObject{Object} for more documentation.

\begin{urbiscript}[firstnumber=last]
var ob = Object.new|
ob.protos;
[00000000] [Object]
ob.addProto(Pair);
[00000000] (nil, nil)
ob.protos;
[00000000] [(nil, nil), Object]
ob.removeProto(Object);
[00000000] (nil, nil)
ob.protos;
[00000000] [(nil, nil)]
\end{urbiscript}

\subsubsection{Inheritance}

Objects inherit their prototypes' slots: \lstinline|getSlot| will also
look in an object prototypes' slots. \lstinline|getSlot| performs a
depth-first traversal of the prototypes hierarchy to find slots. That
is, when looking for a slot in an object:

\begin{itemize}
\item \lstinline|getSlot| checks first if the object itself has the
  requested slot. If so, it returns its value.
\item Otherwise, it applies the same research on every prototype, in
  the order of the prototype list (since addProto inserts in the front
  of the prototype list, the last prototype added has priority). This
  search is recursive: \lstinline|getSlot| will also look in the first
  prototype's prototype, etc before looking in the second
  prototype. If the slot is found in a prototype, it is returned.
\item Finally, if no prototype had the slot, an error is raised.
\end{itemize}

This listing shows how slots are inherited.

\begin{urbiscript}[firstnumber=last]
var a = Object.new|
var b = Object.new|
var c = Object.new|
a.setSlot("x", "slot in a")|
b.setSlot("x", "slot in b")|
// c has no "x" slot
c.getSlot("x");
[00000000:error] !!! lookup failed: x
// c can inherit the "x" slot from a.
c.addProto(a)|
c.getSlot("x");
[00000000] "slot in a"
// b is prepended to the prototype list, and has thus priority
c.addProto(b)|
c.getSlot("x");
[00000000] "slot in b"
// a local slot in c has priority over prototypes
c.setSlot("x", "slot in c")|
c.getSlot("x");
[00000000] "slot in c"
\end{urbiscript}

\subsubsection{Copy on write}

The \lstinline|updateSlot| method has a particular behavior with
respect to prototypes. Although performing an \lstinline|updateSlot|
on a non existent slot is an error, it is valid if the slot is
inherited from a prototype. In this case, the slot is however not
updated in the prototype, but rather created in the object itself,
with the new value. This process is called \dfn{copy on write}; thanks
to it, prototypes are not altered when the slot is overridden in a
child object.

\begin{urbiscript}[firstnumber=last]
var p = Object.new|
var p.slot = 0|
var d = Object.new|
d.addProto(p)|
d.slot;
[00000000] 0
d.slot = 1;
[00000000] 1
// p's slot was not altered
p.slot;
[00000000] 0
// It was copied in d
d.slot;
[00000000] 1
\end{urbiscript}

\subsection{Sending messages}

A \dfn{message} in \us consists in a message name and arguments. One can
send a message to an object with the dot (\lstinline|.|) operator,
followed by the message name (which can be any valid identifier) and
the arguments, as shown below. When there are no
arguments, the parentheses can be omitted. As you might see,
sending messages is very similar to calling methods in classical
languages.

\begin{urbiunchecked}
// Send the message msg to object obj, with arguments arg1 and arg2.
obj.msg(arg1, arg2);
// Send the message msg to object obj, with no arguments.
obj.msg();
// This is strictly equivalent.
obj.msg;
\end{urbiunchecked}

When a message \var{msg} is sent to object \lstinline|obj|:

\begin{itemize}
\item The \var{msg} slot of \lstinline|obj| is retrieved (i.e.,
  \lstinline|obj.getSlot("\var{msg}")|). If the slot is not found, the
  classic lookup error is raised.
\item If the object is a \dfn{routine} (either a primitive, written in
  \Cxx for instance, or a function implemented in \us), it is invoked
  with the message arguments, and the returned value is the result. As
  a consequence, the number of arguments in the message sending must
  match the one required by the routine.
\item Otherwise (the object is not a routine), this object is the
  result of the message sending. There must be no argument.
\end{itemize}

Such message sending is illustrated below.

\begin{urbiscript}[firstnumber=last]
var obj = Object.new|
var obj.a = 42|
var obj.b = function (x) { x + 1 }|
obj.a;
[00000000] 42
obj.a();
[00000000] 42
obj.a(50);
[00000000:error] !!! a: expected 0 argument, given 1
obj.b;
[00000000:error] !!! b: expected 1 argument, given 0
obj.b();
[00000000:error] !!! b: expected 1 argument, given 0
obj.b(50);
[00000000] 51
\end{urbiscript}

\section{Imperative flow control}

\subsection{break}

When encountered within a \lstinline|for| or a \lstinline|while| loop,
\lstinline|break| makes the execution jump after the loop.

\begin{urbiscript}[firstnumber=last]
var i = 5|
for (; true; echo(i))
{
  if (i > 8)
    break;
  i++;
};
[00000000] *** 6
[00000000] *** 7
[00000000] *** 8
[00000000] *** 9
\end{urbiscript}

\subsection{continue}

When encountered within a \lstinline|for| or a \lstinline|while| loop,
\lstinline|continue| short-circuits the rest of the loop-body, and
runs the next iteration (if there remains one).

\begin{urbiscript}[firstnumber=last]
for (var i = 0; i < 8; i++)
{
  if (i % 2 != 0)
    continue;
  echo(i);
};
[00000000] *** 0
[00000000] *** 2
[00000000] *** 4
[00000000] *** 6
\end{urbiscript}

\subsection{do}

The \lstinline|do| construct changes the target (\lstinline|this|)
when evaluating an expression.  It is a convenient means to avoid
repeating the same target several times.

\begin{urbiunchecked}
do (\var{target})
{
  \var{body}
};
\end{urbiunchecked}

It evaluates \var{body}, with \lstinline|this| being \var{target}, as
shown below.  The whole construct evaluates to the value
of \var{body}.

\begin{urbiscript}[firstnumber=last]
do (1024)
{
  assert(this == 1024);
  assert(sqrt == 32);
  setSlot("y", 23);
}.y;
[00000000] 23
\end{urbiscript}


\subsection{if}
\label{sec:lang:if}
As in most programming languages, conditionals are expressed with
\lstinline|if|.

\begin{urbiunchecked}
if (\var{condition}) \var{then-clause}
if (\var{condition}) \var{then-clause} else \var{else-clause}
\end{urbiunchecked}

First \var{condition} is evaluated; if it evaluates to a value which
is true (\autoref{sec:truth}), evaluate \var{then-clause}, otherwise,
if applicable, evaluate \var{else-clause}.

\begin{urbiscript}[firstnumber=last]
if (true) assert(true) else assert(false);
if (false) assert(false) else assert(true);
if (true) assert(true);
\end{urbiscript}

Beware that \emph{there must not be a terminator after the
  \var{then-clause}}:

\begin{urbiscript}[firstnumber=last]
if (true)
  assert(true);
else
  assert(false);
[00000002:error] !!! syntax error, unexpected else
\end{urbiscript}

Contrary to \C/\Cxx, it has value: it also implements the
\lstinline|\var{condition} ? \var{then-clause} : \var{else-clause}|
construct.  Unfortunately, due to syntactic constraints inherited from
\C, it is a \emph{statement}: it cannot be used directly as an
expression.  But as everywhere else in \us, to use a statement where
an expression is expected, use braces:

\begin{urbiscript}[firstnumber=last]
assert(1 + if (true) 3 else 4 == 4);
[00000003:error] !!! syntax error, unexpected if
assert(1 + { if (true) 3 else 4 } == 4);
\end{urbiscript}

The \var{condition} can be any statement list.  Variables which it
declares are visible in both the \var{then-clause} and the
\var{else-clause}, but do not escape the \lstinline|if| construct.

\begin{urbiscript}[firstnumber=last]
assert({if (false) 10 else 20} == 20);
assert({if (true)  10 else 20} == 10);

assert({if (true) 10         } == 10);

assert({if (var x = 10) x + 2 else x - 2} == 12);
assert({if (var x = 0)  x + 2 else x - 2} == -2);

if (var xx = 123) xx | xx;
[00000005:error] !!! lookup failed: xx
\end{urbiscript}

\subsection{for}
\label{sec:lang:for}
\lstinline|for| comes in several flavors.

\subsubsection{C-like for}

\us support the classical \C-like \lstinline|for| construct.

\begin{urbiunchecked}
for (\var{initialization}; \var{condition}; \var{increment})
  \var{body}
\end{urbiunchecked}

It has the exact same behavior as \C's \lstinline|for|:

\begin{enumerate}
\item The \var{initialization} is evaluated.
\item \var{condition} is evaluated. If the result is false, executions
  jump after \lstinline|for|.
\item \var{body} is evaluated. If \lstinline|continue| is encountered,
  execution jumps to point 4. If \lstinline|break| is encountered,
  executions jumps after the \lstinline|for|.
\item The \var{increment} is evaluated.
\item Execution jumps to point 2.
\item The loop evaluates to \lstinline|void|.
\end{enumerate}

\subsubsection{Range-for}
\label{sec:lang:for:each}

\us supports iteration over a collection with another form of the
\lstinline|for| loop.

\begin{urbiunchecked}
for (var \var{name} : \var{collection})
   \var{body};
\end{urbiunchecked}

It evaluates \var{body} for each element in \var{collection}. The loop
evaluates to \lstinline|void|.  Inside \var{body}, the current element
is accessible via the \var{name} local variable. The listing below
illustrates this.

\begin{urbiscript}[firstnumber=last]
for (var x : [0, 1, 2, 3, 4])
  echo(x.sqr);
[00000000] *** 0
[00000000] *** 1
[00000000] *** 4
[00000000] *** 9
[00000000] *** 16
\end{urbiscript}

This form of \lstinline|for| simply sends the \lstinline|each| message
to \var{collection} with one argument: the function that takes the
current element and performs \lstinline|action| over it. Thus, you can
make any object acceptable in a \lstinline|for| by defining an
adequate \lstinline|each| method.

\begin{urbiscript}[firstnumber=last]
var Hobbits = Object.new|
function Hobbits.each (action)
{
  action("Frodo");
  action("Merry");
  action("Pippin");
  action("Sam");
}|
for (var name in Hobbits)
  echo("%s is a hobbit." % [name]);
[00000000] *** Frodo is a hobbit.
[00000000] *** Merry is a hobbit.
[00000000] *** Pippin is a hobbit.
[00000000] *** Sam is a hobbit.
// This for statement is equivalent to:
Hobbits.each(function (name) { echo("%s is a hobbit." % [name]) });
[00000000] *** Frodo is a hobbit.
[00000000] *** Merry is a hobbit.
[00000000] *** Pippin is a hobbit.
[00000000] *** Sam is a hobbit.
\end{urbiscript}

\subsubsection{for $n$-times}
\label{sec:lang:for:n}

\us provides some support for simple replication of computations: it
allow to repeat a loop body $n$-times.  With the exception that the
loop index is not available within the body, \lstinline|for (n)| is
equivalent to \lstinline|for (var i: n)|.  It supports the same
flavors: \lstinline|for;|, \lstinline{for|}, and \lstinline|for&|. The
loop evaluates to \lstinline|void|.

\begin{urbiassert}[firstnumber=last]
{ var res = []; for (3) { res << 1; res << 2 } ; res }
        == [1, 2, 1, 2, 1, 2];

{ var res = []; for|(3) { res << 1; res << 2 } ; res }
        == [1, 2, 1, 2, 1, 2];

{ var res = []; for&(3) { res << 1; res << 2 } ; res }
        == [1, 1, 1, 2, 2, 2];
\end{urbiassert}

Note that since these \lstinline|for| loops are merely anonymous
foreach-style loops, the argument needs not being an integer, any
iterable value can be used.

\begin{urbiassert}[firstnumber=last]
3 == { var r = 0; for ([1, 2, 3]) r += 1; r};
3 == { var r = 0; for ("123")     r += 1; r};
\end{urbiassert}


\subsection{if}

\us supports the usual \lstinline|if| constructs.

\begin{urbiunchecked}
if (\var{condition})
  \var{action};

if (\var{condition})
  \var{action}
else
  \var{otherwise};
\end{urbiunchecked}

If the \var{condition} evaluation is true, \var{action} is
evaluated. Otherwise, in the latter version, \var{otherwise} is
executed.  Contrary to \C/\Cxx, there \emph{must not} be a semicolon
after the \var{action}; it would end the
\lstinline|if|/\lstinline|else| construct prematurely.

\subsection{loop}

Endless loops can be created with \lstinline|loop|, which is
equivalent to \lstinline|while (true)|.  The loop evaluates to
\lstinline|void|.  Both sequential flavors, \lstinline|loop;| and
\lstinline'loop;', are supported.  The default flavor is
\lstinline|loop;|.

\begin{urbiassert}[firstnumber=last]
{
  var n = 10|;
  var res = []|;
  loop;
  {
    n--;
    res << n;
    if (n == 0)
      break
  };
  res
}
==
[9, 8, 7, 6, 5, 4, 3, 2, 1, 0];
\end{urbiassert}

\begin{urbiassert}[firstnumber=last]
{
  var n = 10|;
  var res = []|;
  loop|
  {
    n--;
    res << n;
    if (n == 0)
      break
  };
  res
}
==
[9, 8, 7, 6, 5, 4, 3, 2, 1, 0];
\end{urbiassert}

\subsection{switch}

The \lstinline|switch| statement in \us is similar to \C's one.

\begin{urbiunchecked}
switch (\var{value})
{
  case \var{value_one}:
    \var{action_one};
  case \var{value_two}:
    \var{action_two};
//case ...:
//  ...
  default:
    \var{default_action};
};
\end{urbiunchecked}

It might contain an arbitrary number of cases, and optionally a
default case. The \var{value} is evaluated first, and then the
result is compared sequentially with the evaluation of all cases
values, with the \lstinline|==| operator, until one comparison is
true. If such a match is found, the corresponding action is executed,
and execution jumps after the \lstinline|switch|. Otherwise, the
default case --- if any --- is executed, and execution jumps after the
switch. The switch itself evaluates to case that was evaluated, or to
void if no match was found and there's no default case. The listing below
illustrates \lstinline|switch| usage.

Unlike \C, there are no \lstinline|break| to end \lstinline|case|
clauses: execution will never span over several cases.  Since the
comparisons are performed with the generic \lstinline|==| operator,
\lstinline|switch| can be performed on any comparable data type.

\begin{urbiscript}[firstnumber=last]
function sw(v)
{
  switch (v)
  {
    case "":
      echo("Empty string");
    case "foo":
      "bar";
    default:
      v[0];
  }
}|;
sw("");
[00000000] *** Empty string
sw("foo");
[00000000] "bar"
sw("foobar");
[00000000] "f"
\end{urbiscript}
% $ Pacify emacs math mode.

\subsection{while}

The \lstinline|while| loop is similar to \C's one.

\begin{urbiunchecked}
while (\var{condition})
  \var{body};
\end{urbiunchecked}

If \var{condition} evaluation, is true, \var{body} is evaluated and
execution jumps before the \lstinline|while|, otherwise execution
jumps after the \lstinline|while|.

\begin{urbiscript}[firstnumber=last]
var j = 3|
while (0 < j)
{
  echo(j);
  j--;
};
[00000000] *** 3
[00000000] *** 2
[00000000] *** 1
\end{urbiscript}

The default flavor for \lstinline|while| is \lstinline|while;|.

\subsubsection{while;}

The semantics of

\begin{urbiunchecked}
while; (\var{condition})
  \var{body};
\end{urbiunchecked}

\noindent
is the same as

\begin{urbiunchecked}
\var{condition} | \var{body} ; \var{condition} | \var{body} ; ...
\end{urbiunchecked}

\noindent
as long as \var{cond} evaluates to true, or until \lstinline|break| is
invoked.  If \lstinline|continue| is evaluated, the rest of the body
is skipped, and the next iteration is started.

\begin{urbiscript}[firstnumber=last]
{
  var i = 4|
  while (true)
  {
    i -= 1;
    echo ("in: " + i);
    if (i == 1)
      break
    else if (i == 2)
      continue;
    echo ("out: " + i);
  };
};
[00000000] *** in: 3
[00000000] *** out: 3
[00000000] *** in: 2
[00000000] *** in: 1
\end{urbiscript}


\subsubsection{while|}

The semantics of

\begin{urbiunchecked}
while| (\var{condition})
  \var{body};
\end{urbiunchecked}

\noindent
is the same as

\begin{urbiunchecked}
\var{condition} | \var{body} | \var{condition} | \var{body} | ...
\end{urbiunchecked}

The execution is can be controlled by \lstinline|break| and
\lstinline|continue|.

\begin{urbiscript}[firstnumber=last]
{
  var i = 4|
  while| (true)
  {
    i -= 1;
    echo ("in: " + i);
    if (i == 1)
      break
    else if (i == 2)
      continue;
    echo ("out: " + i);
  };
};
[00000000] *** in: 3
[00000000] *** out: 3
[00000000] *** in: 2
[00000000] *** in: 1
\end{urbiscript}


\section{Exceptions}
\label{sec:lang:except}
\subsection{Throwing exceptions}

Use the \lstinline|throw| keyword to \dfn[exception!throwing]{throw
  exceptions}, as shown below. Thrown exceptions will
break the execution upward until they are caught, or until they reach
the top-level --- as in \Cxx.  Contrary to \Cxx, exceptions reaching
the top-level are printed, and won't abort the kernel --- other and new
connections will continue to execute normally.

\begin{urbiscript}[firstnumber=last]
throw 42;
[00000000:error] !!! 42
function inner() { throw "exn" } |
function outer() { inner() }|
// Exceptions propagate to parent call up to the top-level
outer();
[00000000:error] !!! exn
[00000000:error] !!!    called from: 3.20-26: inner
[00000000:error] !!!    called from: 4.1-7: outer
\end{urbiscript}

\subsection{Catching exceptions}

Exceptions are \dfn[exception!catching]{caught} with the
\lstinline|try|/\lstinline|catch| construct. It consists of a first
block (the \dfn{try-block}), from which we want to catch exceptions,
and one or more catch clauses to stop the exception
(\dfn{catch-blocks}). Each catch clause defines a pattern against
which the thrown exception is matched. If no pattern is specified, the
catch clause matches systematically (equivalent to
\lstinline|catch (...)| in \Cxx).

Exceptions thrown from the \texttt{try} block are matched sequentially
against all catch clauses. The first matching clause is executed, and
control jumps after the whole try/catch block. If no catch clause
matches, the exceptions isn't stopped and continues
upward.

\begin{urbiscript}[firstnumber=last]
function test(e)
{
  try
  { throw e;  }
  catch (0)
  { echo("zero") }
  catch ([var x, var y])
  { echo(x + y) }
} | {};
test(0);
[00002126] *** zero
test([22, 20]);
[00002131] *** 42
test(51);
[00002143:error] !!! 51
[00002143:error] !!!    called from: 12.1-8: test
\end{urbiscript}

\subsection{Inspecting exceptions}

An \refObject{Exception} is a regular object, on which introspection
can be performed.

\begin{urbiscript}[firstnumber=last]
try
{
  Math.cos(3,1415);
}
catch (var e)
{
  echo ("Exception type: %s" % e.'$type');
  if (e.isA(Exception.Arity))
  {
    echo("Routine: %s" % e.routine);
    echo("Number of effective arguments: %s" % e.effective);
  };
};
[00000132] *** Exception type: Arity
[00000133] *** Routine: cos
[00000134] *** Number of effective arguments: 2
\end{urbiscript}

%% FIXME: \subsection{Exceptions and parallelism}

\section{Assertions}
\label{sec:assertions}

\dfn[assertion]{Assertions} allow to embed consistency checks in the
code.  They are particularly useful when developing a program since
they allow early catching of errors.  Yet, they can be costly in
production mode: the run-time cost of verifying every single assertion
might be prohibitive.  Therefore, as in \C-like languages, assertions
are disabled when \lstinline|System.ndebug| is true, see \refObject{System}.

\us supports assertions in two different ways: with a function-like
syntax, which is adequate for single claims, and a block-like syntax,
to group claims together.

The \lstinline|assert(\var{expression})| bounces to the
\lstinline|System.'assert'| function, see \refObject{System}.

\begin{urbiscript}[firstnumber=last]
assert(true);
assert(42);
assert(1 == 1 + 1);
[00000002:error] !!! failed assertion: 1 == 1 . '+'(1) (1 != 2)
\end{urbiscript}

Groups of assertions are more readable when used with the
\lstinline|assert{\var{exp1}; \var{exp2}; ...}| construct.  The
(possibly empty) list of claims may be ended with a semicolon.

\begin{urbiscript}[firstnumber=last]
assert
{
  true;
  42;
  1 == 1 + 1;
};
[00000002:error] !!! failed assertion: 1 . '=='(1 . '+'(1))
\end{urbiscript}

For sake of readability and compactness, this documentation shows
assertion blocks as follows.

\begin{urbiassert}[firstnumber=last]
true;
42;
1 == 1 + 1;
[00000002:error] !!! failed assertion: 1 . '=='(1 . '+'(1))
\end{urbiassert}


\section{Parallel and event-based flow control}

\subsection{\lstinline'at'}
\label{sec:lang:at}
Using the \lstindex{at} construct, one can arm code that will be
triggered each time some condition is true.

The \lstinline'at' construct is as follows:

\begin{urbiunchecked}
at (\var{condition})
  \var{statement1}
onleave
  \var{statement2}
\end{urbiunchecked}

The \var{condition} can be of two different kinds:
\lstinline|\var{e}?(\var{args})| to catch when events are sent, or
\lstinline|\var{exp}| to catch each time a Boolean \var{exp} becomes
true.

The \lstinline|onleave \var{statement2}| part is optional.  Note that,
as is the case for the \lstinline|if| statement, there must not be a
semicolon after \var{statement1} if there is an \lstinline|onleave|
clause.

\subsubsection{\lstinline'at' on Events}
%% FIXME: More details are needed.  Don't bounce elsewhere.
See \autoref{sec:tut:events} for an example of using \lstinline|at|
statements to watch events.

\subsubsection{\lstinline'at' on Boolean Expressions}

The \lstinline|at| construct can be used to watch a given Boolean
expression.

\begin{urbiscript}[firstnumber=1]
var x = 0 |
var x_is_two = false |
at (x == 2)
  x_is_two = true
onleave
  x_is_two = false;

x = 3|;  assert(!x_is_two);
x = 2|;  assert( x_is_two);
x = 2|;  assert( x_is_two);
x = 3|;  assert(!x_is_two);
\end{urbiscript}

It can also wait for some condition to hold long enough:
\lstinline|\var{exp} ~ \var{duration}|, as a condition, denotes the
fact that \var{exp} was true for \var{duration} seconds.

\begin{urbiscript}[firstnumber=1]
var x = 0 |
var x_was_two_for_two_seconds = false |
at (x == 2 ~ 2s)
  x_was_two_for_two_seconds = true
onleave
  x_was_two_for_two_seconds = false;

x = 2       | assert(!x_was_two_for_two_seconds);
sleep(1.5s) | assert(!x_was_two_for_two_seconds);
sleep(1.5s) | assert( x_was_two_for_two_seconds);

x = 3|; sleep(0.1s);  assert(!x_was_two_for_two_seconds);

x = 2       | assert(!x_was_two_for_two_seconds);
sleep(1.5s) | assert(!x_was_two_for_two_seconds);
x = 3|; x = 2|; sleep (1s) | assert(!x_was_two_for_two_seconds);
\end{urbiscript}


\subsubsection{Scoping at \lstinline'at'}

\lstinline'at' statements are not scoped.  But, using a
\refObject{Tag} object, one can control them.  In the following
example, \lstinline|scopeTag| is used to label the \lstinline|at|
statement.  When the function ends, the \lstinline|at| is no longer
active.

\begin{urbiscript}[firstnumber=1]
var x = 0 |
var x_is_two = false |;

{
  scopeTag:
    at (x == 2)
      x_is_two = true
    onleave
      x_is_two = false;
  sleep(2s);
},
x = 2 |; assert(x_is_two);
x = 1 |; assert(!x_is_two);
sleep(3s);
x = 2 | assert(!x_is_two);
\end{urbiscript}

\subsection{\lstinline'every'}

The \lstindex{every} statement enables to execute a block of code
repeatedly, with the given period.

\begin{urbiscript}[firstnumber=last]
// Print out a message every second.
timeout (2.1s)
  every (1s)
    echo("Are you still there?");
[00000000] *** Are you still there?
[00001000] *** Are you still there?
[00002000] *** Are you still there?
\end{urbiscript}

It exists in several flavors.

\subsubsection{\lstinline'every|'}
The whole \lstinline'every|' statement itself remains in foreground:
statements attached after it with \lstinline';' or \lstinline'|' will
not be reached unless you \lstinline'break' out of it.  You may use
\lstinline|continue| to finish one iteration.  In that case, the
following iteration is not immediately started, it will be launched as
expected, at the given period.

% We used to use 100ms instead of 1s, but severely loaded machines
% (the Mac Mini) fail way too often.
\begin{urbiscript}[firstnumber=last]
{
  var count = 4;
  var start = time;
  echo("before");
  every| (1s)
  {
    count -= 1;
    echo("begin: %s @ %1.0fs" % [count, time - start]);
    if (count == 2)
      continue;
    if (count == 0)
      break;
    echo("end:   " + count);
  };
  echo("after");
};
[00000597] *** before
[00000598] *** begin: 3 @ 0s
[00000599] *** end:   3
[00000698] *** begin: 2 @ 1s
[00000798] *** begin: 1 @ 2s
[00000799] *** end:   1
[00000898] *** begin: 0 @ 3s
[00000899] *** after
\end{urbiscript}

The \lstindex{every|} flavor does not let iterations overlap. If an
iteration takes too long, the following iterations are delayed. That
is, the next iterations will start immediately after the end of the
current one, and next iterations will occur normally from this point.

\begin{urbiscript}[firstnumber=last]
{
  var too_long = true|;

  var count = 5;
  // Every other iteration exceeds the period, and will delay the
  // following one.
  every| (1s)
  {
    if (! count -=1)
      break;

    if (too_long)
    {
      too_long = false;
      echo("Long in");
      sleep(1.5s);
      echo("Long out");
    }
    else
    {
      too_long = true;
      echo("Short");
    };
  };
};
[00000000] *** Long in
[00001500] *** Long out
[00001500] *** Short
[00002500] *** Long in
[00004000] *** Long out
[00004000] *** Short
\end{urbiscript}

The flow-control constructs \lstinline|break| and \lstinline|continue|
are supported.

\begin{urbiscript}[firstnumber=last]
{
  var count = 0;
  every| (250ms)
  {
    count += 1;
    if (count == 2)
      continue;
    if (count == 4)
      break;
    echo(count);
  }
};
/*(*/sleep(2s);/*)*/
[00000000] *** 1
[00001500] *** 3
\end{urbiscript}


\subsubsection{\lstinline'every,'}
The default flavor, \lstinline|every,| launches the execution of the
block in the background every given period. Iterations may overlap.

% Cut the previous every, no [firstnumber=last]
\begin{urbiscript}[firstnumber=1]
// If an iteration is longer than the given period, it will overlap
// with the next one.
timeout (2.8s)
  every (1s)
  {
    echo("In");
    sleep(1.5s);
    echo("Out");
  };
[00000000] *** In
[00001000] *** In
[00001500] *** Out
[00002000] *** In
[00002500] *** Out
\end{urbiscript}

\subsection{for}

The \lstinline|for| loops come into several flavors, depending one the
actual kind of \lstinline|for| loop.

\subsubsection{C-for,}
\begin{note}
  This feature is experimental.  It might be changed, or even removed.
  Feedback on its use would be appreciated.
\end{note}

\lstinline|for,| is syntactic sugar for \lstinline|while,|, see
\autoref{sec:lang:while:comma}.

\begin{urbiscript}[firstnumber=last]
for, (var i = 3; 0 < i; i -= 1)
{
  var j = i |
  echo ("in: i = %s, j = %s" % [i, j]);
  sleep(j/10);
  echo ("out: i = %s, j = %s" % [i, j]);
};
echo ("done");
[00000144] *** in: i = 3, j = 3
[00000145] *** in: i = 2, j = 2
[00000145] *** in: i = 1, j = 1
[00000246] *** out: i = 0, j = 1
[00000346] *** out: i = 0, j = 2
[00000445] *** out: i = 0, j = 3
[00000446] *** done
\end{urbiscript}

\begin{urbiscript}[firstnumber=last]
for, (var i = 9; 0 < i; i -= 1)
{
  var j = i;
  if (j % 2)
    continue
  else if (j == 4)
    break
  else
    echo("%s: done" % j)
};
echo("done");
[00000146] *** 8: done
[00000148] *** 6: done
[00000150] *** done
\end{urbiscript}


\subsubsection{range-for\& (:)}
\label{sec:lang:for:each:and}

One can iterate concurrently over the members of a collection.

\begin{urbiscript}[firstnumber=last]
for& (var i: [0, 1, 2])
{
  echo (i * i);
  echo (i * i);
};
[00000000] *** 0
[00000000] *** 1
[00000000] *** 4
[00000000] *** 0
[00000000] *** 1
[00000000] *** 4
\end{urbiscript}

If an iteration executes \lstinline|continue|, it is stopped; the
other iterations are not affected.

\begin{urbiscript}[firstnumber=last]
for& (var i: [0, 1, 2])
{
  var j = i;
  if (j == 1)
    continue;
  echo (j);
};
[00020653] *** 0
[00021054] *** 2
\end{urbiscript}

%%% FIXME:
%%% If an iteration executes \lstinline|break|, all the iterations
%%% including this one, are stopped.
%%%
%%% \begin{urbiscript}[firstnumber=last]
%%% for& (var i: [0, 1, 2])
%%% {
%%%   var j = i;
%%%   sleep(1s);
%%%   if (j == 1)
%%%    { echo ("break");
%%%     break;};
%%%   echo (j);
%%% };
%%% \end{urbiscript}

\subsubsection{for\& (n)}

Since \lstinline|for& (\var{n}) \var{body}| is processed as
\lstinline|for& (var \var{tmp}: \var{n}) \var{body}|, which \var{tmp}
a hidden variable, see \autoref{sec:lang:for:each:and} for details.


\subsection{loop,}
\begin{note}
  This feature is experimental.  It might be changed, or even removed.
  Feedback on its use would be appreciated.
\end{note}

This is syntactic sugar for \lstinline|while,(true)|.  In the
following example, care must be taken that concurrent executions don't
modify \lstinline|n| simultaneously.  This would happen had
\lstinline|;| been used instead of \lstinline'|'.

\begin{urbiassert}[firstnumber=last]
{
  var n = 10|;
  var res = []|;
  loop,
  {
    n-- |
    res << n |
    if (n == 0)
      break
  };
  res.sort
}
==
[0, 1, 2, 3, 4, 5, 6, 7, 8, 9];
\end{urbiassert}

\subsection{\lstinline|waituntil|}
\label{sec:lang:waituntil}

The \lstinline|waituntil| construct is used to hold the execution
until some condition is verified.  Similarly to \lstinline|at|
(\autoref{sec:lang:at}) and the other event-based constructs,
\lstinline|waituntil| may work on events, or on Boolean expressions.

\subsubsection{\lstinline'waituntil' on Events}

When the execution flow enters a \lstinline|waituntil|, the execution
flow is held until the event is fired.  Once caught, the event is
consumed, another \lstinline|waituntil| will require another event
emission.

\begin{urbiscript}[firstnumber=last]
{
  var e = Event.new;
  {
    waituntil (e?);
    echo ("caught e");
  },
  e!;
[00021054] *** caught e
  e!;
};
\end{urbiscript}

In the case of lasting events (see \lstinline|Event.trigger|), the
condition remains verified as long as the event is ``on''.

\begin{urbiscript}[firstnumber=last]
{
  var e = Event.new;
  e.trigger;
  {
    waituntil (e?);
    echo ("caught e");
  };
[00021054] *** caught e
  {
    waituntil (e?);
    echo ("caught e");
  };
[00021054] *** caught e
  {
    waituntil (e?);
    echo ("caught e");
  };
[00021054] *** caught e
};
\end{urbiscript}

The event specification may use pattern-matching to specify the
accepted events.

\begin{urbiscript}[firstnumber=last]
{
  var e = Event.new;
  {
    waituntil (e?(1, var b));
    echo ("caught e(1, %s)" % b);
  },
  e!;
  e!(1);
  e!(2, 2);
  e!(1, 2);
[00021054] *** caught e(1, 2)
  e!(1, 2);
};
\end{urbiscript}

Events sent before do not release the construct.

\begin{urbiscript}[firstnumber=last]
{
  var e = Event.new;
  e!;
  {
    waituntil (e?);
    echo ("caught e");
  },
  e!;
[00021054] *** caught e
};
\end{urbiscript}

\subsubsection{\lstinline'waituntil' on Boolean Expressions}

You may use any expression that evaluates to a truth value as argument
to \lstinline'waituntil'.

\begin{urbiscript}[firstnumber=last]
{
  var foo = Object.new;
  {
    waituntil (foo.hasLocalSlot("bar"));
    echo(foo.getLocalSlot("bar"));
  },
  var foo.bar = 123|;
};
[00021054] *** 123
\end{urbiscript}

\subsection{\lstinline|whenever|}

The \lstinline|whenever| construct really behaves like a never-ending
\lstinline|loop if| construct.  It also works on events and Boolean
expressions, and triggers each time the condition \emph{becomes}
verified.

\begin{urbiunchecked}
whenever (\var{condition})
  \var{statement1}
\end{urbiunchecked}

It supports an optional \lstinline|else| clause, which is run whenever
the condition changes ``from true to false''.

\begin{urbiunchecked}
whenever (\var{condition})
  \var{statement1}
else
  \var{statement2}
\end{urbiunchecked}

The execution of a \lstinline|whenever| clause is ``instantaneous'',
there is no mean to use \samp{,} to put it in background.  It is also
asynchronous with respect to the condition: the emission of an event
is not held until all its watchers have completed their job.

\subsubsection{\lstinline'whenever' on Events}

A \lstinline'whenever' clause can be used to catch events with or
without payloads.

\begin{urbiscript}[firstnumber=1]
var e = Event.new|;
whenever (e?)
  echo("e on")
else
  echo("e off");
e!;
[00021054] *** e on
[00021054] *** e off
e!(1) & e!(2);
[00021054] *** e on
[00021054] *** e off
[00021054] *** e on
[00021054] *** e off
\end{urbiscript}

The payload is available in both bodies.  The pattern-matching and
guard on the payload is available.

\begin{urbiscript}[firstnumber=1]
var e = Event.new|;
whenever (e?("arg", var arg) if arg % 2)
  echo("e (%s) on" % arg)
else
  echo("e (%s) off" % arg);
e!("param", 23);
e!("arg", 52);
e!("arg", 23);
[00000002] *** e (23) on
[00000003] *** e (23) off
e!("arg", 52);
e!("arg", 17);
[00000004] *** e (17) on
[00000005] *** e (17) off
\end{urbiscript}


If the body of the \lstinline|whenever| lasts for a long time, it is
possible that two executions be run concurrently.

\begin{urbiscript}[firstnumber=1]
var e = Event.new|;
whenever (e?(var d))
  {
    echo("e (%s) on begin" % d);
    sleep(d);
    echo("e (%s) on end" % d);
  }
else
  {
    echo("e (%s) off begin" % d);
    sleep(d);
    echo("e (%s) off end" % d);
  };

e!(0.3s) & e!(1s);
sleep(3s);
[00000202] *** e (0.3) on begin
[00000202] *** e (1) on begin
[00000508] *** e (0.3) on end
[00000508] *** e (0.3) off begin
[00000810] *** e (0.3) off end
[00001208] *** e (1) on end
[00001208] *** e (1) off begin
[00002210] *** e (1) off end
\end{urbiscript}

\subsubsection{\lstinline'whenever' on Boolean Expressions}

A \lstinline'whenever' construct will repeatedly evaluate its body as
long as its condition holds.  The number of evaluation of the bodies
is typically non-deterministic, as not only does it depend on how
long the condition holds, but also ``how fast'' the \urbi kernel runs.

\begin{urbiscript}[firstnumber=1]
var x = 0|;
var count = 0|;
var t = Tag.new|;
t:
  whenever (x % 2)
  {
    if (!count)
      echo("x is now odd (%s)" % x);
    count++;
  }
  else
  {
    if (!count)
      echo("x is now even (%s)" % x);
    count++;
  };

t:
  whenever (100 < count)
  {
    count = 0 |
    x++;
  };
waituntil(x == 4);
[00000769] *** x is now even (0)
[00000809] *** x is now odd (1)
[00000846] *** x is now even (2)
[00000886] *** x is now odd (3)
[00000924] *** x is now even (4)
t.stop;
\end{urbiscript}


\subsection{While}
\subsubsection{\lstinline|while,|}
\label{sec:lang:while:comma}

\begin{note}
  This feature is experimental.  It might be changed, or even removed.
  Feedback on its use would be appreciated.
\end{note}

This construct provides a means to run concurrently multiple instances
of statements.  The semantics of

\begin{urbiunchecked}
while, (\var{condition})
  \var{body};
\end{urbiunchecked}

\noindent
is the same as

\begin{urbiunchecked}
\var{condition} | \var{body} , \var{condition} | \var{body} , ...
\end{urbiunchecked}

Attention must be paid to the fact that the (concurrent) iterations
share a common access to the environment, therefore if, for instance,
you want to keep the value of some index variable, use a local
variable inside the loop body:

% Cut the previous every, no [firstnumber=last]
\begin{urbiscript}[firstnumber=1]
{
  var i = 4|
  while, (i)
  {
    var j = i -= 1;
    echo ("in: i = %s, j = %s" % [i, j]);
    sleep(j/10);
    echo ("out: i = %s, j = %s" % [i, j]);
  }|
  echo ("done");
}|
[00000144] *** in: i = 2, j = 3
[00000145] *** in: i = 1, j = 2
[00000145] *** in: i = 0, j = 1
[00000146] *** in: i = 0, j = 0
[00000146] *** out: i = 0, j = 0
[00000246] *** out: i = 0, j = 1
[00000346] *** out: i = 0, j = 2
[00000445] *** out: i = 0, j = 3
[00000446] *** done
\end{urbiscript}

As for the other flavors, \lstinline|continue| skips the current
iteration, and \lstinline|break| ends the loop.  Note that
\lstinline|break| stops all the running iterations.  This semantics is
likely to be changed to ``\lstinline|break| ends the current iteration
and stops the generation of others, but lets the other concurrent
iterations finish'', so do not rely on this feature.

Control flow is passed to the following statement when all the
iterations are done.

\begin{urbiscript}[firstnumber=last]
{
  var i = 10|
  while, (i)
  {
    var j = i -= 1;
    if (j % 2)
      continue
    else if (j == 4)
      break
    else
      echo("%s: done" % j)
  }|
  echo("done");
};
[00000146] *** 8: done
[00000148] *** 6: done
[00000150] *** done
\end{urbiscript}


%% FIXME: \section{Pattern matching}

\section{Trajectories}
\label{sec:lang:traj}

In robotics, \dfn[trajectory]{trajectories} are often used: they are a
means to change the value of a variable (actually, a slot) over time.
This can be done using detached executions, for instance using a
combination of \lstinline|every| and \lstinline|detach|, but \us
provides syntactic sugar to this end.

For instance the following drawing shows how the \lstinline|y|
variable is moved smoothly from its \dfn{initial value}
(\lstinline|0|) to its \dfn{target value} (\lstinline|100|) in 3
seconds (the value given to the \lstinline|smooth| \dfn{attribute}.

\urbitrajectory{smooth}

Trajectories can be frozen and unfrozen, using tags
(\autoref{sec:tut:tags}).  In that case, ``time is suspended'', and
the trajectory resumes as if the trajectory was never interrupted.

\urbitrajectory{smooth-frozen}

When the target value is reached, the trajectory generator is detached
from the variables: changes to the value of the variable no longer
trigger the trajectory generator.

\urbitrajectory{smooth-continued}

See the specifications of \refObject{TrajectoryGenerator} for the list
of supported trajectories.

%%% Local Variables:
%%% mode: latex
%%% coding: utf-8
%%% TeX-master: "urbi-specs"
%%% ispell-personal-dictionary: "../urbi.dict"
%%% End:


\FloatBarrier
\chapter{\us Standard Library}
\label{sec:stdlib}

%% Redefine \section is this chapter so that we don't have to
%% call \labelObject each time.  See the bottom of this file for the
%% restoring of \section.
%%
%% The \index cannot be put inside the \section, it breaks tex4ht when
%% splitting at that boundary.
\let\sectionOrig\section
\renewcommand{\section}[1]
{%
  \sectionOrig{\labelObject{#1}#1}%
  \index{#1@\lstinline{#1}}%
}

%% Copyright (C) 2010, Gostai S.A.S.
%%
%% This software is provided "as is" without warranty of any kind,
%% either expressed or implied, including but not limited to the
%% implied warranties of fitness for a particular purpose.
%%
%% See the LICENSE file for more information.

\section{Barrier}

\lstinline|Barrier| is used to wait until another job raises a signal.
This can be used to implement blocking calls waiting until a resource
is available.

\subsection{Prototypes}

\begin{refObjects}
\item[Object]
\end{refObjects}

\subsection{Construction}

A \lstinline|Barrier| can be created with no argument.  Signals and wait
calls done on this instance are restricted to this instance.

\begin{urbiscript}[firstnumber=1]
Barrier.new;
[00000000] Barrier_0x25d2280
\end{urbiscript}

\subsection{Slots}

\begin{urbiscriptapi}

\item[signal](<payload>)%
  Wake up one of the job waiting for a signal.  The \var{payload} is sent to
  the \var{wait} method.  This method returns the number of job woken up.

\begin{urbiscript}
do (Barrier.new)
{
  echo(wait) &
  echo(wait) &
  assert
  {
    signal(1) == 1;
    signal(2) == 1;
  }
}|;
[00000000] *** 1
[00000000] *** 2
\end{urbiscript}


\item[signalAll](<payload>)%
  Wake up all the jobs waiting for a signal.  The \var{payload} is
  sent to all \var{wait} methods.  Return the number of jobs woken up.

\begin{urbiscript}
do (Barrier.new)
{
  echo(wait) &
  echo(wait) &
  assert
  {
    signalAll(1) == 2;
    signalAll(2) == 0;
  }
}|;
[00000000] *** 1
[00000000] *** 1
\end{urbiscript}


\item[wait]
  Block until a signal is received.  The \var{payload} sent with the signal
  function is returned by the \lstinline|wait| method.

\begin{urbiscript}
do (Barrier.new)
{
  echo(wait) &
  signal(1)
}|;
[00000000] *** 1
\end{urbiscript}

\end{urbiscriptapi}

%%% Local Variables:
%%% mode: latex
%%% TeX-master: "../urbi-sdk"
%%% ispell-dictionary: "american"
%%% ispell-personal-dictionary: "../urbi.dict"
%%% fill-column: 76
%%% End:

%% Copyright (C) 2009-2010, Gostai S.A.S.
%%
%% This software is provided "as is" without warranty of any kind,
%% either expressed or implied, including but not limited to the
%% implied warranties of fitness for a particular purpose.
%%
%% See the LICENSE file for more information.

\section{Binary}

A Binary object, sometimes called a \dfn{blob}, is raw memory,
decorated with a user defined header.

\subsection{Prototypes}
\begin{refObjects}
\item[Object]
\end{refObjects}

\subsection{Construction}

Binaries are usually not made by users, but they are heavily used by
the internal machinery when exchanging Binary UValues.  A binary
features some \lstinline|content| and some \lstinline|keywords|, both
simple \refObject[String]{Strings}.

\begin{urbiscript}[firstnumber=1]
Binary.new("my header", "my content");
[00000001] BIN 10 my header
[:]my content
\end{urbiscript}

Beware that the third line above (\samp{my content}), was output by
the system, although not preceded by a timestamp.

\subsection{Slots}

\begin{urbiscriptapi}
\item['+'](<that>)%
  Return a new Binary whose keywords are those of \this if
  not empty, otherwise those of \var{that}, and whose data is the
  concatenation of both.
\begin{urbiassert}
Binary.new("0", "0") + Binary.new("1", "1")
       == Binary.new("0", "01");
Binary.new("", "0") + Binary.new("1", "1")
       == Binary.new("1", "01");
\end{urbiassert}

\item['=='](<other>)%
  Whether \lstinline|keywords| and \lstinline|data| are equal.
\begin{urbiassert}
Binary.new("0", "0") == Binary.new("0", "0");
Binary.new("0", "0") != Binary.new("0", "1");
Binary.new("0", "0") != Binary.new("1", "0");
\end{urbiassert}

\item[asString]
  Display using the syntactic rules of the UObject/UValue protocol.
  Incoming binaries must use a semicolon to separate the header part
  from the content, while outgoing binaries use a carriage-return.
\begin{urbiscript}
assert(Binary.new("head", "content").asString
       == "BIN 7 head\ncontent");
var b = BIN 7 header;content;
[00000002] BIN 7 header
[:]content
assert(b == Binary.new("header", "content"));
\end{urbiscript}

This syntax (\lstinline|BIN \var{size} \var{header}; \var{content}|)
is \emph{partially} supported in \us, but it is strongly discouraged.
Rather, use the \lstinline|\B(\var{size})(\var{data})| special escape
(see \autoref{sec:lang:string}):

\begin{urbiassert}
Binary.new("head", "\B(7)(content)").asString
       == "BIN 7 head\ncontent";
\end{urbiassert}


\item[data]
  The data carried by the Binary.
\begin{urbiassert}
Binary.new("head", "content").data == "content";
\end{urbiassert}

\item[empty]
  Whether the data is empty.
\begin{urbiassert}
Binary.new("head", "").empty;
!Binary.new("head", "content").empty;
\end{urbiassert}

\item[keywords]
  The headers carried by the Binary.
\begin{urbiassert}
Binary.new("head", "content").keywords == "head";
\end{urbiassert}
\end{urbiscriptapi}


%%% Local Variables:
%%% mode: latex
%%% TeX-master: "../urbi-sdk"
%%% ispell-dictionary: "american"
%%% ispell-personal-dictionary: "../urbi.dict"
%%% fill-column: 76
%%% End:

\section{Boolean}

Booleans does not exist as a real type in \us. On the other hand, all
objects in \us can be evaluated as Boolean value especially the two
specific object \lstinline|true| and \lstinline|false|.

\subsection{Prototypes}

As above, the prototype Boolean does not exists, but \lstinline|true|
and \lstinline|false| have the following prototype.

\begin{itemize}
\item \refObject{Singleton}
\end{itemize}

\subsection{Construction}

Actually you don't want to construct booleans, you just want to know
if a condition is true or false. So the object \lstinline|true| and
\lstinline|false| are the result of all comparison statement.

\begin{urbiscript}
true;
[00000000] true
false;
[00000000] false
2 < 6;
[00000000] true
var x = true.new;
[00000000] true
x === true;
[00000000] true
\end{urbiscript}

As you can see, when you try to clone \lstinline|true| or
\lstinline|false|, you get the object itself and not a copy.

%%% Local Variables:
%%% mode: latex
%%% TeX-master: "../urbi-sdk"
%%% End:

\section{CallMessage}
Capturing a method invocation: its target and arguments.

\subsection{Examples}
\subsubsection{Evaluating an argument several times}
\label{sec:std-callmsg-examples-several}

The following example implements a lazy function which takes an
integer \var{n}, then arguments.  The \var{n}-th argument is evaluated
twice using \lstinline|CallMessage.evalArgAt|.

\begin{urbiscript}
function callTwice
{
  var n = call.evalArgAt(0);
  call.evalArgAt(n);
  call.evalArgAt(n)
} |;

// Call twice echo("foo").
callTwice(1, echo("foo"), echo("bar"));
[00000001] *** foo
[00000002] *** foo

// Call twice echo("bar").
callTwice(2, echo("foo"), echo("bar"));
[00000003] *** bar
[00000004] *** bar
\end{urbiscript}


\subsubsection{Strict Functions}

Strict functions do support \lstinline|call|.

\begin{urbiscript}[firstnumber=last]
function strict(x)
{
  echo("Entering");
  echo("Strict: " + x);
  echo("Lazy:   " + call.evalArgAt(0));
} |;

strict({echo("1"); 1});
[00000011] *** 1
[00000013] *** Entering
[00000012] *** Strict: 1
[00000013] *** 1
[00000014] *** Lazy:   1
\end{urbiscript}


\subsection{Slots}

\begin{itemize}
\item \lstinline|args|\\
  The list of unevaluated arguments.
\begin{urbiscript}[firstnumber=last]
function args { call.args }|
assert
{
  args == [];
  args() == [];
  args({echo(111); 1}) == [Lazy.new(closure() {echo(111); 1})];
  args(1, 2) == [Lazy.new(closure () {1}),
                 Lazy.new(closure () {2})];
};
\end{urbiscript}


\item \lstinline|argsCount|\\
  Return the number of arguments.  Do not evaluate them.
\begin{urbiscript}[firstnumber=last]
function argsCount { call.argsCount }|;
assert
{
  argsCount == 0;
  argsCount() == 0;
  argsCount({echo(1); 1}) == 1;
  argsCount({echo(1); 1}, {echo(2); 2}) == 2;
};
\end{urbiscript}

\item \lstinline|code|\\
  The body of the called function as a \refObject{Code}.
\begin{urbiscript}[firstnumber=last]
function code { call.getSlot("code") }|
assert (code == getSlot("code"));
\end{urbiscript}

\item \lstinline|evalArgAt(\var{n})|\\
  Evaluate the \var{n}-th argument, and return its value.  \var{n}
  must evaluate to an nonnegative integer.  Repeated invocations
  repeat the evaluation, see
  \autoref{sec:std-callmsg-examples-several}.
\begin{urbiscript}[firstnumber=last]
function sumTwice
{
  var n = call.evalArgAt(0);
  call.evalArgAt(n) + call.evalArgAt(n)
}|;

function one () { echo("one"); 1 }|;

sumTwice(1, one, one + one);
[00000008] *** one
[00000009] *** one
[00000010] 2
sumTwice(2, one, one + one);
[00000011] *** one
[00000012] *** one
[00000011] *** one
[00000012] *** one
[00000013] 4

sumTwice(3, one, one);
[00000014:error] !!! evalArgAt: invalid index: 3
sumTwice(3.14, one, one);
[00000015:error] !!! evalArgAt: invalid index: 3.14
\end{urbiscript}

\item \lstinline|evalArgs|\\
  Call \lstinline|evalArgAt| for each argument, return the list of
  values.
\begin{urbiscript}[firstnumber=last]
function twice
{
  call.evalArgs + call.evalArgs
}|;
twice({echo(1); 1}, {echo(2); 2});
[00000011] *** 1
[00000012] *** 2
[00000011] *** 1
[00000012] *** 2
[00000013] [1, 2, 1, 2]
\end{urbiscript}

\item \lstinline|message|\\
  The name under which the function was called.
\begin{urbiscript}[firstnumber=last]
function myself { call.message }|
assert(myself == "myself");
\end{urbiscript}

\end{itemize}


%%% Local Variables:
%%% mode: latex
%%% TeX-master: "../urbi-sdk"
%%% End:

%% Copyright (C) 2009-2010, Gostai S.A.S.
%%
%% This software is provided "as is" without warranty of any kind,
%% either expressed or implied, including but not limited to the
%% implied warranties of fitness for a particular purpose.
%%
%% See the LICENSE file for more information.

\section{Channel}
Returning data, typically asynchronous, with a label so that the
``caller'' can find it in the flow.

\subsection{Prototypes}

\begin{itemize}
\item \refObject{Object}
\end{itemize}

\subsection{Construction}

Channels are created like any other object. The constructor must be
called with a string which will be the label.

\begin{urbiscript}[firstnumber=1]
var ch1 = Channel.new("my_label");
[00000201] Channel_0x7985810

ch1 << 1;
[00000201:my_label] 1

var ch2 = ch1;
[00000201] Channel_0x7985810

ch2 << 1/2;
[00000201:my_label] 0.5
\end{urbiscript}

\subsection{Slots}

\begin{urbiscriptapi}
\item \lstinline|'<<'(\var{value})|\\
  Send \var{value} to \lstinline|this| tagged by its label if non-empty.

\begin{urbiscript}
Channel.new("label") << 42;
[00000000:label] 42

Channel.new("") << 51;
[00000000] 51
\end{urbiscript}

\item[echo](<value>)%
  Same as \lstinline|System.echo(\var{value}, name)|.

\begin{urbiscript}
Channel.new("label").echo(42);
[00000000:label] *** 42

Channel.new("").echo("Foo");
[00000000] *** Foo
\end{urbiscript}

\item[enabled] Whether the Channel is enabled.  Disabled Channels
  produce no output.
\begin{urbiscript}
var c = Channel.new("")|;

c << "enabled";
[00000000] "enabled"

c.enabled = false|;
c << "disabled";

c.enabled = true|;
c << "enabled";
[00000000] "enabled"
\end{urbiscript}

\item[quote] Whether the strings are output escaped (the default)
  instead of raw strings.
\begin{urbiscript}
var d = Channel.new("")|;

assert(d.enabled);
d << "A \"String\"";
[00000000] "A \"String\""

d.quote = false|;
d << "A \"String\"";
[00000000] A "String"
\end{urbiscript}

\item[name] The name of the Channel, used to label the output.
\begin{urbiscript}
assert
{
  Channel.new("").name == "";
  Channel.new("foo").name == "foo";
};
\end{urbiscript}

\item[null] A predefined stream whose \lstinline|enabled| is
  \lstinline|false|.
\begin{urbiscript}
Channel.null << "Message";
\end{urbiscript}


\item[topLevel] A predefined stream for regular output.  Strings are
  output escaped.
\begin{urbiscript}
Channel.topLevel << "Message";
[00015895] "Message"
Channel.topLevel << "\"quote\"";
[00015895] "\"quote\""
\end{urbiscript}

\item[warning] A predefined stream for warning messages.  Strings sent
  to it are not escaped.
\begin{urbiscript}
Channel.warning << "Message";
[00015895:warning] Message
Channel.warning << "\"quote\"";
[00015895:warning] "quote"
\end{urbiscript}
\end{urbiscriptapi}

%%% Local Variables:
%%% mode: latex
%%% TeX-master: "../urbi-sdk"
%%% ispell-dictionary: "american"
%%% ispell-personal-dictionary: "../urbi.dict"
%%% fill-column: 76
%%% End:

%% Copyright (C) 2009-2010, Gostai S.A.S.
%%
%% This software is provided "as is" without warranty of any kind,
%% either expressed or implied, including but not limited to the
%% implied warranties of fitness for a particular purpose.
%%
%% See the LICENSE file for more information.

\section{Code}

Functions written in \us.

\subsection{Prototypes}

\begin{refObjects}
\item[Comparable]
\item[Executable]
\end{refObjects}

\subsection{Construction}

The keywords \lstinline|function| and \lstinline|closure| build Code
instances.

\begin{urbiassert}
function(){}.protos[0] === getSlot("Code");
closure(){}.protos[0] === getSlot("Code");
\end{urbiassert}

\subsection{Slots}

\begin{urbiscriptapi}
\item \lstinline|==(\var{that})|\\
  Whether \lstinline|this| and \var{that} are the same source code.
  It actually checks that both have the same \lstinline|asString|.
\begin{urbiassert}
function () { 1 } == function () { 1 };
function () { 1 } != closure  () { 1 };
closure  () { 1 } != function () { 1 };

function () { 1 + 1 } == function () { 1 + 1 };
function () { 1 + 2 } != function () { 2 + 1 };

function () { 1 } != function { 1 };
function () { 1 } != function (ignored) { 1 };
\end{urbiassert}

\item \lstinline|apply(\var{args})|\\
  Invoke the routine, with all the arguments.  The target,
  \lstinline|this|, will be set to \lstinline|\var{args}[0]| and the
  remaining arguments with be given as arguments.
\begin{urbiassert}
function (x, y) { x+y }.apply([nil, 10, 20]) == 30;
function () { this }.apply([123]) == 123;

// There is Object.apply.
1.apply([this]) == 1;
\end{urbiassert}
\begin{urbiscript}
function () {}.apply([]);
[00000001:error] !!! apply: list of arguments must begin with `this'

function () {}.apply([1, 2]);
[00000002:error] !!! apply: expected 0 argument, given 1
\end{urbiscript}

\item[asString]
  Conversion to \refObject{String}.
\begin{urbiassert}
closure  () { 1 }.asString == "closure () {\n  1\n}";
function () { 1 }.asString == "function () {\n  1\n}";
\end{urbiassert}

\item[bodyString]
  Conversion to \refObject{String} of the routine body.
\begin{urbiassert}
closure  () { 1 }.bodyString == "1";
function () { 1 }.bodyString == "1";
\end{urbiassert}

\end{urbiscriptapi}

%%% Local Variables:
%%% mode: latex
%%% TeX-master: "../urbi-sdk"
%%% ispell-dictionary: "american"
%%% ispell-personal-dictionary: "../urbi.dict"
%%% fill-column: 76
%%% End:

%% Copyright (C) 2009-2010, Gostai S.A.S.
%%
%% This software is provided "as is" without warranty of any kind,
%% either expressed or implied, including but not limited to the
%% implied warranties of fitness for a particular purpose.
%%
%% See the LICENSE file for more information.

\section{Comparable}
Objects that can be compared for equality and inequality.  See also
\refObject{Orderable}.

This object, made to serve as prototype, provides a definition of
\lstinline{!=} based on \lstinline{==}.  \lstinline{Object} provides
a default implementation of \lstinline{==} that bounces on the physical
equality \lstinline{===}.

\begin{urbiscript}[firstnumber=1]
class Foo : Comparable
{
  var value = 0;
  function init (v) { value = v; };
  function '==' (lhs) { value == lhs.value; };
};
[00000000] Foo
Foo.new(1) == Foo.new(1);
[00000000] true
Foo.new(1) == Foo.new(2);
[00000000] false
\end{urbiscript}

\subsection{Slots}

\begin{urbiscriptapi}
\item[==](<that>)
  Whether \lstinline|! (this != that)|.
\begin{urbiscript}
class FortyTwo : Comparable
{
  function '!=' (that) { 42 != that };
}|;
assert
{
  FortyTwo != 51;
  FortyTwo == 42;
};
\end{urbiscript}


\item[!=](<that>)
  Whether \lstinline|! (this == that)|.

\begin{urbiscript}
class FiftyOne : Comparable
{
  function '==' (that) { 51 == that };
}|;
assert
{
  FiftyOne == 51;
  FiftyOne != 42;
};
\end{urbiscript}
\end{urbiscriptapi}

%%% Local Variables:
%%% mode: latex
%%% TeX-master: "../urbi-sdk"
%%% ispell-dictionary: "american"
%%% ispell-personal-dictionary: "../urbi.dict"
%%% fill-column: 76
%%% End:

\section{Control}

\lstinline|Control| is designed as a namespace for control sequences.
Some of these entities are used by the \urbi engine to execute some
\us features; in other words, users are not expected to you use it,
much less change it.

\subsection{Prototypes}

\begin{refObjects}
\item[Object]
\end{refObjects}

\subsection{Slots}

\begin{urbiscriptapi}

\item \lstinline|detach(\var{exp})|
  Detach the evaluation of the expression \var{exp} from the current
  evaluation.  The \var{exp} is evaluated in parallel to the current code
  and keep the current tag which are attached to it.

\begin{urbiscript}
for (var i : [0, 1, 2])
{
  detach({
    echo(i);
    echo(i)
  }) |
  if (i == 2)
    break
};
[00000000] *** 0
[00000000] *** 1
[00000000] *** 0
\end{urbiscript}

\item \lstinline|disown(\var{exp})|%
  Same as \lstinline|detach(\var{exp})| except that tags used to tag
  the \lstinline|disown| call are not inherited inside the expression.

\begin{urbiscript}
for (var i : [0, 1, 2])
{
  disown({
    echo(i);
    echo(i)
  }) |
  if (i == 2)
    break
};
// Give some time to get the output of the detached expressions.
sleep(100ms);
[00000000] *** 0
[00000000] *** 1
[00000000] *** 0
[00000000] *** 2
[00000000] *** 1
[00000000] *** 2
\end{urbiscript}

\item \lstinline|persist(\var{expression}, \var{delay})| Return an
  object whose \var{val} slot evaluates to true if the
  \var{expression} has been continuously true for this \var{delay} and
  false otherwise.

  This function is used to implement

\begin{urbiunchecked}
at (condition ~ delay)
  action
\end{urbiunchecked}

  \noindent
  as

\begin{urbiunchecked}
var u = persist (condition, delay);
at (u.val)
  action
\end{urbiunchecked}

  The \lstinline|persist| action will be controlled by the same tags
  as the initial \lstinline|at| block.


\item \lstinline|finally(\var{action}, \var{reaction})|
  Execute the \var{reaction} function at the end of the evaluation of
  the \var{action} function.  This is equivalent to

\begin{urbiunchecked}
{
  var t = Tag.new |
  at (t.leave?)
    reaction() |
  t: action()
}
\end{urbiunchecked}


\end{urbiscriptapi}


%%% Local Variables:
%%% mode: latex
%%% TeX-master: "../urbi-sdk"
%%% ispell-dictionary: "american"
%%% ispell-personal-dictionary: "../urbi.dict"
%%% fill-column: 76
%%% End:

%% Copyright (C) 2009-2010, 2012, Gostai S.A.S.
%%
%% This software is provided "as is" without warranty of any kind,
%% either expressed or implied, including but not limited to the
%% implied warranties of fitness for a particular purpose.
%%
%% See the LICENSE file for more information.

\section{Date}

This class is meant to record dates in time, with microsecond resolution.
\experimental{}

\subsection{Prototypes}
\begin{refObjects}
\item[Orderable]
\item[Comparable]
\end{refObjects}

\subsection{Construction}

Without argument, newly constructed Dates refer to the current date.

\begin{urbiunchecked}[firstnumber=1]
Date.new;
[00000001] 2010-08-17 14:40:52.549726
\end{urbiunchecked}

With a string argument \var{d}, refers to the date contained in \var{d}.
The string should be formatted as \samp{\var{yyyy}-\var{mm}-\var{dd}
  \var{hh}:\var{mn}:\var{ss}} (see \refSlot{asString}). \var{mn} and
\var{ss} are optional. If the block \samp{\var{hh}:\var{mn}:\var{ss}} is
absent, the behavior is undefined.

\begin{urbiscript}
Date.new("2003-10-10 20:10:50:637");
[00000001] 2003-10-10 20:10:50.637000

Date.new("2003-10-10 20:10:50");
[00000001] 2003-10-10 20:10:50.000000

Date.new("2003-Oct-10 20:10");
[00000002] 2003-10-10 20:10:00.000000

Date.new("2003-10-10 20");
[00000003] 2003-10-10 20:00:00.000000
\end{urbiscript}

Pay attention that the format is rather strict; for instance too many spaces
between day and time result in an error.

\begin{urbiscript}
Date.new("2003-10-10  20:10:50");
[00001968:error] !!! new: cannot convert to date: 2003-10-10  20:10:50
\end{urbiscript}

\subsection{Slots}

\begin{urbiscriptapi}
\item['+'](<that>)%
  The date which corresponds to waiting \refObject{Duration} \var{that}
  after \this.
\begin{urbiassert}
Date.new("2010-08-17 12:00") + 60s == Date.new("2010-08-17 12:01");
\end{urbiassert}

\item['-'](<that>)%
  If \var{that} is a Date, the difference between \this and \var{that} as a
  \refObject{Duration}.
\begin{urbiassert}
Date.new("2010-08-17 12:01") - Date.new("2010-08-17 12:00") ==  60s;
Date.new("2010-08-17 12:00") - Date.new("2010-08-17 12:01") == -60s;
\end{urbiassert}

If \var{that} is a Duration or a Float, the corresponding Date.

\begin{urbiassert}
Date.new("2010-08-17 12:01") - 60s == Date.new("2010-08-17 12:00");
Date.new("2010-08-17 12:01") - 60s
  == Date.new("2010-08-17 12:01") - Duration.new(60s);
\end{urbiassert}

\item['=='](<that>)%
  Equality test.
\begin{urbiassert}
Date.new("2010-08-17 12:00") == Date.new("2010-08-17 12:00");
Date.new("2010-08-17 12:00") != Date.new("2010-08-17 12:01");
\end{urbiassert}

\item['<'](<that>)%
  Order comparison.
\begin{urbiassert}
   Date.new("2010-08-17 12:00") < Date.new("2010-08-17 12:01");
! (Date.new("2010-08-17 12:01") < Date.new("2010-08-17 12:00"));
\end{urbiassert}

\item[asFloat] The duration since the \refSlot{epoch}, as a Float.
\begin{urbiscript}
var d = Date.new("2002-01-20 23:59:59")|;
assert
{
  d.asFloat == d - d.epoch;
  d.asFloat.isA(Float);
};
\end{urbiscript}

\item[asString] Present as \samp{\var{yyyy}-\var{mm}-\var{dd}
    \var{hh}:\var{mn}:\var{ss}.\var{us}} where:
  \begin{itemize}
  \item \var{yyyy} is the four-digit year,
  \item \var{mm} the three letters name of the month (Jan, Feb, ...),
  \item \var{dd} the two-digit day in the month (from 1 to 31),
  \item \var{hh} the two-digit hour (from 0 to 23),
  \item \var{mn} the two-digit number of minutes in the hour (from 0 to 59),
  \item \var{ss} the two-digit number of seconds in the minute (from 0 to
    59), and
  \item \var{iiiiii} the six-digit number of microseconds.
  \end{itemize}
\begin{urbiassert}
Date.new("2009-02-14 00:31:30").asString == "2009-02-14 00:31:30.000000";
\end{urbiassert}

\item[day]
  The day as a \refObject{Float}.
\begin{urbiscript}
{
  var d = Date.new("2010-09-29 17:32:53");
  assert(d.day == 29);
  d.day = 1;
  assert(d == Date.new("2010-09-01 17:32:53"));
};
\end{urbiscript}
\begin{urbiscript}
Date.new("2010-02-01 17:32:53").day = 29;
[00000001:error] !!! updateSlot: Day of month is not valid for year
\end{urbiscript}

\item[epoch]
  A fixed value, the ``origin of times'': January 1st 1970, at
  midnight.
\begin{urbiunchecked}
Date.epoch == Date.new("1970-01-01 00:00");
\end{urbiunchecked}

\item[hour]
  The hour as a \refObject{Float}.
\begin{urbiscript}
{
  var d = Date.new("2010-09-29 17:32:53");
  assert(d.hour == 17);
  d.hour = 8;
  assert(d == Date.new("2010-09-29 08:32:53"));
};
\end{urbiscript}

\item[microsecond]
  The number of microseconds as a \refObject{Float}.
\begin{urbiscript}
{
  var d = Date.new("2010-09-29 17:32:53.123456");
  assert(d.microsecond == 123456);
  d.microsecond = 654321;
  assert(d == Date.new("2010-09-29 17:32:53.654321"));
};
\end{urbiscript}

\item[minute]
  The minute as a \refObject{Float}.
\begin{urbiscript}
{
  var d = Date.new("2010-09-29 17:32:53");
  assert(d.minute == 32);
  d.minute = 12;
  assert(d == Date.new("2010-09-29 17:12:53"));
};
\end{urbiscript}

\item[month]
  The month as a \refObject{Float}.
\begin{urbiscript}
{
  var d = Date.new("2010-09-29 17:32:53");
  assert(d.month == 9);
  d.month = 3;
  assert(d == Date.new("2010-03-29 17:32:53"));
};
\end{urbiscript}

\item[now] The current date. Equivalent to Date.new.
\begin{urbiunchecked}
Date.now;
[00000000] 2012-03-02 15:31:42
\end{urbiunchecked}

\item[second]
  The second as a \refObject{Float}.
\begin{urbiscript}
{
  var d = Date.new("2010-09-29 17:32:53");
  assert(d.second == 53);
  d.second = 37;
  assert(d == Date.new("2010-09-29 17:32:37"));
};
\end{urbiscript}

\item[timestamp] Synonym for \refSlot{asFloat}.

\item[year]
  The year as a \refObject{Float}.
\begin{urbiscript}
{
  var d = Date.new("2010-09-29 17:32:53");
  assert(d.year == 2010);
  d.year = 2000;
  assert(d == Date.new("2000-09-29 17:32:53"));
};
\end{urbiscript}

\end{urbiscriptapi}


%%% Local Variables:
%%% coding: utf-8
%%% mode: latex
%%% TeX-master: "../urbi-sdk"
%%% ispell-dictionary: "american"
%%% ispell-personal-dictionary: "../urbi.dict"
%%% fill-column: 76
%%% End:

%% Copyright (C) 2009-2011, Gostai S.A.S.
%%
%% This software is provided "as is" without warranty of any kind,
%% either expressed or implied, including but not limited to the
%% implied warranties of fitness for a particular purpose.
%%
%% See the LICENSE file for more information.

\section{Dictionary}

A \dfn{dictionary} is an \dfn{associative array}, also known as a \dfn{hash}
in some programming languages.  They are arrays whose indexes are arbitrary
objects.

\subsection{Example}

The following session demonstrates the features of the Dictionary objects.

\begin{urbiscript}[firstnumber=1]
var d = ["one" => 1, "two" => 2];
[00000001] ["one" => 1, "two" => 2]

for (var p : d)
  echo (p.first + " => " + p.second);
[00000003] *** one => 1
[00000002] *** two => 2

"three" in d;
[00000004] false
d["three"];
[00000005:error] !!! missing key: three
d["three"] = d["one"] + d["two"]|;
"three" in d;
[00000006] true
d.getWithDefault("four", 4);
[00000007] 4
\end{urbiscript}

\subsection{Hash values}
\label{sec:dictionary:hash}

Arbitrary objects can be used as dictionary keys. To map to the same cell,
two objects used as keys must have equal hashes (retrieved with the
\refSlot[Object]{hash} method) and be equal to each other (in the
\refSlot[Object]{'=='} sense).

This means that two different objects may have the same hash: the equality
operator (\refSlot[Object]{'=='}) is checked in addition to the hash, to
handle such collision.  However a good hash algorithm should avoid this
case, since it hinders performances.

See \refSlot[Object]{hash} for more detail on how to override hash
values. Most standard value-based classes implement a reasonable hash
function: see \refSlot[Float]{hash}, \refSlot[String]{hash},
\refSlot[List]{hash}, \ldots

\subsection{Prototypes}

\begin{refObjects}
\item[Comparable]
\item[Container]
\item[Object]
\item[RangeIterable]
\end{refObjects}

\subsection{Construction}

The Dictionary constructor takes arguments by pair (key, value).

\begin{urbiscript}
Dictionary.new("one", 1, "two", 2);
[00000000] ["one" => 1, "two" => 2]
Dictionary.new;
[00000000] [ => ]
\end{urbiscript}

There must be an even number of arguments.

\begin{urbiscript}
Dictionary.new("1", 2, "3");
[00000001:error] !!! new: odd number of arguments
\end{urbiscript}

You are encouraged to use the specific syntax for Dictionary literals:

\begin{urbiscript}
["one" => 1, "two" => 2];
[00000000] ["one" => 1, "two" => 2]
[=>];
[00000000] [ => ]
\end{urbiscript}

An extra comma can be added at the end of the list.

\begin{urbiscript}
[
  "one" => 1,
  "two" => 2,
];
[00000000] ["one" => 1, "two" => 2]
\end{urbiscript}

It is guaranteed that the pairs to insert are evaluated left-to-write, key
first, the value.

\begin{urbiassert}
   ["a".fresh => "b".fresh, "c".fresh => "d".fresh]
== ["a_5"     => "b_6",     "c_7"     => "d_8"];
\end{urbiassert}

\subsection{Slots}

\begin{urbiscriptapi}
\item['=='](<that>)%
  Whether \this equals \var{that}.  Expects members to be
  \refObject{Comparable}.
\begin{urbiassert}
[ => ] == [ => ];
["a" => 1, "b" => 2] == ["b" => 2, "a" => 1];
\end{urbiassert}


\item|'[]'|(<key>)%
  Syntactic sugar for \lstinline|get(\var{key})|.

\begin{urbiscript}
assert (["one" => 1]["one"] == 1);
["one" => 1]["two"];
[00000012:error] !!! missing key: two
\end{urbiscript}


\item|'[]='|(<key>, <value>)%
  Syntactic sugar for \lstinline|set(\var{key}, \var{value})|, but returns
  \var{value}.

\begin{urbiassert}
var d = ["one" =>"2"];
(d["one"] = 1) == 1;
d["one"] == 1;
\end{urbiassert}


\item[asBool]
  Negation of \refSlot{empty}.
\begin{urbiassert}
[=>].asBool == false;
["key" => "value"].asBool == true;
\end{urbiassert}


\item[asList]%
  The contents of the dictionary as a \refObject{Pair} list (\var{key},
  \var{value}).

\begin{urbiassert}
["one" => 1, "two" => 2].asList == [("one", 1), ("two", 2)];
\end{urbiassert}

  \noindent
  Since Dictionary derives from \refObject{RangeIterable}, it is easy
  to iterate over a Dictionary using a range-\lstinline|for|
  (\autoref{sec:lang:for:each}).  No particular order is ensured.
\begin{urbiscript}
{
  var res = [];
  for| (var entry: ["one" => 1, "two" => 2])
    res << entry.second;
  assert(res == [1, 2]);
};
\end{urbiscript}


\item[asString] A string representing the dictionary.  There is no guarantee
  on the order of the output.
\begin{urbiassert}
                [=>].asString == "[ => ]";
["a" => 1, "b" => 2].asString == "[\"a\" => 1, \"b\" => 2]";
\end{urbiassert}

\item[elementAdded] An event emitted each time a new element is added to
  the Dictionary.

\item[elementChanged] An event emitted each time the value associated to a
  key of the Dictionary is changed.

\item[elementRemoved] An event emitted each time an element is removed from
  the Dictionary.

\begin{urbiscript}
d = [ => ] |;
at(d.elementAdded?) echo ("added");
at(d.elementChanged?) echo ("changed");
at(d.elementRemoved?) echo ("removed");

d["key1"] = "value1";
[00000001] "value1"
[00000001] *** added

d["key2"] = "value2";
[00000001] "value2"
[00000001] *** added

d["key2"] = "value3";
[00000001] "value3"
[00000001] *** changed

d.erase("key2");
[00000002] ["key1" => "value1"]
[00000001] *** removed

d.clear;
[00000003] [ => ]
[00000001] *** removed

d.clear;
[00000003] [ => ]
\end{urbiscript}

\item[clear]
  Empty the dictionary.

\begin{urbiassert}
["one" => 1].clear.empty;
\end{urbiassert}


\item[empty]
  Whether the dictionary is empty.

\begin{urbiassert}
[=>].empty == true;
["key" => "value"].empty == false;
\end{urbiassert}


\item[erase](<key>) Remove the mapping for \var{key}.
\begin{urbicomment}
removeSlot("d")|;
\end{urbicomment}
\begin{urbiscript}
{
  var d = ["one" => 1, "two" => 2];
  assert
  {
    d.erase("two") === d;
    d == ["one" => 1];
  };

  try
  {
    ["one" => 1, "two" => 2].erase("three");
    echo("never reached");
  }
  catch (var e if e.isA(Dictionary.KeyError))
  {
    assert(e.key == "three")
  };
};
\end{urbiscript}

%% commented until a consensus is reached.
%%
%% \item[extend](<ext>)
%%   Extend with the dictionary \var{ext}.
%%   Return the value of the new dictionary.
%% \begin{urbiscript}
%% d = ["one" => 1, "two" => 2];
%% [00000001] ["one" => 1, "two" => 2]
%% d.extend(["one" => 0, "three" => 3]);
%% [00000002] ["one" => 0, "three" => 3, "two" => 2]
%% \end{urbiscript}

\item[get](<key>)%
  The value associated to \var{key}.  A \lstinline|Dictionary.KeyError|
  exception is thrown if the key is missing.
  % FIXME: the following exception test should be rewritten when (if)
  % we introduce the throw assertion.
\begin{urbiscript}
var d = ["one" => 1, "two" => 2]|;

assert(d.get("one") == 1);
["one" => 1, "two" => 2].get("three");
[00000010:error] !!! missing key: three

try
{
  d.get("three");
  echo("never reached");
}
catch (var e if e.isA(Dictionary.KeyError))
{
  assert(e.key == "three")
};
\end{urbiscript}


\item[getWithDefault](<key>, <defaultValue>)%
  The value associated to \var{key} if it exists, \var{defaultValue}
  otherwise.

\begin{urbiassert}
var d = ["one" => 1, "two" => 2];
d.getWithDefault("one",  -1) == 1;
d.getWithDefault("three", 3) == 3;
\end{urbiassert}


\item[has](<key>)%
  Whether the dictionary has a mapping for \var{key}.

\begin{urbiassert}
var d = ["one" => 1];
d.has("one");
!d.has("zero");
\end{urbiassert}

  The infix operators \lstinline|in| and \lstinline|not in| use
  \lstinline|has| (see \autoref{sec:lang:op:containers}).

\begin{urbiassert}
"one" in     ["one" => 1];
"two" not in ["one" => 1];
\end{urbiassert}


\item[init](<key1>, <value1>, ...)%
  Insert the mapping from \var{key1} to \var{value1} and so forth.

\begin{urbiscript}
Dictionary.clone.init("one", 1, "two", 2);
[00000000] ["one" => 1, "two" => 2]
\end{urbiscript}


\item[keys]%
  The list of all the keys.  No particular order is ensured.  Since
  \refObject{List} features the same function, uniform iteration over
  a List or a Dictionary is possible.
\begin{urbiassert}
var d = ["one" => 1, "two" => 2];
d.keys == ["one", "two"];
\end{urbiassert}


\item[matchAgainst](<handler>, <pattern>)
  Pattern matching on members.  See \refObject{Pattern}.

\begin{urbiscript}
{
  // Match a subset of the dictionary.
  ["a" => var a] = ["a" => 1, "b" => 2];
  // get the matched value.
  assert(a == 1);
};
\end{urbiscript}


\item[set](<key>, <value>)%
  Map \var{key} to \var{value} and return \this so that invocations to
  \refSlot{set} can be chained.  The possibly existing previous mapping is
  overridden.

\begin{urbiscript}
[=>].set("one", 2)
    .set("two", 2)
    .set("one", 1);
[00000000] ["one" => 1, "two" => 2]
\end{urbiscript}


\item[size]
  Number of element in the dictionary.

\begin{urbiassert}
var d = [=>];  d.size == 0;
d["a"] = 10;   d.size == 1;
d["b"] = 20;   d.size == 2;
d["a"] = 30;   d.size == 2;
\end{urbiassert}



\end{urbiscriptapi}


%%% Local Variables:
%%% mode: latex
%%% TeX-master: "../urbi-sdk"
%%% ispell-dictionary: "american"
%%% ispell-personal-dictionary: "../urbi.dict"
%%% fill-column: 76
%%% End:

%% Copyright (C) 2009-2010, Gostai S.A.S.
%%
%% This software is provided "as is" without warranty of any kind,
%% either expressed or implied, including but not limited to the
%% implied warranties of fitness for a particular purpose.
%%
%% See the LICENSE file for more information.

\section{Directory}

A \dfn{Directory} represents a directory of the file system.

\subsection{Prototypes}
\begin{refObjects}
\item[Object]
\end{refObjects}

\subsection{Construction}

A \dfn{Directory} can be constructed with one argument: the path of
the directory using a \refObject{String} or a \refObject{Path}. It can
also be constructed by the method open of \refObject{Path}.

\begin{urbiscript}
Directory.new(".");
[00000001] Directory(".")
Directory.new(Path.new("."));
[00000002] Directory(".")
\end{urbiscript}

\subsection{Slots}
\begin{urbiscriptapi}
\item['/'](<str>)
  \experimental{}

  The \var{str} \refObject{String} is concatenated with the directory path.
  If the resulting path is either a directory or a file, \refSlot{'/'} will
  returns either a \refObject{Directory} or a \refObject{File} object.
\begin{urbiscript}
var dir1 = Directory.create("dir1")|;
var dir2 = Directory.create("dir1/dir2")|;
var file = File.create("dir1/file")|;
dir1 / "dir2";
[00000001] Directory("dir1/dir2")
dir1 / "file";
[00000002] File("dir1/file")
dir1.removeAll;
\end{urbiscript}

\item['<<'](<entity>)
  \experimental{}

  If \var{entity} is a \refObject{Directory} or a \refObject{File},
  \refSlot{'<<'} copies \var{entity} in the \lstinline|this| directory.
  Return \lstinline|this| to allow chained operations.
\begin{urbiscript}
dir1 = Directory.create("dir1")|;
dir2 = Directory.create("dir2")|;
file = File.create("file")|;
dir1 << file << dir2;
[00000001] Directory("dir1")
dir1.content;
[00000003] ["dir2", "file"]
dir2;
[00000004] Directory("dir2")
file;
[00000005] File("file")
dir1.removeAll;
dir2.removeAll;
file.remove;
\end{urbiscript}

\item[asList]
  The contents of the directory as a \refObject{Path} list.  The
  various paths include the name of the directory \this.

\item[asString] A \refObject{String} containing the path of the directory.
\begin{urbiassert}
Directory.new(".").asString == ".";
\end{urbiassert}

\item[asPath] A \refObject{Path} being the path of the directory.

\item[content]
  The contents of the directory as a \refObject{String} list.  The
  strings include only the last component name; they do not contain
  the directory name of \this.

\item[copy](<dirname>)
  Copy recursively all items of the \lstinline|this| directory
  into the directory \var{dirname} after creating it.
\begin{urbiscript}
dir1 = Directory.create("dir1")|;
dir2 = Directory.create("dir1/dir2")|;
file = File.create("dir1/file")|;
var directory1 = dir1.copy("directory1");
[00000001] Directory("directory1")
dir1;
[00000002] Directory("dir1")
directory1.content;
[00000003] ["dir2", "file"]
dir1.removeAll;
directory1.removeAll;
\end{urbiscript}

\item[copyInto](<dirname>)
  \experimental{}

  Copy \lstinline|this| into \var{dirname} without creating it.
\begin{urbiscript}
var dir = Directory.create("dir")|;
dir1 = Directory.create("dir1")|;
dir2 = Directory.create("dir1/dir2")|;
file = File.create("dir1/file")|;
dir1.copyInto(dir);
[00000001] Directory("dir/dir1")
dir1;
[00000002] Directory("dir1")
dir1.content;
[00000003] ["dir2", "file"]
dir.content;
[00000004] ["dir1"]
Directory.new("dir/dir1").content;
[00000005] ["dir2", "file"]
dir.removeAll;
dir1.removeAll;
\end{urbiscript}

\item[clear]
  Remove all children recursively but not the directory itself. After a
  call to \refSlot{clear}, a call to \refSlot{empty} should return
  \lstinline|true|.
\begin{urbiscript}
dir1 = Directory.create("dir1")|;
dir2 = Directory.create("dir1/dir2")|;
var file1 = File.create("dir1/file1")|;
var file2 = File.create("dir1/dir2/file2")|;
dir1.content;
[00000001] ["dir2", "file1"]
dir2.content;
[00000002] ["file2"]
dir1.clear;
assert(dir1.empty);
dir1.remove;
\end{urbiscript}

\item[create](<name>)
  Create the directory \var{name} where \var{name} is either a
  \refObject{String} or a \refObject{Path}. In addition to system errors that
  can occur, errors are raised if directory or file \var{name} already exists.
\begin{urbiscript}
dir = Directory.new("dir");
[00000001:error] !!! new: directory does not exist: "dir"
dir = Directory.create("dir");
[00000002] Directory("dir")
dir = Directory.create("dir");
[00000001:error] !!! create: directory exists: "dir"
dir.content;
[00000003] []
dir.remove;
\end{urbiscript}

\item[createAll](<name>)
  Create the directory \var{name} where \var{name} is either a
  \refObject{String} or a \refObject{Path}. If \var{name} is a
  path (or a \refObject{String} describing a path) no errors are
  raised if one directory doesn't exist or already exists. Instead
  \refSlot{createAll} creates them all as in the Unix \samp{make -p} command.
\begin{urbiscript}
Directory.create("dir1/dir2/dir3");
[00000001:error] !!! create: no such file or directory: "dir1/dir2/dir3"
dir1 = Directory.create("dir1");
[00000002] Directory("dir1")
Directory.createAll("dir1/dir2/dir3");
[00000002] Directory("dir1/dir2/dir3")
dir1.removeAll;
\end{urbiscript}

\item[empty]
  Whether the directory is empty.
\begin{urbiscript}
dir = Directory.create("dir")|;
assert(dir.empty);
File.create("dir/file")|;
assert(!dir.empty);
dir.removeAll;
\end{urbiscript}

\item[exists]
  Whether the directory still exists.
\begin{urbiscript}
dir = Directory.create("dir");
[00000001] Directory("dir")
assert(dir.exists);
dir.remove;
assert(!dir.exists);
\end{urbiscript}

\item[fileCreated](<name>)%
  Event launched when a file is created inside the directory.
  May not exist if not supported by your architecture.

%% firstline is used to separate inotify test from the others.
\begin{urbiscript}[firstnumber=1]
if (Path.new("./dummy.txt").exists)
  File.new("./dummy.txt").remove;

  {
    var d = Directory.new(".");
    waituntil(d.fileCreated?(var name));
    assert
    {
      name == "dummy.txt";
      Path.new(d.asString + "/" + name).exists;
    };
  }
&
  {
    sleep(100ms);
    File.create("./dummy.txt");
  }|;
\end{urbiscript}

\item[fileDeleted](<name>)%
  Event launched when a file is deleted from the directory.  May not exist
  if not supported by your architecture.

\begin{urbiscript}
if (!Path.new("./dummy.txt").exists)
  File.create("./dummy.txt")|;

  {
    var d = Directory.new(".");
    waituntil(d.fileDeleted?(var name));
    assert
    {
      name == "dummy.txt";
      !Path.new(d.asString + "/" + name).exists;
    };
  }
&
  {
    sleep(100ms);
    File.new("./dummy.txt").remove;
  }|;
\end{urbiscript}
%% Use firstline after this test if this is not related to inotify.

\item[basename]
  Return a \refObject{String} containing the path of the directory without
  its dirname.
\begin{urbiscript}[firstnumber=1]
var dir1 = Directory.create("dir1");
[00000001] Directory("dir1")
var dir2 = Directory.create("dir1/dir2");
[00000002] Directory("dir1/dir2")
dir1.basename;
[00000002] "dir1"
dir2.basename;
[00000003] "dir2"
dir1.removeAll;
\end{urbiscript}

\item[lastModifiedDate]
  \experimental{}

  Return a \refObject{Date} object stating when the directory was last modified.

\item[moveInto](<dirname>)
  \experimental{}

  Move \lstinline|this| into \var{dirname} without creating it.
\begin{urbiscript}
dir1 = Directory.create("dir1")|;
dir2 = Directory.create("dir1/dir2")|;
var file = File.create("dir1/file")|;
var dir = Directory.create("dir")|;
dir1.moveInto(dir);
[00000001] Directory("dir/dir1")
dir1;
[00000002] Directory("dir/dir1")
dir1.content;
[00000003] ["dir2", "file"]
dir.content;
[00000004] ["dir1"]
dir.removeAll;
\end{urbiscript}

\item[parent]
  Return the parent of the directory.
\begin{urbiscript}
Directory.create("dir")|;
dir = Directory.create("dir/dir")|;
dir.parent;
[00000001] Directory("dir")
assert(dir.parent.parent.asString == Directory.current.asString);
dir.parent.removeAll;
\end{urbiscript}

\item[remove]
  Remove the directory only if it is empty.
\begin{urbiscript}
dir = Directory.create("dir")|;
File.create("dir/file")|;
dir.remove;
[00000001:error] !!! remove: directory not empty: "dir"
dir.clear;
dir.remove;
assert(!dir.exists);
\end{urbiscript}

\item[removeAll]
  Remove all children recursively including the directory itself.
\begin{urbiscript}
dir1 = Directory.create("dir1")|;
dir2 = Directory.create("dir1/dir2")|;
var file1 = File.create("dir1/file1")|;
var file2 = File.create("dir1/dir2/file2")|;
dir1.removeAll;
assert(!dir1.exists);
\end{urbiscript}

\item[rename]
  Rename or move the directory.
\begin{urbiscript}
dir = Directory.create("dir")|;
File.create("dir/file")|;
dir.rename("other");
[00000001] Directory("other")
dir;
[00000002] Directory("other")
dir.content;
[00000003] ["file"]
dir2 = Directory.create("dir2")|;
dir.rename("dir2/other2");
[00000004] Directory("dir2/other2")
dir;
[00000005] Directory("dir2/other2")
dir.content;
[00000006] ["file"]
dir2.removeAll;
\end{urbiscript}

\item[size]
  \experimental{}

  The size of all the directory content computed recursively.
\begin{urbiscript}
dir = Directory.create("dir")|;
Directory.create("dir/dir")|;
File.save("dir/file", "content");
file1 = File.create("dir/file")|;
File.save("dir/dir/file", "content");
file2 = File.create("dir/dir/file")|;
assert(dir.size() == file1.size() + file2.size());
\end{urbiscript}
\end{urbiscriptapi}

%%% Local Variables:
%%% coding: utf-8
%%% mode: latex
%%% TeX-master: "../urbi-sdk"
%%% ispell-dictionary: "american"
%%% ispell-personal-dictionary: "../urbi.dict"
%%% fill-column: 76
%%% End:

%% Copyright (C) 2009-2010, Gostai S.A.S.
%%
%% This software is provided "as is" without warranty of any kind,
%% either expressed or implied, including but not limited to the
%% implied warranties of fitness for a particular purpose.
%%
%% See the LICENSE file for more information.

\section{Duration}

This class records differences between \refObject[Date]{Dates}.
\experimental{}

\subsection{Prototypes}
\begin{refObjects}
\item[Float]
\end{refObjects}

\subsection{Construction}

Without argument, a null duration.

\begin{urbiscript}[firstnumber=1]
Duration.new;
[00000001] Duration(0s)
Duration.new(1h);
[00023593] Duration(3600s)
\end{urbiscript}

Durations can be negative.

\begin{urbiscript}
Duration.new(-1);
[00000001] Duration(-1s)
\end{urbiscript}


\subsection{Slots}

\begin{urbiscriptapi}
\item[asFloat]
  Return the duration as a \refObject{Float}.
\begin{urbiassert}
Duration.new(1000).asFloat == 1000;
\end{urbiassert}


\item[asString]
  Return the duration as a \refObject{String}.
\begin{urbiassert}
Duration.new(1000).asString == "1000s";
\end{urbiassert}


\item[seconds]
  Return the duration as a \refObject{Float}.
\begin{urbiassert}
Duration.new(1000).seconds == 1000;
\end{urbiassert}
\end{urbiscriptapi}


%%% Local Variables:
%%% coding: utf-8
%%% mode: latex
%%% TeX-master: "../urbi-sdk"
%%% ispell-dictionary: "american"
%%% ispell-personal-dictionary: "../urbi.dict"
%%% fill-column: 76
%%% End:

%% Copyright (C) 2009-2011, Gostai S.A.S.
%%
%% This software is provided "as is" without warranty of any kind,
%% either expressed or implied, including but not limited to the
%% implied warranties of fitness for a particular purpose.
%%
%% See the LICENSE file for more information.

\section{Event}

An \dfn{event} can be ``emitted'' and ``caught'', or ``sent'' and
``received''.  See also \autoref{sec:tut:events}.

\subsection{Examples}

There are several examples of uses of events in the documentation of
event-based constructs.  See \lstinline{at} (\autoref{sec:lang:at}),
\lstinline{waituntil} (\autoref{sec:lang:waituntil}), \lstinline{whenever}
(\autoref{sec:lang:whenever}), and so forth.  The tutorial chapter about
event-based programming contains other examples, see
\autoref{sec:tut:event-prog}.

\subsection{Synchronicity of Event Handling}
\label{sec:event:sync}
A particular emphasis should be put on the \dfn{synchronicity} of the event
handling, i.e., whether the bodies of the event handlers are run before the
control flow returns from the event emission.  By default, (i.e.,
\lstinline|at (e?...)| and \lstinline|e!(...)|/\lstinline|e.emit(...)|)) the
execution is \dfn{asynchronous}, but if either the emitted or the handler is
marked asynchronous (i.e., \lstinline|at sync (e?...)| or
\lstinline|e.syncEmit(...)|), then the execution is \dfn{synchronous}.

Contrast the following examples:

\begin{multicols}{2}
\paragraph{Asynchronous handlers}~

\begin{urbiscript}[xrightmargin=0mm,xleftmargin=0mm]
var e = Event.new |;

at (e?)
  { echo("a"); sleep(20ms); echo("b") }
onleave
  { echo("c"); sleep(20ms); echo("d") };

e! | echo("done");
[00000001] *** done
sleep(25ms);
[00000002] *** a
[00000003] *** c
[00000101] *** b
[00000102] *** d

e.syncEmit | echo("done");
[00000001] *** a
[00000101] *** b
[00000102] *** c
[00000202] *** d
[00000203] *** done
\end{urbiscript}
\columnbreak

\paragraph{Synchronous handlers}~

\begin{urbicomment}
removeSlots("e");
\end{urbicomment}
\begin{urbiscript}[xrightmargin=0mm,xleftmargin=0mm]
var e = Event.new |;

at sync (e?)
  { echo("a"); sleep(20ms); echo("b") }
onleave
  { echo("c"); sleep(20ms); echo("d") };

e! | echo("done");
// No need to sleep.
[00000011] *** a
[00000031] *** b
[00000031] *** c
[00000052] *** d
[00000052] *** done

e.syncEmit | echo("done");
[00000052] *** a
[00000073] *** b
[00000073] *** c
[00000094] *** d
[00000094] *** done
\end{urbiscript}
\end{multicols}

For more information about the synchronicity of event handlers, see
\autoref{sec:lang:at:sync-async}.

\subsection{Sustained Events}
\label{sec:event:sustain}

Events can also be sustained during a time span starting at
\refSlot{trigger} and ending at \lstinline|handler.stop|.  Note that the
\lstinline|onleave|-clauses of the event handlers is not executed right
after the event was first triggered, but rather when it is stopped.

Synchronicity for sustained events is more complex: the
\lstinline|at|-clause is handled asynchronously iff \emph{both} the emission
and the handler are asynchronous, whereas the \lstinline|onleave|-clause is
handled asynchronously iff the emission was synchronous.  Be warned, but do
not depend on this, as in the future we might change this.

\begin{multicols}{2}
\paragraph{Asynchronous Trigger}~

\begin{urbicomment}
removeSlots("e");
\end{urbicomment}
\begin{urbiscript}[xrightmargin=0mm,xleftmargin=0mm]
var e = Event.new|;
at (e?(var v))
  { echo("a"+v); sleep(20ms); echo("b"+v) }
onleave
  { echo("c"+v); sleep(20ms); echo("d"+v) };

var handler = e.trigger("1") | echo("?");
[00000001] *** ?
[00000002] *** a1
[00000102] *** b1
sleep(200ms);
handler.stop | echo("?");
[00000301] *** ?
sleep(25ms);
[00000302] *** c1
[00000402] *** d1

// at and onleave clauses may overlap.
handler = e.trigger("2") | handler.stop;
sleep(25ms);
[00000001] *** a2
[00000002] *** c2
sleep(25ms);
[00000201] *** b2
[00000202] *** d2

handler = e.syncTrigger("3") | echo("?");
[00000002] *** a3
[00000102] *** b3
[00000001] *** ?
handler.stop | echo("?");
[00000302] *** c3
[00000402] *** d3
[00000301] *** ?
\end{urbiscript}
\columnbreak

\paragraph{Synchronous Trigger}~

\begin{urbicomment}
removeSlots("e", "handler");
\end{urbicomment}
\begin{urbiscript}[xrightmargin=0mm,xleftmargin=0mm]
var e = Event.new|;
at sync (e?(var v))
  { echo("a"+v); sleep(20ms); echo("b"+v) }
onleave
  { echo("c"+v); sleep(20ms); echo("d"+v) };

var handler = e.trigger("1") | echo("?");
// No need to sleep.
[00000002] *** a1
[00000102] *** b1
[00000001] *** ?
handler.stop | echo("?");
[00000301] *** ?
sleep(25ms);
[00000302] *** c1
[00000402] *** d1

// at and onleave clauses don't overlap.
handler = e.trigger("2") | handler.stop;
sleep(25ms);
[00000001] *** a2
[00000201] *** b2
[00000002] *** c2
[00000202] *** d2

handler = e.syncTrigger("3") | echo("?");
[00000002] *** a3
[00000102] *** b3
[00000001] *** ?
handler.stop | echo("?");
[00000302] *** c3
[00000402] *** d3
[00000301] *** ?
\end{urbiscript}
\end{multicols}


\subsection{Prototypes}
\begin{refObjects}
\item[Object]
\end{refObjects}

\subsection{Construction}
\label{sec:stdlib:event:ctor}

An \lstinline{Event} is created like any other object.  The constructor
takes no argument.

\begin{urbiscript}[firstnumber=1]
var e = Event.new;
[00000001] Event_0x9ad8118
\end{urbiscript}

\subsection{Slots}
\begin{urbiscriptapi}
\item[asEvent]
  Return \this.

\item[emit](<args>[])%
  Fire an ``instantaneous'' and ``asynchronous'' \refObject{Event}. This
  function is called by the \lstinline|!| operator.  It takes any number of
  arguments, passed to the receiver when the event is caught.
\begin{urbicomment}
removeSlots("e");
\end{urbicomment}
\begin{urbiscript}
var e = Event.new|;
// No handler, lost message.
e.emit;
at (e?)               echo("e");
at (e?())             echo("e()");
at (e?(var x))        echo("e(%s)" % [x]);
at (e?(var x, var y)) echo("e(%s, %s)" % [x, y]);

// This is what e! does.
e.emit;
[00000135] *** e
[00000135] *** e()

// This is what e!() does: same as e!.
e.emit();
[00000138] *** e
[00000138] *** e()

// This is what e!(1, [2]) does.
e.emit(1, [2]);
[00000141] *** e
[00000141] *** e(1, [2])

// This is what e!(1, [2], "three") does.
e.emit(1, [2], "three");
[00000146] *** e
\end{urbiscript}

To sustain an event, see \refSlot{trigger}.  See \autoref{sec:event:sync}
and \refSlot{syncEmit} for details about the synchronicity of the handling.

  %% An event can also be emitted for a certain duration using
  %% \lstinline|~|.  The execution of \lstinline|at| clauses etc., is
  %% asynchronous: the control flow might be released by the
  %% \lstinline|emit| call before all the watchers have finished their
  %% execution.

\item[onEvent](<guard>, <enter>, <leave>, <sync>)%
  This is the low-level routine used to implement the \lstindex|at|
  construct.  Indeed,
  \lstinline|at (\var{e}? if \var{cond}) \var{enter} onleave \var{leave}|
  is (roughly) translated into
\begin{urbiunchecked}
\var{e}
  .onEvent(
    closure (var '$evt', var '$payload')                 { \var{cond}  },
    closure (var '$evt', var '$payload', var '$pattern') { \var{enter} },
    closure (var '$evt', var '$payload', var '$pattern') { \var{leave} },
    false)
\end{urbiunchecked}

\noindent
where the \lstinline|false| would be \lstinline|true| in case of an
\lstinline|at sync| construct.  The \var{cond} discards the event iff it
returns \lstinline{void}.

\begin{urbicomment}
removeSlots("e");
\end{urbicomment}
\begin{urbiscript}
var e = Event.new|;
e.onEvent(
  function (var args[]) { echo("cond 1") | true },
  function (var args[]) { echo("enter 1") },
  function (var args[]) { echo("leave 1") },
  true);

e.onEvent(
  function (var args[]) { echo("cond 2") },
  function (var args[]) { echo("enter 2") },
  function (var args[]) { echo("leave 2") },
  true);

e.emit(12);
[00001619] *** cond 1
[00001619] *** enter 1
[00001619] *** leave 1
[00001619] *** cond 2

var h = e.trigger|;
[00001620] *** cond 1
[00001620] *** enter 1
[00001620] *** cond 2

h.stop;
[00001621] *** leave 1
\end{urbiscript}

This function is internal and it might change in the future.

\item[onSubscribe]%
  This slot is not set by default. You can optionally assign an event to
  it. In this case, it is triggered each time some code starts watching this
  event (by setting up an \lstinline|at| or a \lstinline|waituntil| on it
  for instance).

  Throw a synchronized event. This call awaits that all functions that have
  to react to this event have returned.  This function can have the same
  arguments as \refSlot{emit}.

\item[syncEmit](<args>[])%
  Same as \refSlot{emit} but require a synchronous handling.  See
  \autoref{sec:event:sync} for details.

\item[syncTrigger](<args>[])%
  Same as \refSlot{trigger} but the call will be synchronous (see
  \autoref{sec:event:sync}). The \lstinline|stop| method of the handler
  object will be synchronous as well.  See \autoref{sec:event:sustain} for
  examples.

\item[trigger](<args>[])%
  Fire a sustained event (for an unknown amount of time) and return a
  handler object whose \lstinline|stop| method stops the event. This method
  is asynchronous and the \lstinline|stop| call will be asynchronous as
  well.  See \autoref{sec:event:sustain} for examples.

\item \lstinline+'||'(\var{that})+%
  Logical ``or'' on events: a new Event that triggers whenever \this or
  \that triggers.

\begin{urbiscript}
var e1 = Event.new|;
var e2 = Event.new|;
var e_or = e1 || e2|;
at (e_or?)
  echo("!");
e1!;
[00000004] *** !
e2!;
[00000005] *** !
\end{urbiscript}

\item['<<'](<that>)%
  Watch a \that event status and reproduce it on itself, return \this.  This
  operator is similar to an optimized \lstinline,||=, operator.  Do not make
  events watch for themselves, directly or indirectly.

\begin{urbiscript}
var e3 = Event.new|;
var e4 = Event.new|;
var e_watch = Event.new << e3 << e4 |;
at (e_watch?)
  echo("!");
e3!;
[00000006] *** !
e4!;
[00000007] *** !
\end{urbiscript}


\end{urbiscriptapi}

%%% Local Variables:
%%% mode: latex
%%% TeX-master: "../urbi-sdk"
%%% ispell-dictionary: "american"
%%% ispell-personal-dictionary: "../urbi.dict"
%%% fill-column: 76
%%% End:

\section{Exception}

Exceptions are used to handle errors.  More generally, they are a
means to escape from the normal control-flow to handle exceptional
situations.

The language support for throwing and catching exceptions (using
\lstinline|try|/\lstinline|catch| and \lstinline|throw|, see
\autoref{sec:lang:except}) work perfectly well with any kind of
object, yet it is a good idea to throw only objects that derive from
\lstinline|Exception|.

\subsection{Prototypes}
\begin{itemize}
\item \refObject{Object}
\end{itemize}

\subsection{Construction}

There are several types of exceptions, each of which corresponding to
a particular kind of error.  The top-level object,
\lstinline|Exception|, takes a single argument: an error message.

\begin{urbiscript}
Exception.new("something bad has happened!");
[00000001] Exception `something bad has happened!'
Exception.Arity.new("myRoutine", 1, 10, 23);
[00000002] Arity `myRoutine: expected between 10 and 23 arguments, given 1'
\end{urbiscript}


\subsection{Slots}
\begin{itemize}
\item \lstinline|ArgumentType.new(\var{routine}, \var{index}, \var{expected}, \var{effective})|\\
  Derives from \lstinline|Exception.Type|.  The \var{routine} was
  called with a \var{index}-nth argument of type \var{effective}
  instead of \var{expected}.
\begin{urbiscript}[firstnumber=last]
Exception.ArgumentType.new("myRoutine", 1, "myExceptation", "hisResult");
[00000003] ArgumentType `myRoutine: unexpected "hisResult" for argument 1, expected a String'
\end{urbiscript}

\item \lstinline|Arity.new(\var{routine}, \var{effective}, \var{min}, \var{max} = void)|\\
  The \var{routine} was called with an incorrect number of arguments
  (\var{effective}).  It requires at least \var{min} arguments, and,
  if specified, at most \var{max}.
\begin{urbiscript}[firstnumber=last]
Exception.Arity.new("myRoutine", 1, 10, 23);
[00000004] Arity `myRoutine: expected between 10 and 23 arguments, given 1'
\end{urbiscript}
%% try
%% {
%%   Math.cos(1, 2);
%% }
%% catch (var e)
%% {
%%   assert_eq(e,
%%   Exception.Arity.new("cos", 2, 1));
%% };

\item \lstinline|BadInteger.new(\var{routine}, \var{fmt}, \var{effective})|\\
  The \var{routine} was called with an inappropriate integer
  (\var{effective}).  Use the format \var{fmt} to create an error
  message from \var{effective}.
\begin{urbiscript}[firstnumber=last]
Exception.BadInteger.new("myRoutine", "bad integer: %s", 12);
[00000005] BadInteger `myRoutine: bad integer: 12'
\end{urbiscript}

\item \lstinline|Constness.new(\var{msg})|\\
  An attempt was made to change a constant value.
\begin{urbiscript}[firstnumber=last]
Exception.Constness.new("cannot change this");
[00000006:error] !!! new: expected 0 argument, given 1
\end{urbiscript}

\item \lstinline|FileNotFound.new(\var{name})|\\
  The file named \var{name} cannot be found.
\begin{urbiscript}[firstnumber=last]
Exception.FileNotFound.new("foo");
[00000007] FileNotFound `file not found: foo'
\end{urbiscript}

\item \lstinline|ImplicitTagComponent.new(\var{msg})|\\
  An attempt was made to create an implicit tag, a component of which
  being undefined.
\begin{urbiscript}[firstnumber=last]
Exception.Constness.new("no such implicit tag");
[00000008:error] !!! new: expected 0 argument, given 1
\end{urbiscript}

\item \lstinline|Lookup.new(\var{object}, \var{name})|\\
  A failed name lookup was performed om \var{object} to find a slot
  named \var{name}.  If \lstinline|Exception.Lookup.fixSpelling| is
  true (which is the default), suggest what the user might have meant
  to use.
\begin{urbiscript}[firstnumber=last]
Exception.Lookup.new(Object, "GetSlot");
[00000009] Lookup `lookup failed: Object'
\end{urbiscript}

\item \lstinline|MatchFailure.new|\\
  A pattern matching failed.
\begin{urbiscript}[firstnumber=last]
Exception.MatchFailure.new;
[00000010] MatchFailure `pattern did not match'
\end{urbiscript}

\item \lstinline|Primitive.new(\var{routine}, \var{msg})|\\
  The builtin \var{routine} encountered an error described by
  \var{msg}.
\begin{urbiscript}[firstnumber=last]
Exception.Primitive.new("myRoutine", "cannot do that");
[00000011] Primitive `myRoutine: cannot do that'
\end{urbiscript}

\item \lstinline|Redefinition.new(\var{name})|\\
  An attempt was made to refine a slot named \var{name}.
\begin{urbiscript}[firstnumber=last]
Exception.Redefinition.new("foo");
[00000012] Redefinition `slot redefinition: foo'
\end{urbiscript}

\item \lstinline|Scheduling.new(\var{msg})|\\
  Something really bad has happened with the \urbi task scheduler.
\begin{urbiscript}[firstnumber=last]
Exception.Scheduling.new("cannot schedule");
[00000013] Scheduling `cannot schedule'
\end{urbiscript}

\item \lstinline|Type.new(\var{expected}, \var{effective})|\\
  A value of type \var{effective} was received, while a value of type
  \var{expected} was expected.
\begin{urbiscript}[firstnumber=last]
Exception.Type.new("myExceptation", "hisResult");
[00000014] Type `unexpected "hisResult", expected a String'
\end{urbiscript}

\item \lstinline|UnexpectedVoid.new|\\
  An attempt was made to read the value of \lstinline|void|.
\begin{urbiscript}[firstnumber=last]
Exception.UnexpectedVoid.new;
[00000015] UnexpectedVoid `unexpected void'
var a = void;
a;
[00000016:error] !!! unexpected void
[00000017:error] !!! lookup failed: a
\end{urbiscript}

\end{itemize}


%%% Local Variables:
%%% mode: latex
%%% TeX-master: "../urbi-sdk"
%%% End:

\section{Executable}

This class is used only as a common ancester to \refObject{Primitive}
and \refObject{Code}.

\subsection{Prototypes}
\begin{itemize}
\item \refObject{Object}
\end{itemize}

\subsection{Construction}

There is no point in constructing an Executable.

\subsection{Slots}

\begin{urbiscriptapi}
\item[asExecutable] Return \lstinline|this|.
\end{urbiscriptapi}


%%% Local Variables:
%%% mode: latex
%%% TeX-master: "../urbi-sdk"
%%% ispell-personal-dictionary: "../urbi.dict"
%%% End:

\section{File}

\subsection{Prototypes}
\begin{itemize}
\item \refObject{Object}
\end{itemize}

\subsection{Construction}

Files may be created from a \refObject{String}, or from a
\refObject{Path}.  The file must exist on the file system, and must be
a file.  You may use \lstinline|File.create| to create a file that
does not exist (or to override an existing one).

\begin{urbiscript}
System.system("(echo 1; echo 2) >file.txt")|;
File.new("file.txt");
[00000001] File("file.txt")

File.new(Path.new("file.txt"));
[00000001] File("file.txt")
\end{urbiscript}

You may use \refObject{InputStream} and \refObject{OutputStream} to
read or write to Files.

\subsection{Methods}

\begin{itemize}
\item \lstinline|asList|\\
  Read the file, and return its content as a list of its lines.
\begin{urbiscript}[firstnumber=last]
System.system("(echo 1; echo 2) >file.txt")|;
assert(File.new("file.txt").asList == ["1", "2"]);
\end{urbiscript}

\item \lstinline|asPrintable|\\
\begin{urbiscript}[firstnumber=last]
System.system("(echo 1; echo 2) >file.txt")|;
assert(File.new("file.txt").asPrintable == "File(\"file.txt\")");
\end{urbiscript}


\item \lstinline|asString|\\
  The name of the opened file.
\begin{urbiscript}[firstnumber=last]
System.system("(echo 1; echo 2) >file.txt")|;
assert(File.new("file.txt").asString == "file.txt");
\end{urbiscript}

\item \lstinline|content|\\
  The content of the file as a \refObject{Binary} object.
\begin{urbiscript}[firstnumber=last]
System.system("(echo 1; echo 2) >file.txt")|;
assert(File.new("file.txt").content.data == "1\n2\n");
\end{urbiscript}

\item \lstinline|create(\var{name})|\\
  If the file \var{name} exists, return a File to it, otherwise create
  an empty one, and return a File to it.  See \refObject{OutputStream}
  for methods to add content to the file.
\begin{urbiscript}[firstnumber=last]
System.system("(echo 1; echo 2) >file.txt")|;
assert(File.create("file.txt").asPrintable == "File(\"file.txt\")");
assert(File.new("file.txt").content.data == "1\n2\n");

assert(File.create("new.txt").content.empty);
\end{urbiscript}
\end{itemize}


%%% Local Variables:
%%% mode: latex
%%% TeX-master: "../urbi-sdk"
%%% End:

\section{Float}

A Float is a floating point number.  It is also used, in the current
version of \us, to represent integers.

\subsection{Prototypes}

\begin{refObjects}
\item[Comparable]
\item[Orderable]
\item[RangeIterable]
\end{refObjects}

\subsection{Construction}
\label{sec:float:ctor}

The most common way to create fresh floats is using the literal
syntax.  Numbers are composed of three parts:
\begin{description}
\item[integral] (mandatory) a non empty sequence of (decimal) digits;
\item[fractional] (optional) a period, and a non empty sequence of
  (decimal) digits;
\item[exponent] (optional) either \samp{e} or \samp{E}, an optional
  sign (\samp{+} or \samp{-}), then a non-empty sequence of digits.
\end{description}

In other words, float literals match the
\lstinline|[0-9]+(\.[0-9]+)?([eE][-+]?[0-9]+)?|
regular expression.  For instance:

\begin{urbiassert}
0 == 0000.0000;
// This is actually a call to the unary '+'.
+1 == 1;
0.123456 == 123456 / 1000000;
1e3 == 1000;
1e-3 == 0.001;
1.234e3 == 1234;
\end{urbiassert}

There are also some special numbers, \lstinline|nan|, \lstinline|inf|
(see below).

\begin{urbiassert}
Math.log(0) == -inf;
Math.exp(-inf) == 0;
(inf/inf).asString == "nan";
\end{urbiassert}

A null float can also be obtained with \lstinline|Float|'s
\lstinline|new| method.

\begin{urbiassert}
Float.new == 0;
\end{urbiassert}

\subsection{Slots}

\begin{urbiscriptapi}
\item[abs]
  Absolute value of the target.
\begin{urbiassert}
(-5).abs == 5;
  0 .abs == 0;
  5 .abs == 5;
\end{urbiassert}

\item[acos]
  Arccosine of the target.
\begin{urbiassert}
0.acos == Float.pi/2;
1.acos == 0;
\end{urbiassert}

\item[asBool]
  Whether non null.
\begin{urbiassert}
0.asBool == false;
0.1.asBool == true;
(-0.1).asBool == true;
inf.asBool == true;
nan.asBool == true;
\end{urbiassert}

\item[asFloat]
  Return the target.
\begin{urbiassert}
51.asFloat == 51;
\end{urbiassert}

\item[asList]
  Bounces to \lstinline|seq|.  It is therefore possible to use the
  various flavors of \lstinline|for|-range loops on integers:
\begin{urbiassert}
{
  var res = [];
  for (var i : 3)
    res << i;
  res
}
== [0, 1, 2];

{
  var res = [];
  for|(var i : 3)
    res << i;
  res
}
== [0, 1, 2];

{
  var res = [];
  for&(var i : 3)
    res << i;
  res.sort
}
== [0, 1, 2];
\end{urbiassert}%>>

\item[asin]
  Arcsine of the target.
\begin{urbiassert}
0.asin == 0;
\end{urbiassert}

\item[asString]
  Return a string representing the target.
\begin{urbiassert}
42.asString == "42";
\end{urbiassert}

\item[atan]
  Return the arctangent of the target.
\begin{urbiassert}
0.atan == 0;
1.atan == Float.pi/4;
\end{urbiassert}

\item \lstinline|'bitand'(\var{that})|\\
  The bitwise-and between \lstinline|this| and \var{that}.
\begin{urbiassert}
(3 bitand 6) == 2;
\end{urbiassert}

\item \lstinline|'bitor'(\var{that})|\\
  Bitwise-or between \lstinline|this| and \var{that}.
\begin{urbiassert}
(3 bitor 6) == 7;
\end{urbiassert}

\item[clone]
  Return a fresh Float with the same value as the target.
\begin{urbiscript}
var x = 0;
[00000000] 0
var y = x.clone;
[00000000] 0
x === y;
[00000000] false
\end{urbiscript}

\item[compl]
  The complement to 1 of the target interpreted as a 32 bits integer.
\begin{urbiassert}
compl 0 == 4294967295;
compl 4294967295 == 0;
\end{urbiassert}

\item[cos]
  Cosine of the target.
\begin{urbiassert}
0.cos == 1;
Float.pi.cos == -1;
\end{urbiassert}

\item \lstinline|each(\var{fun})|\\
  Call the functional argument \var{fun} on every integer from 0 to
  target - 1, sequentially.  The number must be non-negative.
\begin{urbiassert}
{
  var res = [];
  3.each(function (i) { res << 100 + i });
  res
}
== [100, 101, 102];

{
  var res = [];
  for(var x : 3) { res << x; sleep(20ms); res << (100 + x); };
  res
}
== [0, 100, 1, 101, 2, 102];

{
  var res = [];
  0.each (function (i) { res << 100 + i });
  res
}
== [];
\end{urbiassert}

\item \lstinline'each|(\var{fun})'\\
  Call the functional argument \var{fun} on every integer from 0 to
  target - 1, with tight sequentiality.  The number must be
  non-negative.
\begin{urbiassert}
{
  var res = [];
  3.'each|'(function (i) { res << 100 + i });
  res
}
== [100, 101, 102];

{
  var res = [];
  for|(var x : 3) { res << x; sleep(20ms); res << (100 + x); };
  res
}
== [0, 100, 1, 101, 2, 102];
\end{urbiassert}%>>>>>>

\item \lstinline|each&(\var{fun})|\\
  Call the functional argument \var{fun} on every integer from 0 to
  target - 1, concurrently.  The number must be non-negative.
\begin{urbiassert}
{
  var res = [];
  for& (var x : 3) { res << x; sleep(30ms); res << (100 + x) };
  res
}
== [0, 1, 2, 100, 101, 102];
\end{urbiassert}%>>>>

\item[exp]
  Exponential of the target.
\begin{urbiscript}
1.exp;
[00000000] 2.71828
\end{urbiscript}

\item \lstinline|format(\var{finfo})|\\
  Format according to the \refObject{FormatInfo} object \var{finfo}.
  The precision, \lstinline|\var{finfo}.precision|, sets the maximum
  number of digits after decimal point when in fixed or scientific
  mode, and in total when in default mode.  Beware that 0 plays a
  special role, as it is not a ``significant'' digit.

  \begin{windows}
    Under Windows the behavior differs slightly.
  \end{windows}
\begin{urbiassert}
"%1.0d" % 0.1 == "0.1";
"%1.0d" % 1.1 == {if (System.Platform.isWindows) "1.1" else "1"};

"%1.0f" % 0.1 == "0";
"%1.0f" % 1.1 == "1";
\end{urbiassert}

\item[inf]
  Return the infinity.
\begin{urbiscript}
Float.inf;
[00000000] inf
\end{urbiscript}

\item[limit_digits]
  Number of digits (in \lstinline|Float.limit_radix| base) in the
  mantissa.
\begin{urbiassert}
Float.limit_digits;
\end{urbiassert}

\item[limit_digits10]
  Number of digits (in decimal base) that can be represented without
  change.
\begin{urbiassert}
Float.limit_digits10;
\end{urbiassert}

\item[limit_epsilon]
  Machine epsilon (the difference between 1 and the least value
  greater than 1 that is representable).
\begin{urbiassert}
1 != 1 + Float.limit_epsilon;
1 == 1 + Float.limit_epsilon / 2;
\end{urbiassert}

\item[limit_max]
  Maximum finite value.
\begin{urbiassert}
Float.limit_max     != Float.inf;
Float.limit_max * 2 == Float.inf;
\end{urbiassert}

\item[limit_max_exponent]
  Maximum integer value for the exponent that generates a normalized
  floating-point number.
\begin{urbiassert}
Float.inf != Float.limit_radix ** (Float.limit_max_exponent - 1);
Float.inf == Float.limit_radix ** Float.limit_max_exponent;
\end{urbiassert}

\item[limit_max_exponent10]
  Maximum integer value such that 10 raised to that power generates a
  normalized finite floating-point number.
\begin{urbiassert}
Float.inf != 10 ** Float.limit_max_exponent10;
Float.inf == 10 ** (Float.limit_max_exponent10 + 1);
\end{urbiassert}

\item[limit_min]
  Minimum positive normalized value.
\begin{urbiassert}
0 != Float.limit_min;
\end{urbiassert}

\item[limit_min_exponent]
  Minimum negative integer value for the exponent that generates a
  normalized floating-point number.
\begin{urbiassert}
0 != Float.limit_radix ** Float.limit_min_exponent;
\end{urbiassert}

\item[limit_min_exponent10]
  Minimum negative integer value such that 10 raised to that power
  generates a normalized floating-point number.
\begin{urbiassert}
0 != 10 ** Float.limit_min_exponent10;
\end{urbiassert}

\item[limit_radix]
  Base of the exponent of the representation.
\begin{urbiassert}
Float.limit_radix == 2;
\end{urbiassert}

\item[log]
  The logarithm of the target.
\begin{urbiassert}
0.log == -inf;
1.log == 0;
1.exp.log == 1;
\end{urbiassert}

\item \lstinline|max(\var{arg1}, ...)|\\
  Bounces to \lstinline|List.max| on \lstinline|[this, \var{arg1}, ...]|.
\begin{urbiassert}
1.max == 1;
1.max(2, 3) == 3;
3.max(1, 2) == 3;
\end{urbiassert}

\item \lstinline|min(\var{arg1}, ...)|\\
  Bounces to \lstinline|List.min| on \lstinline|[this, \var{arg1}, ...]|.
\begin{urbiassert}
1.min == 1;
1.min(2, 3) == 1;
3.min(1, 2) == 1;
\end{urbiassert}

\item[nan]
  The ``not a number'' special float value.  More precisely, this
  returns the ``quiet NaN'', i.e., it is propagated in the various
  computations, it does not raise exceptions.
\begin{urbiscript}
Float.nan;
[00000000] nan
(Float.nan + Float.nan) / (Float.nan - Float.nan);
[00000000] nan
\end{urbiscript}

A {NaN} has one distinctive property over the other Floats: it is
equal to no other float, not even itself.  This behavior is mandated
by the \wref[IEEE_754-2008]{IEEE 754-2008} standard.
\begin{urbiassert}
{ var n = Float.nan; n === n};
{ var n = Float.nan; n  != n};
\end{urbiassert}

\item[pi]
  $\pi$.
\begin{urbiassert}
Float.pi.cos ** 2 + Float.pi.sin ** 2 == 1;
\end{urbiassert}

\item[random]
  A random integer between 0 (included) and the target (excluded).
\begin{urbiscript}
20.map(function (dummy) { 5.random });
[00000000] [1, 2, 1, 3, 2, 3, 2, 2, 4, 4, 4, 1, 0, 0, 0, 3, 2, 4, 3, 2]
\end{urbiscript}

\item[round]
  The target, rounded to the nearest integer.
\begin{urbiassert}
1.6.round == 2;
1.4.round == 1;
\end{urbiassert}

\item[seq]
  The sequence of integers from 0 to \lstinline|this| - 1 as a list.
  The number must be non-negative.
\begin{urbiassert}
3.seq == [0, 1, 2];
0.seq == [];
(-1).seq;
[00004586:error] !!! seq: expected non-negative integer, got -1
\end{urbiassert}

\item[sign]
  Return 1 if \lstinline|this| is positive, 0 if it is null, -1
  otherwise.
\begin{urbiassert}
(-1164).sign == -1;
0.sign       == 0;
(1164).sign  == 1;
\end{urbiassert}

\item[sin]
  The sine of the target.
\begin{urbiassert}
0.sin == 0;
\end{urbiassert}

\item[sqr]
  Square of the target.
\begin{urbiassert}
32.sqr == 1024;
32.sqr == 32 ** 2;
\end{urbiassert}

\item[sqrt]
  The square root of the target.
\begin{urbiassert}
1024.sqrt == 32;
1024.sqrt == 1024 ** 0.5;
\end{urbiassert}

\item[srandom]
  Initialized the seed used by the random function.  As opposed to common
  usage, you should not use
\begin{urbiunchecked}
{
  var now = Date.now.timestamp;
  now.srandom;
  var list1 = 20.map(function (dummy) { 5.random });
  now.srandom;
  var list2 = 20.map(function (dummy) { 5.random });
  assert
  {
    list1 == list2;
  }
};
\end{urbiunchecked}

\item[tan]
  Tangent of the target.
\begin{urbiscript}
assert(0.tan == 0);
(Float.pi/4).tan;
[00000000] 1
\end{urbiscript}

\item \lstinline|times(\var{fun})|\\
  Call the functional argument \var{fun} \lstinline|this| times.

\begin{urbiscript}
3.times(function () { echo("ping")});
[00000000] *** ping
[00000000] *** ping
[00000000] *** ping
\end{urbiscript}

\item[trunc]
  Return the target truncated.
\begin{urbiassert}
1.9.trunc == 1;
(-1.9).trunc == -1;
\end{urbiassert}

\item \lstinline|'^'(\var{that})|\\
  Bitwise exclusive or between \lstinline|this| and \var{that}.
\begin{urbiassert}
(3 ^ 6) == 5;
\end{urbiassert}

\item \lstinline|'>>'(\var{that})|\\%>>
  \lstinline|this| shifted by \var{that} bits towards the right.
\begin{urbiassert}
4 >> 2 == 1;
\end{urbiassert}

\item \lstinline|'<'(\var{that})|\\
  Whether \lstinline|this| is less than \var{b}. The other comparison
  operators (\lstinline|<=|, \lstinline|>|, \ldots) can thus also be
  applied on floats since Float inherits \refObject{Orderable}.
\begin{urbiassert}
  0 < 1;
!(1 < 0);
\end{urbiassert}

\item \lstinline|'<<'(\var{that})|\\
  \lstinline|this| shifted by \var{that} bit towards the left.
\begin{urbiassert}
4 << 2 == 16;
\end{urbiassert}

\item \lstinline|'-'(\var{that})|\\
  \lstinline|this| subtracted by \var{b}.
\begin{urbiassert}
6 - 3 == 3;
\end{urbiassert}

\item \lstinline|'+'(\var{that})|\\
  The sum of \lstinline|this| and \var{that}.
\begin{urbiassert}
1 + 1 == 2;
\end{urbiassert}

\item \lstinline|'/'(\var{that})|\\
  The quotient of \lstinline|this| divided by \var{that}.
\begin{urbiassert}
50 / 10 == 5;
10 / 50 == 0.2;
\end{urbiassert}

\item \lstinline|'%'(\var{that})|\\
  \lstinline|this| modulo \var{b}.
\begin{urbiassert}
50 % 11 == 6;
\end{urbiassert}

\item \lstinline|'*'(\var{that})|\\
  Product of \lstinline|this| by \var{that}.
\begin{urbiassert}
2 * 3 == 6;
\end{urbiassert}

\item \lstinline|'**'(\var{that})|\\
  \lstinline|this| to the \var{that} power (${this}^{that}$).
\begin{urbiassert}
2 ** 10 == 1024;
\end{urbiassert}

\item \lstinline|'=='(\var{that})|\\
  Whether \lstinline|this| equals \var{that}.
\begin{urbiassert}
  1 == 1;
!(1 == 2);
\end{urbiassert}
\end{urbiscriptapi}

%%% Local Variables:
%%% mode: latex
%%% TeX-master: "../urbi-sdk"
%%% ispell-dictionary: "american"
%%% ispell-personal-dictionary: "../urbi.dict"
%%% End:

%% Copyright (C) 2009-2010, Gostai S.A.S.
%%
%% This software is provided "as is" without warranty of any kind,
%% either expressed or implied, including but not limited to the
%% implied warranties of fitness for a particular purpose.
%%
%% See the LICENSE file for more information.

\section{Formatter}

A \dfn{formatter} stores format information of a format string like
used in \code{printf} in the C library or in \code{boost::format}.

\subsection{Prototypes}

\begin{refObjects}
\item[Object]
\end{refObjects}

\subsection{Construction}

Formatters are created with the format string. It cuts the string to
separate regular parts of string and formatting patterns, and stores
them.

\begin{urbiscript}[firstnumber=1]
Formatter.new("Name:%s, Surname:%s;");
[00000001] Formatter ["Name:", %s, ", Surname:", %s, ";"]
\end{urbiscript}

Actually, formatting patterns are translated into
\refObject{FormatInfo}.

\subsection{Slots}

\begin{urbiscriptapi}
\item[asList]
  Return the content of the \dfn{formatter} as a list of strings and
  \refObject{FormatInfo}.
\begin{urbiassert}
Formatter.new("Name:%s, Surname:%s;").asList.asString
       == "[\"Name:\", %s, \", Surname:\", %s, \";\"]";
\end{urbiassert}

-\item \lstinline|'%'(\var{args})|\\
  Use \this as format string and \var{args} as the list of
  arguments, and return the result (a \refObject{String}).  The arity
  of the Formatter (i.e., the number of expected arguments) and the
  size of \var{args} must match exactly.

  This operator concatenates regular strings and the strings that are
  result of \lstinline|asString| called on members of \var{args} with
  the appropriate \refObject{FormatInfo}.
\begin{urbiassert}
Formatter.new("Name:%s, Surname:%s;") % ["Foo", "Bar"]
       == "Name:Foo, Surname:Bar;";
\end{urbiassert}

  If \var{args} is not a \refObject{List}, then the call is equivalent
  to calling \lstinline|'%'([\var{args}])|.
\begin{urbiassert}
Formatter.new("%06.3f") % Math.pi
       == "03.142";
\end{urbiassert}

  Note that \lstinline|String.'%'| provides a nicer interface to this
  operator:
\begin{urbiassert}
"%06.3f" % Math.pi == "03.142";
\end{urbiassert}

  It is nevertheless interesting to use the Formatter for performance
  reasons if the format is reused many times.
\begin{urbiscript}
{
  // Some large database of people.
  var people =
    [["Foo", "Bar" ],
     ["One", "Two" ],
     ["Un",  "Deux"],];
  var f = Formatter.new("Name:%7s, Surname:%7s;");
  for (var p: people)
    echo (f % p);
};
[00031939] *** Name:    Foo, Surname:    Bar;
[00031940] *** Name:    One, Surname:    Two;
[00031941] *** Name:     Un, Surname:   Deux;
\end{urbiscript}
\end{urbiscriptapi}

%%% Local Variables:
%%% coding: utf-8
%%% mode: latex
%%% TeX-master: "../urbi-sdk"
%%% ispell-dictionary: "american"
%%% ispell-personal-dictionary: "../urbi.dict"
%%% fill-column: 76
%%% End:

%% Copyright (C) 2009-2011, Gostai S.A.S.
%%
%% This software is provided "as is" without warranty of any kind,
%% either expressed or implied, including but not limited to the
%% implied warranties of fitness for a particular purpose.
%%
%% See the LICENSE file for more information.

\section{FormatInfo}

A \dfn{format info} is used when formatting a la \code{printf}. It
store the formatting pattern itself and all the format information it
can extract from the pattern.

\subsection{Prototypes}

\begin{refObjects}
\item[Object]
\end{refObjects}

\subsection{Construction}

The constructor expects a string as argument, whose syntax is similar
to \code{printf}'s.  It is detailed below.

\begin{urbiscript}[firstnumber=1]
var f = FormatInfo.new("%+2.3d");
[00000001] %+2.3d
\end{urbiscript}

A formatting pattern must one of the following (brackets denote
optional arguments):
\begin{itemize}
\item \verb&%&\var{options} \var{spec}
\item \verb&%|&\var{options}[\var{spec}]\verb&|&
\end{itemize}

\noindent
\var{options} is a sequence of 0 or several of the following
characters:

\begin{center}
  \begin{tabular}{|c|l|}
    \hline
    \samp{-} & Left alignment.\\
    \samp{=} & Centered alignment.\\
    \samp{+} & Show sign even for positive number.\\
    \samp{ } & If the string does not begin with \samp{+} or \samp{-}, insert
    a space before the converted string.\\
    \samp{0} & Pad with 0's (inserted after sign or base indicator).\\
    \samp{\#} & Show numerical base, and decimal point.\\
    % \samp{'} & Split thousands (\samp{1 000}).\\
    \hline
  \end{tabular}
\end{center}

\noindent
\var{spec} is the conversion character and must be one of the
following:

\begin{center}
  \begin{tabular}{|c|l|}
    \hline
    \samp{s} & Default character, prints normally\\
    \samp{d} & Case modifier: lowercase \\
    \samp{D} & Case modifier: uppercase \\
    \samp{x} & Prints in hexadecimal lowercase \\
    \samp{X} & Prints in hexadecimal uppercase \\
    \samp{o} & Prints in octal\\
    % \samp{b} & Prints in binary\\
    \samp{e} & Prints floats in scientific format\\
    \samp{E} & Prints floats in scientific format uppercase\\
    \samp{f} & Prints floats in fixed format\\
    \hline
  \end{tabular}
\end{center}

When accepted, the format string is decoded, and its features are made
available as separate slots of the FormatInfo object.

\begin{urbiscript}
do (FormatInfo.new("%+'=#06.12X"))
{
  assert
  {
    width == 6;
    precision == 12;
    alt == true;
    prefix == "+";
    alignment == 0;
    group == " ";
    pad == "0";
    uppercase == 1;
    spec == "x";
  };
}|;
\end{urbiscript}

Formats that do not conform raise errors.

\begin{urbiscript}
FormatInfo.new("fgh");
[00000001:error] !!! new: format: pattern "fgh" doesn't begin with %

FormatInfo.new("%mgh");
[00000002:error] !!! new: format: "m" is not a valid conversion type character

FormatInfo.new("%");
[00000003:error] !!! new: format: trailing `%'
\end{urbiscript}



\subsection{Slots}
\begin{urbiscriptapi}
\item[alignment]
  Requested alignment: \lstinline|-1| for left, \lstinline|0| for
  centered, \lstinline|1| for right (default).
\begin{urbiassert}
FormatInfo.new("%s").alignment == 1;
FormatInfo.new("%=s").alignment == 0;
FormatInfo.new("%-s").alignment == -1;
\end{urbiassert}

\item[alt]
  Whether the ``alternative'' display is requested (\samp{\#}).
\begin{urbiassert}
FormatInfo.new("%s").alt == false;
FormatInfo.new("%#s").alt == true;
\end{urbiassert}

\item[group]
  Separator to use for thousands.  Corresponds to the \samp{'}
  \var{option}.
\begin{urbiassert}
FormatInfo.new("%s").group == "";
FormatInfo.new("%'s").group == " ";
\end{urbiassert}

\item[pad]
  The padding character to use for alignment requests.  Defaults to space.
\begin{urbiassert}
FormatInfo.new("%s").pad == " ";
FormatInfo.new("%0s").pad == "0";
\end{urbiassert}

\item[pattern]
  The pattern given to the constructor.
\begin{urbiassert}
FormatInfo.new("%#'12.8s").pattern == "%#'12.8s";
\end{urbiassert}

\item[precision]
  When formatting a \refObject{Float}, the maximum number of digits
  after decimal point when in fixed or scientific mode, and in total
  when in default mode.  When formatting other objects with spec-char
  \samp{s}, the conversion string is truncated to the precision first
  chars. The eventual padding to \lstinline|width| is done after
  truncation.
\begin{urbiassert}
FormatInfo.new("%s").precision == 6;
FormatInfo.new("%23.3s").precision == 3;
\end{urbiassert}

\item[prefix]
  The string to display before positive numbers.  Defaults to empty.
\begin{urbiassert}
FormatInfo.new("%s").prefix == "";
FormatInfo.new("% s").prefix == " ";
FormatInfo.new("%+s").prefix == "+";
\end{urbiassert}

\item[spec]
  The specification character, regardless of the case conversion
  requests.
\begin{urbiassert}
FormatInfo.new("%s").spec == "s";
FormatInfo.new("%23.3s").spec == "s";
FormatInfo.new("%'X").spec == "x";
\end{urbiassert}

\item[uppercase]
  Case conversion: \lstinline|-1| for lower case, \lstinline|0| for no
  conversion (default), \lstinline|1| for conversion to uppercase.
  The value depends on the case of specification character, except for
  \samp{\%s} which corresponds to \lstinline|0|.
\begin{urbiassert}
FormatInfo.new("%s").uppercase == 0;
FormatInfo.new("%d").uppercase == -1;
FormatInfo.new("%D").uppercase == 1;
FormatInfo.new("%x").uppercase == -1;
FormatInfo.new("%X").uppercase == 1;
FormatInfo.new("%|D|").uppercase ==  1;
FormatInfo.new("%|d|").uppercase == -1;
\end{urbiassert}

\item[width]
  Width requested for alignment.
\begin{urbiassert}
FormatInfo.new("%s").width == 0;
FormatInfo.new("%10s").width == 10;
\end{urbiassert}
\end{urbiscriptapi}

%%% Local Variables:
%%% mode: latex
%%% TeX-master: "../urbi-sdk"
%%% ispell-dictionary: "american"
%%% ispell-personal-dictionary: "../urbi.dict"
%%% fill-column: 76
%%% End:

%% Copyright (C) 2009-2010, Gostai S.A.S.
%%
%% This software is provided "as is" without warranty of any kind,
%% either expressed or implied, including but not limited to the
%% implied warranties of fitness for a particular purpose.
%%
%% See the LICENSE file for more information.

\section{Global}

\dfn{Global} is designed for the purpose of being global
namespace. Since \dfn{Global} is a prototype of \dfn{Object} and all
objects are an \dfn{Object}, all slots of \dfn{Global} are accessible from
anywhere.

\subsection{Prototypes}
\begin{itemize}
\item \refSlot{uobjects}, see below.
\item \refSlot[Tag]{tags} (see \refObject{Tag})
\item \refObject{Math}
\item \refObject{System}
\item \refObject{Object}
\end{itemize}

\subsection{Slots}
\begin{urbiscriptapi}
\item[Barrier] See \refObject{Barrier}.
\item[Binary] See \refObject{Binary}.
\item[CallMessage] See \refObject{CallMessage}.

\item[cerr] A predefined stream for error messages.  Strings sent to
  it are not escaped, contrary to regular streams (see
  \lstinline|output| for instance).
\begin{urbiscript}
cerr << "Message";
[00015895:error] Message
cerr << "\"quote\"";
[00015895:error] "quote"
\end{urbiscript}

\item[Channel] See \refObject{Channel}.

\item[clog] A predefined stream for log messages.  Strings are output
  escaped.
\begin{urbiscript}
clog << "Message";
[00015895:clog] "Message"
\end{urbiscript}

\item[Code] See \refObject{Code}.
\item[Comparable] See \refObject{Comparable}.

\item[cout] A predefined stream for output messages.  Strings are
  output escaped.
\begin{urbiscript}
cout << "Message";
[00015895:output] "Message"
cout << "\"quote\"";
[00015895:output] "\"quote\""
\end{urbiscript}

\item[Date] See \refObject{Date}.

\item[detach](<exp>)%
  Bounce to \refSlot[Control]{detach}, see \refObject{Control}.

\item[Dictionary] See \refObject{Dictionary}.
\item[Directory] See \refObject{Directory}.

\item[disown](<exp>)%
  Bounce to \refSlot[Control]{disown}, see \refObject{Control}.

\item[Duration] See \refObject{Duration}.

\item[echo](<value>, <channel> = "")%
  Bounce to \lstinline|lobby.echo|, see \refSlot[Lobby]{echo}.
\begin{urbiscript}
echo("111", "foo");
[00015895:foo] *** 111
echo(222, "");
[00051909] *** 222
echo(333);
[00055205] *** 333
\end{urbiscript}

\item[evaluate] This \refObject{UVar} provides a synchronous interface
  to the \urbi engine: write to it to ``send'' an expression to
  compute it, and ``read'' it to get the result.  This UVar is
  designed to be used from the \Cxx; it makes little sense in \us, use
  \refSlot[System]{eval} instead, if it is really required (see
  \autoref{sec:k122:dollar}).  Since the semantics of the assignment
  requires that it evaluates to the right-hand side argument, reading
  \lstinline|evaluate| after the assignment is needed, which makes
  race conditions likely.  To avoid this, use \lstinline{|} (or better
  yet, do not use \refSlot{evaluate} at all in \us).

\begin{urbiassert}
(evaluate = "1+2;") == "1+2;";
 evaluate == 3;
{ evaluate = "1+2;" | evaluate } == 3;
{ evaluate = "var x = 1;"; x } == 1;
\end{urbiassert}

  Errors raise an exception.

\begin{urbiscript}
evaluate = "1/0;";
[00087671:error] !!! Exception caught while processing notify on Global.evaluate:
[00087671:error] !!! 1.1-3: /: division by 0
[00087671:error] !!!    called from: updateSlot
[00087671] "1/0;"
\end{urbiscript}

\item[Event] See \refObject{Event}.
\item[Exception] See \refObject{Exception}.
\item[Executable] See \refObject{Executable}.
\item[external] An system object used to implement UObject support in
  \us.
\item[false]  See \autoref{sec:truth}.
\item[File] See \refObject{File}.
\item[Finalizable] See \refObject{Finalizable}.
\item[Float] See \refObject{Float}.
\item[FormatInfo] See \refObject{FormatInfo}.
\item[Formatter] See \refObject{Formatter}.

\item[getProperty](<slotName>, <propName>)%
  This wrapper around \refSlot[Object]{getProperty} is actually a by-product
  of the existence of the \refSlot{evaluate} \refObject{UVar}.

\item[Global] See \refObject{Global}.
\item[Group] See \refObject{Group}.
\item[InputStream] See \refObject{InputStream}.
%% FIXME: [00008768] ***   installUpdateHookStack : Code
\item[isdef](<qualifiedIdentifier>)%
  Whether the \var{qualifiedIdentifier} is defined.  It features some
  (fragile) magic to support an argument passed as a literal
  (\lstinline|isdef(foo)|), not a string (\lstinline|isdef("foo")|).  It is
  not recommended to use this feature, which is provided for \us
  compatibility.  See \refSlot[Object]{hasLocalSlot} and
  \refSlot[Object]{hasSlot} for safer alternatives.
\begin{urbiscript}
assert
{
  !isdef(a);
  !isdef(a.b);
  !isdef(a.b.c);
};

var a = Object.new|;
assert
{
   isdef(a);
  !isdef(a.b);
  !isdef(a.b.c);
};

var a.b = Object.new|;
assert
{
   isdef(a);
   isdef(a.b);
  !isdef(a.b.c);
};

var a.b.c = Object.new|;
assert
{
   isdef(a);
   isdef(a.b);
   isdef(a.b.c);
};
\end{urbiscript}


\item[Job] See \refObject{Job}.
\item[Kernel1] See \refObject{Kernel1}.
\item[Lazy] See \refObject{Lazy}.
\item[List] See \refObject{List}.
\item[Loadable] See \refObject{Loadable}.
\item[Lobby] See \refObject{Lobby}.
\item[Math] See \refObject{Math}.
\item[methodToFunction](<name>)%
  Create a function from the method \var{name} so that calling the
  function which arguments \lstinline|(\var{a}, \var{b}, ...)| is that
  same as calling \lstinline|\var{a}.\var{name}(\var{b}, ...)|.
\begin{urbiscript}
var uid_of = methodToFunction("uid")|;
assert
{
  uid_of(Object) == Object.uid;
  uid_of(Global) == Global.uid;
};
var '+_of' = methodToFunction("+")|;
assert
{
  '+_of'( 1,   2)  ==  1  + 2;
  '+_of'("1", "2") == "1" + "2";
  '+_of'([1], [2]) == [1] + [2];
};
\end{urbiscript}

\item[Mutex] See \refObject{Mutex}.
\item[nil] See \refObject{nil}.
\item[Object] See \refObject{Object}.
\item[Orderable] See \refObject{Orderable}.
\item[OutputStream] See \refObject{OutputStream}.
\item[Pair] See \refObject{Pair}.
\item[Path] See \refObject{Path}.
\item[Pattern] See \refObject{Pattern}.

\item[persist](<exp>)%
  Bounce to \refSlot[Control]{persist}, see \refObject{Control}.

\item[Position] See \refObject{Position}.
\item[Primitive] See \refObject{Primitive}.
\item[Process] See \refObject{Process}.
\item[Profiling] See \refObject{Profiling}.
\item[PseudoLazy] See \refObject{PseudoLazy}.
\item[PubSub] See \refObject{PubSub}.
\item[RangeIterable] See \refObject{RangeIterable}.
\item[Regexp] See \refObject{Regexp}.
\item[Semaphore] See \refObject{Semaphore}.
\item[Server] See \refObject{Server}.
\item[Singleton] See \refObject{Singleton}.
\item[Socket] See \refObject{Socket}.
\item[String] See \refObject{String}.
\item[System] See \refObject{System}.
\item[Tag] See \refObject{Tag}.
\item[Timeout] See \refObject{Timeout}.
\item[TrajectoryGenerator] See \refObject{TrajectoryGenerator}.
\item[Triplet] See \refObject{Triplet}.
\item[true]  See \autoref{sec:truth}.
\item[Tuple] See \refObject{Tuple}.
\item[UObject] See \refObject{UObject}.
\item[uobjects] An object whose slots are all the \refObject{UObject}
  bound into the system.

%% FIXME: [00008801] ***   uobjects_handle : Finalizable
%% FIXME: [00008802] ***   UpdateHookStack : Code
\item[UValue] See \refObject{UValue}.

\item[UVar] See \refObject{UVar}.
\item[void] See \refObject{void}.
\item[wall](<value>, <channel> = "")%
  Bounce to \lstinline|lobby.wall|, see \refObject{Lobby}.
\begin{urbiscript}
wall("111", "foo");
[00015895:foo] *** 111
wall(222, "");
[00051909] *** 222
wall(333);
[00055205] *** 333
\end{urbiscript}

\item[warn](<message>)%
  Issue \var{message} on \refSlot[Channel]{warning}.
\begin{urbiscript}
warn("cave canem");
[00015895:warning] *** cave canem
\end{urbiscript}

%% FIXME: [00008806] ***   WeakDictionary : Finalizable
%% FIXME: [00008806] ***   WeakPointer : Finalizable
\end{urbiscriptapi}

%%% Local Variables:
%%% mode: latex
%%% TeX-master: "../urbi-sdk"
%%% ispell-dictionary: "american"
%%% ispell-personal-dictionary: "../urbi.dict"
%%% fill-column: 76
%%% End:

%% Copyright (C) 2009-2011, Gostai S.A.S.
%%
%% This software is provided "as is" without warranty of any kind,
%% either expressed or implied, including but not limited to the
%% implied warranties of fitness for a particular purpose.
%%
%% See the LICENSE file for more information.

\section{Group}
A transparent means to send messages to several objects as if they
were one.

\subsection{Example}

The following session demonstrates the features of the Group
objects.  It first creates the \lstinline|Sample| family of object,
makes a group of such object, and uses that group.

\begin{urbiscript}[firstnumber=1]
class Sample
{
  var value = 0;
  function init(v)    { value = v; };
  function asString() { "<" + value.asString + ">"; };
  function timesTen() { new(value * 10); };
  function plusTwo()  { new(value + 2); };
};
[00000000] <0>

var group = Group.new(Sample.new(1), Sample.new(2));
[00000000] Group [<1>, <2>]
group << Sample.new(3);
[00000000] Group [<1>, <2>, <3>]
group.timesTen.plusTwo;
[00000000] Group [<12>, <22>, <32>]

// Bouncing getSlot and updateSlot.
group.value;
[00000000] Group [1, 2, 3]
group.value = 10;
[00000000] Group [10, 10, 10]

// Bouncing to each&.
var sum = 0|
for& (var v : group)
  sum += v.value;
sum;
[00000000] 30
\end{urbiscript}

\subsection{Prototypes}

\begin{refObjects}
\item[RangeIterable]
\item[Comparable]
\end{refObjects}

\subsection{Construction}

Groups are created like any other object. The constructor can take members
to add to the group.

\begin{urbiscript}
Group.new;
[00000000] Group []
Group.new(1, "two");
[00000000] Group [1, "two"]
\end{urbiscript}

\subsection{Slots}

\begin{urbiscriptapi}
\item['<<'](<member>)%
  Syntactic sugar for \refSlot{add}.


\item['=='](<that>)%
  Whether \lstinline|this.members == \var{that}.members|.
\begin{urbiassert}
               Group.new == Group.new;
Group.new(1, [2], "foo") == Group.new(1, [2], "foo");
         Group.new(1, 2) != Group.new(2, 1);
            Group.new(1) != Group.new(2);
\end{urbiassert}


\item[add](<member>, ...)%
  Add members to \this group, and return \this.
\begin{urbiassert}
var g = Group.new(1, 2);
g.add(3, 4) === g;
g.members == [1, 2, 3, 4];
\end{urbiassert}


\item[asString]
  Report the \lstinline|asString| of the members.
\begin{urbiassert}
Group.new(1, 2).asString == "Group [1, 2]";
\end{urbiassert}


\item[each](<action>)%
  Apply \var{action} to all the members, in sequence, then return the
  Group of the results, in the same order.  Allows to iterate over a
  Group via \lstinline|for|.


\item['each&'](<action>)%
  Apply \var{action} to all the members, concurrently, then return the
  Group of the results.  The order is \emph{not} necessarily the same.
  Allows to iterate over a Group via \lstinline|for&|.


\item[fallback]
  This function is called when a method call on \this
  failed.  It bounces the call to the members of the group, collects
  the results returned as a group.  This allows to chain grouped
  operation in a row.  If the dispatched calls return
  \lstinline|void|, returns a single \lstinline|void|, not a ``group
  of \lstinline|void|''.


\item[getProperty](<slot>, <prop>)%
  Bounced to the members so that \lstinline|this.\var{slot}->\var{prop}|
  actually collects the values of the property \var{prop} of the slots
  \var{slot} of the group members.  See \refSlot[Object]{getProperty}.
\begin{urbiscript}
class C
{
  var val = 0;
}|;

var a = C.new|; var b = C.new|;
var g = Group.new << a << b|;

g.val->prop;
[00010640:error] !!! property lookup failed: val->prop

a.val->prop = 42|;
g.val->prop;
[00010640:error] !!! property lookup failed: val->prop

b.val->prop = 51|;
assert
{
  g.val->prop == Group.new(42, 51);
};
\end{urbiscript}
\begin{urbicomment}
  removeSlots("C", "a", "b", "g");
\end{urbicomment}


\item[hasProperty](<slot>, <prop>)%
  Bounce to the members and return a group for the results.  See
  \refSlot[Object]{hasProperty}.
\begin{urbiscript}
class C
{
  var val = 0;
}|;

var a = C.new|; var b = C.new|;
var g = Group.new << a << b|;

assert
{
  g.hasProperty("val", "prop") == Group.new(false, false);
  a.val->prop = 21;
  g.hasProperty("val", "prop") == Group.new(true, false);
  b.val->prop = 42;
  g.hasProperty("val", "prop") == Group.new(true, true);
};
\end{urbiscript}
\begin{urbicomment}
  removeSlots("C", "a", "b", "g");
\end{urbicomment}


\item[hasSlot](<name>)%
  True if and only if all the members have the slot.

\begin{urbiassert}
var g = Group.new(1, 2);

!g.hasSlot("foo");
 g.hasSlot("+");
 g + 1 == Group.new(2, 3);
\end{urbiassert}


\item[remove](<member>, ...)%
  Remove members from \this group, and return \this.  Non-existing members
  are silently ignored.
\begin{urbiassert}
var g = Group.new(1, 2, 1);
g.remove(1, 3) === g == Group.new(2);
g.remove(2)    === g == Group.new;
\end{urbiassert}


\item[setProperty](<slot>, <prop>, <value>)%
  Bounced to the members so that
  \lstinline|this.\var{slot}->\var{prop} = \var{value}| actually updates the
  value of the property \var{prop} in the slots \var{slot} of the group
  members, and return a group for the collected result.  See
  \refSlot[Object]{setProperty}.

\begin{urbiscript}
class C
{
  var val = 0;
}|;

var g = Group.new << C.new << C.new|;

assert
{
  (g.val->prop = 31) == Group.new(31, 31);
};
\end{urbiscript}
\begin{urbicomment}
  removeSlots("C", "g");
\end{urbicomment}


\item[updateSlot](<name>, <value>)%
  Bounced to the members so that
  \lstinline|this.\var{name} = \var{value}|
  actually updates the value of the slot \var{name} in
  the group members.
\end{urbiscriptapi}

%%% Local Variables:
%%% mode: latex
%%% TeX-master: "../urbi-sdk"
%%% ispell-dictionary: "american"
%%% ispell-personal-dictionary: "../urbi.dict"
%%% fill-column: 76
%%% End:

%% Copyright (C) 2009-2010, Gostai S.A.S.
%%
%% This software is provided "as is" without warranty of any kind,
%% either expressed or implied, including but not limited to the
%% implied warranties of fitness for a particular purpose.
%%
%% See the LICENSE file for more information.

\section{InputStream}

InputStreams are used to read (possibly binary) files by hand.
\refObject{File} provides means to swallow a whole file either as a
single large string, or a list of lines.  \lstinline|InputStream|
provides a more fine-grained interface to read files.

\subsection{Prototypes}
\begin{refObjects}
\item[Object]
\end{refObjects}

\begin{windows}
  Beware that because of limitations in the current implementation,
  one cannot safely read from two different files at the same time
  under Windows.
\end{windows}

\subsection{Construction}

An InputStream is a reading-interface to a file, so its constructor
requires a \refObject{File}.

\begin{urbiscript}[firstnumber=1]
File.save("file.txt", "1\n2\n");
var is = InputStream.new(File.new("file.txt"));
[00000001] InputStream_0x827000
\end{urbiscript}

Bear in mind that open streams should be closed
(\autoref{sec:specs:output-stream:ctor}).

\begin{urbiscript}
is.close;
\end{urbiscript}

\subsection{Slots}

\begin{urbiscriptapi}
\item[close] Close the stream, return void.  Raise an error if the file is
  closed.
\begin{urbiscript}
{
  var i = InputStream.new(File.create("file.txt"));
  assert(i.close.isVoid);
  i.close;
};
[00000001:error] !!! close: stream is closed
\end{urbiscript}

\item[get]%
  Get the next available byte as a \refObject{Float}, or \lstinline|void| if
  the end of file was reached.  Raise an error if the file is closed.
\begin{urbiscript}
{
  File.save("file.txt", "1\n2\n");
  var i = InputStream.new(File.new("file.txt"));
  var x;
  while (!(x = i.get.acceptVoid).isVoid)
    cout << x;
  i.close;
  i.get;
};
[00000001:output] 49
[00000002:output] 10
[00000003:output] 50
[00000004:output] 10
[00000005:error] !!! get: stream is closed
\end{urbiscript}

\item[getChar]%
  Get the next available byte as a \refObject{String}, or \lstinline|void|
  if the end of file was reached.  Raise an error if the file is closed.
\begin{urbiscript}
{
  File.save("file.txt", "1\n2\n");
  var i = InputStream.new(File.new("file.txt"));
  var x;
  while (!(x = i.getChar.acceptVoid).isVoid)
    cout << x;
  i.close;
  i.getChar;
};
[00000001:output] "1"
[00000002:output] "\n"
[00000003:output] "2"
[00000004:output] "\n"
[00000005:error] !!! getChar: stream is closed
\end{urbiscript}

\item[getLine]%
  Get the next available line as a \refObject{String}, or \lstinline|void|
  if the end of file was reached.  The end-of-line characters are trimmed.
  Raise an error if the file is closed.
\begin{urbiscript}
{
  File.save("file.txt", "1\n2\n");
  var i = InputStream.new(File.new("file.txt"));
  var x;
  while (!(x = i.getLine.acceptVoid).isVoid)
    cout << x;
  i.close;
  i.getLine;
};
[00000001:output] "1"
[00000002:output] "2"
[00000005:error] !!! getLine: stream is closed
\end{urbiscript}
\end{urbiscriptapi}


%%% Local Variables:
%%% mode: latex
%%% TeX-master: "../urbi-sdk"
%%% ispell-dictionary: "american"
%%% ispell-personal-dictionary: "../urbi.dict"
%%% fill-column: 76
%%% End:

\section{Kernel1}

This object plays the role of a name-space in which obsolete functions
from \us 1.0 are provided for backward compatibility.  Do not use
these functions, scheduled for removal.

\subsection{Prototypes}
\begin{itemize}
\item \refObject{Singleton}
\end{itemize}

\subsection{Construction}

Since it is a \refObject{Singleton}, you are not expected to build
other instances.

\subsection{Slots}

\begin{itemize}
\item \lstinline|commands|\\
  Ignored for backward compatibility.

\item \lstinline|connections|\\
  Ignored for backward compatibility.

\item \lstinline|copy(\var{binary})|\\
  Obsolete syntax for \lstinline|\var{binary}.copy|, see
  \refObject{Binary}.
\begin{urbiscript}
// copy.
var a = BIN 10;0123456789
[00000001] BIN 10
[:]0123456789

var b = Kernel1.copy(a);
[00000003:warning] *** `copy(binary)' is deprecated, use `binary.copy'
[00000004] BIN 10
[:]0123456789

echo (b);
[00000005] *** BIN 10
[:]0123456789
\end{urbiscript}

\item \lstinline|devices|\\
  Ignored for backward compatibility.

\item \lstinline|events|\\
  Ignored for backward compatibility.

\item \lstinline|functions|\\
  Ignored for backward compatibility.

\item \lstinline|isvoid(\var{obj})|\\
  Obsolete syntax for \lstinline|\var{obj}.isVoid|, see
  \refObject{Object}.

\item \lstinline|noop|\\
  Do nothing.  Use \lstinline|{}| instead.

\item \lstinline|ping|\\
  Return \lstinline|time| verbosely, see \refObject{System}.
\begin{urbiscript}[firstnumber=last]
Kernel1.ping;
[00000421] *** pong time=0.12s
\end{urbiscript}

\item \lstinline|reset|\\
  Ignored for backward compatibility.

\item \lstinline|runningcommands|\\
  Ignored for backward compatibility.

\item \lstinline|seq(\var{number})|\\
  Obsolete syntax for \lstinline|\var{number}.seq|, see
  \refObject{Float}.

\item \lstinline|size(\var{list})|\\
  Obsolete syntax for \lstinline|\var{list}.size|, see
  \refObject{List}.
\begin{urbiassert}[firstnumber=last]
Kernel1.size([1, 2, 3]) == [1, 2, 3].size;
[00000002:warning] *** `size(list)' is deprecated, use `list.size'
\end{urbiassert}

\item \lstinline|strict|\\
  Ignored for backward compatibility.

\item \lstinline|strlen(\var{string})|\\
  Obsolete syntax for \lstinline|\var{string}.length|, see
  \refObject{String}.
\begin{urbiassert}[firstnumber=last]
Kernel1.strlen("123") == "123".length;
[00000002:warning] *** `strlen(string)' is deprecated, use `string.length'
\end{urbiassert}

\item \lstinline|taglist|\\
  Ignored for backward compatibility.

\item \lstinline|undefall|\\
  Ignored for backward compatibility.

\item \lstinline|unstrict|\\
  Ignored for backward compatibility.

\item \lstinline|uservars|\\
  Ignored for backward compatibility.

\item \lstinline|vars|\\
  Ignored for backward compatibility.
\end{itemize}


%%% Local Variables:
%%% mode: latex
%%% TeX-master: "../urbi-sdk"
%%% ispell-personal-dictionary: "../urbi.dict"
%%% End:

%% Copyright (C) 2009-2011, Gostai S.A.S.
%%
%% This software is provided "as is" without warranty of any kind,
%% either expressed or implied, including but not limited to the
%% implied warranties of fitness for a particular purpose.
%%
%% See the LICENSE file for more information.

\section{Lazy}

\dfn{Lazies} are objects that hold a lazy value, that is, a not yet
evaluated value. They provide facilities to evaluate their content only once
(\dfn{memoization}) or several times. Lazy are essentially used in call
messages, to represent lazy arguments, as described in
\autorefObject{CallMessage}.

\subsection{Examples}

\subsubsection{Evaluating once}

One usage of lazy values is to avoid evaluating an expression unless it's
actually needed, because it's expensive or has undesired side effects. The
listing below presents a situation where an expensive-to-compute value
(\lstinline|heavy_computation|) might be needed zero, one or two times. The
objective is to save time by:

\begin{itemize}
\item Not evaluating it if it's not needed.
\item Evaluating it only once if it's needed once or twice.
\end{itemize}

We thus make the wanted expression lazy, and use the \lstinline|value|
method to fetch its value when needed.

\begin{urbiscript}[firstnumber=1]
// This function supposedly performs expensive computations.
function heavy_computation()
{
  echo("Heavy computation");
  return 1 + 1;
}|;

// We want to do the heavy computations only if needed,
// and make it a lazy value to be able to evaluate it "on demand".
var v = Lazy.new(closure () { heavy_computation() });
[00000000] heavy_computation()
/* some code */;
// So far, the value was not needed, and heavy_computation
// was not evaluated.
/* some code */;
// If the value is needed, heavy_computation is evaluated.
v.value();
[00000000] *** Heavy computation
[00000000] 2
// If the value is needed a second time, heavy_computation
// is not reevaluated.
v.value();
[00000000] 2
\end{urbiscript}

\subsubsection{Evaluating several times}

Evaluating a lazy several times only makes sense with lazy arguments and
call messages. See example with call messages in
\autoref{sec:std-callmsg-examples-several}.


\subsection{Caching}

\refObject{Lazy} is meant for functions without argument.  If you need
\dfn{caching} for functions that depend on arguments, it is straightforward
to implement using a \refObject{Dictionary}.  In the future \us might
support dictionaries whose indices are not only strings, but in the
meanwhile, convert the arguments into strings, as the following sample
object demonstrates.

\begin{urbiscript}
class UnaryLazy
{
  function init(f)
  {
    results = [ => ];
    func = f;
  };
  function value(p)
  {
    var sp = p.asString();
    if (results.has(sp))
      return results[sp];
    var res = func(p);
    results[sp] = res |
    res
  };
  var results;
  var func;
} |
// The function to cache.
var inc = function(x) { echo("incing " + x) | x+1 } |
// The function with cache. UnaryLazy simply takes the function as argument.
var p = UnaryLazy.new(inc);
[00062847] UnaryLazy_0x78b750
p.value(1);
[00066758] *** incing 1
[00066759] 2
p.value(1);
[00069058] 2
p.value(2);
[00071558] *** incing 2
[00071559] 3
p.value(2);
[00072762] 3
p.value(1);
[00074562] 2
\end{urbiscript}

\subsection{Prototypes}

\begin{refObjects}
\item[Comparable]
\end{refObjects}

\subsection{Construction}

Lazies are seldom instantiated manually. They are mainly created
automatically when a lazy function call is made (see
\autoref{sec:lang:call}). One can however create a lazy value with the
standard \lstinline|new| method of \lstinline|Lazy|, giving it an
argument-less function which evaluates to the value made lazy.

\begin{urbiscript}
Lazy.new(closure () { /* Value to make lazy */ 0 });
[00000000] 0
\end{urbiscript}

\subsection{Slots}

\begin{urbiscriptapi}
\item['=='](<that>)%
  Whether \this and \var{that} are the same source code and value (an not
  yet evaluated Lazy is never equal to an evaluated one).
\begin{urbiassert}
Lazy.new(closure () { 1 + 1 }) == Lazy.new(closure () { 1 + 1 });
Lazy.new(closure () { 1 + 2 }) != Lazy.new(closure () { 2 + 1 });
\end{urbiassert}
\begin{urbiscript}
{
  var l1 = Lazy.new(closure () { 1 + 1 });
  var l2 = Lazy.new(closure () { 1 + 1 });
  assert (l1 == l2);
  l1.eval();
  assert (l1 != l2);
  l2.eval();
  assert (l1 == l2);
};
\end{urbiscript}


\item[asString]
  The conversion to \refObject{String} of the body of a non-evaluated
  argument.
\begin{urbiassert}
Lazy.new(closure () { echo(1); 1 }).asString() == "echo(1);\n1";
\end{urbiassert}


\item[eval]%
  Force the evaluation of the held lazy value. Two calls to \refSlot{eval}
  will systematically evaluate the expression twice, which can be useful to
  duplicate its side effects.


\item[value]%
  Return the held value, potentially evaluating it before. \refSlot{value}
  performs memoization, that is, only the first call will actually evaluate
  the expression, subsequent calls will return the cached value. Unless you
  want to explicitly trigger side effects from the expression by evaluating
  it several time, this should be preferred over \lstinline|eval| to avoid
  evaluating the expression several times uselessly.
\end{urbiscriptapi}


%%% Local Variables:
%%% coding: utf-8
%%% mode: latex
%%% TeX-master: "../urbi-sdk"
%%% ispell-dictionary: "american"
%%% ispell-personal-dictionary: "../urbi.dict"
%%% fill-column: 76
%%% End:

\section{List}

\lstinline|List|s implement potentially-empty ordered (heterogeneous)
collections of elements.

\subsection{Prototypes}

\begin{itemize}
\item \refObject{Object}
\item \refObject{RangeIterable}
\item \refObject{Orderable}
\end{itemize}

\subsection{Construction}

List can be created with their literal syntax: a possibly empty
sequence of expressions in square brackets, separated by commas.
Non-empty list may actually \emph{terminate} with a comma, rather than
\emph{separate}; in other words, an optional trailing comma is accepted.

\begin{urbiscript}
[]; // The empty list
[00000000] []
[1, "2", [3,],];
[00000000] [1, "2", [3]]
\end{urbiscript}

\subsection{Slots}

\begin{itemize}
\item \lstinline|all(\var{fun})|\\
  % FIXME: link to predicate glossary entry
  Return whether all the members of the target verify the predicate
  \var{fun}.

\begin{urbiassert}[firstnumber=last]
// Are all elements positive?
! [-2, 0, 2, 4].all(function (e) { e > 0 });
// Are all elements even?
[-2, 0, 2, 4].all(function (e) { e % 2 == 0 });
\end{urbiassert}

\item \lstinline|any(\var{fun})|\\
  % FIXME: link to predicate glossary entry
  Whether at least one of the members of the target verifies the
  predicate \var{fun}.

\begin{urbiassert}[firstnumber=last]
// Is there any even element?
! [-3, 1, -1].any(function (e) { e % 2 == 0 });
// Is there any positive element?
[-3, 1, -1].any(function (e) { e > 0 });
\end{urbiassert}

\item \lstinline|asBool|\\
  Whether not empty.
\begin{urbiassert}[firstnumber=last]
[].asBool == false;
[1].asBool == true;
\end{urbiassert}

\item \lstinline|asList|\\
Return the target.

\begin{urbiassert}[firstnumber=last]
[0, 1, 2].asList == [0, 1, 2];
\end{urbiassert}

\item \lstinline|asString|\\
  A string describing the list.  Uses \lstinline|asPrintable| on its
  members, so that, for instance, strings are displayed with quotes.

\begin{urbiassert}[firstnumber=last]
[0, [1], "2"].asString == "[0, [1], \"2\"]";
\end{urbiassert}

\item \lstinline|back|\\
Return the last element of the target. An error if the target is empty.

\begin{urbiscript}[firstnumber=last]
assert([0, 1, 2].back == 2);
[].back;
[00000000:error] !!! back: cannot be applied onto empty list
\end{urbiscript}

\item \lstinline|clear|\\
  Empty the target.

\begin{urbiscript}[firstnumber=last]
var x = [0, 1, 2];
[00000000] [0, 1, 2]
assert(x.clear == []);
\end{urbiscript}

\item \lstinline|each(\var{fun})|\\
  Apply the given functional value \var{fun} on all members,
  sequentially.

\begin{urbiscript}[firstnumber=last]
[0, 1, 2].each(function (v) {echo (v * v); echo (v * v)});
[00000000] *** 0
[00000000] *** 0
[00000000] *** 1
[00000000] *** 1
[00000000] *** 4
[00000000] *** 4
\end{urbiscript}

\item \lstinline|'each&'(\var{fun})|\\
Apply the given functional value on all members simultaneously.

\begin{urbiscript}[firstnumber=last]
[0, 1, 2].'each&'(function (v) {echo (v * v); echo (v * v)});
[00000000] *** 0
[00000000] *** 1
[00000000] *** 4
[00000000] *** 0
[00000000] *** 1
[00000000] *** 4
\end{urbiscript}

\item \lstinline|empty|\\
  Whether the target is empty.

\begin{urbiassert}[firstnumber=last]
[].empty;
! [1].empty;
\end{urbiassert}

\item \lstinline|filter(\var{fun})|\\
  The list of all the members of the target that verify the predicate
  \var{fun}.

\begin{urbiassert}[firstnumber=last]
// Keep only odd numbers.
[0, 1, 2, 3, 4, 5].filter(function (v) {v % 2 == 1}) == [1, 3, 5];
\end{urbiassert}

\item \lstinline|foldl(\var{action}, \var{value})|\\
  \wref[Fold_(higher-order_function)]{Fold},
  also known as \dfn{reduce} or \dfn{accumulate}, computes a result
  from a list.  Starting from \var{value} as the initial result, apply
  repeatedly the binary \var{action} to the current result and the
  next member of the list, from left to right.  For instance, if
  \var{action} were the binary addition and \var{value} were 0, then
  folding a list would compute the sum of the list, including for
  empty lists.

\begin{urbiscript}[firstnumber=last]
[].foldl(function (a, b) { a + b }, 0);
[00000000] 0
[1, 2, 3].foldl(function (a, b) { a + b }, 0);
[00000000] 6
[1, 2, 3].foldl(function (a, b) { a - b }, 0);
[00000000] -6
\end{urbiscript}

\item \lstinline|front|\\
  Return the first element of the target. An error if the target is
  empty.

\begin{urbiscript}[firstnumber=last]
assert([0, 1, 2].front == 0);
[].front;
[00000000:error] !!! front: cannot be applied onto empty list
\end{urbiscript}

\item \lstinline|has(\var{that})|\\
  Whether one of the members of the target equals the argument.

\begin{urbiassert}[firstnumber=last]
[0, 1, 2].has(1);
! [0, 1, 2].has(5);
\end{urbiassert}

\item \lstinline|hasSame(\var{that})|\\
  Return whether one of the member of the target is physically equal
  to the argument.

\begin{urbiscript}[firstnumber=last]
var y = 1;
[00000000:hide] 1
[0, y, 2].hasSame(1);
[00000000] false
[0, y, 2].hasSame(y);
[00000000] true
\end{urbiscript}

\item \lstinline|head|\\
  Synonym for \lstinline|front|.

\item \lstinline|insertBack(\var{that})|\\
  Insert the given element at the end of the target.

\begin{urbiscript}[firstnumber=last]
var z = [0, 1];
[00000000] [0, 1]
assert(z.insertBack(2) == [0, 1, 2]);
assert(z == [0, 1, 2]);
\end{urbiscript}

\item \lstinline|insertFront(\var{that})|\\
  Insert the given element at the beginning of the target.

\begin{urbiscript}[firstnumber=last]
var a = [1, 2];
[00000000] [1, 2]
assert(a.insertFront(0) == [0, 1, 2]);
assert(a == [0, 1, 2]);
\end{urbiscript}

\item \lstinline|join(\var{sep} = "", \var{prefix} = "", \var{suffix} = "")|\\
  Bounces to \lstinline|String.join|, see \refObject{String}.

\begin{urbiassert}[firstnumber=last]
["", "ob", ""].join                == "ob";
["", "ob", ""].join("a")           == "aoba";
["", "ob", ""].join("a", "B", "b") == "Baobab";
\end{urbiassert}

\item \lstinline|keys()|\\
  The list of valid indexes.  This allows uniform iteration over a
  \refObject{Dictionary} or a \refObject{List}.

\begin{urbiscript}[firstnumber=last]
{
  var l = ["a", "b", "c"];
  assert(l.keys == [0, 1, 2]);
  assert({
           var res = [];
           for (var k: l.keys)
             res << l[k];
           res
         }
         == l);
};
\end{urbiscript}

\item \lstinline|map(\var{fun})|\\
Apply the given functional value on every member, and return the list
of results.

\begin{urbiassert}[firstnumber=last]
[0, 1, 2, 3].map(function (v) { v % 2 == 0})
        == [true, false, true, false];
\end{urbiassert}

\item \lstinline|range(\var{begin}, \var{end} = nil)|\\
  Return a sub-range of the list, from the first index included to the
  second index excluded.  An error if out of bounds.  Negative indices
  are valid, and number from the end.

  If \var{end} is \lstinline|nil|, calling \lstinline|range(\var{n})
  is equivalent to calling \lstinline|range(0, \var{n})|.

\begin{urbiscript}[firstnumber=last]
do ([0, 1, 2, 3])
{
  assert
  {
    range(0, 0)   == [];
    range(0, 1)   == [0];
    range(1)      == [0];
    range(1, 3)   == [1, 2];

    range(-3, -2) == [1];
    range(-3, -1) == [1, 2];
    range(-3, 0)  == [1, 2, 3];
    range(-3, 1)  == [1, 2, 3, 0];
    range(-4, 4)  == [0, 1, 2, 3, 0, 1, 2, 3];
  };
}|;
[].range(1, 3);
[00428697:error] !!! range: invalid index: 1
\end{urbiscript}

\item \lstinline|remove(\var{val})|\\
  Remove all elements from the target that equals \var{val}.

\begin{urbiscript}[firstnumber=last]
var c = [0, 1, 0, 2, 0, 3];
[00000000] [0, 1, 0, 2, 0, 3]
assert(c.remove(0) == [1, 2, 3]);
assert(c == [1, 2, 3]);
\end{urbiscript}

\item \lstinline|removeBack|\\
  Remove and return the last element of the target. An error if the
  target is empty.

\begin{urbiscript}[firstnumber=last]
var t = [0, 1, 2];
[00000000] [0, 1, 2]
assert(t.removeBack == 2);
assert(t == [0, 1]);
[].removeBack;
[00000000:error] !!! removeBack: cannot be applied onto empty list
\end{urbiscript}

\item \lstinline|removeById(\var{that})|\\
  Remove all elements from the target that physically equals
  \var{that}.

\begin{urbiscript}[firstnumber=last]
var d = 1;
[00000000] 1
var e = [0, 1, d, 1, 2];
[00000000] [0, 1, 1, 1, 2]
assert(e.removeById(d) == [0, 1, 1, 2]);
assert(e == [0, 1, 1, 2]);
\end{urbiscript}

\item \lstinline|removeFront|\\
Remove and return the first element from the target. An error if the
target is empty.

\begin{urbiscript}[firstnumber=last]
var g = [0, 1, 2];
[00000000] [0, 1, 2]
assert(g.removeFront == 0);
assert(g == [1, 2]);
[].removeFront;
[00000000:error] !!! removeFront: cannot be applied onto empty list
\end{urbiscript}

\item \lstinline|reverse|\\
Return the target with the order of elements inverted.

\begin{urbiassert}[firstnumber=last]
[0, 1, 2].reverse == [2, 1, 0];
\end{urbiassert}

\item \lstinline|size|\\
Return the number of elements in the target.

\begin{urbiassert}[firstnumber=last]
[0, 1, 2].size == 3;
[].size == 0;
\end{urbiassert}

\item \lstinline|sort|\\
Return the target, sorted with respect to the \lstinline|<| criteria.

\begin{urbiassert}[firstnumber=last]
[1, 0, 3, 2].sort == [0, 1, 2, 3];
\end{urbiassert}

\item \lstinline|tail|\\
Return the target, minus the first element. An error if the target is
empty.

\begin{urbiscript}[firstnumber=last]
assert([0, 1, 2].tail == [1, 2]);
[].tail;
[00000000:error] !!! tail: cannot be applied onto empty list
\end{urbiscript}

\item \lstinline|'=='(\var{that})|\\
Check whether all elements in the target and \var{that}, are
equal two by two.

\begin{urbiassert}[firstnumber=last]
[0, 1, 2] == [0, 1, 2];
!([0, 1, 2] == [0, 0, 2]);
\end{urbiassert}

\item \lstinline|'[]'(\var{n})|\\
  Return the \var{n}th member of the target (indexing is
  zero-based). If \var{n} is negative, start from the end.  An error
  if out of bounds.

\begin{urbiscript}[firstnumber=last]
assert(["0", "1", "2"][0] == "0");
assert(["0", "1", "2"][2] == "2");
["0", "1", "2"][3];
[00007061:error] !!! []: invalid index: 3

assert(["0", "1", "2"][-1] == "2");
assert(["0", "1", "2"][-3] == "0");
["0", "1", "2"][-4];
[00007061:error] !!! []: invalid index: -4
\end{urbiscript}

\item \lstinline|'[]='(\var{index}, \var{value})|\\
  Assign \var{value} to the element of the target at the given
  \var{index}.

\begin{urbiscript}[firstnumber=last]
var f = [0, 1, 2];
[00000000] [0, 1, 2]
f[1] = 42;
[00000000] 42
assert(f == [0, 42, 2]);
\end{urbiscript}

\item \lstinline|'*'(\var{n})|\\
  Return the target, concatenated \var{n} times to itself.
\begin{urbiassert}[firstnumber=last]
[0, 1] * 3 == [0, 1, 0, 1, 0, 1];
\end{urbiassert}

  Note that since it is the very same list which is repeatedly
  concatenated (the content is not cloned), side-effects on one item
  will reflect on ``all the items''.

\begin{urbiscript}[firstnumber=last]
var l = [[]] * 3;
[00000000] [[], [], []]
l[0] << 1;
[00000000] [1]
l;
[00000000] [[1], [1], [1]]
\end{urbiscript}

\item \lstinline|'+'(\var{other})|\\
Return the concatenation of the target and the \var{other} list.

\begin{urbiassert}[firstnumber=last]
[0, 1] + [2, 3] == [0, 1, 2, 3];
\end{urbiassert}

\item \lstinline|'-'(\var{other})|\\
Return the target without all element that equals any element in the
\var(other) list.

\begin{urbiassert}[firstnumber=last]
[0, 1, 0, 2, 3] - [1, 2] == [0, 0, 3];
\end{urbiassert}

\item \lstinline|'<<'(\var{that})|\\
  A synonym for \lstinline|insertBack|.

\item \lstinline|'<'(\var{other})|\\
  Return whether the target is inferior to the \var{other} list. A
  list is inferior to another if at least one of its element differs
  from the other, and the first differing element is inferior to the
  other.

\begin{urbiassert}[firstnumber=last]
!([0, 1, 2] < [0, 1, 2]);
!([0, 1, 2] < [0, 0, 2]);
[0, 1, 2] < [0, 2, 2];
\end{urbiassert}

  Since List derives from \refObject{Orderable}, the other order-based
  operators are defined.

\begin{urbiassert}[firstnumber=last]
 [0, 1, 2] <= [0, 1, 2];
 [0, 1, 2] >= [0, 1, 2];
 [0, 1, 2] >  [0, 0, 2];
\end{urbiassert}
\end{itemize}

%%% Local Variables:
%%% mode: latex
%%% TeX-master: "../urbi-sdk"
%%% End:

% LocalWords:  lst asList asString foldl hasSame removeBack popback removeFront
% LocalWords:  popfront insertBack pushback insertFront pushfront urbi sameAs
% LocalWords:  removeById setNth

%% Copyright (C) 2009-2011, Gostai S.A.S.
%%
%% This software is provided "as is" without warranty of any kind,
%% either expressed or implied, including but not limited to the
%% implied warranties of fitness for a particular purpose.
%%
%% See the LICENSE file for more information.

\section{Lobby}

A \dfn{lobby} is the local environment for each (remote or local)
connection to an \urbi server.

\subsection{Examples}

Since every lobby is-a \refObject{Channel}, one can use the methods of
Channel.

\begin{urbiscript}
lobby << 123;
[00478679] 123
lobby << "foo";
[00478679] "foo"
\end{urbiscript}

\subsection{Prototypes}
\begin{itemize}
\item \refSlot[Channel]{topLevel}, an instance of \refObject{Channel}
  with an empty Channel name.
\end{itemize}

\subsection{Construction}

A lobby is implicitly created at each connection. At the top level,
\this is a \dfn{Lobby}.

\begin{urbiscript}
this.protos;
[00000001] [Lobby]
this.protos[0].protos;
[00000003] [Channel_0xADDR]
\end{urbiscript}

Lobbies cannot be cloned, they must be created using \refSlot{create}.

\begin{urbiscript}
Lobby.new;
[00000177:error] !!! new: `Lobby' objects cannot be cloned
Lobby.create;
[00000174] Lobby_0x126450
\end{urbiscript}


\subsection{Slots}
\begin{urbiscriptapi}
\item[authors] Credit the authors of \usdk.


\item[banner] Internal.  Display \usdk banner.
\begin{urbiscript}
lobby.banner;
[00005344] *** ********************************************************
[00005344] *** Urbi SDK version 2.7.1 patch 472 revision b876514
[00005344] *** Copyright (C) 2004-2011 Gostai S.A.S.
[00005344] ***
[00005344] *** This program comes with ABSOLUTELY NO WARRANTY.  It can
[00005344] *** be used under certain conditions.  Type `license;',
[00005344] *** `authors;', or `copyright;' for more information.
[00005344] ***
[00005344] *** Check our community site: http://www.urbiforge.org.
[00005344] *** ********************************************************
\end{urbiscript}


\item[bytesReceived] The number of bytes that were ``input'' to \this.  See
  also \refSlot{receive}.
\begin{urbiscript}
var l = Lobby.create|;
assert (l.bytesReceived == 0);

l.receive("123456789;");
[00000022] 123456789
assert (l.bytesReceived == 10);

l.receive("1234;");
[00000023] 1234
assert (l.bytesReceived == 15);
\end{urbiscript}

\begin{urbicomment}
removeSlot("l")|;
\end{urbicomment}


\item[bytesSent] The number of bytes that were ``output'' by \this.  See
  also \refSlot{send} and \refSlot{write}.
\begin{urbiscript}
var l = Lobby.create|;
assert (l.bytesSent == 0);

l.send("0123456789");
[00011988] 0123456789
// 22 = "[00011988] 0123456789\n".size.
assert (l.bytesSent == 22);

l.send("xx", "label");
[00061783:label] xx
// 20 = "[00061783:label] xx\n".size.
assert (l.bytesSent == 42);

l.write("[01234567]\n");
[01234567]
assert (l.bytesSent == 53);
\end{urbiscript}


\item[connected]
  Whether \this is connected.
\begin{urbiassert}
connected;
\end{urbiassert}


\item[connectionTag] The tag of all code executed in the context of \this.
  This tag applies to \this, but the top-level loop is immune to
  \refSlot[Tag]{stop}, therefore \lstinline|connectionTag| controls every
  thing that was launched from this lobby, yet the lobby itself is still
  usable.
\begin{urbiscript}
every (1s) echo(1), sleep(0.5s); every (1s) echo(2),
sleep(1.2s);
connectionTag.stop;
[00000507] *** 1
[00001008] *** 2
[00001507] *** 1
[00002008] *** 2

"We are alive!";
[00002008] "We are alive!"

every (1s) echo(3), sleep(0.5s); every (1s) echo(4),
sleep(1.2s);
connectionTag.stop;
[00003208] *** 3
[00003710] *** 4
[00004208] *** 3
[00004710] *** 4

"and kicking!";
[00002008] "and kicking!"
\end{urbiscript}

  Of course, a background job may stop a foreground one.
\begin{urbiscript}
{ sleep(1.2s); connectionTag.stop; },
// Note the `;', this is a foreground statement.
every (1s) echo(5);
[00005008] *** 5
[00005508] *** 5

"bye!";
[00006008] "bye!"
\end{urbiscript}


\item[copyright](<deep> = true)%
  Display the copyright of \usdk.  Include copyright information
  about sub-components if \var{deep}.
\begin{urbiscript}
lobby.copyright(false);
[00033102] *** Urbi SDK version 2.7.1 patch 472 revision b876514
[00033102] *** Copyright (C) 2004-2011 Gostai S.A.S.

lobby.copyright;
[00088621] *** Urbi SDK version 2.7.1 patch 472 revision b876514
[00088621] *** Copyright (C) 2004-2011 Gostai S.A.S.
[00088621] ***
[00088621] *** Libport version urbi-sdk-2.7.1 patch 117 revision e96cd13
[00088621] *** Copyright (C) 2006-2011 Gostai S.A.S.
\end{urbiscript}


\item[create]
  Instantiate a new Lobby.
\begin{urbiassert}
Lobby.create.isA(Lobby);
\end{urbiassert}


\item[echo](<value>, <channel> = "")%
  Send \lstinline|\var{value}.asString| to \this, prefixed
  by the \refObject{String} \var{channel} name if specified.  This is
  the preferred way to send informative messages (prefixed with
  \samp{***}).
\begin{urbiscript}
lobby.echo("111", "foo");
[00015895:foo] *** 111
lobby.echo(222, "");
[00051909] *** 222
lobby.echo(333);
[00055205] *** 333
\end{urbiscript}


\item[echoEach](<list>, <channel> = "")%
  Apply \lstinline|echo(\var{m}, \var{channel})| for each member \var{m} of
  \var{list}.
\begin{urbiscript}
lobby.echo([1, "2"], "foo");
[00015895:foo] *** [1, "2"]

lobby.echoEach([1, "2"], "foo");
[00015895:foo] *** 1
[00015895:foo] *** 2

lobby.echoEach([], "foo");
\end{urbiscript}


%\item \lstinline|help|\experimental\\
%  Launch the tutorial.


\item[instances]
  A list of the currently alive lobbies.  It contains at least the Lobby
  object itself, and the current \refSlot{lobby}.
\begin{urbiassert}
lobby in Lobby.instances;
Lobby in Lobby.instances;
\end{urbiassert}


\item[license]
  Display the end user license agreement of the \usdk.
\begin{urbiunchecked}
lobby.license;
[00000000] ***                     GNU AFFERO GENERAL PUBLIC LICENSE
[00000000] ***                        Version 3, 19 November 2007
[00000000] ***
[00000000] ***  Copyright (C) 2007 Free Software Foundation, Inc. <http://fsf.org/>
[00000000] ***  Everyone is permitted to copy and distribute verbatim copies
[00000000] ***  of this license document, but changing it is not allowed.
[00000000] ***
[00000000] ***                             Preamble
[00000000] ***
[00000000] ***   The GNU Affero General Public License is a free, copyleft license for
[00000000] *** software and other kinds of works, specifically designed to ensure
[00000000] *** cooperation with the community in the case of network server software.
[00000000] *** [...]
\end{urbiunchecked}


\item[lobby]
  The lobby of the current connection.  This is typically \this.
\begin{urbiassert}
Lobby.lobby === this;
\end{urbiassert}

  But when several connections are active (e.g., when there are remote
  connections), it can be different from the target of the call.

\begin{urbiscript}
Lobby.create| Lobby.create|;
for (var l : lobbies)
  assert (l.lobby == Lobby.lobby);
\end{urbiscript}


\item[onDisconnect](<lobby>)%
  Event launched when \this is disconnected.  There is a single event
  instance for all the lobby, \refSlot[Lobby]{onDisconnect}, the
  disconnected lobby being passed as argument.


\item[quit] Shut this lobby down, i.e., close the connection.  The
  server is still running, see \refSlot[System]{shutdown} to quit the
  server.


\item[receive](<value>)%
  This is low-level routine.  Pretend the \refObject{String}
  \var{value} was received from the connection.  There is no guarantee
  that \var{value} will be the next program block that will be
  processed: for instance, if you load a file which, in its middle,
  uses \lstinline|lobby.receive("foo")|, then \lstinline|"foo"| will
  be appended after the end of the file.
\begin{urbiscript}
Lobby.create.receive("12;");
[00478679] 12
\end{urbiscript}


\item[remoteIP]
  When \this is connected to a remote server, it's Internet address.


\item[send](<value>, <channel> = "")%
  This is low-level routine.  Send the \refObject{String} \var{value}
  to \this, prefixed by the \refObject{String}
  \var{channel} name if specified.
\begin{urbiscript}
lobby.send("111", "foo");
[00015895:foo] 111
lobby.send("222", "");
[00051909] 222
lobby.send("333");
[00055205] 333
\end{urbiscript}


\item[thanks] Credit the contributors of \usdk.


\item[wall](<value>, <channel> = "")%
  Perform \lstinline|echo(\var{value}, \var{channel})| on all the
  existing lobbies (except Lobby itself).
\begin{urbiscript}[firstnumber=1]
Lobby.wall("111", "foo");
[00015895:foo] *** 111
\end{urbiscript}


\item[write](<value>)%
  This is low-level routine.  Send the \refObject{String} \var{value}
  to the connection.  Note that because of buffering, the output might
  not be visible before an end-of-line character is output.
\begin{urbiscript}
lobby.write("[");
lobby.write("999999999:");
lobby.write("myTag] ");
lobby.write("Hello, World!");
lobby.write("\n");
[999999999:myTag] Hello, World!
\end{urbiscript}
\end{urbiscriptapi}

%%% Local Variables:
%%% coding: utf-8
%%% mode: latex
%%% TeX-master: "../urbi-sdk"
%%% ispell-dictionary: "american"
%%% ispell-personal-dictionary: "../urbi.dict"
%%% fill-column: 76
%%% End:

%% Copyright (C) 2010, 2011, Gostai S.A.S.
%%
%% This software is provided "as is" without warranty of any kind,
%% either expressed or implied, including but not limited to the
%% implied warranties of fitness for a particular purpose.
%%
%% See the LICENSE file for more information.

\section{Location}

This class aggregates two Positions and provides a way to print them as done
in error messages.

\subsection{Prototypes}
\begin{refObjects}
\item[Object]
\end{refObjects}

\subsection{Construction}

Without argument, a newly constructed Location has its Positions initialized
to the first line and the first column.

\begin{urbiscript}[firstnumber=1]
Location.new();
[00000001] 1.1
\end{urbiscript}

With a Position argument \var{p}, the Location will clone the Position into
the begin and end Positions.

\begin{urbiscript}[firstnumber=1]
Location.new(Position.new("file.u",14,25));
[00000001] file.u:14.25
\end{urbiscript}

With two Positions arguments \var{begin} and \var{end}, the Location will
clone both Positions into its own fields.

\begin{urbiscript}[firstnumber=1]
Location.new(Position.new("file.u",14,25), Position.new("file.u",14,35));
[00000001] file.u:14.25-34
\end{urbiscript}

\subsection{Slots}

\begin{urbiscriptapi}
\item['=='](<other>)%
  Compare the begin and end \lstinline|Position|.
\begin{urbiscript}
{
  var p1 = Position.new("file.u",14,25);
  var p2 = Position.new("file.u",16,35);
  var p3 = Position.new("file.u",18,45);
  assert
  {
    Location.new(p1, p3) != Location.new(p1, p2);
    Location.new(p1, p3) == Location.new(p1, p3);
    Location.new(p1, p3) != Location.new(p2, p3);
  };
};
\end{urbiscript}


\item[asString]%
  Present Locations with less variability as possible as either:
  \begin{itemize}
  \item \samp{\var{file}:\var{ll}.\var{cc}}
  \item \samp{\var{file}:\var{ll}.\var{cc}-\var{cc}}
  \item \samp{\var{file}:\var{ll}.\var{cc}-\var{ll}.\var{cc}}
  \end{itemize}
  or the same without file name when the file name is not defined.
\begin{urbiassert}
Location.new(Position.new("file.u",14,25)).asString() == "file.u:14.25";
Location.new(Position.new(14,25)).asString() == "14.25";

Location.new(
  Position.new("file.u", 14, 25),
  Position.new("file.u", 14, 35)
).asString() == "file.u:14.25-34";

Location.new(
  Position.new(14, 25),
  Position.new(14, 35)
).asString() == "14.25-34";

Location.new(
  Position.new("file.u", 14, 25),
  Position.new("file.u", 15, 35)
).asString() == "file.u:14.25-15.34";

Location.new(
  Position.new(14, 25),
  Position.new(15, 35)
).asString() == "14.25-15.34";
\end{urbiassert}


\item[begin]%
  The begin Position used by the Location.  Modifying a copy of this field
  does not modify the Location.
\begin{urbiassert}
Location.new(
  Position.new("file.u", 14, 25),
  Position.new("file.u", 16, 35)
).begin == Position.new("file.u", 14, 25);
\end{urbiassert}


\item[end]%
  The end Position used by the Location.  Modifying a copy of this field
  does not modify the Location.
\begin{urbiassert}
Location.new(
  Position.new("file.u",14,25),
  Position.new("file.u",16,35)
).end == Position.new("file.u",16,35);
\end{urbiassert}
\end{urbiscriptapi}

%%% Local Variables:
%%% coding: utf-8
%%% mode: latex
%%% TeX-master: "../urbi-sdk"
%%% ispell-dictionary: "american"
%%% ispell-personal-dictionary: "../urbi.dict"
%%% fill-column: 76
%%% End:

\section{Math}

This object is actually meant to play the role of a name-space in
which the mathematical functions are defined with a more conventional
notation.  Indeed, in an object-oriented language, writing
\lstinline|pi.cos| makes perfect sense, yet \lstinline|cos(pi)| is
more usual.

\subsection{Prototypes}
\begin{itemize}
\item \refObject{Singleton}
\end{itemize}

\subsection{Construction}

Since it is a \refObject{Singleton}, you are not expected to build
other instances.

\subsection{Slots}

\begin{itemize}
\item \lstinline|abs(\var{float})|\\
  Bounce to \lstinline|\var{float}.abs|.

\item \lstinline|acos(\var{float})|\\
  Bounce to \lstinline|\var{float}.acos|.

\item \lstinline|asin(\var{float})|\\
  Bounce to \lstinline|\var{float}.asin|.

\item \lstinline|atan(\var{float})|\\
  Bounce to \lstinline|\var{float}.atan|.

\item \lstinline|atan2(\var{x}, \var{y})|\\
  Bounce to \lstinline|\var{x}.atan2(\var{y})|.

\item \lstinline|cos(\var{float})|\\
  Bounce to \lstinline|\var{float}.cos|.

\item \lstinline|exp(\var{float})|\\
  Bounce to \lstinline|\var{float}.exp|.

\item \lstinline|inf|\\
  Bounce to \lstinline|Float.inf|.

\item \lstinline|log(\var{float})|\\
  Bounce to \lstinline|\var{float}.log|.

\item \lstinline|nan|\\
  Bounce to \lstinline|Float.nan|.

\item \lstinline|pi|\\
  Bounce to \lstinline|Float.pi|.

\item \lstinline|random(\var{float})|\\
  Bounce to \lstinline|\var{float}.random|.

\item \lstinline|round(\var{float})|\\
  Bounce to \lstinline|\var{float}.round|.

\item \lstinline|sgn(\var{float})|\\
  Bounce to \lstinline|\var{float}.sgn|.

\item \lstinline|sin(\var{float})|\\
  Bounce to \lstinline|\var{float}.sin|.

\item \lstinline|sqr(\var{float})|\\
  Bounce to \lstinline|\var{float}.sqr|.

\item \lstinline|sqrt(\var{float})|\\
  Bounce to \lstinline|\var{float}.sqrt|.

\item \lstinline|tan(\var{float})|\\
  Bounce to \lstinline|\var{float}.tan|.

\item \lstinline|trunc(\var{float})|\\
  Bounce to \lstinline|\var{float}.trunc|.
\end{itemize}


%%% Local Variables:
%%% mode: latex
%%% TeX-master: "../urbi-sdk"
%%% End:

%% Copyright (C) 2009-2011, Gostai S.A.S.
%%
%% This software is provided "as is" without warranty of any kind,
%% either expressed or implied, including but not limited to the
%% implied warranties of fitness for a particular purpose.
%%
%% See the LICENSE file for more information.

\section{nil}

The special entity \lstinline|nil| is an object used to denote an empty
value.  Contrary to \refObject{void}, it is a regular value which can be
read.

\subsection{Prototypes}

\begin{refObjects}
\item[Singleton]
\end{refObjects}

\subsection{Construction}

Being a singleton, \lstinline|nil| is not to be constructed, just used.

\begin{urbiassert}[firstnumber=1]
nil == nil;
\end{urbiassert}

\subsection{Slots}

\begin{urbiscriptapi}
\item[isNil] Whether \this is \refObject{nil}.  I.e., true.  See also
  \refSlot[Object]{isNil}.
\begin{urbiassert}
nil.isNil;
!Object.isNil;  !42.isNil;  !(function () { nil }.isNil);
\end{urbiassert}


\item[isVoid] In order to facilitate the transition from older code to
  newer, return true.  In the future, false will be returned.  Therefore, if
  you really need to check whether \var{foo} is \lstinline|void| but not
  \lstinline|nil|, use
  \lstinline|!\var{foo}.acceptVoid.isNil && \var{foo}.isVoid|.
\begin{urbiassert}
nil.isVoid;
[     Logger     ] nil.isVoid will return false eventually, adjust your code.
[     Logger     ]     For instance replace InputStream loops from
[     Logger     ]       while (!(x = i.get.acceptVoid).isVoid)
[     Logger     ]         cout << x;
[     Logger     ]     to
[     Logger     ]       while (!(x = i.get).isNil)
[     Logger     ]         cout << x;
\end{urbiassert}
\end{urbiscriptapi}

%%% Local Variables:
%%% mode: latex
%%% TeX-master: "../urbi-sdk"
%%% ispell-dictionary: "american"
%%% ispell-personal-dictionary: "../urbi.dict"
%%% fill-column: 76
%%% End:

%% Copyright (C) 2009-2010, Gostai S.A.S.
%%
%% This software is provided "as is" without warranty of any kind,
%% either expressed or implied, including but not limited to the
%% implied warranties of fitness for a particular purpose.
%%
%% See the LICENSE file for more information.

\section{Mutex}

\dfn{Mutex} allow to define critical sections.

\subsection{Prototypes}
\begin{itemize}
\item \refObject{Tag}
\end{itemize}

\subsection{Construction}
A Mutex can be constructed like any other Tag but without name.

\begin{urbiscript}[firstnumber=1]
var m = Mutex.new;
[00000000] Mutex_0x964ed40
\end{urbiscript}

You can define critical sections by tagging your code using the Mutex.

\begin{urbiscript}[firstnumber=1]
var m = Mutex.new |
m: echo("this is critical section");
[00000001] *** this is critical section
\end{urbiscript}

As a critical section, two pieces of code tagged by the same ``Mutex''
will never be executed at the same time.

\subsection{Slots}

\begin{urbiscriptapi}
\item[asMutex]  Return \lstinline|this|.
\begin{urbiscript}
var m1 = Mutex.new;
assert
{
  m1.asMutex === m1;
};
\end{urbiscript}
\end{urbiscriptapi}


%%% Local Variables:
%%% mode: latex
%%% TeX-master: "../urbi-sdk"
%%% ispell-dictionary: "american"
%%% ispell-personal-dictionary: "../urbi.dict"
%%% fill-column: 76
%%% End:

%% Copyright (C) 2009-2011, Gostai S.A.S.
%%
%% This software is provided "as is" without warranty of any kind,
%% either expressed or implied, including but not limited to the
%% implied warranties of fitness for a particular purpose.
%%
%% See the LICENSE file for more information.

\section{Object}

\refObject{Object} includes the mandatory primitives for all objects in \us.
All objects in \us must inherit (directly or indirectly) from it.

\subsection{Prototypes}

\begin{refObjects}
\item[Comparable]
\item[Global]
\end{refObjects}

\subsection{Construction}

A fresh object can be instantiated by cloning \slot{Object} itself.

\begin{urbiscript}[firstnumber=1]
Object.new;
[00000421] Object_0x00000000
\end{urbiscript}

The keyword \lstindex{class} also allows to define objects intended to serve
as prototype of a family of objects, similarly to classes in traditional
object-oriented programming languages (see \autoref{sec:tut:class}).

\begin{urbiscript}
{
  class Foo
  {
    var attr = 23;
  };
  assert
  {
    Foo.localSlotNames == ["asFoo", "attr", "type"];
    Foo.asFoo === Foo;
    Foo.attr == 23;
    Foo.type == "Foo";
  };
};
\end{urbiscript}


\subsection{Slots}

\begin{urbiscriptapi}
\item[acceptVoid]
  Return \this.  See \refObject{void} to know why.
\begin{urbiscript}
{
  var o = Object.new;
  assert(o.acceptVoid === o);
};
\end{urbiscript}


\item[addProto](<proto>)%
  Add \var{proto} into the list of prototypes of \this.  Return \this.
\begin{urbiscript}
do (Object.new)
{
  assert
  {
    addProto(Orderable) === this;
    protos == [Orderable, Object];
  };
}|;
\end{urbiscript}

\item[allProto]%
  A list with \this, its parents, their parents,\ldots
\begin{urbiassert}
123.allProtos.size == 12;
\end{urbiassert}

\item[allSlotNames]
  Deprecated alias for \refSlot{slotNames}.
\begin{urbiassert}
Object.allSlotNames == Object.slotNames;
\end{urbiassert}

\item[apply](<args>)%
  ``Invoke \this''.  The size of the argument list,
  \var{args}, must be one.  This argument is ignored.  This function
  exists for compatibility with \refSlot[Code]{apply}.
\begin{urbiassert}
Object.apply([this]) === Object;
Object.apply([1])    === Object;
\end{urbiassert}

\item[as](<type>)%
  Convert \this to \var{type}.  This is syntactic sugar for
  \lstinline|as\var{Type}| when \var{Type} is the \lstinline|type| of
  \var{type}.
\begin{urbiassert}
     12.as(Float) == 12;
   "12".as(Float) == 12;
    12.as(String) == "12";
Object.as(Object) === Object;
\end{urbiassert}

\item[asBool]
  Whether \this is ``true'', see \autoref{sec:truth}.
\begin{urbiscript}
assert
{
  Global.asBool == true;
  nil.asBool ==    false;
};
void.asBool;
[00000421:error] !!! unexpected void
\end{urbiscript}

\item[bounce](<name>)%
  Return \lstinline|this.\var{name}| transformed from a method into a
  function that takes its target (its ``\this'') as first
  and only argument.  \lstinline|this.\var{name}| must take no
  argument.
\begin{urbiassert}
{ var myCos = Object.bounce("cos"); myCos(0) }    == 0.cos;
{ var myType = bounce("type"); myType(Object); } == "Object";
{ var myType = bounce("type"); myType(3.14); }   == "Float";
\end{urbiassert}

\item[callMessage](<msg>)%
  Invoke the \refObject{CallMessage} \var{msg} on this.
%%% \begin{urbiscript}
%%% function f(var tgt, var msg, var args)
%%% {
%%%   call.target  = tgt;
%%%   call.message = msg;
%%%   call.code = tgt.getSlot(msg);
%%%   call.args    = args;
%%%   call.inspect;
%%%   tgt.callMessage(call);
%%% }|;
%%% assert
%%% {
%%%   f(Object, "type", []) == "Object.f(1, 2)";
%%%
%%% };
%%% \end{urbiscript}
\item[clone]
  Clone \this, i.e., create a fresh, empty, object, which
  sole prototype is \this.
\begin{urbiassert}
Object.clone.protos == [Object];
Object.clone.localSlotNames == [];
\end{urbiassert}

\item[cloneSlot](<from>, <to>)%
  Set the new slot \var{to} using a clone of \var{from}. This can only
  be used into the same object.

\begin{urbiscript}
var foo = Object.new |;
cloneSlot("foo", "bar") |;
assert(!(foo === bar));
\end{urbiscript}

\item[copySlot](<from>, <to>)%
  Same as \lstinline|cloneSlot|, but the slot aren't cloned, so the
  two slot are the same.
\begin{urbiscript}
var moo = Object.new |;
cloneSlot("moo", "loo") |;
assert(!(moo === loo));
\end{urbiscript}

\item[createSlot](<name>)%
  Create an empty slot (which actually means it is bound to
  \lstinline|void|) named \var{name}.  Raise an error if \var{name}
  was already defined.
\begin{urbiscript}
do (Object.new)
{
  assert(!hasLocalSlot("foo"));
  assert(createSlot("foo").isVoid);
  assert(hasLocalSlot("foo"));
}|;
\end{urbiscript}

\item[dump](<depth>)%
  Describe \this: its prototypes and slots.  The argument
  \var{depth} specifies how recursive the description is: the greater,
  the more detailed.  This method is mostly useful for debugging
  low-level issues, for a more human-readable interface, see also
  \refSlot{inspect}.
\begin{urbiscript}
do (2) { var this.attr = "foo"; this.attr->prop = "bar" }.dump(0);
[00015137] *** Float_0x240550 {
[00015137] ***   /* Special slots */
[00015137] ***   protos = Float
[00015137] ***   value = 2
[00015137] ***   /* Slots */
[00015137] ***   attr = String_0x23a750 <...>
[00015137] ***     /* Properties */
[00015137] ***     prop = String_0x23a7a0 <...>
[00015137] ***   }
do (2) { var this.attr = "foo"; this.attr->prop = "bar" }.dump(1);
[00020505] *** Float_0x240550 {
[00020505] ***   /* Special slots */
[00020505] ***   protos = Float
[00020505] ***   value = 2
[00020505] ***   /* Slots */
[00020505] ***   attr = String_0x23a750 {
[00020505] ***     /* Special slots */
[00020505] ***     protos = String
[00020505] ***     /* Slots */
[00020505] ***     }
[00020505] ***     /* Properties */
[00020505] ***     prop = String_0x239330 {
[00020505] ***       /* Special slots */
[00020505] ***       protos = String
[00020505] ***       /* Slots */
[00020505] ***       }
[00020505] ***   }
\end{urbiscript}

\item[getPeriod]
  Deprecated.  Use \refSlot[System]{period} instead.

\item[getProperty](<slotName>, <propName>)%
  The value of the \var{propName} property associated to the slot
  \var{slotName} if defined.  Raise an error otherwise.
\begin{urbiscript}
const var myPi = 3.14|;
assert (getProperty("myPi", "constant"));

getProperty("myPi", "foobar");
[00000045:error] !!! property lookup failed: myPi->foobar
\end{urbiscript}

\item[getLocalSlot](<name>)%
  The value associated to \var{name} in \this, excluding
  its ancestors (contrary to \lstinline|getSlot|).
\begin{urbiscript}
var a = Object.new|;

// Local slot.
var a.slot = 21|;
assert
{
  a.locateSlot("slot") === a;
  a.getLocalSlot("slot") == 21;
};

// Inherited slot are not looked-up.
assert { a.locateSlot("init") == Object };
a.getLocalSlot("init");
[00041066:error] !!! lookup failed: init
\end{urbiscript}

\item[getSlot](<name>)%
  The value associated to \var{name} in \this, possibly
  after a look-up in its prototypes (contrary to
  \lstinline|getLocalSlot|).
\begin{urbiscript}
var b = Object.new|;
var b.slot = 21|;

assert
{
  // Local slot.
  b.locateSlot("slot") === b;
  b.getSlot("slot") == 21;

  // Inherited slot.
  b.locateSlot("init") === Object;
  b.getSlot("init") == Object.getSlot("init");
};

// Unknown slot.
assert { b.locateSlot("ENOENT") == nil; };
b.getSlot("ENOENT");
[00041066:error] !!! lookup failed: ENOENT
\end{urbiscript}

\item[hash]%
  A \refObject{Hash} object for \this.  This default implementation returns
  a different hash for every object, so every key maps to a different
  cells. Classes that have value semantic should override the hash method so
  as objects that are equal (in the \refSlot[Object]{'=='} sense) have the
  same hash. \refSlot[String]{hash} does so for instance; as a consequence
  different String objects with the same value map to the same cell.

  A hash only makes sense as long as the hashed object exists.

\begin{urbiscript}
var o1 = Object.new|
var o2 = Object.new|
assert
{
  o1.hash == o1.hash;
  o1.hash != o2.hash;
};
\end{urbiscript}

\item[hasLocalSlot](<slot>)%
  Whether \this features a slot \var{slot}, locally (not from some
  ancestor).  See also \refSlot{hasSlot}.
\begin{urbiscript}
class Base         { var this.base = 23; } |;
class Derive: Base { var this.derive = 43 } |;
assert(Derive.hasLocalSlot("derive"));
assert(!Derive.hasLocalSlot("base"));
\end{urbiscript}

\item[hasProperty](<slotName>, <propName>)%
  Whether the slot \var{slotName} of \this has a property
  \var{propName}.
\begin{urbiscript}
const var halfPi = pi / 2|;
assert
{
  hasProperty("halfPi", "constant");
  !hasProperty("halfPi", "foobar");
};
\end{urbiscript}

\item[hasSlot](<slot>)%
  Whether \this has the slot \var{slot}, locally, or from
  some ancestor.  See also \refSlot{hasLocalSlot}.

\begin{urbiassert}
Derive.hasSlot("derive");
Derive.hasSlot("base");
!Base.hasSlot("derive");
\end{urbiassert}

\item['$id']% fix color $

\item[inspect](<deep> = false)%
  Describe \this: its prototypes and slots, and their
  properties.  If \var{deep}, all the slots are described, not only
  the local slots. See also \refSlot{dump}.
\begin{urbiscript}
do (2) { var this.attr = "foo"; this.attr->prop = "bar"}.inspect;
[00001227] *** Inspecting 2
[00001227] *** ** Prototypes:
[00001227] ***   0
[00001227] *** ** Local Slots:
[00001228] ***   attr : String
[00001228] ***     Properties:
[00001228] ***      prop : String = "bar"
\end{urbiscript}

\item[isA](<obj>)%
  Whether \this has \var{obj} in his parents.
\begin{urbiassert}
   Float.isA(Orderable);
! String.isA(Float);
\end{urbiassert}

\item[isNil]%
  Whether \this is \refObject{nil}.
\begin{urbiassert}
 nil.isNil;
!  0.isNil;
\end{urbiassert}

\item[isProto]
  Whether \this is a prototype.
\begin{urbiassert}
 Float.isProto;
!   42.isProto;
\end{urbiassert}

\item[isVoid]%
  Whether \this is \lstinline|void|.  See \refObject{void}.
\begin{urbiassert}
void.isVoid;
! 42.isVoid;
\end{urbiassert}

\item[localSlotNames]%
  A list with the names of the local (i.e., not including those of its
  ancestors) slots of \this.  See also \refSlot{slotNames}.
\begin{urbiscript}
var top = Object.new|;
var top.top1 = 1|;
var top.top2 = 2|;
var bot = top.new|;
var bot.bot1 = 10|;
var bot.bot2 = 20|;
assert
{
  top.localSlotNames == ["top1", "top2"];
  bot.localSlotNames == ["bot1", "bot2"];
};
\end{urbiscript}

\item[locateSlot](<slot>)%
  The \slot{Object} that provides \var{slot} to \this, or \lstinline|nil| if
  \this does not feature \var{slot}.
\begin{urbiassert}
locateSlot("locateSlot") == Object;
locateSlot("doesNotExist").isNil;
\end{urbiassert}

\item[print] Send \this to the \refSlot[Channel]{topLevel} channel.
\begin{urbiscript}
1.print;
[00001228] 1
[1, "12"].print;
[00001228] [1, "12"]
\end{urbiscript}

\item[protos]
  The list of prototypes of \this.
\begin{urbiassert}
12.protos == [Float];
\end{urbiassert}

\item[properties](<slotName>)%
  A dictionary of the properties of slot \var{slotName}.  Raise an error if
  the slot does not exist.
\begin{urbiscript}
2.properties("foo");
[00238495:error] !!! lookup failed: foo
do (2) { var foo = "foo" }.properties("foo");
[00238501] ["constant" => false]
do (2) { var foo = "foo" ; foo->bar = "bar" }.properties("foo");
[00238502] ["bar" => "bar", "constant" => false]
\end{urbiscript}

\item[removeLocalSlot](<slot>)%
  Remove \var{slot} from the (local) list of slots of \this, and return
  \this.  Raise an error if \var{slot} does not exist.  See also
  \refSlot{removeSlot}.
\begin{urbiscript}
var base = Object.new|;
var base.slot = "base"|;

var derive = Base.new|;
var derive.slot = "derive"|;

derive.removeLocalSlot("foo");
[00000080:error] !!! lookup failed: foo

assert
{
  derive.removeLocalSlot("slot") === derive;
  derive.localSlotNames == [];
  base.slot == "base";
};

derive.removeLocalSlot("slot");
[00000090:error] !!! lookup failed: slot

assert
{
  base.slot == "base";
};
\end{urbiscript}


\item[removeProperty](<slotName>, <propName>)%
  Remove the property \var{propName} from the slot \var{slotName}.  Raise an
  error if the slot does not exist.  Warn if \var{propName} does not exist;
  in a future release this will be an error.
\begin{urbiscript}
var r = Object.new|;

// Non-existing slot.
r.removeProperty("slot", "property");
[00000072:error] !!! lookup failed: slot

var r.slot = "slot value"|;
// Non-existing property.
r.removeProperty("slot", "property");
[00000081:warning] !!! no such property: slot->property
[00000081:warning] !!!    called from: removeProperty

r.slot->property = "property value"|;
assert
{
  r.hasProperty("slot", "property");
  // Existing property.
  r.removeProperty("slot", "property") == "property value";
  ! r.hasProperty("slot", "property");
};
\end{urbiscript}

\item[removeProto](<proto>)%
  Remove \var{proto} from the list of prototypes of \this, and return \this.
  Do nothing if \var{proto} is not a prototype of \this.
\begin{urbiscript}
do (Object.new)
{
  assert
  {
    addProto(Orderable);
    removeProto(123) === this;
    protos == [Orderable, Object];
    removeProto(Orderable) === this;
    protos == [Object];
  };
}|;
\end{urbiscript}

\item[removeSlot](<slot>)%
  Remove \var{slot} from the (local) list of slots of \this, and return
  \this.  Warn if \var{slot} does not exist; in a future release this will
  be an error.  See also \refSlot{removeLocalSlot}.
\begin{urbiscript}
{
  var base = Object.new;
  var base.slot = "base";

  var derive = Base.new;
  var derive.slot = "derive";

  assert
  {
    derive.removeSlot("foo") === derive;
[00000080:warning] !!! no such local slot: foo
[00000080:warning] !!!    called from: removeSlot
[00000080:warning] !!!    called from: code
[00000080:warning] !!!    called from: eval
[00000080:warning] !!!    called from: value
[00000080:warning] !!!    called from: assertCall

    derive.removeSlot("slot") === derive;
    derive.localSlotNames == [];
    base.slot == "base";
    derive.removeSlot("slot") === derive;
[00000099:warning] !!! no such local slot: slot
[00000099:warning] !!!    called from: removeSlot
[00000099:warning] !!!    called from: code
[00000099:warning] !!!    called from: eval
[00000099:warning] !!!    called from: value
[00000099:warning] !!!    called from: assertCall

    base.slot == "base";
  };
};
\end{urbiscript}


\item[setConstSlot]%
  Like \refSlot{setSlot} but the created slot is const.
\begin{urbiscript}
assert(setConstSlot("fortyTwo", 42) == 42);
fortyTwo = 51;
[00000000:error] !!! cannot modify const slot
\end{urbiscript}

\item[setProperty](<slotName>, <propName>, <value>)%
  Set the property \var{propName} of slot \var{slotName} to \var{value}.
  Raise an error in \var{slotName} does not exist.  Return \var{value}.
  This is what \lstinline|\var{slotName}->\var{propName} = \var{value}|
  actually performs.
\begin{urbiscript}
do (Object.new)
{
  var slot = "slot";
  var value = "value";
  assert
  {
    setProperty("slot", "prop", value) === value;
    "prop" in properties("slot");
    getProperty("slot", "prop") === value;
    slot->prop === value;
    setProperty("slot", "noSuchProperty", value) === value;
  };
}|;
setProperty("noSuchSlot", "prop", "12");
[00000081:error] !!! lookup failed: noSuchSlot
\end{urbiscript}


\item[setProtos](<protos>)%
  Set the list of prototypes of \this to \var{protos}.  Return
  \lstinline|void|.
\begin{urbiscript}
do (Object.new)
{
  assert
  {
    protos == [Object];
    setProtos([Orderable, Object]).isVoid;
    protos == [Orderable, Object];
  };
}|;
\end{urbiscript}

\item[setSlot](<name>, <value>)%
  Create a slot \var{name} mapping to \var{value}. Raise an error if
  \var{name} was already defined.  This is what
  \lstinline|var \var{name} = \var{value}| actually performs.
\begin{urbiassert}
Object.setSlot("theObject", Object) === Object;
Object.theObject === Object;
theObject === Object;
\end{urbiassert}

  If the current job is in redefinition mode, \lstinline|setSlot| on
  an already defined slot is not an error and overwrites the slot like
  \lstinline|updateSlot| would. See the \lstinline|redefinitionMode|
  method in \refObject{System}.

\item[slotNames]%
  A list with the slot names of \this and its ancestors.
\begin{urbiassert}
Object.localSlotNames
  .subset(Object.slotNames);
Object.protos.foldl(function (var r, var p) { r + p.localSlotNames },
                    [])
  .subset(Object.slotNames);
\end{urbiassert}

\item[type]%
  The name of the type of \this.  The \lstinline|class|
  construct defines this slot to the name of the class
  (\autoref{sec:tut:class}).  This is used to display the name of
  ``instances''.
\begin{urbiscript}
class Example {};
[00000081] Example
assert
{
  Example.type == "Example";
};
Example.new;
[00000081] Example_0x6fb2720
\end{urbiscript}

\item[uid]
  The unique id of \this.
\begin{urbiscript}
{
  var foo = Object.new;
  var bar = Object.new;
  assert
  {
    foo.uid == foo.uid;
    foo.uid != bar.uid;
  };
};
\end{urbiscript}

\item[unacceptVoid]%
  Return \this.  See \refObject{void} to know why.
\begin{urbiscript}
{
  var o = Object.new|
  assert(o.unacceptVoid === o);
};
\end{urbiscript}

%%% FIXME: \item[uobjectInit]
\item[updateSlot](<name>, <value>)%
  Map the existing slot named \var{name} to \var{value}. Raise an
  error if \var{name} was not defined.
\begin{urbiassert}
Object.setSlot("one", 1)    == 1;
Object.updateSlot("one", 2) == 2;
Object.one                  == 2;
\end{urbiassert}

\item['&&'](<that>)%
  Short-circuiting logical and. If \this evaluates to true evaluate and
  return \var{that}, otherwise return \this without evaluating \var{that}.
\begin{urbiassert}
(0 && "foo") == 0;
(2 && "foo") == "foo";

(""    && "foo") == "";
("foo" && "bar") == "bar";
\end{urbiassert}

\item['||'](<that>)%
  Short-circuiting logical or. If \this evaluates to false evaluate and
  return \var{that}, otherwise return \this without evaluating \var{that}.
\begin{urbiassert}
(0 || "foo") == "foo";
(2 ||  1/0)  == 2;

(""    || "foo") == "foo";
("foo" || 1/0)   == "foo";
\end{urbiassert}

\item \lstinline|'!'|\\
  Logical negation.  If \this evaluates to false return \lstinline|true| and
  vice-versa.
\begin{urbiassert}
!1 == false;
!0 == true;

!"foo" == false;
!""    == true;
\end{urbiassert}

\item['+='](<that>)%
  Bounce to \lstinline|this '+' \var{that}|.

\item['-='](<that>)%
  Bounce to \lstinline|this '-' \var{that}|.

\item['*='](<that>)%
  Bounce to \lstinline|this '*' \var{that}|.

\item['/='](<that>)%
  Bounce to \lstinline|this '/' \var{that}|.

\item['^='](<that>)%
  Bounce to \lstinline|this '^' \var{that}|.

\item \lstinline|'%='(\var{that})|\\
  Bounce to \lstinline|this '-' \var{that}|.

\item['=='](<that>)%
  Whether \this and \that are equal.  See also \refObject{Comparable} and
  \autoref{sec:lang:operators:comparison}.  By default, bounces to
  \refSlot{'==='}.  This operator \emph{must} be redefined for objects that
  have a value-semantics; for instance two \refObject{String} objects that
  denotes the same string should be equal according to \lstinline|==|,
  although physically different (i.e., not equal according to
  \lstinline|===|).
\begin{urbiscript}
{
  var o1 = Object.new;
  var o2 = Object.new;
  assert
  {
      o1 == o1;
    !(o1 == o2);
      o1 != o2;
    !(o1 != o1);

      1  ==  1;
     "1" == "1";
     [1] == [1];
  };
};
\end{urbiscript}

\item['==='](<that>)%
  Whether \this and \that are exactly the same object (i.e., \this and \that
  are two different means to denote the very same location in memory).  To
  denote equivalence, use \refSlot{'=='}; for instance two \refObject{Float}
  objects that denote 42 can be different objects (in the sense of
  \lstinline|===|), but will be considered equal by \lstinline|==|.  See
  also \refSlot{'==='} and \autoref{sec:lang:operators:comparison}.
\begin{urbiscript}
{
  var o1 = Object.new;
  var o2 = Object.new;
  assert
  {
      o1 === o1;
    !(o1 === o2);

    !( 1  ===  1 );
    !("1" === "1");
    !([1] === [1]);
  };
};
\end{urbiscript}

\item \lstinline|'!=='(\var{that})|\\%
  The negation of \lstinline|\this === \that|, see \refSlot{'==='}.
\begin{urbiscript}
{
  var o1 = Object.new;
  var o2 = Object.new;
  assert
  {
      o1 !== o2;
    !(o1 !== o1);

      1  !==  1;
     "1" !== "1";
     [1] !== [1];
  };
};
\end{urbiscript}

\end{urbiscriptapi}

%%% Local Variables:
%%% coding: utf-8
%%% mode: latex
%%% TeX-master: "../urbi-sdk"
%%% ispell-dictionary: "american"
%%% ispell-personal-dictionary: "../urbi.dict"
%%% fill-column: 76
%%% End:

\section{Orderable}
Objects that have a concept of ``smaller than''.  See also
\refObject{Comparable}.

This object, made to serve as prototype, provides a definition of
\lstinline{<} based on \lstinline{>}, and vice versa; and definition
of \lstinline{<=}/\lstinline{>=} based on
\lstinline{<}/\lstinline{>}\lstinline{==}.  You \strong{must} define
either \lstinline{<} or \lstinline{>}, otherwise invoking either
method will result in endless recursions.

\begin{urbiscript}
class Foo : Orderable
{
  var value = 0;
  function init (v) { value = v; };
  function '<' (lhs)  { value < lhs.value; };
  function asString() { "<" + value.asString + ">"; };
};
[00000000] <0>
var one = Foo.new(1);
[00000001] <1>
var two = Foo.new(2);
[00000002] <2>

assert( (one <= one) &&  (one <= two) && !(two <= one));
assert(!(one >  one) && !(one >  two) &&  (two >  one));
assert( (one >= one) && !(one >= two) &&  (two >= one));
\end{urbiscript}


%%% Local Variables:
%%% mode: latex
%%% TeX-master: "../urbi-sdk"
%%% End:

%% Copyright (C) 2009-2010, Gostai S.A.S.
%%
%% This software is provided "as is" without warranty of any kind,
%% either expressed or implied, including but not limited to the
%% implied warranties of fitness for a particular purpose.
%%
%% See the LICENSE file for more information.

\section{OutputStream}

OutputStreams are used to write (possibly binary) files by hand.

\subsection{Prototypes}
\begin{refObjects}
\item[Object]
\end{refObjects}

\subsection{Construction}

An OutputStream is a writing-interface to a file; its constructor
requires a \refObject{File}.  If the file already exists, content is
\emph{appended} to it.  Remove the file beforehand if you want to
override its content.

\begin{urbiscript}
var o1 = OutputStream.new(File.create("file.txt"));
[00000001] OutputStream_0x827000

var o2 = OutputStream.new(File.new("file.txt"));
[00000002] OutputStream_0x827000
\end{urbiscript}

When an OutputStream is opened on a File, that File cannot be removed.  On
Unix systems, this is handled gracefully (the references to the file are
removed, but the content is still there for the streams that were already
bound to this file); so in practice, the File appears to be removable.  On
Windows, the File cannot be removed at all.  Therefore, do not forget to
close the streams you opened.

\begin{urbiscript}
o1.close;
o2.close;
\end{urbiscript}

\subsection{Slots}

\begin{urbiscriptapi}
\item[<<](<that>)%
  Output \lstinline|\var{this}.asString|.  Return \this to
  enable chains of calls.
\begin{urbiscript}
var o = OutputStream.new(File.create("fresh.txt"))|;
o << 1 << "2" << [3, [4]]|;
o.close;
assert (File.new("fresh.txt").content.data == "12[3, [4]]");
\end{urbiscript}

\item[close]
  Flush the buffers, and close the file.
\begin{urbiassert}
OutputStream.new(File.create("close.txt")).close.isVoid;
\end{urbiassert}

\item[flush]%
  To provide efficient input/output operations, \dfn[buffer]{buffers} are
  used.  As a consequence, what is put into a stream might not be
  immediately saved on the actual file.  To \dfn{flush} a buffer means to
  dump its content to the file.
\begin{urbiscript}
var s = OutputStream.new(File.create("file.txt"))|
s.flush;
s.close;
\end{urbiscript}

\item[put](<byte>)%
  Output the character corresponding to the numeric code \var{byte} in
  \this, and return \this.
\begin{urbiscript}
var f = File.create("put.txt") |
var os = OutputStream.new(f) |

assert
{
  os.put(0)
    .put(255)
    .put(72).put(101).put(108).put(108).put(111)
  === os;
  f.content.data == "\0\xffHello";
};
os.put(12.5);
[00029816:error] !!! put: bad numeric conversion: overflow or non empty fractional part: 12.5
os.put(-1);
[00034840:error] !!! put: bad numeric conversion: negative overflow: -1
os.put(256);
[00039175:error] !!! put: bad numeric conversion: positive overflow: 256
os.close;
\end{urbiscript}

\end{urbiscriptapi}


%%% Local Variables:
%%% mode: latex
%%% TeX-master: "../urbi-sdk"
%%% ispell-dictionary: "american"
%%% ispell-personal-dictionary: "../urbi.dict"
%%% fill-column: 76
%%% End:

\section{Pair}

A \dfn{pair} is a container storing two objects, similar in spirit to
\lstinline|std::pair| in \Cxx.


\begin{itemize}
\item \lstinline|asString|\\
  Generate the string \samp{(\var{first}, \var{second})} using
  \code{asPrintable} to convert members to strings.

\item \lstinline|first|\\
  Return the first member of the pair.

\item \lstinline|init(\var{first}, \var{second})|~\\
  Instantiate a new \lstinline|Pair| containing \var{first} and
  \var{second}.

\item \lstinline|second|\\
  Return the second member of the pair.

\item \lstinline|'[]'(\var{index})|\\
  Return the \var{index}-th element.  \var{index} must be 0 or 1.

\item \lstinline|'[]='(\var{index}, \var{value})|\\
  Set (and return) the \var{index}-th element to \var{value}.
  \var{index} must be 0 or 1.

\item \lstinline|'<'(\var{other})|\\
  Lexicographic comparison between two pairs.

\item \lstinline|'=='(\var{other})|\\
  Whether \lstinline|this| and \lstinline|other| have the same
  contents (equality-wise).
\end{itemize}

See below for an example.

\begin{urbiscript}
var p = Pair.new(1, 2);
[00000001] (1, 2)
p.first - p.second;
[00000002] -1
p[0] = 10 | p;
[00000002] (10, 2)
assert(Pair.new(0, 0) < Pair.new(0, 1));
assert(Pair.new(0, 0) < Pair.new(1, 0));
assert(Pair.new(0, 1) < Pair.new(1, 0));
assert(Pair.new(1, 2) == Pair.new(1, 2));
assert(!(Pair.new(1, 1) == Pair.new(2, 2)));
\end{urbiscript}


%%% Local Variables:
%%% mode: latex
%%% TeX-master: "../urbi-sdk"
%%% End:

\section{Path}

A \dfn{Path} points to a file system entity (directory, file and so
forth).

\subsection{Prototypes}
\begin{refObjects}
\item[Comparable]
\item[Orderable]
\end{refObjects}

\subsection{Construction}

A \lstinline|Path| is constructed with the string that points to the
file system entity. This path can be relative or absolute.

\begin{urbiscript}[firstnumber=1]
Path.new("/path/file.u");
[00000001] Path("/path/file.u")
\end{urbiscript}

Some minor simplifications are made, such as stripping useless
\file{./} occurrences.

\begin{urbiscript}
Path.new("././///.//foo/");
[00000002] Path("foo")
\end{urbiscript}

\subsection{Slots}
\begin{urbiscriptapi}
\item[absolute]
  Whether \lstinline|this| is absolute.
\begin{urbiassert}
Path.new("/abs/path").absolute;
!Path.new("rel/path").absolute;
\end{urbiassert}

\item[asList]
  List of names used in path (directories and possibly file), from
  bottom up. There is no difference between relative path and absolute
  path.
\begin{urbiassert}
Path.new("/path/to/file.u").asList == ["path", "to", "file.u"];
Path.new("/path").asList           == Path.new("path").asList;
\end{urbiassert}

\item[asPrintable]
\begin{urbiassert}
Path.new("file.txt").asPrintable == "Path(\"file.txt\")";
\end{urbiassert}

\item[asString]
  The name of the file.
\begin{urbiassert}
Path.new("file.txt").asString == "file.txt";
\end{urbiassert}

\item[basename]
  Base name of the path.
\begin{urbiassert}
Path.new("/absolute/path/file.u").basename == "file.u";
Path.new("relative/path/file.u").basename  == "file.u";
\end{urbiassert}

\item[cd]
  Change current working directory to \lstinline|this|. Return the new
  current working directory as a \lstinline|Path|.

\item[cwd]
  The current working directory.
% We used to write
% assert(Path.new("/").cd == Path.new("/"));
% assert(Path.cwd         == Path.new("/"));
% which is wrong on Windows, because cwd (like cd) returns Z:/ instead
% of /.
\begin{urbiscript}
{
  // Save current directory.
  var pwd = Path.cwd|
  // Go into ``/''.
  var root = Path.new("/").cd;
  // Current working directory is ``/''.
  assert(Path.cwd == root);
  // Go back to the directory we were in.
  assert(pwd.cd == pwd);
};
\end{urbiscript}

\item[dirname]
  Directory name of the path.
\begin{urbiassert}
Path.new("/abs/path/file.u").dirname == Path.new("/abs/path");
Path.new("rel/path/file.u").dirname  == Path.new("rel/path");
\end{urbiassert}

\item[exists]
  Whether something (a file, a directory, \ldots) exists where
  \lstinline|this| points to.
\begin{urbiassert}
Path.cwd.exists;
Path.new("/").exists;
!Path.new("/this/path/does/not/exists").exists;
\end{urbiassert}

\item[isDir]
  Whether \lstinline|this| is a directory.
\begin{urbiassert}
Path.cwd.isDir;
\end{urbiassert}

\item[isReg]
  Whether \lstinline|this| is a regular file.
\begin{urbiassert}
!Path.cwd.isReg;
\end{urbiassert}

\item[open]
  Open \lstinline|this|. Return either a \dfn{Directory} or a
  \dfn{File} according the type of \lstinline|this|. See
  \refObject{File} and \refObject{Directory}.

\item[readable]
  Whether \lstinline|this| is readable.  Throw if does not even exist.
\begin{urbiassert}
Path.new(".").readable;
\end{urbiassert}

\item[writable]
  Whether \lstinline|this| is writable.  Throw if does not even exist.
\begin{urbiassert}
Path.new(".").writable;
\end{urbiassert}

\item \lstinline|'/'(\var{rhs})|\\
  Create a new \dfn{Path} that is the concatenation of
  \lstinline|this| and \lstinline|\var{rhs}|. \lstinline|\var{rhs}|
  can be a \dfn{Path} or a \dfn{String} and cannot be absolute.
\begin{urbiscript}
assert(Path.new("/foo/bar") / Path.new("baz/qux/quux")
       == Path.new("/foo/bar/baz/qux/quux"));
Path.cwd / Path.new("/tmp/foo");
[00000003:error] !!! /: Rhs of concatenation is absolute: /tmp/foo
\end{urbiscript}

\item \lstinline|'=='(\var{that})|\\
  Same as comparing the string versions of \lstinline|this| and
  \var{that}.  Beware that two paths may be different and point to the
  very same location.
\begin{urbiassert}
  Path.new("/a")  == Path.new("/a");
!(Path.new("/a")  == Path.new("a")  );
\end{urbiassert}

\item \lstinline|'<'(\var{that})|\\
  Same as comparing the string versions of \lstinline|this| and
  \var{that}.
\begin{urbiassert}
  Path.new("/a")   < Path.new("/a/b");
!(Path.new("/a/b") < Path.new("/a")  );
\end{urbiassert}

\end{urbiscriptapi}


%%% Local Variables:
%%% mode: latex
%%% TeX-master: "../urbi-sdk"
%%% ispell-dictionary: "american"
%%% ispell-personal-dictionary: "../urbi.dict"
%%% fill-column: 76
%%% End:

%% Copyright (C) 2009-2010, Gostai S.A.S.
%%
%% This software is provided "as is" without warranty of any kind,
%% either expressed or implied, including but not limited to the
%% implied warranties of fitness for a particular purpose.
%%
%% See the LICENSE file for more information.

\section{Pattern}

\lstinline|Pattern| class is used to make correspondences between a pattern
and another \lstinline|Object|.  The visit is done either on the pattern or
on the element against which the pattern is compared.

\lstinline|Pattern|s are used for the implementation of the pattern matching.
So any class made compatible with the pattern matching implemented by this
class will allow you to use it implicitly in your scripts.

\begin{urbiscript}[firstnumber=1]
[1, var a, var b] = [1, 2, 3];
[00000000] [1, 2, 3]
a;
[00000000] 2
b;
[00000000] 3
\end{urbiscript}

\subsection{Prototypes}

\begin{refObjects}
\item[Object]
\end{refObjects}

\subsection{Construction}

A \lstinline|Pattern| can be created with any object that can be matched.

\begin{urbiscript}
Pattern.new([1]); // create a pattern to match the list [1].
[00000000] Pattern_0x189ea80
Pattern.new(Pattern.Binding.new("a")); // match anything into "a".
[00000000] Pattern_0x18d98b0
\end{urbiscript}

\subsection{Slots}

\begin{urbiscriptapi}
\item[Binding]
  A class used to create pattern variables.

\begin{urbiscript}
Pattern.Binding.new("a");
[00000000] var a
\end{urbiscript}


\item[bindings]

  A \lstinline|Dictionary| filled by the match function for each
  \refSlot{Binding} contained inside the pattern.

\begin{urbiscript}
{
  var p = Pattern.new([Pattern.Binding.new("a"), Pattern.Binding.new("b")]);
  assert (p.match([1, 2]));
  p.bindings
};
[00000000] ["a" => 1, "b" => 2]
\end{urbiscript}


\item[match](<value>)%

  Use \var{value} to unify the current pattern with this value.
  Return the status of the match.
  \begin{itemize}


\item[matchPattern](<pattern>, <value>)%

  This function is used as a callback function to store all bindings
  in the same place.  This function is useful inside objects that
  implement a \lstinline|match| or \lstinline|matchAgainst| function
  that need to continue the match deeper.  Return the status of the
  match (a Boolean).

  The \var{pattern} should provide a method
  \lstinline|match(\var{handler},\var{value})| otherwise the value method
  \lstinline|matchAgainst(\var{handler}, \var{pattern})| is used.  If none
  are provided the \lstinline|'=='| operator is used.

  To see how to use it, you can have a look at the implementation of
  \refSlot[List]{matchAgainst}.

%% This function is indirectly tested with the match of Pattern.Binding
%% inside lists.


\item[pattern]
  The pattern given at the creation.
\begin{urbiassert}
Pattern.new(1).pattern == 1;
Pattern.new([1, 2]).pattern == [1, 2];
{
  var pattern = [1, Pattern.Binding.new("a")];
  Pattern.new(pattern).pattern === pattern
};
\end{urbiassert}


\item If the match is correct, then the \var{bindings} member will
      contain the result of every matched values.
    \item If the match is incorrect, then the \var{bindings} member should
      not be used.
  \end{itemize}
  If the pattern contains multiple \refSlot{Binding} with the same name,
  then the behavior is undefined.

\begin{urbiassert}
Pattern.new(1).match(1);
Pattern.new([1, 2]).match([1, 2]);
! Pattern.new([1, 2]).match([1, 3]);
! Pattern.new([1, 2]).match([1, 2, 3]);
Pattern.new(Pattern.Binding.new("a")).match(0);
Pattern.new([1, Pattern.Binding.new("a")]).match([1, 2]);
! Pattern.new([1, Pattern.Binding.new("a")]).match(0);
\end{urbiassert}
\end{urbiscriptapi}

%%% Local Variables:
%%% mode: latex
%%% TeX-master: "../urbi-sdk"
%%% ispell-dictionary: "american"
%%% ispell-personal-dictionary: "../urbi.dict"
%%% fill-column: 76
%%% End:

%% Copyright (C) 2010, 2011, Gostai S.A.S.
%%
%% This software is provided "as is" without warranty of any kind,
%% either expressed or implied, including but not limited to the
%% implied warranties of fitness for a particular purpose.
%%
%% See the LICENSE file for more information.

\section{Position}

This class is used to handle file locations with a line, column and file
name.

\subsection{Prototypes}
\begin{refObjects}
\item[Object]
\end{refObjects}

\subsection{Construction}

Without argument, a newly constructed Position has its fields initialized to
the first line and the first column.

\begin{urbiscript}[firstnumber=1]
Position.new;
[00000001] 1.1
\end{urbiscript}

With a position argument \var{p}, the newly constructed Position is a clone
of \var{p}.

\begin{urbiscript}
Position.new(Position.new(2, 3));
[00000001] 2.3
\end{urbiscript}

With two float arguments \var{l} and \var{c}, the newly constructed
Position has its line and column defined and an empty file name.

\begin{urbiscript}
Position.new(2, 3);
[00000001] 2.3
\end{urbiscript}

With three arguments \var{f}, \var{l} and \var{c}, the newly
constructed Position has its file name, line and column defined.

\begin{urbiscript}
Position.new("file.u", 2, 3);
[00000001] file.u:2.3
\end{urbiscript}

\subsection{Slots}

\begin{urbiscriptapi}
\item['+'](<n>)%
  A new Position shifted from \var{n} columns to the right of \this.  The
  minimal value of the new position column is 1.
\begin{urbiassert}
Position.new(2, 3) + 2  == Position.new(2, 5);
Position.new(2, 3) + -4 == Position.new(2, 1);
\end{urbiassert}


\item['-'](<n>)%
  A new Position shifted from \var{n} columns to the left of \this.  The
  minimal value of the new Position column is 1.
\begin{urbiassert}
Position.new(2, 3) - 1  == Position.new(2, 2);
Position.new(2, 3) - -4 == Position.new(2, 7);
\end{urbiassert}


\item['<'](<that>)%
  Order comparison of lines and columns.
\begin{urbiassert}
Position.new(2, 3) < Position.new(2, 4);
Position.new(2, 3) < Position.new(3, 1);
\end{urbiassert}


\item['=='](<that>)%
  Compare the lines and columns of two Positions.
\begin{urbiassert}
Position.new(2, 3)        == Position.new(2, 3);
Position.new("a.u", 2, 3) == Position.new("b.u", 2, 3);
Position.new(2, 3)        != Position.new(2, 2);
\end{urbiassert}


\item[asString]
  Present as \samp{\var{file}:\var{line}.\var{column}}, the file name is
  omitted if it is not defined.
\begin{urbiassert}
Position.new("file.u", 2, 3).asString == "file.u:2.3";
\end{urbiassert}


\item[column]
  The column number of the Position.
\begin{urbiassert}
Position.new(2, 3).column == 3;
\end{urbiassert}


\item[columns](<n>)%
  Identical to \lstinline|'+'(\var{n})|.
\begin{urbiassert}
Position.new(2, 3).columns(2)  == Position.new(2, 5);
Position.new(2, 3).columns(-4) == Position.new(2, 1);
\end{urbiassert}


\item[file]
  The \refObject{Path} of the Position file.
\begin{urbiassert}
Position.new("file.u", 2, 3).file == Path.new("file.u");
Position.new(2, 3).file == nil;
\end{urbiassert}


\item[line]
  The line number of the Position.
\begin{urbiassert}
Position.new(2, 3).line == 2;
\end{urbiassert}


\item[lines](<n>)%
  Add \var{n} lines and reset the column number to 1.
\begin{urbiassert}
Position.new(2, 3).lines(2)  == Position.new(4, 1);
Position.new(2, 3).lines(-1) == Position.new(1, 1);
\end{urbiassert}
\end{urbiscriptapi}


%%% Local Variables:
%%% coding: utf-8
%%% mode: latex
%%% TeX-master: "../urbi-sdk"
%%% ispell-dictionary: "american"
%%% ispell-personal-dictionary: "../urbi.dict"
%%% fill-column: 76
%%% End:

\section{Primitive}
\Cxx routine callable from \us.

\subsection{Prototypes}
\begin{itemize}
\item \refObject{Executable}
\item \refObject{Object}
\end{itemize}

\subsection{Construction}

It is not possible to construct a Primitive.

\subsection{Slots}

\begin{urbiscriptapi}
\item \lstinline|apply(\var{args})|\\
  Invoke a primitive.  The argument list, \var{args}, must start with
  the target.
\begin{urbiassert}
Float.getSlot("+").isA(Global.getSlot("Primitive"));
Float.getSlot("+").apply([1, 2]) == 3;

String.getSlot("+").isA(Global.getSlot("Primitive"));
String.getSlot("+").apply(["1", "2"]);
\end{urbiassert}

\item[asPrimitive] Return \lstinline|this|.
\begin{urbiassert}
Float.getSlot("+").asPrimitive === Float.getSlot("+");
\end{urbiassert}
\end{urbiscriptapi}


%%% Local Variables:
%%% mode: latex
%%% TeX-master: "../urbi-sdk"
%%% ispell-personal-dictionary: "../urbi.dict"
%%% End:

%% Copyright (C) 2010, Gostai S.A.S.
%%
%% This software is provided "as is" without warranty of any kind,
%% either expressed or implied, including but not limited to the
%% implied warranties of fitness for a particular purpose.
%%
%% See the LICENSE file for more information.

\section{Process}

A Process is a separated task handled by the underneath operating
system.

\begin{windows}
  Process is not yet supported under Windows.
\end{windows}

\subsection{Prototypes}
\begin{itemize}
\item \refObject{Object}
\end{itemize}

\subsection{Example}

The following examples runs the \command{cat} program, a Unix standard
command that simply copies on its (standard) output its (standard)
input.

\begin{urbiscript}
var p = Process.new("cat", []);
[00000004] Process cat
\end{urbiscript}

\noindent
Just created, this process is not running yet.  Use \lstinline|run| to
launch it.

\begin{urbiscript}
p.status;
[00000005] not started

p.run;
p.status;
[00000006] running
\end{urbiscript}

\noindent
Then we feed its input, named \lstinline|stdin| in the Unix
tradition, and close its input.

\begin{urbiscript}
p.stdin << "1\n" |
p.stdin << "2\n" |
p.stdin << "3\n" |;

p.status;
[00000007] running

p.stdin.close;
\end{urbiscript}

\noindent
At this stage, the status of the process is unknown, as it is running
asynchronously.  If it has had enough time to ``see'' that its input
is closed, then it will have finished, otherwise we might have to wait
for awhile.  The method \lstinline|join| means ``wait for the process
to finish''.

\begin{urbiscript}
p.join;

p.status;
[00000008] exited with status 0
\end{urbiscript}

\noindent
Finally we can check its output.

\begin{urbiscript}
p.stdout.asList;
[00000009] ["1", "2", "3"]
\end{urbiscript}

\subsection{Construction}

A Process needs a program name to run and a possibly-empty list of
command line arguments.  Calling \lstinline|run| is required to
execute the process.

\begin{urbiscript}
Process.new("cat", []);
[00000004] Process cat

Process.new("cat", ["--version"]);
[00000004] Process cat
\end{urbiscript}

\subsection{Slots}

\begin{urbiscriptapi}
\item[asProcess] Return \this.
\begin{urbiscript}
do (Process.new("cat", []))
{
  assert (asProcess === this);
}|;
\end{urbiscript}

\item[asString] \lstinline|Process| and the name of the program.
\begin{urbiassert}
Process.new("cat", ["--version"]).asString
  == "Process cat";
\end{urbiassert}

\item[done] Whether the process has completed its execution.
\begin{urbiscript}
do (Process.new("sleep", ["1"]))
{
  assert (!done);
  run;
  assert (!done);
  join;
  assert (done);
}|;
\end{urbiscript}


\item[join] Wait for the process to finish.  Changes its status.
\begin{urbiscript}
do (Process.new("sleep", ["2"]))
{
  var t0 = System.time;
  assert (status.asString == "not started");
  run;
  assert (status.asString == "running");
  join;
  assert (t0 + 2s <= System.time);
  assert (status.asString == "exited with status 0");
}|;
\end{urbiscript}

\item[kill] If the process is not \lstinline|done|, interrupt it (with
  a \lstinline|SIGKILL| in Unix parlance).  You still have to wait for
  its termination with \lstinline|join|.
\begin{urbiscript}
do (Process.new("sleep", ["1"]))
{
  run;
  kill;
  join;
  assert (done);
  assert (status.asString == "killed by signal 9");
}|;
\end{urbiscript}


\item[name] The (base) name of the program the process runs.
\begin{urbiassert}
Process.new("cat", ["--version"]).name == "cat";
\end{urbiassert}

\item[run] Launch the process.  Changes it status.  A process can only
  be run once.
\begin{urbiscript}
do (Process.new("sleep", ["1"]))
{
  assert (status.asString == "not started");
  run;
  assert (status.asString == "running");
  join;
  assert (status.asString == "exited with status 0");
  run;
}|;
[00021972:error] !!! run: Process was already run
\end{urbiscript}

\item[runTo]
  %%% FIXME:
\item[status] An object whose slots describe the status of the
  process.
  %%% FIXME:
\item[stderr] An \refObject{InputStream} (the output of the Process is
  an input for \urbi) to the standard error stream of the process.
\begin{urbiscript}
do (Process.new("urbi-send", ["--no-such-option"]))
{
  run;
  join;
  assert
  {
    stderr.asList ==
    ["urbi-send: invalid option: --no-such-option",
     "Try `urbi-send --help' for more information."];
  };
}|;
\end{urbiscript}

\item[stdin] An \refObject{OutputStream} (the input of the Process is
  an output for \urbi) to the standard input stream of the process.
\begin{urbiscript}
do (Process.new(System.programName, ["--version"]))
{
  run;
  join;
  assert
  {
    stdout.asList[1] == "Copyright (C) 2004-2011 Gostai S.A.S..";
  };
}|;
\end{urbiscript}

\item[stdout] An \refObject{InputStream} (the output of the Process is
  an input for \urbi) to the standard output stream of the process.
\begin{urbiscript}
do (Process.new("cat", []))
{
  run;
  stdin << "Hello, World!\n";
  stdin.close;
  join;
  assert (stdout.asList == ["Hello, World!"]);
}|;
\end{urbiscript}
\end{urbiscriptapi}


%%% Local Variables:
%%% coding: utf-8
%%% mode: latex
%%% TeX-master: "../urbi-sdk"
%%% ispell-dictionary: "american"
%%% ispell-personal-dictionary: "../urbi.dict"
%%% fill-column: 76
%%% End:

%% Copyright (C) 2010, Gostai S.A.S.
%%
%% This software is provided "as is" without warranty of any kind,
%% either expressed or implied, including but not limited to the
%% implied warranties of fitness for a particular purpose.
%%
%% See the LICENSE file for more information.

\section{PubSub}

\lstinline|PubSub| provides an abstraction over \lstinline|Barrier|
\refObject{Barrier} to queue signals for each subscriber.

\subsection{Prototypes}

\begin{refObjects}
\item[Object]
\end{refObjects}

\subsection{Construction}

A \lstinline|PubSub| can be created with no arguments.  Values can be
published and read by each subscriber.

\begin{urbiscript}[firstnumber=1]
var ps = PubSub.new();
[00000000] PubSub_0x28c1bc0
\end{urbiscript}

\subsection{Slots}

\begin{urbiscriptapi}
\item[publish](<ev>)%
  Queue the value \var{ev} to the queue of each subscriber.  This method
  returns the value \var{ev}.

\begin{urbiscript}
{
  var sub = ps.subscribe();
  assert
  {
    ps.publish(2) == 2;
    sub.getOne() == 2;
  };
  ps.unsubscribe(sub)
}|;
\end{urbiscript}


\item[subscribe] Create a \refSlot{Subscriber} and insert it inside the list
  of subscribers.

\begin{urbiscript}
var sub = ps.subscribe() |
ps.subscribers == [sub];
[00000000] true
\end{urbiscript}


\item[Subscriber] See \refObject{PubSub.Subscriber}.


\item[subscribers] Field containing the list of \refSlot{Subscriber} which
  are watching published values.  This field only exists in instances of
  \lstinline|PubSub|.


\item[unsubscribe](<sub>)%
  Remove a subscriber from the list of subscriber watching the published
  values.

\begin{urbiscript}
ps.unsubscribe(sub) |
ps.subscribers;
[00000000] []
\end{urbiscript}
\end{urbiscriptapi}

%%% Local Variables:
%%% coding: utf-8
%%% mode: latex
%%% TeX-master: "../urbi-sdk"
%%% ispell-dictionary: "american"
%%% ispell-personal-dictionary: "../urbi.dict"
%%% fill-column: 76
%%% End:

%% Copyright (C) 2010, Gostai S.A.S.
%%
%% This software is provided "as is" without warranty of any kind,
%% either expressed or implied, including but not limited to the
%% implied warranties of fitness for a particular purpose.
%%
%% See the LICENSE file for more information.

\section{PubSub.Subscriber}

\lstinline|Subscriber| is created by \lstinline|PubSub.subscribe|.  It
provides methods to access to the list of values published by
\lstinline|PubSub| instances.

\subsection{Prototypes}

\begin{refObjects}
\item[Object]
\end{refObjects}

\subsection{Construction}

A \lstinline|PubSub.Subscriber| can be created with a call to
\lstinline|PubSub.subscribe|.  This way of creating a
\lstinline|Subscriber| adds the subscriber as a watcher of values
published on the instance of \lstinline|PubSub|.

\begin{urbiscript}[firstnumber=1]
var ps = PubSub.new |;
var sub = ps.subscribe;
[00000000] Subscriber_0x28607c0
\end{urbiscript}

%% PubSub.Subscriber.new is not documented because there is no method to add
%% a subscriber into the list of subscribers of a PubSub instance except by
%% doing it in a dirty way.  This may allow to have a subscriber watching
%% multiple published values.
\subsection{Slots}

\begin{urbiscriptapi}

%% \lstinline|enqueue(\var{ev})| is not documented because I assume it is an
%% implementation detail of PubSub.
\item[getOne]
  Block until a value is accessible and return it.  If a value is already
  queued, then the method returns it without blocking.

\begin{urbiscript}
echo(sub.getOne) &
ps.publish(3);
[00000000] *** 3
\end{urbiscript}


\item[getAll] Block until a value is accessible.  Return the list of
  queued values.  If the values are already queued, then return them
  without blocking.

\begin{urbiscript}
ps.publish(4) |
ps.publish(5) |
echo(sub.getAll);
[00000000] *** [4, 5]
\end{urbiscript}

\end{urbiscriptapi}

%%% Local Variables:
%%% mode: latex
%%% TeX-master: "../../urbi-sdk"
%%% ispell-dictionary: "american"
%%% ispell-personal-dictionary: "../../urbi.dict"
%%% fill-column: 76
%%% End:

%% Copyright (C) 2010, Gostai S.A.S.
%%
%% This software is provided "as is" without warranty of any kind,
%% either expressed or implied, including but not limited to the
%% implied warranties of fitness for a particular purpose.
%%
%% See the LICENSE file for more information.

\section{Profiling}

\lstinline|Profiling| is useful to get an idea of the efficiency of some
small pieces of code.

\subsection{Prototypes}

\begin{refObjects}
\item[Object]
\end{refObjects}

\subsection{Construction}

A \lstinline|Profiling| can be created with two arguments.  The first
argument is the expression which has to be profiled and the second is the
number of iteration it should be run.

Creating a \lstinline|Profiling| session prints the result of the profiled
expression, the number of iterations, the number of cycles and the time of
the evaluation.  The number of cycles corresponds to the number of time the
job is scheduled.

\begin{urbiunchecked}[firstnumber=1]
Profiling.new({1| 2| 3| 4}, 10000);
[00000000] Profiling information
  Expression:       1 | 2 | 3 | 4
  Iterations:       10000
  Cycles:           10000
  Total time:       1.00098 s
  Single iteration: 0.000100098 s
                    1 cycles


Profiling.new({1; 2; 3; 4}, 10000);
[00000000] Profiling information
  Expression:       1;
2;
3;
4
  Iterations:       10000
  Cycles:           40000
  Total time:       1.45856 s
  Single iteration: 0.000145856 s
                    4 cycles
\end{urbiunchecked}

%% \subsection{Slots}

%% \begin{urbiscriptapi}
%% \item \lstinline|timen(\var{expr}, \var{niter})|
%%   Profile the evaluation of the expression.  The reported message is bugged
%%   when it reports the expression.
%% \end{urbiscriptapi}
%%% Local Variables:
%%% mode: latex
%%% TeX-master: "../urbi-sdk"
%%% ispell-dictionary: "american"
%%% ispell-personal-dictionary: "../urbi.dict"
%%% fill-column: 76
%%% End:

\section{RangeIterable}

This object is meant to be used as a prototype for objects that
support an \lstinline|asList| method, to use range-based
\lstinline|for| loops (\autoref{sec:lang:for:each}).

\subsection{Prototypes}

\begin{itemize}
\item \refObject{Object}
\end{itemize}

\subsection{Slots}

\begin{itemize}
\item \lstinline|each(\var{fun})|\\
  Apply the given functional value \var{fun} on all ``members'',
  sequentially.  Corresponds to range-\lstinline|for| loops.
\begin{urbiscript}[firstnumber=1]
class range : RangeIterable
{
  var asList = [10, 20, 30];
}|;
for (var i : range)
  echo (i);
[00000000] *** 10
[00000000] *** 20
[00000000] *** 30
\end{urbiscript}

\item \lstinline|each&(\var{fun})|\\
  Apply the given functional value \var{fun} on all ``members'', in
  parallel, starting all the computations simultaneously.  Corresponds
  to range-\lstinline|for&| loops.
\begin{urbiscript}
{
  var res = [];
  for& (var i : range)
    res << i;
  assert(res.sort == [10, 20, 30]);
};
\end{urbiscript}

\item \lstinline|each|(\var{fun})|\\
  Apply the given functional value \var{fun} on all ``members'', with
  tight sequentially.  Corresponds to range-\lstinline'for|' loops.
\begin{urbiscript}
{
  var res = [];
  for| (var i : range)
    res << i;
  assert(res == [10, 20, 30]);
};
\end{urbiscript}
\end{itemize}

%%% Local Variables:
%%% mode: latex
%%% TeX-master: "../urbi-sdk"
%%% ispell-dictionary: "american"
%%% ispell-personal-dictionary: "../urbi.dict"
%%% End:

%% Copyright (C) 2009-2010, Gostai S.A.S.
%%
%% This software is provided "as is" without warranty of any kind,
%% either expressed or implied, including but not limited to the
%% implied warranties of fitness for a particular purpose.
%%
%% See the LICENSE file for more information.

\section{Regexp}

A Regexp is an object which allow you to match strings with a regular
expression.

\subsection{Prototypes}
\begin{itemize}
\item \refObject{Container}
\item \refObject{Object}
\end{itemize}

\subsection{Construction}
\label{stdlib:regexp:ctor}

A \lstinline{Regexp} is created with the regular expression once and
for all, and it can be used many times to match with other strings.

\begin{urbiscript}[firstnumber=1]
Regexp.new(".");
[00000001] Regexp(".")
\end{urbiscript}

\us supports Perl regular expressions, see
\href{http://perldoc.perl.org/perlre.html}{the perlre man page}.
Expressions cannot be empty.

\subsection{Slots}
\begin{urbiscriptapi}
\item[asPrintable] A string that shows that \this is a Regexp, and its
  value.
\begin{urbiassert}
           Regexp.new("abc").asPrintable == "Regexp(\"abc\")";
Regexp.new("\\d+(\\.\\d+)?").asPrintable == "Regexp(\"\\\\d+(\\\\.\\\\d+)?\")";
\end{urbiassert}

\item[asString] The regular expression that was compiled.
\begin{urbiassert}
           Regexp.new("abc").asString == "abc";
Regexp.new("\\d+(\\.\\d+)?").asString == "\\d+(\\.\\d+)?";
\end{urbiassert}

\item[has](<str>)%
  An experimental alias to \refSlot{match}, so that the infix operators
  \lstinline|in| and \lstinline|not in| can be used (see
  \autoref{sec:lang:operators:containers}).
\begin{urbiassert}
"23.03"     in Regexp.new("^\\d+\\.\\d+$");
"-3.14" not in Regexp.new("^\\d+\\.\\d+$");
\end{urbiassert}

\item[match](<str>)%
  Whether \this matches \var{str}.
\begin{urbiscript}
// Ordinary characters
var r = Regexp.new("oo")|
assert
{
  r.match("oo");
  r.match("foobar");
  !r.match("bazquux");
};

// ^, anchoring at the beginning of line.
r = Regexp.new("^oo")|
assert
{
  r.match("oops");
  !r.match("woot");
};

// $, anchoring at the end of line.
r = Regexp.new("oo$")|
assert
{
  r.match("foo");
  !r.match("mooh");
};

// *, greedy repetition, 0 or more.
r = Regexp.new("fo*bar")|
assert
{
  r.match("fbar");
  r.match("fooooobar");
  !r.match("far");
};

// (), grouping.
r = Regexp.new("f(oo)*bar")|
assert
{
  r.match("foooobar");
  !r.match("fooobar");
};
\end{urbiscript}

\item[matches]%
  If the latest \refSlot{match} was successful, the matched groups, as
  delimited by parentheses in the regular expression; the first element
  being the whole match.  Otherwise, the empty list.  See also
  \refSlot{'[]'}.

\begin{urbiscript}
var re = Regexp.new("([a-zA-Z0-9._]+)@([a-zA-Z0-9._]+)")|;
assert
{
  re.match("Someone <someone@somewhere.com>");
  re.matches == ["someone@somewhere.com", "someone", "somewhere.com"];

  "does not match" not in re;
  re.matches == [];
};
\end{urbiscript}

\item|'[]'|(<n>)%
  Same as \lstinline|this.matches[\var{n}]|.
\begin{urbiscript}
var d = Regexp.new("(1+)(2+)(3+)")|;
assert
{
  "01223334" in d;
  d[0] == "122333";
  d[1] == "1";
  d[2] == "22";
  d[3] == "333";
};
d[4];
[00000009:error] !!! []: out of bound index: 4
\end{urbiscript}

\end{urbiscriptapi}

%%% Local Variables:
%%% mode: latex
%%% TeX-master: "../urbi-sdk"
%%% ispell-dictionary: "american"
%%% ispell-personal-dictionary: "../urbi.dict"
%%% fill-column: 76
%%% End:

%% Copyright (C) 2010, Gostai S.A.S.
%%
%% This software is provided "as is" without warranty of any kind,
%% either expressed or implied, including but not limited to the
%% implied warranties of fitness for a particular purpose.
%%
%% See the LICENSE file for more information.

\section{Semaphore}

\lstinline|Semaphore| are useful to limit the number of access to a limited
number of resources.

\subsection{Prototypes}

\begin{refObjects}
\item[Object]
\end{refObjects}

\subsection{Construction}

A \lstinline|Semaphore| can be created with as argument the number of
processes allowed to enter critical sections at the same time.

\begin{urbiscript}[firstnumber=1]
Semaphore.new(1);
[00000000] Semaphore_0x8c1e80
\end{urbiscript}

\subsection{Slots}

\begin{urbiscriptapi}

\item \lstinline|criticalSection(function () {\var{code}})|
  Put the piece of \var{code} inside a critical section which can be
  executed simultaneously at most the number of time given at the creation
  of the \lstinline|Semaphore|.  This method is similar to a call to
  \lstinline|acquire| and a call to \lstinline|release| when the code ends
  by any means.

\begin{urbiscript}
{
  var s = Semaphore.new(1);
  for& (var i : [0, 1, 2, 3])
  {
    s.criticalSection(function () {
      echo("start " + i);
      echo("end " + i);
    })
  }
};
[00000000] *** start 0
[00000000] *** end 0
[00000000] *** start 1
[00000000] *** end 1
[00000000] *** start 2
[00000000] *** end 2
[00000000] *** start 3
[00000000] *** end 3


{
  var s = Semaphore.new(2);
  for& (var i : [0, 1, 2, 3])
  {
    s.criticalSection(function () {
      echo("start " + i);

      // Illustrate that processes can be intertwined
      sleep(i * 100ms);

      echo("end " + i);
    })
  }
};
[00000000] *** start 0
[00000000] *** start 1
[00000000] *** end 0
[00000000] *** start 2
[00000000] *** end 1
[00000000] *** start 3
[00000000] *** end 2
[00000000] *** end 3
\end{urbiscript}


\item[acquire] Wait to enter a critical section delimited by the execution
  of \lstinline|acquire| and \lstinline|release|.  Enter the critical
  section when the number of processes inside it goes below the maximum
  allowed.

\item[p] Historical synonym for \lstinline|acquire|.

\item[release] Leave a critical section delimited by the execution of
  \lstinline|p| and \lstinline|v|.

\begin{urbiscript}
{
  var s = Semaphore.new(1);
  for& (var i : [0, 1, 2, 3])
  {
    s.acquire;
    echo("start " + i);
    echo("end " + i);
    s.release;
  }
};
[00000000] *** start 0
[00000000] *** end 0
[00000000] *** start 1
[00000000] *** end 1
[00000000] *** start 2
[00000000] *** end 2
[00000000] *** start 3
[00000000] *** end 3
\end{urbiscript}

\item[v] Historical synonym for \lstinline|release|.
\end{urbiscriptapi}

%%% Local Variables:
%%% mode: latex
%%% TeX-master: "../urbi-sdk"
%%% ispell-dictionary: "american"
%%% ispell-personal-dictionary: "../urbi.dict"
%%% fill-column: 76
%%% End:

%% Copyright (C) 2009-2010, Gostai S.A.S.
%%
%% This software is provided "as is" without warranty of any kind,
%% either expressed or implied, including but not limited to the
%% implied warranties of fitness for a particular purpose.
%%
%% See the LICENSE file for more information.

\section{Server}

A \dfn{Server} can listen to incoming connections.  See
\refObject{Socket} for an example.

\subsection{Prototypes}
\begin{refObjects}
\item[Object]
\end{refObjects}

\subsection{Construction}

A \lstinline|Server| is constructed with no argument. At creation, a
new \lstinline|Server| has its own slot \lstinline|connection|. This
slot is an event that is launched when a connection establishes.

\begin{urbiscript}
var s = Server.new|
s.localSlotNames;
[00000001] ["connection"]
\end{urbiscript}

\subsection{Slots}
\begin{urbiscriptapi}
\item[connection]
  The event launched at each incoming connection. This event is
  launched with one argument: the socket of the established connection.
\begin{urbiscript}
at (s.connection?(var socket))
{
  // This code is run at each connection. 'socket' is the incoming
  // connection.
};
\end{urbiscript}

\item[host]
  The host on which \lstinline|this| is listening. Raise an error if
  \lstinline|this| is not listening.
\begin{urbiscript}
Server.host;
[00000003:error] !!! host: server not listening
\end{urbiscript}

\item \lstinline|listen(\var{host}, \var{port})|\\
  Listen incoming connections with \var{host} and \var{port}.

\item[port]
  The port on which \lstinline|this| is listening. Raise an error if
  \lstinline|this| is not listening.
\begin{urbiscript}
Server.port;
[00000004:error] !!! port: server not listening
\end{urbiscript}

\item[sockets]
  The list of the sockets created at each incoming connection.
\end{urbiscriptapi}

%%% Local Variables:
%%% mode: latex
%%% TeX-master: "../urbi-sdk"
%%% ispell-dictionary: "american"
%%% ispell-personal-dictionary: "../urbi.dict"
%%% fill-column: 76
%%% End:

%% Copyright (C) 2009-2010, Gostai S.A.S.
%%
%% This software is provided "as is" without warranty of any kind,
%% either expressed or implied, including but not limited to the
%% implied warranties of fitness for a particular purpose.
%%
%% See the LICENSE file for more information.

\section{Singleton}

A \dfn{singleton} is a prototype that cannot be cloned. All prototypes
derived of \lstinline{Singleton} are also singletons.

\subsection{Prototypes}
\begin{refObjects}
\item[Object]
\end{refObjects}

\subsection{Construction}

To be a singleton, the object must have \lstinline{Singleton} as a
prototype. The common way to do this is
%
\lstinline{var s = Singleton.new},
%
but this does not work : \lstinline|s| is not a new singleton, it is
the \lstinline|Singleton| itself since it cannot be cloned. There are
two other ways:

\begin{urbiscript}[firstnumber=1]
// Defining a new class and specifying Singleton as a parent.
class NewSingleton1: Singleton
{
  var asString = "NewSingleton1";
}|
var s1 = NewSingleton1.new;
[00000001] NewSingleton1
assert(s1 === NewSingleton1);
assert(NewSingleton1 !== Singleton);

// Create a new Object and set its prototype by hand.
var NewSingleton2 = Object.new|
var NewSingleton2.asString = "NewSingleton2"|
NewSingleton2.protos = [Singleton]|
var s2 = NewSingleton2.new;
[00000001] NewSingleton2
assert(s2 === NewSingleton2);
assert(NewSingleton2 !== Singleton);
\end{urbiscript}

\subsection{Slots}
\begin{urbiscriptapi}
\item[clone]
  Return \lstinline|this|.

\item \lstinline|'new'|\\
  Return \lstinline|this|.
\end{urbiscriptapi}

%%% Local Variables:
%%% mode: latex
%%% TeX-master: "../urbi-sdk"
%%% ispell-dictionary: "american"
%%% ispell-personal-dictionary: "../urbi.dict"
%%% fill-column: 76
%%% End:

\section{Socket}

A \dfn{Socket} can manage asynchronous input/output network
connections.

\subsection{Example}

The following example demonstrates how both the \refObject{Server} and
\refObject{Socket} object work.

This simple example will establish a dialogue between
\lstinline|server| and \lstinline|client|.  The following object,
\lstinline|Dialogue|, contains the script of this exchange.  It is put
into \lstinline|Global| so that both the server and client can read
it.  \lstinline|Dialogue.reply(var \var{s})| returns the reply to a
message \var{s}.

\begin{urbiscript}
class Global.Dialogue
{
  var lines =
  [
    "Hi!",
    "Hey!",
    "Hey you doin'?",
    "Whazaaa!",
    "See ya.",
  ]|;

  function reply(var s)
  {
    for (var i: lines.size - 1)
      if (s == lines[i])
        return lines[i + 1];
    "off";
  }
}|;
\end{urbiscript}

The server, an instance of \refObject{Server}, expects incoming
connections, notified by the socket's \lstinline|connection?| event.
Once the connection establish, it listens to the \lstinline|socket|
for incoming messages, notified by the \lstinline|received?| event.
Its reaction to this event is to send the following line of the
dialogue.  At the end of the dialogue, the socket is disconnected.

\begin{urbiscript}
var server =
  do (Server.new)
  {
    at (connection?(var socket))
      at (socket.received?(var data))
      {
        var reply = Dialogue.reply(data);
        socket.write(reply);
        echo("server: " + reply);
        if (reply == "off")
          socket.disconnect;
      };
  }|;
\end{urbiscript}

The client, an instance of \refObject{Socket} expects incoming
messages, notified by the \lstinline|received?| event.  Its reaction
is to send the following line of the dialogue.

\begin{urbiscript}
var client =
  do (Socket.new)
  {
    at (received?(var data))
    {
      var reply = Dialogue.reply(data);
      write(reply);
      echo("client: " + reply);
    };
  }|;
\end{urbiscript}

As of today, \us's socket machinery requires to be regularly polled.

\begin{urbiscript}
every (100ms)
  Socket.poll,
\end{urbiscript}

The server is then activated, listening to incoming connections on a
port that will be chosen by the system amongst the free ones.

\begin{urbiscript}
server.listen("localhost", "0");
clog << "connecting to %s:%s" % [server.host, server.port];
\end{urbiscript}

The client connects to the server, and initiates the dialogue.

\begin{urbiscript}
client.connect(server.host, server.port);
echo("client: " + Dialogue.lines[0]);
client.write(Dialogue.lines[0]);
[00000003] *** client: Hi!
\end{urbiscript}

Because this dialogue is asynchronous, the easiest way to wait for the
dialogue to finish is to wait for the \lstinline|disconnected?| event.

\begin{urbiscript}
waituntil(client.disconnected?);
[00000004] *** server: Hey!
[00000005] *** client: Hey you doin'?
[00000006] *** server: Whazaaa!
[00000007] *** client: See ya.
[00000008] *** server: off
\end{urbiscript}

\subsection{Prototypes}
\begin{itemize}
\item \refObject{Object}
\end{itemize}

\subsection{Construction}

A \lstinline|Socket| is constructed with no argument. At creation, a
new \lstinline|Socket| has four own slots: \lstinline|connected|,
\lstinline|disconnected|, \lstinline|error| and \lstinline|received|.

\begin{urbiscript}
var s = Socket.new|
\end{urbiscript}

\subsection{Slots}
\begin{itemize}

\item \lstinline|connect(\var{host}, \var{port})|\\
  Connect \lstinline|this| to \var{host} and \var{port}.  The
  \var{port} can be either an integer, or a string that denotes
  symbolic ports, such as \lstinline|"smtp"|, or \lstinline|"ftp"| and
  so forth.

\item \lstinline|connected|\\
  Event launched when the connection is established.

\item \lstinline|connectSerial(\var{device}, \var{baudrate})|\\
  Connect \lstinline|this| to the serial port \var{device}, with given
  \var{baudrate}.

\item \lstinline|disconnect|\\
  Close the connection.

\item \lstinline|disconnected|\\
  Event launched when a disconnection happens.

\item \lstinline|error|\\
  Event launched when an error happens. This event is launched with
  the error message in argument. The event \lstinline|disconnected| is
  also always launched.

\item \lstinline|host|\\
  The host of the connection.

\item \lstinline|isConnected|\\
  Whether \lstinline|this| is connected.

\item \lstinline|poll|\\
  Runs the event processing loop to execute ready handlers. By
  default, the loop is executed every \lstinline|pollInterval|.

\item \lstinline|pollInterval|\\
  Each \lstinline|pollInterval| amout of time, \lstinline|Socket.poll|
  is called. If \lstinline|pollInterval| equals zero,
  \lstinline|Socket.poll| is not called.

\item \lstinline|port|\\
  The port of the connection.

\item \lstinline|received|\\
  Event launched when \lstinline|this| has received data. The data is
  given by argument to the event.

\item \lstinline|write(\var{data})|\\
  Sends \var{data} trough the connection.

\end{itemize}

%%% Local Variables:
%%% mode: latex
%%% TeX-master: "../urbi-sdk"
%%% ispell-personal-dictionary: "../urbi.dict"
%%% End:

\section{String}

A \dfn{string} is a sequence of characters.

\subsection{Prototypes}
\begin{itemize}
\item \refObject{Comparable}
\item \refObject{Orderable}
\item \refObject{RangeIterable}
\end{itemize}

\subsection{Construction}
Fresh Strings can easily be built using the literal syntax.  Several
escaping sequences (the traditional ones and \us specific ones) allow
to insert special characters.  Consecutive string literals are merged
together.  See \autoref{sec:us-syn-lit-string} for details and
examples.

A null String can also be obtained with \lstinline|String|'s
\lstinline|new| method.

\begin{urbiscript}
assert(String.new == "");
assert(String == "");
assert("123".new == "123");
\end{urbiscript}

\subsection{Slots}
\begin{itemize}
\item \lstinline|asFloat|\\
  If the whole content of \lstinline|this| is an integer, return its
  value, otherwise return an error.
\begin{urbiscript}[firstnumber=last]
assert("23.03".asFloat == 23.03);

"123abc".asFloat;
[00000001:error] !!! asFloat: unable to convert to float: "123abc"
\end{urbiscript}

\item \lstinline|asList|\\
  Return a List of one-letter Strings that, concataneted, equal
  \lstinline|this|.  This allows to use \lstinline|for| to iterate
  over the string.
\begin{urbiscript}[firstnumber=last]
assert("123".asList == ["1", "2", "3"]);
for (var v : "123")
  echo(v);
[00000001] *** 1
[00000001] *** 2
[00000001] *** 3
\end{urbiscript}

\item \lstinline|asPrintable|\\
  Return \lstinline|this| as a literal (escaped) string.
\begin{urbiscript}[firstnumber=last]
assert("foo".asPrintable == "\"foo\"");
assert("foo".asPrintable.asPrintable == "\"\\\"foo\\\"\"");
\end{urbiscript}

\item \lstinline|asString|\\
  Return \lstinline|this|.
\begin{urbiscript}[firstnumber=last]
assert("\"foo\"".asString == "\"foo\"");
\end{urbiscript}

\item \lstinline|distance(\var{other})|\\
  Return the
  \href{http://en.wikipedia.org/wiki/Damerau-Levenshtein_distance}
  {Damerau-Levenshtein distance} between \lstinline|this| and
  \var{other}.  The more alike the strings are, the smaller the
  distance is.
\begin{urbiscript}[firstnumber=last]
assert("foo".distance("foo") == 0);
assert("bar".distance("baz") == 1);
assert("foo".distance("bar") == 3);
\end{urbiscript}

\item \lstinline|fresh|\\
  Return a String that has never been used as an identifier, prefixed
  by \lstinline|this|.  It can safely be used with
  \lstinline|Object.setSlot| and so forth.
\begin{urbiscript}[firstnumber=last]
assert(String.fresh == "_5");
assert("foo".fresh == "foo_6");
\end{urbiscript}

\item Character handling functions\\
  Here is a map of how the original 127-character ASCII set is
  considered by each function (a \textbullet{} indicates that the function
  returns true if all characters of \lstinline|this| are on the
  row).

\begin{tabular}{|l||l||c|c|c|c|c|c|c|c|c|c|c|}
  \hline
  &&&&&&&&&&&&\\
  ASCII values & Characters & \begin{sideways}iscntrl\end{sideways}
    & \begin{sideways}isspace\end{sideways}
    & \begin{sideways}isupper\end{sideways}
    & \begin{sideways}islower\end{sideways}
    & \begin{sideways}isalpha\end{sideways}
    & \begin{sideways}isdigit\end{sideways}
    & \begin{sideways}isxdigit\end{sideways}
    & \begin{sideways}isalnum\end{sideways}
    & \begin{sideways}ispunct\end{sideways}
    & \begin{sideways}isgraph\end{sideways}
    & \begin{sideways}print \end{sideways}\\ \hline \hline
  0x00 .. 0x08 & & \textbullet & & & & & & & & & &\\ \hline
  0x09 .. 0x0D & \textbackslash{}t, \textbackslash{}f,
  \textbackslash{}v, \textbackslash{}n, \textbackslash{}r &
  \textbullet & \textbullet & & & & & & & & &\\ \hline
  0x0E .. 0x1F & & \textbullet & & & & & & & & & &\\ \hline
  0x20 & space (' ') & & \textbullet & & & & & & & & & \textbullet\\ \hline
  0x21 .. 0x2F & \verb|!"#$%&'()*+,-./| & & & & & & & & & \textbullet & \textbullet & \textbullet\\ \hline
  0x30 .. 0x39 & \verb|0-9| & & & & & & \textbullet & \textbullet & \textbullet & & \textbullet & \textbullet\\ \hline
  0x3a .. 0x40 & \verb|:;<=>?@| & & & & & & & & & \textbullet & \textbullet & \textbullet\\ \hline
  0x41 .. 0x46 & \verb|A-F| & & & \textbullet & & \textbullet & & \textbullet & \textbullet & & \textbullet & \textbullet\\ \hline
  0x47 .. 0x5A & \verb|G-Z| & & & \textbullet & & \textbullet & & & \textbullet & & \textbullet & \textbullet\\ \hline
  0x5B .. 0x60 & \verb|[\]^{}_`| & & & & & & & & & \textbullet & \textbullet & \textbullet\\ \hline
  0x61 .. 0x66 & \verb|a-f| & & & & \textbullet & \textbullet & & \textbullet & \textbullet & & \textbullet & \textbullet\\ \hline
  0x67 .. 0x7A & \verb|g-z| & & & & \textbullet & \textbullet & & & \textbullet & & \textbullet & \textbullet\\ \hline
  0x7B .. 0x7E & \verb-{|}~- & & & & & & & & & \textbullet & \textbullet & \textbullet\\ \hline
  0x7F & (DEL) &  \textbullet & & & & & & & & & &\\
  \hline
\end{tabular}

\begin{urbiscript}[firstnumber=last]
assert("".isDigit);
assert("0123456789".isDigit);
assert(!"a".isDigit);

assert("".isLower);
assert("lower".isLower);
assert(! "Not Lower".isLower);

assert("".isUpper);
assert("UPPER".isUpper);
assert(! "Not Upper".isUpper);
\end{urbiscript}

\item \lstinline|join(\var{list}, \var{prefix}, \var{suffix})|\\
  Glue the result of \lstinline|asString| applied to the members of
  \var{list}, separated by \lstinline|this|, and embedded in a pair
  \var{prefix}/var{suffix}.
\begin{urbiscript}[firstnumber=last]
assert("|".join([1, 2, 3], "(", ")")      == "(1|2|3)");
assert(", ".join([1, [2], "3"], "[", "]") == "[1, [2], 3]");
\end{urbiscript}

\item \lstinline|replace(\var{from}, \var{to})|\\
  Replace every occurrence of the string \var{from} in
  \lstinline|this| by \var{to}, and return the result.
  \lstinline|this| is not modified.
\begin{urbiscript}[firstnumber=last]
assert("Hello, World!".replace("Hello", "Bonjour")
                      .replace("World!", "Monde !") ==
       "Bonjour, Monde !");
\end{urbiscript}

\item \lstinline|size|\\
  Return the size of the string.
\begin{urbiscript}[firstnumber=last]
assert("foo".size == 3);
assert("".size == 0);
\end{urbiscript}

\item \lstinline|toLower|\\
  Make lower case every upper case character in \lstinline|this| and
  return the result.  \lstinline|this| is not modified.
\begin{urbiscript}[firstnumber=last]
assert("Hello == World!".toLower, "hello, world!");
\end{urbiscript}

\item \lstinline|toUpper|\\
  Make upper case every lower case character in \lstinline|this| and
  return the result.  \lstinline|this| is not modified.
\begin{urbiscript}[firstnumber=last]
assert("Hello == World!".toUpper, "HELLO, WORLD!");
\end{urbiscript}

\item \lstinline|'%'(\var{args})|\\
  It is an equivalent of \lstinline|Formatter.new(this) % \var{args}|.
  See \refObject{Formatter}.
%  This construct is actually more
%  powerful than this, since it relies on
%  \href{http://www.boost.org/doc/libs/1_39_0/libs/format/doc/format.html,
%    Boost.Format}.  For instance:
\begin{urbiscript}[firstnumber=last]
assert("%s + %s = %s" % [1, 2, 3] == "1 + 2 = 3");
\end{urbiscript}

\item \lstinline|'*'(\var{n})|\\
  Concatenate \lstinline|this| \var{n} times.
\begin{urbiscript}[firstnumber=last]
assert("foo" * 0 == "");
assert("foo" * 1 == "foo");
assert("foo" * 3 == "foofoofoo");
\end{urbiscript}

\item \lstinline|'+'(\var{other})|\\
  Concatenate \lstinline|this| and \lstinline|\var{other}.asString|.
\begin{urbiscript}[firstnumber=last]
assert("foo" + "bar" == "foobar");
assert("foo" + "" == "foo");
assert("foo" + 3 == "foo3");
assert("foo" + [1, 2, 3] == "foo[1, 2, 3]");
\end{urbiscript}

\item \lstinline|'<'(\var{other})|\\
  Whether \lstinline|this| is lexicographically before \var{other},
  which must be a String.
\begin{urbiscript}[firstnumber=last]
assert("" < "a");
assert(!("a" < ""));
assert("a" < "b");
assert(!("a" < "a"));
\end{urbiscript}

\item \lstinline|'[]'(\var{from})|\\
  \lstinline|'[]'(\var{from}, \var{to})|\\
  Return the substring starting at \var{from}, up to and not including
  \var{to} (which defaults to \var{to} + 1).
\begin{urbiscript}[firstnumber=last]
assert("foobar"[0, 3] == "foo");
assert("foobar"[0] == "f");
\end{urbiscript}

\item \lstinline|'[]='(\var{from}, \var{other})|\\
  \lstinline|'[]='(\var{from}, \var{to}, \var{other})|\\
  Replace the substring starting at \var{from}, up to and not including
  \var{to} (which defaults to \var{to} + 1), by \var{other}.  Return
  \var{other}.

  Beware that this routine is imperative: it changes the value of
  \lstinline|this|.
\begin{urbiscript}[firstnumber=last]
var s1 = "foobar" | var s2 = s1 |
assert((s1[0, 3] = "quux") == "quux");
assert(s1 == "quuxbar");
assert(s2 == "quuxbar");
assert((s1[4, 7] = "") == "");
assert(s2 == "quux");
\end{urbiscript}
\end{itemize}

%%% Local Variables:
%%% mode: latex
%%% TeX-master: "../urbi-sdk"
%%% End:

%% Copyright (C) 2009-2012, Gostai S.A.S.
%%
%% This software is provided "as is" without warranty of any kind,
%% either expressed or implied, including but not limited to the
%% implied warranties of fitness for a particular purpose.
%%
%% See the LICENSE file for more information.

\section{System}
Details on the architecture the \urbi server runs on.

\subsection{Prototypes}
\begin{refObjects}
\item[Object]
\end{refObjects}

\subsection{Slots}
\begin{urbiscriptapi}
\item[_exit](<status>)%
  Shut the server down brutally: the connections are not closed, and
  the resources are not explicitly released (the operating system
  reclaims most of them: memory, file descriptors and so forth).
  Architecture dependent.


\item[arguments] The list of the command line arguments passed to the user
  script.  This is especially useful in scripts.
\begin{shell}[alsolanguage={[Interactive]urbiscript}]
$ cat >echo <<EOF
#! /usr/bin/env urbi
System.arguments;
shutdown;
EOF
$ chmod +x echo
$ ./echo 1 2 3
[00000172] ["1", "2", "3"]
$ ./echo -x 12 -v "foo"
[00000172] ["-x", "12", "-v", "foo"]
\end{shell}

\begin{urbicomment}
//#timeout 3
import System.*;
\end{urbicomment}

\item['assert'](<assertion>)%
  Unless \refSlot{ndebug} is true, throw an error if
  \var{assertion} is not verified.  See also the assertion support in
  \us, \autoref{sec:lang:assert}.
\begin{urbiscript}
'assert'(true);
'assert'(42);
'assert'(1 == 1 + 1);
[00000002:error] !!! failed assertion: 1.'=='(1.'+'(1))
\end{urbiscript}


\item[assert_](<assertion>, <message>)%
  If \var{assertion} does not evaluate to true, throw the failure
  \var{message}.
\begin{urbiscript}
assert_(true,       "true failed");
assert_(42,         "42 failed");
assert_(1 == 1 + 1, "one is not two");
[00000001:error] !!! failed assertion: one is not two
\end{urbiscript}


\item[assert_op](<operator>, <lhs>, <rhs>)%
  Deprecated, use \lstinline|assert| instead, see \autoref{sec:lang:assert}.


%% \item[breakpoint]


\item[currentRunner]  An obsolete alias for \refSlot[Job]{current}.


\item[cycle]%
  The number of execution cycles since the beginning. \experimental
\begin{urbiscript}
{
  var first = cycle ; var second = cycle ;
  assert(first + 1 == second);
  first = cycle | second = cycle ;
  assert(first == second);
};
\end{urbiscript}


\item[env]
  A \refObject{Dictionary} containing the current
  environment of \urbi.  See also \refSlot{env.init}.
\begin{urbiassert}
(env["MyVar"] = 12) == "12";
env["MyVar"] == "12";

// A child process that uses the environment variable.
System.system("exit $MyVar") >> 8 ==
       {if (Platform.isWindows) 0 else 12};
(env["MyVar"] = 23) == "23";
System.system("exit $MyVar") >> 8 ==
       {if (Platform.isWindows) 0 else 23};

// Defining to empty is defining, unless you are on Windows.
(env["MyVar"] ="") == "";
env["MyVar"].isNil == Platform.isWindows;

env["UndefinedEnvironmentVariable"].isNil;
!env["PATH"].isNil;

(env["MyVar"] = 12) == "12";
!env["MyVar"].isNil;
env.erase("MyVar") == "12";
env["MyVar"].isNil;
\end{urbiassert}


\item[env.init]%
  Refresh the \urbi environment by fetching all the environment variables.
  Beware that \refSlot{env} is not updated when calling \lstinline{getenv},
  \lstinline{setenv} or \lstinline{unsetenv} from the C library.  Initialize
  it first and then manipulate your environment as a simple
  \refObject{Dictionary}.
\begin{urbiassert}
env.init == env;
!env["USER"].isNil;
\end{urbiassert}


\item[eval](<source>, <target> = this)%
  Evaluate the \us \var{source}, and return its result.  See also
  \refSlot{loadFile}.  The \var{source} must be complete, yet the terminator
  (e.g., \samp{;}) is not required.
\begin{urbiassert}
eval("1+2") == 1+2;
eval("\"x\" * 10") == "x" * 10;
eval("eval(\"1\")") == 1;
eval("{ var x = 1; x + x; }") == 2;
\end{urbiassert}

The evaluation is performed in the context of the current object (\this) or
\var{target} if specified.  In particular, to create local variables, create
scopes.
\begin{urbiassert}
// Create a slot in the current object.
eval("var a = 23;") == 23;
this.a == 23;

eval("var a = 3", Global) == 3;
Global.a == 3;
\end{urbiassert}

  Exceptions are thrown on error (including syntax errors).
\begin{urbiscript}
// Scanner errors.
eval("#");
[00000004:error] !!! 1.1: syntax error: invalid character: `#'
[00000005:error] !!!    called from: eval

// Syntax errors.
eval("3; 1 * * 2");
[00000002:error] !!! 1.8: syntax error: unexpected *
[00000003:error] !!!    called from: eval

// Exceptions.
eval("1/0");
[00008316:error] !!! 1.1-3: /: division by 0
[00008316:error] !!!    called from: eval
try
{
  eval("1/0")
}
catch (var e)
{
  assert
  {
    e.isA(Exception.Primitive);
    e.location.asString  == "1.1-3";
    e.routine            == "/";
    e.message            == "division by 0";
  }
};
\end{urbiscript}

  Warnings are reported.

\begin{urbiscript}
eval("new Object");
[00001388:warning] !!! 1.1-10: `new Obj(x)' is deprecated, use `Obj.new(x)'
[00001388:warning] !!!    called from: eval
[00001388] Object_0x1001b2320
\end{urbiscript}

  Nested calls to \refSlot{eval} behave as expected.  The locations in the
  inner calls refer to the position inside the evaluated string.

\begin{urbiscript}
eval("/");
[00001028:error] !!! 1.1: syntax error: unexpected /
[00001028:error] !!!    called from: eval

eval("eval(\"/\")");
[00001032:error] !!! 1.1: syntax error: unexpected /
[00001032:error] !!!    called from: 1.1-9: eval
[00001032:error] !!!    called from: eval

eval("eval(\"eval(\\\"/\\\")\")");
[00001035:error] !!! 1.1: syntax error: unexpected /
[00001035:error] !!!    called from: 1.1-9: eval
[00001035:error] !!!    called from: 1.1-19: eval
[00001035:error] !!!    called from: eval
\end{urbiscript}


\item[getenv](<name>)%
  Deprecated function use \lstinline|env[\var{name}]| instead.
  The value of the environment variable \var{name} as a \refObject{String}
  if set, \refObject{nil} otherwise.  See also \refSlot{env}, \refSlot{setenv}
  and \refSlot{unsetenv}.
\begin{urbiassert}
getenv("UndefinedEnvironmentVariable").isNil;
[01234567:warning] !!! `System.getenv(that)' is deprecated, use `System.env[that]'
!getenv("PATH").isNil;
[01234567:warning] !!! `System.getenv(that)' is deprecated, use `System.env[that]'
\end{urbiassert}


\item[getLocale](<category>)%
  A \refObject{String} denoting the locale set for \var{category}, or
  raise an error.  See \refSlot{setLocale} for more details.
\begin{urbiscript}
getLocale("LC_IMAGINARY");
[00006328:error] !!! getLocale: invalid category: LC_IMAGINARY
\end{urbiscript}


\item[load](<file>, <target> = this)%
  Look for \var{file} in the \urbi path (\autoref{sec:tools:envvars}), and
  load it in the context of \var{target}.  See also \refSlot{loadFile}.
  Throw a \refSlot[Exception]{FileNotFound} error if the file cannot be
  found.  Return the last value of the file.
\begin{urbiscript}
// Create the file ``123.u'' that contains exactly ``var t = 123;''.
File.save("123.u", "var t = 123;");
assert
{
  load("123.u") == 123;
  this.t == 123;

  load("123.u", Global) == 123;
  Global.t == 123;
};
\end{urbiscript}


\item[loadFile](<file>, <target> = this)%
  Load the \us file \var{file} in the context of \var{target}.  Behaves like
  \refSlot{eval} applied to the content of \var{file}.  Throw a
  \refSlot[Exception]{FileNotFound} error if the file cannot be found.
  Return the last value of the file.
\begin{urbiscript}
// Create the file ``123.u'' that contains exactly ``var y = 123;''.
File.save("123.u", "var y = 123;");
assert
{
  loadFile("123.u") == 123;
  this.y == 123;

  loadFile("123.u", Global) == 123;
  Global.y == 123;
};

\end{urbiscript}


\item[loadLibrary](<library>)%
  Load the library \var{library}, to be found in
  \refSlot[UObject]{searchPath}.  The \var{library} may be a
  \refObject{String} or a \refObject{Path}.  The \Cxx symbols are made
  available to the other \Cxx components.  See also \refSlot{loadModule}.


\item[loadModule](<module>)%
  Load the \UObject \var{module}.  Same as \refSlot{loadLibrary}, except
  that the low-level \Cxx symbols are not made ``global'' (in the sense of
  the shared library loader).


\item[lobbies] Bounce to \refSlot[Lobby]{instances}.


\item[lobby] Bounce to \refSlot[Lobby]{lobby}.


\item[maybeLoad](<file>, <channel> = Channel.null)%
  Look for \var{file} in the \urbi path (\autoref{sec:tools:envvars}).
  If the file is found announce on \var{Channel} that \var{file} is
  about to be loaded, and load it.

\begin{urbiscript}
// Create the file ``123.u'' that contains exactly ``123;''.
File.save("123.u", "123;");
assert
{
  maybeLoad("123.u") == 123;
  maybeLoad("u.123").isVoid;
};
\end{urbiscript}


\item[ndebug] If true, do not evaluate the assertions.  See
  \autoref{sec:lang:assert}.
\begin{urbiscript}
function one() { echo("called!"); 1 }|;
assert(!System.ndebug);

assert(one);
[00000617] *** called!

// Beware of copy-on-write.
System.ndebug = true|;
assert(one);

System.ndebug = false|;
assert(one);
[00000622] *** called!
\end{urbiscript}


%% \item[nonInterruptible]


\item[PackageInfo] See \refObject{System.PackageInfo}.


\item[period] The \dfn{period} of the \urbi kernel.  Influences the
  trajectories (\refObject{TrajectoryGenerator}), and the \UObject
  monitoring.  Defaults to 20ms.
\begin{urbiassert}
System.period == 20ms;
\end{urbiassert}


\item[Platform] See \refObject{System.Platform}.


\item[profile](<function>)%
  Compute some measures during the execution of \var{function} and return
  the results as a \refObject{Profile} object. A \refObject{Profile} details
  information about time, function calls and scheduling.


\item[programName] The path under which the \urbi process was called.
  This is typically \file{.../urbi} (\autoref{sec:tools:urbi}) or
  \file{.../urbi-launch} (\autoref{sec:tools:urbi-launch}).
\begin{urbiassert}
Path.new(System.programName).basename
  in ["urbi", "urbi.exe", "urbi-launch", "urbi-launch.exe"];
\end{urbiassert}


\item[reboot] Restart the \urbi server.  Architecture dependent.


\item[redefinitionMode] Switch the current job in redefinition mode
  until the end of the current scope.  While in redefinition mode,
  setSlot on already existing slots will overwrite the slot instead of
  erring.

\begin{urbiscript}
var Global.x = 0;
[00000001] 0
{
  System.redefinitionMode;
  // Not an error
  var Global.x = 1;
  echo(Global.x);
};
[00000002] *** 1
// redefinitionMode applies only to the scope.
var Global.x = 0;
[00000003:error] !!! slot redefinition: x
\end{urbiscript}


\item[requireFile](<file>, <target>)%
  Load \var{file} in the context of \var{target} if it was not loaded before
  (with \refSlot{load} or \refSlot{requireFile}). Unlike \refSlot{load},
  \lstinline{requireFile} always returns \lstinline|void|. If \var{file} is
  being loaded concurrently \lstinline{requireFile} waits until the loading
  is finished.

\begin{urbiscript}
// Create the file ``test.u'' that echoes a string.
File.save("test1.u", "echo(\"test 1\"); 1;");
requireFile("test1.u");
[00000001] *** test 1
requireFile("test1.u");
// File is not re-loaded

File.save("test2.u", "echo(\"test 2\"); 2;");
load("test2.u");
[00000004] *** test 2
[00000004] 2
requireFile("test2.u");
load("test2.u");
[00000006] *** test 2
[00000006] 2
\end{urbiscript}

  The \var{target} is not taken into account to check whether the file has
  already been loaded: if you require twice the same file with two different
  targets, it will be loaded only for the first.

\begin{urbiscript}
requireFile("test2.u", Global);
\end{urbiscript}


\item[resetStats]%
  Reinitialize the \refSlot{stats} computation.
\begin{urbiassert}
 0  < System.stats["cycles"];
System.resetStats.isVoid;
 1 == System.stats["cycles"];
\end{urbiassert}


\item[scopeTag] Bounce to \refSlot[Tag]{scope}.


\item[searchFile](<file>)%
  Look for \var{file} in the \refSlot{searchPath} and return its
  \refObject{Path}.  Throw a \refSlot[Exception]{FileNotFound} error if the
  file cannot be found.
\begin{urbiscript}
// Create the file ``123.u'' that contains exactly ``123;''.
File.save("123.u", "123;");
assert
{
  searchFile("123.u") == Path.cwd / Path.new("123.u");
};
\end{urbiscript}


\item[searchPath] The \urbi path (i.e., the directories where the \us files
  are looked for, see \autoref{sec:tools:envvars}) as a \refObject{List} of
  \refObject[Path]{Paths}.
\begin{urbiassert}
System.searchPath.isA(List);
System.searchPath[0].isA(Path);
\end{urbiassert}


\item[setenv](<name>, <value>)%
  Deprecated, use \lstinline|env[\var{name}] = \var{value}| instead.  Set
  the environment variable \var{name} to \lstinline|\var{value}.asString|,
  and return this value.  See also \refSlot{env}, \refSlot{getenv} and
  \refSlot{unsetenv}.
  \begin{windows}
    Under Windows, setting to an empty value is equivalent to
    making undefined.
  \end{windows}

\begin{urbiassert}
setenv("MyVar", 12) == "12";
[00000001:warning] !!! `System.setenv(var, value)' is deprecated, \
[:]                       use `System.env[var] = value'
env["MyVar"] == "12";
\end{urbiassert}


\item[setLocale](<category>, <locale> = "")%
  Change the system's \dfn{locale} for the \var{category} to \var{locale}
  and return void.  If \var{locale} is empty, then use the locale specified
  by the user's environment (e.g., the environment variables).  The
  \var{category} can be:
  \begin{sublist}
    \begin{envs}
    \item[LC\_ALL] Overrides all the following categories.
    \item[LC\_COLLATE] Controls how string sorting is performed.
    \item[LC\_CTYPE] Change what characters are considered as letters and so
      on.
    \item[LC\_MESSAGES] The catalog of translated messages.  This category
      is not supported by Microsoft Windows.
    \item[LC\_MONETARY] How to format monetary values.
    \item[LC\_NUMERIC] Set a locale for formatting numbers.
    \item[LC\_TIME] Set a locale for formatting dates and times.
    \end{envs}
  \end{sublist}
  With \command{urbi} is run, it does \emph{not} change its locales: it
  defaults to the ``good old C mode'', which corresponds to the \samp{C} (or
  \samp{POSIX}) locale.  See also \refSlot{getLocale}.
\begin{urbicomment}
// Be sure not to depend on the user environment.
for (var i : ["ALL", "COLLATE", "CTYPE", "MESSAGES", "MONETARY",
              "NUMERIC", "TIME"])
  env.erase("LC_" + i);
\end{urbicomment}
\begin{urbiassert}
// Initially they are all set to "C".
getLocale("LC_ALL")     == "C";
getLocale("LC_CTYPE")   == "C";
getLocale("LC_NUMERIC") == "C";

// Windows does not understand the "fr_FR" locale, it supports "French"
// which actually denotes "French_France.1252".
var fr_FR =
  { if (System.Platform.isWindows) "French_France.1252" else "fr_FR.utf8" };
// Changing one via the environment does not affect the others.
(env["LC_CTYPE"] = fr_FR) == fr_FR;
getLocale("LC_CTYPE")   == "C";
setLocale("LC_CTYPE").isVoid;
getLocale("LC_CTYPE")   == fr_FR;
getLocale("LC_NUMERIC") == "C";

// Changing one via setLocale does not change the others either.
setLocale("LC_CTYPE", "C").isVoid;
getLocale("LC_CTYPE")   == "C";
getLocale("LC_NUMERIC") == "C";

// The environment variable LC_ALL overrides all the others.
env["LC_ALL"] = fr_FR;
setLocale("LC_ALL").isVoid;
getLocale("LC_ALL")     == fr_FR;
getLocale("LC_CTYPE")   == fr_FR;
getLocale("LC_NUMERIC") == fr_FR;

// Explicit changes of LC_ALL overrides all the others.
setLocale("LC_ALL", "C").isVoid;
getLocale("LC_ALL")     == "C";
getLocale("LC_CTYPE")   == "C";
getLocale("LC_NUMERIC") == "C";
\end{urbiassert}

  On invalid requests, raise an error.
\begin{urbiscript}
setLocale("LC_IMAGINARY");
[00006328:error] !!! setLocale: invalid category: LC_IMAGINARY

env["LC_ALL"] = "elfic"|;
setLocale("LC_ALL");
[00024950:error] !!! setLocale: cannot set locale LC_ALL to elfic

setLocale("LC_ALL", "klingon");
[00074958:error] !!! setLocale: cannot set locale LC_ALL to klingon
\end{urbiscript}


\item[shiftedTime] The number of seconds elapsed since the \urbi server was
  launched.  Contrary to \refSlot{time}, time spent in frozen code is not
  counted.
\begin{urbiassert}
{ var t0 = shiftedTime | sleep(1s) | shiftedTime - t0 }.round ~= 1;

  1 ==
  {
    var t = Tag.new|;
    var t0 = time|;
    var res;
    t: { sleep(1s) | res = shiftedTime - t0 },
    t.freeze;
    sleep(1s);
    t.unfreeze;
    sleep(1s);
    res.round;
  };
\end{urbiassert}


\item[shutdown](<exit_status> = 0)%
  Have the \urbi server shut down, with exit status \var{exit\_status}.  All
  the connections are closed, the resources are released.  Architecture
  dependent.


\item[sleep](<duration> = inf)%
  Suspend the execution for \var{duration} seconds.  No CPU cycle is wasted
  during this wait. If no \var{duration} is given the execution is suspended
  indefinitely.

\begin{urbiassert}
(time - {sleep(1s); time}).round == -1;
\end{urbiassert}


\item[spawn](<function>, <clear>)%
  Deprecated internal function.  Bounces to
  \lstinline|\var{function}.spawn(\var{clear})|, see \refSlot[Code]{spawn}.
\begin{urbiassert}
System.spawn(closure () { echo(1) }, true).isA(Job);
[00016657:warning] !!! `System.spawn' is deprecated, use `Code.spawn'
[00016659] *** 1
\end{urbiassert}


\item[stats]%
  A \refObject{Dictionary} containing information about the execution cycles
  of \urbi.  This is an internal feature made for developers, it might be
  changed without notice.  See also \refSlot{resetStats}.  These statistics
  make no sense in \option{--fast} mode (\autoref{sec:tools:urbi:opt}).
\begin{urbicomment}
//#no-fast
\end{urbicomment}
\begin{urbiassert}
var stats = System.stats;

stats.isA(Dictionary);
stats.keys.sort == ["cycles",
                    "cyclesMin", "cyclesMean", "cyclesMax",
                    "cyclesVariance", "cyclesStdDev"].sort;
// Number of cycles.
0 < stats["cycles"];
// Cycles duration.
0 <= stats["cyclesMin"] <= stats["cyclesMean"] <= stats["cyclesMax"];

stats["cyclesVariance"].isA(Float);
stats["cyclesStdDev"].isA(Float);
\end{urbiassert}


%% \item[stopall]


\item[system](<command>)%
  Ask the operating system to run the \var{command}.  This is typically used
  to start new processes.  The exact syntax of \var{command} depends on your
  system.  On Unix systems, this is typically \file{/bin/sh}, while Windows
  uses \file{command.exe}.

  Return the exit status.

  \begin{windows}
    Under Windows, the exit status is always 0.
  \end{windows}

\begin{urbiassert}
System.system("exit 0") == 0;
System.system("exit 23") >> 8
       == { if (System.Platform.isWindows) 0 else 23 };
\end{urbiassert}


\item[time] The number of seconds elapsed since the \urbi server was
  launched.  See also \refObject{Date}.  In presence of a frozen
  \refObject{Tag}, see also \refSlot{shiftedTime}.
\begin{urbiassert}
{ var t0 = time | sleep(1s) | time - t0 }.round ~= 1;

  2 ==
  {
    var t = Tag.new|;
    var t0 = time|;
    var res;
    t: { sleep(1s) | res = time - t0 },
    t.freeze;
    sleep(1s);
    t.unfreeze;
    sleep(1s);
    res.round;
  };
\end{urbiassert}


\item[timeReference]%
  The ``origin of time'' of this run of \urbi, as a \refObject{Date}.  It is
  a constant during the run.  Basically, \lstinline|System.time| is about
  \lstinline|Date.now - System.timeReference|.  See also \refSlot{time} and
  \refSlot[Date]{now}.

\begin{urbiscript}
var t1 = System.timeReference|;
sleep(1s);
var t2 = System.timeReference|;
assert
{
  t1 == t2;
  t1.isA(Date);
  (Date.now - (System.timeReference + System.time)) < 0.5s;
};
\end{urbiscript}
\begin{urbicomment}
removeSlots("t1", "t2");
\end{urbicomment}

\item[unsetenv](<name>)%
  Deprecated use \lstinline|env.erase (\var{name})| instead.
  Undefine the environment variable \var{name}, return its previous value.
  See also \refSlot{env}, \refSlot{getenv} and \refSlot{setenv}.

\begin{urbiassert}
(env["MyVar"] = 12) == "12";
!env["MyVar"].isNil;
unsetenv("MyVar") == "12";
[01234567:warning] !!! `System.unsetenv(var)' is deprecated, use `System.env.erase(var)'
env["MyVar"].isNil;
\end{urbiassert}


\item[version]%
  The version of \usdk.  A string composed of two or more numbers separated
  by periods: \samp{"\packageVersion"}.
\begin{urbiassert}
System.version in Regexp.new("\\d+(\\.\\d+)+");
\end{urbiassert}
\end{urbiscriptapi}

%%% Local Variables:
%%% coding: utf-8
%%% mode: latex
%%% TeX-master: "../urbi-sdk"
%%% ispell-dictionary: "american"
%%% ispell-personal-dictionary: "../urbi.dict"
%%% fill-column: 76
%%% End:

%% Copyright (C) 2009-2010, Gostai S.A.S.
%%
%% This software is provided "as is" without warranty of any kind,
%% either expressed or implied, including but not limited to the
%% implied warranties of fitness for a particular purpose.
%%
%% See the LICENSE file for more information.

\section{System.Platform}
A description of the platform (the computer) the server is running on.

\subsection{Prototypes}
\begin{refObjects}
\item[Object]
\end{refObjects}

\subsection{Slots}
\begin{urbiscriptapi}
\item[host] The type of system \usdk runs on.  Composed of the CPU, the
  vendor, and the OS.
\begin{urbiassert}
System.Platform.host ==
  "%s-%s-%s" % [System.Platform.hostCpu,
                System.Platform.hostVendor,
                System.Platform.hostOs];
\end{urbiassert}

\item[hostAlias] The name of the system \usdk runs on as the person who
  compiled it decided to name it.  Typically empty, it is fragile to depend
  on it.
\begin{urbiassert}
System.Platform.hostAlias.isA(String);
\end{urbiassert}

\item[hostCpu] The CPU type of system \usdk runs on.  The following values
  are those for which Gostai provides binary builds.
\begin{urbiassert}
System.Platform.hostCpu in ["i386", "i686", "x86_64"];
\end{urbiassert}

\item[hostOs] The OS type of system \usdk runs on.  For instance
  \lstinline|darwin9.8.0| or \lstinline|linux-gnu| or \lstinline|mingw32|.

\item[hostVendor] The vendor type of system \usdk runs on.  The following
  values are those for which Gostai provides binary builds.
\begin{urbiassert}
System.Platform.hostVendor in ["apple", "pc", "unknown"];
\end{urbiassert}

\item[isWindows] Whether running under Windows.
\begin{urbiassert}
System.Platform.isWindows in [true, false];
\end{urbiassert}

\item[kind] Either \code{"POSIX"} or \code{"WIN32"}.
\begin{urbiassert}
System.Platform.kind in ["POSIX", "WIN32"];
\end{urbiassert}
\end{urbiscriptapi}

%%% Local Variables:
%%% mode: latex
%%% TeX-master: "../../urbi-sdk"
%%% ispell-dictionary: "american"
%%% ispell-personal-dictionary: "../../urbi.dict"
%%% fill-column: 76
%%% End:

%% Copyright (C) 2009-2011, Gostai S.A.S.
%%
%% This software is provided "as is" without warranty of any kind,
%% either expressed or implied, including but not limited to the
%% implied warranties of fitness for a particular purpose.
%%
%% See the LICENSE file for more information.

\section{Tag}

A \dfn{tag} is an object meant to label blocks of code in order to control
them externally.  Tagged code can be frozen, resumed, stopped\ldots See also
\autoref{sec:tut:tags}.

\subsection{Examples}

\subsubsection{Stop}
\label{sec:specs:tag:stop}

To \dfn{stop} a tag means to kill all the code currently running that it
labels.  It does not affect ``newcomers''.

\begin{urbiscript}[firstnumber=1]
var t = Tag.new|;
var t0 = time|;
t: every(1s) echo("foo"),
sleep(2.2s);
[00000158] *** foo
[00001159] *** foo
[00002159] *** foo

t.stop;
// Nothing runs.
sleep(2.2s);

t: every(1s) echo("bar"),
sleep(2.2s);
[00000158] *** bar
[00001159] *** bar
[00002159] *** bar

t.stop;
\end{urbiscript}

\refSlot[Tag]{stop} can be used to inject a return value to a tagged
expression.

\begin{urbiscript}[firstnumber=1]
var t = Tag.new|;
var res;
detach(res = { t: every(1s) echo("computing") })|;
sleep(2.2s);
[00000001] *** computing
[00000002] *** computing
[00000003] *** computing

t.stop("result");
assert(res == "result");
\end{urbiscript}

Be extremely cautious, the precedence rules can be misleading:
\lstinline|\var{var} = \var{tag}: \var{exp}| is read as
\lstinline|(\var{var} = \var{tag}): \var{exp}| (i.e., defining \var{var} as
an alias to \var{tag} and using it to tag \var{exp}), not as
\lstinline|\var{var} = { \var{tag}: \var{exp} }|.  Contrast the following
example, which is most probably an error from the user, with the previous,
correct, one.

\begin{urbiscript}[firstnumber=1]
var t = Tag.new("t")|;
var res;
res = t: every(1s) echo("computing"),
sleep(2.2s);
[00000001] *** computing
[00000002] *** computing
[00000003] *** computing

t.stop("result");
assert(res == "result");
[00000004:error] !!! failed assertion: res == "result" (Tag<t> != "result")
\end{urbiscript}


\subsubsection{Block/unblock}
\label{sec:specs:tag:block}

To \dfn{block} a tag means:
\begin{itemize}
\item Stop running pieces of code it labels (as with
  \refSlot{stop}).
\item Ignore new pieces of code it labels (this differs from
  \refSlot{stop}).
\end{itemize}

One can \dfn{unblock} the tag.  Contrary to
\refSlot{freeze}/\refSlot{unfreeze}, tagged code does not resume the
execution.

\begin{urbiscript}[firstnumber=1]
var ping = Tag.new("ping")|;
ping:
  every (1s)
    echo("ping"),
assert(!ping.blocked);
sleep(2.1s);
[00000000] *** ping
[00002000] *** ping
[00002000] *** ping

ping.block;
assert(ping.blocked);

ping:
  every (1s)
    echo("pong"),

// Neither new nor old code runs.
ping.unblock;
assert(!ping.blocked);
sleep(2.1s);

// But we can use the tag again.
ping:
  every (1s)
    echo("ping again"),
sleep(2.1s);
[00004000] *** ping again
[00005000] *** ping again
[00006000] *** ping again
\end{urbiscript}

As with \refSlot{stop}, one can force the value of stopped
expressions.

\begin{urbiassert}[firstnumber=1]
{
  var t = Tag.new;
  var res = [];
  for (3)
    detach(res << {t: sleep});
  t.block("foo");
  res;
}
==
["foo", "foo", "foo"];
\end{urbiassert}

\subsubsection{Freeze/unfreeze}
\label{sec:specs:tag:freeze}

To \dfn{freeze} a tag means holding the execution of code it labels.
This applies to code already being run, and ``arriving'' pieces of code.

\begin{urbiscript}[firstnumber=1]
var t = Tag.new|;
var t0 = time|;
t: every(1s) echo("time   : %.0f" % (time - t0)),
sleep(2.2s);
[00000158] *** time   : 0
[00001159] *** time   : 1
[00002159] *** time   : 2

t.freeze;
assert(t.frozen);
t: every(1s) echo("shifted: %.0f" % (shiftedTime - t0)),
sleep(2.2s);
// The tag is frozen, nothing is run.

// Unfreeze the tag: suspended code is resumed.
// Note the difference between "time" and "shiftedTime".
t.unfreeze;
assert(!t.frozen);
sleep(2.2s);
[00004559] *** shifted: 2
[00005361] *** time   : 5
[00005560] *** shifted: 3
[00006362] *** time   : 6
[00006562] *** shifted: 4
\end{urbiscript}


\subsubsection{Scope tags}
\label{sec:specs:tag:scope}

Scopes feature a \lstindex{scopeTag}, i.e., a tag which will be stop
when the execution reaches the end of the current scope.  This is
handy to implement cleanups, how ever the scope was exited from.

\begin{urbiscript}[firstnumber=1]
{
  var t = scopeTag;
  t: every(1s)
      echo("foo"),
  sleep(2.2s);
};
[00006562] *** foo
[00006562] *** foo
[00006562] *** foo

{
  var t = scopeTag;
  t: every(1s)
      echo("bar"),
  sleep(2.2s);
  throw 42;
};
[00006562] *** bar
[00006562] *** bar
[00006562] *** bar
[00006562:error] !!! 42
sleep(2s);
\end{urbiscript}

\subsubsection{Enter/leave events}
\label{sec:specs:tag:enter-leave}

Tags provide two events, \refSlot{enter} and \refSlot{leave}, that trigger
whenever flow control enters or leaves statements the tag.

\begin{urbiscript}[firstnumber=1]
var t = Tag.new("t");
[00000000] Tag<t>

at (t.enter?)
  echo("enter");
at (t.leave?)
  echo("leave");

t: {echo("inside"); 42};
[00000000] *** enter
[00000000] *** inside
[00000000] *** leave
[00000000] 42
\end{urbiscript}

This feature is fundamental; it is a concise and safe way to ensure code
will be executed upon exiting a chunk of code (like \acro{raii} in \Cxx or
\lstinline|finally| in Java). The exit code will be run no matter what the
reason for leaving the block was: natural exit, exceptions, flow control
statements like \lstinline|return| or \lstinline|break|, \ldots

For instance, suppose we want to make sure we turn the gas off when
we're done cooking. Here is the \emph{bad} way to do it:

\begin{urbiscript}
{
  function cook()
  {
    turnGasOn();
    // Cooking code ...
    turnGasOff();
  }|

  enterTheKitchen();
  cook();
  leaveTheKitchen();
};
\end{urbiscript}

This \lstinline|cook| function is wrong because there are several situations
where we could leave the kitchen with gas still turned on. Consider the
following cooking code:

\begin{urbiscript}
{
  function cook()
  {
    turnGasOn();

    if (mealReady)
    {
      echo("The meal is already there, nothing to do!");
      // Oops ...
      return;
    };

    for (var i in recipe)
      if (i in kitchen)
        putIngredient(i)
      else
        // Oops ...
        throw Exception("missing ingredient: %s" % i);

    // ...

    turnGasOff();
  }|
};
\end{urbiscript}

Here, if the meal was already prepared, or if an ingredient is missing, we
will leave the \lstinline|cook| function without executing the
\lstinline|turnGasOff| statement, through the \lstinline|return| statement
or the exception.  One correct way to ensure gas is necessarily turned off
is:

\begin{urbiscript}
{
  function cook()
  {
    var withGas = Tag.new("withGas");

    at (withGas.enter?)
      turnGasOn();
    // Even if exceptions are thrown or return is called,
    // the gas will be turned off.
    at (withGas.leave?)
      turnGasOff();

    withGas: {
      // Cooking code...
    }
  }|
};
\end{urbiscript}

Alternatively, the \lstinline|try|/\lstinline|finally| construct provides an
elegant means to achieve the same result (\autoref{sec:lang:except:finally}).

\begin{urbiscript}
{
  function cook()
  {
    try
    {
      turnGasOn();
      // Cooking code...
    }
    finally
    {
      // Even if exceptions are thrown or return is called,
      // the gas will be turned off.
      turnGasOff();
    }
  }|
};
\end{urbiscript}

\subsubsection{Begin/end}
\label{sec:specs:tag:begin-end}

The \refSlot{begin} and \refSlot{end} methods enable to monitor when code is
executed.  The following example illustrates the proper use of
\refSlot{enter} and \refSlot{leave} events
(\autoref{sec:specs:tag:enter-leave}), which are used to implement this
feature.

\begin{urbiscript}
var myTag = Tag.new("myTag");
[00000000] Tag<myTag>

myTag.begin: echo(1);
[00000000] *** myTag: begin
[00000000] *** 1

myTag.end: echo(2);
[00000000] *** 2
[00000000] *** myTag: end

myTag.begin.end: echo(3);
[00000000] *** myTag: begin
[00000000] *** 3
[00000000] *** myTag: end
\end{urbiscript}

\subsection{Construction}
\label{stdlib:tag:ctor}

As any object, tags are created using \lstinline{new} to create derivatives
of the \lstinline{Tag} object.  The name is optional, it makes easier to
display a tag and remember what it is.

\begin{urbiscript}[firstnumber=1]
// Anonymous tag.
var t1 = Tag.new;
[00000001] Tag<tag_8>

// Named tag.
var t2 = Tag.new("cool name");
[00000001] Tag<cool name>
\end{urbiscript}

\subsection{Slots}

\begin{urbiscriptapi}
\item[begin]
  A sub-tag that prints out "tag\_name: begin" each time flow control
  enters the tagged code. See \autoref{sec:specs:tag:begin-end}.

\item[block](<result> = void)%
  Block any code tagged by \this.  Blocked tags can be
  unblocked using \refSlot{unblock}.  If some \var{result} was
  specified, let stopped code return \var{result} as value.  See
  \autoref{sec:specs:tag:block}.

\item[blocked]
  Whether code tagged by \this is blocked.  See
  \autoref{sec:specs:tag:block}.

\item[end]
  A sub-tag that prints out "tag\_name: end" each time flow control
  leaves the tagged code. See \autoref{sec:specs:tag:begin-end}.

\item[enter] An event triggered each time the flow control enters the
  tagged code.  See \autoref{sec:specs:tag:enter-leave}.

\item[freeze] Suspend code tagged by \this, already running or
  forthcoming.  Frozen code can be later unfrozen using \refSlot{unfreeze}.
  See \autoref{sec:specs:tag:freeze}.

\item[frozen]
  Whether the tag is frozen. See  \autoref{sec:specs:tag:freeze}.

\item[leave] An event triggered each time flow control leaves the
  tagged code.  See \autoref{sec:specs:tag:enter-leave}.

\item[scope] Return a fresh Tag whose \refSlot{stop} will be invoked a the
  end of the current scope.  This function is likely to be removed.  See
  \autoref{sec:specs:tag:scope}.

\item[stop](<result> = void)%
  Stop any code tagged by \this.  If some \var{result} was
  specified, let stopped code return \var{result} as value.
  See \autoref{sec:specs:tag:stop}.

\item[tags] All the undeclared tags are created as slots in this
  object.  Using this feature is discouraged.
\begin{urbiscript}
{
  assert ("brandNewTag" not in Tag.tags.localSlotNames);
  brandNewTag: 1;
  assert ("brandNewTag" in Tag.tags.localSlotNames);
  assert (Tag.tags.brandNewTag.isA(Tag));
};
\end{urbiscript}

\item[unblock]
  Unblock \this.  See \autoref{sec:specs:tag:block}.

\item[unfreeze]
  Unfreeze code tagged by \this.  See
  \autoref{sec:specs:tag:freeze}.
\end{urbiscriptapi}

\subsection{Hierarchical tags}

Tags can be arranged in a parent/child relationship: any operation done on a
tag --- freezing, stopping, \ldots is also performed on its descendants.
Another way to see it is that tagging a piece of code with a child will also
tag it with the parent. To create a child Tag, simply clone its parent.

\begin{urbiscript}
var parent = Tag.new |
var child = parent.clone |

// Stopping parent also stops children.
{
  parent: {sleep(100ms); echo("parent")},
  child:  {sleep(100ms); echo("child")},
  parent.stop;
  sleep(200ms);
  echo("end");
};
[00000001] *** end

// Stopping child has no effect on parent.
{
  parent: {sleep(100ms); echo("parent")},
  child:  {sleep(100ms); echo("child")},
  child.stop;
  sleep(200ms);
  echo("end");
};
[00000002] *** parent
[00000003] *** end
\end{urbiscript}

Hierarchical tags are commonly laid out in slots so as to reflect their tag
hierarchy.

\begin{urbiunchecked}
var a = Tag.new;
var a.b = a.clone;
var a.b.c = a.b.clone;

a:     foo; // Tagged by a
a.b:   bar; // Tagged by a and b
a.b.c: baz; // Tagged by a, b and c
\end{urbiunchecked}

% FIXME: If we ever restore some sugar to create hierarchical tags, document
% it here

%%% Local Variables:
%%% mode: latex
%%% TeX-master: "../urbi-sdk"
%%% ispell-dictionary: "american"
%%% ispell-personal-dictionary: "../urbi.dict"
%%% fill-column: 76
%%% End:

%% Copyright (C) 2009-2011, Gostai S.A.S.
%%
%% This software is provided "as is" without warranty of any kind,
%% either expressed or implied, including but not limited to the
%% implied warranties of fitness for a particular purpose.
%%
%% See the LICENSE file for more information.

\section{Timeout}

Timeout objects can be used as \refObject[Tag]{Tags} to execute some code in
limited time.  See also the \lstinline|timeout| construct
(\autoref{sec:lang:timeout}).

\subsection{Examples}

Use it as a tag:

\begin{urbiscript}
var t = Timeout.new(300ms);
[00000000] Timeout_0x133ec0
t:{
  echo("This will be displayed.");
  sleep(500ms);
  echo("This will not.");
};
[00000000] *** This will be displayed.
[00000007:error] !!! Timeout_0x133ec0 has timed out.
\end{urbiscript}

The same Timeout, \lstinline|t| can be reused.  It is armed again each
time it is used to tag some code.

\begin{urbiscript}
t: { echo("Open"); sleep(1s); echo("Close"); };
[00000007] *** Open
[00000007:error] !!! Timeout_0x133ec0 has timed out.
t: { echo("Open"); sleep(1s); echo("Close"); };
[00000007] *** Open
[00000007:error] !!! Timeout_0x133ec0 has timed out.
\end{urbiscript}

%%% FIXME: Something is wrong here.
%%%
%%%A Timeout cannot be used several times at the same time though.
%%%
%%%\begin{urbiscript}
%%%// Note the commas.
%%%t: { echo("Open"); sleep(1s); echo("Close"); },
%%%[00035424:error] !!! Timeout_0x44e89e0 is already running
%%%t: { echo("Open"); sleep(1s); echo("Close"); },
%%%[00000007] *** Open
%%%[00000007:error] !!! Timeout_0x133ec0 has timed out.
%%%sleep(2s);
%%%\end{urbiscript}
Even if exceptions have been disabled, you can check whether the
count-down expired with \lstinline|timedOut|.

\begin{urbiscript}
t:sleep(500ms);
[00000007:error] !!! Timeout_0x133ec0 has timed out.
if (t.timedOut)
  echo("The Timeout expired.");
[00000000] *** The Timeout expired.
\end{urbiscript}

\subsection{Prototypes}
\begin{refObjects}
\item[Tag]
\end{refObjects}

\subsection{Construction}
At construction, a Timeout takes a duration, and a \refObject{Boolean}
stating whether an exception should be thrown on timeout (by default,
it does).

\begin{urbiscript}
Timeout.new(300ms);
[00000000] Timeout_0x953c1e0
Timeout.new(300ms, false);
[00000000] Timeout_0x953c1e0
\end{urbiscript}

\subsection{Slots}
\begin{urbiscriptapi}
\item[asTimeout] Return \this.
\begin{urbiscript}
{
  var t = Timeout.new(10);
  assert
  {
    Timeout.asTimeout === Timeout;
          t.asTimeout === t;
  };
};
\end{urbiscript}

\item[launch]
  Fire \this.

%% FIXME: there is more.
\end{urbiscriptapi}

%%% Local Variables:
%%% mode: latex
%%% TeX-master: "../urbi-sdk"
%%% ispell-dictionary: "american"
%%% ispell-personal-dictionary: "../urbi.dict"
%%% fill-column: 76
%%% End:

%% Copyright (C) 2010, 2011, Gostai S.A.S.
%%
%% This software is provided "as is" without warranty of any kind,
%% either expressed or implied, including but not limited to the
%% implied warranties of fitness for a particular purpose.
%%
%% See the LICENSE file for more information.

\section{TrajectoryGenerator}

The trajectory generators change the value of a given variable from an
\dfn{initial value} to a \dfn{target value}.  They can be
\dfn{open-loop}, i.e., the intermediate values depend only on the
initial and/or target value of the variable; or \dfn{closed-loop},
i.e., the intermediate values also depend on the current value value
of the variable.

Open-loop trajectories are insensitive to changes made elsewhere to
the variable.  Closed-loop trajectories \emph{are} sensitive to
changes made elsewhere to the variable --- for instance when the human
physically changes the position of a robot's motor.

Trajectory generators are not made to be used directly, rather use the
``continuous assignment'' syntax (\autoref{sec:lang:traj}).


\subsection{Examples}
\label{sec:traj:examples}

%% \subsubsection{Trajectory Name}
%% -------------------------------
\let\subsubsectionOrig\subsubsection
\renewcommand{\subsubsection}[1]
{%
  \subsubsectionOrig{\label{sec:traj:#1}#1}%
  %% It is on purpose that we pass [] to lstinline, because we do in many
  %% other places, and that would result in index entries that makeindex is
  %% unable to merge.
  \index{#1@\lstinline[]{#1}}%
}

\subsubsection{Accel}

The \lstinline{Accel} trajectory reaches a target value at a fixed
acceleration (\lstinline{accel} attribute).

\urbitrajectory{accel}

\subsubsection{Cos}

The \lstinline{Cos} trajectory implements a cosine around the target
value, given an amplitude (\lstinline{ampli} attribute) and period
(\lstinline{cos} attribute).

This trajectory is not ``smooth'': the initial value of the variable
is not taken into account.

\urbitrajectory{cos}

\subsubsection{Sin}

The \lstinline{Sin} trajectory implements a sine around the target
value, given an amplitude (\lstinline{ampli} attribute) and period
(\lstinline{sin} attribute).

This trajectory is not ``smooth'': the initial value of the variable
is not taken into account.

\urbitrajectory{sin}

\subsubsection{Smooth}

The \lstinline{Smooth} trajectory implements a sigmoid.  It changes
the variable from its current value to the target value ``smoothly''
in a given amount of time (\lstinline{smooth} attribute).

\urbitrajectory{smooth}

\subsubsection{Speed}

The \lstinline{Speed} trajectory changes the value of the variable
from its current value to the target value at a fixed speed (the
\lstinline{speed} attribute).

\urbitrajectory{speed}

If the \lstinline{adaptive} attribute is set to true, then the
duration of the trajectory is constantly reevaluated.

\urbitrajectory{speed-adaptive}

\subsubsection{Time}

The \lstinline{Time} trajectory changes the value of the variable from
its current value to the target value within a given duration (the
\lstinline{time} attribute).

\urbitrajectory{time}

If the \lstinline{adaptive} attribute is set to true, then the
duration of the trajectory is constantly reevaluated.

\urbitrajectory{time-adaptive}

%% Restore the definition of \subsubsection.
\let\subsubsection\subsubsectionOrig

\subsubsection{Trajectories and Tags}

Trajectories can be managed using \refObject[Tag]{Tags}.  Stopping or blocking
a tag that manages a trajectory kill the trajectory.

\urbitrajectory{cos-stop}
\urbitrajectory{cos-block}

When a trajectory is frozen, its local time is frozen too, the movement
proceeds from where it was rather than from where it would have been had it
been not frozen.

\urbitrajectory{cos-freeze}


\subsection{Prototypes}
\begin{refObjects}
\item[Object]
\end{refObjects}

\subsection{Construction}

You are not expected to construct trajectory generators by hand, using
modifiers is the recommended way to construct trajectories.  See
\autoref{sec:lang:traj} for details about trajectories, and see
\autoref{sec:traj:examples} for an extensive set of examples.

\subsection{Slots}

\begin{urbiscriptapi}
\item[Accel] This class implements the \lstinline|Accel| trajectory
  (\autoref{sec:traj:Accel}).  It derives from
  \refSlot{OpenLoop}.

\item[ClosedLoop] This class factors the implementation of the
  \dfn{closed-loop} trajectories.  It derives from
  \lstinline|TrajectoryGenerator|.

\item[OpenLoop] This class factors the implementation of the \dfn{open-loop}
  trajectories.  It derives from \lstinline|TrajectoryGenerator|.

\item[Sin] This class implements the \lstinline|Cos| and \lstinline|Sin|
  trajectories (\autoref{sec:traj:Cos}, \autoref{sec:traj:Sin}).  It derives
  from \refSlot{OpenLoop}.

\item[Smooth] This class implements the \lstinline|Smooth| trajectory
  (\autoref{sec:traj:Smooth}).  It derives from
  \refSlot{OpenLoop}.

\item[SpeedAdaptive] This class implements the \lstinline|Speed| trajectory
  when the \lstinline|adaptive| attribute is given
  (\autoref{sec:traj:Speed}).  It derives from
  \refSlot{ClosedLoop}.

\item[Time] This class implements the non-adaptive \lstinline|Speed| and
  \lstinline|Time| trajectories (\autoref{sec:traj:Speed},
  \autoref{sec:traj:Time}).  It derives from
  \refSlot{OpenLoop}.

\item[TimeAdaptive] This class implements the \lstinline|Time| trajectory
  when the \lstinline|adaptive| attribute is given
  (\autoref{sec:traj:Time}).  It derives from \refSlot{ClosedLoop}.
\end{urbiscriptapi}

%%% Local Variables:
%%% mode: latex
%%% TeX-master: "../urbi-sdk"
%%% ispell-dictionary: "american"
%%% ispell-personal-dictionary: "../urbi.dict"
%%% fill-column: 76
%%% End:

\section{Triplet}

A \dfn{triplet} (or \dfn{triple}) is a container storing three
objects.


\subsection{Prototype}
\begin{itemize}
\item \refObject{Tuple}
\end{itemize}

\subsection{Construction}

A \lstinline|Triplet| is constructed with three arguments.

\begin{urbiscript}[firstnumber=1]
Triplet.new(1, 2, 3);
[00000001] (1, 2, 3)

Triplet.new(1, 2);
[00000003:error] !!! Triplet.init: expected 3 arguments, given 2

Triplet.new(1, 2, 3, 4);
[00000003:error] !!! Triplet.init: expected 3 arguments, given 4
\end{urbiscript}

\subsection{Slots}
\begin{itemize}
\item \lstinline|first|\\
  Return the first member of the pair.
\begin{urbiassert}
Triplet.new(1, 2, 3).first == 1;
Triplet[0] === Triplet.first;
\end{urbiassert}

\item \lstinline|second|\\
  Return the second member of the triplet.
\begin{urbiassert}
Triplet.new(1, 2, 3).second == 2;
Triplet[1] === Triplet.second;
\end{urbiassert}

\item \lstinline|third|\\
  Return the third member of the triplet.
\begin{urbiassert}
Triplet.new(1, 2, 3).third == 3;
Triplet[2] === Triplet.third;
\end{urbiassert}
\end{itemize}



%%% Local Variables:
%%% mode: latex
%%% TeX-master: "../urbi-sdk"
%%% ispell-dictionary: "american"
%%% ispell-personal-dictionary: "../urbi.dict"
%%% End:

\section{Tuple}

A \dfn{tuple} is a container storing a fixed number of objects.
Examples include \refObject{Pair} and \refObject{Triplet}.

\subsection{Prototype}
\begin{itemize}
\item \refObject{Object}
\end{itemize}

\subsection{Construction}

The \lstinline|Tuple| object is not meant to be instantiated, its main
purpose is to share code for its descendants, such as \refObject{Pair}.
Yet it accepts its members as a list.

\begin{urbiscript}[firstnumber=1]
var t = Tuple.new([1, 2, 3]);
[00000000] (1, 2, 3)
\end{urbiscript}

The output generated for a \lstinline|Tuple| can also be used to create a
\lstinline|Tuple|.  Expressions are put inside parenthesis and separated by
commas.  One extra comma is allowed after the last element.  To avoid
confusion between a 1 member \lstinline|Tuple| and a parenthesized
expression, the extra comma must be added.  \lstinline|Tuple| with no
expressions are also accepted.

\begin{urbiscript}
// not a Tuple
(1);
[00000000] 1

// Tuples
();
[00000000] ()
(1,);
[00000000] (1,)
(1, 2);
[00000000] (1, 2)
(1, 2, 3, 4,);
[00000000] (1, 2, 3, 4)
\end{urbiscript}


\subsection{Slots}
\begin{urbiscriptapi}
\item[asString]
  Generate the string \samp{(\var{first}, \var{second}, ...)} using
  \code{asPrintable} to convert members to strings.

\item \lstinline|'[]'(\var{index})|\\
  Return the \var{index}-th element.  \var{index} must be in bounds.
\begin{urbiassert}
(1, 2, 3)[0] == 1;
(1, 2, 3)[1] == 2;
\end{urbiassert}

\item \lstinline|'[]='(\var{index}, \var{value})|\\
  Set (and return) the \var{index}-th element to \var{value}.
  \var{index} must be in bounds.
\begin{urbiscript}
{
  var t = (1, 2, 3);
  assert
  {
    (t[0] = 2) == 2;
    t == (2, 2, 3);
  };
};
\end{urbiscript}

\item \lstinline|'<'(\var{other})|\\
  Lexicographic comparison between two tuples.
\begin{urbiassert}
(0, 0) < (0, 1);
(0, 0) < (1, 0);
(0, 1) < (1, 0);
\end{urbiassert}

\item \lstinline|'=='(\var{other})|\\
  Whether \lstinline|this| and \lstinline|other| have the same
  contents (equality-wise).
\begin{urbiassert}
  (1, 2) == (1, 2);
!((1, 1) == (2, 2));
\end{urbiassert}
\end{urbiscriptapi}



%%% Local Variables:
%%% mode: latex
%%% TeX-master: "../urbi-sdk"
%%% ispell-dictionary: "american"
%%% ispell-personal-dictionary: "../urbi.dict"
%%% End:

\section{void}

%%% Local Variables:
%%% mode: latex
%%% TeX-master: "../urbi-sdk"
%%% End:


%% Restore the definition of \section.
\let\section\sectionOrig

%%% Local Variables:
%%% mode: latex
%%% TeX-master: "../urbi-sdk"
%%% ispell-dictionary: "american"
%%% ispell-personal-dictionary: "../urbi.dict"
%%% End:

\FloatBarrier

\chapter{\urbi SDK specifications}
\label{sec:sdk}

\clearpage
\phantomsection % otherwise hyperlinks to previous chapter.
\addcontentsline{toc}{chapter}{\indexname}
\printindex

\end{document}
