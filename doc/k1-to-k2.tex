\chapter{Migration from \us 1 to \us 2}

This chapter is intended to people who want to migrate programs in \us
1 into program \us 2.  Backward compatibility is \emph{mostly}
ensured, but there are a number of constructs in \us 1 that were
removed because they prevented the introduction of a cleaner construct
in \us 2.  When possible, \us 2 supports the remaining \us 1
constructs.

\section{One to one translation}

\subsection{delete Foo}

In order to maintain an analogy with the \Cxx language, \us used to
support \lstinline{delete Foo}, but this was removed for a number of
reasons:
\begin{itemize}
\item \us 2 features genuine local variables for which
  \lstinline{delete} makes no sense.
\item in \Cxx \lstinline{delete} really targets the object: destroy
  yourself, then the system will reclaim the memory.  In \us one
  cannot destroy an object and reclaim the memory, it is the task of
  the system to notice objects that are no longer used, and to reclaim
  the memory.  This is called \dfn{garbage collection}.  Therefore in
  \us \lstinline{delete} is actually bounced to \lstinline{removeSlot}
  sent to the owner of the object.
\item \lstinline{delete} is an unsafe feature that makes only sense in
  pointer-based languages such as \C and \Cxx.  It enables nice bugs
  such as:
\begin{urbiscript}
var this.a := A.new;
// ...
delete this.a;
// ...
cout << this.a;
\end{urbiscript}
\end{itemize}

For these reasons, and others, \lstinline{delete Foo} was removed.  If
your goal is to remove the \emph{name} Foo, run
{\lstinline{removeSlot("Foo")}} (\autoref{sec:tut:objects}) --- the
garbage collector will reclaim memory if there are no other use of
\lstinline{Foo}.

\subsection{new Foo}

See \autoref{sec:tut:ctor} for details on \code{new}.  The construct
\lstinline{new Foo} is no longer supported because it is ambiguous:
what does \lstinline{new a(1,2).b(3,4)} mean?  Is
\lstinline{a(1,2).b} the object to clone and \lstinline{(3,4)} are the
arguments of the constructor?  Or is it the result of
\lstinline{a(1,2).b(3,4)} that must be cloned?

In temporary version \us 2 used to support this \code{new} construct,
but too many users got it wrong, and we decided to keep only the
simpler, safer, and more consistent method-call-style construct:
\lstinline{Foo.new}.  Every single possible interpretation of
\lstinline{new a(1,2).b(3,4)} is reported below, unambiguously.
\begin{itemize}
\item \lstinline{a(1,2).b(3,4).new}
\item \lstinline{a(1,2).b.new(3,4)}
\item \lstinline{a(1,2).new.b(3,4)}
\item \lstinline{a.new(1,2).b(3,4)}
\item \lstinline{new.a(1,2).b(3,4)}
\end{itemize}


\section{Cookbook}


%%% Local Variables:
%%% mode: latex
%%% TeX-master: "urbi-sdk"
%%% End:
