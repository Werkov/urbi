%% Copyright (C) 2010, 2011, Gostai S.A.S.
%%
%% This software is provided "as is" without warranty of any kind,
%% either expressed or implied, including but not limited to the
%% implied warranties of fitness for a particular purpose.
%%
%% See the LICENSE file for more information.

\chapter{Notations}
\label{sec:notations}

This chapter introduces the various notations that are used in the
document.

\section{Words}

\begin{itemize}
\item \var{name}\\
  Depending on the context, a \dfn{variable} name (i.e., an identifier
  in \Cxx or \us), or a \dfn{meta-variable} name.  A meta-variable
  denotes a place where some syntactic construct may be entered.  For
  instance, in \lstinline|while (\var{expression}) \var{statement}|,
  \var{expression} and \var{statement} do not denote two variable
  names, but two placeholders which can be filled with an arbitrary
  expression, and an arbitrary statement.  For instance:

  \lstinline|while (!tasks.empty) { tasks.removeFront.process }|.

\item \env{name}\\
  An environment variable name, e.g., \env{PATH}.
\item \lstinline|code|\\
  A piece of \us or \Cxx code.
\item \file{name}\\
  A file name.
\end{itemize}

\section{Frames}

\subsection{\Cxx Code}
\label{sec:notations:cxx}

\Cxx source code is presented in frames as follows.

\begin{cxx}
class Int
{
public:
  Foo(int v = 0)
    : val_(v)
  {}

  void operator(int v)
  {
    std::swap(v, val_);
    return v;
  }

  int operator() const
  {
    return val_;
  }

private:
  int val_;
};
\end{cxx}

\subsection{Grammar Excerpts}
\label{sec:notations:bnf}

The grammar fragments are written in Extend Backus-Naur Form.  The symbol
\lstindex{::=} separates the left-hand symbol from the right-hand side part
of the rule.  Infix \lstinline{|} denotes alternation, postfix-\lstinline{*}
0-or-more repetition, postfix-\lstinline{+} 1-or-more repetition, and
postfix-\lstinline{?} denotes optional parts.  Terminal symbols are written
in double-quotes, and non-terminals in angle-brackets.  Parentheses group.

\begin{bnf}
<grammar> ::= <rule>+
<rule> ::= <symbol> "::=" <rhs>
<rhs> ::= <rhs> "|" <rhs>
        | <rhs> ("?" | "*" | "+")
        | "(" <rhs> ")"
        | <symbol>
\end{bnf}

\subsection{\Java Code}
\label{sec:notations:java}

\Java source code is presented in frames as follows.

\begin{java}
import liburbi.main.*;
public class Main
{
    /// Load urbijava library.
    static
    {
        System.loadLibrary("urbijava");
    }

    public static void main(String argv[])
    {
      // Does nothing for now.
    }
}
\end{java}

\subsection{Shell Sessions}
\label{sec:notations:shell}

Interactive sessions with a (Unix) shell are represented as follows.

\begin{shell}
$ echo toto
toto
\end{shell}

The user entered \samp{echo toto}, and the system answered
\samp{toto}.  \samp{\$} is the \dfn{prompt}: it starts the lines where
the system invites the user to enter her commands.

\subsection{\us Sessions}
\label{sec:notations:us}

Interactive sessions with \urbi are represented as follows.

\begin{urbiscript}[firstnumber=1]
echo("toto");
[00000001] *** toto
\end{urbiscript}

Contrary to shell interaction (see \autoref{sec:notations:shell}),
there is no prompt that marks the user-entered lines (here
\lstinline|echo("toto");|, but, on the contrary, answers from the \urbi
server start with a label that includes a timestamp (here
\samp{00000001}), and possibly a channel name, \samp{output} in the
following example.

\begin{urbiscript}
cout << "toto";
[00000002:output] "toto"
\end{urbiscript}


\subsection{\us Assertions}
\label{sec:notations:urbiassert}

\us features assertions blocks, see \autoref{sec:lang:assert}.  The
following assertion frame:

\begin{urbiassert}
true;
1 < 2;
1 + 2 * 3 == 7;
"foobar"[0, 3] == "foo";
[1, 2, 3].map (function (a) { a * a }) == [1, 4, 9];
[ => ].empty;
\end{urbiassert}

\noindent
actually denotes the following assertion-block in an \us-session
frame (\autoref{sec:lang:assert}):

\begin{urbiscript}
assert
{
  true;
  1 < 2;
  1 + 2 * 3 == 7;
  "foobar"[0, 3] == "foo";
  [1, 2, 3].map (function (a) { a * a }) == [1, 4, 9];
  [ => ].empty;
};
\end{urbiscript}

%%% Local Variables:
%%% mode: latex
%%% TeX-master: "urbi-sdk"
%%% ispell-dictionary: "american"
%%% ispell-personal-dictionary: "../urbi.dict"
%%% fill-column: 76
%%% End:
