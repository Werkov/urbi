%% Copyright (C) 2010, 2011, Gostai S.A.S.
%%
%% This software is provided "as is" without warranty of any kind,
%% either expressed or implied, including but not limited to the
%% implied warranties of fitness for a particular purpose.
%%
%% See the LICENSE file for more information.

\chapter{Bioloid}
\label{sec:bioloid}

\section{Introduction}

The \dfn{Bioloid} is a a robot construction kit made of servomotors, sensors
and frame elements. You can find more information on the manufacturer's
official website at \url{http://www.robotis.com/zbxe/bioloid_en}.

Urbi cannot run directly on the Bioloid controller (CM5, CM510). Urbi runs
on your computer and talks to the Bioloid controller over serial link
(RS232, or wireless depending on your configuration). Urbi can also talk
directly to the motor bus if you have an usb2dynamixel.

\section{Installing Urbi for Bioloid}

\subsection{Flashing the firmware}

Urbi for Bioloid is using a custom firmware in the CM-5 controller. You must
upload the new firmware in your CM-5 using the procedure below. This
operation is reversible.

Download the firmware file from our website at
\url{http://www.gostai.com/downloads/urbi-for-bioloid.zip}.

\begin{itemize}
\item Start the \emph{robot terminal} software from Robotis (installed with
  your Bioloid software).
\item Make sure your CM-5 batteries are fully charged or that it is plugged
  to a power supply.
\item Connect your CM-5 to the serial port of your computer.
\item Turn off your CM-5 (using the red on/off button).
\item Maintain the \samp{\#} key pressed on your keyboard while turning the
  CM-5 back on. A message like this one should be displayed:

\begin{verbatim}
  SYSTEM O.K. (CM5 boot loader V1.xx)
  -#######
\end{verbatim}

\item Press enter, type load and press enter again. Display should be:
\begin{verbatim}
  SYSTEM O.K. (CM5 boot loader V1.xx)
  -#######
  -load
  Write Address: 00000000
  Ready..
\end{verbatim}

\item In the menu, select ``Files'', ``Transmit file'' (Control-T) and
  select the firmware file you downloaded. After a few seconds, display
  should turn to:
\begin{verbatim}
  Ready..Success
  (and other information)
\end{verbatim}
\end{itemize}

\subsection{Getting Urbi and Urbi for Bioloid}

Look at \url{http://www.urbiforge.com/index.php/Robots/Bioloid} for
up-to-date information on how to download Urbi for Bioloid.

\section{First steps}

\subsection{Starting up}
Connect your CM-5 to the PC and turn it on before starting Urbi.

When you start \urbi, the initialization script in \file{global.u} will
connect to the CM-5 (edit the file to change the port if required), scan for
devices and instantiate them. It will create:

\begin{itemize}
\item one \lstinline|bioloid| object representing the connection to the CM5.
\item one object per motor (\refObject{AX12}) named
  \lstinline|motor\var{id}| and \lstinline|motors[\var{id}]| where \var{id}
  is the motor identifier.
\item one object per sensor (\refObject{AXS1}) named
  \lstinline|sensor\var{id}| and \lstinline|sensors[\var{id}]| where
  \var{id} is the sensor \var{id} (e.g., \lstinline|sensor100|,
  \lstinline|sensors[100]|).
\end{itemize}

Once initialized, do not disconnect any motor that was detected or
everything will run very slow.

At this point you might want to give motor names more adapted to your model.
Here is an example:

\begin{urbiunchecked}
do (Global) // Make them available to everyone.
{
  var wheelL = motors[1];  // Left wheel is motor ID 1.
  var wheelR = motors[2];  // Right wheel is motor ID 2.
  var headYaw = motors[7]; // Head yaw rotation is motor ID 7.
};
\end{urbiunchecked}


\subsection{Motor features}
\labelObject{AX12}
\def\currentObject{AX12}

All the AX12 features are exposed in \us. The following section lists
the main slots with code examples explaining how to use them. You can use
\refSlot[Object]{localSlotNames} and refer to the AX12 documentation for
more information.

\begin{urbiscriptapi}
\item[val] The current motor position when read, the target position when
  written to.

\begin{urbiunchecked}
// Move motor6 to 90 degrees.
motor6.val = 90deg;
// Move to 0 in 5 seconds.
motor6.val = 0 time:5s;
// Ticking clock.
at (motor6.val > 0) echo("tick") onleave echo("tack");
motor6.val = 0 sin:1s ampli:20deg,
\end{urbiunchecked}

\item[cwLimit, ccwLimit]%
  Set min and max reachable angles. When both equal to 0, the motor is put
  in continuous rotation mode. In this mode, writing to \refSlot{val} has no
  effect: the motor moves at the speed given by the \refSlot{speed} slot.

\item[speed] When read, gives the current rotation speed. When written to,
  set the speed at which following commands will be executed. In continuous
  rotation mode, start moving the motor at given speed.
\begin{urbiunchecked}
// Set speed to two radians/second.
motor6.speed = 2;
// Move to 90deg.
motor6.val = 90deg;
// Switch to continuous rotation mode.
motor6.cwLimit = motor6.ccwLimit = 0;
// Start moving counter-clockwise at 1 radian/seconds.
motor6.speed = 1;
\end{urbiunchecked}

\item[torque]
  Current torque the motor is giving.

\item[load]
  Shut down the motor when 0.

\begin{urbiunchecked}
//Turn the motor off if the torque is too high.
at(motor6.torque > 5) motor6.load = 0;
\end{urbiunchecked}

\end{urbiscriptapi}

\subsection{Sensor features}
\labelObject{AXS1}
\def\currentObject{AXS1}

All the AXS1 features are exposed in urbiscript. The following section lists
the main slots with code examples explaining how to use them.  You can use
\refSlot[Object]{localSlotNames} and refer to the AXS1 documentation for
more information.

\begin{urbiscriptapi}

\item[IRLeft, IRCenter, IRRight] Amount of infrared received by the sensor
  using the internal emitter. This sensors detect light reflected by objects
  nearby in the direction of the sensor.

\item[lightLeft, lightCenter, lightRight] Amount of infrared received by the
  sensor without the emitter. Detects infrared light sources such as
  incandescent light, candles, ...

\item[buzzerIndex]
  Play a note when written to.

\item[buzzerTime]
  Set length of next played note.

\item[clapCount]
  Number of successive claps detected by the microphone.

\item[soundVolume]
  Volume of sound received by the microphone.

\begin{urbiunchecked}
// The louder the sound, the faster we go.
at (sensor100.soundVolume->changed?)
  motor6.speed = sensor100.soundVolume * 6;

// Stop after 4 claps.
at (sensor100.clapCount == 4)
  motor6.speed = 0;
\end{urbiunchecked}

\item[soundVolumeMax] Holds maximum sound volume recorded so far. Write 0 to
  this slot to reset.

\end{urbiscriptapi}


%%% Local Variables:
%%% coding: utf-8
%%% mode: latex
%%% TeX-master: "../urbi-sdk"
%%% ispell-dictionary: "american"
%%% ispell-personal-dictionary: "../urbi.dict"
%%% fill-column: 76
%%% End:
