%% Copyright (C) 2010, 2011, Gostai S.A.S.
%%
%% This software is provided "as is" without warranty of any kind,
%% either expressed or implied, including but not limited to the
%% implied warranties of fitness for a particular purpose.
%%
%% See the LICENSE file for more information.

\chapter{Mindstorms NXT}
\label{sec:nxt}
\setHtmlFileName{nxt}

This documentation contains information about the LEGO Mindstorms NXT \urbi
engine.  The first two sections provide a short tutorial for using \urbi to
control the TriBot robot.  The remainder of the documentation contains an
exhaustive reference for the LEGO Mindstorms NXT \urbi engine devices.

\section{Launching \urbi for Mindstorms NXT}

First you should build the TriBot model from the LEGO Mindstorms NXT basic
model.

Install the engine Mindstorms by untaring it into the Urbi engine runtime
folder. Follow instructions in \file{INSTALL} and \file{README} files.

The file \file{TriBot.u} is loaded by default when server is launched. You
can remove it by modifying \file{global.u}.

\section{First steps with \urbi to control Mindstorms NXT}

This section is a short tutorial for using \urbi with the LEGO TriBot model
with the default configuration provided with the \urbi server. Users
familiar with \urbi should look directly to the next section for an
extensive list of the available devices.

\subsection{Make basic movements}

We will first move the wheels of the robot.

In \urbi, all the motor's speed and the sensor's value in a robot are associated
with variables. You can set the motor's speed or get the sensor's value by
assigning or reading the values of the corresponding variables.

In the default server layout for the TriBot model, the left wheel is assigned
to variable wheelL and the right wheel is assigned to wheelR\footnote{%
%%
  To be precise, sensors and motors are associated with objects. The left
  motor is associated with the wheelL object and the left motor speed is
  associated with the wheelL.speed slot of this object.
%%
}. The values of these variables control the corresponding wheel
speed. First put the TriBot upside-down to avoid any accident, then you can
move the left wheel just doing:
\begin{urbiunchecked}
wheelL.speed = 50;
\end{urbiunchecked}

Don't omit the semicolon at the end of the line, or the command will not be
executed. You can also move the right wheel (a negative value is a move
backward):
\begin{urbiunchecked}
wheelR.speed = -50;
\end{urbiunchecked}

A group is also defined so that you can give orders to several motors at the
same time. In our case, the group wheels contains both robot wheels. Setting
wheels to a value will set both wheels to this value:
\begin{urbiunchecked}
wheels.speed = 0;
\end{urbiunchecked}

\noindent
will stop both wheels.  The third motor of the TriBot is associated with the
claw variable.

\subsection{Improving the movements}
The commands given so far simply set a speed to the wheels, but you can use
\urbi to make precise sequences of movements. It is for example possible to set
a speed for a given time, then stop.

To achieve that, \urbi provides a command to wait for a given time: sleep
(duration). And you can make sequences of commands separated with semicolon
that will be executed one after another. So the following line:
\begin{urbiunchecked}
wheels.speed = 50;
sleep(10s);
wheels.speed = 0;
\end{urbiunchecked}

\noindent
will make the robot to go forward during 10 seconds then stop.

It is also possible with \urbi to control the way the values will change. In the
previous examples, the motor speed goes from 0 to 50 as fast as possible. You
can use a modifier to the variable assignation to control the way the variable
value will change. For example, if you want to reach a value in a given time,
use the time: modifier:
\begin{urbiunchecked}
wheels.speed = 50 time:3s;
wheels.speed = 0 time:3s;
\end{urbiunchecked}

Using this code, the robot will accelerate to reach the speed of 50 after 3
seconds, then slow down to stop after another 3 seconds.  Other modifiers
are available in \urbi, described in the \urbi documentation.

\subsection{Reading sensors}
In all the previous steps the commands don't take the environment into account.
To do that, you need to get the sensor values. In \urbi, you just have to type
the associated variable name. For example when you type:
\begin{urbiunchecked}
sonar.val;
\end{urbiunchecked}

The server returns a message with the ultrasonic sensor value:
\begin{urbiunchecked}
[00146711] 48
\end{urbiunchecked}

In this message, 146711 is the time (from the server clock) at which the value
was read, 48.00000 is the value of the sonar sensor, in this case the distance
to the obstacle in front of it.
You also need a way to react to the sensor values. In \urbi, this is achieved
using event catching commands. These commands check some condition and launch
other commands when these conditions are verified. For example the following
command:
\begin{urbiunchecked}
at (sonar.val < 50)
  wheels.speed = 0;
\end{urbiunchecked}

\noindent
will stop the robot if an obstacle is detected at less than 50 centimeters.
Note that this command remains active in the server. If you move the robot
and set a forward speed:
\begin{urbiunchecked}
wheels.speed = 50;
\end{urbiunchecked}
\noindent
the robot will stop again when another obstacle is encountered.  The other
sensors available for the TriBot are the sound sensor decibel, the bumper
sensor bumper and the light sensor light. The following command:
\begin{urbiunchecked}
at (bumper.val == 1)
  wheels.speed = -50;
\end{urbiunchecked}
\noindent
will make the robot go backward if something hits the bumper.  Note that the
at command reacts only when the condition becomes true. If you want to loop
the execution of an action whenever the condition is true, use the whenever
command (see \autoref{sec:nxt:sounds}).


\subsection{Tagging commands}
The \lstinline{at} commands you entered before are still active in the
server. You need a way to stop them if you want to do something else. The
tagging mechanism will enable you to do that.  A tag in a name associated to
a command. To tag a command with the name \lstinline{myTag}, create it and
simply prefix the command. For example:
\begin{urbiunchecked}
var myTag = Tag.new|;
myTag: at (decibel.val > 0.7)
  wheels.speed = 0;
\end{urbiunchecked}
\noindent
will launch the event catching command using the decibel sensor with tag
\lstinline{myTag}.  Later you will be able to stop that command:
\begin{urbiunchecked}
myTag.stop;
\end{urbiunchecked}
\noindent
will stop the previous command and the robot will not react to sound anymore.
It if also possible to freeze temporarily a command:
\begin{urbiunchecked}
myTag.freeze;
\end{urbiunchecked}
\noindent
will stop the commands tagged with myTag, but it will not be deleted, and
\begin{urbiunchecked}
myTag.unfreeze;
\end{urbiunchecked}
\noindent
will resume the command.

\subsection{Playing sounds}
\label{sec:nxt:sounds}
The LEGO Mindstorms NXT is able to play sounds. The object beeper is associated
with this capacity. However, you can't just assign a value to the variable
beeper, because playing a sound requires several parameters. You therefore have
to call a method of the beeper object:
\begin{urbiunchecked}
beeper.play(200, 3s);
\end{urbiunchecked}

This will play a beep at 200Hz during 3 seconds (beep's frequency must be
between 200Hz and 14000Hz).  The parameters of the method may also be the
result of operations, or depend on other variables. For example, it is
possible to play a sound which frequency depends on the obstacle distance:
\begin{urbiunchecked}
var myTag = Tag.new|;
myTag: whenever (sonar.val < 100)
  beeper.play(200 + 6 * sonar.val, 3ms);
\end{urbiunchecked}

This plays beeps with a lower frequency as the obstacle comes closer (under
1 meter).

\subsection{Cyclic moves}
We now introduce a modifier in \urbi that makes it possible to do cyclic
moves of sinusoidal shape. This modifier has the particularity to make the
assignation never terminate. To be able to enter new commands after this
modifier is used, you need to put this command in background so that the
\urbi server continues processing new commands. This is achieved by
terminating the command with a comma instead a semicolon. More details about
these command separators will be given in \autoref{sec:nxt:para}.  The sinus
modifier can be used this way:
\begin{urbiunchecked}
var myTag = Tag.new|;
myTag: wheels.speed = 0 sin:10s ampli:100,
\end{urbiunchecked}

This command will make the speed of the wheels oscillate around 0, between -100
et 100 with a period of 10s. Use a tag to be able to stop the command. To make
the speed change from 0 to 100 with a period of 3 seconds, use the command:
\begin{urbiunchecked}
var tag2 = Tag.new;
tag2: wheels.speed = 50 sin:3s ampli:50,
\end{urbiunchecked}



\subsection{Parallelism}
\label{sec:nxt:para}
\urbi handles command parallelism in a very simple way. Parallelism is enforced
through various command separators:

\begin{itemize}
\item As usual in programming languages, two commands separated by a
  semicolon will be executed one after the other:
\begin{urbiunchecked}
wheels.speed = 100;
sleep(2s);
wheels.speed = 0;
\end{urbiunchecked}

\noindent
will turn the wheels, wait 2 seconds and then stop.

\item In \urbi, two commands separated by \lstinline|&| will be executed in
  parallel and will start exactly at the same time:
\begin{urbiunchecked}
wheelL.speed = 50 & wheelR.speed = -50;
\end{urbiunchecked}

\noindent
will move at the same time the left wheel forward and the right wheel
backward (so the robot will turn).

\item The comma makes the first action to go in background and starts the
  next one as soon as possible, without enforcing a simultaneous start of
  the two commands:
\begin{urbiunchecked}
wheelL.speed = 50 time:10s,
wheelR.speed = -40 time:10s,
\end{urbiunchecked}

\noindent
will not wait that wheelL grows to 50 to make wheelR decrease to -40, but
will not enforce that the commands terminate at the same time, which would
be the case with the \lstinline|&| separator.
\end{itemize}

This explain why the comma was necessary in the previous section. If we
finish a command that last forever with a semicolon, the server will wait
forever before starting a new command. If you happen to make this mistake,
it is still possible to open a new connection to the \urbi server using
another client and using the stop tag; command.

Here is an example where we use parallelism to move the TriBot awkwardly
while playing sounds:
\begin{urbiunchecked}
wheelR.speed = 0 sin:3s ampli:50 &
wheelL.speed = 0 sin:4s ampli:30 &
beeper.play(200,3s),
\end{urbiunchecked}

Parallelism handling is a key feature of \urbi, which has in fact four
command separators.

\subsection{Using functions}
As in most programming language, it is possible to write functions in
\urbi. The functions are useful for making more complex programs where the
same commands are used several times.  The following example defines a
function which moves the robot for a given time at a given speed:
\begin{urbiunchecked}
function Global.forward(speed,timer)
{
  wheels.speed = speed;
  sleep(timer);
  wheels.speed = 0;
},
\end{urbiunchecked}


Once defined, this function can be called simply:
\begin{urbiunchecked}
Global.forward(100,3000);
\end{urbiunchecked}

More details about functions, objects and variables in \urbi are available
in the \urbi tutorial.

\subsection{Loading files}
When making larger \urbi programs, it is easier to save them in files and to
command the server to load the files.  The files are simple text files in
which you write \urbi commands. These files should be saved in the data/
sub-directory of your sever directory. The files should use the .u extension
which is the extension used for \urbi script files.  An example
\file{demo.u} is provided with the server. To make the \urbi server to
execute this script, type the \urbi command:
\begin{urbiunchecked}
load("demo.u");
\end{urbiunchecked}


\subsection{Conclusion}
This chapter has highlighted the main features of \urbi with the Mindstorms
TriBot robot. The next chapter details all the devices available for the
TriBot in the default layout.

\section{Default layout reference}

This chapter describes the objects defined in the \file{TriBot.u} layout
shipped with the \urbi Mindstorms NXT server.
\subsection{Motors}
Three motors are defined: \lstinline{wheelL}, \lstinline{wheelR} and
\lstinline{claw}.  A group, \lstinline{wheels}, is defined to group
\lstinline{wheelL} and \lstinline{wheelR}:
\begin{urbiunchecked}
wheels.speed = 10;
\end{urbiunchecked}

is equivalent to
\begin{urbiunchecked}
  wheelL.speed = 10 & wheelR.speed = 10;
\end{urbiunchecked}

The slot .speed allows you to set the current speed of the motor. This value
is set between -100 and 100.

Motor position (in degrees) can be accessed using the .val slot. This
slot is set to 0 when you turn the robot on.

\subsection{Sensors}
Sensors are grouped in the sensors group and all hardware devices (sensors +
motors + battery + beeper) are grouped in the hardware group.
\subsubsection{Bumper}
The switch device in front of the robot is called bumper. Its value is 1 if
it is pressed, 0 otherwise.
\subsubsection{Sonar}
The ultrasonic sensor is called sonar.  Its value is the distance measured
by the sensor in centimeters between 0 and 255. When the read operation
fails, 255 is returned.
\subsubsection{Decibel}
The sound sensor is called decibel.  Its value relates the level of ambient
sound. It is between 0 and 1.  Two different modes can be used by changing
the .mode slot to \lstinline{"DB"} or \lstinline{"DBA"}.
\begin{description}
\item[DBA] is a mode measuring only frequencies between 200 and 14000Hz.
\item[DB] measures a wider band.
\end{description}
Default value is \lstinline{"DBA"}.

\subsubsection{Light}
The light sensor is called light.  The value returned is between 0 an 1
representing the amount of light measured.  Three different modes are
available by changing the .mode slot.
\begin{description}
\item[Reflector] means the sensor lights on its led and measures the light
  reflected.
\item[Ambient] lights off the led and measures the ambient light.
\item[Normal] lights off the led too and return the raw value measured.
\end{description}
Default value is \lstinline{"Reflector"}.

\subsection{Battery}
The battery device is called battery.  Its value is the current battery
level between 0 and 1, 1 is full charge.
\subsection{Beeper}
The beeper device is called beeper.  You can request the beeper to play a
sound using the play method:
\begin{urbiunchecked}
beeper.play(frequency, time);
\end{urbiunchecked}

\lstinline{frequency} is an integer between 200 and 14000, which is the
frequency of the sound. time is an integer, which is the duration of the
sound in milliseconds.  A duration of 0 is infinite.  The command returns
immediately. If you want to wait until the end of the beep, use:
\begin{urbiunchecked}
{
  beeper.play(frequency, duration);
  sleep(duration)
},
\end{urbiunchecked}

\subsection{Command}
The Command UObject is not a device. It makes it possible to send direct
commands as you do with the NXT SDK. You will find more information on
\url{http:// mindstorms.lego.com/}. This object has methods to send and
receive data to the NXT:
\begin{urbiunchecked}
Command.send(buff);
\end{urbiunchecked}
where:
\begin{itemize}
\item \var{buff} is a valid command buffer (list of integers between 0 and
  255).
\end{itemize}
For example:
\begin{urbiunchecked}
Command.send([3, 10, 10, 0 , 0]);
\end{urbiunchecked}
will play a beep.


\begin{urbiunchecked}
\var{answer} = Command.request(\var{buff}, \var{size}) ;
\end{urbiunchecked}

where:
\begin{itemize}
\item \var{buff} is a valid command buffer (requiring an answer)
\item \var{size} is the size of the return buffer (it must be the exact
  size)
\item \var{answer} is a list containing the values of the return buffer ([ ]
  is returned when the request failed)
\end{itemize}

This command is designed for expert users. Using it is not recommended if you
don't know what you are doing and might result in server crashes.
Here is an example:
\begin{urbiunchecked}
answer = Command.request([7,0], 15);
\end{urbiunchecked}

that returns the information on input port 1.

\section{How to make its own layout}

In this chapter, the method to build a custom layout for a different robot
is explained. A good understanding of the objects in \urbi in useful and the
reading of the \urbi tutorial is advised.

There are three kinds of devices provided by the Mindstorms NXT server:
motors, sensors and other devices. There is one type of motor called
\lstinline{Servo}. There are four types of sensors: \lstinline{Switch},
\lstinline{SoundSensor}, \lstinline{UltraSonicSensor} and
\lstinline{LightSensor}. Two other devices are available:
\lstinline{Battery} and \lstinline{Beeper}. A complete description of these
devices is available in \autoref{sec:nxt:specs}.  Writing a layout entails
instancing particular objects from the devices above.

\subsection{Instancing Motors}
The Servo init constructor has the following syntax:
\begin{urbiunchecked}
Servo.new(\var{port});
\end{urbiunchecked}
\noindent
where \var{port} is either \lstinline{"A"}, \lstinline{"B"} or
\lstinline{"C"}. The created object makes it possible to control the motor
connected to the port in parameter.  As an example, in the current
\file{TriBot.u}, three Servo objects are created:
\begin{urbiunchecked}
wheelL = Servo.new("C");
wheelR = Servo.new("A");
claw = Servo.new("B");
\end{urbiunchecked}

The wheel left is on port C, the wheel right on A and the claw is on port B.

\subsection{Instancing sensors}
To create a Switch object, the syntax is the following:
\begin{urbiunchecked}
Switch.new(\var{port});
\end{urbiunchecked}
\noindent
where \var{port} can be 1, 2, 3 or 4.

In \file{TriBot.u} the switch is used as a bumper plugged on the port 4 so
the instancing is:
\begin{urbiunchecked}
bumper  = Switch.new(4);
\end{urbiunchecked}

UltraSonicSensor objects are created the same way:
\begin{urbiunchecked}
UltraSonicSensor.new(\var{port});
\end{urbiunchecked}

\var{port} can be 1, 2, 3 or 4.

In \file{TriBot.u} an ultrasonic sensor is connected on port 2:
\begin{urbiunchecked}
sonar = UltraSonicSensor.new(2);
\end{urbiunchecked}

The SoundSensor object has an additional parameter:
\begin{urbiunchecked}
SoundSensor.new(\var{port}, \var{mode});
\end{urbiunchecked}

\var{port} can be 1, 2, 3 or 4.  \var{mode} sets the mode of the sensor. It
can be \lstinline{"DB"} or \lstinline{"DBA"}, meaning ``decibel'' or
``decibel adjusted'' (the latter records sounds in the frequency range
200-14000Hz).  In \file{TriBot.u}, the sound sensor is on port 1 using
decibel adjusted:
\begin{urbiunchecked}
decibel = SoundSensor.new(1,"DBA");
\end{urbiunchecked}

LightSensor also has two parameters:
\begin{urbiunchecked}
LightSensor.new(\var{port}, \var{mode});
\end{urbiunchecked}

\var{port} can be 1, 2, 3 or 4. \var{mode} sets the sensor mode and it can
be \lstinline{"Ambient"} (so the sensor measures the ambient light),
\lstinline{"Reflector"} (the sensor lights on its led and measures the
reflected light) and \lstinline{"Normal"} (the sensor returns its raw
value).

In \file{TriBot.u}, the light sensor is in reflector mode on port 1:
\begin{urbiunchecked}
light = LightSensor.new(3,"Reflector");
\end{urbiunchecked}


\subsection{Other devices}
The Battery device makes it possible to get the current battery level.
\begin{urbiunchecked}
Battery.new;
\end{urbiunchecked}

In \file{TriBot.u}:
\begin{urbiunchecked}
battery = Battery.new();
\end{urbiunchecked}

The Beeper device makes it possible to emit customized beeps from the NXT
speaker.
\begin{urbiunchecked}
Beeper.new();
\end{urbiunchecked}

In \file{TriBot.u}:
\begin{urbiunchecked}
beeper = Beeper.new();
\end{urbiunchecked}


\section{Available UObject Devices}
\label{sec:nxt:specs}
This section exhaustively describes the UObjects available in the LEGO
Mindstorms NXT \urbi server.

\subsection{Servo}
\defObject{Servo}
Servo describes a motor. A Servo instance contains the following slots:

\begin{urbiscriptapi}
\item[val] Motor's position.

\item[speed] Motor's speed, from -100 to 100.

\item[port] The port where the servo is plugged. It can be \lstinline{"A"},
  \lstinline{"B"} or \lstinline{"C"}. You can change the value while the
  server is running. Beware, if you change the port, that will free the old
  port (so you will be able to create another device on it) and will busy
  the new one (so you won't be able to create another device on it).

\item[init](<port>) The UObject constructor. \var{port} is a port name. (See
  previous slot port). Example:
\begin{urbiunchecked}
wheel = Servo.new("A");
\end{urbiunchecked}
\end{urbiscriptapi}


\subsection{UltraSonicSensor}
\defObject{UltraSonicSensor}
UltraSonicSensor describes a sonar sensor. An UltraSonicSensor instance
contains the following slots:

\begin{urbiscriptapi}
\item[val] Distance in cm (from 0 to 255).

\item[port] The port where the servo is plugged. It can be 1, 2, 3 or 4. You
  can change the value while the server is running. Beware, if you change
  the port, that will free the old port (so you will be able to create
  another device on it) and will busy the new one (so you won't be able to
  create another device on it).

\item[init](<port>) The UObject constructor. \var{port} is a port name. (See
  previous slot port). Example:
\begin{urbiunchecked}
sonar = UltraSonicSensor.new(1);
\end{urbiunchecked}

\end{urbiscriptapi}

\subsection{SoundSensor}
\defObject{SoundSensor}

SoundSensor describes a sound sensor. A SoundSensor instance contains the
following slots:

\begin{urbiscriptapi}

\item[val] The sound level measured (from 0 to 1).

\item[mode] \lstinline{"DB"} is decibel or \lstinline{"DBA"} is decibel
  audible (measures only in the audible frequencies range).

\item[port] The port where the servo is plugged. It can be 1, 2, 3 or 4. You
  can change the value while the server is running. Beware, if you change
  the port, that will free the old port (so you will be able to create
  another device on it) and will busy the new one (so you won't be able to
  create another device on it).


\item[init](<port>, <mode>) The UObject constructor. \var{port} is a port
  name. (See previous slot port). \var{mode} is a mode name. (See
  previous slot mode). Example:
\begin{urbiunchecked}
decibel = SoundSensor.new(1, "DB");
\end{urbiunchecked}

\end{urbiscriptapi}

\subsection{LightSensor}
\defObject{LightSensor}

LightSensor describes a light sensor. A LightSensor instance contains the
following slots:

\begin{urbiscriptapi}
\item[val] The light level measured (from 0 to 1).

\item[mode]
  \begin{itemize}
  \item \lstinline{"Ambient"} measures the ambient light.
  \item \lstinline{"Reflector"} lights on a led and measures the reflection.
  \item \lstinline{"Normal"} returns the raw value
  \end{itemize}

\item[port] The port where the servo is plugged. It can be 1, 2, 3 or 4. You
  can change the value while the server is running. Beware, if you change
  the port, that will free the old port (so you will be able to create
  another device on it) and will busy the new one (so you won't be able to
  create another device on it).

\item[init](<port>, <mode>) The UObject constructor. \var{port} is a port
  name. (See previous slot port). \var{mode} is a mode name. (See previous
  slot mode). Example:
\begin{urbiunchecked}
light = LightSensor.new(1, "Ambient");
\end{urbiunchecked}
\end{urbiscriptapi}

\subsection{Switch}
\defObject{Switch}

Switch describes a touch sensor. A Switch instance contains the following
slots:

\begin{urbiscriptapi}
\item[val] The switch status (0 or 1).

\item[port] The port where the servo is plugged. It can be 1, 2, 3 or 4. You
  can change the value while the server is running. Beware, if you change
  the port, that will free the old port (so you will be able to create
  another device on it) and will busy the new one (so you won't be able to
  create another device on it).

\item[init](<port>) The UObject constructor. \var{port} is a port name. (See
  previous slot port). Example:
\begin{urbiunchecked}
bumper = Switch.new(1);
\end{urbiunchecked}

\end{urbiscriptapi}

\subsection{Battery}
\defObject{Battery}

Battery describes a battery device. A Battery instance contains the
following slots:
\begin{urbiscriptapi}
\item[val] Battery remaining power between 0 and 1.
\item[init] The UObject constructor. Example:
\begin{urbiunchecked}
battery = Battery.new;
\end{urbiunchecked}
\end{urbiscriptapi}

\subsection{Beeper}
\defObject{Beeper}

Beeper describes a speaker device. A Beeper instance contains the following
methods:

\begin{urbiscriptapi}
\item[init] The UObject constructor.  Example:
\begin{urbiunchecked}
beeper = Beeper.new;
\end{urbiunchecked}
\item[play](<frequency>, <duration>) Plays a beep of a custom frequency for
  a custom duration.  \var{frequency} is the frequency of the beep (in Hz),
  between 200 and 14000.  \var{duration} is the duration of the beep (in
  seconds), 0 is infinite.  Example:
\begin{urbiunchecked}
beeper.play(200,1s);
\end{urbiunchecked}
\end{urbiscriptapi}

\subsection{Command}
\defObject{Command}

Command allows you to use expert commands. The Command UObject contains the
following methods:
\begin{urbiscriptapi}
\item[send](<buffer>) Sends a command without requiring any answer.
  \var{buffer} must be a list of integers between 0 and 255.  Example:
\begin{urbiunchecked}
Command.send([3,10,10,0,0]);
\end{urbiunchecked}
\noindent
that plays a beep.

\item[request](<bufferIn>, <sizeOut>)%
  Send a command requiring an answer, and return the buffer out. This is
  only recommended to expert users. Beware of the \var{sizeOut}
  parameter. Indeed if it is too long, the request will normally fail and
  when the size is too short, the command may work but all Mindstorms
  internal buffers will be shift back (so this will bend all other
  requests).

  \var{bufferIn} is the buffer of the command. \var{sizeOut} is the size of
  the answer.

  Example:
\begin{urbiunchecked}
answer = Command.request([7,0], 15);
\end{urbiunchecked}

that returns the information on input port 0.
\end{urbiscriptapi}

\subsection{Instances}
Here are the different instances in the \file{TriBot.u} layout.

\begin{center}
  \begin{tabular}{|l|l|l|}
    \hline
    Instance & UObject          & Description    \\
    \hline
    wheelL   & Servo            & Left wheel     \\
    wheelR   & Servo            & Right wheel    \\
    claw     & Servo            & Claw           \\
    sonar    & UltraSoundSensor & Distance sensor\\
    decibel  & SoundSensor      & Sound sensor   \\
    light    & LightSensor      & Light sensor   \\
    bumper   & Switch           & Touch sensor   \\
    beeper   & Beeper           & Emits beeps    \\
    battery  & Battery          & Battery        \\
    \hline
  \end{tabular}
\end{center}

\subsection{Groups}
Here are the different groups provided in the current layout.

\begin{center}
  \begin{tabular}{|l|l|l|}
    \hline
    Group Name & Description     \\
    \hline
    wheels     & The two wheels  \\
    motors     & All the motors  \\
    sensors    & All the sensors \\
    hardware   & All the devices \\
    \hline
  \end{tabular}
\end{center}

%%% Local Variables:
%%% coding: utf-8
%%% mode: latex
%%% TeX-master: "../urbi-sdk"
%%% ispell-dictionary: "american"
%%% ispell-personal-dictionary: "../urbi.dict"
%%% fill-column: 76
%%% End:
