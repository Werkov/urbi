%% Copyright (C) 2010, 2011, 2012, Gostai S.A.S.
%%
%% This software is provided "as is" without warranty of any kind,
%% either expressed or implied, including but not limited to the
%% implied warranties of fitness for a particular purpose.
%%
%% See the LICENSE file for more information.

\chapter{Notations}
\label{sec:notations}

This chapter defines the \dfn{notations} used in this document.

\section{Words}

\begin{itemize}
\item \lstinline|code|\\
  A \dfn{piece of code} (\us, \Java, \Cxx\ldots).

\item \textcmt{comment}\\
  A \dfn{comment} in some programming language.  For instance,
  \lstinline|/* hello /* world */ ! */| is a comment in \us.

\item \env{environment-variable}\\
  An \dfn{environment variable} name, e.g., \env{PATH}.

\item \file{file-name}\\
  A \dfn{file name}.

\item \textkwd{keyword}\\
  A \dfn{keyword} in some programming language.  For instance,
  \lstinline|watch| is an \us keyword.

\item \var{meta-variable}\\
  Depending on the context, a \dfn{variable} name (i.e., an identifier in
  \Cxx or \us), or a \dfn{meta-variable} name.  A meta-variable denotes a
  place where some syntactic construct may be entered.  For instance, in
  \lstinline|while (\var{expression}) \var{statement}|, \var{expression} and
  \var{statement} do not denote two variable names, but two placeholders
  which can be filled with an arbitrary expression, and an arbitrary
  statement.  For instance:

  \lstinline|while (!tasks.empty) { tasks.removeFront.process }|.

\item \textstr{string}\\
  A \dfn{string} in some programming language.  For instance,
  \lstinline|"Hello, world!"| is a string in \us.
\end{itemize}

\section{Frames}

\subsection{\Cxx Code}
\label{sec:notations:cxx}

\Cxx source code is presented in frames as follows.

\begin{cxx}
class Int
{
public:
  Foo(int v = 0)
    : val_(v)
  {}

  void operator(int v)
  {
    std::swap(v, val_);
    return v;
  }

  int operator() const
  {
    return val_;
  }

private:
  int val_;
};
\end{cxx}

%% Using lstlistings inside an \ifthen does not work, make it another file.
\subsection{Grammar Excerpts}
\label{sec:notations:bnf}

The \dfn{grammar fragments} are written in \dfn{EBNF} (\dfn{Extended
  Backus-Naur Form}).  The symbol \lstindex{::=} separates the left-hand
symbol from the right-hand side part of the rule.  Infix \lstinline{|}
denotes alternation, postfix-\lstinline{*} 0-or-more repetition,
postfix-\lstinline{+} 1-or-more repetition, and postfix-\lstinline{?}
denotes optional parts.  Terminal symbols are written in double-quotes, and
non-terminals in angle-brackets.  Parentheses group.

The following frame defines the grammar syntax expressed in the same grammar
syntax.

\begin{bnf}
<grammar> ::= <rule>+

<rule> ::= <symbol> "::=" <rhs>

<rhs> ::= <rhs>*
        | <rhs> "|" <rhs>
        | <rhs> ("?" | "*" | "+")
        | "(" <rhs> ")"
        | <symbol>

<symbol> ::= "<" <identifier> ">"
           | '"' <escaped-character>* '"'
           | "'" <escaped-character>* "'"
\end{bnf}

\ifthen{\boolean{urbisdk}\AND\boolean{urbiThree}}{%
  \autoref{sec:grammar} provides the grammar of \us.}

\subsection{\Java Code}
\label{sec:notations:java}

\Java source code is presented in frames as follows.

\begin{java}
import liburbi.main.*;
public class Main
{
    /// Load urbijava library.
    static
    {
        System.loadLibrary("urbijava");
    }

    public static void main(String argv[])
    {
      // Does nothing for now.
    }
}
\end{java}

\subsection{Shell Sessions}
\label{sec:notations:shell}

Interactive sessions with a (Unix) shell are represented as follows.

\begin{shell}
$ echo toto
toto
\end{shell}

The user entered \samp{echo toto}, and the system answered \samp{toto}.
\samp{\$} is the \dfn{prompt}: it starts the lines where the system invites
the user to enter her commands.

\subsection{\us Sessions}
\label{sec:notations:us}

Interactive sessions with \urbi are represented as follows.

\begin{urbiscript}[firstnumber=1]
echo("toto");
[00000001] *** toto
\end{urbiscript}

Contrary to shell interaction (see \autoref{sec:notations:shell}), there is
no prompt that marks the user-entered lines (here \lstinline|echo("toto");|,
but, on the contrary, answers from the \urbi server start with a label that
includes a timestamp (here \samp{00000001}), and possibly a channel name,
\samp{output} in the following example.

\begin{urbiscript}
cout << "toto";
[00000002:output] "toto"
\end{urbiscript}


\subsection{\us Assertions}
\label{sec:notations:urbiassert}

The following \dfn{assertion frame}:

\begin{urbiassert}
true;
1 < 2;
1 + 2 * 3 == 7;
"foobar"[0, 3] == "foo";
[1, 2, 3].map (function (a) { a * a }) == [1, 4, 9];
[ => ].empty;
\end{urbiassert}

\noindent
denotes the following assertion-block\ifthen{\boolean{urbisdk}}{ (see
  \autoref{sec:lang:assert})} in an \us-session frame:

\begin{urbiscript}
assert
{
  true;
  1 < 2;
  1 + 2 * 3 == 7;
  "foobar"[0, 3] == "foo";
  [1, 2, 3].map (function (a) { a * a }) == [1, 4, 9];
  [ => ].empty;
};
\end{urbiscript}

%%% Local Variables:
%%% coding: utf-8
%%% mode: latex
%%% TeX-master: "urbi-sdk"
%%% ispell-dictionary: "american"
%%% ispell-personal-dictionary: "../../urbi.dict"
%%% fill-column: 76
%%% End:
