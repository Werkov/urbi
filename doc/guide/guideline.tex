%% Copyright (C) 2011, Gostai S.A.S.
%%
%% This software is provided "as is" without warranty of any kind,
%% either expressed or implied, including but not limited to the
%% implied warranties of fitness for a particular purpose.
%%
%% See the LICENSE file for more information.

\chapter{Urbi Guideline}
\label{sec:guideline}

\section{\us Programming Guideline}

\subsection{Prefer Expressions to Statements}

Code like this:

\begin{urbiunchecked}
if (delta)
  cmd = "go";
else if (alpha)
  cmd = "turn"
else
  cmd = "stop";
\end{urbiunchecked}

\noindent
is more legible as follows:

\begin{urbiunchecked}
cmd =
  {
    if      (delta) "go"
    else if (alpha) "turn"
    else            "stop"
  };
\end{urbiunchecked}


\subsection{Avoid \lstinline|return|}
\label{sec:guideline:return}

The \lstinline|return| statement is actually costly, because it also in
charge of stopping all the event-handlers, detached code, and code sent into
background (via \samp{,} or \samp{\&}).

In a large number of cases, \lstinline|return| is actually used uselessly.
For instance instead of:

\begin{urbiscript}
function inBounds1(var x, var low, var high)
{
  if (x < low)
    return false;
  if (high < x)
    return false;
  return true;
}|;
assert
{
  inBounds1(1, 0, 2);    inBounds1(0, 0, 2);  inBounds1(2, 0, 2);
  !inBounds1(0, 1, 2);  !inBounds1(3, 0, 2);
};
\end{urbiscript}

\noindent
write

\begin{urbiscript}
function inBounds2(var x, var low, var high)
{
  if (x < low)
    false
  else if (high < x)
    false
  else
    true;
}|;
assert
{
  inBounds2(1, 0, 2);    inBounds2(0, 0, 2);  inBounds2(2, 0, 2);
  !inBounds2(0, 1, 2);  !inBounds2(3, 0, 2);
};
\end{urbiscript}

\noindent
or better yet, simply evaluate the Boolean expression.  As a matter of fact,
returning a Boolean is often the sign that you ought to ``return'' a Boolean
expression rather that evaluate it in a conditional, and return either true
or false.

\begin{urbiscript}
function inBounds3(var x, var low, var high)
{
  low <= x && x <= high
}|;
assert
{
  inBounds3(1, 0, 2);    inBounds3(0, 0, 2);  inBounds3(2, 0, 2);
  !inBounds3(0, 1, 2);  !inBounds3(3, 0, 2);
};
\end{urbiscript}


%%% Local Variables:
%%% mode: latex
%%% TeX-master: "urbi-sdk"
%%% ispell-dictionary: "american"
%%% ispell-personal-dictionary: "../urbi.dict"
%%% fill-column: 76
%%% End:
