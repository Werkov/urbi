\chapter{Frequently Asked Questions}
\label{sec:faq}

\section{Tools}
\subsection{urbi-launch and urbi}
\subsubsection{The server dies with ``stack exhaustion''}
\index{stack!size}
Your program might be deeply recursive, or use large temporary
objects.  Use \option{--stack-size} to augment the stack size, see
\autoref{sec:tools:urbi}.

Note that one stack is allocated per ``light thread''.  This can
explain why programs that heavily rely on concurrency might succeed
where sequential programs can fail.  For instance the following
program is very likely to quickly exhaust the (single) stack.

\begin{urbiunchecked}
function consume (num)
{
  if (num)
    consume(num - 1) | consume(num - 1)
} |
consume (512);
\end{urbiunchecked}

But if you use \lstinline{&} instead of \lstinline{|}, then each
recursive call to \lstinline{consume} will be spawn with a fresh
stack, and therefore none will run out of stack space:

\begin{urbiunchecked}
function consume (num)
{
  if (num)
    consume(num - 1) & consume(num - 1)
} |
consume (512);
\end{urbiunchecked}

However your machine will run out of resources: this heavily
concurrent program aims at creating no less than $2^{513}$ threads,
about $2.68 \times 10^{156}$ (a 156-digit long number, by far larger
than the number of atoms in the observable universe, estimated to
$10^{80}$).

\section{\us}
\subsection{Objects lifetime}

\subsubsection{How do I create a new Object derivative?}
Urbi is based on prototypes. To create a new Object derivative (which
will inherit all the Object methods), you can do:

\begin{urbiscript}[firstnumber=last]
var myObject = Object.new;
[00000001] Object_0x76543210
\end{urbiscript}

\subsubsection{How do I destroy an Object?}
There is no \lstindex{delete} in Urbi, for a number of reasons (see
\autoref{sec:k122:delete}).  Objects are deleted when they are no
longer used/referenced to.

In practice, users who want to ``delete an object'' actually want to
remove a slot --- see \autoref{sec:tut:objects}.  Users who want to
clear an object can empty it --- see \autoref{sec:k122:delete}.

Note that \lstinline{myObject = nil} does not explicitly destroy the
object bound to the name \lstinline{myObject}, yet it may do
so provided that \lstinline{myObject} was the last and only reference
to this object.

\subsection{Slots and variables}

\subsubsection{Is the lobby a scope?}

One frequently asked question is what visibility do variables have in
\us, especially when they are declared at the top-level interactive
loop.  In this section, we will see the mechanisms behind slots, local
variables and scoping to fully explain this behavior and determine how
to proceed to give the right visibility to variables.

For instance, this code might seem confusing at first:

\begin{urbiscript}[firstnumber=last]
var mind = 42;
[00000001] 42
function get()
{
  echo(mind);
}|;
get();
[00000000] *** 42
function Object.get()
{
  echo(mind)
}|;
// Where is my mind?
Object.get;
[00000000:error] !!! lookup failed: mind
\end{urbiscript}

\paragraph{Local variables, slots and targets}
The first point is to understand the difference between local
variables and slots. Slots are simply object fields: a name in an
object referring to another object, like members in \Cxx. They can be
defined with the \lstinline|setSlot| method, or with the
\lstinline|var| keyword.

\begin{urbiscript}[firstnumber=last]
// Add an `x' slot in Object, with value 51.
Object.setSlot("x", 51);
[00000000] 51
// This is an equivalent version, for the `y' slot.
var Object.y = 51;
[00000000] 51

// We can access these slots with the dot operator.
Object.x + Object.y;
[00000000] 102
\end{urbiscript}

On the other hand, local variables are not stored in an object, but in
the execution stack: their lifetime spans from their declaration point
to the end of the current scope. They are declared with the `var'
keyword.

\begin{urbiscript}[firstnumber=last]
function foo()
{
  // Declare an `x' local variable, with value 51.
  var x = 51;
  // `x' isn't stored in any object. It's simply
  // available until the end of the scope.
  echo(x);
}|;
\end{urbiscript}

You probably noticed that in the last two code snippets, we used the
\lstinline|var| keyword to declare both a slot in Object and a local
variable. The rule is simple: \lstinline|var| declares a slot if an
owning object is specified with the dot notation, as in %
\lstinline|var owner.slot|, and a local variable if only an
unqualified name is given, as in \lstinline|var name|.

\begin{urbiscript}[firstnumber=last]
{
  // Store a `kyle' slot in Object.
  var Object.kyle = 42;
  // Declare a local variable, limited to this scope.
  var kenny = 42;
}; // End of scope.
[00000000] 42

// Kyle survived.
echo(Object.kyle);
[00000000] *** 42

// Oh my God, they killed Kenny.
echo(kenny);
[00000000:error] !!! lookup failed: kenny
\end{urbiscript}

There is however an exception to this rule: \lstinline|do| and
\lstinline|class| scopes are designed to define a target where to
store slots. Thus, in \lstinline|do| and \lstinline|class| scopes,
even unqualified \lstinline|var| uses declare slots in the target.

\begin{urbiscript}[firstnumber=last]
// Classical scope.
{
  var arm = 64; // Local to the scope.
};
[00000000] 64

// Do scope, with target Object
do (Object)
{
  var chocolate = 64; // Stored as a slot in Object.
};
[00000000] Object

// No arm...
echo(arm);
[00000000:error] !!! lookup failed: arm
// ... but still chocolate!
echo(chocolate);
[00000000] *** 64
\end{urbiscript}

Last tricky rule you must keep in mind: the toplevel of your
connection --- your interactive session --- is a %
\lstinline|do (lobby)| scope. That is, when you type \lstinline|var x|
directly in your connection, it stores an \lstinline|x| slot in the
\lstinline|lobby| object. So, what is this \dfn{lobby}? It's precisely
the object designed to store your top-level variables. Every \urbi
server has an unique \refObject{Lobby} (note the capital), and every
connection has its \lstinline|lobby| that inherits the
\lstinline|Lobby|. Thus, variables stored in \lstinline|Lobby| are
accessible from any connection, while variables stored in a
connection's \lstinline|lobby| are local to this connection.

To fully understand how lobbies and the toplevel work, we must
understand how calls --- message passing --- work in \us.  In \us,
every call has a target. For instance, in \lstinline|Object.x|,
\lstinline|Object| is the target of the \lstinline|x| call. If no
target is specified, as in \lstinline|x| alone, the target defaults to
\lstinline|this|, yielding \lstinline|this.x|. Knowing this rules,
plus the fact that at the toplevel \lstinline|this| is
\lstinline|lobby|, we can understand better what happens when defining
and accessing variables at the toplevel:

\begin{urbiscript}[firstnumber=last]
// Since we are at the toplevel, this stores x in the lobby.
// It is equivalent to `var lobby.x'.
var x = "hello";
[00000000] "hello"

// This is an unqualified call, and is thus
// equivalent to `this.x'.
// That is, `lobby.x' would be equivalent.
x;
[00000000] "hello"
\end{urbiscript}

\paragraph{Solving the tricky example}
We now know all the scoping rules required to explain the behavior of
the first code snippet. First, let's determine why the first access to
\lstinline|mind| works:

\begin{urbiscript}[firstnumber=last]
// This is equivalent to `var lobby.myMind = 42'.
var myMind = 42;
[00000001] 42
// This is equivalent to `function lobby.getMine...'
function getMine()
{
  // This is equivalent to `echo(this.myMind)'
  echo(myMind);
}|;
// This is equivalent to `this.getMine()', i.e. `lobby.getMine()'.
getMine();
[00000000] *** 42
\end{urbiscript}

Step by step:
\begin{itemize}
\item We create a \lstinline|myMind| slot in \lstinline|lobby|, with
  value 42.
\item We create a \lstinline|getMine| function in \lstinline|lobby|.
\item We call the lobby's \lstinline|getMine| method.
\item We access \lstinline|this.myMind| from within the method. Since
  the method was called with \lstinline|lobby| as targetMine,
  \lstinline|this| is \lstinline|lobby|, and \lstinline|lobby.x|
  resolves to the previously defined 42.
\end{itemize}

We can also explain why the second test fails:

\begin{urbiscript}[firstnumber=last]
// Create the `hisMind' slot in the lobby.
var hisMind = 42;
[00000000] 42
// Define a `getHis' method in `Object'.
function Object.getHis()
{
  // Equivalent to echo(this.hisMind).
  echo(hisMind)
}|;
// Call Object's getHis method.
Object.getHis;
[00000000:error] !!! lookup failed: hisMind
\end{urbiscript}

Step by step:
\begin{itemize}
\item We create a \lstinline|hisMind| slot in \lstinline|lobby|, with
  value 42, like before.
\item We create a \lstinline|getHis| function in \lstinline|Object|.
\item We call Object's \lstinline|getHis| method.
\end{itemize}

In the method, \lstinline|this| is \lstinline|Object|. Thus
\lstinline|hisMind|, which is \lstinline|this.hisMind|, fails because
Object has no such slot.

The key to understanding this behavior is that any unqualified call
--- unless it refers to a local variable --- is destined to
\lstinline|this|. Thus, variables stored in the lobby are only
accessible from the toplevel, or from functions that are targeted on
the lobby.

\paragraph{So, where to store global variables?}
From these rules, we can deduce a simple statement: since unqualified
slots are searched in \lstinline|this|, for a slot to be global, it
must always be accessible through \lstinline|this|. One way to achieve
this is to store the slot in \lstinline|Object|, the ancestor of any
object:

\begin{urbiscript}[firstnumber=last]
var Object.global = 1664;
[00000000] 1664

function any_object()
{
  // This is equivalent to echo(this.global)
  echo(global);
}|;
\end{urbiscript}

In the previous example, typing \lstinline|global| will look for the
\lstinline|global| slot in \lstinline|this|. Since \lstinline|this|
necessarily inherits \lstinline|Object|, it will necessarily be found.

This solution would work; however, storing all global variables in
\lstinline|Object| wouldn't be very clean. \lstinline|Object| is
rather designed to hold methods shared by all objects. Instead, a
\lstinline|Global| object exists. This object is a prototype of
Object, so all his slots are accessible from Object, and thus from
anywhere. So, creating a genuine global variable is as simple as
storing it in \lstinline|Global|:

\begin{urbiscript}[firstnumber=last]
var Global.g = "I'm global!";
[00000000] "I'm global!"
\end{urbiscript}

Note that you might want to reproduce the \lstinline|Global| system
and create your own object to store your related variables in a more
tidy fashion. This is for instance what is done for mathematical
constants:

\begin{urbiscript}[firstnumber=last]
// Store all constants here
class Constants
{
  var Pi = 3.14;
  var Euler = 2.17;
  var One = 1;
  // ...
}|;
// Make them global by making them accessible from Global.
Global.addProto(Constants);
[00000000] Global

// Test it.
Global.Pi;
[00000000] 3.14
Pi;
[00000000] 3.14
function Object.testPi() { echo(Pi) }|;
42.testPi;
[00000000] *** 3.14
\end{urbiscript}

\subsubsection{How do I add a new slot in an object?}
To add a slot to an object \lstinline{O}, you have to use the
\lstinline{var} keyword, which is syntactic sugar for the
\lstinline{setSlot} method:

\begin{urbiscript}[firstnumber=last]
var O2 = Object.new |
// Syntax...
var O2.mySlot1 = 42;
[00000001] 42

// and semantics.
O2.setSlot("mySlot2", 23);
[00000001] 23
\end{urbiscript}

Note that in a method, \lstinline{this} designates the current
object.  It is needed to distinguish the name of a slot in the current
object, versus a local variable name:

\begin{urbiscript}[firstnumber=last]
{
  // Create a new slot in the current object.
  var this.bar = 42;

  // Create a local variable, which will not be known anymore
  // after we exit the current scope.
  var qux = 23;
}|
qux;
[00000001:error] !!! lookup failed: qux
bar;
[00000001] 42
\end{urbiscript}


\subsubsection{How do I modify a slot of my object?}
Use the \lstinline|=| operator, which is syntactic sugar for the
\lstinline|updateSlot| method.

\begin{urbiscript}[firstnumber=last]
class O
{
  var mySlot = 42;
}|
// Sugarful.
O.mySlot = 51;
[00000001] 51

// Sugarfree.
O.updateSlot("mySlot", 23);
[00000001] 23
\end{urbiscript}

\subsubsection{How do I create or modify a local variable?}
Use \lstinline|var| and \lstinline|=|.

\begin{urbiscript}[firstnumber=last]
// In two steps: definition, and initial assignment.
var myLocalVariable;
myLocalVariable = "foo";
[00000001] "foo"
// In a single step: definition with an initial value.
var myOtherLocalVariable = "bar";
[00000001] "bar"
\end{urbiscript}


\subsubsection{How do I make a constructor?}
\index{constructor}
You can define a method called \lstinline{init} which will be called
automatically by \lstinline{new}. For example:

\begin{urbiunchecked}
class myObject
{
  function init(x, y)
  {
    var this.x = x;
    var this.y = y;
  };
};
myInstance = myObject.new(10, 20);
\end{urbiunchecked}


\subsubsection{How can I manipulate the list of prototypes of my objects?}
The \lstindex{protos} method returns a list (which can be manipulated)
containing the list of your object prototype.

\begin{urbiunchecked}
var myObject = Object.new;
myObject.protos;
[00000001] [Object]
\end{urbiunchecked}

\subsubsection{How can I know the slots available for a given object?}
The \lstindex{localSlotNames} and \lstindex{allSlotNames} methods
return respectively the local slot names and the local+inherited slot
names.

\subsubsection{How do I create a new function?}
Functions are first class objects. That means that you can add them as
any other slot in an object:

\begin{urbiunchecked}
var myObject = Object.new;
var myObject.myFunction = function (x, y)
  { echo ("myFunction called with " + x + " and " + y) };
\end{urbiunchecked}

You can also use the following notation to add a function to your
object:

\begin{urbiunchecked}
var myObject = Object.new;
function myObject.myFunction (x, y) { /* ... */ };
\end{urbiunchecked}

\noindent
or even group definitions within a \lstinline{do} scope, which will
automatically define new slots instead of local variables and
functions:

\begin{urbiunchecked}
var myObject = Object.new;
do (myObject)
{
  function myFunction (x, y) { /* ... */ };
};
\end{urbiunchecked}

\noindent
or group those two statements by using a convenient \lstinline{class}
scope:

\begin{urbiunchecked}
class myObject
{
  function myFunction (x, y) { /* ... */ };
};
\end{urbiunchecked}


\subsection{Tags}
\index{tag}
\subsubsection{How do I create a tag?}
See \autoref{stdlib:tag:ctor}.

\subsubsection{How do I stop a tag?}

Use the \lstinline|stop| method.
\begin{urbiunchecked}
myTag.stop;
\end{urbiunchecked}

\subsubsection{Can tagged statements return a value?}
By default, tagged statements will return the value of last evaluated
expression if they have not been stopped:

\begin{urbiunchecked}
var myTag = Tag.new;
var res;
res = { myTag: { do_something; 42 } };
\end{urbiunchecked}

After the execution, \lstinline{res} will contain 42.

However, when a tag is stopped, all the statements tagged with that
tag may be forced to return a value, by giving it as a parameter to
\lstinline{stop}:

\begin{urbiunchecked}
var myTag = Tag.new;
var res;
{ res = { myTag: sleep(inf) } }, // Sleep forever (infinite amount of time),
                                 // note the "," to put the tagged statement in
                                 // the background
myTag.stop(42);
res;
[00000001] 42
\end{urbiunchecked}

\subsection{Events}
\index{event}
\subsubsection{How do I create an event?}
Events are objects, and must be created as any object by using
\lstinline{new} to create derivatives of the \lstinline{Event} object.

\begin{urbiunchecked}
var ev = Event.new;
\end{urbiunchecked}

\subsubsection{How do I emit an event?}
Use the \lstinline|!| operator.

\begin{urbiunchecked}
ev!(1, "foo");
\end{urbiunchecked}

\subsubsection{How do I catch an event?}
Use the \lstinline|at(\var{event}?\var{args})| construct.

\begin{urbiunchecked}
at(ev?(1, var msg))
  echo ("Received event with 1 and message " + msg);
\end{urbiunchecked}

The \lstinline{?} marker indicates that we are looking for an event
instead of a boolean condition. The construct \lstinline{var msg}
indicates that the \lstinline{msg} variable will be bound (as a local
variable) in the body part of the \lstinline{at} construct, with
whatever value is present in the event that triggered the
\lstinline{at}.

\subsection{Lists}
\index{list}
\subsubsection{How can I iterate over a list?}

Use the \lstinline{for} construct (\autoref{sec:lang:for:each}), or
the \lstinline|each| method (\refObject{List}):

\begin{urbiscript}[firstnumber=last]
for (var i: [10, 11, 12]) echo (i);
[00000001] *** 10
[00000002] *** 11
[00000003] *** 12
\end{urbiscript}

\subsection{\lstinline|at| and \lstinline|waituntil| with expressions}
\label{sec:faq:atexp}
Because the development of \us was focused on other features, the
initial implementation for the \lstinline|waituntil(\var{cond})| and
the \lstinline|at(\var{cond})| features are expensive: the expressions
\var{cond} are repeatedly evaluated at every ``cycle''.  As a
consequence, \strong{the whole \us program is slowed down!}.  And of
course this considerably hinders the battery lifetime.

In the future, they will be evaluated only when they are likely to
change.

There are at least two means to avoid this problems.

\subsubsection{Events}
The key \us feature for event-based programming are... events (see
\refObject{Event} and \autoref{sec:tut:events}).  The use of
\lstinline|at| and \lstinline|waituntil| with events is well
optimized, and incurs no cost on the whole program: it is effective
only when events are fired.

There, rather than observing the values of a variable for changes, it
is preferable to subscribe to the variable to be notified when it is
changed, and then to check that your conditions are verified.  For
instance, instead of the costly following program:

\begin{urbiscript}[firstnumber=last]
{
  var x = 0;
  at (x == 1)
    echo ("one!");
  for (var i : [0, 1, 2])
  {
    echo (i);
    x = i
  };
};
[00000003] *** 0
[00000003] *** 1
[00000003] *** one!
[00000003] *** 2
\end{urbiscript}

\noindent
use the following implementation:

\begin{urbiscript}[firstnumber=last]
{
  var x = 0;
  at (x->changed?)
    if (x == 1)
      echo ("one!");
  for (var i : [0, 1, 2])
  {
    echo (i);
    x = i
  };
};
[00000003] *** 0
[00000003] *** 1
[00000003] *** one!
[00000003] *** 2
\end{urbiscript}

You may also use the event guard using %
\lstinline|at (\var{event}? if \var{cond})|:

% I have no idea why I need this.  Looks like a bug in lstlisting...
\clearpage
\begin{urbiscript}[firstnumber=last]
{
  var x = 0;
  at (x->changed? if x == 1)
      echo ("one!");
  for (var i : [0, 1, 2])
  {
    echo (i);
    x = i
  };
};
[00000003] *** 0
[00000003] *** 1
[00000003] *** one!
[00000003] *** 2
\end{urbiscript}

The optimization of \lstinline|waituntil(\var{cond})| can be done
similarly.

\subsubsection{Loops}

Alternatively, \lstinline|waituntil| can be simulated with a
\lstinline|while|, but rather that looping intensively on an empty
body, you may \lstinline|sleep| for a little while, say 20ms.  This is
especially useful when there is no value that changes, such as the
creation of a new slot.

\begin{urbiscript}[firstnumber=last]
waituntil(Object.hasSlot("foo")) | echo (Object.foo),
var Object.foo = "foo"|;
sleep(100ms);
[00000003] *** foo
\end{urbiscript}

\begin{urbiscript}[firstnumber=last]
while (!Object.hasSlot("bar"))
  sleep(20ms) |
echo (Object.bar),
var Object.bar = "bar"|;
sleep(100ms);
[00000003] *** bar
\end{urbiscript}

\section{UObjects}
\index{UObject}
\subsection{Is the UObject API Thread-Safe?}
\index{thread-safety}
We are receiving a lot of questions on thread-safety issues in UObject
code. So here comes a quick explanation on how things work in plugin
and remote mode, with a focus on those questions.

\subsubsection{Plugin mode}

In \dfn{plugin mode}, all the UObject callbacks (timer, bound
functions, notifyChange and notifyAccess targets) are called
synchronously in the same thread that executes \us code. All reads and
writes to \urbi variables, through \dfn{UVar}, are done
synchronously. \strong{The UObject plugin API is not thread-safe.} If
your code uses other threads, they must not attempt to read or to
write UVars.

If your other threads need to write to UVars, or to send \us code, the
simplest solution is to queue those requests in a list (locked by a
mutex), and dequeue and execute those requests in the update function
of an UObject.

\subsubsection{Remote mode}

\paragraph{Execution model}

In \dfn{remote mode}, a single thread is also used to handle all
UObject callbacks, for all the UObjects in the same executable. It
means that two bound functions registered from the same executable
will never execute in parallel. Consider this sample \Cxx function:

\begin{cxx}
int MyObject::test(int delay)
{
  static const int callNumber = 0;
  int call = ++callNumber;
  std::cerr << "in "  << call << ": " << time() << std::endl;
  sleep(delay);
  std::cerr << "out " << call << ": " << time() << std::endl;
  return 0;
}
\end{cxx}

If this function is bound in a remote uobject, the following code:

\begin{cxx}
MyObject.test(1), MyObject.test(1)
\end{cxx}

\noindent
will produce the following output (assuming the first call to
\lstinline|time| returns 1000).

\begin{lstlisting}
in 1: 1000
out 1: 1001
in 2: 1001
out 2: 1002
\end{lstlisting}

However, the execution of the \urbi kernel is not ``stuck'' while the
remote function executes, as the following code demonstrates:

\begin{urbiunchecked}
var t = Tag.new;
test(1) | t.stop,
t:every(300ms)
  cerr << "running";
\end{urbiunchecked}

The corresponding output is (mixing the kernel and the remote outputs):

\begin{lstlisting}
[0] running
in 1: 1000
[300] running
[600] running
[900] running
out 1: 1001
\end{lstlisting}

As you can see, \urbi semantics is respected (the execution flow is
stuck until the return value from the function is returned), but the
kernel is not stuck: other pieces of code are still running.

\paragraph{Thread-safety}

The liburbi and the UObject API in remote mode are thread safe. All
operations can be performed in any thread. As always, care must be
taken for all non-atomic operations. For exemple, the following
function is not thread safe:

\begin{cxx}
void
writeToVar(UClient* cl, std::string varName, std::string value)
{
  (*cl) << varName << " = " << value << ";";
}
\end{cxx}

Two simultaneous calls to this function from different threads can
result in the two messages being mixed.  The following implementation
of the same function is thread-safe however:

\begin{cxx}
void
writeToVar(UClient* cl, std::string varName, std::string value)
{
  std::stringstream s;
  s << varName << " = " << value << ";";
  (*cl) << s.str();
}
\end{cxx}

\noindent
since a single call to UClient's \lstinline[language=C++]|operator <<|
is thread-safe.


\subsection{Shared library dependencies}
\subsubsection{urbi-launch fails with 'myuobject: file not found'. What can I do?}
If you are sure the file is there, the most probable cause is an
undefined symbol in your shared library. A libltdl quirk prevents us
from displaying a more accurate error message.  You can use a tool
named \command{ltrace} to obtain the exact error message.  Ltrace is a
standard package on most Linux distributions.  Run it with
\samp{ltrace -C -s 1024 urbi-launch ...}, and look for lines
containing \samp{dlerror} in the output. One will contain the exact
message that occurred while trying to load your shared library.

Under Mac OS X, the \env{DYLD\_PRINT\_APIS} environment variable can
be set to 1 to make the shared library loader more verbose and display
all its attempts to load a file to stderr.

Note that this problem is Mac OS X and Linux specific.

\section{Miscellaneous}
\subsection{Releases}
\subsubsection{How can I know what has changed since the latest beta release I got?}
The file \file{RELEASE-NOTES.txt} at the top of the distribution
contains the list of user-visible changes between consecutive
releases.

\subsection{Bugs}
\subsubsection{How do I report a bug?}
See the file \file{REPORTING-BUGS.txt} at the top of the
distribution. We insist on keeping \email{k2-beta@lists.gostai.com} in
copy at all times, because it helps us to track conversations with our
users.


%%% Local Variables:
%%% mode: latex
%%% TeX-master: "urbi-sdk"
%%% End:
