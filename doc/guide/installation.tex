\chapter{Installation}
\label{sec:installation}

\section{Download}

Various pre-compiled packages are provided.  They are named
\begin{center}
\file{urbi-sdk-\var{version}-\var{arch}-\var{os}-\var{compiler}.\var{ext}}
\end{center}
where
\begin{description}
\item[\var{version}] specifies the exact revision of Urbi that you are
  using.  It can be simple, \code{2.0}, or more complex,
  \code{2.0-beta3-137-g28f8880}.  In that case,
  \begin{description}
  \item[2.0] is the version of the urbiscript Kernel,
  \item[beta3] designates the third pre-release,
  \item[137] is the number of changes since beta3 (not counting
    changes in sub-packages),
  \item[g28f8880] is a version control identifier, used internally to
    track the exact version that is being tested by our users.
  \end{description}
\item[\var{arch}] describes the architecture, the \acronym{cpu}:
  \code{ARM}, \code{ppc}, or \code{x86}.
\item[\var{os}] is the operating system: \code{linux} for GNU/Linux,
  \code{osx} for Mac OS X, or \code{windows} for Microsoft Windows.
%% FIXME: We do not display the compiler name in the packages.
\item[\var{compiler}] is the tool chain used to compile the programs:
  \code{gcc4} for the GNU Compiler Collection 4.x, \code{vcxx2005} for
  Microsoft Visual \Cxx 2005, \code{vcxx2008} for Microsoft Visual
  \Cxx 2008.
\item[\var{ext}] is the package format extension: \code{tar.gz} for
  Unix style tar balls (uncompressed with \command{tar zxf
    \var{tarfile}}), and \code{zip} for Windows style zip files.
\end{description}

\section{Install}

The package is \dfn{relocatable}, i.e., it does not need to be put at
a specific location, nor does it need special environment variables to
be set.  It is not necessary to be a super-user to install it.  The
\dfn{root} of the package, denoted by \var{urbi-root} hereafter, is
the absolute name of the directory which contains the package.

\subsection{GNU/Linux and Mac OS X}

Decompress the package where you want to install it:

\begin{shell}
$ rm -rf \var{urbi-root}
$ cd /tmp
$ tar zxf \var{path-to}/urbi-sdk-2.0-linux-x86-gcc4.tar.gz
$ mv urbi-sdk-2.0-linux-x86-gcc4 \var{urbi-root}
\end{shell}

This directory, \var{urbi-root}, should contain \file{bin},
\file{FAQ.txt} and so forth.  Do not move things around inside this
directory.  In order to have an easy access to the \urbi programs, set
up your \env{PATH}:

% !!! "\var{urbi\_root}/bin:$PATH" does not escape the \var because it is
% inside a string.
\begin{shell}
$ export PATH="urbi\_root/bin:$PATH"
\end{shell}%$

\subsection{Windows}

Decompress the zip file wherever you want.

\section{Checks}

Running the various programs is the quickest way to test your urbi
installation.

\subsection{Linux and MacOS}

\begin{shell}
# Check that urbi is properly set up.
$ urbi --version

# Check that urbi-launch is properly installed.
$ urbi-launch --version
# Check that urbi-launch can find its dependencies.
$ urbi-launch -- --version

# Check that Urbi can compute.
$ urbi -e '1+2*3; shutdown;'
[00000175] 7
\end{shell}%$

\subsection{Windows}

Execute the script \file{urbi.bat}, located at the root of the
uncompressed package. It should open a terminal with an interactive
Urbi session.

% FIXME: add a listing with the urbi banner under windows?

%%% Local Variables:
%%% mode: latex
%%% TeX-master: "../urbi-sdk"
%%% ispell-dictionary: "american"
%%% ispell-personal-dictionary: "../urbi.dict"
%%% End:
