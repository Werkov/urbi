%% Copyright (C) 2010, 2011, Gostai S.A.S.
%%
%% This software is provided "as is" without warranty of any kind,
%% either expressed or implied, including but not limited to the
%% implied warranties of fitness for a particular purpose.
%%
%% See the LICENSE file for more information.

\chapter{Building \usdk}
\label{sec:build}
\setHtmlFileName{build}

\lstnewenvironment{package}[1][]
  {% Don't do that: we don't want to show "C++ program" in marginpar
   % \cxxPre%
    \lstset{language={},
      style=UrbiSDKEnv,
      #1}}
  {\cxxPost}

This section is meant for people who want to \emph{build} the \usdk.  If
you just want to install a pre-built \usdk, see \autoref{sec:installation}.

\section{Requirements}
\label{sec:build:req}

This section details the dependencies of this package.  Some of them are
required to \dfn{bootstrap} the package (\autoref{sec:build:req:bootstrap}),
i.e., to build it from the repository.  Others are required to \dfn{build}
the package (\autoref{sec:build:req:build}), i.e., to compile it from a
tarball, or after a bootstrap.  Finally, some are needed to run the test
suite (\autoref{sec:build:req:check}).

The reader in a hurry can simply run one of the following commands,
depending on her environment.

\paragraph{Debian or Ubuntu}~\\
\begin{shell}
sudo apt-get install  \
   aspell aspell-en autoconf automake bc ccache colordiff  \
   coreutils cvs doxygen flex g++ gettext git-core gnuplot  \
   graphviz help2man imagemagick libboost-all-dev  \
   liblinphone-dev libx11-dev make pkg-config  \
   python-docutils python-yaml python2.6 socat  \
   sun-java6-jdk swig tex4ht texinfo texlive-base  \
   texlive-binaries texlive-latex-extra transfig valgrind

\end{shell}

\paragraph{MacPorts}~\\
\begin{shell}
sudo port install  \
   ImageMagick aspell aspell-dict-en autoconf automake bc  \
   boost ccache colordiff coreutils cvs doxygen flex gcc42  \
   gettext git-core gmake gnuplot graphviz help2man libxslt  \
   linphone pkgconfig py26-docutils py26-yaml python26  \
   python_select socat swig swig-java texinfo texlive  \
   texlive-bin-extra texlive-htmlxml texlive-latex-extra  \
   transfig valgrind-devel

\end{shell}


\subsection{Bootstrap}
\label{sec:build:req:bootstrap}

To bootstrap this package from its repository, and then to compile it,
the following tools are needed.

\begin{description}
\item[Autoconf 2.64 or later]~\\

\begin{package}
package: autoconf
\end{package}

\item[Automake 1.11.1 or later] Note that if you have to install Automake by
  hand (as opposed to ``with your distribution's system''), you have to tell
  its \command{aclocal} that it should also look at the files from the
  system's aclocal.  If \file{/usr/local/bin/aclocal} is the one you just
  installed, and \file{/usr/bin/aclocal} is the system's one, then run
  something like this:

\begin{shell}
$ dirlist=$(/usr/local/bin/aclocal --print-ac-dir)/dirlist
$ sudo mkdir -p $(dirname $dirlist)
$ sudo /usr/bin/aclocal --print-ac-dir >>$dirlist
\end{shell}

\begin{package}
package: automake
\end{package}

\item[Cvs]
  This surprising requirement comes from the system Bison uses to fetch
  the current version of the message translations.
\begin{package}
package: cvs
\end{package}

\item[colordiff] Not a requirement, but a useful addition.  Used if exists.
\begin{package}
package: colordiff
\end{package}

\item[Git 1.6 or later]
  Beware that older versions behave poorly with submodules.
\begin{package}
package: git-core
\end{package}

\item[Gettext 1.17]
  Required to bootstrap Bison.  In particular it provides autopoint.
\begin{package}
package: gettext
\end{package}

\item[GNU sha1sum]
  We need the GNU version of sha1sum.
\begin{package}
package: coreutils
\end{package}

\item[Help2man]
Needed by Bison.
\begin{package}
package: help2man
\end{package}

\item[Python] You need \file{xml/sax}, which seems to be part only of
  Python 2.6.  Using \command{python\_select} can be useful to state
  that you want to use Python 2.6 by default (\samp{sudo
    python\_select python26}).
\begin{package}
deb: python2.6
MacPorts: python26 python_select
\end{package}

\item[Texinfo] Needed to compile Bison.
\begin{package}
package: texinfo
\end{package}

\item[yaml for Python] The AST is generated from a description written
  in YAML.  Our (Python) tools need to read these files to generate
  the AST classes.  See \url{http://pyyaml.org/wiki/PyYAML}.  The
  installation procedure on Cygwin is:

\begin{shell}
$ cd /tmp
$ wget http://pyyaml.org/download/pyyaml/PyYAML-3.09.zip
$ unzip PyYAML-3.09.zip
$ cd pyYAML-3.09
$ python setup.py install
\end{shell}

\begin{package}
MacPorts: py26-yaml
deb:   python-yaml
\end{package}

On OS X you may also need to specify the \env{PYTHONPATH}:

\begin{shell}
export PYTHONPATH="\
/opt/local/Library/Frameworks/Python.framework/Versions/2.6\
/lib/python2.6/site-packages:$PYTHONPATH"
\end{shell}
\end{description}

\subsection{Build}
\label{sec:build:req:build}

\begin{description}
\item[Boost 1.38 or later] Don't try to build it yourself, ask your
  distribution's management to install it.
\begin{package}
deb: libboost-all-dev
MacPorts: boost
windows: http://www.boostpro.com/download
\end{package}

\item[Ccache]
  Not a requirement, but it's better to have it.
\begin{package}
package: ccache
\end{package}

\item[Doxygen] Needed if you enable the Doxygen documentation (via
  \command{configure}'s option \option{--enable-documentation=doxygen,...}).
\begin{package}
package: doxygen
\end{package}

\item[Flex 2.5.35 or later]
  Beware that 2.5.33 has bugs than prevents this package from building
  correctly.
\begin{package}
package: flex
\end{package}

\item[G++ 4.0 or later] GCC 4.2 or later is a better option.  You may have a
  problem with GCC 4.4 which rejects some Bison generated code.  See
  \autoref{sec:faq:build:uninitialized}.
\begin{package}
deb: g++
MacPorts: gcc42
\end{package}

\item[Gnuplot]
  Required to generate charts of trajectories for the documentation.
\begin{package}
package: gnuplot
\end{package}

\item[GraphViz] Used to generate some of the figures in the
  documentation.  There is no GraphViz package for Cygwin, so download
  the MSI file from GraphViz' site, and install it.  Then change your
  path to go into its bin directory.

\begin{shell}
wget http://www.graphviz.org/pub/graphviz/stable/windows/graphviz-2.26.msi
PATH=/cygdrive/c/Program\ Files/Graphviz2.26/bin:$PATH
\end{shell}

\begin{package}
package: graphviz
\end{package}

\item[ImageMagick] Used to convert some of the figures in the
  documentation.
\begin{package}
MacPorts: ImageMagick
deb: imagemagick
\end{package}

\item[JDK] In order to compile support for Java UObjects, you need Sun's
  JDK.  In particular the \file{jni.h} must be available.  On OS X Snow
  Leopard, you need the ``Java Developer'' package from Apple.  See
  \url{http://connect.apple.com/}.
\begin{package}
deb: sun-java6-jdk
\end{package}

\item[GNU Make] Although we use Automake that does provide portability
  across flavors of Make, we do use GNU Make extensions.  Actually, the most
  common version, 3.81, behaves improperly on our Makefiles, so be sure to
  use GNU Make 3.82.
\begin{package}
package: make
\end{package}

\item[oRTP] The UObject middleware can use the RTP protocol to provide
  better throughput for streams.  The implementation relies on the oRTP
  library which is now part of the linphone package.  oRTP is not needed,
  but it is strongly recommended.
\begin{package}
MacPorts: linphone
deb: liblinphone-dev
\end{package}

\item[PDFLaTeX] TeX Live is the most common \TeX{} distribution nowadays.
  We use \command{pdfcrop} from \file{texlive-bin-extra}.
\begin{package}
deb: texlive-base texlive-latex-extra texlive-binaries
MacPorts: texlive texlive-latex-extra texlive-bin-extra
\end{package}

\item[pkg-config] To find some libraries (such as oRTP), we use
  \command{pkg-config}.
\begin{package}
deb: pkg-config
MacPorts: pkgconfig
\end{package}

\item[py-docutils] This is not a requirement, but it's better to have
  it.  Used by the test suite.  Unfortunately the name of the package
  varies between distributions.  It provides \command{rst2html}.
\begin{package}
deb: python-docutils
MacPorts: py26-docutils
\end{package}

\item[ROS CTurtle or later] Needed to enable our ROS bridge.  Versions
  CTurtle, Diamondback, and Electric Emys (1.2.X, 1.4.X, 1.6.X) are known to
  work.  The packages and installation processes are detailed at
  \url{http://www.ros.org/wiki/ROS}.

\item[SWIG 1.3.36 or later] Used to generate the Java support for UObjects.
  Versions 1.3.36, 1.3.40, and 2.0.0 are known to work.
\begin{package}
package: swig
MacPorts: swig-java
\end{package}

\item[TeX4HT] Used to generate the HTML documentation.
\begin{package}
deb: tex4ht
MacPorts: texlive-htmlxml
\end{package}

\item[Transfig] Needed to convert some figures for documentation
  (using fig2dev).
\begin{package}
package: transfig
\end{package}

\item[xsltproc] This is not a requirement, but it's better to have it. Used
  to generate reports about the \us grammar.
\begin{package}
MacPorts: libxslt
\end{package}
\end{description}

\subsection{Check}
\label{sec:build:req:check}

\begin{description}
\item[aspell] Needed by the test suite.
\begin{package}
deb: aspell aspell-en
MacPorts: aspell aspell-dict-en
\end{package}

\item[bc]
  Needed by the test suite.
\begin{package}
package: bc
\end{package}

\item[socat] Needed by the test suite to send messages to an \urbi server.
\begin{package}
package: socat
\end{package}

\item[Valgrind] Not needed, but if present, used by the test suite.
\begin{package}
deb: valgrind
MacPorts: valgrind-devel
\end{package}

\end{description}



\section{Check out}

Get the open source version tarball from \url{\downloadUrl/urbi/2.x/} and
uncompress it.  This version is bootstrapped, you can directly proceed to
the ``Configure'' step.

\section{Bootstrap}
Run

\begin{shell}
$ ./bootstrap
\end{shell}

\section{Configure}

Do not compile where the source files are (the test suite of a sub-component
of ours, Libport, is known to fail in that situation).  So, \strong{compile
  in another directory than the one containing the sources}, for instance as
follows.

\begin{shell}
$ mkdir _build
$ cd _build
$ ../configure \var{options}...
\end{shell}

Also, do not compile in a directory whose name, or the name of some of its
ancestor, use ``special characters'': stick to plain ASCII.  This is a
limitation of the Java VM which seems to be unable to traverse properly
directories with UTF-8 names when the locale is set to C
(\autoref{sec:faq:jvm}).

\subsection{configuration options}
See \samp{../configure --help} for detailed information.  Unless you
want to do funky stuff, you probably need no option.

To use \command{ccache}, pass \samp{CC='ccache gcc' CXX='ccache g++'} as
arguments to configure:

\begin{shell}
$ ../configure CC='ccache gcc' CXX='ccache g++' ...
\end{shell}

For the records, here are some of the \option{--enable-}/\option{--disable-}
specific options.

\begin{options}
\item{enable-bindings} Whether to generate the bindings to other languages,
  such as UObject Java support.  Disabling is
  fine:\option{--disable-bindings}.
\item{enable-compilation-mode=\var{mode}} Specify the flavor of the build.
  You should use \option{--enable-compilation-mode=speed}.
\item{enable-doc=\var{formats}} Specify what formats of the documentation
  must be built.  Disabling is fine: \option{--disable-doc}.
\item{enable-doc-sections=\var{sections}} If the PDF or HTML documentation
  is to be generated, specify what optional sections of the documentation
  must be built.
\item{enable-examples} Whether to build the sample programs
  (\command{urbi-ball-tracking} and so forth).  Disabling is fine:
  \option{--disable-examples}.
\item{enable-library-suffix=\var{suffix}} Specify that libraries should be
  named \file{libfoo\var{suffix}.*} instead of \file{libfoo.*}.  If
  \var{suffix} is \code{auto} (or \code{yes}, or if no argument is passed),
  then use the same suffix as Boost (e.g., \file{-vc80-d} for Visual 2005,
  debug libraries).  If \var{suffix} is \code{autodebug}, then just append
  \file{-d} for debug builds.
\item{enable-sdk-remote} Whether to build the Urbi SDK Remote libraries.
  Disabling is fine: \option{--disable-sdk-remote}.  Some of the components
  that are shared between \urbi and SDK Remote will be built anyway, so
  there will be compilations to perform in \file{sdk-remote}.
\item{enable-ufloat=\var{kind}} Do not use this option, it is experimental
  and unreliable.
\end{options}


\subsection{Windows: Cygwin}
The builds for Windows use our wrappers.  These wrappers use a database to
store the dependencies (actually, to speed up their computation).  We use
Perl, and its DBI module.  So be sort to have sqlite, and DBI.

\begin{shell}
$ perl -MCPAN -e 'install Bundle::DBI'
\end{shell}

It might fail.  The most important part is

\begin{shell}
$ perl -MCPAN -e 'install DBD::SQLite'
\end{shell}

It might suffice\ldots

\subsection{Building For Windows}

We support two builds: using Wine on top of Unix, and using Cygwin on
top of Windows.

Both builds use our wrappers around MSVC's tools (\command{cl.exe},
\command{dumpbin.exe}, \command{link.exe}).  It is still unclear whether it
was a good or a bad idea, but the wrappers use the same names as the tools
themselves.  To set up Libtool properly, you will need to pass the following
as argument to configure:

\begin{shell}
AR=lib.exe
CC='ccache cl.exe'
CC_FOR_BUILD=gcc
CXX='ccache cl.exe'
LD=link.exe
DUMPBIN='dumpbin.exe
RANLIB=:
host_alias=mingw32
--host=mingw32
\end{shell}

where: \file{cl.exe}, \file{dumpbin.exe}, \file{lib.exe}, and
\file{link.exe} are the wrappers.

Since we are cross-compiling, we also need to specify \env{CC\_FOR\_BUILD}
so that \command{config.guess} can properly guess the type of the build
machine.

\subsection{Building for Windows using Cygwin}

We use our wrappers, which is something that Libtool cannot know.  So we
need to tell it that we are on Windows with Cygwin, and pretend we use GCC,
so we pretend we run mingw.

The following options have been used with success to compile \usdk with
Visual C++ 2005.  Adjust to your own case (in particular the location of
Boost and the \samp{vcxx8} part).

\begin{shell}
$ ../configure                                          \
  -C                                                    \
  --prefix=/usr/local/gostai                            \
  --enable-compilation-mode=speed                       \
  --enable-shared                                       \
  --disable-static                                      \
  --enable-dependency-tracking                          \
  --with-boost=/cygdrive/c/gt-win32-2/d/boost_1_39      \
  AR=lib.exe                                            \
  CC='ccache cl.exe'                                    \
  CC_FOR_BUILD=gcc                                      \
  CXX='ccache cl.exe'                                   \
  LD=link.exe                                           \
  DUMPBIN=dumpbin.exe                                   \
  RANLIB=:                                              \
  host_alias=mingw32                                    \
  --host=mingw32
\end{shell}

\section{Compile}
Should be straightforward.

\begin{shell}
$ make -j3
\end{shell}

Using \command{distcc} and \command{ccache} is recommended.

\section{Install}

Running \samp{make install} works as expected.  It is a good idea to check
that your installation works properly: run \samp{make installcheck} (see
\autoref{sec:build:check}).


\section{Relocatable}

If you intend to make a relocatable tree of \usdk (i.e., a self-contained
tree that can be moved around), then run \samp{make relocatable} after
\samp{make install}.

This step \emph{requires} that you use a \code{DESTDIR} (see the Automake
documentation for more information).  Basically, the sequence of commands
should look like:

\begin{shell}
$ DESTDIR=/tmp/install
$ rm -rf $DESTDIR
$ make install DESTDIR=$DESTDIR
$ make relocatable DESTDIR=$DESTDIR
$ make installcheck DESTDIR=$DESTDIR
\end{shell}

You may now move this tree around and expect the executable to work
properly.

\section{Run}

There are some special variables in addition to the environment
variables documented in the manual.

\begin{envs}
\item[URBI\_ACCEPT\_BINARY\_MISMATCH] As a safety net, Urbi checks that
  loaded modules were compiled with exactly the same version of Urbi SDK.
  Define this variable to skip this check, at your own risks.

\item[URBI\_CHECK\_MODE] Skip lines in input that look like \urbi output.  A
  way to accept test files (\file{*.chk}) as input.

\item[URBI\_DESUGAR] Display the desugared ASTs instead of the
  original one.

\item[URBI\_IGNORE\_URBI\_U] Ignore failures (such as differences between
  kernel revision and \file{urbi.u} revision) during the initialization.

\item[URBI\_INTERACTIVE] Force the interactive mode, as if
  \option{--interactive} was passed.

\item[URBI\_LAUNCH] The path to \command{urbi-launch} that
  \command{urbi.exe} will exec.

\item[URBI\_NO\_ICE\_CATCHER] Don't try to catch SEGVs.

\item[URBI\_PARSER] Enable Bison parser traces.

\item[URBI\_REPORT] Display statistics about execution rounds performed by
  the kernel.

\item[URBI\_ROOT\_LIB\var{name}] The location of the libraries to load,
  without the extension.  The \code{LIB\var{name}} are: LIBJPEG4URBI,
  LIBPLUGIN (libuobject plugin), LIBPORT, LIBREMOTE (libuobject remote),
  LIBSCHED, LIBSERIALIZE, LIBURBI.

\item[URBI\_SCANNER] Enable Flex scanner traces.

\item[URBI\_TEXT\_MODE] Forbid binary communications with UObjects.

\item[URBI\_TOPLEVEL] Force the display the result of the top-level
  evaluation into the lobby.
\end{envs}

\section{Check}
\label{sec:build:check}
There are several test suites that will be run if you run \samp{make
  check} (\samp{-j4} works on most machines).

To rerun only the tests that failed, use \samp{make recheck}.  Some tests
have explicit dependencies, and they are not rerun if nothing was changed
(unless they had failed the previous time).  It is therefore expected that
after one (long) run of \samp{make check}, the second one will be
``instantaneous'': the previous log files are reused.

Note that running \samp{make} in the top-level is needed when you change
something deep in the package.  If you forget this \samp{make} run, the
timestamps on which the test suite depends will not be updated, and
therefore the results of the previous state of the package will be used,
instead of your fresh changes.

Some tests are extremely ``touchy''.  Because the test suite exercises \urbi
under extreme conditions, some tests may fail not because of a problem in
\urbi, but because of non-determinism in the test itself.  In this case,
another run of \samp{make check} will give an opportunity for the test to
pass (remind that the tests that passed will not be run again).  Also, using
\samp{make check -j16} is a sure means to have the \urbi scheduler behave
insufficiently well for the test to pass.  \strong{Do not send bug reports
  for such failures.}.  Before reporting bugs, make sure that the failures
remain after a few \samp{make check -j1} invocations.


\subsection{Lazy test suites}
The test suites are declared as ``lazy'', i.e., unless its
dependencies changed, a successful test will be run only once ---
failing tests do not ``cache'' their results.  Because spelling out
the dependencies is painful, we rely on a few timestamps:

\begin{files}
\item[libraries.stamp] Updated when a library is updated (libport,
  libuobject, etc.).

\item[executables.stamp] Updated when an executable is updated
  (\command{urbi-launch}, etc.).  Depends on libraries.stamp.

\item[urbi.stamp] When \urbi sources (\file{share/urbi/*.u}) are updated.

\item[all.stamp] Updated when any of the three aforementioned
  timestamps is.
\end{files}

These timestamps are updated \strong{only} when \command{make} is run
in the top-level.  Therefore, the following sequence does not work as
expected:

\begin{shell}
$ make check -C tests     # All passes.
$ emacs share/urbi/foo.u
$ make check -C tests     # All passes again.
\end{shell}

\noindent
because the timestamps were not given a chance to notice that some
\urbi source changed, so Make did not notice the tests really needed to
be rerun and \strong{the tests were not run}.

You may either just update the timestamps:

\begin{shell}
$ make check -C tests     # All passes.
$ emacs share/urbi/foo.u
$ make stamps             # Update the timestamps.
$ make check -C tests     # All passes again.
\end{shell}

\noindent
or completely disable the test suite laziness:

\begin{shell}
$ make check -C tests LAZY_TEST_SUITE=
\end{shell}

\subsection{Partial test suite runs}
You can run each test suite individually by hand as follows:
\begin{files}
\item[sdk-remote/libport/test-suite.log] Tests libport.
\begin{shell}
$ make check -C sdk-remote/libport
\end{shell}

\item[sdk-remote/src/tests/test-suite.log]
Checks liburbi, and some of the executables we ship.  Requires the
kernel to be compiled in order to be able to test some of the uobjects.

\begin{shell}
$ make check -C sdk-remote/src/tests
\end{shell}

\item[tests/test-suite.log]
Tests the kernel and uobjects.
\begin{shell}
$ make check -C tests
\end{shell}

\noindent
Partial runs can be invoked:

\begin{shell}
$ make check -C tests TESTS='2.x/echo.chk'
\end{shell}

\noindent
wildcards are supported:

\begin{shell}
$ make check -C tests TESTS='2.x/*'
\end{shell}

\noindent
To check remote uobjects tests:

\begin{shell}
$ make check -C tests TESTS='uob/remote/*'
\end{shell}

\item[doc/tests/test-suite.log] The snippets of code displayed in the
  documentation are transformed into test files.
\begin{shell}
$ make check -C doc
\end{shell}

Partial runs for the doc tests:

\begin{shell}
$ make -C doc check-TESTS TESTS='tests/specs/float-00.chk'
\end{shell}
\end{files}

%%% Local Variables:
%%% coding: utf-8
%%% mode: latex
%%% TeX-master: "../urbi-sdk"
%%% ispell-dictionary: "american"
%%% ispell-personal-dictionary: "../urbi.dict"
%%% fill-column: 76
%%% End:
