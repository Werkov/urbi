%% Copyright (C) 2010, 2011, 2012, Gostai S.A.S.
%%
%% This software is provided "as is" without warranty of any kind,
%% either expressed or implied, including but not limited to the
%% implied warranties of fitness for a particular purpose.
%%
%% See the LICENSE file for more information.

\chapter{Building \usdk}
\label{sec:build}
\setHtmlFileName{build}

\lstnewenvironment{package}[1][]
  {% Don't do that: we don't want to show "C++ program" in marginpar
   % \cxxPre%
    \lstset{language={},
      style=UrbiSDKEnv,
      #1}}
  {\cxxPost}
\label{sec:build:configure}
This section is meant for people who want to \emph{build} the \usdk.  If
you just want to install a pre-built \usdk, see \autoref{sec:installation}.

A foreword that applies to any package, not just \usdk: \strong{building or
  checking as root is a bad idea}.  Build as a regular user, and run
\samp{sudo make install} just for the install time \emph{if you need
  privileges to install to the chosen destination}.


\section{Building}
\label{sec:build:req}
\usdk uses qibuild as its buildsystem.

\subsection{Getting and installing qibuild}

The simplest and recommended way if you have python and pip is:

\begin{shell}
$ pip install qibuild
\end{shell}

Otherwise you can build it from sources:

\begin{shell}
$ git clone https://github.com/aldebaran/qibuild
$ cd qibuild
$ python setup.py install # without pip
\end{shell}

\subsection{Setting up an urbi worktree}

\begin{shell}
$ mkdir work
$ cd work
$ qisrc init
$ qisrc add https://github.com/aldebaran/libport.git
$ qisrc add https://github.com/aldebaran/urbi.git
$ qisrc add https://github.com/aldebaran/libjpeg.git
\end{shell}

\subsection{Building}

\begin{shell}
$ qibuild configure urbi
$ qibuild make urbi
\end{shell}

Add \command --release to make a release build.

\subsection{Modifying the grammar or the AST}

The build procedure above will use pregenerated files for the grammar
and the AST files.

If you wish to modify ugrammar.y, utoken.l, or ast.yml, you must enable
grammar/ast processing by passing \command -DHAVE\_CUSTOM\_BISON=On. This currently
requires custom patches into bison that you can find in urbi-sdk 2.7.5 source
tarball.

\subsection{Installing}

\begin{shell}
$ qibuild install urbi <INSTALL\_DIR> # add --release for release build
\end{shell}

\section{Run}
\label{sec:build:run}

You can now run the \command urbi-launch binary, either from the build directory or
the install directory. See \autoref{sec:tools:urbi-launch} for informations
on its command line.

In addition to the ``public'' environment variables
(\autoref{sec:tools:env}), some other, reserved for developers, alter the
behavior of the programs.

\begin{envs}
\item[URBI\_ACCEPT\_BINARY\_MISMATCH] As a safety net, Urbi checks that
  loaded modules were compiled with exactly the same version of Urbi SDK.
  Define this variable to skip this check, at your own risks.

\item[URBI\_CHECK\_MODE] Skip lines in input that look like \urbi output.  A
  way to accept test files (\file{*.chk}) as input.

\item[URBI\_DESUGAR] Display the desugared ASTs instead of the
  original one.

\item[URBI\_DOC] Where to find the file{doc} directory, which contains
  \file{THANKS.txt} and so forth.

\item[URBI\_IGNORE\_URBI\_U] Ignore failures (such as differences between
  kernel revision and \file{urbi.u} revision) during the initialization.

\item[URBI\_INTERACTIVE] Force the interactive mode, as if
  \option{--interactive} was passed.

\item[URBI\_LAUNCH] The path to \command{urbi-launch} that
  \command{urbi.exe} will exec.

\item[URBI\_NO\_ICE\_CATCHER] Don't try to catch SEGVs.

\item[URBI\_PARSER] Enable Bison parser traces.  Obsolete, use the
  \code{Urbi.Parser} category and \env{GD\_CATEGORY} instead
  (\autoref{sec:tools:env}).

\item[URBI\_REPORT] Display statistics about execution rounds performed by
  the kernel.

\item[URBI\_ROOT\_LIB\var{name}] The location of the libraries to load,
  without the extension.  The \code{LIB\var{name}} are: LIBJPEG4URBI,
  LIBPLUGIN (libuobject plugin), LIBPORT, LIBREMOTE (libuobject remote),
  LIBSCHED, LIBSERIALIZE, LIBURBI.

\item[URBI\_SCANNER] Enable Flex scanner traces.  Obsolete, use the
  \code{Urbi.Scanner} category and \env{GD\_CATEGORY} instead
  (\autoref{sec:tools:env}).

\item[URBI\_SHARE] Where to find the file{share} directory, which contains
  \file{images/gostai-logo}, \file{urbi/urbi.u} and so forth.

\item[URBI\_SHARE] Where to find the file{share} directory, which contains
  \file{images/gostai-logo}, \file{urbi/urbi.u} and so forth.

\item[URBI\_TEXT\_MODE] Forbid binary communications with UObjects.

\item[URBI\_TOPLEVEL] Force the display the result of the top-level
  evaluation into the lobby.
\end{envs}

\section{Check}
\label{sec:build:check}

\paragraph{Root}
Running the test-suite as a super-user (root) is a bad idea
(\autoref{sec:build}): some tests check that \usdk respects file
permissions, which of course cannot work if you are omnipotent.

\paragraph{Parallel Tests}
There are several test suites that will be run if you run \samp{qibuild 
  test} (\samp{-j4} works on most machines).

Some tests are extremely ``touchy''.  Because the test suite exercises \urbi
under extreme conditions, some tests may fail not because of a problem in
\urbi, but because of non-determinism in the test itself.  In this case,
another run of \samp{qibuild test} will give an opportunity for the test to
pass (remind that the tests that passed will not be run again).  Also, using
\samp{qibuild test -j16} is a sure means to have the \urbi scheduler behave
insufficiently well for the test to pass.  \strong{Do not send bug reports
  for such failures.}.  Before reporting bugs, make sure that the failures
remain after a few \samp{qibuild test -j1} invocations.



\section{Debug}
\label{sec:build:debug}

\urbi can be debugged with \command{gdb}.  It is highly recommended to use
\option{--enable-compilation-mode=debug} to configure if you intend to
debug (see \autoref{sec:build:configure}).

When compiled in debug mode, gdb extensions would be installed in the
library directory and inside the share directory.  These extensions are also
available if you run gdb inside the root of the source directory.

These extensions provide pretty printing of some \Cxx objects, such as AST
nodes, \urbi objects, intrusive pointers.  Additional commands are provided
for printing an equivalent of the \us backtrace and for adding and removing
breakpoints set on \us.

To benefit from these extensions, install a recent gdb (higher than 7.2).
Older versions may not be able to load the extensions, or may not provide
the features on which the extensions rely.  To avoid disturbance in case the
extension cause much trouble than benefits, gdb can be run with extra
arguments to disable these extensions.

\begin{shell}
$ gdb -xe 'set auto-load-scripts no' urbi
\end{shell}

The following sections assume that gdb is recent enough.

\subsection{GDB commands}

\begin{description}
\item[urbi stack] Print the backtrace of the current coroutine.  The
  backtrace is indexed by the \Cxx frame numbers and includes frame which
  are manipulating the sources.  Each primitive object is pretty printed
  inside this backtrace.

\begin{verbatim}[language=gdb]
(gdb) b urbi::object::system_backtrace
Breakpoint 1 at 0x7ffff4be229d: file system.cc, line 305.
(gdb) c
Continuing.

//#push 1 "input.u"
function foo (x) { backtrace }|;
for(var i: [1]) foo([i.asString => i]);


Breakpoint 1, urbi::object::system_backtrace () at system.cc:305
305           runner::Job& r = runner();
(gdb) urbi stack
#12 [input.u:1.20-29] Lobby_0x7ffff7f03208.backtrace()
#16 [input.u:1.20-29] Call backtrace
#19 [input.u:1.20-29] Scope
#22 [input.u:2.17-39] Lobby_0x7ffff7f03208.foo(["1" => 1])
#26 [input.u:2.17-39] Call foo
#29 [input.u:2.17-39] Scope
#32 [flower/flower.cc:132.19-25] Scope
#35 [??] Code_0x7ffff7fbdfb8.each(1)
#48 [input.u:2.12-39] [1].each(Code_0x7ffff7fbdfb8)
#52 [input.u:2.12-39] Call each
#55 [input.u:2.1-40] Stmt
\end{verbatim}

\item[urbi break] Set a breakpoint on an \us function.  A breakpoint is
  defined with at least a message and additional optional information such
  as a file, a line, a target and arguments can be used as condition for the
  breakpoint.  Be aware that the target and the arguments are lexically
  compared with the pretty-printed versions.

\begin{verbatim}[language=gdb]
(gdb) urbi break input.u:2 Lobby_0x7ffff7f03208.foo(["2" => 2])
UBreakpoint 1:
        Location: input.u:2
        Lobby_0x7ffff7f03208.foo(["2" => 2])
(gdb) c
Continuing.

//#push 1 "input.u"
function foo (x) { backtrace }|;
for(var i: [1, 2, 3]) foo([i.asString => i]);

[00219506:backtrace] foo (input.u:2.23-44)
[00219506:backtrace] each (input.u:2.12-44)
UBreakpoint 1: [input.u:2.23-45] Lobby_0x7ffff7f03208.foo(["2" => 2])
Inside Job 0x6a2f70
(gdb)
\end{verbatim}


\item[urbi delete] Remove a breakpoint set by \command{urbi-break}.  It
  expects one argument: the \urbi breakpoint number.

\begin{verbatim}[language=gdb]
(gdb) urbi break foo
UBreakpoint 2:
        <any>.foo(<any>)
(gdb) urbi delete 2
UBreakpoint 2:
        Removed
\end{verbatim}

\item[urbi call] Evaluate its arguments as an urbiscript call to eval.
  Live objects can be manipulated easily by using the object name followed
  by its address.

\begin{verbatim}[language=gdb]
(gdb) urbi call Lobby_0x7ffff7f03208.echo(42)
[00219506] *** 42
\end{verbatim}

\item[urbi print] Do the same as \command{urbi call} except that the result
  of the expression is printed at the end of the evaluation.  Objects
  printed like that are likely to be destroyed at the end of the command
  unless the value has been stored in a global variable.

\begin{verbatim}[language=gdb]
(gdb) urbi print { var a = "1"; var b = [1]; [a => b] }
["1" => [1]]
\end{verbatim}

\item[urbi continue] Alias to \command{continue}.
\item[urbi finish] Execute until the end of the current function.
\item[urbi next] Execute until next \urbi function call.
\item[urbi next job] Execute until a function call is executed in another job.
\item[urbi step] Execute next \urbi function call until it return to the
  current function scope or leave the current scope.

\begin{verbatim}[language=gdb]
(gdb) urbi break foo
UBreakpoint 1:
        <any>.foo(<any>)
(gdb) urbi continue

//#push 1 "input.u"
function foo () {
  [1].map(closure (x) { x }); // yield
  [2].map(closure (x) { x }); // yield
  [3].map(closure (x) { x })
}|
/* first */ foo & foo /* second */;

UBreakpoint 1: [input.u:6.1-4] Lobby_0x7ffff35137a8.foo()
Inside Job 0x6a2f70
(gdb) #% Go deeper inside foo
(gdb) urbi next
[input.u:2.3-29] [1].map(Code_0x7ffff3659338)
(gdb) urbi next
[urbi/list.u:129.18-130.18] [1].each|(Code_0x7ffff3658960)
(gdb) #% The current coroutine will hold on the next function call after map
(gdb) urbi finish
UBreakpoint 1: [input.u:6.7-10] Lobby_0x7ffff35137a8.foo()
Inside Job 0x9fab50
(gdb) #% The breakpoint has hit on the second job calling foo.
(gdb) #% Ignore this job, with "urbi continue"
(gdb) urbi continue
[input.u:3.3-29] [2].map(Code_0x7ffff365ca10)
Inside Job 0x6a2f70
(gdb) #% Back into the first job where the finish has ended.
(gdb) #% Step over the map call.
(gdb) urbi step
[input.u:4.3-29] [3].map(Code_0x7ffff36600e8)
(gdb) #% Switch to the next job to be executed.
(gdb) urbi next job
[input.u:4.3-29] [3].map(Code_0x7ffff365d6b8)
Inside Job 0x9fab50
(gdb)
\end{verbatim}

\end{description}


%%% Local Variables:
%%% coding: utf-8
%%% mode: latex
%%% TeX-master: "../urbi-sdk"
%%% ispell-dictionary: "american"
%%% ispell-personal-dictionary: "../urbi.dict"
%%% fill-column: 76
%%% End:
