%% Copyright (C) 2010, Gostai S.A.S.
%%
%% This software is provided "as is" without warranty of any kind,
%% either expressed or implied, including but not limited to the
%% implied warranties of fitness for a particular purpose.
%%
%% See the LICENSE file for more information.

\chapter{Building \usdk}
\label{sec:build}
\setHtmlFileName{build}

\lstnewenvironment{package}[1][]
  {\cxxPre%
    \lstset{language={},
      style=UrbiSDKEnv,
      #1}}
  {\cxxPost}

This section is meant for people who want to \emph{build} the \usdk.  If
you just want to install a pre-built \usdk, see \autoref{sec:installation}.

\section{Requirements}
\label{sec:build:req}

This section lists all the dependencies of this package.  Some of them are
required to \dfn{bootstrap} the package, i.e., to build it from the
repository.  Others are required to \dfn{build} the package, i.e., to
compile it from a tarball, or after a bootstrap.

\subsection{Bootstrap}

To bootstrap this package from its repository, and then to compile it,
the following tools are needed.

\begin{description}
\item[Autoconf 2.64 or later]~\\

\begin{package}
package: autoconf
\end{package}

\item[Automake 1.11.1 or later] Note that if you have to install Automake by
  hand (as opposed to ``with your distribution's system''), you have to tell
  its \command{aclocal} that it should also look at the files from the
  system's aclocal.  If \file{/usr/local/bin/aclocal} is the one you just
  installed, and \file{/usr/bin/aclocal} is the system's one, then run
  something like this:

\begin{shell}
$ dirlist=$(/usr/local/bin/aclocal --print-ac-dir)/dirlist
$ sudo mkdir -p $(dirname $dirlist)
$ sudo /usr/bin/aclocal --print-ac-dir >>$dirlist
\end{shell}

\begin{package}
package: automake
\end{package}

\item[Cvs]
  This surprising requirement comes from the system Bison uses to fetch
  the current version of the message translations.
\begin{package}
package: cvs
\end{package}

\item[Git 1.6 or later]
  Beware that older versions behave poorly with submodules.
\begin{package}
package: git-core
\end{package}

\item[Gettext 1.17]
  Required to bootstrap Bison.  In particular it provides autopoint.
\begin{package}
package: gettext
\end{package}

\item[GNU sha1sum]
  We need the GNU version of sha1sum.
\begin{package}
package: coreutils
\end{package}

\item[Help2man]
Needed by Bison.
\begin{package}
package: help2man
\end{package}

\item[Python] You need \file{xml/sax}, which seems to be part only of
  Python 2.6.  Using \command{python\_select} can be useful to state
  that you want to use Python 2.6 by default (\samp{sudo
    python\_select python26}).
\begin{package}
deb: python2.6
MacPorts: python26
MacPorts: python_select
\end{package}

\item[Texinfo] Needed to compile Bison.
\begin{package}
package: texinfo
\end{package}

\item[yaml for Python] The AST is generated from a description written
  in YAML.  Our (Python) tools need to read these files to generate
  the AST classes.  See \url{http://pyyaml.org/wiki/PyYAML}.  The
  installation procedure on Cygwin is:

\begin{shell}
$ cd /tmp
$ wget http://pyyaml.org/download/pyyaml/PyYAML-3.09.zip
$ unzip PyYAML-3.09.zip
$ cd pyYAML-3.09
$ python setup.py install
\end{shell}

\begin{package}
MacPorts: py26-yaml
deb:   python-yaml
\end{package}

At least on OS X you also need to specify the PYTHONPATH:

\begin{shell}
export PYTHONPATH="\
/opt/local/Library/Frameworks/Python.framework/Versions/2.6\
/lib/python2.6/site-packages:$PYTHONPATH"
\end{shell}
\end{description}

\subsection{Build}

\begin{description}
\item[bc]
  Needed by the test suite.
\begin{package}
package: bc
\end{package}

\item[Boost 1.38 or later] Don't try to build it yourself, ask your
  distribution's management to install it.
\begin{package}
deb: libboost-dev
MacPorts: boost
windows: http://www.boostpro.com/download
\end{package}

\item[Ccache]
  Not a requirement, but it's better to have it.
\begin{package}
package: ccache
\end{package}

\item[Flex 2.5.35 or later]
  Beware that 2.5.33 has bugs than prevents this package from building
  correctly.
\begin{package}
package: flex
\end{package}

\item[G++ 4.0 or later] GCC 4.2 or later is a better option.  Beware that we
  have a problem with GCC-4.4 which rejects some Bison generated code.  See
  \autoref{sec:faq:build:uninitialized}.
\begin{package}
deb: g++-4.2
MacPorts: gcc42
\end{package}

\item[Gnuplot]
  Required to generate charts of trajectories for the documentation.
\begin{package}
package: gnuplot
\end{package}

\item[GraphViz] Used to generate some of the figures in the
  documentation.  There is no GraphViz package for Cygwin, so download
  the MSI file from GraphViz' site, and install it.  Then change your
  path to go into its bin directory.

\begin{shell}
wget http://www.graphviz.org/pub/graphviz/stable/windows/graphviz-2.26.msi
PATH=/cygdrive/c/Program\ Files/Graphviz2.26/bin:$PATH
\end{shell}

\begin{package}
package: graphviz
\end{package}

\item[ImageMagick] Used to convert some of the figures in the
  documentation.
\begin{package}
MacPorts: ImageMagick
deb: imagemagick
\end{package}

\item[PDFLaTeX] TeX Live is the most common \TeX{} distribution nowadays.
\begin{package}
deb: texlive-base texlive-latex-extra
MacPorts: texlive texlive_texmf-full
\end{package}

\item[py-docutils] This is not a requirement, but it's better to have
  it.  Used by the test suite.  Unfortunately the name of the package
  varies between distributions.  It provides \command{rst2html}.

\begin{package}
MacPorts: py26-docutils
\end{package}

\item[socat] Needed by the test suite to send messages to an \urbi server.
\begin{package}
package: socat
\end{package}

\item[TeX4HT] Used to generate the HTML documentation.
\begin{package}
deb: tex4ht
MacPorts: texlive_texmf-full
\end{package}

\item[Transfig] Needed to convert some figures for documentation
  (using fig2dev).
\begin{package}
package: transfig
\end{package}

\item[Valgrind] Not needed, but if present, used by the test suite.
\begin{package}
package: valgrind
\end{package}

\item[xsltproc] This is not a requirement, but it's better to have it.
\end{description}

\section{Check out}

Get the open source version tarball from
\url{http://www.urbiforge.com/index.php/Main/Downloads} and uncompress it.
With this version, the bootstrap step can be skipped.

\section{Bootstrap}
Run

\begin{shell}
$ ./bootstrap
\end{shell}

\section{Configure}
You should compile in another directory than the one containing the sources.

\begin{shell}
$ mkdir _build
$ cd _build
$ ../configure OPTIONS...
\end{shell}

\subsection{configuration options}
See \samp{../configure --help} for detailed information.  Unless you
want to do funky stuff, you probably need no option.

To use ccache, pass \samp{CC='ccache gcc' CXX='ccache g++'} as
arguments to configure:

\begin{shell}
$ ../configure CC='ccache gcc' CXX='ccache g++' ...
\end{shell}

\subsection{Windows: Cygwin}
The builds for windows use our wrappers.  These wrappers use a
database to store the dependencies (actually, to speed up their
computation).  We use Perl, and its DBI module.  So be sort to have
sqlite, and DBI.

\begin{shell}
$ perl -MCPAN -e 'install Bundle::DBI'
\end{shell}

It might fail.  The most important part is

\begin{shell}
$ perl -MCPAN -e 'install DBD::SQLite'
\end{shell}

It might suffice, I don't know...

\subsection{building For Windows}

We support two builds: using Wine on top of Unix, and using Cygwin on
top of Windows.

Both builds use our wrappers around MSVC's cl.exe and link.exe.  It
is still unclear whether it was a good or a bad idea, but the wrappers
use the same names.  Yet libtool will also need to use the genuine
link.exe.  So to set up Libtool properly, you will need to pass the
following as argument to configure:

\begin{shell}
$ AR=lib.exe                                                    \
  CC='ccache cl.exe'                                            \
  CC_FOR_BUILD=gcc                                              \
  CXX='ccache cl.exe'                                           \
  LD=link.exe                                                   \
  DUMPBIN='/cygdrive/c/vcxx8/VC/bin/link.exe -dump -symbols'    \
  RANLIB=:                                                      \
  host_alias=mingw32                                            \
  --host=mingw32
\end{shell}

where:
\begin{itemize}
\item \file{lib.exe}, \file{cl.exe}, \file{link.exe} are the wrappers
\item \file{/cygdrive/c/vcxx8/VC/bin/link.exe} is the genuine MSVC
  tool.
\end{itemize}

Since we are cross-compiling, we also need to specify
\env{CC\_FOR\_BUILD} so that \command{config.guess} can properly guess
the type of the build machine.

\subsection{Building for Windows using Cygwin}

We use our cl.exe wrappers, which is something that Libtool cannot
know.  So we need to tell it that we are on Windows with Cygwin, and
pretend we use GCC, so we pretend we run mingw.

The following options have been used with success to compile \usdk
with Visual C++ 2005.  Adjust to your own case (in particular the
location of Boost).

\begin{shell}
$ ../configure                                                  \
  -C                                                            \
  --prefix=/usr/local/gostai                                    \
  --enable-compilation-mode=debug                               \
  --enable-shared                                               \
  --disable-static                                              \
  --enable-dependency-tracking                                  \
  --with-boost=/cygdrive/c/gt-win32-2/d/boost_1_39              \
  AR=lib.exe                                                    \
  CC='ccache cl.exe'                                            \
  CC_FOR_BUILD=gcc                                              \
  CXX='ccache cl.exe'                                           \
  LD=link.exe                                                   \
  DUMPBIN='/cygdrive/c/vcxx8/VC/bin/link.exe -dump -symbols'    \
  RANLIB=:                                                      \
  host_alias=mingw32                                            \
  --host=mingw32
\end{shell}

%%% FIXME: Obsolete I think.
%%%
%%% For some reason, the building of \file{libuobject.dll} will fail.
%%% You need to remove the reference to \file{libltld.la} from the
%%% libtool invocation.  This will be simplified later.

\section{Compile}
Should be straightforward.

\begin{shell}
$ make -j3
\end{shell}

Using \command{distcc} and \command{ccache} is recommended.

\section{Run}

There are some special variables in addition to the environment
variables documented in the manual.

\begin{envs}
\item[URBI\_CONSOLE\_MODE] Skip lines in input that look like shell
  output.  A way to accept *.chk as input.

\item[URBI\_DESUGAR] Display the desugared ASTs instead of the
  original one.

\item[URBI\_IGNORE\_URBI\_U] Ignore failures (such as differences between
  kernel revision and \file{urbi.u} revision) during the initialization.

\item[URBI\_LAUNCH] The path to \command{urbi-launch} that
  \command{urbi.exe} will exec.

\item[URBI\_NO\_ICE\_CATCHER] Don't try to catch SEGVs.

\item[URBI\_PARSER] Enable Bison parser traces.

\item[URBI\_REPORT] Display stats about execution rounds performed by
  the kernel.

\item[URBI\_ROOT\_LIB\var{name}] The location of the libraries to
  load, without the extension.  The \code{LIB\var{name}} are: LIBJPEG,
  LIBPLUGIN (libuobject plugin), LIBPORT, LIBREMOTE (libuobject
  remote), LIBSCHED, LIBSERIALIZE, LIBURBI.

\item[URBI\_SCANNER] Enable Flex scanner traces.

\item[URBI\_TOPLEVEL] Force the display the result of the top-level
  evaluation into the lobby.
\end{envs}

\section{Check}
\label{sec:build:check}
There are several test suites that will be run if you run \samp{make
  check} (\samp{-j4} works on most machines).

To rerun only the tests that failed, use \samp{make recheck}.  Some tests
have explicit dependencies, and they are not rerun if nothing was changed
(unless they had failed the previous time).  It is therefore expected that
after one (long) run of \samp{make check}, the second one will be
``instantaneous'': the previous log files are reused.

Note that running \samp{make} in the top-level is needed when you change
something deep in the package.  If you forget this \samp{make} run, the
timestamps on which the test suite depends will not be updated, and
therefore the results of the previous state of the package will be used,
instead of your fresh changes.

Some tests are extremely ``touchy''.  Because the test suite exercises \urbi
under extreme conditions, some tests may fail not because of a problem in
\urbi, but because of nondeterminism in the test itself.  In this case,
another run of \samp{make check} will give an opportunity for the test to
pass (remind that the tests that passed will not be run again).  Also, using
\samp{make check -j16} is a sure means to have the \urbi scheduler behave
insufficiently well for the test to pass.  \strong{Do not send bug reports
  for such failures.}.  Before reporting bugs, make sure that the failures
remain after a few \samp{make check -j1} invocations.


\subsection{Lazy test suites}
The test suites are declared as ``lazy'', i.e., unless its
dependencies changed, a successful test will be run only once ---
failing tests do not ``cache'' their results.  Because spelling out
the dependencies is painful, we rely on a few timestamps:

\begin{files}
\item[libraries.stamp] Updated when a library is updated (libport,
  libuobject, etc.).

\item[executables.stamp] Updated when an executable is updated
  (\command{urbi-launch}, etc.).  Depends on libraries.stamp.

\item[urbi.stamp] When \urbi sources (\file{share/urbi/*.u}) are updated.

\item[all.stamp] Updated when any of the three aforementioned
  timestamps is.
\end{files}

These timestamps are updated \strong{only} when \command{make} is run
in the top-level.  Therefore, the following sequence does not work as
expected:

\begin{shell}
$ make check -C tests     # All passes.
$ emacs share/urbi/foo.u
$ make check -C tests     # All passes again.
\end{shell}

\noindent
because the timestamps were not given a chance to notice that some
\urbi source changed, so Make did not notice the tests really needed to
be rerun and \strong{the tests were not run}.

You may either just update the timestamps:

\begin{shell}
$ make check -C tests     # All passes.
$ emacs share/urbi/foo.u
$ make stamps             # Update the timestamps.
$ make check -C tests     # All passes again.
\end{shell}

or completely disable the test suite laziness:

\begin{shell}
$ make check -C tests LAZY_TEST_SUITE=
\end{shell}

\subsection{Partial test suite runs}
You can run each test suite individually by hand as follows:
\begin{files}
\item[sdk-remote/libport/test-suite.log] Tests libport.
\begin{shell}
$ make check -C sdk-remote/libport
\end{shell}

\item[sdk-remote/src/tests/test-suite.log]
Checks liburbi, and some of the executables we ship.  Requires the
kernel to be compiled in order to be able to test some of the uobjects.

\begin{shell}
$ make check -C sdk-remote/src/tests
\end{shell}

\item[tests/test-suite.log]
Tests the kernel and uobjects.
\begin{shell}
$ make check -C tests
\end{shell}

\noindent
Partial runs can be invoked:

\begin{shell}
$ make check -C tests TESTS='2.x/echo.chk'
\end{shell}

\noindent
wildcards are supported:

\begin{shell}
$ make check -C tests TESTS='2.x/*'
\end{shell}

\noindent
To check remote uobjects tests:

\begin{shell}
$ make check -C tests TESTS='uob/remote/*'
\end{shell}

\item[doc/tests/test-suite.log] The snippets of code displayed in the
  documentation are transformed into test files.
\begin{shell}
$ make check -C doc
\end{shell}
\end{files}

%%% Local Variables:
%%% mode: latex
%%% TeX-master: "../urbi-sdk"
%%% ispell-dictionary: "american"
%%% ispell-personal-dictionary: "../urbi.dict"
%%% fill-column: 76
%%% End:
