%% Copyright (C) 2009-2011, Gostai S.A.S.
%%
%% This software is provided "as is" without warranty of any kind,
%% either expressed or implied, including but not limited to the
%% implied warranties of fitness for a particular purpose.
%%
%% See the LICENSE file for more information.

\chapter{Migration from urbiscript 1 to urbiscript 2}
\label{sec:k1}

This chapter is intended to people who want to migrate programs in \us
1 to \us 2.  Backward compatibility is \emph{mostly} ensured, but some
\us 1 constructs were removed because they prevented the introduction
of cleaner constructs in \us 2.  When possible, \us 2 supports the
remaining \us 1 constructs.  The \refObject{Kernel1} object contains
functions that support some \us 1 features.

% \section{One to one translation}

\section{\$(Foo)}
\label{sec:k1:dollar}

This construct was designed to build identifiers at run-time.  This
used to be a common idiom to work around some limitations of \us 1
which are typically \emph{no longer needed in \us 2}.  For instance,
genuine local variables are simpler and safer to use than identifiers
forged by hand to be unique.  In order to associate information to a
string, use a \refObject{Dictionary}.

If you really need to forge identifiers at run-time, use
\lstinline{setSlot}, \lstinline{updateSlot}, and \lstinline{getSlot},
which all work with strings, and possibly \lstinline{asString}, which
converts arbitrary expressions into strings.  The
following table lists common patterns.

\begin{center}
  \begin{tabular}{|l|l|}
    \hline
    \textbf{Deprecated} & \textbf{Updated}  \\
    \hline
    \lstinline|var $(Foo) = ...;| & \lstinline|setSlot(Foo.asString, ...);|   \\
    \lstinline|$(Foo) = ...;|     & \lstinline|updateSlot(Foo.asString, ...);|\\
    \lstinline|$(Foo)|            & \lstinline|getSlot(Foo.asString);|\\
    \hline
  \end{tabular}%$
\end{center}

\section{delete Foo}
\label{sec:k1:delete}
In order to maintain an analogy with the \Cxx language, \us used to
support \lstinline{delete Foo}, but this was removed for a number of
reasons:
\begin{itemize}
\item \us 2 features genuine local variables for which
  \lstinline{delete} makes no sense.
\item in \Cxx \lstinline{delete} really targets the object: destroy
  yourself, then the system will reclaim the memory.  In \us one cannot
  destroy an object and reclaim the memory, it is the task of the system to
  notice objects that are no longer used, and to reclaim the memory.  This
  is called \dfn{garbage collection}.  Therefore in \us \lstinline{delete}
  is actually bounced to \refSlot[Object]{removeLocalSlot} sent to the owner
  of the object.
\item \lstinline{delete} is an unsafe feature that makes only sense in
  pointer-based languages such as \langC and \Cxx.  It enables nice bugs
  such as:
\begin{urbiunchecked}
var this.a := A.new;
// ...
delete this.a;
// ...
cout << this.a;
\end{urbiunchecked}
\end{itemize}

For these reasons, and others, \lstinline{delete Foo} was removed.  To
remove the \emph{name} Foo, run {\lstinline{removeLocalSlot("Foo")}}
(\autoref{sec:tut:objects}) --- the garbage collector will reclaim memory if
there are no other use of \lstinline{Foo}.  To remove the contents of
\lstinline{Foo}, you remove all its slots one by one:

\begin{urbiscript}[firstnumber=1]
class Foo
{
  var a = 12;
  var b = 23;
}|;
function Object.removeAllSlots()
{
  for (var s: localSlotNames)
    removeLocalSlot(s);
}|;
Foo.removeAllSlots;
Foo.localSlotNames;
[00000000] []
\end{urbiscript}

\section{emit Foo}
\label{sec:k1:emit}
The keyword \lstinline{emit} is deprecated in favor of \lstinline{!}.

\begin{center}
  \begin{tabular}{|l|l|}
    \hline
    \textbf{Deprecated} & \textbf{Updated}  \\
    \hline
    \lstinline|emit e;|         & \lstinline|e!;|          \\
    \lstinline|emit e(a);|      & \lstinline|e!(a);|       \\
    \lstinline|emit e ~ 1s;|    & \lstinline|e! ~ 1s;|     \\
    \lstinline|emit e(a) ~ 1s;| & \lstinline|e!(a) ~ 1s;|  \\
    \hline
  \end{tabular}
\end{center}

The \lstinline{?} construct is changed for symmetry.

\begin{center}
  \begin{tabular}{|l|l|}
    \hline
    \textbf{Deprecated} & \textbf{Updated}  \\
    \hline
    \lstinline|at (?e)|                  & \lstinline|at (e?)|\\
    \lstinline|at (?e(var a))|           & \lstinline|at (e?(var a))|\\
    \lstinline|at (?e(var a) if a == 2)| & \lstinline|at (e?(var a) if a == 2)|\\
    \hline
  \end{tabular}
\end{center}


\section{eval(Foo)}
\lstinline{eval} is still supported, but its use is discouraged: one
can often easily do without.  For instance, \lstinline{eval} was often
used to manipulate forged identifiers; see \autoref{sec:k1:dollar}
for means of getting rid of them.

\section{foreach}
The same feature with a slightly different syntax is now provided by
\lstinline|for|.  See \autoref{sec:lang:foreach}.

\section{group}
Where support for groups was a built-in feature in \us 1, it is now
provided by the standard library, see \refObject{Group}.  Instead of
\begin{urbiunchecked}
group myGroup {a, b, c}
\end{urbiunchecked}
\noindent
write
\begin{urbiunchecked}
var myGroup = Group.new(a,b,c)
\end{urbiunchecked}

\section{loopn}
\label{sec:k1:loopn}
The same feature and syntax is now provided by \lstinline|for|.  See
\autoref{sec:lang:forn}.

\section{new Foo}

See \autoref{sec:tut:ctor} for details on \lstinline{new}.  The construct
\lstinline{new Foo} is no longer supported because it is (too) ambiguous:
what does \lstinline{new a(1,2).b(3,4)} mean?  Is
\lstinline{a(1,2).b} the object to clone and \lstinline{(3,4)} are the
arguments of the constructor?  Or is it the result of
\lstinline{a(1,2).b(3,4)} that must be cloned?

In temporary versions, \us 2 used to support this \lstinline{new} construct,
but too many users got it wrong, and we decided to keep only the
simpler, safer, and more consistent method-call-style construct:
\lstinline{Foo.new}.  Every single possible interpretation of
\lstinline{new a(1,2).b(3,4)} is reported below, unambiguously.
\begin{itemize}
\item \lstinline{a(1,2).b(3,4).new}
\item \lstinline{a(1,2).b.new(3,4)}
\item \lstinline{a(1,2).new.b(3,4)}
\item \lstinline{a.new(1,2).b(3,4)}
\item \lstinline{new.a(1,2).b(3,4)}
\end{itemize}

\section{self}
For consistency with the \Cxx syntax, \us now uses \lstinline{this}.

\section{stop Foo}

Use \lstinline{Foo.stop} instead, see \refObject{Tag}.

\section{\# line}

Use \lstinline{//#line} instead, see \autoref{sec:specs:synclines}.

\section{tag+end}

To detect the end of a statement, instead of

\begin{urbiunchecked}
mytag+end: { echo ("foo") },
\end{urbiunchecked}

\noindent
use the \lstinline{leave?} method of the \refObject{Tag} object:

\begin{urbiscript}
{
  var mytag = Tag.new("mytag");
  at (mytag.leave?)
    Channel.new("mytag") << "code has finished";
  mytag: { echo ("foo") },
};
[00000002] *** foo
[00000003:mytag] "code has finished"
\end{urbiscript}



%%% Local Variables:
%%% coding: utf-8
%%% mode: latex
%%% TeX-master: "urbi-sdk"
%%% ispell-dictionary: "american"
%%% ispell-personal-dictionary: "../urbi.dict"
%%% fill-column: 76
%%% End:
