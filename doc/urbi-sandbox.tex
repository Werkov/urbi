\documentclass[openright,twoside,11pt]{book}
  \usepackage{gostai-documentation}
  \usepackage{rotating}


\title{Urbi Documentation Sandbox}
\titleImage{figs/urbi}
\subtitle{Version \VcsDescription}
\author{Gostai}

\begin{document}

\maketitle

This document is meant to be used to toy with \LaTeX{}.

\tableofcontents
\part{First Part}
\chapter{First Chapter}

This is the first chapter.

\newcommand{\tilt}[1]{\multicolumn{1}{c}{\begin{turn}{60}\textbf{#1}\end{turn}}}
\newcommand{\operatorhead}{
  \tilt{Operator}
  & \tilt{Use}
  & \tilt{Associativity}
  & \tilt{Original semantics}
  & \tilt{Equivalence}
}


\newcommand{\operator}[6][ ]{\lstinline@#2@&\lstinline@#3@&#4&#5&\lstinline@#6@#1\\}

\newcommand{\operatordot}    {\operator  {.}    {a.b}              {-}     {Message sending}          {Not redefinable}       }
\newcommand{\operatordota}   {\operator  {.}    {a.b(args)}        {-}     {Message sending}          {Not redefinable}       }


\begin{tabular}{|c|c|c|c|c|c|}
  \hline
  \operatorhead
  \\
  \hline
  \operatordot
  \operatordota
  \hline
\end{tabular}

\section{First Section}
\newcommand{\optionHelp}
  {Display the help message and exit successfully.}

\newcommand{\optionVersion}
  {Display version information and exit successfully.}

\begin{options}
\item[h]{help} \optionHelp
  \begin{sublist}
    \begin{description}
    \item[foo] foo.
    \item[bar] bar.
    \end{description}
  \end{sublist}

  \begin{sublist}
    \begin{itemize}
    \item foo
    \item bar
    \end{itemize}
  \end{sublist}

\item{version} \optionVersion
  \begin{sublist}
    \begin{description}
    \item[foo] foo.
    \item[bar] bar.
    \end{description}

    \begin{itemize}
    \item foo
    \item bar
    \end{itemize}
  \end{sublist}
\end{options}

\subsection{First Subsection}
\subsubsection{Subsubsection I.1.1.1.a}
Body of Subsection I.1.1.1.a.

\begin{urbiassert}[firstnumber=1]
1 == 1;
\end{urbiassert}

\subsubsection{Subsubsection I.1.1.1.b}
Body of Subsection I.1.1.1.b.

\begin{urbiassert}[firstnumber=1]
1 == 1;
\end{urbiassert}

\subsection{Second Subsection}
\begin{urbiscript}
0 != 1;
[00001234] true
// Check the syntax highlighting colors.
if (1 == 0) "yes" else "no";
[00001234] "no"
\end{urbiscript}

\section{Second Section}
\begin{urbiunchecked}[escapeinside=<>]
#<Été>;
[00048238:error] !!! invalid character: `#'
[00048239:error] !!! invalid character: `\xc3'
[00048239:error] !!! invalid character: `\x89'
[00048239:error] !!! invalid character: `\xc3'
[00048239:error] !!! invalid character: `\xa9'
\end{urbiunchecked}

\chapter{Second Chapter}

This is the second chapter.

\urbitrajectory{smooth}

\urbitrajectory{accel}

\begin{urbiscriptapi}
\item[count] Return the count.
\item[launch]
  Fire \lstinline|this|.
\item \lstinline|launch|~\\
  Fire \lstinline|this|.

\end{urbiscriptapi}

\part{Second Part}

This is the second part.

\chapter{Chapter II.1}
Body of Chapter II.1.

\ifx\ifHtml\undefined\else
  \let\subsubsectionSave\subsubsection
  \let\subsubsection\faqsection
\fi

\section{Section II.1.1}
Body of Section II.1.1.

\subsection{Subsection II.1.1.1}
Body of Subsection II.1.1.1.

\subsubsection{FAQ II.1.1.1.a}
Body of Subsection II.1.1.1.a.

\subsubsection{FAQ II.1.1.1.b}
Body of Subsection II.1.1.1.b.

\subsection{Subsection II.1.1.2}
Body of Subsection II.1.1.2.

\subsubsection{FAQ II.1.1.2.a}
Body of Subsection II.1.1.2.a.

\subsubsection{FAQ II.1.1.2.b}
Body of Subsection II.1.1.2.b.


\ifx\ifHtml\undefined\else
  \let\subsubsection\subsubsectionSave
\fi

\section{Section II.1.2}
Body of Section II.1.2.

\chapter{Chapter II.2}
Body of Chapter II.2.


\end{document}

%%% Local Variables:
%%% mode: latex
%%% coding: utf-8
%%% TeX-master: t
%%% ispell-dictionary: "american"
%%% ispell-personal-dictionary: "urbi.dict"
%%% End:
