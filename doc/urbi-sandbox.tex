%% Copyright (C) 2010, 2011, Gostai S.A.S.
%%
%% This software is provided "as is" without warranty of any kind,
%% either expressed or implied, including but not limited to the
%% implied warranties of fitness for a particular purpose.
%%
%% See the LICENSE file for more information.

\documentclass[openright,twoside,11pt]{book}
%  \usepackage[svgnames]{xcolor}  \usepackage{experiment}
  \usepackage{urbi-sdk}
  \usepackage{lipsum}

\title{Urbi Documentation Sandbox}
\subtitle{Version \VcsDescription}
\author{Gostai}

\abstract{This document is not to be published, its point is to toy with
  \LaTeX{}.  \lipsum[1]}


\begin{document}

\maketitle

\tableofcontents
\part{First Part}

\chapter{First Chapter}

This is the first chapter\footnote{Footnotes are ugly.}.

\grammar{class-exp, lvalue-end, block-exp}

\begin{verbatim}
for (var i: 10)
  echo(i);
\end{verbatim}

\begin{urbiunchecked}
'if' + 'else' + 'bitand';
'if'+'else'+'bitand';
'if'.'else'.'bitand';
.'else';

x xor_eq a;

a.'bitand'(b);
a' "foo";
a'' "foo";
a''' "foo";
"else";

"foo";
[00048238] "foo"
\end{urbiunchecked}

\grammar{}

\begin{verbatim}
for (var i: 10)
  echo(i);
\end{verbatim}

\begin{verbatim}[language=gdb]
(gdb) urbi break input.u:2 Lobby_0x7ffff7f03208.foo(["2" => 2])
UBreakpoint 1:
        Location: input.u:2
        Lobby_0x7ffff7f03208.foo(["2" => 2])
(gdb) c
Continuing.

//#push 1 "input.u"
function foo (x) { backtrace }|;
for(var i: [1, 2, 3]) foo([i.asString => i]);

[00219506:backtrace] foo (input.u:2.23-44)
[00219506:backtrace] each (input.u:2.12-44)
UBreakpoint 1: [input.u:2.23-45] Lobby_0x7ffff7f03208.foo(["2" => 2])
Inside Job 0x6a2f70
(gdb) urbi stack
#12 [input.u:1.20-29] Lobby_0x7ffff7f03208.backtrace()
#16 [input.u:1.20-29] Call backtrace
#55 [input.u:2.1-40] Stmt
(gdb) #% comment
\end{verbatim}


\begin{urbiassert}
1+2 == 3;
\end{urbiassert}


\section{\usdk 2.5}
\release{2.5}{2010-11-xx}

\subsection{New Features}
\begin{itemize}
\item More types of empty statements are warned about.  For instance \urbi
  used to accept silently \lstinline|if (foo);|.  It now warns about the empty
  body, and recommends writing \lstinline|if (foo) {};|.
\end{itemize}

\section{\usdk 2.4}
\release{2.4}{2010-11-xx}

\subsection{New Features}
\begin{itemize}
\item More types of empty statements are warned about.  For instance \urbi
  used to accept silently \lstinline|if (foo);|.  It now warns about the empty
  body, and recommends writing \lstinline|if (foo) {};|.
\end{itemize}

\section{\usdk 2.3}
\release{2.3}{2010-11-xx}

\subsection{New Features}
\begin{itemize}
\item More types of empty statements are warned about.  For instance \urbi
  used to accept silently \lstinline|if (foo);|.  It now warns about the empty
  body, and recommends writing \lstinline|if (foo) {};|.
\end{itemize}


\sectionObject{Bar}

\begin{urbiscriptapi}
\item[bar] This is Bar.bar.
\item[foo](<x>) This is Bar.foo with arguments.
\item[set](<key>, <value>) Map \var{key} to \var{value} and return \this so
  that invocations to \lstinline|set| can be chained.  The possibly existing
  previous mapping is overridden.
\item \lstinline|I do it myself|
\item beware (not to catch the first word)
\item|slot[foo][bar]| a slot
\item|fun[foo][bar]|(<var>) a function
\end{urbiscriptapi}

\sectionObject{Foo}

\begin{urbiscriptapi}
\item[bar] This is Foo.bar.
\item[foo] This is Foo.foo.
\end{urbiscriptapi}

\section{Second Section}

\begin{urbiunchecked}[escapeinside=<>]
#<Été>;
[00048238:error] !!! invalid character: `#'
[00048239:error] !!! invalid character: `\xc3'
[00048239:error] !!! invalid character: `\x89'
[00048239:error] !!! invalid character: `\xc3'
[00048239:error] !!! invalid character: `\xa9'
\end{urbiunchecked}

\chapter{Second Chapter}

This is the second chapter.

\urbitrajectory{smooth}

\urbitrajectory{accel}

\begin{urbiscriptapi}
\item[count] Return the count.
\item[launch]
  Fire \this.
\item \lstinline|launch|~\\
  Fire \this.
\end{urbiscriptapi}

\let\sectionOrig\section
\renewcommand{\section}[1]{\clearpage\sectionObject{#1}}
\section{Pair}

A \dfn{pair} is a container storing two objects, similar in spirit to
\lstinline|std::pair| in \Cxx.


\begin{itemize}
\item \lstinline|asString|\\
  Generate the string \samp{(\var{first}, \var{second})} using
  \code{asPrintable} to convert members to strings.

\item \lstinline|first|\\
  Return the first member of the pair.

\item \lstinline|init(\var{first}, \var{second})|~\\
  Instantiate a new \lstinline|Pair| containing \var{first} and
  \var{second}.

\item \lstinline|second|\\
  Return the second member of the pair.

\item \lstinline|'[]'(\var{index})|\\
  Return the \var{index}-th element.  \var{index} must be 0 or 1.

\item \lstinline|'[]='(\var{index}, \var{value})|\\
  Set (and return) the \var{index}-th element to \var{value}.
  \var{index} must be 0 or 1.

\item \lstinline|'<'(\var{other})|\\
  Lexicographic comparison between two pairs.

\item \lstinline|'=='(\var{other})|\\
  Whether \lstinline|this| and \lstinline|other| have the same
  contents (equality-wise).
\end{itemize}

See below for an example.

\begin{urbiscript}
var p = Pair.new(1, 2);
[00000001] (1, 2)
p.first - p.second;
[00000002] -1
p[0] = 10 | p;
[00000002] (10, 2)
assert(Pair.new(0, 0) < Pair.new(0, 1));
assert(Pair.new(0, 0) < Pair.new(1, 0));
assert(Pair.new(0, 1) < Pair.new(1, 0));
assert(Pair.new(1, 2) == Pair.new(1, 2));
assert(!(Pair.new(1, 1) == Pair.new(2, 2)));
\end{urbiscript}


%%% Local Variables:
%%% mode: latex
%%% TeX-master: "../urbi-sdk"
%%% End:

\let\section\sectionOrig


\part{Second Part}
\label{sec:second}
This is the second part.

\chapter{Chapter II.1}

You want to use \refSlot[Foo]{foo}, and possibly \refSlot[Bar]{foo}.
\settocbibname{\label{sec:bib}The Bibliography}
\setindexname{\label{sec:index}The Index}
\bibliographystyle{plain}
\bibliography{comp.lang.urbi}
\nocite{*}

%% Copyright (C) 2010, 2011, 2012, Gostai S.A.S.
%%
%% This software is provided "as is" without warranty of any kind,
%% either expressed or implied, including but not limited to the
%% implied warranties of fitness for a particular purpose.
%%
%% See the LICENSE file for more information.

\chapter{Notations}
\label{sec:notations}

This chapter defines the \dfn{notations} used in this document.

\section{Words}

\begin{itemize}
\item \lstinline|code|\\
  A \dfn{piece of code} (\us, \Java, \Cxx\ldots).

\item \textcmt{comment}\\
  A \dfn{comment} in some programming language.  For instance,
  \lstinline|/* hello /* world */ ! */| is a comment in \us.

\item \env{environment-variable}\\
  An \dfn{environment variable} name, e.g., \env{PATH}.

\item \file{file-name}\\
  A \dfn{file name}.

\item \textkwd{keyword}\\
  A \dfn{keyword} in some programming language.  For instance,
  \lstinline|watch| is an \us keyword.

\item \var{meta-variable}\\
  Depending on the context, a \dfn{variable} name (i.e., an identifier in
  \Cxx or \us), or a \dfn{meta-variable} name.  A meta-variable denotes a
  place where some syntactic construct may be entered.  For instance, in
  \lstinline|while (\var{expression}) \var{statement}|, \var{expression} and
  \var{statement} do not denote two variable names, but two placeholders
  which can be filled with an arbitrary expression, and an arbitrary
  statement.  For instance:

  \lstinline|while (!tasks.empty) { tasks.removeFront.process }|.

\item \textstr{string}\\
  A \dfn{string} in some programming language.  For instance,
  \lstinline|"Hello, world!"| is a string in \us.
\end{itemize}

\section{Frames}

\subsection{\Cxx Code}
\label{sec:notations:cxx}

\Cxx source code is presented in frames as follows.

\begin{cxx}
class Int
{
public:
  Foo(int v = 0)
    : val_(v)
  {}

  void operator(int v)
  {
    std::swap(v, val_);
    return v;
  }

  int operator() const
  {
    return val_;
  }

private:
  int val_;
};
\end{cxx}

%% Using lstlistings inside an \ifthen does not work, make it another file.
\subsection{Grammar Excerpts}
\label{sec:notations:bnf}

The \dfn{grammar fragments} are written in \dfn{EBNF} (\dfn{Extended
  Backus-Naur Form}).  The symbol \lstindex{::=} separates the left-hand
symbol from the right-hand side part of the rule.  Infix \lstinline{|}
denotes alternation, postfix-\lstinline{*} 0-or-more repetition,
postfix-\lstinline{+} 1-or-more repetition, and postfix-\lstinline{?}
denotes optional parts.  Terminal symbols are written in double-quotes, and
non-terminals in angle-brackets.  Parentheses group.

The following frame defines the grammar syntax expressed in the same grammar
syntax.

\begin{bnf}
<grammar> ::= <rule>+

<rule> ::= <symbol> "::=" <rhs>

<rhs> ::= <rhs>*
        | <rhs> "|" <rhs>
        | <rhs> ("?" | "*" | "+")
        | "(" <rhs> ")"
        | <symbol>

<symbol> ::= "<" <identifier> ">"
           | '"' <escaped-character>* '"'
           | "'" <escaped-character>* "'"
\end{bnf}

\ifthen{\boolean{urbisdk}\AND\boolean{urbiThree}}{%
  \autoref{sec:grammar} provides the grammar of \us.}

\subsection{\Java Code}
\label{sec:notations:java}

\Java source code is presented in frames as follows.

\begin{java}
import liburbi.main.*;
public class Main
{
    /// Load urbijava library.
    static
    {
        System.loadLibrary("urbijava");
    }

    public static void main(String argv[])
    {
      // Does nothing for now.
    }
}
\end{java}

\subsection{Shell Sessions}
\label{sec:notations:shell}

Interactive sessions with a (Unix) shell are represented as follows.

\begin{shell}
$ echo toto
toto
\end{shell}

The user entered \samp{echo toto}, and the system answered \samp{toto}.
\samp{\$} is the \dfn{prompt}: it starts the lines where the system invites
the user to enter her commands.

\subsection{\us Sessions}
\label{sec:notations:us}

Interactive sessions with \urbi are represented as follows.

\begin{urbiscript}[firstnumber=1]
echo("toto");
[00000001] *** toto
\end{urbiscript}

Contrary to shell interaction (see \autoref{sec:notations:shell}), there is
no prompt that marks the user-entered lines (here \lstinline|echo("toto");|,
but, on the contrary, answers from the \urbi server start with a label that
includes a timestamp (here \samp{00000001}), and possibly a channel name,
\samp{output} in the following example.

\begin{urbiscript}
cout << "toto";
[00000002:output] "toto"
\end{urbiscript}


\subsection{\us Assertions}
\label{sec:notations:urbiassert}

The following \dfn{assertion frame}:

\begin{urbiassert}
true;
1 < 2;
1 + 2 * 3 == 7;
"foobar"[0, 3] == "foo";
[1, 2, 3].map (function (a) { a * a }) == [1, 4, 9];
[ => ].empty;
\end{urbiassert}

\noindent
denotes the following assertion-block\ifthen{\boolean{urbisdk}}{ (see
  \autoref{sec:lang:assert})} in an \us-session frame:

\begin{urbiscript}
assert
{
  true;
  1 < 2;
  1 + 2 * 3 == 7;
  "foobar"[0, 3] == "foo";
  [1, 2, 3].map (function (a) { a * a }) == [1, 4, 9];
  [ => ].empty;
};
\end{urbiscript}

%%% Local Variables:
%%% coding: utf-8
%%% mode: latex
%%% TeX-master: "urbi-sdk"
%%% ispell-dictionary: "american"
%%% ispell-personal-dictionary: "../../urbi.dict"
%%% fill-column: 76
%%% End:

\chapterIndex
\end{document}

%%% Local Variables:
%%% mode: latex
%%% coding: utf-8
%%% TeX-master: t
%%% ispell-dictionary: "american"
%%% ispell-personal-dictionary: "urbi.dict"
%%% fill-column: 76
%%% End:
