%% Copyright (C) 2010, Gostai S.A.S.
%%
%% This software is provided "as is" without warranty of any kind,
%% either expressed or implied, including but not limited to the
%% implied warranties of fitness for a particular purpose.
%%
%% See the LICENSE file for more information.

\begin{partDescription}{part:uobject}
  {
    %
    This part covers the \urbi architecture: its core components
    (client/server architecture), how its middleware works, how to
    include extensions as UObjects (\Cxx components) and so forth.

    No knowledge of the \us language is needed.  As a matter of fact,
    \urbi can be used as a standalone middleware architecture to
    orchestrate the execution of existing components.

    Yet \us is a feature that ``comes for free'': it is easy using it
    to experiment, prototype, and even program fully-featured
    applications that orchestrate native components.  The interested
    reader should read either the \us user manual
    (\autoref{part:tut}), or the reference manual
    (\autoref{sec:lang}).
    %
  }
%\item[sec:uob:quick] This chapter, self-contained, shows the potential
%  of \urbi used as a middleware.
\item[sec:uob:api] This section shows the various steps of writing an
  \urbi \Cxx component using the UObject API.
\item[sec:uob:uses]
  Interfacing a servomotor device as an example on how to use the
  UObject architecture as a middleware.
\end{partDescription}

%%% Local Variables:
%%% mode: latex
%%% TeX-master: "../urbi-sdk"
%%% ispell-dictionary: "american"
%%% ispell-personal-dictionary: "../urbi.dict"
%%% fill-column: 76
%%% End:
