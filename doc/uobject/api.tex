\chapter{The UObject API}
\label{sec:uob:api}

\section{Creating a class, binding variables and functions}
\label{sec:uob:api:bind}

Let's illustrate those concepts by defining a simple object:
\lstinline{adder}. This object has one variable \lstinline{v}, and a
method \lstinline{add} that returns the sum of this variable and its
argument.

\begin{itemize}
\item First the required include and namespace:

\begin{cxx}
#include <urbi/uobject.hh>
\end{cxx}

\item Then we declare our \lstinline{adder} class:
\begin{cxx}
class adder : public urbi::UObject // Must inherit from UObject.
{
  public:
   // The class must have a single constructor taking a string.
   adder (const std::string&);

   // Our variable.
   urbi::UVar v;

   // Our method.
   double add (double);
};
\end{cxx}
\item The implementation of the constructor and our \lstinline{add}
  method:
\begin{cxx}
// the constructor defines what is available from Urbi
adder::adder (const std::string& s)
  : UObject (s) // required
{
  // Bind the variable.
  UBindVar (adder, v);

  // Bind the function.
  UBindFunction (adder, add);
}

double
adder::add (double rhs)
{
  return v + rhs;
}
\end{cxx}
\item And register this class:
\begin{cxx}
// Register the class to the Urbi kernel.
UStart (adder);
\end{cxx}
\end{itemize}

To summarize:

\begin{itemize}
\item Declare your object class as inheriting from
  \lstinline{urbi::UObject}.
\item Declare a single constructor taking a string, and pass this
  string to the constructor of \lstinline{urbi::UObject}.
\item Declare the variables you want to share with \urbi with the type
  \lstinline{urbi::UVar}.
\item In the constructor, use the macros
  \lstinline|UBindVar(\var{class-name}, \var{variable-name})|
  for each \lstinline{UVar} you want as an instance variable, and
  \lstinline|UBindFunction(\var{class-name}, \var{function-name})| for
  each function you want to bind.
\item Call the macro \lstinline{UStart} for each object.
\end{itemize}

\section{Creating new instances}

When you start an \urbi server, an object of each class registered
with \lstinline{UStart} is created with the same name as the
class. New instances can be created from \urbi using the
\lstinline|new| method. For each instance created in \urbi, a
corresponding instance of the \Cxx object is created. You can get the
arguments passed to the constructor by defining and binding a method
named \lstinline|init| with the appropriate number of arguments.

\section{Notification of a variable change or access}

You can register a function that will be called each time a variable
is modified or accessed (for embedded components only) by calling
\lstinline{UNotifyChange} and \lstinline{UNotifyAccess}, passing
either an \lstinline{UVar} or a variable name as first argument, and a
member function of your \lstinline{UObject} as second argument. This
function can take zero or one argument: a \lstinline{UVar} reference
pointing to the \lstinline{UVar} being accessed or modified. The
\lstinline{notifyChange} callback function is called after the
variable value is changed, whereas the \lstinline{notifyAccess}
callback is called before the variable is accessed, giving you the
possibility to update its value.

\section{Timers}
\label{sec:uob:timers}

The API gives you two methods to have a function called periodically:
\begin{itemize}
\item \lstinline|urbi::UObject::USetUpdate(ufloat \var{period})|\\
  Set up a timer that calls the virtual method
  \lstinline{UObject::update()} with the specified period (in
  milliseconds).  Disable updates if \var{period} is -1.

\item \lstinline|urbi::UObject::USetTimer<T>(ufloat \var{period}), int (T::*\var{fun})()|\\
  Invoke an UObject member function \var{fun} every \var{period}
  milliseconds.  \var{fun} is a regular member-function pointer, for
  instance \lstinline|MyUObject::my_function|.
\end{itemize}

\section{The special case of sensor/effector variables}

In \urbi, a variable can have a different meaning depending on whether
you are reading or writing it: you can use the same variable to
represent the target value of an effector and the current value
measured by an associated sensor. This special mode is activated by
the \lstinline{UObject} defining the variable by calling
\lstinline{UOwned} after calling \lstinline{UBindVar}. This call has
the following effects:
\begin{itemize}
\item When \urbi code or code in other modules read the variable, they
  read the current value.
\item When \urbi code or code in other modules write the variable,
  they set the target value.
\item When the module that called \lstinline|UOwned| reads the
  variable, it reads the target value. When it writes the variable, it
  writes the current value.
\end{itemize}

\section{Using \urbi variables}

You can read or write any \urbi variable by creating an
\lstinline{UVar} passing the variable name to the constructor. Change
the value by writing any compatible type to the \lstinline{UVar}, and
access the value by casting the \lstinline{UVar} to any compatible
type. However, some care must be taken in remote mode: changes on the
variable coming from \urbi code or an other module are only visible if
you have called \lstinline{UNotifyChange} on this variable. Otherwise
the \lstinline{UVar} is not synchronized. Alternatively, in remote
mode, you can get the value on demand by calling
\lstinline|UNotifyOnRequest (\var{variable}, \var{function})|. Then to get an
updated value, call the \lstinline{requestValue} method on the
\lstinline{UVar}, and your callback function will be called as soon as
the value is available. You can read and write all the \urbi
properties of an \lstinline{UVar} by reading and writing the
appropriate \lstinline{UProp} object in the \lstinline{UVar}.

\section{Using binary types}

\urbi can store binary objects of any type in a generic container, and
provides specific structures for sound and images. The generic
containers is called \lstinline{UBinary} and is defined in the
\lstinline{urbi/uobject.hh} header. It contains an enum field type
giving the type of the binary (\lstinline{UNKNOWN}, \lstinline{SOUND}
or \lstinline{IMAGE}), and an union of a \lstinline{USound} and
\lstinline{UImage} struct containing a pointer to the data, the size
of the data and type-specific meta-information. Reading a
\lstinline{UBinary} from a \lstinline{UVar}, and writing a
\lstinline{UBinary} to a \lstinline{UVar} performs a deep-copy of the
data. In plugin mode, you can access directly the buffer used by the
kernel by casting the \lstinline{UVar} to an \lstinline{UImage} or a
\lstinline{USound}. You can then write to the buffer, but you can not
change any other information.

\section{Using hubs to group objects}

Sometimes, you need to perform actions for a group of
\lstinline{UObjects}, for instance devices that need to be updated
together. The API provides the \lstinline{UObjectHub} class for this
purpose. To create a hub, simply declare a subclass of
\lstinline{UObjectHub}, and register it by calling once the macro
\lstinline|UStartHub (\var{class-name})|. A single instance of this class
will then be created upon server start-up. \lstinline{UObject}
instances can then register to this hub by calling
\lstinline|URegister (\var{hub-class-name})|. Timers can be attached to
\lstinline{UObjectHub} the same way as to \lstinline{UObject} (see
\autoref{sec:uob:timers}). The kernel will call the \lstinline{update()}
method of all \lstinline{UObject} before calling the
\lstinline{update()} method of the hub. A hub instance can be
retrieved by calling \lstinline{getUObjectHub (string classname)}. The
hub also holds the list of registered UObject in its members
attribute.

\section{Sending \urbi code}

If you need to send \urbi code to the server, the \lstinline{URBI()}
macro is available, as well as the \lstinline{send()} function. You
can either pass it a string, or directly \urbi code inside a pair of
parentheses:

\begin{urbifixme}
send ("tag:1+1;");

URBI (( at (someevent (x)) { sometag:echo x; }; ));
\end{urbifixme}


%%% Local Variables:
%%% mode: latex
%%% TeX-master: "../urbi-sdk"
%%% End:

% LocalWords:  UObject API UVar UBinary USound UImage liburbi UValue namespace
% LocalWords:  urbi const UBindVar UBindFunction rhs UStart init UNotifyChange
% LocalWords:  UNotifyAccess notifyChange notifyAccess USetUpdate USetTimer pos
% LocalWords:  UOwned UNotifyOnRequest requestValue UProp enum struct plugin px
% LocalWords:  UObjects UObjectHub UStartHub URegister getUObjectHub classname
% LocalWords:  someevent sometag bool getPosition setPosition setPID PID Uowned
% LocalWords:  valueChanged valueAccessed pidChanged headPan ampli getVal vals
% LocalWords:  setPositions servohub addValue realloc sizeof destructor RGB py
% LocalWords:  getWidth getHeight getImage accessFrame softdevice BallTracker
% LocalWords:  varname cameraval newImage detectBall
