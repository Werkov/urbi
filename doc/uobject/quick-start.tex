\newenvironment{todo}{%
  \begin{quote}\itshape%
  }{%
  \end{quote}%
}

\chapter{Quick Start}
\label{sec:uob:quick}

This chapter presents \urbi SDK with a specific focus on its
middleware features.  It is self-contained in order to help readers
quickly grasp the potential of \urbi used as a middleware.  References
to other sections of this document are liberally provided to point the
reader to the more complete documentation; they should be ignored
during the first reading.

\section{\urbi as Middleware}

\urbi SDK is meant to make easier the orchestration of independent,
concurent, components.  It was first designed for robotics: it
provides all the needed features to coordonate the execution of
various components (actuators, sensors, software devices that provide
features such as text-to-speach, face recognition and so forth).
Traditional languages such as \Cxx are definitely very adequate to
program the local, low-level, handling of these hardward or software
devices; indeed one needs efficiency, small memory footprint, and
access to low-level hardware details.  Yet, when it comes to
orchestration and to coordination between components, in a word, when
it comes to \emph{address concurrency}, such languages are no longer
an adequate model.

Rather, one has to select a middleware infrastructure in order to be
able to use remote components as if they were local, to allow
concurrent execution, to make synchronous or asynchronous requests and
so forth.  The \dfn{UObject} architecture provide exactly this: a
common API which allows conformant components to be used seemlessly in
highly concurrent settings.  Components need not be designed with
UObjects in mind, rather, UObjects are typically ``shells'' around
``regular'' components.

As a quite magical feature that comes for free, components with an
UObject interface are naturally supported by the \us programming
language.  This can be a tremendous help: one can interact with these
components (making queries, changing them, observing their state,
monitoring various kinds of events and so forth), which provides a
huge speed-up during development.

Finally, note that, although made with robots in mind, the UObject
architecture is well suited to tame any heavily concurrent
environment, such as video games.

\subsection{The \urbi Architecture}

\begin{todo}
  * Description approche Urbi, philosophie (orchestration etc) => un
  exemple de code Urbiscript qui orchestre 3 UObject de manière simple
  mais non triviale, avec at, \&, events, tags et qui servira de fil
  conducteur au quickstart (le but du quickstart est de refaire cet
  exemple initial).
\end{todo}

\section{UObject Basics}

As a simple running example, consider a (very) basic factory.  Raw
material delivered to the factory is push into some assembly machine,
which takes some time.

As a firth component of this factory, consider the following
implementation of the core engine of the factory.
\newcommand{\factoryDir}{\uobjectsDir/factory/factory.uob}
\lstinputlisting[language=C++]{\factoryDir/factory.hh}
\lstinputlisting[language=C++]{\factoryDir/factory.cc}

\subsection{Wrapping into an UObject}
\begin{todo}
  * Description basique de l'API UObject et de l'approche UObject
  (binding explicite)
\end{todo}

\lstinputlisting[language=C++]{\factoryDir/ufactory.hh}
\lstinputlisting[language=C++]{\factoryDir/ufactory.cc}

\subsection{Asynchronicity}
\begin{todo}
  * Asynchronisme dans UObject: notifychange, timers
\end{todo}

\subsection{Running Components}
\begin{todo}
  * Compilation et branchement de UObject (urbi-launch): expliquer les
  deux modes: distant / plugé

  * Cycle complet de lancement d'une appli Urbi: lancement d'urbi,
  execution d'urbi.ini, chargement dynamique des UObjects
\end{todo}

\section{Using \us}

\subsection{The \us Scripting Language}
\begin{todo}
  * Introduction à Urbiscript (rappeler que c'est avant tout un
  langage comme les autres, avec if/for/while, etc), puis passer
  rapidement aux elements clefs de la prog evenementielle dans Urbi:
  at/whenever
\end{todo}

\subsection{Concurrency}
\begin{todo}
  * Parallelisme explicite: \&

  * Tags et controle d'execution

  * Channels et liburbi (les principes, renvoyer à la doc liburbi pour
  l'API)

  * All together, on reprend l'exemple du début et on détaille le code
  des UObjects utilisés, puis le code urbiscript.
\end{todo}

\section{Conclusion}
\begin{todo}
  * Conclusion note: expliquer qu'on peut embedder l'engine Urbi dans
  une appli native C++ (tout le monde le demande), qu'on peut faire
  des bridges génériques avec d'autres archis à composants (citer
  CORBA), donner le lien pour télécharger, le lien vers la doc du
  langage urbiscript, vers la doc de UObject, etc.
\end{todo}


%%% Local Variables:
%%% mode: latex
%%% TeX-master: "../urbi-sdk"
%%% End:
