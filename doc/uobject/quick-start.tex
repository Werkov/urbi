%% Copyright (C) 2009-2011, Gostai S.A.S.
%%
%% This software is provided "as is" without warranty of any kind,
%% either expressed or implied, including but not limited to the
%% implied warranties of fitness for a particular purpose.
%%
%% See the LICENSE file for more information.

\chapter{Quick Start}
\label{sec:uob:quick}

This chapter presents \usdk with a specific focus on its middleware
features.  It is self-contained in order to help readers quickly grasp the
potential of \urbi used as a middleware.  References to other sections of
this document are liberally provided to point the reader to the more
complete documentation; they should be ignored during the first reading.

\newcommand{\machineDir}{\uobjectsDir/test/machine.uob}
% firstline to skip the license header.
\newcommand{\machineFile}[1]
{\lstinputlisting[language=C++,firstline=11,basicstyle=\ttfamily\footnotesize]{\machineDir/#1}}

\section{UObject Basics}

As a simple running example, consider a (very) basic factory.  Raw material
delivered to the factory is pushed into some assembly machines, which takes
some time.

\subsection{The Objects to Bind into \urbi}

As a firth component of this factory, the core machine of the factory is
implemented as follows.  This class is pure regular \Cxx, it uses no \urbi
feature at all.

The header \file{machine.hh}, which declares \lstinline|Machine|, is
traditional.  The documentation uses the Doxygen syntax.

\machineFile{machine.hh}

The implementation file, \file{machine.cc}, is equally straightforward.

\machineFile{machine.cc}

\subsection{Wrapping into an UObject}

By \dfn{binding} a UObject, we mean using the UObject API to declare objects
to the \urbi world.  These objects have member variables (also known as
\dfn{attributes}) and/or member functions (also known as \dfn{methods}) all
of them or some of them being declared to \urbi.

One could modify the \lstinline|Machine| class to make it a UObject, yet we
recommend wrapping pure \Cxx classes into a different, wrapping, UObject.
It is strongly suggested to aggregate the native \Cxx objects in the UObject
--- rather than trying to derive from it.  By convention, we prepend a ``U''
to the name of the base class, hence the \lstinline|UMachine| class.  This
class provides a simplified interface, basically restricted to what will be
exposed to \urbi.  It must derive from \lstinline|urbi::UObject|.

\machineFile{umachine.hh}

The implementation of \lstinline|UMachine| is simple.  It uses some of the
primitives used in the binding process (\autoref{sec:uob:api:bind}):
\begin{itemize}
\item \lstinline|UStart(\var{class})| declares classes that are UObjects;
  eventually such classes will appear in \us as
  \lstinline|uobjects.\var{class}| (but since \refObject{Global} derives
  from \refObject{uobjects} you may just use the simpler name \var{class}).
  Use it once.

\item \lstinline|UBindFunction(\var{class}, \var{function})| declares a
  \var{function}.  Eventually bound in the \us world as
  \lstinline|uobjects.\var{class}.\var{function}|.

\item Similarly, \lstinline|UBindVar(\var{class}, \var{variable})| declares
  a \var{variable}.
\end{itemize}

\urbi relies on the prototype model for object-oriented programming, which
is somewhat different from the traditional \Cxx class-based model
(\autoref{sec:tut:value}).  This is reflected by the presence of \emph{two}
different constructors:
\begin{itemize}
\item \lstinline|UMachine::UMachine|, the \Cxx constructor invoked for every
  single instance of the UObject.  It is always invoked by the \urbi system
  when instantiating a UObject, \emph{including} the prototype itself.  It
  sole argument is its name (an internal detail you need not be aware of).
  The \Cxx constructor must register the \lstinline|UMachine::init|
  function.  It may also bind functions and variables.

\item \lstinline|UMachine::init|, the \urbi constructor invoked each time a
  new clone of \lstinline|UMachine| is made, i.e., for every instance except
  the first one.

  Functions and variables that do not make sense for the initial prototype
  (which might not be fully functional) should be bound here, rather than in
  the \Cxx constructor.
\end{itemize}

The following listing is abundantly commented, and is easy to grasp.

\machineFile{umachine.cc}

\subsection{Running Components}
\label{sec:uob:quick:run}
%  * Compilation et branchement de UObject (urbi-launch): expliquer les
%  deux modes: distant / pluge'
%
%  * Cycle complet de lancement d'une appli Urbi: lancement d'urbi,
%  execution d'urbi.ini, chargement dynamique des UObjects


As a first benefit from using the \urbi environment, this source code is
already runnable!  No \lstinline|main| function is needed, the \urbi system
provides one.

\subsubsection{Compiling}
Of course, beforehand, we need to compile this UObject into some loadable
module.  The \urbi modules are \dfn{shared objects}, i.e., libraries that
can be loaded on demand (and unloaded) during the execution of the program.
Their typical file names depend on the architecture: \file{machine.so} on
most Unix platforms (including Mac OS X), and \file{machine.dll} on Windows.
To abstract away from these differences, we will simply use the base name,
\file{machine} with the \urbi tool chain.

There are several options to compile our machine as a UObject.  If you are
using Microsoft Visual Studio, the \usdk installer created a ``UObject''
project template accessible through the ``New project'' wizard.  Otherwise,
you can use directly your regular compiler tool chain.  You may also use
\command{umake-shared} from the \file{umake-*} family of programs
(\autoref{sec:tools:umake:wrappers}):

\begin{shell}
$ ls machine.uob
machine.cc  machine.hh  umachine.cc  umachine.hh
$ umake-shared machine.uob -o machine
# ... Lots of compilation log messages ...
$ ls
_ubuild-machine.so  machine.la  machine.so  machine.uob
\end{shell}

The various files are:
\begin{description}
\item[\file{machine.uob}] Merely by convention, the sources of our UObject
  are in a \file{*.uob} directory.
\item[\file{machine.so}] A shared dlopen-module.  This is the ``true''
  product of the \command{umake-shared} invocation.  Its default name can be
  quite complex (\file{uobject-i386-apple-darwin9.7.0.so} on my machine), as
  it will encode information about the architecture of your machine; if you
  don't need such accuracy, use the option \option{--output}/\option{-o} to
  specify the output file name.

  \command{umake-shared} traversed \file{machine.uob} to gather and process
  relevant files (source files, headers, libraries and object files) in
  order to produce this output file.

\item[\file{\_ubuild-machine.so}] this is a temporary directory in which the
  compilation takes place.  It can be safely removed by hand, or using
  \command{umake-deepclean} (\autoref{sec:tools:umake:wrappers}).

\item[\file{machine.la}] a \href{http://www.gnu.org/software/libtool/}{GNU
    Libtool} file that contains information such as dependencies on other
  libraries.  While this file should be useless most of the time, we
  recommend against removing it: it may help understand some problems.
\end{description}

\subsubsection{Running UObjects}

There are several means to toy with this simple UObject.  You can use
\command{urbi-launch} (\autoref{sec:tools:urbi-launch}) to plug the UMachine
in an \urbi server and enter an interactive session.

\begin{shell}[alsolanguage={[interactive]urbiscript}]
# Launch an Urbi server with UMachine plugged in.
$ urbi-launch --start machine -- --interactive
[00000000] *** ********************************************************
[00000000] *** Urbi SDK version 2.7 rev. a6a1ec5
[00000000] *** Copyright (C) 2004-2011 Gostai S.A.S.
[00000000] ***
[00000000] *** This program comes with ABSOLUTELY NO WARRANTY.  It can
[00000000] *** be used under certain conditions.  Type `license;',
[00000000] *** `authors;', or `copyright;' for more information.
[00000000] ***
[00000000] *** Check our community site: http://www.urbiforge.org.
[00000000] *** ********************************************************
var f = UMachine.new(1s);
[00020853] UMachine_0x1899c90
f.assemble(["Hello, ", "World!"]);
[00038705] "Hello, World!"
shutdown;
\end{shell}

You may also launch the machine UObject in the background, as a network
component:

\begin{shell}[alsolanguage={[interactive]urbiscript}]
$ urbi-launch --start machine --host 0.0.0.0 --port 54000 &
\end{shell}

\noindent
and interact with it using your favorite client (\command{telnet},
\command{netcat}, \command{socat}, \ldots), or using the \command{urbi-send}
tool (\autoref{sec:tools:urbi-send}).

\begin{shell}[alsolanguage={[interactive]urbiscript}]
$ urbi-send --port 54000                       \
            -e 'UMachine.assemble([12, 34]);'  \
            --quit
[00126148] "1234"
[00000000:client_error] End of file
$ urbi-send --port 54000                     \
            -e 'var f = UMachine.new(1s)|'   \
            -e 'f.assemble(["ab", "cd"]);'   \
            --quit
[00146159] "abcd"
[00000000:client_error] End of file
\end{shell}


% \subsection{Asynchronicity}
% * Asynchronisme dans UObject: notifychange, timers


\section{Using \us}

% Introduction a` Urbiscript (rappeler que c'est avant tout un langage
% comme les autres, avec if/for/while, etc), puis passer rapidement
% aux elements clefs de la prog evenementielle dans Urbi: at/whenever

\us is a programming language primarily designed for robotics.  Its syntax
is inspired by that of \Cxx: if you know \C, \Cxx, \Java or \Cs, writing \us
programs is easy.  It's a dynamic object-oriented (\autoref{sec:tut:object})
scripting language, which makes it well suited for high-level application.
It supports and emphasizes parallel (\autoref{sec:tut:concurrent}) and
event-based programming (\autoref{sec:tut:event-prog}), which are very
popular paradigms in robotics, by providing core primitives and language
constructs.

Thanks to its client/server approach, one can easily interact with a robot,
to monitor it, or to experiment live changes in the \us programs.

Courtesy of the UObject architecture, \us is fully integrated with \Cxx. As
already seen in the above examples (\autoref{sec:uob:quick:run}), you can
bind \Cxx classes in \us seamlessly. \us is also integrated with many other
languages such as \Java (\autoref{sec:uob:apijava}), \MatLab or \Python.
The UObject framework also naturally provides \us with support for
distributed architectures: objects can run in different processes, possibly
on remote computers.

%% Everything that follows is bullshit: UObject invocations are
%% *not* in different threads, so we are definitely doomed here.
%%
\subsection{The \us Scripting Language}

The following example shows how one can easily interface UObjects into the
\us language.  The following simple class (actually, a genuine object, in
\us ``classes are objects'', see \autoref{sec:tut:object}) aggregates two
assembly machines, a fast one, and a slow one.  This class demonstrates
usual object-oriented, sequential, features.

\begin{urbiscript}[firstnumber=1]
//#plug test/machine
class TwoMachineFactory
{
  // A shorthand common to all the Two Machine factories.
  var UMachine = uobjects.UMachine;

  // Default machines.
  var fastMachine = UMachine.new(10ms);
  var slowMachine = UMachine.new(100ms);

  // The urbiscript constructor.
  // Build two machines, a fast one, and a slow one.
  function init(fast = 10ms, slow = 100ms)
  {
    // Make sure fast <= slow.
    if (slow < fast)
      [fast, slow] = [slow, fast];
    // Two machines for each instance of TwoMachineFactory.
    fastMachine = UMachine.new(fast);
    slowMachine = UMachine.new(slow);
  };

  // Wrappers to make invocation of the machine simpler.
  function fast(input) { fastMachine.assemble(input) };
  function slow(input) { slowMachine.assemble(input) };

  // Use the slow machine for large jobs.
  function assemble(input)
  {
    var res|
    var machine|
    if (5 < input.size)
      { machine = "slow" | res = slow(input); }
    else
      { machine = "fast" | res = fast(input); } |
    echo ("Used the %s machine (%s => %s)" % [machine, input, res]) |
    res
  };
};
[00000001] TwoMachineFactory
\end{urbiscript}

Using this class is straightforward.

\begin{urbiscript}
var f = TwoMachineFactory.new|;
f.assemble([1, 2, 3, 4, 5, 6]);
[00000002] *** Used the slow machine ([1, 2, 3, 4, 5, 6] => 123456)
[00000003] "123456"
f.assemble([1]);
[00000004] *** Used the fast machine ([1] => 1)
[00000005] "1"
\end{urbiscript}

The genuine power of \us is when concurrency comes into play.

\subsection{Concurrency}
% * Parallelisme explicite: \&
%
% * Tags et controle d'execution
%
% * Channels et liburbi (les principes, renvoyer a` la doc liburbi pour
% l'API)
%
% * All together, on reprend l'exemple du de'but et on de'taille le code
% des UObjects utilise's, puis le code urbiscript.

Why should we wait for the slow job to finish if we have a fast machine
available?  To do so, we must stop requesting a \emph{sequential}
composition between both calls.  We did that by using the sequential
operator, \samp{;}.  In \us, there exists its concurrent counter-part:
\samp{,}.  Indeed, running \lstinline|\var{a}, \var{b}| means ``launch the
program \var{a} and then launch the program \var{b}''.  The \us Manual
contains a whole section devoted to explaining these operators
(\autoref{sec:tut:concurrent}).

\subsubsection{First Attempt}
\label{sec:uob:quick:fst}
Let's try it:

\begin{urbiscript}
f.assemble([1, 2, 3, 4, 5, 6]),
f.assemble([1]),
[00000002] *** Used the slow machine ([1, 2, 3, 4, 5, 6] => 123456)
[00000004] *** Used the fast machine ([1] => 1)
\end{urbiscript}

\noindent
This is a complete failure.

Why?

Since \urbi cannot expect that your code is thread-safe, by default all
calls to UObject features are \dfn{synchronous}, or \dfn{blocking}:
\emph{the whole \urbi system is suspended until the function returns}.
There is a single thread of execution for \urbi, and when calling a function
from a (plugged) UObject, that thread of execution is devoted to evaluated
the code.

See for instance below that, in even though \lstinline|f.assemble| is slow
and launched in background, the almost instantaneous display of
\lstinline|ping| is not performed immediately.

\begin{urbiscript}
echo(f.slow([1, 2, 3, 4, 5, 6])),
echo("ping");
[00000002] *** 123456
[00000002] *** ping
\end{urbiscript}

There are several means to address this unintended behavior.  If the base
library provides a threaded API (in our example, the \lstinline|Machine|
class, not the \lstinline|UMachine| UObject wrapper), then you could use
it.  Yet we don't recommend it, as it takes away from \urbi the possibility
to really control concurrency issues (for instance it cannot turn
non-blocking functions into blocking functions).

A better option is to ask \urbi to turn blocking function calls into
non-blocking ones.

\subsubsection{Second Attempt: Threaded Functions}

If you read carefully the body of the \lstinline|UMachine::init| function,
you will find the following piece of code:

\begin{cxx}
UBindFunction(UMachine, assemble);
UBindThreadedFunctionRename
  (UMachine, assemble, "threadedAssemble", urbi::LOCK_FUNCTION);
\end{cxx}

\noindent
Both calls bind the function \lstinline|UMachine::assemble|, but the second
one will run the call in a separate thread.  Since multiple calls to a
single function (or different functions of a single object etc.) are likely
to fail, a locking model must be provided.  Here,
\lstinline|urbi::LOCK_FUNCTION| means that concurrent calls to
\lstinline|UMachine::assemble| must be serialized: one at a time.

To use this threaded version of \lstinline|assemble|, we can simply patch
our \lstinline|TwoMachineFactory| class:

\begin{urbiscript}
do (TwoMachineFactory)
{
  fast = function (input) { fastMachine.threadedAssemble(input) };
  slow = function (input) { slowMachine.threadedAssemble(input) };
}|;
\end{urbiscript}

Let's try again where we failed previously (\autoref{sec:uob:quick:fst}):

\begin{urbiscript}
f.assemble([1, 2, 3, 4, 5, 6]),
f.assemble([1]),
[00000004] *** Used the fast machine ([1] => 1)
[00000002] *** Used the slow machine ([1, 2, 3, 4, 5, 6] => 123456)
sleep(200ms);
\end{urbiscript}

Victory!  The fast machine answered first.

You may have noticed that the result is no longer reported.  Indeed, the \us
interactive shell only displays the result of synchronous expressions (i.e.,
those ending with a \samp{;}): asynchronous answers are confusing (see the
inversion here).

\refObject[s]{Channel} are useful to ``send'' asynchronous answers.

\begin{urbiscript}
var c1 = Channel.new("c1")|;
var c2 = Channel.new("c2")|;
c1 << f.assemble([10, 20, 30, 40, 50, 60]),
c2 << f.assemble([100]),
sleep(200ms);
[00000535] *** Used the fast machine ([100] => 100)
[00000535:c2] "100"
[00000625] *** Used the slow machine ([10, 20, 30, 40, 50, 60] => 102030405060)
[00000625:c1] "102030405060"
\end{urbiscript}

\section{Conclusion}

This section gave only a quick glance to all the power that \urbi provides
to support concurrency.  Actually, it makes plenty of sense to embed an
\urbi engine in a native \Cxx program, and to delegate the concurrency
issues to it.  Thanks to its middleware and client/server architecture, it
is then possible to connect it to remote components of different kinds, such
as using ROS for instance.

Then, a host of features are at hand, ready to be used when you need them:
event-driven programming, automatic monitoring of expressions, interactive
sessions, \ldots and, last but not least, the \us programming language.

% * Conclusion note: expliquer qu'on peut embedder l'engine Urbi dans
% une appli native C++ (tout le monde le demande), qu'on peut faire
% des bridges ge'ne'riques avec d'autres archis a` composants (citer
% CORBA), donner le lien pour te'le'charger, le lien vers la doc du
% langage urbiscript, vers la doc de UObject, etc.



%%% Local Variables:
%%% coding: utf-8
%%% mode: latex
%%% TeX-master: "../urbi-sdk"
%%% ispell-dictionary: "american"
%%% ispell-personal-dictionary: "../urbi.dict"
%%% fill-column: 76
%%% End:
