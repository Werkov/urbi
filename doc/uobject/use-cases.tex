\chapter{Use Cases}
\label{sec:uob:uses}
\section{Writing a Servomotor Device}

Let's write a \lstinline{UObject} for a servomotor device whose
underlying API is:

\begin{cxxapi}
\item[bool initialize (int id)]
  Initialize the servomotor with given ID.
\item[double getPosition (int id)]
  Read servomotor of given id position.
\item[void setPosition (int id, double pos)]
  Send a command to servomotor.
\item[void setPID (int id, int p, int i, int d)]
  Set P, I, and D arguments.
\end{cxxapi}

First our header. Our servo device provides an attribute named
\lstinline{val}, the standard \urbi name, and two ways to set PID
gain: a method, and three variables.

\begin{cxx}
class servo : public urbi::UObject // must inherit UObject
{
public:
  // the class must have a single constructor taking a string
  servo(const std::string&);

  // Urbi  constructor
  int init(int id);

  // main attribute
  urbi::UVar val;

  // position variables:
  //  P gain
  urbi::UVar P;
  //  I gain
  urbi::UVar I;
  //  D gain
  urbi::UVar D;

  // callback for val change
  int valueChanged(UVar& v);
  //callback for val access
  int valueAccessed(UVar& v);
  // callback for PID change
  int pidChanged(UVar& v);

  // method to change all values
  void setPID(int p, int i, int d);

  // motor ID
  int id_;
};
\end{cxx}

The constructor only registers init, so that our default instance
\lstinline{servo} does nothing, and can only be used to create new
instances.

\begin{cxx}
servo::servo (const std::string& s)
  : urbi::UObject (s)
{
   // register init
   UBindFunction (servo, init);
}
\end{cxx}

The \lstinline{init} function, called in a new instance each time a
new \urbi instance is created, registers the four variables
(\lstinline{val}, \lstinline{P}, \lstinline{I} and \lstinline{D}), and
sets up callback functions.

\begin{cxx}
// Urbi constructor.
int
servo::init (int id)
{
  id_ = id;

  if (!initialize (id))
    return 1;

  UBindVar (servo, val);

  // val is both a sensor and an actuator.
  Uowned (val);

  // Set blend mode to mix.
  val.blend = urbi::UMIX;

  // Register variables.
  UBindVar (servo, P);
  UBindVar (servo, I);
  UBindVar (servo, D);

  // Register functions.
  UBindFunction (servo, setPID);

  // Register callbacks on functions.
  UNotifyChange (val, &servo::valueChanged);
  UNotifyAccess (val, &servo::valueAccessed);
  UNotifyChange (P, &servo::pidChanged);
  UNotifyChange (I, &servo::pidChanged);
  UNotifyChange (D, &servo::pidChanged);

  return 0;
}
\end{cxx}

Then we define our callback methods. \lstinline{servo::valueChanged}
will be called each time the \lstinline{val} variable is modified,
just after the value is changed: we use this method to send our servo
commands. \lstinline{servo::valueAccessed} is called just before the
value is going to be read. In this function we request the current
value from the servo, and set \lstinline{val} accordingly.

\begin{cxx}
// Called each time val is written to.
int
servo::valueChanged (urbi::UVar& v)
{
  // v is a reference to our class member val: you can use both
  // indifferently.
  setPosition (id, (double)val);

  // The return value is ignored for now, just return 0.
  return 0;
}

// Called each time val is read.
int
servo::valueAccessed (urbi::UVar& v)
{
  // v is a reference to val.
  val = getPosition (id);

  return 0;
}
\end{cxx}

\lstinline{servo::pidChanged} is called each time one of the PID
variables is written to. The function \lstinline{servo::setPID} can be
called directly from \urbi.

\begin{cxx}
int
servo::pidChanged (urbi::UVar& v)
{
  setPID(id, (int)P, (int)I, (int)D);

  return 0;
}

void
servo::setPID (int p, int i, int d)
{
  setPID (id, p, i, d);
  P = p;
  I = i;
  D = d;
}

// Register servo class to the Urbi kernel.
UStart (servo);
\end{cxx}

That's it, compile this module, and you can use it within \us:

\begin{urbifixme}
// Create a new instance.  Calls init (1).
headPan = new servo (1);

// Calls setPID ().
headPan.setPID (8,2,1);

// Calls valueChanged ().
headPan.val = 13;

// Calls valueAccessed ().
headPan.val * 12;

// Periodically calls valueChanged ().
headPan.val = 0 sin:1s ampli:20,

// Periodically calls valueAccessed ().
at (headPan.val < 0)
  echo ("left");
\end{urbifixme}

The sample code above has one problem: \lstinline{valueAccessed} and
\lstinline{valueChanged} are called each time the value is read or
written from \urbi, which can happen quite often. This is a problem if
sending the actual command (\lstinline{setPosition} in our example)
takes time to execute. There are two solutions to this issue.

\subsection{Caching}

One solution is to remember the last time the value was read/written,
and not apply the new command before a fixed time. Note that the
kernel is doing this automatically for \lstinline{UOwned}'d variables
that are in a blend mode different than \lstinline{normal}. So the
easiest solution to the above problem is likely to set the variable to
the \lstinline{mix} blending mode. The unavoidable drawback is that
commands are not applied immediately, but only after a small delay.

\subsection{Using Timers}

Instead of updating/fetching the value on demand, you can chose to do
it periodically based on a timer. A small difference between the two
API methods comes in handy for this case: the \lstinline{update()}
virtual method called periodically after being set up by
\lstinline{USetUpdate(interval)} is called just after one pass of
\urbi code execution, whereas the timers set up by
\lstinline{USetTimer} are called just before one pass of \urbi code
execution. So the ideal solution is to read your sensors in the second
callback, and write to your actuators in the first. Our previous
example (omitting PID handling for clarity) can be rewritten. The
header becomes:

\begin{cxx}
// Inherit from UObject.
class servo : public urbi::UObject
{
public:
  // The class must have a single constructor taking a string.
  servo (const std::string&)

  // Urbi constructor.
  int init (int id);

  // Called periodically.
  virtual int update ();
  // Called periodically.
  int getVal ();

  // Our position variable.
  urbi::UVar val;

  // Motor ID.
  int id_;
};
\end{cxx}

Constructor is unchanged, \lstinline{init} becomes:

\begin{cxx}
// Urbi constructor.
int
servo::init (int id)
{
  id_ = id;

  if (!initialize (id))
    return 0;

  UBindVar (servo,val);
  // Val is both a sensor and an actuator.
  UOwned(val);

  // Will call update () periodically.
  USetUpdate(1);
  // Idem for getVal ().
  USetTimer (1, &servo::getVal);

  return 0;
}
\end{cxx}

\lstinline{valueChanged} becomes \lstinline{update} and
\lstinline{valueAccessed} becomes \lstinline{getVal}. Instead of being
called on demand, they are now called periodically. The period of the
call cannot be lower than the value returned by
\lstinline{Object.getPeriod;}
so you can set it to 0 to mean ``as fast as is useful''.

\section{Using Hubs to Group Objects}

Now, suppose that, for our previous example, we can speed things up by
sending all the servomotor commands at the same time, using the
following method that takes two arrays of ids and positions.

\begin{cxx}
void setPositions(int count, int* ids, double* positions);
\end{cxx}

A hub is the perfect way to handle this task. The UObject header stays
the same. We add a hub declaration:

\begin{cxx}
class servohub : public urbi::UObjectHub
{
public:
  // The class must have a single constructor taking a string.
  servohub (const std::string&);

  // Called periodically.
  virtual int update ();

  // Called by servo.
  void addValue (int id, double val);

  int* ids;
  double* vals;
  int size;
  int count;
};
\end{cxx}

\lstinline{servo::update} becomes a call to the \lstinline{addValue}
method of the hub:

\begin{cxx}
int
servo::update()
{
  ((servohub*)getUObjectHub ("servohub"))->addValue (id, (double)val);
};
\end{cxx}

The following line can be added to the servo \lstinline{init} method,
although it has no use in our specific example:

\begin{cxx}
URegister(servohub);
\end{cxx}

Finally, the implementation of our hub methods is:

\begin{cxx}
servohub::servohub (const std::string& s)
  : UObjectHub (s)
  , ids   (0)
  , vals  (0)
  , size  (0)
  , count (0)
{
  // setup our timer
  USetUpdate (1);
}

int
servohub::update ()
{
  // Called periodically.
  setPositions (count, ids, vals);

  // Reset position counter.
  count = 0;

  return 0;
}

void
servohub::addValue (int id, double val)
{
  if (count + 1 < size)
  {
    // Allocate more memory.
    ids = (int*) realloc (ids, (count + 1) * sizeof (int));
    vals = (double*) realloc (vals, (count + 1) * sizeof (double));
    size = count + 1;
  }
  ids[count] = id;
  vals[count++] = val;
}

UStartHub (servohub);
\end{cxx}

Periodically, the \lstinline{update} method is called on each servo
instance, which adds commands to the hub arrays, then the
\lstinline{update} method of the hub is called, actually sending the
command and resetting the array.

\subsection{Alternate Implementation}

Alternatively, to demonstrate the use of the members hub variable, we
can entirely remove the \lstinline{update} method in the servo class
(and the \lstinline{USetUpdate()} call in \lstinline{init}), and
rewrite the hub \lstinline{update} method the following way:

\begin{cxx}
int servohub::update()
{
  //called periodically
  for (UObjectList::iterator i = members.begin ();
       i != members.end ();
       ++i)
    addValue (((servo*)*i)->id, (double)((servo*)*i)->val);
  setPositions(count, ids, vals);
  // reset position counter
  count = 0;

  return 0;
}
\end{cxx}

\section{Writing a Camera Device}

A camera device is an UObject whose \lstinline{val} field is a binary
object. The \urbi kernel itself doesn't make any difference between
all the possible binary formats and data type, but the API provides
image-specific structures for convenience. You must be careful about
memory management. The \lstinline{UBinary} structure handles its own
memory: copies are deep, and the destructor frees the associated
buffer. The \lstinline{UImage} and \lstinline{USound} structures do
not.

Let's suppose we have an underlying camera API with the following functions:
\begin{cxxapi}
\item[bool initialize (int id)]\\
  Initialize the camera with given ID.
\item[int getWidth (int id)]\\
  Return image width.
\item[int getHeight (int id)]\\
  Return image height.
\item[char* getImage (int id)]\\
  Get image buffer of format RGB24.  The buffer returned is always the
  same and doesn't have to be freed.
\end{cxxapi}

Our device code can be written as follows:
\begin{cxx}
// Inherit from UObject.
class Camera : public urbi::UObject
{
public:
  // The class must have a single constructor taking a string.
  Camera(const std::string&);

  // Urbi constructor.
  int init (int id);

  // Our image variable and dimensions.
  urbi::UVar val;
  urbi::UVar width;
  urbi::UVar height;

  // Called on access.
  int getVal (UVar&);

  // Called periodically.
  virtual int update ();

  // Frame counter for caching.
  int frame;
  // Frame number of last access.
  int accessFrame;
  // Camera id.
  int id_;
  // Storage for last captured image.
  UBinary bin;
};
\end{cxx}

The constructor only registers \lstinline{init}:

\begin{cxx}
Camera::Camera (const std::string& s)
  : urbi::UObject (s)
  , frame (0)
{
  UBindFunction (Camera, init);
}
\end{cxx}

The \lstinline{init} function binds the variable, a function called on
access, and sets a timer up on update. It also initializes the
\lstinline{UBinary} structure.

\begin{cxx}
int
Camera::init (int id)
{
  //urbi constructor
  id_ = id;
  frame = 0;
  accessFrame = 0;

  if (!initialize (id))
    return 0;

  UBindVar (Camera, val);
  UBindVar (Camera, width);
  UBindVar (Camera, height);
  width = getWidth (id);
  height = getHeight (id);

  UNotifyAccess (val, &Camera::getVal);

  bin.type = BINARY_IMAGE;
  bin.image.width = width;
  bin.image.height = height;
  bin.image.imageFormat = IMAGE_RGB;
  bin.image.size = width * height * 3;

  // Call update () periodically.
  USetUpdate (1);

  return 0;
}
\end{cxx}

The \lstinline{update} function simply updates the frame counter:

\begin{cxx}
int
Camera::update ()
{
  ++frame;
  return 0;
}
\end{cxx}

The \lstinline{getVal} updates the camera value, only if it hasn't
already been called this frame, which provides a simple caching
mechanism to avoid performing the potentially long operation of
acquiring an image too often.

\begin{cxx}
int
Camera::getVal(urbi::UVar&)
{
  if (frame == accessFrame)
    return 1;

  bin.image.data = getImage (id);
  // Assign image to bin.
  val = bin;
}

UStart(Camera);
\end{cxx}

The image data is copied inside the kernel when proceeding this way.

Be careful, suppose that we had created the \lstinline{UBinary}
structure inside the \lstinline{getVal} method, our buffer would have
been freed at the end of the function. To avoid this, set it to 0
after assigning the \lstinline{UBinary} to the \lstinline{UVar}.

\subsection{Optimization in Plugin Mode}

In plugin mode, it is possible to access the buffer used by the kernel
by casting the \lstinline{UVar} to a \lstinline{UImage}. You can
modify the content of the kernel buffer but no other argument.

\section{Writing a Speaker or Microphone Device}

Sound handling works similarly to image manipulation, the
\lstinline{USound} structure is provided for this purpose. The
recommended way to implement a microphone is to fill the
\lstinline{UObject} val variable with the sound data corresponding to
one kernel period. If you do so, the \urbi code
\lstinline{loop tag:micro.val,} will produce the expected result.

\section{Writing a Softdevice: Ball Detection}

Algorithms that require intense computation can be written in \Cxx but
still be usable within \urbi: they acquire their data using
\lstinline{UVar} referencing other modules' variables, and output
their results to other \lstinline{UVar}. Let's consider the case of a
ball detector device that takes an image as input, and outputs the
coordinates of a ball if one is found.

The header is defined like:

\begin{cxx}
class BallTracker : public urbi::UObject
{
public:
  BallTracker (const std::string&);
  int init (const std::string& varname);

  // Is the ball visible?
  urbi::UVar visible;

  // Ball coordinates.
  urbi::UVar x;
  urbi::UVar y;
 };
\end{cxx}

The constructor only registers \lstinline{init}:

\begin{cxx}
// The constructor registers init only.
BallTracker::BallTracker (const::string& s)
  : urbi::UObject (s)
{
  UBindFunction (BallTracker, init);
}
\end{cxx}

The \lstinline{init} function binds the variables and a callback on
update of the image variable passed as a argument.

\begin{cxx}
int
BallTracker::init (const std::string& cameraval)
{
  UBindVar (BallTracker, visible);
  UBindVar (BallTracker, x);
  UBindVar (BallTracker, y);
  UNotifyChange (cameraval, &BallTracker::newImage);

  visible = 0;

  return 0;
}
\end{cxx}

The \lstinline{newImage} function runs the detection algorithm on the
image in its argument, and updates the variables.

\begin{cxx}
int
BallTracker::newImage (urbi::UVar& v)
{
  //cast to UImage
  urbi::UImage i = v;
  int px,py;
  bool found = detectBall (i.data, i.width, i.height, &px, &py);

  if (found)
  {
    visible = 1;
    x = px / i.width;
    y = py / i.height;
  }
  else
    visible = 0;

  return 0;
}
\end{cxx}

%%% Local Variables:
%%% mode: latex
%%% TeX-master: "../urbi-sdk"
%%% End:

% LocalWords:  UObject API UVar UBinary USound UImage liburbi UValue namespace
% LocalWords:  urbi const UBindVar UBindFunction rhs UStart init UNotifyChange
% LocalWords:  UNotifyAccess notifyChange notifyAccess USetUpdate USetTimer pos
% LocalWords:  UOwned UNotifyOnRequest requestValue UProp enum struct plugin px
% LocalWords:  UObjects UObjectHub UStartHub URegister getUObjectHub classname
% LocalWords:  someevent sometag bool getPosition setPosition setPID PID Uowned
% LocalWords:  valueChanged valueAccessed pidChanged headPan ampli getVal vals
% LocalWords:  setPositions servohub addValue realloc sizeof destructor RGB py
% LocalWords:  getWidth getHeight getImage accessFrame softdevice BallTracker
% LocalWords:  varname cameraval newImage detectBall
