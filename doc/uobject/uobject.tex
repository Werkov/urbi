\part[The Middleware Architecture: UObjects]
     {The Middleware Architecture:\\UObjects}
\label{part:uobject}

\chapter*{Introduction}
This part guides you through the various steps of writing an \urbi
\Cxx component using the UObject API. This API can be used to add new
objects written in \Cxx to the \urbi language, and to interact from
\Cxx with the objects that are already defined. We cover the use cases
of controlling a physical device (servomotor, speaker, camera...), and
interfacing higher-lever components (voice recognition, object
detection, ...) with \urbi.

The API defines the UObject class. Each instance of a derived class in
your \Cxx code will correspond to an \urbi object sharing some of its
methods and attributes. The API provides methods to declare which
elements of your object are to be shared. To share a variable with
\urbi, you have to give it the type UVar. This type is a container that
provides cast and equal operators for all types known to \urbi: double,
string and char*, and the binary-holding structures UBinary, USound
and UImage. This type can also read from and write to the liburbi
UValue class. The API provides methods to set up callbacks functions
that will be notified when a variable is modified or read from \urbi
code. Instance methods of any prototype can be rendered accessible
from \urbi, providing all the parameters types and the return type can
be converted to/from UValue.

The next section is a walk through covering all the aspects of the
UObject API, and the rest of the chapter describes many use cases in
details.

%% Copyright (C) 2009-2011, Gostai S.A.S.
%%
%% This software is provided "as is" without warranty of any kind,
%% either expressed or implied, including but not limited to the
%% implied warranties of fitness for a particular purpose.
%%
%% See the LICENSE file for more information.

\chapter{Quick Start}
\label{sec:uob:quick}

This chapter presents \usdk with a specific focus on its middleware
features.  It is self-contained in order to help readers quickly grasp the
potential of \urbi used as a middleware.  References to other sections of
this document are liberally provided to point the reader to the more
complete documentation; they should be ignored during the first reading.

\newcommand{\machineDir}{\uobjectsDir/test/machine.uob}
% firstline to skip the license header.
\newcommand{\machineFile}[1]
{\cxxPre%
\lstinputlisting[language=C++,style=UrbiSDKEnv,firstline=11]{\machineDir/#1}%
\cxxPost}

\section{UObject Basics}
\label{sec:uob:quick:basics}

As a simple running example, consider a (very) basic factory.  Raw material
delivered to the factory is pushed into some assembly machines, which takes
some time.

\subsection{The Objects to Bind into \urbi}

As a firth component of this factory, the core machine of the factory is
implemented as follows.  This class is pure regular \Cxx, it uses no \urbi
feature at all.

The header \file{machine.hh}, which declares \lstinline|Machine|, is
traditional.  The documentation uses the Doxygen syntax.

\machineFile{machine.hh}

The implementation file, \file{machine.cc}, is equally straightforward.

\machineFile{machine.cc}

\subsection{Wrapping into an UObject}

By \dfn{binding} a UObject, we mean using the UObject API to declare objects
to the \urbi world.  These objects have member variables (also known as
\dfn{attributes}) and/or member functions (also known as \dfn{methods}) all
of them or some of them being declared to \urbi.

One could modify the \lstinline|Machine| class to make it a UObject, yet we
recommend wrapping pure \Cxx classes into a different, wrapping, UObject.
It is strongly suggested to aggregate the native \Cxx objects in the UObject
--- rather than trying to derive from it.  By convention, we prepend a ``U''
to the name of the base class, hence the \lstinline|UMachine| class.  This
class provides a simplified interface, basically restricted to what will be
exposed to \urbi.  It must derive from \lstinline|urbi::UObject|.

\machineFile{umachine.hh}

The implementation of \lstinline|UMachine| is simple.  It uses some of the
primitives used in the binding process (\autoref{sec:uob:api:bind}):
\begin{itemize}
\item \lstinline|UStart(\var{class})| declares classes that are UObjects;
  eventually such classes will appear in \us as
  \lstinline|uobjects.\var{class}| (but since \refObject{Global} derives
  from \refObject{uobjects} you may just use the simpler name \var{class}).
  Use it once.

\item \lstinline|UBindFunction(\var{class}, \var{function})| declares a
  \var{function}.  Eventually bound in the \us world as
  \lstinline|uobjects.\var{class}.\var{function}|.

\item Similarly, \lstinline|UBindVar(\var{class}, \var{variable})| declares
  a \var{variable}.
\end{itemize}

\urbi relies on the prototype model for object-oriented programming, which
is somewhat different from the traditional \Cxx class-based model
(\autoref{sec:tut:value}).  This is reflected by the presence of \emph{two}
different constructors:
\begin{itemize}
\item \lstinline|UMachine::UMachine|, the \Cxx constructor invoked for every
  single instance of the UObject.  It is always invoked by the \urbi system
  when instantiating a UObject, \emph{including} the prototype itself.  It
  sole argument is its name (an internal detail you need not be aware of).
  The \Cxx constructor must register the \lstinline|UMachine::init|
  function.  It may also bind functions and variables.

\item \lstinline|UMachine::init|, the \urbi constructor invoked each time a
  new clone of \lstinline|UMachine| is made, i.e., for every instance except
  the first one.

  Functions and variables that do not make sense for the initial prototype
  (which might not be fully functional) should be bound here, rather than in
  the \Cxx constructor.
\end{itemize}

The following listing is abundantly commented, and is easy to grasp.

\machineFile{umachine.cc}

\subsection{Running Components}
\label{sec:uob:quick:run}
%  * Compilation et branchement de UObject (urbi-launch): expliquer les
%  deux modes: distant / pluge'
%
%  * Cycle complet de lancement d'une appli Urbi: lancement d'urbi,
%  execution d'urbi.ini, chargement dynamique des UObjects


As a first benefit from using the \urbi environment, this source code is
already runnable!  No \lstinline|main| function is needed, the \urbi system
provides one.

\subsubsection{Compiling}
Of course, beforehand, we need to compile this UObject into some loadable
module.  The \urbi modules are \dfn{shared objects}, i.e., libraries that
can be loaded on demand (and unloaded) during the execution of the program.
Their typical file names depend on the architecture: \file{machine.so} on
most Unix platforms (including Mac OS X), and \file{machine.dll} on Windows.
To abstract away from these differences, we will simply use the base name,
\file{machine} with the \urbi tool chain.

There are several options to compile our machine as a UObject.  If you are
using Microsoft Visual Studio, the \usdk installer created a ``UObject''
project template accessible through the ``New project'' wizard.  Otherwise,
you can use directly your regular compiler tool chain.  You may also use
\command{umake-shared} from the \file{umake-*} family of programs
(\autoref{sec:tools:umake:wrappers}):

\begin{shell}
$ ls machine.uob
machine.cc  machine.hh  umachine.cc  umachine.hh
$ umake-shared machine.uob -o machine
# ... Lots of compilation log messages ...
$ ls
_ubuild-machine.so  machine.la  machine.so  machine.uob
\end{shell}

The various files are:
\begin{description}
\item[\file{machine.uob}] Merely by convention, the sources of our UObject
  are in a \file{*.uob} directory.
\item[\file{machine.so}] A shared dlopen-module.  This is the ``true''
  product of the \command{umake-shared} invocation.  Its default name can be
  quite complex (\file{uobject-i386-apple-darwin9.7.0.so} on my machine), as
  it will encode information about the architecture of your machine; if you
  don't need such accuracy, use the option \option{--output}/\option{-o} to
  specify the output file name.

  \command{umake-shared} traversed \file{machine.uob} to gather and process
  relevant files (source files, headers, libraries and object files) in
  order to produce this output file.

\item[\file{\_ubuild-machine.so}] this is a temporary directory in which the
  compilation takes place.  It can be safely removed by hand, or using
  \command{umake-deepclean} (\autoref{sec:tools:umake:wrappers}).

\item[\file{machine.la}] a \href{http://www.gnu.org/software/libtool/}{GNU
    Libtool} file that contains information such as dependencies on other
  libraries.  While this file should be useless most of the time, we
  recommend against removing it: it may help understand some problems.
\end{description}

\subsubsection{Running UObjects}

There are several means to toy with this simple UObject.  You can use
\command{urbi-launch} (\autoref{sec:tools:urbi-launch}) to plug the UMachine
in an \urbi server and enter an interactive session.

\begin{shell}[alsolanguage={[interactive]urbiscript}]
# Launch an Urbi server with UMachine plugged in.
$ urbi-launch --start machine -- --interactive
[00000000] *** ********************************************************
[00000000] *** Urbi SDK version 2.7 rev. a6a1ec5
[00000000] *** Copyright (C) 2004-2011 Gostai S.A.S.
[00000000] ***
[00000000] *** This program comes with ABSOLUTELY NO WARRANTY.  It can
[00000000] *** be used under certain conditions.  Type `license;',
[00000000] *** `authors;', or `copyright;' for more information.
[00000000] ***
[00000000] *** Check our community site: http://www.urbiforge.org.
[00000000] *** ********************************************************
var f = UMachine.new(1s);
[00020853] UMachine_0x1899c90
f.assemble(["Hello, ", "World!"]);
[00038705] "Hello, World!"
shutdown;
\end{shell}

You may also launch the machine UObject in the background, as a network
component:

\begin{shell}[alsolanguage={[interactive]urbiscript}]
$ urbi-launch --start machine --host 0.0.0.0 --port 54000 &
\end{shell}

\noindent
and interact with it using your favorite client (\command{telnet},
\command{netcat}, \command{socat}, \ldots), or using the \command{urbi-send}
tool (\autoref{sec:tools:urbi-send}).

\begin{shell}[alsolanguage={[interactive]urbiscript}]
$ urbi-send --port 54000                       \
            -e 'UMachine.assemble([12, 34]);'  \
            --quit
[00126148] "1234"
[00000000:client_error] End of file
$ urbi-send --port 54000                     \
            -e 'var f = UMachine.new(1s)|'   \
            -e 'f.assemble(["ab", "cd"]);'   \
            --quit
[00146159] "abcd"
[00000000:client_error] End of file
\end{shell}


% \subsection{Asynchronicity}
% * Asynchronisme dans UObject: notifychange, timers


\section{Using \us}

% Introduction a` Urbiscript (rappeler que c'est avant tout un langage
% comme les autres, avec if/for/while, etc), puis passer rapidement
% aux elements clefs de la prog evenementielle dans Urbi: at/whenever

\us is a programming language primarily designed for robotics.  Its syntax
is inspired by that of \Cxx: if you know \langC, \Cxx, \Java or \Cs, writing \us
programs is easy.  It's a dynamic object-oriented (\autoref{sec:tut:object})
scripting language, which makes it well suited for high-level application.
It supports and emphasizes parallel (\autoref{sec:tut:concurrent}) and
event-based programming (\autoref{sec:tut:event-prog}), which are very
popular paradigms in robotics, by providing core primitives and language
constructs.

Thanks to its client/server approach, one can easily interact with a robot,
to monitor it, or to experiment live changes in the \us programs.

Courtesy of the UObject architecture, \us is fully integrated with \Cxx. As
already seen in the above examples (\autoref{sec:uob:quick:run}), you can
bind \Cxx classes in \us seamlessly. \us is also integrated with many other
languages such as \Java (\autoref{sec:uob:apijava}), \MatLab or \Python.
The UObject framework also naturally provides \us with support for
distributed architectures: objects can run in different processes, possibly
on remote computers.

%% Everything that follows is bullshit: UObject invocations are
%% *not* in different threads, so we are definitely doomed here.
%%
\subsection{The \us Scripting Language}

The following example shows how one can easily interface UObjects into the
\us language.  The following simple class (actually, a genuine object, in
\us ``classes are objects'', see \autoref{sec:tut:object}) aggregates two
assembly machines, a fast one, and a slow one.  This class demonstrates
usual object-oriented, sequential, features.

\begin{urbicomment}
//#plug test/machine
//#no-fast
\end{urbicomment}
\begin{urbiscript}
class TwoMachineFactory
{
  // A shorthand common to all the Two Machine factories.
  var UMachine = uobjects.UMachine;

  // Default machines.
  var fastMachine = UMachine.new(10ms);
  var slowMachine = UMachine.new(100ms);

  // The urbiscript constructor.
  // Build two machines, a fast one, and a slow one.
  function init(fast = 10ms, slow = 100ms)
  {
    // Make sure fast <= slow.
    if (slow < fast)
      [fast, slow] = [slow, fast];
    // Two machines for each instance of TwoMachineFactory.
    fastMachine = UMachine.new(fast);
    slowMachine = UMachine.new(slow);
  };

  // Wrappers to make invocation of the machine simpler.
  function fast(input) { fastMachine.assemble(input) };
  function slow(input) { slowMachine.assemble(input) };

  // Use the slow machine for large jobs.
  function assemble(input)
  {
    var res|
    var machine|
    if (5 < input.size)
      { machine = "slow" | res = slow(input); }
    else
      { machine = "fast" | res = fast(input); } |
    echo ("Used the %s machine (%s => %s)" % [machine, input, res]) |
    res
  };
};
[00000001] TwoMachineFactory
\end{urbiscript}

Using this class is straightforward.

\begin{urbiscript}
var f = TwoMachineFactory.new|;
f.assemble([1, 2, 3, 4, 5, 6]);
[00000002] *** Used the slow machine ([1, 2, 3, 4, 5, 6] => 123456)
[00000003] "123456"
f.assemble([1]);
[00000004] *** Used the fast machine ([1] => 1)
[00000005] "1"
\end{urbiscript}

The genuine power of \us is when concurrency comes into play.

\subsection{Concurrency}
% * Parallelisme explicite: \&
%
% * Tags et controle d'execution
%
% * Channels et liburbi (les principes, renvoyer a` la doc liburbi pour
% l'API)
%
% * All together, on reprend l'exemple du de'but et on de'taille le code
% des UObjects utilise's, puis le code urbiscript.

Why should we wait for the slow job to finish if we have a fast machine
available?  To do so, we must stop requesting a \emph{sequential}
composition between both calls.  We did that by using the sequential
operator, \samp{;}.  In \us, there exists its concurrent counter-part:
\samp{,}.  Indeed, running \lstinline|\var{a}, \var{b}| means ``launch the
program \var{a} and then launch the program \var{b}''.  The \us Manual
contains a whole section devoted to explaining these operators
(\autoref{sec:tut:concurrent}).

\subsubsection{First Attempt}
\label{sec:uob:quick:fst}
Let's try it:

\begin{urbiscript}
f.assemble([1, 2, 3, 4, 5, 6]),
f.assemble([1]),
[00000002] *** Used the slow machine ([1, 2, 3, 4, 5, 6] => 123456)
[00000004] *** Used the fast machine ([1] => 1)
\end{urbiscript}

\noindent
This is a complete failure.

Why?

Since \urbi cannot expect that your code is thread-safe, by default all
calls to UObject features are \dfn{synchronous}, or \dfn{blocking}:
\emph{the whole \urbi system is suspended until the function returns}.
There is a single thread of execution for \urbi, and when calling a function
from a (plugged) UObject, that thread of execution is devoted to evaluated
the code.

See for instance below that, in even though \lstinline|f.assemble| is slow
and launched in background, the almost instantaneous display of
\lstinline|ping| is not performed immediately.

\begin{urbiscript}
echo(f.slow([1, 2, 3, 4, 5, 6])),
echo("ping");
[00000002] *** 123456
[00000002] *** ping
\end{urbiscript}

There are several means to address this unintended behavior.  If the base
library provides a threaded API (in our example, the \lstinline|Machine|
class, not the \lstinline|UMachine| UObject wrapper), then you could use
it.  Yet we don't recommend it, as it takes away from \urbi the possibility
to really control concurrency issues (for instance it cannot turn
non-blocking functions into blocking functions).

A better option is to ask \urbi to turn blocking function calls into
non-blocking ones.

\subsubsection{Second Attempt: Threaded Functions}

If you read carefully the body of the \lstinline|UMachine::init| function,
you will find the following piece of code:

\begin{cxx}
UBindFunction(UMachine, assemble);
UBindThreadedFunctionRename
  (UMachine, assemble, "threadedAssemble", urbi::LOCK_FUNCTION);
\end{cxx}

\noindent
Both calls bind the function \lstinline|UMachine::assemble|, but the second
one will run the call in a separate thread.  Since multiple calls to a
single function (or different functions of a single object etc.) are likely
to fail, a locking model must be provided.  Here,
\lstinline|urbi::LOCK_FUNCTION| means that concurrent calls to
\lstinline|UMachine::assemble| must be serialized: one at a time.

To use this threaded version of \lstinline|assemble|, we can simply patch
our \lstinline|TwoMachineFactory| class:

\begin{urbiscript}
do (TwoMachineFactory)
{
  fast = function (input) { fastMachine.threadedAssemble(input) };
  slow = function (input) { slowMachine.threadedAssemble(input) };
}|;
\end{urbiscript}

Let's try again where we failed previously (\autoref{sec:uob:quick:fst}):

\begin{urbiscript}
f.assemble([1, 2, 3, 4, 5, 6]),
f.assemble([1]),
[00000004] *** Used the fast machine ([1] => 1)
[00000002] *** Used the slow machine ([1, 2, 3, 4, 5, 6] => 123456)
sleep(200ms);
\end{urbiscript}

Victory!  The fast machine answered first.

You may have noticed that the result is no longer reported.  Indeed, the \us
interactive shell only displays the result of synchronous expressions (i.e.,
those ending with a \samp{;}): asynchronous answers are confusing (see the
inversion here).

\refObject[Channel]{Channels} are useful to ``send'' asynchronous answers.

\begin{urbiscript}
var c1 = Channel.new("c1")|;
var c2 = Channel.new("c2")|;
c1 << f.assemble([10, 20, 30, 40, 50, 60]),
c2 << f.assemble([100]),
sleep(200ms);
[00000535] *** Used the fast machine ([100] => 100)
[00000535:c2] "100"
[00000625] *** Used the slow machine ([10, 20, 30, 40, 50, 60] => 102030405060)
[00000625:c1] "102030405060"
\end{urbiscript}

\section{Conclusion}

This section gave only a quick glance to all the power that \urbi provides
to support concurrency.  Actually, it makes plenty of sense to embed an
\urbi engine in a native \Cxx program, and to delegate the concurrency
issues to it.  Thanks to its middleware and client/server architecture, it
is then possible to connect it to remote components of different kinds, such
as using ROS for instance.

Then, a host of features are at hand, ready to be used when you need them:
event-driven programming, automatic monitoring of expressions, interactive
sessions, \ldots and, last but not least, the \us programming language.

% * Conclusion note: expliquer qu'on peut embedder l'engine Urbi dans
% une appli native C++ (tout le monde le demande), qu'on peut faire
% des bridges ge'ne'riques avec d'autres archis a` composants (citer
% CORBA), donner le lien pour te'le'charger, le lien vers la doc du
% langage urbiscript, vers la doc de UObject, etc.



%%% Local Variables:
%%% coding: utf-8
%%% mode: latex
%%% TeX-master: "../urbi-sdk"
%%% ispell-dictionary: "american"
%%% ispell-personal-dictionary: "../urbi.dict"
%%% fill-column: 76
%%% End:

\chapter{The UObject API}
\label{sec:uob:api}

\section{Creating a class, binding variables and functions}
\label{sec:uob:api:bind}

Let's illustrate those concepts by defining a simple object:
\lstinline{adder}. This object has one variable \lstinline{v}, and a
method \lstinline{add} that returns the sum of this variable and its
argument.

\begin{itemize}
\item First the required include and namespace:

\begin{cxx}
#include <urbi/uobject.hh>
\end{cxx}

\item Then we declare our \lstinline{adder} class:
\begin{cxx}
class adder : public urbi::UObject // Must inherit from UObject.
{
  public:
   // The class must have a single constructor taking a string.
   adder (const std::string&);

   // Our variable.
   urbi::UVar v;

   // Our method.
   double add (double);
};
\end{cxx}
\item The implementation of the constructor and our \lstinline{add}
  method:
\begin{cxx}
// the constructor defines what is available from Urbi
adder::adder (const std::string& s)
  : UObject (s) // required
{
  // Bind the variable.
  UBindVar (adder, v);

  // Bind the function.
  UBindFunction (adder, add);
}

double
adder::add (double rhs)
{
  return v + rhs;
}
\end{cxx}
\item And register this class:
\begin{cxx}
// Register the class to the Urbi kernel.
UStart (adder);
\end{cxx}
\end{itemize}

To summarize:

\begin{itemize}
\item Declare your object class as inheriting from
  \lstinline{urbi::UObject}.
\item Declare a single constructor taking a string, and pass this
  string to the constructor of \lstinline{urbi::UObject}.
\item Declare the variables you want to share with \urbi with the type
  \lstinline{urbi::UVar}.
\item In the constructor, use the macros
  \lstinline|UBindVar(\var{class-name}, \var{variable-name})|
  for each \lstinline{UVar} you want as an instance variable, and
  \lstinline|UBindFunction(\var{class-name}, \var{function-name})| for
  each function you want to bind.
\item Call the macro \lstinline{UStart} for each object.
\end{itemize}

\section{Creating new instances}

When you start an \urbi server, an object of each class registered
with \lstinline{UStart} is created with the same name as the
class. New instances can be created from \urbi using the
\lstinline|new| method. For each instance created in \urbi, a
corresponding instance of the \Cxx object is created. You can get the
arguments passed to the constructor by defining and binding a method
named \lstinline|init| with the appropriate number of arguments.

\section{Notification of a variable change or access}

You can register a function that will be called each time a variable
is modified or accessed (for embedded components only) by calling
\lstinline{UNotifyChange} and \lstinline{UNotifyAccess}, passing
either an \lstinline{UVar} or a variable name as first argument, and a
member function of your \lstinline{UObject} as second argument. This
function can take zero or one argument: a \lstinline{UVar} reference
pointing to the \lstinline{UVar} being accessed or modified. The
\lstinline{notifyChange} callback function is called after the
variable value is changed, whereas the \lstinline{notifyAccess}
callback is called before the variable is accessed, giving you the
possibility to update its value.

\section{Timers}
\label{sec:uob:timers}

The API gives you two methods to have a function called periodically:
\begin{itemize}
\item \lstinline|urbi::UObject::USetUpdate(ufloat \var{period})|\\
  Set up a timer that calls the virtual method
  \lstinline{UObject::update()} with the specified period (in
  milliseconds).  Disable updates if \var{period} is -1.

\item \lstinline|urbi::UObject::USetTimer<T>(ufloat \var{period}), int (T::*\var{fun})()|\\
  Invoke an UObject member function \var{fun} every \var{period}
  milliseconds.  \var{fun} is a regular member-function pointer, for
  instance \lstinline|MyUObject::my_function|.
\end{itemize}

\section{The special case of sensor/effector variables}

In \urbi, a variable can have a different meaning depending on whether
you are reading or writing it: you can use the same variable to
represent the target value of an effector and the current value
measured by an associated sensor. This special mode is activated by
the \lstinline{UObject} defining the variable by calling
\lstinline{UOwned} after calling \lstinline{UBindVar}. This call has
the following effects:
\begin{itemize}
\item When \urbi code or code in other modules read the variable, they
  read the current value.
\item When \urbi code or code in other modules write the variable,
  they set the target value.
\item When the module that called \lstinline|UOwned| reads the
  variable, it reads the target value. When it writes the variable, it
  writes the current value.
\end{itemize}

\section{Using \urbi variables}

You can read or write any \urbi variable by creating an
\lstinline{UVar} passing the variable name to the constructor. Change
the value by writing any compatible type to the \lstinline{UVar}, and
access the value by casting the \lstinline{UVar} to any compatible
type. However, some care must be taken in remote mode: changes on the
variable coming from \urbi code or an other module are only visible if
you have called \lstinline{UNotifyChange} on this variable. Otherwise
the \lstinline{UVar} is not synchronized. Alternatively, in remote
mode, you can get the value on demand by calling
\lstinline|UNotifyOnRequest (\var{variable}, \var{function})|. Then to get an
updated value, call the \lstinline{requestValue} method on the
\lstinline{UVar}, and your callback function will be called as soon as
the value is available. You can read and write all the \urbi
properties of an \lstinline{UVar} by reading and writing the
appropriate \lstinline{UProp} object in the \lstinline{UVar}.

\section{Using binary types}

\urbi can store binary objects of any type in a generic container, and
provides specific structures for sound and images. The generic
containers is called \lstinline{UBinary} and is defined in the
\lstinline{urbi/uobject.hh} header. It contains an enum field type
giving the type of the binary (\lstinline{UNKNOWN}, \lstinline{SOUND}
or \lstinline{IMAGE}), and an union of a \lstinline{USound} and
\lstinline{UImage} struct containing a pointer to the data, the size
of the data and type-specific meta-information. Reading a
\lstinline{UBinary} from a \lstinline{UVar}, and writing a
\lstinline{UBinary} to a \lstinline{UVar} performs a deep-copy of the
data. In plugin mode, you can access directly the buffer used by the
kernel by casting the \lstinline{UVar} to an \lstinline{UImage} or a
\lstinline{USound}. You can then write to the buffer, but you can not
change any other information.

\section{Using hubs to group objects}

Sometimes, you need to perform actions for a group of
\lstinline{UObjects}, for instance devices that need to be updated
together. The API provides the \lstinline{UObjectHub} class for this
purpose. To create a hub, simply declare a subclass of
\lstinline{UObjectHub}, and register it by calling once the macro
\lstinline|UStartHub (\var{class-name})|. A single instance of this class
will then be created upon server start-up. \lstinline{UObject}
instances can then register to this hub by calling
\lstinline|URegister (\var{hub-class-name})|. Timers can be attached to
\lstinline{UObjectHub} the same way as to \lstinline{UObject} (see
\autoref{sec:uob:timers}). The kernel will call the \lstinline{update()}
method of all \lstinline{UObject} before calling the
\lstinline{update()} method of the hub. A hub instance can be
retrieved by calling \lstinline{getUObjectHub (string classname)}. The
hub also holds the list of registered UObject in its members
attribute.

\section{Sending \urbi code}

If you need to send \urbi code to the server, the \lstinline{URBI()}
macro is available, as well as the \lstinline{send()} function. You
can either pass it a string, or directly \urbi code inside a pair of
parentheses:

\begin{urbifixme}
send ("tag:1+1;");

URBI (( at (someevent (x)) { sometag:echo x; }; ));
\end{urbifixme}


%%% Local Variables:
%%% mode: latex
%%% TeX-master: "../urbi-sdk"
%%% End:

% LocalWords:  UObject API UVar UBinary USound UImage liburbi UValue namespace
% LocalWords:  urbi const UBindVar UBindFunction rhs UStart init UNotifyChange
% LocalWords:  UNotifyAccess notifyChange notifyAccess USetUpdate USetTimer pos
% LocalWords:  UOwned UNotifyOnRequest requestValue UProp enum struct plugin px
% LocalWords:  UObjects UObjectHub UStartHub URegister getUObjectHub classname
% LocalWords:  someevent sometag bool getPosition setPosition setPID PID Uowned
% LocalWords:  valueChanged valueAccessed pidChanged headPan ampli getVal vals
% LocalWords:  setPositions servohub addValue realloc sizeof destructor RGB py
% LocalWords:  getWidth getHeight getImage accessFrame softdevice BallTracker
% LocalWords:  varname cameraval newImage detectBall

\chapter{Use Cases}
\label{sec:uob:uses}
\section{Writing a Servomotor Device}

Let's write a \lstinline{UObject} for a servomotor device whose
underlying API is:

\begin{cxxapi}
\item[bool initialize (int id)]
  Initialize the servomotor with given ID.
\item[double getPosition (int id)]
  Read servomotor of given id position.
\item[void setPosition (int id, double pos)]
  Send a command to servomotor.
\item[void setPID (int id, int p, int i, int d)]
  Set P, I, and D arguments.
\end{cxxapi}

First our header. Our servo device provides an attribute named
\lstinline{val}, the standard \urbi name, and two ways to set PID
gain: a method, and three variables.

\begin{cxx}
class servo : public urbi::UObject // must inherit UObject
{
public:
  // the class must have a single constructor taking a string
  servo(const std::string&);

  // Urbi  constructor
  int init(int id);

  // main attribute
  urbi::UVar val;

  // position variables:
  //  P gain
  urbi::UVar P;
  //  I gain
  urbi::UVar I;
  //  D gain
  urbi::UVar D;

  // callback for val change
  int valueChanged(UVar& v);
  //callback for val access
  int valueAccessed(UVar& v);
  // callback for PID change
  int pidChanged(UVar& v);

  // method to change all values
  void setPID(int p, int i, int d);

  // motor ID
  int id_;
};
\end{cxx}

The constructor only registers init, so that our default instance
\lstinline{servo} does nothing, and can only be used to create new
instances.

\begin{cxx}
servo::servo (const std::string& s)
  : urbi::UObject (s)
{
   // register init
   UBindFunction (servo, init);
}
\end{cxx}

The \lstinline{init} function, called in a new instance each time a
new \urbi instance is created, registers the four variables
(\lstinline{val}, \lstinline{P}, \lstinline{I} and \lstinline{D}), and
sets up callback functions.

\begin{cxx}
// Urbi constructor.
int
servo::init (int id)
{
  id_ = id;

  if (!initialize (id))
    return 1;

  UBindVar (servo, val);

  // val is both a sensor and an actuator.
  Uowned (val);

  // Set blend mode to mix.
  val.blend = urbi::UMIX;

  // Register variables.
  UBindVar (servo, P);
  UBindVar (servo, I);
  UBindVar (servo, D);

  // Register functions.
  UBindFunction (servo, setPID);

  // Register callbacks on functions.
  UNotifyChange (val, &servo::valueChanged);
  UNotifyAccess (val, &servo::valueAccessed);
  UNotifyChange (P, &servo::pidChanged);
  UNotifyChange (I, &servo::pidChanged);
  UNotifyChange (D, &servo::pidChanged);

  return 0;
}
\end{cxx}

Then we define our callback methods. \lstinline{servo::valueChanged}
will be called each time the \lstinline{val} variable is modified,
just after the value is changed: we use this method to send our servo
commands. \lstinline{servo::valueAccessed} is called just before the
value is going to be read. In this function we request the current
value from the servo, and set \lstinline{val} accordingly.

\begin{cxx}
// Called each time val is written to.
int
servo::valueChanged (urbi::UVar& v)
{
  // v is a reference to our class member val: you can use both
  // indifferently.
  setPosition (id, (double)val);

  // The return value is ignored for now, just return 0.
  return 0;
}

// Called each time val is read.
int
servo::valueAccessed (urbi::UVar& v)
{
  // v is a reference to val.
  val = getPosition (id);

  return 0;
}
\end{cxx}

\lstinline{servo::pidChanged} is called each time one of the PID
variables is written to. The function \lstinline{servo::setPID} can be
called directly from \urbi.

\begin{cxx}
int
servo::pidChanged (urbi::UVar& v)
{
  setPID(id, (int)P, (int)I, (int)D);

  return 0;
}

void
servo::setPID (int p, int i, int d)
{
  setPID (id, p, i, d);
  P = p;
  I = i;
  D = d;
}

// Register servo class to the Urbi kernel.
UStart (servo);
\end{cxx}

That's it, compile this module, and you can use it within \us:

\begin{urbiunchecked}
// Create a new instance.  Calls init (1).
headPan = new servo (1);

// Calls setPID ().
headPan.setPID (8,2,1);

// Calls valueChanged ().
headPan.val = 13;

// Calls valueAccessed ().
headPan.val * 12;

// Periodically calls valueChanged ().
headPan.val = 0 sin:1s ampli:20,

// Periodically calls valueAccessed ().
at (headPan.val < 0)
  echo ("left");
\end{urbiunchecked}

The sample code above has one problem: \lstinline{valueAccessed} and
\lstinline{valueChanged} are called each time the value is read or
written from \urbi, which can happen quite often. This is a problem if
sending the actual command (\lstinline{setPosition} in our example)
takes time to execute. There are two solutions to this issue.

\subsection{Caching}

One solution is to remember the last time the value was read/written,
and not apply the new command before a fixed time. Note that the
kernel is doing this automatically for \lstinline{UOwned}'d variables
that are in a blend mode different than \lstinline{normal}. So the
easiest solution to the above problem is likely to set the variable to
the \lstinline{mix} blending mode. The unavoidable drawback is that
commands are not applied immediately, but only after a small delay.

\subsection{Using Timers}

Instead of updating/fetching the value on demand, you can chose to do
it periodically based on a timer. A small difference between the two
API methods comes in handy for this case: the \lstinline{update()}
virtual method called periodically after being set up by
\lstinline{USetUpdate(interval)} is called just after one pass of
\urbi code execution, whereas the timers set up by
\lstinline{USetTimer} are called just before one pass of \urbi code
execution. So the ideal solution is to read your sensors in the second
callback, and write to your actuators in the first. Our previous
example (omitting PID handling for clarity) can be rewritten. The
header becomes:

\begin{cxx}
// Inherit from UObject.
class servo : public urbi::UObject
{
public:
  // The class must have a single constructor taking a string.
  servo (const std::string&)

  // Urbi constructor.
  int init (int id);

  // Called periodically.
  virtual int update ();
  // Called periodically.
  int getVal ();

  // Our position variable.
  urbi::UVar val;

  // Motor ID.
  int id_;
};
\end{cxx}

Constructor is unchanged, \lstinline{init} becomes:

\begin{cxx}
// Urbi constructor.
int
servo::init (int id)
{
  id_ = id;

  if (!initialize (id))
    return 0;

  UBindVar (servo,val);
  // Val is both a sensor and an actuator.
  UOwned(val);

  // Will call update () periodically.
  USetUpdate(1);
  // Idem for getVal ().
  USetTimer (1, &servo::getVal);

  return 0;
}
\end{cxx}

\lstinline{valueChanged} becomes \lstinline{update} and
\lstinline{valueAccessed} becomes \lstinline{getVal}. Instead of being
called on demand, they are now called periodically. The period of the
call cannot be lower than the value returned by
\lstinline{Object.getPeriod;}
so you can set it to 0 to mean ``as fast as is useful''.

\section{Using Hubs to Group Objects}

Now, suppose that, for our previous example, we can speed things up by
sending all the servomotor commands at the same time, using the
following method that takes two arrays of ids and positions.

\begin{cxx}
void setPositions(int count, int* ids, double* positions);
\end{cxx}

A hub is the perfect way to handle this task. The UObject header stays
the same. We add a hub declaration:

\begin{cxx}
class servohub : public urbi::UObjectHub
{
public:
  // The class must have a single constructor taking a string.
  servohub (const std::string&);

  // Called periodically.
  virtual int update ();

  // Called by servo.
  void addValue (int id, double val);

  int* ids;
  double* vals;
  int size;
  int count;
};
\end{cxx}

\lstinline{servo::update} becomes a call to the \lstinline{addValue}
method of the hub:

\begin{cxx}
int
servo::update()
{
  ((servohub*)getUObjectHub ("servohub"))->addValue (id, (double)val);
};
\end{cxx}

The following line can be added to the servo \lstinline{init} method,
although it has no use in our specific example:

\begin{cxx}
URegister(servohub);
\end{cxx}

Finally, the implementation of our hub methods is:

\begin{cxx}
servohub::servohub (const std::string& s)
  : UObjectHub (s)
  , ids   (0)
  , vals  (0)
  , size  (0)
  , count (0)
{
  // setup our timer
  USetUpdate (1);
}

int
servohub::update ()
{
  // Called periodically.
  setPositions (count, ids, vals);

  // Reset position counter.
  count = 0;

  return 0;
}

void
servohub::addValue (int id, double val)
{
  if (count + 1 < size)
  {
    // Allocate more memory.
    ids = (int*) realloc (ids, (count + 1) * sizeof (int));
    vals = (double*) realloc (vals, (count + 1) * sizeof (double));
    size = count + 1;
  }
  ids[count] = id;
  vals[count++] = val;
}

UStartHub (servohub);
\end{cxx}

Periodically, the \lstinline{update} method is called on each servo
instance, which adds commands to the hub arrays, then the
\lstinline{update} method of the hub is called, actually sending the
command and resetting the array.

\subsection{Alternate Implementation}

Alternatively, to demonstrate the use of the members hub variable, we
can entirely remove the \lstinline{update} method in the servo class
(and the \lstinline{USetUpdate()} call in \lstinline{init}), and
rewrite the hub \lstinline{update} method the following way:

\begin{cxx}
int servohub::update()
{
  //called periodically
  for (UObjectList::iterator i = members.begin ();
       i != members.end ();
       ++i)
    addValue (((servo*)*i)->id, (double)((servo*)*i)->val);
  setPositions(count, ids, vals);
  // reset position counter
  count = 0;

  return 0;
}
\end{cxx}

\section{Writing a Camera Device}

A camera device is an UObject whose \lstinline{val} field is a binary
object. The \urbi kernel itself doesn't make any difference between
all the possible binary formats and data type, but the API provides
image-specific structures for convenience. You must be careful about
memory management. The \lstinline{UBinary} structure handles its own
memory: copies are deep, and the destructor frees the associated
buffer. The \lstinline{UImage} and \lstinline{USound} structures do
not.

Let's suppose we have an underlying camera API with the following functions:
\begin{cxxapi}
\item[bool initialize (int id)]\\
  Initialize the camera with given ID.
\item[int getWidth (int id)]\\
  Return image width.
\item[int getHeight (int id)]\\
  Return image height.
\item[char* getImage (int id)]\\
  Get image buffer of format RGB24.  The buffer returned is always the
  same and doesn't have to be freed.
\end{cxxapi}

Our device code can be written as follows:
\begin{cxx}
// Inherit from UObject.
class Camera : public urbi::UObject
{
public:
  // The class must have a single constructor taking a string.
  Camera(const std::string&);

  // Urbi constructor.
  int init (int id);

  // Our image variable and dimensions.
  urbi::UVar val;
  urbi::UVar width;
  urbi::UVar height;

  // Called on access.
  int getVal (UVar&);

  // Called periodically.
  virtual int update ();

  // Frame counter for caching.
  int frame;
  // Frame number of last access.
  int accessFrame;
  // Camera id.
  int id_;
  // Storage for last captured image.
  UBinary bin;
};
\end{cxx}

The constructor only registers \lstinline{init}:

\begin{cxx}
Camera::Camera (const std::string& s)
  : urbi::UObject (s)
  , frame (0)
{
  UBindFunction (Camera, init);
}
\end{cxx}

The \lstinline{init} function binds the variable, a function called on
access, and sets a timer up on update. It also initializes the
\lstinline{UBinary} structure.

\begin{cxx}
int
Camera::init (int id)
{
  //urbi constructor
  id_ = id;
  frame = 0;
  accessFrame = 0;

  if (!initialize (id))
    return 0;

  UBindVar (Camera, val);
  UBindVar (Camera, width);
  UBindVar (Camera, height);
  width = getWidth (id);
  height = getHeight (id);

  UNotifyAccess (val, &Camera::getVal);

  bin.type = BINARY_IMAGE;
  bin.image.width = width;
  bin.image.height = height;
  bin.image.imageFormat = IMAGE_RGB;
  bin.image.size = width * height * 3;

  // Call update () periodically.
  USetUpdate (1);

  return 0;
}
\end{cxx}

The \lstinline{update} function simply updates the frame counter:

\begin{cxx}
int
Camera::update ()
{
  ++frame;
  return 0;
}
\end{cxx}

The \lstinline{getVal} updates the camera value, only if it hasn't
already been called this frame, which provides a simple caching
mechanism to avoid performing the potentially long operation of
acquiring an image too often.

\begin{cxx}
int
Camera::getVal(urbi::UVar&)
{
  if (frame == accessFrame)
    return 1;

  bin.image.data = getImage (id);
  // Assign image to bin.
  val = bin;
}

UStart(Camera);
\end{cxx}

The image data is copied inside the kernel when proceeding this way.

Be careful, suppose that we had created the \lstinline{UBinary}
structure inside the \lstinline{getVal} method, our buffer would have
been freed at the end of the function. To avoid this, set it to 0
after assigning the \lstinline{UBinary} to the \lstinline{UVar}.

\subsection{Optimization in Plugin Mode}

In plugin mode, it is possible to access the buffer used by the kernel
by casting the \lstinline{UVar} to a \lstinline{UImage}. You can
modify the content of the kernel buffer but no other argument.

\section{Writing a Speaker or Microphone Device}

Sound handling works similarly to image manipulation, the
\lstinline{USound} structure is provided for this purpose. The
recommended way to implement a microphone is to fill the
\lstinline{UObject} val variable with the sound data corresponding to
one kernel period. If you do so, the \urbi code
\lstinline{loop tag:micro.val,} will produce the expected result.

\section{Writing a Softdevice: Ball Detection}

Algorithms that require intense computation can be written in \Cxx but
still be usable within \urbi: they acquire their data using
\lstinline{UVar} referencing other modules' variables, and output
their results to other \lstinline{UVar}. Let's consider the case of a
ball detector device that takes an image as input, and outputs the
coordinates of a ball if one is found.

The header is defined like:

\begin{cxx}
class BallTracker : public urbi::UObject
{
public:
  BallTracker (const std::string&);
  int init (const std::string& varname);

  // Is the ball visible?
  urbi::UVar visible;

  // Ball coordinates.
  urbi::UVar x;
  urbi::UVar y;
 };
\end{cxx}

The constructor only registers \lstinline{init}:

\begin{cxx}
// The constructor registers init only.
BallTracker::BallTracker (const::string& s)
  : urbi::UObject (s)
{
  UBindFunction (BallTracker, init);
}
\end{cxx}

The \lstinline{init} function binds the variables and a callback on
update of the image variable passed as a argument.

\begin{cxx}
int
BallTracker::init (const std::string& cameraval)
{
  UBindVar (BallTracker, visible);
  UBindVar (BallTracker, x);
  UBindVar (BallTracker, y);
  UNotifyChange (cameraval, &BallTracker::newImage);

  visible = 0;

  return 0;
}
\end{cxx}

The \lstinline{newImage} function runs the detection algorithm on the
image in its argument, and updates the variables.

\begin{cxx}
int
BallTracker::newImage (urbi::UVar& v)
{
  //cast to UImage
  urbi::UImage i = v;
  int px,py;
  bool found = detectBall (i.data, i.width, i.height, &px, &py);

  if (found)
  {
    visible = 1;
    x = px / i.width;
    y = py / i.height;
  }
  else
    visible = 0;

  return 0;
}
\end{cxx}

%%% Local Variables:
%%% mode: latex
%%% TeX-master: "../urbi-sdk"
%%% ispell-personal-dictionary: "../urbi.dict"
%%% End:



%%% Local Variables:
%%% mode: latex
%%% TeX-master: "../urbi-sdk"
%%% ispell-dictionary: "american"
%%% ispell-personal-dictionary: "../urbi.dict"
%%% End:

