\chapter{Getting Started}
\label{sec:tut:started}
\us comes with a set of tools, two of which being of particular
importance:
\begin{description}
\item[\dfn{urbi}] launches an \urbi server.  There are several means
  to interact with it, which we will see later.
\item[\dfn{urbi-launch}] runs \urbi components, the UObjects, and connects
  them to an \urbi server.
\end{description}

Please, first make sure that these tools are properly installed.  If
you encounter problems, please see \autoref{sec:installation}, which
is more detailed, and possibly \autoref{sec:faq} for frequently asked
questions.

\begin{shell}
# Make sure urbi is properly installed.
$ urbi --version
Urbi Kernel version preview/2.0/beta3-425 rev. 000913e
Copyright (C) 2005-2009 Gostai SAS.

URBI SDK Remote version preview/1.6/beta1-666 rev. 92ec3b4
Copyright (C) 2004-2009 Gostai SAS.

Libport version preview/1.0/beta1-1048 rev. f1c5170
Copyright (C) 2005-2009 Gostai SAS.
\end{shell}%$


There are several means to interact with a server spawned by
\command{urbi}, see \autoref{sec:tools:urbi} for details.  First of
all, you may use the options
\option{-e}/\option{--expression \var{code}} and
\option{-f}/\option{--file \var{file}} to send some \var{code} or the
contents of some \var{file} to the newly run server.  You may combine
any number of these options, but beware that being event-driven, the
server does not ``know'' when a program ends.  Therefore, batch
programs should end by calling \lstinline{shutdown}:

\begin{shell}[alsolanguage={[interactive]Urbi}]
# A classical program.
$ urbi -e 'echo("Hello, World!");' -e 'shutdown;'
[00000004] *** Hello, World!
\end{shell}%$

To run an interactive session, use option
\option{-i}/\option{--interactive}.  Like most interactive
interpreters, \urbi will evaluate the given commands and print out the
results.

\begin{shell}[alsolanguage={[interactive]Urbi}]
$ urbi -i
[00000000:start] *** **********************************************************
[00000000:start] *** Urbi Kernel version preview/2.0/beta3-425 rev. 000913e
[00000000:start] *** Copyright (C) 2005-2009 Gostai SAS.
[00000000:start] ***
[00000000:start] *** URBI SDK Remote version preview/1.6/beta1-666 rev. 92ec3b4
[00000000:start] *** Copyright (C) 2004-2009 Gostai SAS.
[00000000:start] ***
[00000000:start] *** Libport version preview/1.0/beta1-1048 rev. f1c5170
[00000000:start] *** Copyright (C) 2005-2009 Gostai SAS.
[00000000:start] ***
[00000000:start] *** URBI comes with ABSOLUTELY NO WARRANTY.
[00000000:start] *** This software can be used under certain conditions;
[00000000:start] *** see LICENSE file for details.
[00000000:start] ***
[00000000:start] *** See http://www.urbiforge.com for news and updates.
[00000000:start] *** **********************************************************
[00000000:start] ***
[00000000:ident] *** ID: U3154688
[00000000:start] *** Urbi is up and running.
1+2;
[00000000] 3
shutdown;
\end{shell}%$

The output from the server is prefixed by a number surrounded by
square brackets: this is the date (in milliseconds since the server
was launched) at which that line was sent by the server. This is
useful at occasions, since \urbi is meant to run many parallel
commands.  But since these timestamps are irrelevant in most examples,
they will often be filled with zeroes through this documentation.

The program \command{rlwrap} provides additional services (history of
commands, advanced command line edition etc.); run \samp{rlwrap
  urbi -i}.

In either case the server can also be made available for network-based
interactions using option \option{--port \var{port}}.  Note that while
\lstinline{shutdown} asks the server to quit, \lstinline{quit} only
quits one interactive session.  In the following example the server is
still available for other, possibly concurrent, sessions.

\begin{shell}[alsolanguage={[interactive]Urbi}]
$ urbi --port 54000 &
[1] 77024
$ telnet localhost 54000
Trying 127.0.0.1...
Connected to localhost.
Escape character is '^]'.
[00000758:start] *** **********************************************************
[00000758:start] *** Urbi Kernel version preview/2.0/beta3-425 rev. 000913e
[00000758:start] *** Copyright (C) 2005-2009 Gostai SAS.
[00000758:start] ***
[00000758:start] *** URBI SDK Remote version preview/1.6/beta1-666 rev. 92ec3b4
[00000758:start] *** Copyright (C) 2004-2009 Gostai SAS.
[00000758:start] ***
[00000758:start] *** Libport version preview/1.0/beta1-1048 rev. f1c5170
[00000758:start] *** Copyright (C) 2005-2009 Gostai SAS.
[00000758:start] ***
[00000758:start] *** URBI comes with ABSOLUTELY NO WARRANTY.
[00000758:start] *** This software can be used under certain conditions;
[00000758:start] *** see LICENSE file for details.
[00000758:start] ***
[00000758:start] *** See http://www.urbiforge.com for news and updates.
[00000758:start] *** **********************************************************
[00000758:start] ***
[00000758:ident] *** ID: U348160
12345679*8;
[00018032] 98765432
quit;
\end{shell}%$

The program \command{urbi-send} provides a nice interface to
interact with a running server.

\begin{shell}[alsolanguage={[interactive]Urbi}]
$ urbi-send -P 54000 -e '1+2*3;' -e 'quit;'
[00018032:] 7
# Have the server shutdown;
$ urbi-send -P 54000 -e 'shutdown;'
\end{shell}

\medskip

You can now send commands to your \urbi server. If at any point you
get lost, or want a fresh start, you can simply close and reopen your
connection to the server to get a clean environment.  In some cases,
particularly if you made global changes in the environment, it is
simpler start anew: shut your current server down, and spawn a new
one.

%%% Local Variables:
%%% mode: latex
%%% TeX-master: "../urbi-sdk"
%%% ispell-dictionary: "american"
%%% ispell-personal-dictionary: "../urbi.dict"
%%% End:
