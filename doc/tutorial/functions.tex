\chapter{Advanced Functions and Scoping}
\label{sec:tut:function}

This section presents advanced uses of functions and scoping, as well
as their combo: lexical closures, which prove to be a very powerful
tool.

\section{Scopes as expressions}

Contrary to other languages from the \C family, scopes are
expressions: they can be used where values are expected, just as
\lstinline|1 + 1| or \lstinline|"foo"|.
They evaluate to the value of their last expression, or
\lstinline|void| if they are
empty. The following listing illustrates the use of scopes as
expressions. Note that the last semicolon inside a scope is optional.

\begin{urbiscript}
// Scopes evaluate to the last expression they contain.
{ 1; 2; 3};
[00000000] 3
// They are expressions.
echo({1; 2; 3});
[00000000] *** 3
\end{urbiscript}

\section{Advanced scoping}

Scopes can be nested. Variables can be redefined in sub-scopes. In
this case, the inner variables hide the outer ones, as illustrated
below.

\begin{urbiscript}[firstnumber=last]
var x = 0;   // Define the outer x.
[00000000] 0
{
  var x = 1; // Define an inner x.
  x = 2;     // These refer to
  echo(x);   // the inner x
};
[00000000] *** 2
x;           // This is the outer x again.
[00000000] 0
{
  x = 3;     // This is still the outer x.
  echo(x);
};
[00000000] *** 3
x;
[00000000] 3
\end{urbiscript}

\section{Local functions}

Functions can be defined anywhere local variables can --- that is,
about anywhere. These functions' visibility are limited to the scope
they're defined in, like variables. This enables for instance to write
local helper functions like ``max2'' in the example belox.

\begin{urbiscript}[firstnumber=last]
function max3(a, b, c) // Max of three values
{
  function max2(a, b)
  {
    if (a > b)
      return a
    else
      return b
  };
  max2(a, max2(b, c));
} | {};
\end{urbiscript}

\section{Lexical closures}

A \dfn{closure} is the capture by a function of a variable external to this
function. \us supports lexical closure: functions can refer to outer
local variables, as long as they are visible (in scope) from where
the function is defined.

\begin{urbiscript}[firstnumber=last]
function printSalaries(rate)
{
  var charges = 100;
  function computeSalary(hours)
  {
    // Here rate and charges are captured
    // from the environment by closure
    rate * hours - charges
  };

  echo("Alice's salary is " + computeSalary(35));
  echo("Bob's salary is " + computeSalary(30));
} | {};
printSalaries(15);
[00000000] *** Alice's salary is 425
[00000000] *** Bob's salary is 350
\end{urbiscript}

Closures can also write to captured variables, as shown below.

% Fixme: these are not actually closures because of the toplevel ...
\begin{urbiscript}[firstnumber=last]
var a = 0;
[00000000] 0
var b = 0;
[00000000] 0
function add(n)
{
  // x and y are updated by closure
  a += n;
  b += n;
  void
} | {};
add(25);
add(25);
add(1);
a;
[00000000] 51
b;
[00000000] 51
\end{urbiscript}

Closure can be really powerful tools in some situation, and they are
even more useful when combined with functional programing, as
described in \autoref{sec:tut:functional}.

%\section{Conclusion}


%%% Local Variables:
%%% mode: latex
%%% TeX-master: "../urbi-sdk"
%%% End:
