\chapter{Flow control constructs}
\label{sec:tut:flow}

In this section, we'll introduce some flow control structures that
will prove handy later. Most of them are inspired by \C/\Cxx.

\section{if}

The \lstinline{if} construct is the same has \C/\Cxx's one. The
\lstinline{if} keyword is followed by a condition between parentheses and
an expression, and optionally the \lstinline{else} keyword and another
expression. If the condition evaluates to true, the first expression
is evaluated. Otherwise, the second expression is evaluated if
present.

\begin{urbiscript}
if (true)
  echo("ok");
[00000000] *** ok
if (false)
  echo("ko")
else
  echo("ok");
[00000000] *** ok
\end{urbiscript}

The \lstinline|if| construct is an expression: it has a value.

\begin{urbiscript}[firstnumber=last]
echo({ if (false) "a" else "b" });
[00000000] *** b
\end{urbiscript}

\section{while}

The \lstinline{while} construct is, again, the same as in \C/\Cxx. The
\lstinline{while} keyword is followed by a condition between parentheses
and an expression. If the expression evaluation is false, the
execution jumps after the while block; otherwise, the expression is
evaluated and control jumps before the while block.

% FIXME: += wasn't introduced
\begin{urbiscript}[firstnumber=last]
var x = 2;
[00000000] 2
while (x < 40)
{
  x += 10;
  echo(x);
};
[00000000] *** 12
[00000000] *** 22
[00000000] *** 32
[00000000] *** 42
\end{urbiscript}

\section{for}

The \lstinline{for} keyword supports different constructs, as in
languages such as Java, \Cs, or even the forthcoming \Cxx revision.

The first construct is hardly more than syntactic suggar for a
\lstinline{while} loop.

\begin{urbiscript}[firstnumber=last]
for (var x = 2; x < 40; x += 10)
  echo(x);
[00000000] *** 2
[00000000] *** 12
[00000000] *** 22
[00000000] *** 32
\end{urbiscript}

The second construct allows to iterate over members of a collection,
such as a list. The \lstinline{for} keyword, followed by
\lstinline|var|, an identifier, a semicolon (or \lstinline|in|), an
expression and a scope, executes the scope for every element in the
collection resulting of the evaluation of the expression, with the
variable named with the identifier referring to the list
elements.

\begin{urbiscript}[firstnumber=last]
for (var e : [1, 2, 3]) { echo(e) };
[00000000] *** 1
[00000000] *** 2
[00000000] *** 3
\end{urbiscript}

\section{switch}

The syntax of the \lstinline|switch| construct is similar to \C/\Cxx's
one, except it works on any kind of object, not only integral
ones. Comparison is done by semantic equality (operator
\lstinline{==}). Execution will jump out of the
\lstinline|switch|-block after a case has been executed (no need to
\lstinline{break}).  Also, contrary to \Cxx, the whole construct has a
value: that of the matching \lstinline{case}.

\begin{urbiscript}[firstnumber=last]
switch ("bar")
{
  case "foo":  0;
  case "bar":  1;
  case "baz":  2;
  case "qux":  3;
};
[00000000] 1
\end{urbiscript}

\section{do}
\label{section:constructs:do}

A \lstinline|do| scope is a shorthand to perform several actions on an
object.

\begin{urbiscript}[firstnumber=last]
var o1 = Object.clone;
[00000000] Object_0x100000
var o1.one = 1;
[00000000] 1
var o1.two = 2;
[00000000] 2
echo(o1.uid);
[00000000] *** 0x100000
\end{urbiscript}

The same result can be obtained with a short \lstinline|do| scope,
that redirect method calls to their target, as in the listing
below. This is similar to the \pascal
``\lstinline[language=Pascal]{with}'' construct.  The value of the
\lstinline{do}-block is the target itself.

\begin{urbiscript}[firstnumber=last]
var o2 = Object.clone;
[00000000] Object_0x100000
// All the message in this scope are destined to o.
do (o2)
{
  var one = 1; // var is a shortcut for the setSlot
  var two = 2; // message, so it applies on obj too.
  echo(uid);
};
[00000000] *** 0x100000
[00000000] Object_0x100000
\end{urbiscript}


%\section{Conclusion} FIXME?

%%% Local Variables:
%%% mode: latex
%%% TeX-master: "../urbi-sdk"
%%% End:
