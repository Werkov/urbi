\begin{partDescription}{part:tut}
  {%
    This part is the \us user manual, also sometimes referred to as
    the ``\us tutorial''.  It teaches the reader how to program in
    \us.  It goes from the basis to concurrent and
    event-based programming.  No specific knowledge is expected.
    There is no need for a \Cxx compiler, as \UObject will not be
    covered here (see \autoref{part:uobject}).  The reference manual
    contains a terse and complete definition of the \urbi environment
    (\autoref{part:specs}).
    %
  }
\item[sec:tut:first]
  First contact with \us.
\item[sec:tut:value]
  A quick introduction to objects and values.
\item[sec:tut:flow]
  Basic control flow: \lstinline{if}, \lstinline{for} and the like.
\item[sec:tut:function]
  Details about functions, scoped, and lexical closures.
\item[sec:tut:object]
  A more in-depth introduction to object-oriented programming in \us.
\item[sec:tut:functional]
  Functions are first-class citizens.
\item[sec:tut:concurrent]
  The \us operators for concurrency, tags.
\item[sec:tut:event-prog]
  Support for event-driven concurrency in \us.
\end{partDescription}


%%% Local Variables:
%%% mode: latex
%%% TeX-master: "../urbi-sdk"
%%% ispell-dictionary: "american"
%%% ispell-personal-dictionary: "../urbi.dict"
%%% End:
