\chapter{Objective programming, \us object model}
\label{sec:tut:object}

This section presents object programing in \us: the prototype-based
object model of \us, and how to define and use classes.

\section{Prototype-based programing in \us}

You're probably already familiar with class-based object programing,
since this is the \Cxx model.  Classes and objects are very differents
entities.  Types are static entities that do not exist at runtime,
while objects are dynamic entities that do not exist at compile time.

Prototype-based object programing is different: the difference between
classes and objects, between types and values, is blurred. Instead,
you have an
object, that is already an instance, and that you might clone to
obtain a new one that you can modify afterward. Prototype-based
programming was introduced by the Self language, and is used in
several popular script languages such as \io or \js.

Class-based programming can be considered with an industrial
metaphore: classes are moulds, from which objects are generated.
Prototype-based programming is more biological: an prototype object is
cloned into another object which which can be modified during its
lifetime.

Consider pairs for instance (see \refObject{Pair}). Pairs hold two
values, \lstinline|first| and \lstinline|second|, like an
\lstinline{std::pair} in \Cxx. Since \us is prototype-based, there is
no pair class. Instead, \lstinline|Pair| is really a pair (object).

\begin{urbiscript}
Pair;
[00000000] (nil, nil)
\end{urbiscript}

We can see here that \lstinline|Pair| is a pair whose two values are
equal to \lstinline|nil| --- which is a reasonable default value. To
get a pair of our own, we simply clone \lstinline|Pair|.  We can then
use it as a regular pair.

\begin{urbiscript}
var p = Pair.clone;
[00000000] (nil, nil)
p.first = "101010";
[00000000] "101010"
p.second = true;
[00000000] true
p;
[00000000] ("101010", true)
Pair;
[00000000] (nil, nil)
\end{urbiscript}

Since \lstinline|Pair| is a regular pair object, you can modify and
use it at will. Yet this is not a good idea, since you will alter your
base prototype, which alters any derivative, future and even past.

\begin{urbiscript}
var before = Pair.clone;
[00000000] (nil, nil)
Pair.first = false;
[00000000] false
var after = Pair.clone;
[00000000] (false, nil)
before;
[00000000] (false, nil)
// before and after share the same first: that of Pair.
assert(Pair.first === before.first);
assert(Pair.first === after.first);
\end{urbiscript}

\section{Prototypes and slot lookup}

In prototype-based language, 'is-a' relations (being an instance of
some type) and inheritance relations (extending another type) are
simplified in a single relation: prototyping. You can inspect an
object prototypes with the \lstinline{protos} method.

\begin{urbiscript}
var p = Pair.clone;
[00000000] (nil, nil)
p.protos;
[00000000] [(nil, nil)]
\end{urbiscript}

As expected, our fresh pair has one prototype, \lstinline|(nil, nil)|,
which is how \lstinline|Pair| displays itself. We can check this as
presented below.

\begin{urbiscript}[firstnumber=last]
// List.head returns the first element.
p.protos.head;
[00000000] (nil, nil)
// Check that the prototype is really Pair.
p.protos.head === Pair;
[00000000] true
\end{urbiscript}

Prototypes are the base of the slot lookup mechanism. Slot lookup is
the action of finding an object slot when \lstinline{getSlot} or the
dot notation is used. So far, when we typed
\lstinline|\var{obj}.\var{slot}|, \var{slot} was always a slot of
\var{obj}.  Yet, this call can be valid even if \var{obj} has no
\var{slot} slot, because slots are also looked up in prototypes. For
instance, \lstinline|p|, our clone of \lstinline|Pair|, has no
\lstinline|first| or \lstinline|second| slots. Yet,
\lstinline|p.first| and \lstinline|p.second| work, because these slots
are present in \lstinline|Pair|, which is \lstinline|p|'s
prototype. This is illustrated below.

\begin{urbiscript}
var p = Pair.clone;
[00000000] (nil, nil)
// p has no slots of its own.
p.localSlotNames;
[00000000] []
// Yet this works.
p.first;
// This is because p has Pair for prototype, and Pair has a 'first' slot.
p.protos.head === Pair;
[00000000] true
"first" in Pair.localSlotNames && "second" in Pair.localSlotNames;
[00000000] true
\end{urbiscript}

As shown here, the \lstinline{clone} method simply creates an empty
object, with its target as prototype. The new object has the exact
same behavior as the cloned on thanks to slot lookup.

Let's experience slot lookup by ourselves. In \us, you can add and
remove prototypes from an object thanks to \lstinline{addProto} and
\lstinline{removeProto}.

\begin{urbiscript}
// We create a fresh object.
var c = Object.clone;
[00000000] Object_0x1
// As expected, it has no 'slot' slot.
c.slot;
[00000000:error] !!! lookup failed: slot
var p = Object.clone;
[00000000] Object_0x2
var p.slot = 0;
[00000000] 0
c.addProto(p);
[00000000] Object_0x1
// Now, 'slot' is found in c, because it is inherited from p.
c.slot;
[00000000] 0
c.removeProto(p);
[00000000] Object_0x1
// Back to our good old lookup error.
c.slot;
[00000000:error] !!! lookup failed: slot
\end{urbiscript}

The slot lookup algorithm in \us in a depth-first traversal of the
object prototypes tree. Formally, when the \var{s} slot is requested
from \var{x}:

\begin{itemize}
\item If \var{x} itself has the slot, the requested value is found.
\item Otherwise, the same lookup algorithm is applied on all
  prototypes, most recent first.
\end{itemize}

Thus, slots from the last prototype added take precedence over other
prototype's slots.

\begin{urbiscript}
var proto1 = Object.clone;
[00000000] Object_0x10000000
var proto2 = Object.clone;
[00000000] Object_0x20000000
var o = Object.clone;
[00000000] Object_0x30000000
o.addProto(proto1);
[00000000] Object_0x30000000
o.addProto(proto2);
[00000000] Object_0x30000000
// We give o an x slot through proto1.
var proto1.x = 0;
[00000000] 0
o.x;
[00000000] 0
// proto2 is visited first during lookup.
// Thus its "x" slot takes precedence over proto1's.
var proto2.x = 1;
[00000000] 1
o.x;
[00000000] 1
// Of course, o's own slots have the highest precedence.
var o.x = 2;
[00000000] 2
o.x;
[00000000] 2
\end{urbiscript}

You can check where in the prototype hierarchy a slot is found with
the \lstinline{locateSlot} method. This is a very
handful tool when inspecting an object.

\begin{urbiscript}
var p = Pair.clone;
[00000000] (nil, nil)
// Check that the 'first' slot is found in Pair
p.locateSlot("first") === Pair;
[00000000] true
// Where does locateSlot itself come from? Object itself!
p.locateSlot("locateSlot");
[00000000] Object
\end{urbiscript}

The prototype model is rather simple: creating a fresh object simply
consists in cloning a model object, a prototype, that was provided to
you.  Moreover, you can add behavior to an object at any time with a
simple \lstinline{addProto}: you can make any object a fully
functional \lstinline|Pair| with a simple
\lstinline|myObj.addProto(Pair)|.

\section{Copy on write}

One point might be bothering you though: what if you want to update a
slot value in a clone of your prototype?

Say we implement a simple prototype, with an \var{x} slot equal to
\lstinline|0|, and clone it twice. We have three object with an
\var{x} slot, yet only one actual \lstinline|0| integer. Will
modifying \var{x} in one of the clone change the prototype's \var{x},
thus altering the prototype and the other clone as well?

The answer is, of course, no, as illustrated below.

\begin{urbiscript}
var proto = Object.clone;
[00000000] Object_0x1
var proto.x = 0;
[00000000] 0
var o1 = proto.clone;
[00000000] Object_0x2
var o2 = proto.clone;
[00000000] Object_0x3
// Are we modifying proto's x slot here?
o1.x = 1;
[00000000] 1
// Obviously not
o2.x;
[00000000] 0
proto.x;
[00000000] 0
o1.x;
[00000000] 1
\end{urbiscript}

This work thanks to copy-on-write: slots are first duplicated to the
local object when they're updated, as we can check below.

\begin{urbiscript}[firstnumber=last]
// This is the continuation of previous example.

// As expected, o2 finds "x" in proto
o2.locateSlot("x") === proto;
[00000000] true
// Yet o1 doesn't anymore
o1.locateSlot("x") === proto;
[00000000] false
// Because the slot was duplicated locally
o1.locateSlot("x") === o1;
[00000000] true
\end{urbiscript}

This is why, when we cloned Pair earlier, and modified the ``first''
slot of our fresh Pair, we didn't alter Pair one all its other clones.

% BIG FIXME: classes, classes ... c'mon, these are not classes but
% prototypes ...
\section{Defining classes}

Now that we know the internals of \us's object model, we can start
defining our own classes. For instance, we can define our own
\lstinline{Pair} class. We just have to create a pair, with its first and
second slots. For this we use the \lstinline{do} scope described in
\autoref{section:constructs/do}. The listing below defines a new
\lstinline{Pair} class. The \lstinline{asString} function is simply used to
customize pairs printing - don't give it too much attention for now.

\begin{urbiscript}
var MyPair = Object.clone;
[00000000] Object_0x1
do (MyPair)
{
  var first = nil;
  var second = nil;
  function asString ()
  {
    "MyPair: " + first + ", " + second
  };
} | {};
// We just defined a pair
MyPair;
[00000000] MyPair: nil, nil
// Let's try it out
var p = MyPair.clone;
[00000000] MyPair: nil, nil
p.first = 0;
[00000000] 0
p;
[00000000] MyPair: 0, nil
MyPair;
[00000000] MyPair: nil, nil
\end{urbiscript}

That's it, we defined a pair that can be cloned at will! \us
provides a shorthand to define classes as we did above: the
\lstinline{class} keyword.

\begin{urbiscript}
class MyPair
{
  var first = nil;
  var second = nil;
  function asString() { "(" + first + ", " + second + ")"; };
};
[00000000] (nil, nil)
\end{urbiscript}

The \lstinline{class} keyword simply creates MyPair with Object.clone,
and provides you with a \lstinline|do (MyPair)| scope. It actually also
predefines a few slots, but this is not the point here.

\section{Constructors}
\label{sec:tut:ctor}
As we've seen, we can use the clone method on any object to obtain an
identical object. Yet, some class provide more elaborated
constructors, accessible by calling \lstinline{new} instead of
\lstinline{clone}, potentially passing it arguments.

\begin{urbiscript}
var p = Pair.new("foo", false);
[00000000] ("foo", false)
\end{urbiscript}

While \lstinline{clone} guarantees you obtain a empty fresh object
inheriting from the prototype, \lstinline{new} behavior is left to the
discretion of the cloned prototype --- although its behavior is the
same as \lstinline{clone} by default.

To define such constructors, prototypes only need to provide an
\lstinline{init} method, that will be called with the arguments given to
new. For instance, we can improve our previous \lstinline{Pair} class
with a constructor.

\begin{urbiscript}
class MyPair
{
  var first = nil;
  var second = nil;
  function init(f, s) { first = f;   second = s;  };
  function asString() { "(" + first + ", " + second + ")"; };
};
[00000000] (nil, nil)
MyPair.new(0, 1);
[00000000] (0, 1)
\end{urbiscript}

%\section{Conclusion}

%%% Local Variables:
%%% mode: latex
%%% TeX-master: "../urbi-sdk"
%%% End:
