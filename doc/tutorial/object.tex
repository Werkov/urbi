%% Copyright (C) 2009-2012, Gostai S.A.S.
%%
%% This software is provided "as is" without warranty of any kind,
%% either expressed or implied, including but not limited to the
%% implied warranties of fitness for a particular purpose.
%%
%% See the LICENSE file for more information.

\chapter{Objective Programming, \us Object Model}
\label{sec:tut:object}

This section presents object programming in \us: the prototype-based
object model of \us, and how to define and use classes.

\section{Prototype-Based Programming in \us}

You're probably already familiar with class-based object programming, since
this is the \Cxx, \Java, \Cs model.  Classes and objects are very different
entities.  Classes and types are static entities that do not exist at
run-time, while objects are dynamic entities that do not exist at compile
time.

Prototype-based object programming is different: the difference between
classes and objects, between types and values, is blurred. Instead, you have
an object, that is already an instance, and that you might clone to obtain a
new one that you can modify afterward.  Prototype-based programming was
introduced by the Self language, and is used in several popular script
languages such as \Io or \Js.

Class-based programming can be considered with an industrial metaphor:
classes are molds, from which objects are generated.  Prototype-based
programming is more biological: a prototype object is cloned into another
object which can be modified during its lifetime.

Consider pairs for instance (see \refObject{Pair}). Pairs hold two values,
\lstinline|first| and \lstinline|second|, like an \lstinline{std::pair} in
\Cxx. Since \us is prototype-based, there is no pair class. Instead,
\lstinline|Pair| is really a pair (object).

\begin{urbiscript}[firstnumber=1]
Pair;
[00000000] (nil, nil)
\end{urbiscript}

We can see here that \lstinline|Pair| is a pair whose two values are equal
to \lstinline|nil| --- which is a reasonable default value. To get a pair of
our own, we simply clone \lstinline|Pair|.  We can then use it as a regular
pair.

\begin{urbiscript}
var p = Pair.clone();
[00000000] (nil, nil)
p.first = "101010";
[00000000] "101010"
p.second = true;
[00000000] true
p;
[00000000] ("101010", true)
Pair;
[00000000] (nil, nil)
\end{urbiscript}

Since \lstinline|Pair| is a regular pair object, you can modify and use it
at will. Yet this is not a good idea, since you will alter your base
prototype, which alters any derivative, future and even past.

\begin{urbiscript}
var before = Pair.clone();
[00000000] (nil, nil)
Pair.first = false;
[00000000] false
var after = Pair.clone();
[00000000] (false, nil)
before;
[00000000] (false, nil)
// before and after share the same first: that of Pair.
assert(Pair.first === before.first);
assert(Pair.first === after.first);
\end{urbiscript}
\begin{urbicomment}
Pair.first = nil;
\end{urbicomment}
\section{Prototypes and Slot Lookup}

In prototype-based language, \emph{is-a} relations (being an instance of
some type) and inheritance relations (extending another type) are simplified
in a single relation: prototyping. You can inspect an object prototypes with
the \lstinline{protos} method.

\begin{urbiscript}[firstnumber=1]
var p = Pair.clone();
[00000000] (nil, nil)
p.protos;
[00000000] [(nil, nil)]
\end{urbiscript}

As expected, our fresh pair has one prototype, \lstinline|(nil, nil)|, which
is how \lstinline|Pair| displays itself. We can check this as presented
below.

\begin{urbiscript}
// List.head returns the first element.
p.protos.head();
[00000000] (nil, nil)
// Check that the prototype is really Pair.
p.protos.head() === Pair;
[00000000] true
\end{urbiscript}

Prototypes are the base of the slot lookup mechanism. Slot lookup is the
action of finding an object slot when the dot notation is used.  So far,
when we typed \lstinline|\var{obj}.\var{slot}|, \var{slot} was always a slot
of \var{obj}.  Yet, this call can be valid even if \var{obj} has no
\var{slot} slot, because slots are also looked up in prototypes. For
instance, \lstinline|p|, our clone of \lstinline|Pair|, has no
\lstinline|first| or \lstinline|second| slots. Yet, \lstinline|p.first| and
\lstinline|p.second| work, because these slots are present in
\lstinline|Pair|, which is \lstinline|p|'s prototype. This is illustrated
below.

\begin{urbiscript}[firstnumber=1]
var p = Pair.clone();
[00000000] (nil, nil)
// p has no slots of its own.
p.localSlotNames();
[00000000] []
// Yet this works.
p.first;
// This is because p has Pair for prototype, and Pair has a 'first' slot.
p.protos.head() === Pair;
[00000000] true
"first" in Pair.localSlotNames() && "second" in Pair.localSlotNames();
[00000000] true
\end{urbiscript}

As shown here, the \lstinline{clone} method simply creates an empty object,
with its target as prototype. The new object has the exact same behavior as
the cloned on thanks to slot lookup.

Let's experience slot lookup by ourselves. In \us, you can add and remove
prototypes from an object thanks to \lstinline{addProto} and
\lstinline{removeProto}.

\begin{urbiscript}[firstnumber=1]
// We create a fresh object.
var c = Object.clone();
[00000000] Object_0x1
// As expected, it has no 'slot' slot.
c.slot;
[00000000:error] !!! lookup failed: slot
var p = Object.clone();
[00000000] Object_0x2
var p.slot = 0;
[00000000] 0
c.addProto(p);
[00000000] Object_0x1
// Now, 'slot' is found in c, because it is inherited from p.
c.slot;
[00000000] 0
c.removeProto(p);
[00000000] Object_0x1
// Back to our good old lookup error.
c.slot;
[00000000:error] !!! lookup failed: slot
\end{urbiscript}

The slot lookup algorithm in \us in a depth-first traversal of the object
prototypes tree. Formally, when the \var{s} slot is requested from \var{x}:

\begin{itemize}
\item If \var{x} itself has the slot, the requested value is found.
\item Otherwise, the same lookup algorithm is applied on all prototypes,
  most recent first.
\end{itemize}

Thus, slots from the last prototype added take precedence over other
prototype's slots.

\begin{urbiscript}[firstnumber=1]
var proto1 = Object.clone();
[00000000] Object_0x10000000
var proto2 = Object.clone();
[00000000] Object_0x20000000
var o = Object.clone();
[00000000] Object_0x30000000
o.addProto(proto1);
[00000000] Object_0x30000000
o.addProto(proto2);
[00000000] Object_0x30000000
// We give o an x slot through proto1.
var proto1.x = 0;
[00000000] 0
o.x;
[00000000] 0
// proto2 is visited first during lookup.
// Thus its "x" slot takes precedence over proto1's.
var proto2.x = 1;
[00000000] 1
o.x;
[00000000] 1
// Of course, o's own slots have the highest precedence.
var o.x = 2;
[00000000] 2
o.x;
[00000000] 2
\end{urbiscript}

You can check where in the prototype hierarchy a slot is found with the
\lstinline{locateSlot} method. This is a very handful tool when inspecting
an object.

\begin{urbiscript}[firstnumber=1]
var p = Pair.clone();
[00000000] (nil, nil)
// Check that the 'first' slot is found in Pair
p.locateSlot("first") === Pair;
[00000000] true
// Where does locateSlot itself come from? Object itself!
p.locateSlot("locateSlot");
[00000000] Object
\end{urbiscript}

The prototype model is rather simple: creating a fresh object simply
consists in cloning a model object, a prototype, that was provided to you.
Moreover, you can add behavior to an object at any time with a simple
\lstinline{addProto}: you can make any object a fully functional
\lstinline|Pair| with a simple \lstinline|myObj.addProto(Pair)|.

\section{Copy on Write}

One point might be bothering you though: what if you want to update a slot
value in a clone of your prototype?

Say we implement a simple prototype, with an \var{x} slot equal to
\lstinline|0|, and clone it twice. We have three objects with an \var{x}
slot, yet only one actual \lstinline|0| integer. Will modifying \var{x} in
one of the clone change the prototype's \var{x}, thus altering the prototype
and the other clone as well?

The answer is, of course, no, as illustrated below.

\begin{urbiscript}[firstnumber=1]
var proto = Object.clone();
[00000000] Object_0x1
var proto.x = 0;
[00000000] 0
var o1 = proto.clone();
[00000000] Object_0x2
var o2 = proto.clone();
[00000000] Object_0x3
// Are we modifying proto's x slot here?
o1.x = 1;
[00000000] 1
// Obviously not
o2.x;
[00000000] 0
proto.x;
[00000000] 0
o1.x;
[00000000] 1
\end{urbiscript}

This work thanks to \dfn{copy-on-write}: slots are first duplicated to the
local object when they're updated, as we can check below.

\begin{urbiscript}
// This is the continuation of previous example.

// As expected, o2 finds "x" in proto
o2.locateSlot("x") === proto;
[00000000] true
// Yet o1 doesn't anymore
o1.locateSlot("x") === proto;
[00000000] false
// Because the slot was duplicated locally
o1.locateSlot("x") === o1;
[00000000] true
\end{urbiscript}

This is why, when we cloned Pair earlier, and modified the ``first'' slot of
our fresh Pair, we didn't alter Pair one all its other clones.

\section{Defining Pseudo-Classes}
\label{sec:tut:class}

Now that we know the internals of \us's object model, we can start defining
our own classes.

But wait, we just said there are no classes in prototype-based
object-oriented languages!  That is true: there are no classes in the sense
of \Cxx, i.e., compile-time entities that are not objects.  Instead,
prototype-based languages rely on the existence of a canonical object (the
\dfn{prototype}) from which (pseudo) \emph{instances} are derived.  Yet,
since the syntactic inspiration for \us comes from languages such as \Java,
\Cxx and so forth, it is nevertheless the \lstinline|class| keyword that is
used to define the pseudo-classes, i.e., prototypes.

As an example, we define our own \lstinline{Pair} class. We just have to
create a pair, with its \lstinline|first| and \lstinline|second| slots. For
this we use the \lstinline{do} scope described in
\autoref{sec:constructs:do}. The listing below defines a new
\lstinline{Pair} class. The \lstinline{asString} function is simply used to
customize pairs printing --- don't give it too much attention for now.

\begin{urbiscript}[firstnumber=1]
var MyPair = Object.clone();
[00000000] Object_0x1
do (MyPair)
{
  var first = nil;
  var second = nil;
  function asString ()
  {
    "MyPair: " + first + ", " + second
  };
}|;
// We just defined a pair
MyPair;
[00000000] MyPair: nil, nil
// Let's try it out
var p = MyPair.clone();
[00000000] MyPair: nil, nil
p.first = 0;
[00000000] 0
p;
[00000000] MyPair: 0, nil
MyPair;
[00000000] MyPair: nil, nil
\end{urbiscript}

That's it, we defined a pair that can be cloned at will! \us provides a
shorthand to define classes as we did above: the \lstinline{class} keyword.

\begin{urbiscript}[firstnumber=1]
class MyPair
{
  var first = nil;
  var second = nil;
  function asString() { "(" + first + ", " + second + ")"; };
};
[00000000] (nil, nil)
\end{urbiscript}

The \lstinline{class} keyword simply creates \lstinline|MyPair| with
\lstinline|Object.clone|, and provides you with a \lstinline|do (MyPair)|
scope. It actually also pre-defines a few slots, but this is not the point
here.

It is also possible to specify a proto for the newly created ``class'',
using the same syntax as \Java and \Cxx:

\begin{urbiscript}
class Top
{
  var top = "top";
};
[00000000] Top

class Bottom : Top
{
  var bottom = "bottom";
};
[00000000] Bottom

Bottom.new().top;
[00000000] "top"
\end{urbiscript}

For more details, see \autoref{sec:lang:class}.

\section{Constructors}
\label{sec:tut:ctor}
As we've seen, we can use the \lstinline|clone| method on any object to
obtain an identical object. Yet, some classes provide more elaborate
constructors, accessible by calling \lstinline{new} instead of
\lstinline{clone}, potentially passing arguments.

\begin{urbiscript}[firstnumber=1]
var p = Pair.new("foo", false);
[00000000] ("foo", false)
\end{urbiscript}

While \lstinline{clone} guarantees you obtain an empty fresh object
inheriting from the prototype, \lstinline{new} behavior is left to the
discretion of the cloned prototype --- although its behavior is the same as
\lstinline{clone} by default.

To define such constructors, prototypes only need to provide an
\lstinline{init} method, that will be called with the arguments given to
new. For instance, we can improve our previous \lstinline{Pair} class with a
constructor.

\begin{urbiscript}[firstnumber=1]
class MyPair
{
  var first = nil;
  var second = nil;
  function init(f, s) { first = f;   second = s;  };
  function asString() { "(" + first + ", " + second + ")"; };
};
[00000000] (nil, nil)
MyPair.new(0, 1);
[00000000] (0, 1)
\end{urbiscript}

\section{Operators}
\label{sec:tut:operators}

In \us, operators such as \lstinline|+|, \lstinline|&&| and others, are
regular functions that benefit from a bit of syntactic sugar.  To be more
precise, \lstinline|\var{a}+\var{b}| is exactly the same as
\lstinline|\var{a}.'+'(\var{b})|.  The rules to resolve slot names apply
too, i.e., the \lstinline|'+'| slot is looked for in \var{a}, then in its
prototypes.

The following example provides arithmetic between pairs.

\begin{urbiscript}
class ArithPair
{
  var first = nil;
  var second = nil;
  function init(f, s) { first = f;   second = s;  };
  function asString() { "(" + first + ", " + second + ")"; };
  function '+'(rhs) { new(first + rhs.first, second + rhs.second); };
  function '-'(rhs) { new(first - rhs.first, second - rhs.second); };
  function '*'(rhs) { new(first * rhs.first, second * rhs.second); };
  function '/'(rhs) { new(first / rhs.first, second / rhs.second); };
};
[00000000] (nil, nil)
ArithPair.new(1, 10) + ArithPair.new(2, 20) * ArithPair.new(3, 30);
[00000000] (7, 610)
\end{urbiscript}

\section{Properties}
\label{sec:tut:prop}

Sometimes one needs to attach attributes to a variable, and not the value
it contains, for instance to store whether the variable is constant or not.
In \us we use objects named \refObject{Slot}s that stands before the actual
value, and persist when a new value is assigned to the variable. Variables of
the \refObject{Slot} are called \lstinline{properties}.

\subsection{Features of Values}
In following example, we attach some random slot \lstinline|foo| to the
value pointed to by the slot \lstinline|x|.

\begin{urbiscript}
var x = 123;
[00000000] 123
var x.foo = 42;
[00000000] 42
\end{urbiscript}

If \lstinline|y| is another slot to the value of \lstinline|x|, then it
provides the same \lstinline|foo| feature:

\begin{urbiscript}
var y = x;
[00000000] 123
y.foo;
[00000000] 42
// The value in the slots x and y are the same object
x===y;
[00000000] true
x.foo = 43 | y.foo;
[00000001] 43
\end{urbiscript}

If \lstinline|x| is bound to a new object (e.g., 456), then the feature
\lstinline|foo| is no longer present, since it's a feature of the
\emph{value} (i.e., 123), and not one of the slot (i.e., \lstinline|x|).

\begin{urbiscript}
x = 456;
[00000000] 456
x.foo;
[00000000:error] !!! lookup failed: foo
\end{urbiscript}

Of course, \lstinline|y|, which is still linked to the original value (123),
answers to queries to \lstinline|foo|.

\begin{urbiscript}
y.foo;
[00000000] 43
\end{urbiscript}

\subsection{Features of Slots}

If, on the contrary you want to attach a feature to the slot-as-a-name,
rather than to the value it contains, use the \dfn{properties}.  The syntax
is \lstinline|\var{slotName}->\var{propertyName}|.

\begin{urbiscript}
x = 123;
[00000000] 123
x->foo = 42;
[00000000] 42
x->foo;
[00000000] 42
\end{urbiscript}

Copying the value contained by a slot does \emph{not} propagate the
properties of the slot:

\begin{urbiscript}
y = x;
[00000000] 123
y->foo;
[00000000:error] !!! property lookup failed: y->foo
\end{urbiscript}

And if you assign a new value to a slot, the properties of the slot are
preserved:

\begin{urbiscript}
x = 456;
[00000000] 456
x->foo = 42;
[00000000] 42
\end{urbiscript}

\section{Getters and setters}
\label{sec:tut:getter}

All the properties with a special meaning are described in \refObject{Slot}.
Two of particular interest are \lstinline{oget} and \lstinline{oset}. They can
be used to define a getter and a setter function, which are called each time
the Slot is read or written to, respectively, thus implementing the
\lstinline{property} semantics as defined in \Js. \us also borrowed the \Js
shortcut syntax \lstinline{get foo} and \lstinline{set foo} to define setter
and getter properties.

Consider this example:

\begin{urbiscript}
class Vector
{
  var x=0;
  var y=0;
  function init(x=0, y=0)
  {
    this.x = x; // Here we use copy on write to create our own slot x.
    var this.y = y; // Here we force local slot creation, both are valid.
  };
  // Return the L2 norm of the vector
  get norm() // sugar for norm->oget = function()
  {
    (x*x+y*y).sqrt()
  };
  // Scale the vector so that its norm becomes newNorm
  set norm(newNorm) // sugar for norm->oset = function(newNorm)
  {
    var oldNorm = norm;
    x *= newNorm/oldNorm;
    y *= newNorm/oldNorm;
  };
  function asString()
  {
    "<" + x + "," + y + ">"
  };
};
[00000000] <0,0>
var v = Vector.new(1, 1);
[00016149] <1,1>
v.norm;
[00018036] 1.41421
v.norm = 2.sqrt() | v;
[00177142] <1,1>
\end{urbiscript}

If you use a setter without a getter, the value returned by your function
will be stored if it is not void, as demonstrated in
\refSlot[Slot]{set}.

\subsection{When to use properties?}

The \lstinline{c#} documentation provides a very good explanation of when to use
properties that we took the liberty to reproduce verbatim here:

In most cases, properties represent data, and methods perform actions.
Properties are accessed like fields, which makes them easier to use.
If a method takes no arguments and returns an object's state information,
or accepts a single argument to set some part of an object's state,
it is a good candidate for becoming a property.

Properties should behave as if they are fields; if the method cannot,
it should not be changed to a property.
Methods are preferable to properties in the following situations:

\begin{itemize}
\item The method performs a time-consuming operation. The method is perceivably slower than the time it takes to set or get a field's value.
\item The method performs a conversion. Accessing a field does not return a converted version of the data it stores.
\item The "Get" method has an observable side effect. Retrieving a field's value does not produce any side effects.
\item The order of execution is important. Setting the value of a field does not rely on other operations having occurred.
\item Calling the method twice in succession creates different results.
The method is static but returns an object that can be changed by the caller. Retrieving a field's value does not allow the caller to change the data stored by the field.

\end{itemize}
%%% Local Variables:
%%% coding: utf-8
%%% mode: latex
%%% TeX-master: "../urbi-sdk"
%%% ispell-dictionary: "american"
%%% ispell-personal-dictionary: "../urbi.dict"
%%% fill-column: 76
%%% End:
