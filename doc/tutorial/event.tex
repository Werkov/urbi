\chapter{Event-based Programming}
\label{sec:tut:event-prog}

When dealing with highly interactive agent programming, sequential
programming is inconvenient. We want to react to external, random
events, not execute code linearly with a predefined flow. \us has a
strong support for event-based programming.

\section{Event related constructs}

The first construct we will study is the \lstinline|at| keyword. Given
a condition, and an expression, \lstinline|at| will evaluate the
expression every time the condition becomes true. That is,
when a rising edge occurs on the condition.

\begin{urbiscript}
var x = 0;
[00000000] 0
at (x > 5)
  echo("ping");
x = 5;
[00000000] 5
// This triggers the event
x = 6;
[00000000] 6
[00000000] *** ping
// Does not trigger, since the condition is already true.
x = 7;
[00000000] 7
// The condition becomes false here.
x = 3;
[00000000] 3

x = 10;
[00000000] 10
[00000000] *** ping
\end{urbiscript}

An \lstinline|onleave| block can be appended to execute an expression
when the expression \emph{becomes} false --- that is, on falling edges.

\begin{urbiscript}
var x = false;
[00000000] false
at (x)
  echo("x")
onleave
  echo("!x");
x = true;
[00000000] true
[00000000] *** x
x = false;
[00000000] false
[00000000] *** !x
\end{urbiscript}

See \autoref{sec:lang:at} for more details on \lstinline|at|
statements.

The \lstinline|whenever| construct is similar to \lstinline|at|,
except the expression evaluation is systematically restarted when it
finishes as long as the condition stands true.

\begin{urbiscript}
var x = false;
[00000000] false
whenever (x)
{
  echo("ping");
  sleep(1s);
};
x = true;
[00000000] true
sleep(3s);
// Whenever keeps triggering
[00000000] *** ping
[00001000] *** ping
[00002000] *** ping
x = false;
[00002000] false
// Whenever stops triggering
\end{urbiscript}

Just like \lstinline|at| has \lstinline|onleave|, \lstinline|whenever|
has \lstinline|else|: the given expression is evaluated as long as the
condition is false.

\begin{urbiscript}
var x = false;
[00002000] false
whenever (x)
{
  echo("ping");
  sleep(1s);
}
else
{
  echo("pong");
  sleep(1s);
};
sleep (3s);
[00000000] *** pong
[00001000] *** pong
[00002000] *** pong
x = true;
[00003000] true
sleep (3s);
[00003000] *** ping
[00004000] *** ping
[00005000] *** ping
x = false;
[00006000] false
sleep (2s);
[00006000] *** pong
[00007000] *** pong
\end{urbiscript}

\section{Events}
\label{sec:tut:events}
\us enables you to define events, that can be caught with the
\lstinline|at| and \lstinline|whenever| constructs we saw earlier. You
can create events by cloning the \lstinline|Event| prototype. They can
then be emitted with the \lstinline|!| keyword.

\begin{urbiscript}
var myEvent = Event.new;
[00000000] Event_0x0
at (myEvent?)
  echo("ping");
myEvent!;
[00000000] *** ping
// events work well with parallelism
myEvent! & myEvent!;
[00000000] *** ping
[00000000] *** ping
\end{urbiscript}

Both \lstinline|at| and \lstinline|whenever| have the same behavior on
punctual events. However, if you emit an event for a given duration,
\lstinline|whenever| will keep triggering for this duration, contrary
to \lstinline|at|.

\begin{urbiscript}
var myEvent = Event.new;
[00000000] Event_0x0
whenever (myEvent?)
{
  echo("ping (whenever)")|
  sleep(200ms)
};
at (myEvent?)
{
  echo("ping (at)")|
  sleep(200ms)
};
// Emit myEvent for .3 second.
myEvent! ~ 300ms;
[00000000] *** ping (at)
[00000000] *** ping (whenever)
[00000100] *** ping (whenever)
\end{urbiscript}

%\section{Valued events}

%%% Local Variables:
%%% mode: latex
%%% TeX-master: "../urbi-sdk"
%%% End:
