%% Copyright (C) 2009-2012, Gostai S.A.S.
%%
%% This software is provided "as is" without warranty of any kind,
%% either expressed or implied, including but not limited to the
%% implied warranties of fitness for a particular purpose.
%%
%% See the LICENSE file for more information.

\chapter{Event-based Programming}
\label{sec:tut:event-prog}

When dealing with highly interactive agent programming, sequential
programming is inconvenient. We want to react to external, random
events, not execute code linearly with a predefined flow. \us has a
strong support for event-based programming.

\section{Watchdog constructs}

The first construct we will study uses the \lstinline|at| keyword. Given a
condition and a statement, \lstinline|at| will evaluate the statement each
time the condition \emph{becomes} true. That is, when a rising edge occurs
on the condition.

\begin{urbiscript}[firstnumber=1]
var x = 0;
[00000000] 0
at (x > 5)
  echo("ping");
x = 5;
[00000000] 5
// This triggers the event.
x = 6;
[00000000] 6
[00000000] *** ping
// Does not trigger, since the condition is already true.
x = 7;
[00000000] 7
// The condition becomes false here.
x = 3;
[00000000] 3

x = 10;
[00000000] 10
[00000000] *** ping
\end{urbiscript}

An \lstinline|onleave| block can be appended to execute an expression when
the expression \emph{becomes} false --- that is, on falling edges.

\begin{urbiscript}[firstnumber=1]
var x = false;
[00000000] false
at (x)
  echo("x")
onleave
  echo("!x");
x = true;
[00000000] true
[00000000] *** x
x = false;
[00000000] false
[00000000] *** !x
\end{urbiscript}

See \autoref{sec:lang:at} for more details on \lstinline|at| statements.



\section{Events}
\label{sec:tut:events}

In addition to monitoring an expression with a watchdog, \us enables you to
define events that can be caught with the \lstinline|at| construct we saw earlier. You can create events by
instantiating the \refObject{Event} prototype. They can then be emitted with
the \lstinline|!| keyword.

\subsection{Emitting Events}

\begin{urbiscript}[firstnumber=1]
var myEvent = Event.new();
[00000000] Event_0xb5579008
at (myEvent?)
  echo("ping");
myEvent!;
[00000000] *** ping
// events work well with parallelism
myEvent! & myEvent!;
[00000000] *** ping
[00000000] *** ping
\end{urbiscript}

\lstinline|myEvent!| triggers an instantaneous event, which has no effect on
\lstinline|at| registered after it:

\begin{urbiscript}[firstnumber=1]
var ev = Event.new();
[00000000] Event_0xb5579008
ev!;
at(ev?) echo("ping");
// Nothing happens unless ev is emitted again.
\end{urbiscript}

Events can be emited for a given duration using the \lstinline|~(duration)|
construct after the \lstinline|!| :

\begin{urbiscript}[firstnumber=1]
var event = Event.new();
[00000000] Event_0xb5579008
// emit event for 1 second in parallel with the scope
event!~(1s),
// Use an at 100ms later, the event is still in triggered state.
sleep(100ms) | at(event?) echo("ping");
sleep(1s);
[00000000] *** ping
\end{urbiscript}

Note in the example above that \lstinline|at| only triggers once for the whole
duration of the emit.

\subsection{Emitting events with a payload}
\label{sec:tut:events:payload}

Events behave very much like ``channels'': listeners use \lstinline|at|,
and producers use \lstinline|!|.  Fortunately, the
messages can include a \dfn{payload}, i.e., something sent in the
``message''.  The Event then behaves very much like an identifier of the
message type.  To send/catch the payload, just pass arguments to
\lstinline|!| and \lstinline|?|:

\begin{urbiscript}[firstnumber=1]
var event = Event.new();
[00000000] Event_0x00000001

at (event?(var payload))
  echo("received: " + payload)
onleave
  echo("had received: " + payload);

event!(1);
[00000008] *** received: 1
[00000009] *** had received: 1

event!(["string", 124]);
[00000010] *** received: ["string", 124]
[00000011] *** had received: ["string", 124]
\end{urbiscript}

Like functions, events have an \dfn{arity}, i.e., they depend on the number
of arguments: \lstinline|at (\var{event}?(\var{arg}))| will only match
emissions whose payload contain exactly one argument, i.e.,
\lstinline|\var{event}!(\var{arg})|.

\begin{urbiscript}
// Too many arguments.
event!(1, 2);

// Not enough arguments.
event!;
event!();
\end{urbiscript}

Event handlers that do not specify their arity (i.e., without parentheses)
match event emissions of any arity.

\begin{urbiscript}
at (event?)
  echo("received an event")
onleave
  echo("had received an event");

event!;
[00000014] *** received an event
[00000015] *** had received an event

event!(1);
[00000016] *** received: 1
[00000017] *** had received: 1
[00000018] *** received an event
[00000019] *** had received an event

event!(1, 2);
[00000020] *** received an event
[00000021] *** had received an event
\end{urbiscript}

Actually, the feature is much more powerful than this: full pattern matching
applies, as with the \lstinline|switch|/\lstinline|case| construct.

\begin{urbiscript}
var e = Event.new()|;

at (e?)
  echo("e");

at (e?(var x))
  echo("e(x)");

at (e?(1))
  echo("e(1)");

at (e?(var x) if x.isA(Float) && x % 2)
  echo("e(odd)");

// Payload must be a list of three members, the first two being 1 and 2, and
// the third one being greater than 2, when converted as a Float.
at (e?([1, 2, var x]) if 2 < x.asFloat())
  echo("e([1, 2, x = %s])" % x);

e!;
[00000845] *** e

e!(0);
[00011902] *** e
[00011902] *** e(x)

e!(1);
[00023327] *** e
[00023327] *** e(x)
[00023327] *** e(1)
[00023327] *** e(odd)

e!([1, 2, 1]);
[00024327] *** e
[00024327] *** e(x)

e!([1, 2, 3]);
[00025327] *** e
[00025327] *** e(x)
[00025327] *** e([1, 2, x = 3])

e!([1, 2, "4"]);
[00026327] *** e
[00026327] *** e(x)
[00026327] *** e([1, 2, x = 4])
\end{urbiscript}

Remember \lstinline{stopif}\ref{sec:tut:stopif} and
\lstinline{freezeif}\ref{sec:tut:freezeif} from the previous section? They also
accept an event in their condition:

\begin{urbiscript}[firstnumber=1]
var e = Event.new()|;
var counter = 3|;
stopif(e?(0)) while(true) { echo(counter); e!(counter); counter--;};
[00026327] *** 3
[00026327] *** 2
[00026327] *** 1
[00026327] *** 0
\end{urbiscript}

\subsection{waituntil}

Sometimes it is useful to wait until a condition becomes true, or an event
triggers. This task can be accomplished with the
\lstinline{waituntil}\ref{sec:lang:waituntil} construct:

\begin{urbiscript}
counter = 0|;
waituntil(counter == 2) | echo("Now!"), // in parallel
for (var i: 4) { echo(i); counter = i};
[00000001] *** 0
[00000001] *** 1
[00000001] *** 2
[00000001] *** Now!
[00000001] *** 3
\end{urbiscript}




%%% Local Variables:
%%% coding: utf-8
%%% mode: latex
%%% TeX-master: "../urbi-sdk"
%%% ispell-dictionary: "american"
%%% ispell-personal-dictionary: "../urbi.dict"
%%% fill-column: 76
%%% End:
