\chapter{Functional programming}
\label{sec:tut:functional}

\us support functional programming through first class functions and
lambda expressions.

\section{First class functions}

\us has first class functions. That is, function are regular values,
just like an integer or a string, enabling you to store them or pass
them as arguments. For instance, you don't need to write
\lstinline|function object.f(){/* ... */}| to insert a function in an
object, you can simply use \lstinline{setSlot}.

% FIXME: doesn't work, toplevel, puke, etc
\begin{urbiscript}
var o = Object.clone | {};
// Here we can use f as any regular value
o.setSlot("m1", function () { echo("Hello") }) | {};
// This is strictly equivalent
var o.m2 = function () { echo("Hello") } | {};
o.m1;
[00000000] *** Hello
o.m2;
[00000000] *** Hello
\end{urbiscript}

This enables to write powerful pieces of code, like functions that
take function as argument. For instance, consider the \lstinline{all}
function: given a list and a function, it applies the function to each
element of the list, and returns whether all calls returned true. This
enables to check very simply if all elements in a list verify a
predicate.

\begin{urbiscript}[firstnumber=last]
function all(list, predicate)
{
  for (var elt : list)
    if (!predicate(elt))
      return false;
  return true;
} | {};
// Check if all elements in a list are positive.
function positive(x) { x >= 0 } | {};
all([1, 2, 3], getSlot("positive"));
[00000000] true
all([1, 2, -3], getSlot("positive"));
[00000000] false
\end{urbiscript}

It turns out that \lstinline|all| already exists: instead of
\lstinline|all(\var{list}, \var{predicate})|, use
\lstinline|\var{list}.all(\var{predicate})|, see \refObject{List}.

\section{Lambda functions}

Another nice feature is the ability to write lambda functions, which
are anonymous functions. You can create a functional value as an
expression, without naming it, with the syntax shown below.

\begin{urbiscript}[firstnumber=last]
// Create an anonymous function
function (x) {x + 1} | {};
// This enable to easily pass function
// to our "all" function:
[1, 2, 3].all(function (x) { x > 0});
[00000000] true
\end{urbiscript}

In fact, the \lstinline{function} construct we saw earlier is only a
shorthand for a variable assignment.

\begin{lstlisting}[firstnumber=last]
// This ...
function obj.f (/*...*/) {/*...*/};
// ... is actually a shorthand for this
var obj.f = function (/*...*/) {/* ... */};
\end{lstlisting}

% This should maybe be outside the functional section. It is also
% incomplete.
\section{Lazy arguments}

Most popular programming languages use strict arguments evaluation:
arguments are evaluated before functions are called. Other languages
use lazy evaluation: argument are evaluated by the function only when
needed. In \us, evaluation is strict by default, but you can ask a
function not to evaluate its arguments, and do it by hand. This works
by not specifying formal arguments. The function is provided with a
\lstinline{call} object that enables you to evaluate arguments.

\begin{urbiscript}[firstnumber=last]
// Note the lack of formal arguments specification
function first
{
  // Evaluate only the first argument.
  call.evalArgAt(0);
} | {};
first(echo("first"), echo("second"));
[00000000] *** first
function reverse
{
  call.evalArgAt(1);
  call.evalArgAt(0);
} | {};
reverse(echo("first"), echo("second"));
[00000000] *** second
[00000000] *** first
\end{urbiscript}

A good example are logic operators. Although \Cxx is a strict
language, it uses a few logic operators. For instance, the logical and
(\lstinline{&&}) does not evaluate its right operand if the left
operand is false (the result will be false anyway).

\us logic operator mimic this behavior. The listing below shows how
one can implement such a behavior.

\begin{urbiscript}[firstnumber=last]
function myAnd
{
  if (call.evalArgAt(0))
    call.evalArgAt(1)
  else
    false;
}|;

function f()
{
  echo("f executed");
  return true;
}|;

myAnd(false, f());
[00000000] false

myAnd(true, f());
[00000000] *** f executed
[00000000] true
\end{urbiscript}

%\section{Pattern matching}

%%% Local Variables:
%%% mode: latex
%%% TeX-master: "../urbi-sdk"
%%% End:
