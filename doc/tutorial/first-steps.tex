\chapter{First steps}
\label{sec:tut:first}
This section introduces the most basic notions to write \us code. Some
aspects are presented only minimally.  The goal of this section is to
bootstrap yourself with the \us language, to be able to study more
in-depth examples afterward.

\section{Comments\index{comment}}

Commenting your code is crucial, so let's start by learning how to do
this in \us. Comments are ignored by the interpreter, and
can be left as documentation, reminder, \ldots \us supports \C and
\Cxx style style comments:

\begin{itemize}
\item \C style comments start with \texttt{/*} and end with \texttt{*/}.
\item \Cxx style comments start with \texttt{//} and last until the
  end of the line.
\end{itemize}

\begin{urbiscript}
1; // This is a C++ style comment.
[00000000] 1
2 + /* This is a C-style comment. */ 2;
[00000000] 4
/* Contrary to C/C++, this type of comment /* does nest */. */
3;
[00000000] 3
\end{urbiscript}


\section{Literal values}

As already seen, we can evaluate literal integers. \us supports
several other literals, such as:

\begin{description}
\item[floats] floating point numbers.
\item[strings] character strings.
\item[lists] ordered collection of values.
\item[nil] neutral value. Think of it as the value that fits anywhere.
\item[void] absence of value. Think of it as the value that fits nowhere.
\end{description}

These literal values can be obtained with the syntax presented below.

\begin{urbiscript}[firstnumber=last]
42; // Integer literal.
[00000000] 42
3.14; // Floating point number literal.
[00000000] 3.14
"string"; // Character string literal.
[00000000] "string"
[1, 2, "a", "b"]; // List literal.
[00000000] [1, 2, "a", "b"]
nil;
void;
\end{urbiscript}

This listing highlights some point:
\begin{itemize}
\item Lists in \us are heterogeneous. That is, one list can hold
  values of different types.
\item The printing of nil and void is empty.
\end{itemize}

\section{Function calls}

You can call functions with the classical, mathematical notation.

\begin{urbiscript}[firstnumber=last]
cos(0); // Compute cosine
[00000000] 1
max(1, 3); // Get the maximum of the arguments.
[00000000] 3
max(1, 3, 4, 2);
[00000000] 4
\end{urbiscript}

Again, the result of the evaluation are printed out. You can see here
that function in \us can be variadic, that is, take different number
of arguments, such as the \lstinline{max} function. Let's now try the
\lstinline{echo} function, that prints out its argument.

\begin{urbiscript}[firstnumber=last]
echo("Hello world!");
[00000000] *** Hello world!
\end{urbiscript}

The server prints out \lstinline{Hello world!}, as expected. Note that
this output is still prepended with the time stamp. Since echo returns
void, no evaluation result is printed.

\section{Variables\index{variable}}
Variables can be introduced with the \lstinline{var} keyword, given a
name and an initial value. They can be assigned new values with the
\lstinline{=} operator.

\begin{urbiscript}[firstnumber=last]
var x = 42;
[00000000] 42
echo(x);
[00000000] *** 42
x = 51;
[00000000] 51
x;
[00000000] 51
\end{urbiscript}

Note that, just as in \Cxx, affectation returns the affected value, so
you can write code like ``\lstinline|x = y = 0|''. The rule for valid
identifiers is also the same as in \Cxx: they may contain alphanumeric
characters and underscores, but they may not start with a digit.

You may omit the initialization value, in which case it defaults to
\lstinline|void|.

\begin{urbiscript}[firstnumber=last]
var y;
y;
// Remember, the interpreter remains silent
// because void is printed out as nothing.
// You can convince yourself that x is actually
// void with the following methods.
y.asString;
[00000000] "void"
y.isVoid;
[00000000] true
\end{urbiscript}

\section{Scopes\index{scope}}
Scopes are introduced with curly brackets (\lstinline|{}|).  They can
contain any number of statements. Variables declared in a scope only
exist within this scope.

\begin{urbiscript}[firstnumber=last]
{
  var x = "test";
  echo(x);
};
[00000000] *** test
// x is no longer defined here
\end{urbiscript}

Note that the interpreter waits for the whole scope to be inputted to
evaluate it. Also note the mandatory terminating semicolon after the
closing curly bracket.

\section{Method calls}

Methods are called on objects with the dot (\lstinline{.}) notation as in
\Cxx. Method calls can be chained. Methods with no arguments don't
require the parentheses.

\begin{urbiscript}[firstnumber=last]
0.cos();
[00000000] 1
"a-b-c".split("-");
[00000000] ["a", "b", "c"]
// Empty parentheses are optional
"foo".length();
[00000000] 3
"foo".length;
[00000000] 3
// Method call can be chained
"".length.cos;
[00000000] 1
\end{urbiscript}

In \lstinline|obj.method|, we say that \lstinline{obj} is the
\dfn{target}, and that we are sending him the \lstinline{method}
\dfn{message}.

\section{Function definition}

You know how to call routines, let's learn how to write
some. Functions can be declared thanks to the \lstinline{function}
keyword, followed by the comma separated, parentheses surrounded list
of formal arguments, and the body between curly brackets.

\begin{urbiscript}[firstnumber=last]
// Define myFunction
function myFunction()
{
  echo("Hello world");
  echo("from my function!");
};
[00000000] function () {
[:]  echo("Hello world");
[:]  echo("from my function!");
[:]}

// Invoke it
myFunction();
[00000000] *** Hello world
[00000000] *** from my function!
\end{urbiscript}

Note the strange output after you defined the function. \us seems to
be printing the function you just typed in again. This is because
a function definition evaluates to the freshly created function.

Functions are first class citizen: they are values, just as
\lstinline{0} or \lstinline{"foobar"}.  The evaluation of a function
definition yields the new function, and as always, the interpreter
prints out the evaluation result, thus showing you the function
again:

\begin{urbiscript}[firstnumber=last]
// Work in a scope.
{
  // Define f
  function f()
  {
    echo("f")
  };
  // This does not invoke f, it returns its value.
  f;
};
[00000000] function () {
[:]  echo("f")
[:]}
{
  // Define f
  function f()
  {
    echo("Hello World");
  };
  // This actually calls f
  f();
};
[00000000] *** Hello World
\end{urbiscript}

Here you can see that \lstinline{f} is actually a simple value. You can just
evaluate it to see its value, that is, its body. By adding the
parentheses, you can actually call the function. This is a difference
with methods calling, where empty parentheses are optional: method are
always evaluated, you cannot retrieve their functional value --- of
course, you can with a different construct, but that's not the point
here.

Since this output is often irrelevant, most of the time it is hidden
in this documentation using the \lstinline'|{};' trick (or even
\lstinline'|;'): when evaluating \lstinline'\var{code} | {};', the
server first evaluates \var{code}, then evaluates \lstinline'{}' and
return its value, \lstinline'void', which prints to nothing.

\begin{urbiscript}[firstnumber=last]
function sum(a, b, c)
{
  return a + b + c;
} | {};
sum(20, 2, 20);
[00000000] 42
\end{urbiscript}

The \lstinline{return} keyword enables to return a value from the
function. If no \lstinline{return} statement is executed, the
evaluation of the last expression is returned.

\begin{urbiscript}[firstnumber=last]
function succ(i) { i + 1 } | {};
succ(50);
[00000000] 51
\end{urbiscript}

\section{Conclusion}

You're now up and running with basic \us code, and we can dive in
details into advanced \us code.

%%% Local Variables:
%%% mode: latex
%%% TeX-master: "../urbi-sdk"
%%% End:
